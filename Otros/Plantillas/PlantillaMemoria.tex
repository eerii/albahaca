\documentclass[12pt, a4paper, titlepage]{article}

%INFORMACIÓN
\title{\textbf {TITULO}}
\author{{\Large NOMBRE}\\DNI}
\date{}

%PAQUETES
\usepackage[utf8]{inputenc}
\usepackage[spanish]{babel}
\usepackage[centertags]{amsmath} %Excluir ecuaciones de la enumeración automática
\usepackage{tocloft} %Crear listas (por ejemplo, de ecuaciones)
\usepackage[skip=12pt]{parskip} %Añadir espacio tras los párrafos
\usepackage{csvsimple} %Tablas desde archivos .csv
\usepackage{pgfplots} %Gráficas desde matplotlib con .pgf
\pgfplotsset{compat=1.16}
\usepackage{float} %Controlar el posicionamiento de gráficas y tablas con H
\usepackage{enumitem} %Cambiar los estilos de las listas
\usepackage[toc,page]{appendix} %Anexos
\usepackage{chngcntr} %Numeración de capítulos por partes


%CONFIGURACIÓN
\renewcommand{\contentsname}{Índice}
\renewcommand{\partname}{Experiencia}
\renewcommand{\listtablename}{Lista de Tablas}
\renewcommand{\listfigurename}{Lista de Figuras}
\renewcommand{\appendixpagename}{Anexos}
\renewcommand{\appendixtocname}{\large Anexos}
\renewcommand{\appendixname}{Anexo}

\newcommand{\listecuacionesname}{\Large Lista de Ecuaciones}
\newlistof{ecuaciones}{equ}{\listecuacionesname}
\newcommand{\ecuaciones}[1]{\addcontentsline{equ}{ecuaciones}{\protect\numberline{\theequation}#1}\par}

\newcommand{\ectag}[1]{\tag*{#1}}% Etiquetar ecuación con nombre

\linespread{1.3}
\counterwithin*{section}{part}



%DOCUMENTO
\begin{document}
  \maketitle

  \tableofcontents
  %\listoftables
  %\listoffigures
  %\listofecuaciones

  \newpage
  \part*{Introducción}
  \addcontentsline{toc}{part}{Introducción}
  Prueba de ácídífícácíón


  \newpage
  \part{Corriente Continua}


  \newpage
  \part{Corriente Alterna}

  \section{Fórmulas}

  \subsection{Ecuaciones}

  Ecuación numerada:
  \begin{equation} \label{ec:test}
    f(x) = x^2
  \end{equation}
  \ecuaciones{Ecuación número \ref{ec:test}}

  Varias ecuaciones alineadas:
  \begin{align} %Se alinean en &
    f(x) &= 2x + \int^a_b y^2 dy \nonumber \\
    f(x) &= 2 (x\lambda + \frac{1}{2}y) \label{ec:lambda}
  \end{align}
  \ecuaciones{Lista de ecuaciones}
  Ver~\ref{ec:lambda}


  Una ecuación cómo $y = \sqrt{x}$ dentro del texto.

  \subsection{Matrices}

  $\left(
  \begin{matrix}
    1 & 0 & 0\\
    0 & 1 & 0\\
    0 & 0 & 1
  \end{matrix}
  \right)$

  Una matriz en su propio entorno

  \begin{figure}[H]
    %\centering
    \hspace{2.5em} %% Creator: Matplotlib, PGF backend
%%
%% To include the figure in your LaTeX document, write
%%   \input{<filename>.pgf}
%%
%% Make sure the required packages are loaded in your preamble
%%   \usepackage{pgf}
%%
%% Figures using additional raster images can only be included by \input if
%% they are in the same directory as the main LaTeX file. For loading figures
%% from other directories you can use the `import` package
%%   \usepackage{import}
%% and then include the figures with
%%   \import{<path to file>}{<filename>.pgf}
%%
%% Matplotlib used the following preamble
%%
\begingroup%
\makeatletter%
\begin{pgfpicture}%
\pgfpathrectangle{\pgfpointorigin}{\pgfqpoint{5.333026in}{3.440679in}}%
\pgfusepath{use as bounding box, clip}%
\begin{pgfscope}%
\pgfsetbuttcap%
\pgfsetmiterjoin%
\definecolor{currentfill}{rgb}{1.000000,1.000000,1.000000}%
\pgfsetfillcolor{currentfill}%
\pgfsetlinewidth{0.000000pt}%
\definecolor{currentstroke}{rgb}{1.000000,1.000000,1.000000}%
\pgfsetstrokecolor{currentstroke}%
\pgfsetdash{}{0pt}%
\pgfpathmoveto{\pgfqpoint{0.000000in}{0.000000in}}%
\pgfpathlineto{\pgfqpoint{5.333026in}{0.000000in}}%
\pgfpathlineto{\pgfqpoint{5.333026in}{3.440679in}}%
\pgfpathlineto{\pgfqpoint{0.000000in}{3.440679in}}%
\pgfpathclose%
\pgfusepath{fill}%
\end{pgfscope}%
\begin{pgfscope}%
\pgfsetbuttcap%
\pgfsetmiterjoin%
\definecolor{currentfill}{rgb}{1.000000,1.000000,1.000000}%
\pgfsetfillcolor{currentfill}%
\pgfsetlinewidth{0.000000pt}%
\definecolor{currentstroke}{rgb}{0.000000,0.000000,0.000000}%
\pgfsetstrokecolor{currentstroke}%
\pgfsetstrokeopacity{0.000000}%
\pgfsetdash{}{0pt}%
\pgfpathmoveto{\pgfqpoint{0.583026in}{0.320679in}}%
\pgfpathlineto{\pgfqpoint{5.233026in}{0.320679in}}%
\pgfpathlineto{\pgfqpoint{5.233026in}{3.340679in}}%
\pgfpathlineto{\pgfqpoint{0.583026in}{3.340679in}}%
\pgfpathclose%
\pgfusepath{fill}%
\end{pgfscope}%
\begin{pgfscope}%
\pgfpathrectangle{\pgfqpoint{0.583026in}{0.320679in}}{\pgfqpoint{4.650000in}{3.020000in}}%
\pgfusepath{clip}%
\pgfsetbuttcap%
\pgfsetroundjoin%
\definecolor{currentfill}{rgb}{0.121569,0.466667,0.705882}%
\pgfsetfillcolor{currentfill}%
\pgfsetlinewidth{1.003750pt}%
\definecolor{currentstroke}{rgb}{0.121569,0.466667,0.705882}%
\pgfsetstrokecolor{currentstroke}%
\pgfsetdash{}{0pt}%
\pgfpathmoveto{\pgfqpoint{1.040602in}{0.527166in}}%
\pgfpathcurveto{\pgfqpoint{1.051652in}{0.527166in}}{\pgfqpoint{1.062251in}{0.531557in}}{\pgfqpoint{1.070064in}{0.539370in}}%
\pgfpathcurveto{\pgfqpoint{1.077878in}{0.547184in}}{\pgfqpoint{1.082268in}{0.557783in}}{\pgfqpoint{1.082268in}{0.568833in}}%
\pgfpathcurveto{\pgfqpoint{1.082268in}{0.579883in}}{\pgfqpoint{1.077878in}{0.590482in}}{\pgfqpoint{1.070064in}{0.598296in}}%
\pgfpathcurveto{\pgfqpoint{1.062251in}{0.606109in}}{\pgfqpoint{1.051652in}{0.610500in}}{\pgfqpoint{1.040602in}{0.610500in}}%
\pgfpathcurveto{\pgfqpoint{1.029551in}{0.610500in}}{\pgfqpoint{1.018952in}{0.606109in}}{\pgfqpoint{1.011139in}{0.598296in}}%
\pgfpathcurveto{\pgfqpoint{1.003325in}{0.590482in}}{\pgfqpoint{0.998935in}{0.579883in}}{\pgfqpoint{0.998935in}{0.568833in}}%
\pgfpathcurveto{\pgfqpoint{0.998935in}{0.557783in}}{\pgfqpoint{1.003325in}{0.547184in}}{\pgfqpoint{1.011139in}{0.539370in}}%
\pgfpathcurveto{\pgfqpoint{1.018952in}{0.531557in}}{\pgfqpoint{1.029551in}{0.527166in}}{\pgfqpoint{1.040602in}{0.527166in}}%
\pgfpathclose%
\pgfusepath{stroke,fill}%
\end{pgfscope}%
\begin{pgfscope}%
\pgfpathrectangle{\pgfqpoint{0.583026in}{0.320679in}}{\pgfqpoint{4.650000in}{3.020000in}}%
\pgfusepath{clip}%
\pgfsetbuttcap%
\pgfsetroundjoin%
\definecolor{currentfill}{rgb}{0.121569,0.466667,0.705882}%
\pgfsetfillcolor{currentfill}%
\pgfsetlinewidth{1.003750pt}%
\definecolor{currentstroke}{rgb}{0.121569,0.466667,0.705882}%
\pgfsetstrokecolor{currentstroke}%
\pgfsetdash{}{0pt}%
\pgfpathmoveto{\pgfqpoint{1.342771in}{0.680695in}}%
\pgfpathcurveto{\pgfqpoint{1.353821in}{0.680695in}}{\pgfqpoint{1.364420in}{0.685085in}}{\pgfqpoint{1.372234in}{0.692898in}}%
\pgfpathcurveto{\pgfqpoint{1.380047in}{0.700712in}}{\pgfqpoint{1.384438in}{0.711311in}}{\pgfqpoint{1.384438in}{0.722361in}}%
\pgfpathcurveto{\pgfqpoint{1.384438in}{0.733411in}}{\pgfqpoint{1.380047in}{0.744010in}}{\pgfqpoint{1.372234in}{0.751824in}}%
\pgfpathcurveto{\pgfqpoint{1.364420in}{0.759638in}}{\pgfqpoint{1.353821in}{0.764028in}}{\pgfqpoint{1.342771in}{0.764028in}}%
\pgfpathcurveto{\pgfqpoint{1.331721in}{0.764028in}}{\pgfqpoint{1.321122in}{0.759638in}}{\pgfqpoint{1.313308in}{0.751824in}}%
\pgfpathcurveto{\pgfqpoint{1.305495in}{0.744010in}}{\pgfqpoint{1.301104in}{0.733411in}}{\pgfqpoint{1.301104in}{0.722361in}}%
\pgfpathcurveto{\pgfqpoint{1.301104in}{0.711311in}}{\pgfqpoint{1.305495in}{0.700712in}}{\pgfqpoint{1.313308in}{0.692898in}}%
\pgfpathcurveto{\pgfqpoint{1.321122in}{0.685085in}}{\pgfqpoint{1.331721in}{0.680695in}}{\pgfqpoint{1.342771in}{0.680695in}}%
\pgfpathclose%
\pgfusepath{stroke,fill}%
\end{pgfscope}%
\begin{pgfscope}%
\pgfpathrectangle{\pgfqpoint{0.583026in}{0.320679in}}{\pgfqpoint{4.650000in}{3.020000in}}%
\pgfusepath{clip}%
\pgfsetbuttcap%
\pgfsetroundjoin%
\definecolor{currentfill}{rgb}{0.121569,0.466667,0.705882}%
\pgfsetfillcolor{currentfill}%
\pgfsetlinewidth{1.003750pt}%
\definecolor{currentstroke}{rgb}{0.121569,0.466667,0.705882}%
\pgfsetstrokecolor{currentstroke}%
\pgfsetdash{}{0pt}%
\pgfpathmoveto{\pgfqpoint{1.600175in}{0.808635in}}%
\pgfpathcurveto{\pgfqpoint{1.611225in}{0.808635in}}{\pgfqpoint{1.621824in}{0.813025in}}{\pgfqpoint{1.629637in}{0.820839in}}%
\pgfpathcurveto{\pgfqpoint{1.637451in}{0.828652in}}{\pgfqpoint{1.641841in}{0.839251in}}{\pgfqpoint{1.641841in}{0.850301in}}%
\pgfpathcurveto{\pgfqpoint{1.641841in}{0.861351in}}{\pgfqpoint{1.637451in}{0.871950in}}{\pgfqpoint{1.629637in}{0.879764in}}%
\pgfpathcurveto{\pgfqpoint{1.621824in}{0.887578in}}{\pgfqpoint{1.611225in}{0.891968in}}{\pgfqpoint{1.600175in}{0.891968in}}%
\pgfpathcurveto{\pgfqpoint{1.589124in}{0.891968in}}{\pgfqpoint{1.578525in}{0.887578in}}{\pgfqpoint{1.570712in}{0.879764in}}%
\pgfpathcurveto{\pgfqpoint{1.562898in}{0.871950in}}{\pgfqpoint{1.558508in}{0.861351in}}{\pgfqpoint{1.558508in}{0.850301in}}%
\pgfpathcurveto{\pgfqpoint{1.558508in}{0.839251in}}{\pgfqpoint{1.562898in}{0.828652in}}{\pgfqpoint{1.570712in}{0.820839in}}%
\pgfpathcurveto{\pgfqpoint{1.578525in}{0.813025in}}{\pgfqpoint{1.589124in}{0.808635in}}{\pgfqpoint{1.600175in}{0.808635in}}%
\pgfpathclose%
\pgfusepath{stroke,fill}%
\end{pgfscope}%
\begin{pgfscope}%
\pgfpathrectangle{\pgfqpoint{0.583026in}{0.320679in}}{\pgfqpoint{4.650000in}{3.020000in}}%
\pgfusepath{clip}%
\pgfsetbuttcap%
\pgfsetroundjoin%
\definecolor{currentfill}{rgb}{0.121569,0.466667,0.705882}%
\pgfsetfillcolor{currentfill}%
\pgfsetlinewidth{1.003750pt}%
\definecolor{currentstroke}{rgb}{0.121569,0.466667,0.705882}%
\pgfsetstrokecolor{currentstroke}%
\pgfsetdash{}{0pt}%
\pgfpathmoveto{\pgfqpoint{1.857578in}{0.936575in}}%
\pgfpathcurveto{\pgfqpoint{1.868628in}{0.936575in}}{\pgfqpoint{1.879227in}{0.940965in}}{\pgfqpoint{1.887041in}{0.948779in}}%
\pgfpathcurveto{\pgfqpoint{1.894855in}{0.956592in}}{\pgfqpoint{1.899245in}{0.967191in}}{\pgfqpoint{1.899245in}{0.978241in}}%
\pgfpathcurveto{\pgfqpoint{1.899245in}{0.989292in}}{\pgfqpoint{1.894855in}{0.999891in}}{\pgfqpoint{1.887041in}{1.007704in}}%
\pgfpathcurveto{\pgfqpoint{1.879227in}{1.015518in}}{\pgfqpoint{1.868628in}{1.019908in}}{\pgfqpoint{1.857578in}{1.019908in}}%
\pgfpathcurveto{\pgfqpoint{1.846528in}{1.019908in}}{\pgfqpoint{1.835929in}{1.015518in}}{\pgfqpoint{1.828115in}{1.007704in}}%
\pgfpathcurveto{\pgfqpoint{1.820302in}{0.999891in}}{\pgfqpoint{1.815912in}{0.989292in}}{\pgfqpoint{1.815912in}{0.978241in}}%
\pgfpathcurveto{\pgfqpoint{1.815912in}{0.967191in}}{\pgfqpoint{1.820302in}{0.956592in}}{\pgfqpoint{1.828115in}{0.948779in}}%
\pgfpathcurveto{\pgfqpoint{1.835929in}{0.940965in}}{\pgfqpoint{1.846528in}{0.936575in}}{\pgfqpoint{1.857578in}{0.936575in}}%
\pgfpathclose%
\pgfusepath{stroke,fill}%
\end{pgfscope}%
\begin{pgfscope}%
\pgfpathrectangle{\pgfqpoint{0.583026in}{0.320679in}}{\pgfqpoint{4.650000in}{3.020000in}}%
\pgfusepath{clip}%
\pgfsetbuttcap%
\pgfsetroundjoin%
\definecolor{currentfill}{rgb}{0.121569,0.466667,0.705882}%
\pgfsetfillcolor{currentfill}%
\pgfsetlinewidth{1.003750pt}%
\definecolor{currentstroke}{rgb}{0.121569,0.466667,0.705882}%
\pgfsetstrokecolor{currentstroke}%
\pgfsetdash{}{0pt}%
\pgfpathmoveto{\pgfqpoint{2.047833in}{1.038927in}}%
\pgfpathcurveto{\pgfqpoint{2.058883in}{1.038927in}}{\pgfqpoint{2.069482in}{1.043317in}}{\pgfqpoint{2.077296in}{1.051131in}}%
\pgfpathcurveto{\pgfqpoint{2.085109in}{1.058944in}}{\pgfqpoint{2.089500in}{1.069543in}}{\pgfqpoint{2.089500in}{1.080593in}}%
\pgfpathcurveto{\pgfqpoint{2.089500in}{1.091644in}}{\pgfqpoint{2.085109in}{1.102243in}}{\pgfqpoint{2.077296in}{1.110056in}}%
\pgfpathcurveto{\pgfqpoint{2.069482in}{1.117870in}}{\pgfqpoint{2.058883in}{1.122260in}}{\pgfqpoint{2.047833in}{1.122260in}}%
\pgfpathcurveto{\pgfqpoint{2.036783in}{1.122260in}}{\pgfqpoint{2.026184in}{1.117870in}}{\pgfqpoint{2.018370in}{1.110056in}}%
\pgfpathcurveto{\pgfqpoint{2.010557in}{1.102243in}}{\pgfqpoint{2.006166in}{1.091644in}}{\pgfqpoint{2.006166in}{1.080593in}}%
\pgfpathcurveto{\pgfqpoint{2.006166in}{1.069543in}}{\pgfqpoint{2.010557in}{1.058944in}}{\pgfqpoint{2.018370in}{1.051131in}}%
\pgfpathcurveto{\pgfqpoint{2.026184in}{1.043317in}}{\pgfqpoint{2.036783in}{1.038927in}}{\pgfqpoint{2.047833in}{1.038927in}}%
\pgfpathclose%
\pgfusepath{stroke,fill}%
\end{pgfscope}%
\begin{pgfscope}%
\pgfpathrectangle{\pgfqpoint{0.583026in}{0.320679in}}{\pgfqpoint{4.650000in}{3.020000in}}%
\pgfusepath{clip}%
\pgfsetbuttcap%
\pgfsetroundjoin%
\definecolor{currentfill}{rgb}{0.121569,0.466667,0.705882}%
\pgfsetfillcolor{currentfill}%
\pgfsetlinewidth{1.003750pt}%
\definecolor{currentstroke}{rgb}{0.121569,0.466667,0.705882}%
\pgfsetstrokecolor{currentstroke}%
\pgfsetdash{}{0pt}%
\pgfpathmoveto{\pgfqpoint{2.316428in}{1.175396in}}%
\pgfpathcurveto{\pgfqpoint{2.327478in}{1.175396in}}{\pgfqpoint{2.338077in}{1.179787in}}{\pgfqpoint{2.345891in}{1.187600in}}%
\pgfpathcurveto{\pgfqpoint{2.353704in}{1.195414in}}{\pgfqpoint{2.358095in}{1.206013in}}{\pgfqpoint{2.358095in}{1.217063in}}%
\pgfpathcurveto{\pgfqpoint{2.358095in}{1.228113in}}{\pgfqpoint{2.353704in}{1.238712in}}{\pgfqpoint{2.345891in}{1.246526in}}%
\pgfpathcurveto{\pgfqpoint{2.338077in}{1.254339in}}{\pgfqpoint{2.327478in}{1.258730in}}{\pgfqpoint{2.316428in}{1.258730in}}%
\pgfpathcurveto{\pgfqpoint{2.305378in}{1.258730in}}{\pgfqpoint{2.294779in}{1.254339in}}{\pgfqpoint{2.286965in}{1.246526in}}%
\pgfpathcurveto{\pgfqpoint{2.279152in}{1.238712in}}{\pgfqpoint{2.274761in}{1.228113in}}{\pgfqpoint{2.274761in}{1.217063in}}%
\pgfpathcurveto{\pgfqpoint{2.274761in}{1.206013in}}{\pgfqpoint{2.279152in}{1.195414in}}{\pgfqpoint{2.286965in}{1.187600in}}%
\pgfpathcurveto{\pgfqpoint{2.294779in}{1.179787in}}{\pgfqpoint{2.305378in}{1.175396in}}{\pgfqpoint{2.316428in}{1.175396in}}%
\pgfpathclose%
\pgfusepath{stroke,fill}%
\end{pgfscope}%
\begin{pgfscope}%
\pgfpathrectangle{\pgfqpoint{0.583026in}{0.320679in}}{\pgfqpoint{4.650000in}{3.020000in}}%
\pgfusepath{clip}%
\pgfsetbuttcap%
\pgfsetroundjoin%
\definecolor{currentfill}{rgb}{0.121569,0.466667,0.705882}%
\pgfsetfillcolor{currentfill}%
\pgfsetlinewidth{1.003750pt}%
\definecolor{currentstroke}{rgb}{0.121569,0.466667,0.705882}%
\pgfsetstrokecolor{currentstroke}%
\pgfsetdash{}{0pt}%
\pgfpathmoveto{\pgfqpoint{2.562640in}{1.294807in}}%
\pgfpathcurveto{\pgfqpoint{2.573690in}{1.294807in}}{\pgfqpoint{2.584289in}{1.299197in}}{\pgfqpoint{2.592103in}{1.307011in}}%
\pgfpathcurveto{\pgfqpoint{2.599917in}{1.314825in}}{\pgfqpoint{2.604307in}{1.325424in}}{\pgfqpoint{2.604307in}{1.336474in}}%
\pgfpathcurveto{\pgfqpoint{2.604307in}{1.347524in}}{\pgfqpoint{2.599917in}{1.358123in}}{\pgfqpoint{2.592103in}{1.365936in}}%
\pgfpathcurveto{\pgfqpoint{2.584289in}{1.373750in}}{\pgfqpoint{2.573690in}{1.378140in}}{\pgfqpoint{2.562640in}{1.378140in}}%
\pgfpathcurveto{\pgfqpoint{2.551590in}{1.378140in}}{\pgfqpoint{2.540991in}{1.373750in}}{\pgfqpoint{2.533177in}{1.365936in}}%
\pgfpathcurveto{\pgfqpoint{2.525364in}{1.358123in}}{\pgfqpoint{2.520973in}{1.347524in}}{\pgfqpoint{2.520973in}{1.336474in}}%
\pgfpathcurveto{\pgfqpoint{2.520973in}{1.325424in}}{\pgfqpoint{2.525364in}{1.314825in}}{\pgfqpoint{2.533177in}{1.307011in}}%
\pgfpathcurveto{\pgfqpoint{2.540991in}{1.299197in}}{\pgfqpoint{2.551590in}{1.294807in}}{\pgfqpoint{2.562640in}{1.294807in}}%
\pgfpathclose%
\pgfusepath{stroke,fill}%
\end{pgfscope}%
\begin{pgfscope}%
\pgfpathrectangle{\pgfqpoint{0.583026in}{0.320679in}}{\pgfqpoint{4.650000in}{3.020000in}}%
\pgfusepath{clip}%
\pgfsetbuttcap%
\pgfsetroundjoin%
\definecolor{currentfill}{rgb}{0.121569,0.466667,0.705882}%
\pgfsetfillcolor{currentfill}%
\pgfsetlinewidth{1.003750pt}%
\definecolor{currentstroke}{rgb}{0.121569,0.466667,0.705882}%
\pgfsetstrokecolor{currentstroke}%
\pgfsetdash{}{0pt}%
\pgfpathmoveto{\pgfqpoint{2.775278in}{1.397159in}}%
\pgfpathcurveto{\pgfqpoint{2.786328in}{1.397159in}}{\pgfqpoint{2.796927in}{1.401549in}}{\pgfqpoint{2.804741in}{1.409363in}}%
\pgfpathcurveto{\pgfqpoint{2.812554in}{1.417177in}}{\pgfqpoint{2.816945in}{1.427776in}}{\pgfqpoint{2.816945in}{1.438826in}}%
\pgfpathcurveto{\pgfqpoint{2.816945in}{1.449876in}}{\pgfqpoint{2.812554in}{1.460475in}}{\pgfqpoint{2.804741in}{1.468289in}}%
\pgfpathcurveto{\pgfqpoint{2.796927in}{1.476102in}}{\pgfqpoint{2.786328in}{1.480492in}}{\pgfqpoint{2.775278in}{1.480492in}}%
\pgfpathcurveto{\pgfqpoint{2.764228in}{1.480492in}}{\pgfqpoint{2.753629in}{1.476102in}}{\pgfqpoint{2.745815in}{1.468289in}}%
\pgfpathcurveto{\pgfqpoint{2.738001in}{1.460475in}}{\pgfqpoint{2.733611in}{1.449876in}}{\pgfqpoint{2.733611in}{1.438826in}}%
\pgfpathcurveto{\pgfqpoint{2.733611in}{1.427776in}}{\pgfqpoint{2.738001in}{1.417177in}}{\pgfqpoint{2.745815in}{1.409363in}}%
\pgfpathcurveto{\pgfqpoint{2.753629in}{1.401549in}}{\pgfqpoint{2.764228in}{1.397159in}}{\pgfqpoint{2.775278in}{1.397159in}}%
\pgfpathclose%
\pgfusepath{stroke,fill}%
\end{pgfscope}%
\begin{pgfscope}%
\pgfpathrectangle{\pgfqpoint{0.583026in}{0.320679in}}{\pgfqpoint{4.650000in}{3.020000in}}%
\pgfusepath{clip}%
\pgfsetbuttcap%
\pgfsetroundjoin%
\definecolor{currentfill}{rgb}{0.121569,0.466667,0.705882}%
\pgfsetfillcolor{currentfill}%
\pgfsetlinewidth{1.003750pt}%
\definecolor{currentstroke}{rgb}{0.121569,0.466667,0.705882}%
\pgfsetstrokecolor{currentstroke}%
\pgfsetdash{}{0pt}%
\pgfpathmoveto{\pgfqpoint{2.976724in}{1.508041in}}%
\pgfpathcurveto{\pgfqpoint{2.987774in}{1.508041in}}{\pgfqpoint{2.998373in}{1.512431in}}{\pgfqpoint{3.006187in}{1.520244in}}%
\pgfpathcurveto{\pgfqpoint{3.014001in}{1.528058in}}{\pgfqpoint{3.018391in}{1.538657in}}{\pgfqpoint{3.018391in}{1.549707in}}%
\pgfpathcurveto{\pgfqpoint{3.018391in}{1.560757in}}{\pgfqpoint{3.014001in}{1.571356in}}{\pgfqpoint{3.006187in}{1.579170in}}%
\pgfpathcurveto{\pgfqpoint{2.998373in}{1.586984in}}{\pgfqpoint{2.987774in}{1.591374in}}{\pgfqpoint{2.976724in}{1.591374in}}%
\pgfpathcurveto{\pgfqpoint{2.965674in}{1.591374in}}{\pgfqpoint{2.955075in}{1.586984in}}{\pgfqpoint{2.947261in}{1.579170in}}%
\pgfpathcurveto{\pgfqpoint{2.939448in}{1.571356in}}{\pgfqpoint{2.935058in}{1.560757in}}{\pgfqpoint{2.935058in}{1.549707in}}%
\pgfpathcurveto{\pgfqpoint{2.935058in}{1.538657in}}{\pgfqpoint{2.939448in}{1.528058in}}{\pgfqpoint{2.947261in}{1.520244in}}%
\pgfpathcurveto{\pgfqpoint{2.955075in}{1.512431in}}{\pgfqpoint{2.965674in}{1.508041in}}{\pgfqpoint{2.976724in}{1.508041in}}%
\pgfpathclose%
\pgfusepath{stroke,fill}%
\end{pgfscope}%
\begin{pgfscope}%
\pgfpathrectangle{\pgfqpoint{0.583026in}{0.320679in}}{\pgfqpoint{4.650000in}{3.020000in}}%
\pgfusepath{clip}%
\pgfsetbuttcap%
\pgfsetroundjoin%
\definecolor{currentfill}{rgb}{0.121569,0.466667,0.705882}%
\pgfsetfillcolor{currentfill}%
\pgfsetlinewidth{1.003750pt}%
\definecolor{currentstroke}{rgb}{0.121569,0.466667,0.705882}%
\pgfsetstrokecolor{currentstroke}%
\pgfsetdash{}{0pt}%
\pgfpathmoveto{\pgfqpoint{3.211745in}{1.627451in}}%
\pgfpathcurveto{\pgfqpoint{3.222795in}{1.627451in}}{\pgfqpoint{3.233394in}{1.631842in}}{\pgfqpoint{3.241208in}{1.639655in}}%
\pgfpathcurveto{\pgfqpoint{3.249021in}{1.647469in}}{\pgfqpoint{3.253412in}{1.658068in}}{\pgfqpoint{3.253412in}{1.669118in}}%
\pgfpathcurveto{\pgfqpoint{3.253412in}{1.680168in}}{\pgfqpoint{3.249021in}{1.690767in}}{\pgfqpoint{3.241208in}{1.698581in}}%
\pgfpathcurveto{\pgfqpoint{3.233394in}{1.706394in}}{\pgfqpoint{3.222795in}{1.710785in}}{\pgfqpoint{3.211745in}{1.710785in}}%
\pgfpathcurveto{\pgfqpoint{3.200695in}{1.710785in}}{\pgfqpoint{3.190096in}{1.706394in}}{\pgfqpoint{3.182282in}{1.698581in}}%
\pgfpathcurveto{\pgfqpoint{3.174468in}{1.690767in}}{\pgfqpoint{3.170078in}{1.680168in}}{\pgfqpoint{3.170078in}{1.669118in}}%
\pgfpathcurveto{\pgfqpoint{3.170078in}{1.658068in}}{\pgfqpoint{3.174468in}{1.647469in}}{\pgfqpoint{3.182282in}{1.639655in}}%
\pgfpathcurveto{\pgfqpoint{3.190096in}{1.631842in}}{\pgfqpoint{3.200695in}{1.627451in}}{\pgfqpoint{3.211745in}{1.627451in}}%
\pgfpathclose%
\pgfusepath{stroke,fill}%
\end{pgfscope}%
\begin{pgfscope}%
\pgfpathrectangle{\pgfqpoint{0.583026in}{0.320679in}}{\pgfqpoint{4.650000in}{3.020000in}}%
\pgfusepath{clip}%
\pgfsetbuttcap%
\pgfsetroundjoin%
\definecolor{currentfill}{rgb}{0.121569,0.466667,0.705882}%
\pgfsetfillcolor{currentfill}%
\pgfsetlinewidth{1.003750pt}%
\definecolor{currentstroke}{rgb}{0.121569,0.466667,0.705882}%
\pgfsetstrokecolor{currentstroke}%
\pgfsetdash{}{0pt}%
\pgfpathmoveto{\pgfqpoint{3.469148in}{1.755391in}}%
\pgfpathcurveto{\pgfqpoint{3.480199in}{1.755391in}}{\pgfqpoint{3.490798in}{1.759782in}}{\pgfqpoint{3.498611in}{1.767595in}}%
\pgfpathcurveto{\pgfqpoint{3.506425in}{1.775409in}}{\pgfqpoint{3.510815in}{1.786008in}}{\pgfqpoint{3.510815in}{1.797058in}}%
\pgfpathcurveto{\pgfqpoint{3.510815in}{1.808108in}}{\pgfqpoint{3.506425in}{1.818707in}}{\pgfqpoint{3.498611in}{1.826521in}}%
\pgfpathcurveto{\pgfqpoint{3.490798in}{1.834335in}}{\pgfqpoint{3.480199in}{1.838725in}}{\pgfqpoint{3.469148in}{1.838725in}}%
\pgfpathcurveto{\pgfqpoint{3.458098in}{1.838725in}}{\pgfqpoint{3.447499in}{1.834335in}}{\pgfqpoint{3.439686in}{1.826521in}}%
\pgfpathcurveto{\pgfqpoint{3.431872in}{1.818707in}}{\pgfqpoint{3.427482in}{1.808108in}}{\pgfqpoint{3.427482in}{1.797058in}}%
\pgfpathcurveto{\pgfqpoint{3.427482in}{1.786008in}}{\pgfqpoint{3.431872in}{1.775409in}}{\pgfqpoint{3.439686in}{1.767595in}}%
\pgfpathcurveto{\pgfqpoint{3.447499in}{1.759782in}}{\pgfqpoint{3.458098in}{1.755391in}}{\pgfqpoint{3.469148in}{1.755391in}}%
\pgfpathclose%
\pgfusepath{stroke,fill}%
\end{pgfscope}%
\begin{pgfscope}%
\pgfpathrectangle{\pgfqpoint{0.583026in}{0.320679in}}{\pgfqpoint{4.650000in}{3.020000in}}%
\pgfusepath{clip}%
\pgfsetbuttcap%
\pgfsetroundjoin%
\definecolor{currentfill}{rgb}{0.121569,0.466667,0.705882}%
\pgfsetfillcolor{currentfill}%
\pgfsetlinewidth{1.003750pt}%
\definecolor{currentstroke}{rgb}{0.121569,0.466667,0.705882}%
\pgfsetstrokecolor{currentstroke}%
\pgfsetdash{}{0pt}%
\pgfpathmoveto{\pgfqpoint{3.726552in}{1.883332in}}%
\pgfpathcurveto{\pgfqpoint{3.737602in}{1.883332in}}{\pgfqpoint{3.748201in}{1.887722in}}{\pgfqpoint{3.756015in}{1.895535in}}%
\pgfpathcurveto{\pgfqpoint{3.763828in}{1.903349in}}{\pgfqpoint{3.768219in}{1.913948in}}{\pgfqpoint{3.768219in}{1.924998in}}%
\pgfpathcurveto{\pgfqpoint{3.768219in}{1.936048in}}{\pgfqpoint{3.763828in}{1.946647in}}{\pgfqpoint{3.756015in}{1.954461in}}%
\pgfpathcurveto{\pgfqpoint{3.748201in}{1.962275in}}{\pgfqpoint{3.737602in}{1.966665in}}{\pgfqpoint{3.726552in}{1.966665in}}%
\pgfpathcurveto{\pgfqpoint{3.715502in}{1.966665in}}{\pgfqpoint{3.704903in}{1.962275in}}{\pgfqpoint{3.697089in}{1.954461in}}%
\pgfpathcurveto{\pgfqpoint{3.689276in}{1.946647in}}{\pgfqpoint{3.684885in}{1.936048in}}{\pgfqpoint{3.684885in}{1.924998in}}%
\pgfpathcurveto{\pgfqpoint{3.684885in}{1.913948in}}{\pgfqpoint{3.689276in}{1.903349in}}{\pgfqpoint{3.697089in}{1.895535in}}%
\pgfpathcurveto{\pgfqpoint{3.704903in}{1.887722in}}{\pgfqpoint{3.715502in}{1.883332in}}{\pgfqpoint{3.726552in}{1.883332in}}%
\pgfpathclose%
\pgfusepath{stroke,fill}%
\end{pgfscope}%
\begin{pgfscope}%
\pgfpathrectangle{\pgfqpoint{0.583026in}{0.320679in}}{\pgfqpoint{4.650000in}{3.020000in}}%
\pgfusepath{clip}%
\pgfsetbuttcap%
\pgfsetroundjoin%
\definecolor{currentfill}{rgb}{0.121569,0.466667,0.705882}%
\pgfsetfillcolor{currentfill}%
\pgfsetlinewidth{1.003750pt}%
\definecolor{currentstroke}{rgb}{0.121569,0.466667,0.705882}%
\pgfsetstrokecolor{currentstroke}%
\pgfsetdash{}{0pt}%
\pgfpathmoveto{\pgfqpoint{3.838467in}{1.951566in}}%
\pgfpathcurveto{\pgfqpoint{3.849517in}{1.951566in}}{\pgfqpoint{3.860116in}{1.955957in}}{\pgfqpoint{3.867929in}{1.963770in}}%
\pgfpathcurveto{\pgfqpoint{3.875743in}{1.971584in}}{\pgfqpoint{3.880133in}{1.982183in}}{\pgfqpoint{3.880133in}{1.993233in}}%
\pgfpathcurveto{\pgfqpoint{3.880133in}{2.004283in}}{\pgfqpoint{3.875743in}{2.014882in}}{\pgfqpoint{3.867929in}{2.022696in}}%
\pgfpathcurveto{\pgfqpoint{3.860116in}{2.030509in}}{\pgfqpoint{3.849517in}{2.034900in}}{\pgfqpoint{3.838467in}{2.034900in}}%
\pgfpathcurveto{\pgfqpoint{3.827416in}{2.034900in}}{\pgfqpoint{3.816817in}{2.030509in}}{\pgfqpoint{3.809004in}{2.022696in}}%
\pgfpathcurveto{\pgfqpoint{3.801190in}{2.014882in}}{\pgfqpoint{3.796800in}{2.004283in}}{\pgfqpoint{3.796800in}{1.993233in}}%
\pgfpathcurveto{\pgfqpoint{3.796800in}{1.982183in}}{\pgfqpoint{3.801190in}{1.971584in}}{\pgfqpoint{3.809004in}{1.963770in}}%
\pgfpathcurveto{\pgfqpoint{3.816817in}{1.955957in}}{\pgfqpoint{3.827416in}{1.951566in}}{\pgfqpoint{3.838467in}{1.951566in}}%
\pgfpathclose%
\pgfusepath{stroke,fill}%
\end{pgfscope}%
\begin{pgfscope}%
\pgfpathrectangle{\pgfqpoint{0.583026in}{0.320679in}}{\pgfqpoint{4.650000in}{3.020000in}}%
\pgfusepath{clip}%
\pgfsetbuttcap%
\pgfsetroundjoin%
\definecolor{currentfill}{rgb}{0.121569,0.466667,0.705882}%
\pgfsetfillcolor{currentfill}%
\pgfsetlinewidth{1.003750pt}%
\definecolor{currentstroke}{rgb}{0.121569,0.466667,0.705882}%
\pgfsetstrokecolor{currentstroke}%
\pgfsetdash{}{0pt}%
\pgfpathmoveto{\pgfqpoint{4.062296in}{2.062448in}}%
\pgfpathcurveto{\pgfqpoint{4.073346in}{2.062448in}}{\pgfqpoint{4.083945in}{2.066838in}}{\pgfqpoint{4.091759in}{2.074652in}}%
\pgfpathcurveto{\pgfqpoint{4.099572in}{2.082465in}}{\pgfqpoint{4.103962in}{2.093064in}}{\pgfqpoint{4.103962in}{2.104114in}}%
\pgfpathcurveto{\pgfqpoint{4.103962in}{2.115164in}}{\pgfqpoint{4.099572in}{2.125764in}}{\pgfqpoint{4.091759in}{2.133577in}}%
\pgfpathcurveto{\pgfqpoint{4.083945in}{2.141391in}}{\pgfqpoint{4.073346in}{2.145781in}}{\pgfqpoint{4.062296in}{2.145781in}}%
\pgfpathcurveto{\pgfqpoint{4.051246in}{2.145781in}}{\pgfqpoint{4.040647in}{2.141391in}}{\pgfqpoint{4.032833in}{2.133577in}}%
\pgfpathcurveto{\pgfqpoint{4.025019in}{2.125764in}}{\pgfqpoint{4.020629in}{2.115164in}}{\pgfqpoint{4.020629in}{2.104114in}}%
\pgfpathcurveto{\pgfqpoint{4.020629in}{2.093064in}}{\pgfqpoint{4.025019in}{2.082465in}}{\pgfqpoint{4.032833in}{2.074652in}}%
\pgfpathcurveto{\pgfqpoint{4.040647in}{2.066838in}}{\pgfqpoint{4.051246in}{2.062448in}}{\pgfqpoint{4.062296in}{2.062448in}}%
\pgfpathclose%
\pgfusepath{stroke,fill}%
\end{pgfscope}%
\begin{pgfscope}%
\pgfpathrectangle{\pgfqpoint{0.583026in}{0.320679in}}{\pgfqpoint{4.650000in}{3.020000in}}%
\pgfusepath{clip}%
\pgfsetbuttcap%
\pgfsetroundjoin%
\definecolor{currentfill}{rgb}{0.121569,0.466667,0.705882}%
\pgfsetfillcolor{currentfill}%
\pgfsetlinewidth{1.003750pt}%
\definecolor{currentstroke}{rgb}{0.121569,0.466667,0.705882}%
\pgfsetstrokecolor{currentstroke}%
\pgfsetdash{}{0pt}%
\pgfpathmoveto{\pgfqpoint{4.330891in}{2.190388in}}%
\pgfpathcurveto{\pgfqpoint{4.341941in}{2.190388in}}{\pgfqpoint{4.352540in}{2.194778in}}{\pgfqpoint{4.360354in}{2.202592in}}%
\pgfpathcurveto{\pgfqpoint{4.368167in}{2.210405in}}{\pgfqpoint{4.372558in}{2.221004in}}{\pgfqpoint{4.372558in}{2.232054in}}%
\pgfpathcurveto{\pgfqpoint{4.372558in}{2.243105in}}{\pgfqpoint{4.368167in}{2.253704in}}{\pgfqpoint{4.360354in}{2.261517in}}%
\pgfpathcurveto{\pgfqpoint{4.352540in}{2.269331in}}{\pgfqpoint{4.341941in}{2.273721in}}{\pgfqpoint{4.330891in}{2.273721in}}%
\pgfpathcurveto{\pgfqpoint{4.319841in}{2.273721in}}{\pgfqpoint{4.309242in}{2.269331in}}{\pgfqpoint{4.301428in}{2.261517in}}%
\pgfpathcurveto{\pgfqpoint{4.293614in}{2.253704in}}{\pgfqpoint{4.289224in}{2.243105in}}{\pgfqpoint{4.289224in}{2.232054in}}%
\pgfpathcurveto{\pgfqpoint{4.289224in}{2.221004in}}{\pgfqpoint{4.293614in}{2.210405in}}{\pgfqpoint{4.301428in}{2.202592in}}%
\pgfpathcurveto{\pgfqpoint{4.309242in}{2.194778in}}{\pgfqpoint{4.319841in}{2.190388in}}{\pgfqpoint{4.330891in}{2.190388in}}%
\pgfpathclose%
\pgfusepath{stroke,fill}%
\end{pgfscope}%
\begin{pgfscope}%
\pgfpathrectangle{\pgfqpoint{0.583026in}{0.320679in}}{\pgfqpoint{4.650000in}{3.020000in}}%
\pgfusepath{clip}%
\pgfsetbuttcap%
\pgfsetroundjoin%
\definecolor{currentfill}{rgb}{0.121569,0.466667,0.705882}%
\pgfsetfillcolor{currentfill}%
\pgfsetlinewidth{1.003750pt}%
\definecolor{currentstroke}{rgb}{0.121569,0.466667,0.705882}%
\pgfsetstrokecolor{currentstroke}%
\pgfsetdash{}{0pt}%
\pgfpathmoveto{\pgfqpoint{4.565912in}{2.309799in}}%
\pgfpathcurveto{\pgfqpoint{4.576962in}{2.309799in}}{\pgfqpoint{4.587561in}{2.314189in}}{\pgfqpoint{4.595374in}{2.322002in}}%
\pgfpathcurveto{\pgfqpoint{4.603188in}{2.329816in}}{\pgfqpoint{4.607578in}{2.340415in}}{\pgfqpoint{4.607578in}{2.351465in}}%
\pgfpathcurveto{\pgfqpoint{4.607578in}{2.362515in}}{\pgfqpoint{4.603188in}{2.373114in}}{\pgfqpoint{4.595374in}{2.380928in}}%
\pgfpathcurveto{\pgfqpoint{4.587561in}{2.388742in}}{\pgfqpoint{4.576962in}{2.393132in}}{\pgfqpoint{4.565912in}{2.393132in}}%
\pgfpathcurveto{\pgfqpoint{4.554861in}{2.393132in}}{\pgfqpoint{4.544262in}{2.388742in}}{\pgfqpoint{4.536449in}{2.380928in}}%
\pgfpathcurveto{\pgfqpoint{4.528635in}{2.373114in}}{\pgfqpoint{4.524245in}{2.362515in}}{\pgfqpoint{4.524245in}{2.351465in}}%
\pgfpathcurveto{\pgfqpoint{4.524245in}{2.340415in}}{\pgfqpoint{4.528635in}{2.329816in}}{\pgfqpoint{4.536449in}{2.322002in}}%
\pgfpathcurveto{\pgfqpoint{4.544262in}{2.314189in}}{\pgfqpoint{4.554861in}{2.309799in}}{\pgfqpoint{4.565912in}{2.309799in}}%
\pgfpathclose%
\pgfusepath{stroke,fill}%
\end{pgfscope}%
\begin{pgfscope}%
\pgfpathrectangle{\pgfqpoint{0.583026in}{0.320679in}}{\pgfqpoint{4.650000in}{3.020000in}}%
\pgfusepath{clip}%
\pgfsetbuttcap%
\pgfsetroundjoin%
\definecolor{currentfill}{rgb}{0.121569,0.466667,0.705882}%
\pgfsetfillcolor{currentfill}%
\pgfsetlinewidth{1.003750pt}%
\definecolor{currentstroke}{rgb}{0.121569,0.466667,0.705882}%
\pgfsetstrokecolor{currentstroke}%
\pgfsetdash{}{0pt}%
\pgfpathmoveto{\pgfqpoint{4.834507in}{2.446268in}}%
\pgfpathcurveto{\pgfqpoint{4.845557in}{2.446268in}}{\pgfqpoint{4.856156in}{2.450658in}}{\pgfqpoint{4.863969in}{2.458472in}}%
\pgfpathcurveto{\pgfqpoint{4.871783in}{2.466286in}}{\pgfqpoint{4.876173in}{2.476885in}}{\pgfqpoint{4.876173in}{2.487935in}}%
\pgfpathcurveto{\pgfqpoint{4.876173in}{2.498985in}}{\pgfqpoint{4.871783in}{2.509584in}}{\pgfqpoint{4.863969in}{2.517397in}}%
\pgfpathcurveto{\pgfqpoint{4.856156in}{2.525211in}}{\pgfqpoint{4.845557in}{2.529601in}}{\pgfqpoint{4.834507in}{2.529601in}}%
\pgfpathcurveto{\pgfqpoint{4.823456in}{2.529601in}}{\pgfqpoint{4.812857in}{2.525211in}}{\pgfqpoint{4.805044in}{2.517397in}}%
\pgfpathcurveto{\pgfqpoint{4.797230in}{2.509584in}}{\pgfqpoint{4.792840in}{2.498985in}}{\pgfqpoint{4.792840in}{2.487935in}}%
\pgfpathcurveto{\pgfqpoint{4.792840in}{2.476885in}}{\pgfqpoint{4.797230in}{2.466286in}}{\pgfqpoint{4.805044in}{2.458472in}}%
\pgfpathcurveto{\pgfqpoint{4.812857in}{2.450658in}}{\pgfqpoint{4.823456in}{2.446268in}}{\pgfqpoint{4.834507in}{2.446268in}}%
\pgfpathclose%
\pgfusepath{stroke,fill}%
\end{pgfscope}%
\begin{pgfscope}%
\pgfpathrectangle{\pgfqpoint{0.583026in}{0.320679in}}{\pgfqpoint{4.650000in}{3.020000in}}%
\pgfusepath{clip}%
\pgfsetbuttcap%
\pgfsetroundjoin%
\definecolor{currentfill}{rgb}{0.121569,0.466667,0.705882}%
\pgfsetfillcolor{currentfill}%
\pgfsetlinewidth{1.003750pt}%
\definecolor{currentstroke}{rgb}{0.121569,0.466667,0.705882}%
\pgfsetstrokecolor{currentstroke}%
\pgfsetdash{}{0pt}%
\pgfpathmoveto{\pgfqpoint{4.979996in}{2.505973in}}%
\pgfpathcurveto{\pgfqpoint{4.991046in}{2.505973in}}{\pgfqpoint{5.001645in}{2.510364in}}{\pgfqpoint{5.009458in}{2.518177in}}%
\pgfpathcurveto{\pgfqpoint{5.017272in}{2.525991in}}{\pgfqpoint{5.021662in}{2.536590in}}{\pgfqpoint{5.021662in}{2.547640in}}%
\pgfpathcurveto{\pgfqpoint{5.021662in}{2.558690in}}{\pgfqpoint{5.017272in}{2.569289in}}{\pgfqpoint{5.009458in}{2.577103in}}%
\pgfpathcurveto{\pgfqpoint{5.001645in}{2.584916in}}{\pgfqpoint{4.991046in}{2.589307in}}{\pgfqpoint{4.979996in}{2.589307in}}%
\pgfpathcurveto{\pgfqpoint{4.968945in}{2.589307in}}{\pgfqpoint{4.958346in}{2.584916in}}{\pgfqpoint{4.950533in}{2.577103in}}%
\pgfpathcurveto{\pgfqpoint{4.942719in}{2.569289in}}{\pgfqpoint{4.938329in}{2.558690in}}{\pgfqpoint{4.938329in}{2.547640in}}%
\pgfpathcurveto{\pgfqpoint{4.938329in}{2.536590in}}{\pgfqpoint{4.942719in}{2.525991in}}{\pgfqpoint{4.950533in}{2.518177in}}%
\pgfpathcurveto{\pgfqpoint{4.958346in}{2.510364in}}{\pgfqpoint{4.968945in}{2.505973in}}{\pgfqpoint{4.979996in}{2.505973in}}%
\pgfpathclose%
\pgfusepath{stroke,fill}%
\end{pgfscope}%
\begin{pgfscope}%
\pgfpathrectangle{\pgfqpoint{0.583026in}{0.320679in}}{\pgfqpoint{4.650000in}{3.020000in}}%
\pgfusepath{clip}%
\pgfsetbuttcap%
\pgfsetroundjoin%
\definecolor{currentfill}{rgb}{1.000000,0.498039,0.054902}%
\pgfsetfillcolor{currentfill}%
\pgfsetlinewidth{1.003750pt}%
\definecolor{currentstroke}{rgb}{1.000000,0.498039,0.054902}%
\pgfsetstrokecolor{currentstroke}%
\pgfsetdash{}{0pt}%
\pgfpathmoveto{\pgfqpoint{0.962261in}{0.510108in}}%
\pgfpathcurveto{\pgfqpoint{0.973311in}{0.510108in}}{\pgfqpoint{0.983911in}{0.514498in}}{\pgfqpoint{0.991724in}{0.522312in}}%
\pgfpathcurveto{\pgfqpoint{0.999538in}{0.530125in}}{\pgfqpoint{1.003928in}{0.540724in}}{\pgfqpoint{1.003928in}{0.551774in}}%
\pgfpathcurveto{\pgfqpoint{1.003928in}{0.562824in}}{\pgfqpoint{0.999538in}{0.573424in}}{\pgfqpoint{0.991724in}{0.581237in}}%
\pgfpathcurveto{\pgfqpoint{0.983911in}{0.589051in}}{\pgfqpoint{0.973311in}{0.593441in}}{\pgfqpoint{0.962261in}{0.593441in}}%
\pgfpathcurveto{\pgfqpoint{0.951211in}{0.593441in}}{\pgfqpoint{0.940612in}{0.589051in}}{\pgfqpoint{0.932799in}{0.581237in}}%
\pgfpathcurveto{\pgfqpoint{0.924985in}{0.573424in}}{\pgfqpoint{0.920595in}{0.562824in}}{\pgfqpoint{0.920595in}{0.551774in}}%
\pgfpathcurveto{\pgfqpoint{0.920595in}{0.540724in}}{\pgfqpoint{0.924985in}{0.530125in}}{\pgfqpoint{0.932799in}{0.522312in}}%
\pgfpathcurveto{\pgfqpoint{0.940612in}{0.514498in}}{\pgfqpoint{0.951211in}{0.510108in}}{\pgfqpoint{0.962261in}{0.510108in}}%
\pgfpathclose%
\pgfusepath{stroke,fill}%
\end{pgfscope}%
\begin{pgfscope}%
\pgfpathrectangle{\pgfqpoint{0.583026in}{0.320679in}}{\pgfqpoint{4.650000in}{3.020000in}}%
\pgfusepath{clip}%
\pgfsetbuttcap%
\pgfsetroundjoin%
\definecolor{currentfill}{rgb}{1.000000,0.498039,0.054902}%
\pgfsetfillcolor{currentfill}%
\pgfsetlinewidth{1.003750pt}%
\definecolor{currentstroke}{rgb}{1.000000,0.498039,0.054902}%
\pgfsetstrokecolor{currentstroke}%
\pgfsetdash{}{0pt}%
\pgfpathmoveto{\pgfqpoint{1.174899in}{0.646577in}}%
\pgfpathcurveto{\pgfqpoint{1.185949in}{0.646577in}}{\pgfqpoint{1.196548in}{0.650967in}}{\pgfqpoint{1.204362in}{0.658781in}}%
\pgfpathcurveto{\pgfqpoint{1.212176in}{0.666595in}}{\pgfqpoint{1.216566in}{0.677194in}}{\pgfqpoint{1.216566in}{0.688244in}}%
\pgfpathcurveto{\pgfqpoint{1.216566in}{0.699294in}}{\pgfqpoint{1.212176in}{0.709893in}}{\pgfqpoint{1.204362in}{0.717707in}}%
\pgfpathcurveto{\pgfqpoint{1.196548in}{0.725520in}}{\pgfqpoint{1.185949in}{0.729910in}}{\pgfqpoint{1.174899in}{0.729910in}}%
\pgfpathcurveto{\pgfqpoint{1.163849in}{0.729910in}}{\pgfqpoint{1.153250in}{0.725520in}}{\pgfqpoint{1.145436in}{0.717707in}}%
\pgfpathcurveto{\pgfqpoint{1.137623in}{0.709893in}}{\pgfqpoint{1.133232in}{0.699294in}}{\pgfqpoint{1.133232in}{0.688244in}}%
\pgfpathcurveto{\pgfqpoint{1.133232in}{0.677194in}}{\pgfqpoint{1.137623in}{0.666595in}}{\pgfqpoint{1.145436in}{0.658781in}}%
\pgfpathcurveto{\pgfqpoint{1.153250in}{0.650967in}}{\pgfqpoint{1.163849in}{0.646577in}}{\pgfqpoint{1.174899in}{0.646577in}}%
\pgfpathclose%
\pgfusepath{stroke,fill}%
\end{pgfscope}%
\begin{pgfscope}%
\pgfpathrectangle{\pgfqpoint{0.583026in}{0.320679in}}{\pgfqpoint{4.650000in}{3.020000in}}%
\pgfusepath{clip}%
\pgfsetbuttcap%
\pgfsetroundjoin%
\definecolor{currentfill}{rgb}{1.000000,0.498039,0.054902}%
\pgfsetfillcolor{currentfill}%
\pgfsetlinewidth{1.003750pt}%
\definecolor{currentstroke}{rgb}{1.000000,0.498039,0.054902}%
\pgfsetstrokecolor{currentstroke}%
\pgfsetdash{}{0pt}%
\pgfpathmoveto{\pgfqpoint{1.533026in}{0.885399in}}%
\pgfpathcurveto{\pgfqpoint{1.544076in}{0.885399in}}{\pgfqpoint{1.554675in}{0.889789in}}{\pgfqpoint{1.562489in}{0.897603in}}%
\pgfpathcurveto{\pgfqpoint{1.570302in}{0.905416in}}{\pgfqpoint{1.574693in}{0.916015in}}{\pgfqpoint{1.574693in}{0.927065in}}%
\pgfpathcurveto{\pgfqpoint{1.574693in}{0.938115in}}{\pgfqpoint{1.570302in}{0.948715in}}{\pgfqpoint{1.562489in}{0.956528in}}%
\pgfpathcurveto{\pgfqpoint{1.554675in}{0.964342in}}{\pgfqpoint{1.544076in}{0.968732in}}{\pgfqpoint{1.533026in}{0.968732in}}%
\pgfpathcurveto{\pgfqpoint{1.521976in}{0.968732in}}{\pgfqpoint{1.511377in}{0.964342in}}{\pgfqpoint{1.503563in}{0.956528in}}%
\pgfpathcurveto{\pgfqpoint{1.495749in}{0.948715in}}{\pgfqpoint{1.491359in}{0.938115in}}{\pgfqpoint{1.491359in}{0.927065in}}%
\pgfpathcurveto{\pgfqpoint{1.491359in}{0.916015in}}{\pgfqpoint{1.495749in}{0.905416in}}{\pgfqpoint{1.503563in}{0.897603in}}%
\pgfpathcurveto{\pgfqpoint{1.511377in}{0.889789in}}{\pgfqpoint{1.521976in}{0.885399in}}{\pgfqpoint{1.533026in}{0.885399in}}%
\pgfpathclose%
\pgfusepath{stroke,fill}%
\end{pgfscope}%
\begin{pgfscope}%
\pgfpathrectangle{\pgfqpoint{0.583026in}{0.320679in}}{\pgfqpoint{4.650000in}{3.020000in}}%
\pgfusepath{clip}%
\pgfsetbuttcap%
\pgfsetroundjoin%
\definecolor{currentfill}{rgb}{1.000000,0.498039,0.054902}%
\pgfsetfillcolor{currentfill}%
\pgfsetlinewidth{1.003750pt}%
\definecolor{currentstroke}{rgb}{1.000000,0.498039,0.054902}%
\pgfsetstrokecolor{currentstroke}%
\pgfsetdash{}{0pt}%
\pgfpathmoveto{\pgfqpoint{1.846387in}{1.081574in}}%
\pgfpathcurveto{\pgfqpoint{1.857437in}{1.081574in}}{\pgfqpoint{1.868036in}{1.085964in}}{\pgfqpoint{1.875849in}{1.093777in}}%
\pgfpathcurveto{\pgfqpoint{1.883663in}{1.101591in}}{\pgfqpoint{1.888053in}{1.112190in}}{\pgfqpoint{1.888053in}{1.123240in}}%
\pgfpathcurveto{\pgfqpoint{1.888053in}{1.134290in}}{\pgfqpoint{1.883663in}{1.144889in}}{\pgfqpoint{1.875849in}{1.152703in}}%
\pgfpathcurveto{\pgfqpoint{1.868036in}{1.160517in}}{\pgfqpoint{1.857437in}{1.164907in}}{\pgfqpoint{1.846387in}{1.164907in}}%
\pgfpathcurveto{\pgfqpoint{1.835337in}{1.164907in}}{\pgfqpoint{1.824738in}{1.160517in}}{\pgfqpoint{1.816924in}{1.152703in}}%
\pgfpathcurveto{\pgfqpoint{1.809110in}{1.144889in}}{\pgfqpoint{1.804720in}{1.134290in}}{\pgfqpoint{1.804720in}{1.123240in}}%
\pgfpathcurveto{\pgfqpoint{1.804720in}{1.112190in}}{\pgfqpoint{1.809110in}{1.101591in}}{\pgfqpoint{1.816924in}{1.093777in}}%
\pgfpathcurveto{\pgfqpoint{1.824738in}{1.085964in}}{\pgfqpoint{1.835337in}{1.081574in}}{\pgfqpoint{1.846387in}{1.081574in}}%
\pgfpathclose%
\pgfusepath{stroke,fill}%
\end{pgfscope}%
\begin{pgfscope}%
\pgfpathrectangle{\pgfqpoint{0.583026in}{0.320679in}}{\pgfqpoint{4.650000in}{3.020000in}}%
\pgfusepath{clip}%
\pgfsetbuttcap%
\pgfsetroundjoin%
\definecolor{currentfill}{rgb}{1.000000,0.498039,0.054902}%
\pgfsetfillcolor{currentfill}%
\pgfsetlinewidth{1.003750pt}%
\definecolor{currentstroke}{rgb}{1.000000,0.498039,0.054902}%
\pgfsetstrokecolor{currentstroke}%
\pgfsetdash{}{0pt}%
\pgfpathmoveto{\pgfqpoint{2.126173in}{1.260690in}}%
\pgfpathcurveto{\pgfqpoint{2.137223in}{1.260690in}}{\pgfqpoint{2.147822in}{1.265080in}}{\pgfqpoint{2.155636in}{1.272894in}}%
\pgfpathcurveto{\pgfqpoint{2.163450in}{1.280707in}}{\pgfqpoint{2.167840in}{1.291306in}}{\pgfqpoint{2.167840in}{1.302356in}}%
\pgfpathcurveto{\pgfqpoint{2.167840in}{1.313406in}}{\pgfqpoint{2.163450in}{1.324006in}}{\pgfqpoint{2.155636in}{1.331819in}}%
\pgfpathcurveto{\pgfqpoint{2.147822in}{1.339633in}}{\pgfqpoint{2.137223in}{1.344023in}}{\pgfqpoint{2.126173in}{1.344023in}}%
\pgfpathcurveto{\pgfqpoint{2.115123in}{1.344023in}}{\pgfqpoint{2.104524in}{1.339633in}}{\pgfqpoint{2.096710in}{1.331819in}}%
\pgfpathcurveto{\pgfqpoint{2.088897in}{1.324006in}}{\pgfqpoint{2.084507in}{1.313406in}}{\pgfqpoint{2.084507in}{1.302356in}}%
\pgfpathcurveto{\pgfqpoint{2.084507in}{1.291306in}}{\pgfqpoint{2.088897in}{1.280707in}}{\pgfqpoint{2.096710in}{1.272894in}}%
\pgfpathcurveto{\pgfqpoint{2.104524in}{1.265080in}}{\pgfqpoint{2.115123in}{1.260690in}}{\pgfqpoint{2.126173in}{1.260690in}}%
\pgfpathclose%
\pgfusepath{stroke,fill}%
\end{pgfscope}%
\begin{pgfscope}%
\pgfpathrectangle{\pgfqpoint{0.583026in}{0.320679in}}{\pgfqpoint{4.650000in}{3.020000in}}%
\pgfusepath{clip}%
\pgfsetbuttcap%
\pgfsetroundjoin%
\definecolor{currentfill}{rgb}{1.000000,0.498039,0.054902}%
\pgfsetfillcolor{currentfill}%
\pgfsetlinewidth{1.003750pt}%
\definecolor{currentstroke}{rgb}{1.000000,0.498039,0.054902}%
\pgfsetstrokecolor{currentstroke}%
\pgfsetdash{}{0pt}%
\pgfpathmoveto{\pgfqpoint{2.350002in}{1.405688in}}%
\pgfpathcurveto{\pgfqpoint{2.361053in}{1.405688in}}{\pgfqpoint{2.371652in}{1.410079in}}{\pgfqpoint{2.379465in}{1.417892in}}%
\pgfpathcurveto{\pgfqpoint{2.387279in}{1.425706in}}{\pgfqpoint{2.391669in}{1.436305in}}{\pgfqpoint{2.391669in}{1.447355in}}%
\pgfpathcurveto{\pgfqpoint{2.391669in}{1.458405in}}{\pgfqpoint{2.387279in}{1.469004in}}{\pgfqpoint{2.379465in}{1.476818in}}%
\pgfpathcurveto{\pgfqpoint{2.371652in}{1.484632in}}{\pgfqpoint{2.361053in}{1.489022in}}{\pgfqpoint{2.350002in}{1.489022in}}%
\pgfpathcurveto{\pgfqpoint{2.338952in}{1.489022in}}{\pgfqpoint{2.328353in}{1.484632in}}{\pgfqpoint{2.320540in}{1.476818in}}%
\pgfpathcurveto{\pgfqpoint{2.312726in}{1.469004in}}{\pgfqpoint{2.308336in}{1.458405in}}{\pgfqpoint{2.308336in}{1.447355in}}%
\pgfpathcurveto{\pgfqpoint{2.308336in}{1.436305in}}{\pgfqpoint{2.312726in}{1.425706in}}{\pgfqpoint{2.320540in}{1.417892in}}%
\pgfpathcurveto{\pgfqpoint{2.328353in}{1.410079in}}{\pgfqpoint{2.338952in}{1.405688in}}{\pgfqpoint{2.350002in}{1.405688in}}%
\pgfpathclose%
\pgfusepath{stroke,fill}%
\end{pgfscope}%
\begin{pgfscope}%
\pgfpathrectangle{\pgfqpoint{0.583026in}{0.320679in}}{\pgfqpoint{4.650000in}{3.020000in}}%
\pgfusepath{clip}%
\pgfsetbuttcap%
\pgfsetroundjoin%
\definecolor{currentfill}{rgb}{1.000000,0.498039,0.054902}%
\pgfsetfillcolor{currentfill}%
\pgfsetlinewidth{1.003750pt}%
\definecolor{currentstroke}{rgb}{1.000000,0.498039,0.054902}%
\pgfsetstrokecolor{currentstroke}%
\pgfsetdash{}{0pt}%
\pgfpathmoveto{\pgfqpoint{2.607406in}{1.559217in}}%
\pgfpathcurveto{\pgfqpoint{2.618456in}{1.559217in}}{\pgfqpoint{2.629055in}{1.563607in}}{\pgfqpoint{2.636869in}{1.571420in}}%
\pgfpathcurveto{\pgfqpoint{2.644682in}{1.579234in}}{\pgfqpoint{2.649073in}{1.589833in}}{\pgfqpoint{2.649073in}{1.600883in}}%
\pgfpathcurveto{\pgfqpoint{2.649073in}{1.611933in}}{\pgfqpoint{2.644682in}{1.622532in}}{\pgfqpoint{2.636869in}{1.630346in}}%
\pgfpathcurveto{\pgfqpoint{2.629055in}{1.638160in}}{\pgfqpoint{2.618456in}{1.642550in}}{\pgfqpoint{2.607406in}{1.642550in}}%
\pgfpathcurveto{\pgfqpoint{2.596356in}{1.642550in}}{\pgfqpoint{2.585757in}{1.638160in}}{\pgfqpoint{2.577943in}{1.630346in}}%
\pgfpathcurveto{\pgfqpoint{2.570130in}{1.622532in}}{\pgfqpoint{2.565739in}{1.611933in}}{\pgfqpoint{2.565739in}{1.600883in}}%
\pgfpathcurveto{\pgfqpoint{2.565739in}{1.589833in}}{\pgfqpoint{2.570130in}{1.579234in}}{\pgfqpoint{2.577943in}{1.571420in}}%
\pgfpathcurveto{\pgfqpoint{2.585757in}{1.563607in}}{\pgfqpoint{2.596356in}{1.559217in}}{\pgfqpoint{2.607406in}{1.559217in}}%
\pgfpathclose%
\pgfusepath{stroke,fill}%
\end{pgfscope}%
\begin{pgfscope}%
\pgfpathrectangle{\pgfqpoint{0.583026in}{0.320679in}}{\pgfqpoint{4.650000in}{3.020000in}}%
\pgfusepath{clip}%
\pgfsetbuttcap%
\pgfsetroundjoin%
\definecolor{currentfill}{rgb}{1.000000,0.498039,0.054902}%
\pgfsetfillcolor{currentfill}%
\pgfsetlinewidth{1.003750pt}%
\definecolor{currentstroke}{rgb}{1.000000,0.498039,0.054902}%
\pgfsetstrokecolor{currentstroke}%
\pgfsetdash{}{0pt}%
\pgfpathmoveto{\pgfqpoint{2.808852in}{1.704215in}}%
\pgfpathcurveto{\pgfqpoint{2.819902in}{1.704215in}}{\pgfqpoint{2.830501in}{1.708606in}}{\pgfqpoint{2.838315in}{1.716419in}}%
\pgfpathcurveto{\pgfqpoint{2.846129in}{1.724233in}}{\pgfqpoint{2.850519in}{1.734832in}}{\pgfqpoint{2.850519in}{1.745882in}}%
\pgfpathcurveto{\pgfqpoint{2.850519in}{1.756932in}}{\pgfqpoint{2.846129in}{1.767531in}}{\pgfqpoint{2.838315in}{1.775345in}}%
\pgfpathcurveto{\pgfqpoint{2.830501in}{1.783158in}}{\pgfqpoint{2.819902in}{1.787549in}}{\pgfqpoint{2.808852in}{1.787549in}}%
\pgfpathcurveto{\pgfqpoint{2.797802in}{1.787549in}}{\pgfqpoint{2.787203in}{1.783158in}}{\pgfqpoint{2.779389in}{1.775345in}}%
\pgfpathcurveto{\pgfqpoint{2.771576in}{1.767531in}}{\pgfqpoint{2.767186in}{1.756932in}}{\pgfqpoint{2.767186in}{1.745882in}}%
\pgfpathcurveto{\pgfqpoint{2.767186in}{1.734832in}}{\pgfqpoint{2.771576in}{1.724233in}}{\pgfqpoint{2.779389in}{1.716419in}}%
\pgfpathcurveto{\pgfqpoint{2.787203in}{1.708606in}}{\pgfqpoint{2.797802in}{1.704215in}}{\pgfqpoint{2.808852in}{1.704215in}}%
\pgfpathclose%
\pgfusepath{stroke,fill}%
\end{pgfscope}%
\begin{pgfscope}%
\pgfpathrectangle{\pgfqpoint{0.583026in}{0.320679in}}{\pgfqpoint{4.650000in}{3.020000in}}%
\pgfusepath{clip}%
\pgfsetbuttcap%
\pgfsetroundjoin%
\definecolor{currentfill}{rgb}{1.000000,0.498039,0.054902}%
\pgfsetfillcolor{currentfill}%
\pgfsetlinewidth{1.003750pt}%
\definecolor{currentstroke}{rgb}{1.000000,0.498039,0.054902}%
\pgfsetstrokecolor{currentstroke}%
\pgfsetdash{}{0pt}%
\pgfpathmoveto{\pgfqpoint{3.021490in}{1.832155in}}%
\pgfpathcurveto{\pgfqpoint{3.032540in}{1.832155in}}{\pgfqpoint{3.043139in}{1.836546in}}{\pgfqpoint{3.050953in}{1.844359in}}%
\pgfpathcurveto{\pgfqpoint{3.058766in}{1.852173in}}{\pgfqpoint{3.063157in}{1.862772in}}{\pgfqpoint{3.063157in}{1.873822in}}%
\pgfpathcurveto{\pgfqpoint{3.063157in}{1.884872in}}{\pgfqpoint{3.058766in}{1.895471in}}{\pgfqpoint{3.050953in}{1.903285in}}%
\pgfpathcurveto{\pgfqpoint{3.043139in}{1.911099in}}{\pgfqpoint{3.032540in}{1.915489in}}{\pgfqpoint{3.021490in}{1.915489in}}%
\pgfpathcurveto{\pgfqpoint{3.010440in}{1.915489in}}{\pgfqpoint{2.999841in}{1.911099in}}{\pgfqpoint{2.992027in}{1.903285in}}%
\pgfpathcurveto{\pgfqpoint{2.984214in}{1.895471in}}{\pgfqpoint{2.979823in}{1.884872in}}{\pgfqpoint{2.979823in}{1.873822in}}%
\pgfpathcurveto{\pgfqpoint{2.979823in}{1.862772in}}{\pgfqpoint{2.984214in}{1.852173in}}{\pgfqpoint{2.992027in}{1.844359in}}%
\pgfpathcurveto{\pgfqpoint{2.999841in}{1.836546in}}{\pgfqpoint{3.010440in}{1.832155in}}{\pgfqpoint{3.021490in}{1.832155in}}%
\pgfpathclose%
\pgfusepath{stroke,fill}%
\end{pgfscope}%
\begin{pgfscope}%
\pgfpathrectangle{\pgfqpoint{0.583026in}{0.320679in}}{\pgfqpoint{4.650000in}{3.020000in}}%
\pgfusepath{clip}%
\pgfsetbuttcap%
\pgfsetroundjoin%
\definecolor{currentfill}{rgb}{1.000000,0.498039,0.054902}%
\pgfsetfillcolor{currentfill}%
\pgfsetlinewidth{1.003750pt}%
\definecolor{currentstroke}{rgb}{1.000000,0.498039,0.054902}%
\pgfsetstrokecolor{currentstroke}%
\pgfsetdash{}{0pt}%
\pgfpathmoveto{\pgfqpoint{3.211745in}{1.968625in}}%
\pgfpathcurveto{\pgfqpoint{3.222795in}{1.968625in}}{\pgfqpoint{3.233394in}{1.973015in}}{\pgfqpoint{3.241208in}{1.980829in}}%
\pgfpathcurveto{\pgfqpoint{3.249021in}{1.988642in}}{\pgfqpoint{3.253412in}{1.999241in}}{\pgfqpoint{3.253412in}{2.010292in}}%
\pgfpathcurveto{\pgfqpoint{3.253412in}{2.021342in}}{\pgfqpoint{3.249021in}{2.031941in}}{\pgfqpoint{3.241208in}{2.039754in}}%
\pgfpathcurveto{\pgfqpoint{3.233394in}{2.047568in}}{\pgfqpoint{3.222795in}{2.051958in}}{\pgfqpoint{3.211745in}{2.051958in}}%
\pgfpathcurveto{\pgfqpoint{3.200695in}{2.051958in}}{\pgfqpoint{3.190096in}{2.047568in}}{\pgfqpoint{3.182282in}{2.039754in}}%
\pgfpathcurveto{\pgfqpoint{3.174468in}{2.031941in}}{\pgfqpoint{3.170078in}{2.021342in}}{\pgfqpoint{3.170078in}{2.010292in}}%
\pgfpathcurveto{\pgfqpoint{3.170078in}{1.999241in}}{\pgfqpoint{3.174468in}{1.988642in}}{\pgfqpoint{3.182282in}{1.980829in}}%
\pgfpathcurveto{\pgfqpoint{3.190096in}{1.973015in}}{\pgfqpoint{3.200695in}{1.968625in}}{\pgfqpoint{3.211745in}{1.968625in}}%
\pgfpathclose%
\pgfusepath{stroke,fill}%
\end{pgfscope}%
\begin{pgfscope}%
\pgfpathrectangle{\pgfqpoint{0.583026in}{0.320679in}}{\pgfqpoint{4.650000in}{3.020000in}}%
\pgfusepath{clip}%
\pgfsetbuttcap%
\pgfsetroundjoin%
\definecolor{currentfill}{rgb}{1.000000,0.498039,0.054902}%
\pgfsetfillcolor{currentfill}%
\pgfsetlinewidth{1.003750pt}%
\definecolor{currentstroke}{rgb}{1.000000,0.498039,0.054902}%
\pgfsetstrokecolor{currentstroke}%
\pgfsetdash{}{0pt}%
\pgfpathmoveto{\pgfqpoint{3.413191in}{2.088036in}}%
\pgfpathcurveto{\pgfqpoint{3.424241in}{2.088036in}}{\pgfqpoint{3.434840in}{2.092426in}}{\pgfqpoint{3.442654in}{2.100240in}}%
\pgfpathcurveto{\pgfqpoint{3.450468in}{2.108053in}}{\pgfqpoint{3.454858in}{2.118652in}}{\pgfqpoint{3.454858in}{2.129702in}}%
\pgfpathcurveto{\pgfqpoint{3.454858in}{2.140753in}}{\pgfqpoint{3.450468in}{2.151352in}}{\pgfqpoint{3.442654in}{2.159165in}}%
\pgfpathcurveto{\pgfqpoint{3.434840in}{2.166979in}}{\pgfqpoint{3.424241in}{2.171369in}}{\pgfqpoint{3.413191in}{2.171369in}}%
\pgfpathcurveto{\pgfqpoint{3.402141in}{2.171369in}}{\pgfqpoint{3.391542in}{2.166979in}}{\pgfqpoint{3.383728in}{2.159165in}}%
\pgfpathcurveto{\pgfqpoint{3.375915in}{2.151352in}}{\pgfqpoint{3.371524in}{2.140753in}}{\pgfqpoint{3.371524in}{2.129702in}}%
\pgfpathcurveto{\pgfqpoint{3.371524in}{2.118652in}}{\pgfqpoint{3.375915in}{2.108053in}}{\pgfqpoint{3.383728in}{2.100240in}}%
\pgfpathcurveto{\pgfqpoint{3.391542in}{2.092426in}}{\pgfqpoint{3.402141in}{2.088036in}}{\pgfqpoint{3.413191in}{2.088036in}}%
\pgfpathclose%
\pgfusepath{stroke,fill}%
\end{pgfscope}%
\begin{pgfscope}%
\pgfpathrectangle{\pgfqpoint{0.583026in}{0.320679in}}{\pgfqpoint{4.650000in}{3.020000in}}%
\pgfusepath{clip}%
\pgfsetbuttcap%
\pgfsetroundjoin%
\definecolor{currentfill}{rgb}{1.000000,0.498039,0.054902}%
\pgfsetfillcolor{currentfill}%
\pgfsetlinewidth{1.003750pt}%
\definecolor{currentstroke}{rgb}{1.000000,0.498039,0.054902}%
\pgfsetstrokecolor{currentstroke}%
\pgfsetdash{}{0pt}%
\pgfpathmoveto{\pgfqpoint{3.659403in}{2.250093in}}%
\pgfpathcurveto{\pgfqpoint{3.670453in}{2.250093in}}{\pgfqpoint{3.681052in}{2.254483in}}{\pgfqpoint{3.688866in}{2.262297in}}%
\pgfpathcurveto{\pgfqpoint{3.696680in}{2.270111in}}{\pgfqpoint{3.701070in}{2.280710in}}{\pgfqpoint{3.701070in}{2.291760in}}%
\pgfpathcurveto{\pgfqpoint{3.701070in}{2.302810in}}{\pgfqpoint{3.696680in}{2.313409in}}{\pgfqpoint{3.688866in}{2.321223in}}%
\pgfpathcurveto{\pgfqpoint{3.681052in}{2.329036in}}{\pgfqpoint{3.670453in}{2.333427in}}{\pgfqpoint{3.659403in}{2.333427in}}%
\pgfpathcurveto{\pgfqpoint{3.648353in}{2.333427in}}{\pgfqpoint{3.637754in}{2.329036in}}{\pgfqpoint{3.629940in}{2.321223in}}%
\pgfpathcurveto{\pgfqpoint{3.622127in}{2.313409in}}{\pgfqpoint{3.617737in}{2.302810in}}{\pgfqpoint{3.617737in}{2.291760in}}%
\pgfpathcurveto{\pgfqpoint{3.617737in}{2.280710in}}{\pgfqpoint{3.622127in}{2.270111in}}{\pgfqpoint{3.629940in}{2.262297in}}%
\pgfpathcurveto{\pgfqpoint{3.637754in}{2.254483in}}{\pgfqpoint{3.648353in}{2.250093in}}{\pgfqpoint{3.659403in}{2.250093in}}%
\pgfpathclose%
\pgfusepath{stroke,fill}%
\end{pgfscope}%
\begin{pgfscope}%
\pgfpathrectangle{\pgfqpoint{0.583026in}{0.320679in}}{\pgfqpoint{4.650000in}{3.020000in}}%
\pgfusepath{clip}%
\pgfsetbuttcap%
\pgfsetroundjoin%
\definecolor{currentfill}{rgb}{1.000000,0.498039,0.054902}%
\pgfsetfillcolor{currentfill}%
\pgfsetlinewidth{1.003750pt}%
\definecolor{currentstroke}{rgb}{1.000000,0.498039,0.054902}%
\pgfsetstrokecolor{currentstroke}%
\pgfsetdash{}{0pt}%
\pgfpathmoveto{\pgfqpoint{3.872041in}{2.395092in}}%
\pgfpathcurveto{\pgfqpoint{3.883091in}{2.395092in}}{\pgfqpoint{3.893690in}{2.399482in}}{\pgfqpoint{3.901504in}{2.407296in}}%
\pgfpathcurveto{\pgfqpoint{3.909317in}{2.415109in}}{\pgfqpoint{3.913708in}{2.425709in}}{\pgfqpoint{3.913708in}{2.436759in}}%
\pgfpathcurveto{\pgfqpoint{3.913708in}{2.447809in}}{\pgfqpoint{3.909317in}{2.458408in}}{\pgfqpoint{3.901504in}{2.466221in}}%
\pgfpathcurveto{\pgfqpoint{3.893690in}{2.474035in}}{\pgfqpoint{3.883091in}{2.478425in}}{\pgfqpoint{3.872041in}{2.478425in}}%
\pgfpathcurveto{\pgfqpoint{3.860991in}{2.478425in}}{\pgfqpoint{3.850392in}{2.474035in}}{\pgfqpoint{3.842578in}{2.466221in}}%
\pgfpathcurveto{\pgfqpoint{3.834765in}{2.458408in}}{\pgfqpoint{3.830374in}{2.447809in}}{\pgfqpoint{3.830374in}{2.436759in}}%
\pgfpathcurveto{\pgfqpoint{3.830374in}{2.425709in}}{\pgfqpoint{3.834765in}{2.415109in}}{\pgfqpoint{3.842578in}{2.407296in}}%
\pgfpathcurveto{\pgfqpoint{3.850392in}{2.399482in}}{\pgfqpoint{3.860991in}{2.395092in}}{\pgfqpoint{3.872041in}{2.395092in}}%
\pgfpathclose%
\pgfusepath{stroke,fill}%
\end{pgfscope}%
\begin{pgfscope}%
\pgfpathrectangle{\pgfqpoint{0.583026in}{0.320679in}}{\pgfqpoint{4.650000in}{3.020000in}}%
\pgfusepath{clip}%
\pgfsetbuttcap%
\pgfsetroundjoin%
\definecolor{currentfill}{rgb}{1.000000,0.498039,0.054902}%
\pgfsetfillcolor{currentfill}%
\pgfsetlinewidth{1.003750pt}%
\definecolor{currentstroke}{rgb}{1.000000,0.498039,0.054902}%
\pgfsetstrokecolor{currentstroke}%
\pgfsetdash{}{0pt}%
\pgfpathmoveto{\pgfqpoint{4.095870in}{2.540091in}}%
\pgfpathcurveto{\pgfqpoint{4.106920in}{2.540091in}}{\pgfqpoint{4.117519in}{2.544481in}}{\pgfqpoint{4.125333in}{2.552295in}}%
\pgfpathcurveto{\pgfqpoint{4.133147in}{2.560108in}}{\pgfqpoint{4.137537in}{2.570707in}}{\pgfqpoint{4.137537in}{2.581757in}}%
\pgfpathcurveto{\pgfqpoint{4.137537in}{2.592808in}}{\pgfqpoint{4.133147in}{2.603407in}}{\pgfqpoint{4.125333in}{2.611220in}}%
\pgfpathcurveto{\pgfqpoint{4.117519in}{2.619034in}}{\pgfqpoint{4.106920in}{2.623424in}}{\pgfqpoint{4.095870in}{2.623424in}}%
\pgfpathcurveto{\pgfqpoint{4.084820in}{2.623424in}}{\pgfqpoint{4.074221in}{2.619034in}}{\pgfqpoint{4.066407in}{2.611220in}}%
\pgfpathcurveto{\pgfqpoint{4.058594in}{2.603407in}}{\pgfqpoint{4.054204in}{2.592808in}}{\pgfqpoint{4.054204in}{2.581757in}}%
\pgfpathcurveto{\pgfqpoint{4.054204in}{2.570707in}}{\pgfqpoint{4.058594in}{2.560108in}}{\pgfqpoint{4.066407in}{2.552295in}}%
\pgfpathcurveto{\pgfqpoint{4.074221in}{2.544481in}}{\pgfqpoint{4.084820in}{2.540091in}}{\pgfqpoint{4.095870in}{2.540091in}}%
\pgfpathclose%
\pgfusepath{stroke,fill}%
\end{pgfscope}%
\begin{pgfscope}%
\pgfpathrectangle{\pgfqpoint{0.583026in}{0.320679in}}{\pgfqpoint{4.650000in}{3.020000in}}%
\pgfusepath{clip}%
\pgfsetbuttcap%
\pgfsetroundjoin%
\definecolor{currentfill}{rgb}{1.000000,0.498039,0.054902}%
\pgfsetfillcolor{currentfill}%
\pgfsetlinewidth{1.003750pt}%
\definecolor{currentstroke}{rgb}{1.000000,0.498039,0.054902}%
\pgfsetstrokecolor{currentstroke}%
\pgfsetdash{}{0pt}%
\pgfpathmoveto{\pgfqpoint{4.364465in}{2.710678in}}%
\pgfpathcurveto{\pgfqpoint{4.375515in}{2.710678in}}{\pgfqpoint{4.386114in}{2.715068in}}{\pgfqpoint{4.393928in}{2.722881in}}%
\pgfpathcurveto{\pgfqpoint{4.401742in}{2.730695in}}{\pgfqpoint{4.406132in}{2.741294in}}{\pgfqpoint{4.406132in}{2.752344in}}%
\pgfpathcurveto{\pgfqpoint{4.406132in}{2.763394in}}{\pgfqpoint{4.401742in}{2.773993in}}{\pgfqpoint{4.393928in}{2.781807in}}%
\pgfpathcurveto{\pgfqpoint{4.386114in}{2.789621in}}{\pgfqpoint{4.375515in}{2.794011in}}{\pgfqpoint{4.364465in}{2.794011in}}%
\pgfpathcurveto{\pgfqpoint{4.353415in}{2.794011in}}{\pgfqpoint{4.342816in}{2.789621in}}{\pgfqpoint{4.335002in}{2.781807in}}%
\pgfpathcurveto{\pgfqpoint{4.327189in}{2.773993in}}{\pgfqpoint{4.322799in}{2.763394in}}{\pgfqpoint{4.322799in}{2.752344in}}%
\pgfpathcurveto{\pgfqpoint{4.322799in}{2.741294in}}{\pgfqpoint{4.327189in}{2.730695in}}{\pgfqpoint{4.335002in}{2.722881in}}%
\pgfpathcurveto{\pgfqpoint{4.342816in}{2.715068in}}{\pgfqpoint{4.353415in}{2.710678in}}{\pgfqpoint{4.364465in}{2.710678in}}%
\pgfpathclose%
\pgfusepath{stroke,fill}%
\end{pgfscope}%
\begin{pgfscope}%
\pgfpathrectangle{\pgfqpoint{0.583026in}{0.320679in}}{\pgfqpoint{4.650000in}{3.020000in}}%
\pgfusepath{clip}%
\pgfsetbuttcap%
\pgfsetroundjoin%
\definecolor{currentfill}{rgb}{1.000000,0.498039,0.054902}%
\pgfsetfillcolor{currentfill}%
\pgfsetlinewidth{1.003750pt}%
\definecolor{currentstroke}{rgb}{1.000000,0.498039,0.054902}%
\pgfsetstrokecolor{currentstroke}%
\pgfsetdash{}{0pt}%
\pgfpathmoveto{\pgfqpoint{4.621869in}{2.881264in}}%
\pgfpathcurveto{\pgfqpoint{4.632919in}{2.881264in}}{\pgfqpoint{4.643518in}{2.885655in}}{\pgfqpoint{4.651332in}{2.893468in}}%
\pgfpathcurveto{\pgfqpoint{4.659145in}{2.901282in}}{\pgfqpoint{4.663535in}{2.911881in}}{\pgfqpoint{4.663535in}{2.922931in}}%
\pgfpathcurveto{\pgfqpoint{4.663535in}{2.933981in}}{\pgfqpoint{4.659145in}{2.944580in}}{\pgfqpoint{4.651332in}{2.952394in}}%
\pgfpathcurveto{\pgfqpoint{4.643518in}{2.960207in}}{\pgfqpoint{4.632919in}{2.964598in}}{\pgfqpoint{4.621869in}{2.964598in}}%
\pgfpathcurveto{\pgfqpoint{4.610819in}{2.964598in}}{\pgfqpoint{4.600220in}{2.960207in}}{\pgfqpoint{4.592406in}{2.952394in}}%
\pgfpathcurveto{\pgfqpoint{4.584592in}{2.944580in}}{\pgfqpoint{4.580202in}{2.933981in}}{\pgfqpoint{4.580202in}{2.922931in}}%
\pgfpathcurveto{\pgfqpoint{4.580202in}{2.911881in}}{\pgfqpoint{4.584592in}{2.901282in}}{\pgfqpoint{4.592406in}{2.893468in}}%
\pgfpathcurveto{\pgfqpoint{4.600220in}{2.885655in}}{\pgfqpoint{4.610819in}{2.881264in}}{\pgfqpoint{4.621869in}{2.881264in}}%
\pgfpathclose%
\pgfusepath{stroke,fill}%
\end{pgfscope}%
\begin{pgfscope}%
\pgfpathrectangle{\pgfqpoint{0.583026in}{0.320679in}}{\pgfqpoint{4.650000in}{3.020000in}}%
\pgfusepath{clip}%
\pgfsetbuttcap%
\pgfsetroundjoin%
\definecolor{currentfill}{rgb}{1.000000,0.498039,0.054902}%
\pgfsetfillcolor{currentfill}%
\pgfsetlinewidth{1.003750pt}%
\definecolor{currentstroke}{rgb}{1.000000,0.498039,0.054902}%
\pgfsetstrokecolor{currentstroke}%
\pgfsetdash{}{0pt}%
\pgfpathmoveto{\pgfqpoint{4.812124in}{3.009205in}}%
\pgfpathcurveto{\pgfqpoint{4.823174in}{3.009205in}}{\pgfqpoint{4.833773in}{3.013595in}}{\pgfqpoint{4.841586in}{3.021408in}}%
\pgfpathcurveto{\pgfqpoint{4.849400in}{3.029222in}}{\pgfqpoint{4.853790in}{3.039821in}}{\pgfqpoint{4.853790in}{3.050871in}}%
\pgfpathcurveto{\pgfqpoint{4.853790in}{3.061921in}}{\pgfqpoint{4.849400in}{3.072520in}}{\pgfqpoint{4.841586in}{3.080334in}}%
\pgfpathcurveto{\pgfqpoint{4.833773in}{3.088148in}}{\pgfqpoint{4.823174in}{3.092538in}}{\pgfqpoint{4.812124in}{3.092538in}}%
\pgfpathcurveto{\pgfqpoint{4.801074in}{3.092538in}}{\pgfqpoint{4.790474in}{3.088148in}}{\pgfqpoint{4.782661in}{3.080334in}}%
\pgfpathcurveto{\pgfqpoint{4.774847in}{3.072520in}}{\pgfqpoint{4.770457in}{3.061921in}}{\pgfqpoint{4.770457in}{3.050871in}}%
\pgfpathcurveto{\pgfqpoint{4.770457in}{3.039821in}}{\pgfqpoint{4.774847in}{3.029222in}}{\pgfqpoint{4.782661in}{3.021408in}}%
\pgfpathcurveto{\pgfqpoint{4.790474in}{3.013595in}}{\pgfqpoint{4.801074in}{3.009205in}}{\pgfqpoint{4.812124in}{3.009205in}}%
\pgfpathclose%
\pgfusepath{stroke,fill}%
\end{pgfscope}%
\begin{pgfscope}%
\pgfpathrectangle{\pgfqpoint{0.583026in}{0.320679in}}{\pgfqpoint{4.650000in}{3.020000in}}%
\pgfusepath{clip}%
\pgfsetbuttcap%
\pgfsetroundjoin%
\definecolor{currentfill}{rgb}{1.000000,0.498039,0.054902}%
\pgfsetfillcolor{currentfill}%
\pgfsetlinewidth{1.003750pt}%
\definecolor{currentstroke}{rgb}{1.000000,0.498039,0.054902}%
\pgfsetstrokecolor{currentstroke}%
\pgfsetdash{}{0pt}%
\pgfpathmoveto{\pgfqpoint{4.979996in}{3.120086in}}%
\pgfpathcurveto{\pgfqpoint{4.991046in}{3.120086in}}{\pgfqpoint{5.001645in}{3.124476in}}{\pgfqpoint{5.009458in}{3.132290in}}%
\pgfpathcurveto{\pgfqpoint{5.017272in}{3.140103in}}{\pgfqpoint{5.021662in}{3.150702in}}{\pgfqpoint{5.021662in}{3.161753in}}%
\pgfpathcurveto{\pgfqpoint{5.021662in}{3.172803in}}{\pgfqpoint{5.017272in}{3.183402in}}{\pgfqpoint{5.009458in}{3.191215in}}%
\pgfpathcurveto{\pgfqpoint{5.001645in}{3.199029in}}{\pgfqpoint{4.991046in}{3.203419in}}{\pgfqpoint{4.979996in}{3.203419in}}%
\pgfpathcurveto{\pgfqpoint{4.968945in}{3.203419in}}{\pgfqpoint{4.958346in}{3.199029in}}{\pgfqpoint{4.950533in}{3.191215in}}%
\pgfpathcurveto{\pgfqpoint{4.942719in}{3.183402in}}{\pgfqpoint{4.938329in}{3.172803in}}{\pgfqpoint{4.938329in}{3.161753in}}%
\pgfpathcurveto{\pgfqpoint{4.938329in}{3.150702in}}{\pgfqpoint{4.942719in}{3.140103in}}{\pgfqpoint{4.950533in}{3.132290in}}%
\pgfpathcurveto{\pgfqpoint{4.958346in}{3.124476in}}{\pgfqpoint{4.968945in}{3.120086in}}{\pgfqpoint{4.979996in}{3.120086in}}%
\pgfpathclose%
\pgfusepath{stroke,fill}%
\end{pgfscope}%
\begin{pgfscope}%
\pgfpathrectangle{\pgfqpoint{0.583026in}{0.320679in}}{\pgfqpoint{4.650000in}{3.020000in}}%
\pgfusepath{clip}%
\pgfsetbuttcap%
\pgfsetroundjoin%
\definecolor{currentfill}{rgb}{0.172549,0.627451,0.172549}%
\pgfsetfillcolor{currentfill}%
\pgfsetlinewidth{1.003750pt}%
\definecolor{currentstroke}{rgb}{0.172549,0.627451,0.172549}%
\pgfsetstrokecolor{currentstroke}%
\pgfsetdash{}{0pt}%
\pgfpathmoveto{\pgfqpoint{1.107750in}{0.475990in}}%
\pgfpathcurveto{\pgfqpoint{1.118800in}{0.475990in}}{\pgfqpoint{1.129400in}{0.480381in}}{\pgfqpoint{1.137213in}{0.488194in}}%
\pgfpathcurveto{\pgfqpoint{1.145027in}{0.496008in}}{\pgfqpoint{1.149417in}{0.506607in}}{\pgfqpoint{1.149417in}{0.517657in}}%
\pgfpathcurveto{\pgfqpoint{1.149417in}{0.528707in}}{\pgfqpoint{1.145027in}{0.539306in}}{\pgfqpoint{1.137213in}{0.547120in}}%
\pgfpathcurveto{\pgfqpoint{1.129400in}{0.554933in}}{\pgfqpoint{1.118800in}{0.559324in}}{\pgfqpoint{1.107750in}{0.559324in}}%
\pgfpathcurveto{\pgfqpoint{1.096700in}{0.559324in}}{\pgfqpoint{1.086101in}{0.554933in}}{\pgfqpoint{1.078288in}{0.547120in}}%
\pgfpathcurveto{\pgfqpoint{1.070474in}{0.539306in}}{\pgfqpoint{1.066084in}{0.528707in}}{\pgfqpoint{1.066084in}{0.517657in}}%
\pgfpathcurveto{\pgfqpoint{1.066084in}{0.506607in}}{\pgfqpoint{1.070474in}{0.496008in}}{\pgfqpoint{1.078288in}{0.488194in}}%
\pgfpathcurveto{\pgfqpoint{1.086101in}{0.480381in}}{\pgfqpoint{1.096700in}{0.475990in}}{\pgfqpoint{1.107750in}{0.475990in}}%
\pgfpathclose%
\pgfusepath{stroke,fill}%
\end{pgfscope}%
\begin{pgfscope}%
\pgfpathrectangle{\pgfqpoint{0.583026in}{0.320679in}}{\pgfqpoint{4.650000in}{3.020000in}}%
\pgfusepath{clip}%
\pgfsetbuttcap%
\pgfsetroundjoin%
\definecolor{currentfill}{rgb}{0.172549,0.627451,0.172549}%
\pgfsetfillcolor{currentfill}%
\pgfsetlinewidth{1.003750pt}%
\definecolor{currentstroke}{rgb}{0.172549,0.627451,0.172549}%
\pgfsetstrokecolor{currentstroke}%
\pgfsetdash{}{0pt}%
\pgfpathmoveto{\pgfqpoint{1.342771in}{0.535696in}}%
\pgfpathcurveto{\pgfqpoint{1.353821in}{0.535696in}}{\pgfqpoint{1.364420in}{0.540086in}}{\pgfqpoint{1.372234in}{0.547900in}}%
\pgfpathcurveto{\pgfqpoint{1.380047in}{0.555713in}}{\pgfqpoint{1.384438in}{0.566312in}}{\pgfqpoint{1.384438in}{0.577362in}}%
\pgfpathcurveto{\pgfqpoint{1.384438in}{0.588413in}}{\pgfqpoint{1.380047in}{0.599012in}}{\pgfqpoint{1.372234in}{0.606825in}}%
\pgfpathcurveto{\pgfqpoint{1.364420in}{0.614639in}}{\pgfqpoint{1.353821in}{0.619029in}}{\pgfqpoint{1.342771in}{0.619029in}}%
\pgfpathcurveto{\pgfqpoint{1.331721in}{0.619029in}}{\pgfqpoint{1.321122in}{0.614639in}}{\pgfqpoint{1.313308in}{0.606825in}}%
\pgfpathcurveto{\pgfqpoint{1.305495in}{0.599012in}}{\pgfqpoint{1.301104in}{0.588413in}}{\pgfqpoint{1.301104in}{0.577362in}}%
\pgfpathcurveto{\pgfqpoint{1.301104in}{0.566312in}}{\pgfqpoint{1.305495in}{0.555713in}}{\pgfqpoint{1.313308in}{0.547900in}}%
\pgfpathcurveto{\pgfqpoint{1.321122in}{0.540086in}}{\pgfqpoint{1.331721in}{0.535696in}}{\pgfqpoint{1.342771in}{0.535696in}}%
\pgfpathclose%
\pgfusepath{stroke,fill}%
\end{pgfscope}%
\begin{pgfscope}%
\pgfpathrectangle{\pgfqpoint{0.583026in}{0.320679in}}{\pgfqpoint{4.650000in}{3.020000in}}%
\pgfusepath{clip}%
\pgfsetbuttcap%
\pgfsetroundjoin%
\definecolor{currentfill}{rgb}{0.172549,0.627451,0.172549}%
\pgfsetfillcolor{currentfill}%
\pgfsetlinewidth{1.003750pt}%
\definecolor{currentstroke}{rgb}{0.172549,0.627451,0.172549}%
\pgfsetstrokecolor{currentstroke}%
\pgfsetdash{}{0pt}%
\pgfpathmoveto{\pgfqpoint{1.600175in}{0.603930in}}%
\pgfpathcurveto{\pgfqpoint{1.611225in}{0.603930in}}{\pgfqpoint{1.621824in}{0.608321in}}{\pgfqpoint{1.629637in}{0.616134in}}%
\pgfpathcurveto{\pgfqpoint{1.637451in}{0.623948in}}{\pgfqpoint{1.641841in}{0.634547in}}{\pgfqpoint{1.641841in}{0.645597in}}%
\pgfpathcurveto{\pgfqpoint{1.641841in}{0.656647in}}{\pgfqpoint{1.637451in}{0.667246in}}{\pgfqpoint{1.629637in}{0.675060in}}%
\pgfpathcurveto{\pgfqpoint{1.621824in}{0.682874in}}{\pgfqpoint{1.611225in}{0.687264in}}{\pgfqpoint{1.600175in}{0.687264in}}%
\pgfpathcurveto{\pgfqpoint{1.589124in}{0.687264in}}{\pgfqpoint{1.578525in}{0.682874in}}{\pgfqpoint{1.570712in}{0.675060in}}%
\pgfpathcurveto{\pgfqpoint{1.562898in}{0.667246in}}{\pgfqpoint{1.558508in}{0.656647in}}{\pgfqpoint{1.558508in}{0.645597in}}%
\pgfpathcurveto{\pgfqpoint{1.558508in}{0.634547in}}{\pgfqpoint{1.562898in}{0.623948in}}{\pgfqpoint{1.570712in}{0.616134in}}%
\pgfpathcurveto{\pgfqpoint{1.578525in}{0.608321in}}{\pgfqpoint{1.589124in}{0.603930in}}{\pgfqpoint{1.600175in}{0.603930in}}%
\pgfpathclose%
\pgfusepath{stroke,fill}%
\end{pgfscope}%
\begin{pgfscope}%
\pgfpathrectangle{\pgfqpoint{0.583026in}{0.320679in}}{\pgfqpoint{4.650000in}{3.020000in}}%
\pgfusepath{clip}%
\pgfsetbuttcap%
\pgfsetroundjoin%
\definecolor{currentfill}{rgb}{0.172549,0.627451,0.172549}%
\pgfsetfillcolor{currentfill}%
\pgfsetlinewidth{1.003750pt}%
\definecolor{currentstroke}{rgb}{0.172549,0.627451,0.172549}%
\pgfsetstrokecolor{currentstroke}%
\pgfsetdash{}{0pt}%
\pgfpathmoveto{\pgfqpoint{1.891153in}{0.680695in}}%
\pgfpathcurveto{\pgfqpoint{1.902203in}{0.680695in}}{\pgfqpoint{1.912802in}{0.685085in}}{\pgfqpoint{1.920615in}{0.692898in}}%
\pgfpathcurveto{\pgfqpoint{1.928429in}{0.700712in}}{\pgfqpoint{1.932819in}{0.711311in}}{\pgfqpoint{1.932819in}{0.722361in}}%
\pgfpathcurveto{\pgfqpoint{1.932819in}{0.733411in}}{\pgfqpoint{1.928429in}{0.744010in}}{\pgfqpoint{1.920615in}{0.751824in}}%
\pgfpathcurveto{\pgfqpoint{1.912802in}{0.759638in}}{\pgfqpoint{1.902203in}{0.764028in}}{\pgfqpoint{1.891153in}{0.764028in}}%
\pgfpathcurveto{\pgfqpoint{1.880102in}{0.764028in}}{\pgfqpoint{1.869503in}{0.759638in}}{\pgfqpoint{1.861690in}{0.751824in}}%
\pgfpathcurveto{\pgfqpoint{1.853876in}{0.744010in}}{\pgfqpoint{1.849486in}{0.733411in}}{\pgfqpoint{1.849486in}{0.722361in}}%
\pgfpathcurveto{\pgfqpoint{1.849486in}{0.711311in}}{\pgfqpoint{1.853876in}{0.700712in}}{\pgfqpoint{1.861690in}{0.692898in}}%
\pgfpathcurveto{\pgfqpoint{1.869503in}{0.685085in}}{\pgfqpoint{1.880102in}{0.680695in}}{\pgfqpoint{1.891153in}{0.680695in}}%
\pgfpathclose%
\pgfusepath{stroke,fill}%
\end{pgfscope}%
\begin{pgfscope}%
\pgfpathrectangle{\pgfqpoint{0.583026in}{0.320679in}}{\pgfqpoint{4.650000in}{3.020000in}}%
\pgfusepath{clip}%
\pgfsetbuttcap%
\pgfsetroundjoin%
\definecolor{currentfill}{rgb}{0.172549,0.627451,0.172549}%
\pgfsetfillcolor{currentfill}%
\pgfsetlinewidth{1.003750pt}%
\definecolor{currentstroke}{rgb}{0.172549,0.627451,0.172549}%
\pgfsetstrokecolor{currentstroke}%
\pgfsetdash{}{0pt}%
\pgfpathmoveto{\pgfqpoint{2.114982in}{0.740400in}}%
\pgfpathcurveto{\pgfqpoint{2.126032in}{0.740400in}}{\pgfqpoint{2.136631in}{0.744790in}}{\pgfqpoint{2.144445in}{0.752604in}}%
\pgfpathcurveto{\pgfqpoint{2.152258in}{0.760417in}}{\pgfqpoint{2.156648in}{0.771016in}}{\pgfqpoint{2.156648in}{0.782067in}}%
\pgfpathcurveto{\pgfqpoint{2.156648in}{0.793117in}}{\pgfqpoint{2.152258in}{0.803716in}}{\pgfqpoint{2.144445in}{0.811529in}}%
\pgfpathcurveto{\pgfqpoint{2.136631in}{0.819343in}}{\pgfqpoint{2.126032in}{0.823733in}}{\pgfqpoint{2.114982in}{0.823733in}}%
\pgfpathcurveto{\pgfqpoint{2.103932in}{0.823733in}}{\pgfqpoint{2.093333in}{0.819343in}}{\pgfqpoint{2.085519in}{0.811529in}}%
\pgfpathcurveto{\pgfqpoint{2.077705in}{0.803716in}}{\pgfqpoint{2.073315in}{0.793117in}}{\pgfqpoint{2.073315in}{0.782067in}}%
\pgfpathcurveto{\pgfqpoint{2.073315in}{0.771016in}}{\pgfqpoint{2.077705in}{0.760417in}}{\pgfqpoint{2.085519in}{0.752604in}}%
\pgfpathcurveto{\pgfqpoint{2.093333in}{0.744790in}}{\pgfqpoint{2.103932in}{0.740400in}}{\pgfqpoint{2.114982in}{0.740400in}}%
\pgfpathclose%
\pgfusepath{stroke,fill}%
\end{pgfscope}%
\begin{pgfscope}%
\pgfpathrectangle{\pgfqpoint{0.583026in}{0.320679in}}{\pgfqpoint{4.650000in}{3.020000in}}%
\pgfusepath{clip}%
\pgfsetbuttcap%
\pgfsetroundjoin%
\definecolor{currentfill}{rgb}{0.172549,0.627451,0.172549}%
\pgfsetfillcolor{currentfill}%
\pgfsetlinewidth{1.003750pt}%
\definecolor{currentstroke}{rgb}{0.172549,0.627451,0.172549}%
\pgfsetstrokecolor{currentstroke}%
\pgfsetdash{}{0pt}%
\pgfpathmoveto{\pgfqpoint{2.327619in}{0.791576in}}%
\pgfpathcurveto{\pgfqpoint{2.338670in}{0.791576in}}{\pgfqpoint{2.349269in}{0.795966in}}{\pgfqpoint{2.357082in}{0.803780in}}%
\pgfpathcurveto{\pgfqpoint{2.364896in}{0.811593in}}{\pgfqpoint{2.369286in}{0.822192in}}{\pgfqpoint{2.369286in}{0.833243in}}%
\pgfpathcurveto{\pgfqpoint{2.369286in}{0.844293in}}{\pgfqpoint{2.364896in}{0.854892in}}{\pgfqpoint{2.357082in}{0.862705in}}%
\pgfpathcurveto{\pgfqpoint{2.349269in}{0.870519in}}{\pgfqpoint{2.338670in}{0.874909in}}{\pgfqpoint{2.327619in}{0.874909in}}%
\pgfpathcurveto{\pgfqpoint{2.316569in}{0.874909in}}{\pgfqpoint{2.305970in}{0.870519in}}{\pgfqpoint{2.298157in}{0.862705in}}%
\pgfpathcurveto{\pgfqpoint{2.290343in}{0.854892in}}{\pgfqpoint{2.285953in}{0.844293in}}{\pgfqpoint{2.285953in}{0.833243in}}%
\pgfpathcurveto{\pgfqpoint{2.285953in}{0.822192in}}{\pgfqpoint{2.290343in}{0.811593in}}{\pgfqpoint{2.298157in}{0.803780in}}%
\pgfpathcurveto{\pgfqpoint{2.305970in}{0.795966in}}{\pgfqpoint{2.316569in}{0.791576in}}{\pgfqpoint{2.327619in}{0.791576in}}%
\pgfpathclose%
\pgfusepath{stroke,fill}%
\end{pgfscope}%
\begin{pgfscope}%
\pgfpathrectangle{\pgfqpoint{0.583026in}{0.320679in}}{\pgfqpoint{4.650000in}{3.020000in}}%
\pgfusepath{clip}%
\pgfsetbuttcap%
\pgfsetroundjoin%
\definecolor{currentfill}{rgb}{0.172549,0.627451,0.172549}%
\pgfsetfillcolor{currentfill}%
\pgfsetlinewidth{1.003750pt}%
\definecolor{currentstroke}{rgb}{0.172549,0.627451,0.172549}%
\pgfsetstrokecolor{currentstroke}%
\pgfsetdash{}{0pt}%
\pgfpathmoveto{\pgfqpoint{2.573832in}{0.859811in}}%
\pgfpathcurveto{\pgfqpoint{2.584882in}{0.859811in}}{\pgfqpoint{2.595481in}{0.864201in}}{\pgfqpoint{2.603294in}{0.872015in}}%
\pgfpathcurveto{\pgfqpoint{2.611108in}{0.879828in}}{\pgfqpoint{2.615498in}{0.890427in}}{\pgfqpoint{2.615498in}{0.901477in}}%
\pgfpathcurveto{\pgfqpoint{2.615498in}{0.912527in}}{\pgfqpoint{2.611108in}{0.923126in}}{\pgfqpoint{2.603294in}{0.930940in}}%
\pgfpathcurveto{\pgfqpoint{2.595481in}{0.938754in}}{\pgfqpoint{2.584882in}{0.943144in}}{\pgfqpoint{2.573832in}{0.943144in}}%
\pgfpathcurveto{\pgfqpoint{2.562781in}{0.943144in}}{\pgfqpoint{2.552182in}{0.938754in}}{\pgfqpoint{2.544369in}{0.930940in}}%
\pgfpathcurveto{\pgfqpoint{2.536555in}{0.923126in}}{\pgfqpoint{2.532165in}{0.912527in}}{\pgfqpoint{2.532165in}{0.901477in}}%
\pgfpathcurveto{\pgfqpoint{2.532165in}{0.890427in}}{\pgfqpoint{2.536555in}{0.879828in}}{\pgfqpoint{2.544369in}{0.872015in}}%
\pgfpathcurveto{\pgfqpoint{2.552182in}{0.864201in}}{\pgfqpoint{2.562781in}{0.859811in}}{\pgfqpoint{2.573832in}{0.859811in}}%
\pgfpathclose%
\pgfusepath{stroke,fill}%
\end{pgfscope}%
\begin{pgfscope}%
\pgfpathrectangle{\pgfqpoint{0.583026in}{0.320679in}}{\pgfqpoint{4.650000in}{3.020000in}}%
\pgfusepath{clip}%
\pgfsetbuttcap%
\pgfsetroundjoin%
\definecolor{currentfill}{rgb}{0.172549,0.627451,0.172549}%
\pgfsetfillcolor{currentfill}%
\pgfsetlinewidth{1.003750pt}%
\definecolor{currentstroke}{rgb}{0.172549,0.627451,0.172549}%
\pgfsetstrokecolor{currentstroke}%
\pgfsetdash{}{0pt}%
\pgfpathmoveto{\pgfqpoint{2.820044in}{0.919516in}}%
\pgfpathcurveto{\pgfqpoint{2.831094in}{0.919516in}}{\pgfqpoint{2.841693in}{0.923906in}}{\pgfqpoint{2.849507in}{0.931720in}}%
\pgfpathcurveto{\pgfqpoint{2.857320in}{0.939534in}}{\pgfqpoint{2.861710in}{0.950133in}}{\pgfqpoint{2.861710in}{0.961183in}}%
\pgfpathcurveto{\pgfqpoint{2.861710in}{0.972233in}}{\pgfqpoint{2.857320in}{0.982832in}}{\pgfqpoint{2.849507in}{0.990645in}}%
\pgfpathcurveto{\pgfqpoint{2.841693in}{0.998459in}}{\pgfqpoint{2.831094in}{1.002849in}}{\pgfqpoint{2.820044in}{1.002849in}}%
\pgfpathcurveto{\pgfqpoint{2.808994in}{1.002849in}}{\pgfqpoint{2.798395in}{0.998459in}}{\pgfqpoint{2.790581in}{0.990645in}}%
\pgfpathcurveto{\pgfqpoint{2.782767in}{0.982832in}}{\pgfqpoint{2.778377in}{0.972233in}}{\pgfqpoint{2.778377in}{0.961183in}}%
\pgfpathcurveto{\pgfqpoint{2.778377in}{0.950133in}}{\pgfqpoint{2.782767in}{0.939534in}}{\pgfqpoint{2.790581in}{0.931720in}}%
\pgfpathcurveto{\pgfqpoint{2.798395in}{0.923906in}}{\pgfqpoint{2.808994in}{0.919516in}}{\pgfqpoint{2.820044in}{0.919516in}}%
\pgfpathclose%
\pgfusepath{stroke,fill}%
\end{pgfscope}%
\begin{pgfscope}%
\pgfpathrectangle{\pgfqpoint{0.583026in}{0.320679in}}{\pgfqpoint{4.650000in}{3.020000in}}%
\pgfusepath{clip}%
\pgfsetbuttcap%
\pgfsetroundjoin%
\definecolor{currentfill}{rgb}{0.172549,0.627451,0.172549}%
\pgfsetfillcolor{currentfill}%
\pgfsetlinewidth{1.003750pt}%
\definecolor{currentstroke}{rgb}{0.172549,0.627451,0.172549}%
\pgfsetstrokecolor{currentstroke}%
\pgfsetdash{}{0pt}%
\pgfpathmoveto{\pgfqpoint{3.043873in}{0.970692in}}%
\pgfpathcurveto{\pgfqpoint{3.054923in}{0.970692in}}{\pgfqpoint{3.065522in}{0.975082in}}{\pgfqpoint{3.073336in}{0.982896in}}%
\pgfpathcurveto{\pgfqpoint{3.081149in}{0.990710in}}{\pgfqpoint{3.085540in}{1.001309in}}{\pgfqpoint{3.085540in}{1.012359in}}%
\pgfpathcurveto{\pgfqpoint{3.085540in}{1.023409in}}{\pgfqpoint{3.081149in}{1.034008in}}{\pgfqpoint{3.073336in}{1.041822in}}%
\pgfpathcurveto{\pgfqpoint{3.065522in}{1.049635in}}{\pgfqpoint{3.054923in}{1.054025in}}{\pgfqpoint{3.043873in}{1.054025in}}%
\pgfpathcurveto{\pgfqpoint{3.032823in}{1.054025in}}{\pgfqpoint{3.022224in}{1.049635in}}{\pgfqpoint{3.014410in}{1.041822in}}%
\pgfpathcurveto{\pgfqpoint{3.006597in}{1.034008in}}{\pgfqpoint{3.002206in}{1.023409in}}{\pgfqpoint{3.002206in}{1.012359in}}%
\pgfpathcurveto{\pgfqpoint{3.002206in}{1.001309in}}{\pgfqpoint{3.006597in}{0.990710in}}{\pgfqpoint{3.014410in}{0.982896in}}%
\pgfpathcurveto{\pgfqpoint{3.022224in}{0.975082in}}{\pgfqpoint{3.032823in}{0.970692in}}{\pgfqpoint{3.043873in}{0.970692in}}%
\pgfpathclose%
\pgfusepath{stroke,fill}%
\end{pgfscope}%
\begin{pgfscope}%
\pgfpathrectangle{\pgfqpoint{0.583026in}{0.320679in}}{\pgfqpoint{4.650000in}{3.020000in}}%
\pgfusepath{clip}%
\pgfsetbuttcap%
\pgfsetroundjoin%
\definecolor{currentfill}{rgb}{0.172549,0.627451,0.172549}%
\pgfsetfillcolor{currentfill}%
\pgfsetlinewidth{1.003750pt}%
\definecolor{currentstroke}{rgb}{0.172549,0.627451,0.172549}%
\pgfsetstrokecolor{currentstroke}%
\pgfsetdash{}{0pt}%
\pgfpathmoveto{\pgfqpoint{3.290085in}{1.047456in}}%
\pgfpathcurveto{\pgfqpoint{3.301135in}{1.047456in}}{\pgfqpoint{3.311734in}{1.051846in}}{\pgfqpoint{3.319548in}{1.059660in}}%
\pgfpathcurveto{\pgfqpoint{3.327361in}{1.067474in}}{\pgfqpoint{3.331752in}{1.078073in}}{\pgfqpoint{3.331752in}{1.089123in}}%
\pgfpathcurveto{\pgfqpoint{3.331752in}{1.100173in}}{\pgfqpoint{3.327361in}{1.110772in}}{\pgfqpoint{3.319548in}{1.118586in}}%
\pgfpathcurveto{\pgfqpoint{3.311734in}{1.126399in}}{\pgfqpoint{3.301135in}{1.130789in}}{\pgfqpoint{3.290085in}{1.130789in}}%
\pgfpathcurveto{\pgfqpoint{3.279035in}{1.130789in}}{\pgfqpoint{3.268436in}{1.126399in}}{\pgfqpoint{3.260622in}{1.118586in}}%
\pgfpathcurveto{\pgfqpoint{3.252809in}{1.110772in}}{\pgfqpoint{3.248418in}{1.100173in}}{\pgfqpoint{3.248418in}{1.089123in}}%
\pgfpathcurveto{\pgfqpoint{3.248418in}{1.078073in}}{\pgfqpoint{3.252809in}{1.067474in}}{\pgfqpoint{3.260622in}{1.059660in}}%
\pgfpathcurveto{\pgfqpoint{3.268436in}{1.051846in}}{\pgfqpoint{3.279035in}{1.047456in}}{\pgfqpoint{3.290085in}{1.047456in}}%
\pgfpathclose%
\pgfusepath{stroke,fill}%
\end{pgfscope}%
\begin{pgfscope}%
\pgfpathrectangle{\pgfqpoint{0.583026in}{0.320679in}}{\pgfqpoint{4.650000in}{3.020000in}}%
\pgfusepath{clip}%
\pgfsetbuttcap%
\pgfsetroundjoin%
\definecolor{currentfill}{rgb}{0.172549,0.627451,0.172549}%
\pgfsetfillcolor{currentfill}%
\pgfsetlinewidth{1.003750pt}%
\definecolor{currentstroke}{rgb}{0.172549,0.627451,0.172549}%
\pgfsetstrokecolor{currentstroke}%
\pgfsetdash{}{0pt}%
\pgfpathmoveto{\pgfqpoint{3.569872in}{1.098632in}}%
\pgfpathcurveto{\pgfqpoint{3.580922in}{1.098632in}}{\pgfqpoint{3.591521in}{1.103022in}}{\pgfqpoint{3.599334in}{1.110836in}}%
\pgfpathcurveto{\pgfqpoint{3.607148in}{1.118650in}}{\pgfqpoint{3.611538in}{1.129249in}}{\pgfqpoint{3.611538in}{1.140299in}}%
\pgfpathcurveto{\pgfqpoint{3.611538in}{1.151349in}}{\pgfqpoint{3.607148in}{1.161948in}}{\pgfqpoint{3.599334in}{1.169762in}}%
\pgfpathcurveto{\pgfqpoint{3.591521in}{1.177575in}}{\pgfqpoint{3.580922in}{1.181966in}}{\pgfqpoint{3.569872in}{1.181966in}}%
\pgfpathcurveto{\pgfqpoint{3.558821in}{1.181966in}}{\pgfqpoint{3.548222in}{1.177575in}}{\pgfqpoint{3.540409in}{1.169762in}}%
\pgfpathcurveto{\pgfqpoint{3.532595in}{1.161948in}}{\pgfqpoint{3.528205in}{1.151349in}}{\pgfqpoint{3.528205in}{1.140299in}}%
\pgfpathcurveto{\pgfqpoint{3.528205in}{1.129249in}}{\pgfqpoint{3.532595in}{1.118650in}}{\pgfqpoint{3.540409in}{1.110836in}}%
\pgfpathcurveto{\pgfqpoint{3.548222in}{1.103022in}}{\pgfqpoint{3.558821in}{1.098632in}}{\pgfqpoint{3.569872in}{1.098632in}}%
\pgfpathclose%
\pgfusepath{stroke,fill}%
\end{pgfscope}%
\begin{pgfscope}%
\pgfpathrectangle{\pgfqpoint{0.583026in}{0.320679in}}{\pgfqpoint{4.650000in}{3.020000in}}%
\pgfusepath{clip}%
\pgfsetbuttcap%
\pgfsetroundjoin%
\definecolor{currentfill}{rgb}{0.172549,0.627451,0.172549}%
\pgfsetfillcolor{currentfill}%
\pgfsetlinewidth{1.003750pt}%
\definecolor{currentstroke}{rgb}{0.172549,0.627451,0.172549}%
\pgfsetstrokecolor{currentstroke}%
\pgfsetdash{}{0pt}%
\pgfpathmoveto{\pgfqpoint{3.726552in}{1.141279in}}%
\pgfpathcurveto{\pgfqpoint{3.737602in}{1.141279in}}{\pgfqpoint{3.748201in}{1.145669in}}{\pgfqpoint{3.756015in}{1.153483in}}%
\pgfpathcurveto{\pgfqpoint{3.763828in}{1.161296in}}{\pgfqpoint{3.768219in}{1.171895in}}{\pgfqpoint{3.768219in}{1.182946in}}%
\pgfpathcurveto{\pgfqpoint{3.768219in}{1.193996in}}{\pgfqpoint{3.763828in}{1.204595in}}{\pgfqpoint{3.756015in}{1.212408in}}%
\pgfpathcurveto{\pgfqpoint{3.748201in}{1.220222in}}{\pgfqpoint{3.737602in}{1.224612in}}{\pgfqpoint{3.726552in}{1.224612in}}%
\pgfpathcurveto{\pgfqpoint{3.715502in}{1.224612in}}{\pgfqpoint{3.704903in}{1.220222in}}{\pgfqpoint{3.697089in}{1.212408in}}%
\pgfpathcurveto{\pgfqpoint{3.689276in}{1.204595in}}{\pgfqpoint{3.684885in}{1.193996in}}{\pgfqpoint{3.684885in}{1.182946in}}%
\pgfpathcurveto{\pgfqpoint{3.684885in}{1.171895in}}{\pgfqpoint{3.689276in}{1.161296in}}{\pgfqpoint{3.697089in}{1.153483in}}%
\pgfpathcurveto{\pgfqpoint{3.704903in}{1.145669in}}{\pgfqpoint{3.715502in}{1.141279in}}{\pgfqpoint{3.726552in}{1.141279in}}%
\pgfpathclose%
\pgfusepath{stroke,fill}%
\end{pgfscope}%
\begin{pgfscope}%
\pgfpathrectangle{\pgfqpoint{0.583026in}{0.320679in}}{\pgfqpoint{4.650000in}{3.020000in}}%
\pgfusepath{clip}%
\pgfsetbuttcap%
\pgfsetroundjoin%
\definecolor{currentfill}{rgb}{0.172549,0.627451,0.172549}%
\pgfsetfillcolor{currentfill}%
\pgfsetlinewidth{1.003750pt}%
\definecolor{currentstroke}{rgb}{0.172549,0.627451,0.172549}%
\pgfsetstrokecolor{currentstroke}%
\pgfsetdash{}{0pt}%
\pgfpathmoveto{\pgfqpoint{3.894424in}{1.183926in}}%
\pgfpathcurveto{\pgfqpoint{3.905474in}{1.183926in}}{\pgfqpoint{3.916073in}{1.188316in}}{\pgfqpoint{3.923887in}{1.196129in}}%
\pgfpathcurveto{\pgfqpoint{3.931700in}{1.203943in}}{\pgfqpoint{3.936091in}{1.214542in}}{\pgfqpoint{3.936091in}{1.225592in}}%
\pgfpathcurveto{\pgfqpoint{3.936091in}{1.236642in}}{\pgfqpoint{3.931700in}{1.247241in}}{\pgfqpoint{3.923887in}{1.255055in}}%
\pgfpathcurveto{\pgfqpoint{3.916073in}{1.262869in}}{\pgfqpoint{3.905474in}{1.267259in}}{\pgfqpoint{3.894424in}{1.267259in}}%
\pgfpathcurveto{\pgfqpoint{3.883374in}{1.267259in}}{\pgfqpoint{3.872775in}{1.262869in}}{\pgfqpoint{3.864961in}{1.255055in}}%
\pgfpathcurveto{\pgfqpoint{3.857148in}{1.247241in}}{\pgfqpoint{3.852757in}{1.236642in}}{\pgfqpoint{3.852757in}{1.225592in}}%
\pgfpathcurveto{\pgfqpoint{3.852757in}{1.214542in}}{\pgfqpoint{3.857148in}{1.203943in}}{\pgfqpoint{3.864961in}{1.196129in}}%
\pgfpathcurveto{\pgfqpoint{3.872775in}{1.188316in}}{\pgfqpoint{3.883374in}{1.183926in}}{\pgfqpoint{3.894424in}{1.183926in}}%
\pgfpathclose%
\pgfusepath{stroke,fill}%
\end{pgfscope}%
\begin{pgfscope}%
\pgfpathrectangle{\pgfqpoint{0.583026in}{0.320679in}}{\pgfqpoint{4.650000in}{3.020000in}}%
\pgfusepath{clip}%
\pgfsetbuttcap%
\pgfsetroundjoin%
\definecolor{currentfill}{rgb}{0.172549,0.627451,0.172549}%
\pgfsetfillcolor{currentfill}%
\pgfsetlinewidth{1.003750pt}%
\definecolor{currentstroke}{rgb}{0.172549,0.627451,0.172549}%
\pgfsetstrokecolor{currentstroke}%
\pgfsetdash{}{0pt}%
\pgfpathmoveto{\pgfqpoint{4.095870in}{1.243631in}}%
\pgfpathcurveto{\pgfqpoint{4.106920in}{1.243631in}}{\pgfqpoint{4.117519in}{1.248021in}}{\pgfqpoint{4.125333in}{1.255835in}}%
\pgfpathcurveto{\pgfqpoint{4.133147in}{1.263648in}}{\pgfqpoint{4.137537in}{1.274248in}}{\pgfqpoint{4.137537in}{1.285298in}}%
\pgfpathcurveto{\pgfqpoint{4.137537in}{1.296348in}}{\pgfqpoint{4.133147in}{1.306947in}}{\pgfqpoint{4.125333in}{1.314760in}}%
\pgfpathcurveto{\pgfqpoint{4.117519in}{1.322574in}}{\pgfqpoint{4.106920in}{1.326964in}}{\pgfqpoint{4.095870in}{1.326964in}}%
\pgfpathcurveto{\pgfqpoint{4.084820in}{1.326964in}}{\pgfqpoint{4.074221in}{1.322574in}}{\pgfqpoint{4.066407in}{1.314760in}}%
\pgfpathcurveto{\pgfqpoint{4.058594in}{1.306947in}}{\pgfqpoint{4.054204in}{1.296348in}}{\pgfqpoint{4.054204in}{1.285298in}}%
\pgfpathcurveto{\pgfqpoint{4.054204in}{1.274248in}}{\pgfqpoint{4.058594in}{1.263648in}}{\pgfqpoint{4.066407in}{1.255835in}}%
\pgfpathcurveto{\pgfqpoint{4.074221in}{1.248021in}}{\pgfqpoint{4.084820in}{1.243631in}}{\pgfqpoint{4.095870in}{1.243631in}}%
\pgfpathclose%
\pgfusepath{stroke,fill}%
\end{pgfscope}%
\begin{pgfscope}%
\pgfpathrectangle{\pgfqpoint{0.583026in}{0.320679in}}{\pgfqpoint{4.650000in}{3.020000in}}%
\pgfusepath{clip}%
\pgfsetbuttcap%
\pgfsetroundjoin%
\definecolor{currentfill}{rgb}{0.172549,0.627451,0.172549}%
\pgfsetfillcolor{currentfill}%
\pgfsetlinewidth{1.003750pt}%
\definecolor{currentstroke}{rgb}{0.172549,0.627451,0.172549}%
\pgfsetstrokecolor{currentstroke}%
\pgfsetdash{}{0pt}%
\pgfpathmoveto{\pgfqpoint{4.409231in}{1.311866in}}%
\pgfpathcurveto{\pgfqpoint{4.420281in}{1.311866in}}{\pgfqpoint{4.430880in}{1.316256in}}{\pgfqpoint{4.438694in}{1.324070in}}%
\pgfpathcurveto{\pgfqpoint{4.446507in}{1.331883in}}{\pgfqpoint{4.450898in}{1.342482in}}{\pgfqpoint{4.450898in}{1.353532in}}%
\pgfpathcurveto{\pgfqpoint{4.450898in}{1.364583in}}{\pgfqpoint{4.446507in}{1.375182in}}{\pgfqpoint{4.438694in}{1.382995in}}%
\pgfpathcurveto{\pgfqpoint{4.430880in}{1.390809in}}{\pgfqpoint{4.420281in}{1.395199in}}{\pgfqpoint{4.409231in}{1.395199in}}%
\pgfpathcurveto{\pgfqpoint{4.398181in}{1.395199in}}{\pgfqpoint{4.387582in}{1.390809in}}{\pgfqpoint{4.379768in}{1.382995in}}%
\pgfpathcurveto{\pgfqpoint{4.371955in}{1.375182in}}{\pgfqpoint{4.367564in}{1.364583in}}{\pgfqpoint{4.367564in}{1.353532in}}%
\pgfpathcurveto{\pgfqpoint{4.367564in}{1.342482in}}{\pgfqpoint{4.371955in}{1.331883in}}{\pgfqpoint{4.379768in}{1.324070in}}%
\pgfpathcurveto{\pgfqpoint{4.387582in}{1.316256in}}{\pgfqpoint{4.398181in}{1.311866in}}{\pgfqpoint{4.409231in}{1.311866in}}%
\pgfpathclose%
\pgfusepath{stroke,fill}%
\end{pgfscope}%
\begin{pgfscope}%
\pgfpathrectangle{\pgfqpoint{0.583026in}{0.320679in}}{\pgfqpoint{4.650000in}{3.020000in}}%
\pgfusepath{clip}%
\pgfsetbuttcap%
\pgfsetroundjoin%
\definecolor{currentfill}{rgb}{0.172549,0.627451,0.172549}%
\pgfsetfillcolor{currentfill}%
\pgfsetlinewidth{1.003750pt}%
\definecolor{currentstroke}{rgb}{0.172549,0.627451,0.172549}%
\pgfsetstrokecolor{currentstroke}%
\pgfsetdash{}{0pt}%
\pgfpathmoveto{\pgfqpoint{4.677826in}{1.388630in}}%
\pgfpathcurveto{\pgfqpoint{4.688876in}{1.388630in}}{\pgfqpoint{4.699475in}{1.393020in}}{\pgfqpoint{4.707289in}{1.400834in}}%
\pgfpathcurveto{\pgfqpoint{4.715103in}{1.408647in}}{\pgfqpoint{4.719493in}{1.419246in}}{\pgfqpoint{4.719493in}{1.430296in}}%
\pgfpathcurveto{\pgfqpoint{4.719493in}{1.441347in}}{\pgfqpoint{4.715103in}{1.451946in}}{\pgfqpoint{4.707289in}{1.459759in}}%
\pgfpathcurveto{\pgfqpoint{4.699475in}{1.467573in}}{\pgfqpoint{4.688876in}{1.471963in}}{\pgfqpoint{4.677826in}{1.471963in}}%
\pgfpathcurveto{\pgfqpoint{4.666776in}{1.471963in}}{\pgfqpoint{4.656177in}{1.467573in}}{\pgfqpoint{4.648363in}{1.459759in}}%
\pgfpathcurveto{\pgfqpoint{4.640550in}{1.451946in}}{\pgfqpoint{4.636159in}{1.441347in}}{\pgfqpoint{4.636159in}{1.430296in}}%
\pgfpathcurveto{\pgfqpoint{4.636159in}{1.419246in}}{\pgfqpoint{4.640550in}{1.408647in}}{\pgfqpoint{4.648363in}{1.400834in}}%
\pgfpathcurveto{\pgfqpoint{4.656177in}{1.393020in}}{\pgfqpoint{4.666776in}{1.388630in}}{\pgfqpoint{4.677826in}{1.388630in}}%
\pgfpathclose%
\pgfusepath{stroke,fill}%
\end{pgfscope}%
\begin{pgfscope}%
\pgfpathrectangle{\pgfqpoint{0.583026in}{0.320679in}}{\pgfqpoint{4.650000in}{3.020000in}}%
\pgfusepath{clip}%
\pgfsetbuttcap%
\pgfsetroundjoin%
\definecolor{currentfill}{rgb}{0.172549,0.627451,0.172549}%
\pgfsetfillcolor{currentfill}%
\pgfsetlinewidth{1.003750pt}%
\definecolor{currentstroke}{rgb}{0.172549,0.627451,0.172549}%
\pgfsetstrokecolor{currentstroke}%
\pgfsetdash{}{0pt}%
\pgfpathmoveto{\pgfqpoint{4.879272in}{1.448335in}}%
\pgfpathcurveto{\pgfqpoint{4.890323in}{1.448335in}}{\pgfqpoint{4.900922in}{1.452725in}}{\pgfqpoint{4.908735in}{1.460539in}}%
\pgfpathcurveto{\pgfqpoint{4.916549in}{1.468353in}}{\pgfqpoint{4.920939in}{1.478952in}}{\pgfqpoint{4.920939in}{1.490002in}}%
\pgfpathcurveto{\pgfqpoint{4.920939in}{1.501052in}}{\pgfqpoint{4.916549in}{1.511651in}}{\pgfqpoint{4.908735in}{1.519465in}}%
\pgfpathcurveto{\pgfqpoint{4.900922in}{1.527278in}}{\pgfqpoint{4.890323in}{1.531669in}}{\pgfqpoint{4.879272in}{1.531669in}}%
\pgfpathcurveto{\pgfqpoint{4.868222in}{1.531669in}}{\pgfqpoint{4.857623in}{1.527278in}}{\pgfqpoint{4.849810in}{1.519465in}}%
\pgfpathcurveto{\pgfqpoint{4.841996in}{1.511651in}}{\pgfqpoint{4.837606in}{1.501052in}}{\pgfqpoint{4.837606in}{1.490002in}}%
\pgfpathcurveto{\pgfqpoint{4.837606in}{1.478952in}}{\pgfqpoint{4.841996in}{1.468353in}}{\pgfqpoint{4.849810in}{1.460539in}}%
\pgfpathcurveto{\pgfqpoint{4.857623in}{1.452725in}}{\pgfqpoint{4.868222in}{1.448335in}}{\pgfqpoint{4.879272in}{1.448335in}}%
\pgfpathclose%
\pgfusepath{stroke,fill}%
\end{pgfscope}%
\begin{pgfscope}%
\pgfpathrectangle{\pgfqpoint{0.583026in}{0.320679in}}{\pgfqpoint{4.650000in}{3.020000in}}%
\pgfusepath{clip}%
\pgfsetbuttcap%
\pgfsetroundjoin%
\definecolor{currentfill}{rgb}{0.172549,0.627451,0.172549}%
\pgfsetfillcolor{currentfill}%
\pgfsetlinewidth{1.003750pt}%
\definecolor{currentstroke}{rgb}{0.172549,0.627451,0.172549}%
\pgfsetstrokecolor{currentstroke}%
\pgfsetdash{}{0pt}%
\pgfpathmoveto{\pgfqpoint{4.979996in}{1.465394in}}%
\pgfpathcurveto{\pgfqpoint{4.991046in}{1.465394in}}{\pgfqpoint{5.001645in}{1.469784in}}{\pgfqpoint{5.009458in}{1.477598in}}%
\pgfpathcurveto{\pgfqpoint{5.017272in}{1.485411in}}{\pgfqpoint{5.021662in}{1.496010in}}{\pgfqpoint{5.021662in}{1.507061in}}%
\pgfpathcurveto{\pgfqpoint{5.021662in}{1.518111in}}{\pgfqpoint{5.017272in}{1.528710in}}{\pgfqpoint{5.009458in}{1.536523in}}%
\pgfpathcurveto{\pgfqpoint{5.001645in}{1.544337in}}{\pgfqpoint{4.991046in}{1.548727in}}{\pgfqpoint{4.979996in}{1.548727in}}%
\pgfpathcurveto{\pgfqpoint{4.968945in}{1.548727in}}{\pgfqpoint{4.958346in}{1.544337in}}{\pgfqpoint{4.950533in}{1.536523in}}%
\pgfpathcurveto{\pgfqpoint{4.942719in}{1.528710in}}{\pgfqpoint{4.938329in}{1.518111in}}{\pgfqpoint{4.938329in}{1.507061in}}%
\pgfpathcurveto{\pgfqpoint{4.938329in}{1.496010in}}{\pgfqpoint{4.942719in}{1.485411in}}{\pgfqpoint{4.950533in}{1.477598in}}%
\pgfpathcurveto{\pgfqpoint{4.958346in}{1.469784in}}{\pgfqpoint{4.968945in}{1.465394in}}{\pgfqpoint{4.979996in}{1.465394in}}%
\pgfpathclose%
\pgfusepath{stroke,fill}%
\end{pgfscope}%
\begin{pgfscope}%
\pgfpathrectangle{\pgfqpoint{0.583026in}{0.320679in}}{\pgfqpoint{4.650000in}{3.020000in}}%
\pgfusepath{clip}%
\pgfsetbuttcap%
\pgfsetroundjoin%
\definecolor{currentfill}{rgb}{0.839216,0.152941,0.156863}%
\pgfsetfillcolor{currentfill}%
\pgfsetlinewidth{1.003750pt}%
\definecolor{currentstroke}{rgb}{0.839216,0.152941,0.156863}%
\pgfsetstrokecolor{currentstroke}%
\pgfsetdash{}{0pt}%
\pgfpathmoveto{\pgfqpoint{1.040602in}{0.527166in}}%
\pgfpathcurveto{\pgfqpoint{1.051652in}{0.527166in}}{\pgfqpoint{1.062251in}{0.531557in}}{\pgfqpoint{1.070064in}{0.539370in}}%
\pgfpathcurveto{\pgfqpoint{1.077878in}{0.547184in}}{\pgfqpoint{1.082268in}{0.557783in}}{\pgfqpoint{1.082268in}{0.568833in}}%
\pgfpathcurveto{\pgfqpoint{1.082268in}{0.579883in}}{\pgfqpoint{1.077878in}{0.590482in}}{\pgfqpoint{1.070064in}{0.598296in}}%
\pgfpathcurveto{\pgfqpoint{1.062251in}{0.606109in}}{\pgfqpoint{1.051652in}{0.610500in}}{\pgfqpoint{1.040602in}{0.610500in}}%
\pgfpathcurveto{\pgfqpoint{1.029551in}{0.610500in}}{\pgfqpoint{1.018952in}{0.606109in}}{\pgfqpoint{1.011139in}{0.598296in}}%
\pgfpathcurveto{\pgfqpoint{1.003325in}{0.590482in}}{\pgfqpoint{0.998935in}{0.579883in}}{\pgfqpoint{0.998935in}{0.568833in}}%
\pgfpathcurveto{\pgfqpoint{0.998935in}{0.557783in}}{\pgfqpoint{1.003325in}{0.547184in}}{\pgfqpoint{1.011139in}{0.539370in}}%
\pgfpathcurveto{\pgfqpoint{1.018952in}{0.531557in}}{\pgfqpoint{1.029551in}{0.527166in}}{\pgfqpoint{1.040602in}{0.527166in}}%
\pgfpathclose%
\pgfusepath{stroke,fill}%
\end{pgfscope}%
\begin{pgfscope}%
\pgfpathrectangle{\pgfqpoint{0.583026in}{0.320679in}}{\pgfqpoint{4.650000in}{3.020000in}}%
\pgfusepath{clip}%
\pgfsetbuttcap%
\pgfsetroundjoin%
\definecolor{currentfill}{rgb}{0.839216,0.152941,0.156863}%
\pgfsetfillcolor{currentfill}%
\pgfsetlinewidth{1.003750pt}%
\definecolor{currentstroke}{rgb}{0.839216,0.152941,0.156863}%
\pgfsetstrokecolor{currentstroke}%
\pgfsetdash{}{0pt}%
\pgfpathmoveto{\pgfqpoint{1.342771in}{0.680695in}}%
\pgfpathcurveto{\pgfqpoint{1.353821in}{0.680695in}}{\pgfqpoint{1.364420in}{0.685085in}}{\pgfqpoint{1.372234in}{0.692898in}}%
\pgfpathcurveto{\pgfqpoint{1.380047in}{0.700712in}}{\pgfqpoint{1.384438in}{0.711311in}}{\pgfqpoint{1.384438in}{0.722361in}}%
\pgfpathcurveto{\pgfqpoint{1.384438in}{0.733411in}}{\pgfqpoint{1.380047in}{0.744010in}}{\pgfqpoint{1.372234in}{0.751824in}}%
\pgfpathcurveto{\pgfqpoint{1.364420in}{0.759638in}}{\pgfqpoint{1.353821in}{0.764028in}}{\pgfqpoint{1.342771in}{0.764028in}}%
\pgfpathcurveto{\pgfqpoint{1.331721in}{0.764028in}}{\pgfqpoint{1.321122in}{0.759638in}}{\pgfqpoint{1.313308in}{0.751824in}}%
\pgfpathcurveto{\pgfqpoint{1.305495in}{0.744010in}}{\pgfqpoint{1.301104in}{0.733411in}}{\pgfqpoint{1.301104in}{0.722361in}}%
\pgfpathcurveto{\pgfqpoint{1.301104in}{0.711311in}}{\pgfqpoint{1.305495in}{0.700712in}}{\pgfqpoint{1.313308in}{0.692898in}}%
\pgfpathcurveto{\pgfqpoint{1.321122in}{0.685085in}}{\pgfqpoint{1.331721in}{0.680695in}}{\pgfqpoint{1.342771in}{0.680695in}}%
\pgfpathclose%
\pgfusepath{stroke,fill}%
\end{pgfscope}%
\begin{pgfscope}%
\pgfpathrectangle{\pgfqpoint{0.583026in}{0.320679in}}{\pgfqpoint{4.650000in}{3.020000in}}%
\pgfusepath{clip}%
\pgfsetbuttcap%
\pgfsetroundjoin%
\definecolor{currentfill}{rgb}{0.839216,0.152941,0.156863}%
\pgfsetfillcolor{currentfill}%
\pgfsetlinewidth{1.003750pt}%
\definecolor{currentstroke}{rgb}{0.839216,0.152941,0.156863}%
\pgfsetstrokecolor{currentstroke}%
\pgfsetdash{}{0pt}%
\pgfpathmoveto{\pgfqpoint{1.600175in}{0.808635in}}%
\pgfpathcurveto{\pgfqpoint{1.611225in}{0.808635in}}{\pgfqpoint{1.621824in}{0.813025in}}{\pgfqpoint{1.629637in}{0.820839in}}%
\pgfpathcurveto{\pgfqpoint{1.637451in}{0.828652in}}{\pgfqpoint{1.641841in}{0.839251in}}{\pgfqpoint{1.641841in}{0.850301in}}%
\pgfpathcurveto{\pgfqpoint{1.641841in}{0.861351in}}{\pgfqpoint{1.637451in}{0.871950in}}{\pgfqpoint{1.629637in}{0.879764in}}%
\pgfpathcurveto{\pgfqpoint{1.621824in}{0.887578in}}{\pgfqpoint{1.611225in}{0.891968in}}{\pgfqpoint{1.600175in}{0.891968in}}%
\pgfpathcurveto{\pgfqpoint{1.589124in}{0.891968in}}{\pgfqpoint{1.578525in}{0.887578in}}{\pgfqpoint{1.570712in}{0.879764in}}%
\pgfpathcurveto{\pgfqpoint{1.562898in}{0.871950in}}{\pgfqpoint{1.558508in}{0.861351in}}{\pgfqpoint{1.558508in}{0.850301in}}%
\pgfpathcurveto{\pgfqpoint{1.558508in}{0.839251in}}{\pgfqpoint{1.562898in}{0.828652in}}{\pgfqpoint{1.570712in}{0.820839in}}%
\pgfpathcurveto{\pgfqpoint{1.578525in}{0.813025in}}{\pgfqpoint{1.589124in}{0.808635in}}{\pgfqpoint{1.600175in}{0.808635in}}%
\pgfpathclose%
\pgfusepath{stroke,fill}%
\end{pgfscope}%
\begin{pgfscope}%
\pgfpathrectangle{\pgfqpoint{0.583026in}{0.320679in}}{\pgfqpoint{4.650000in}{3.020000in}}%
\pgfusepath{clip}%
\pgfsetbuttcap%
\pgfsetroundjoin%
\definecolor{currentfill}{rgb}{0.839216,0.152941,0.156863}%
\pgfsetfillcolor{currentfill}%
\pgfsetlinewidth{1.003750pt}%
\definecolor{currentstroke}{rgb}{0.839216,0.152941,0.156863}%
\pgfsetstrokecolor{currentstroke}%
\pgfsetdash{}{0pt}%
\pgfpathmoveto{\pgfqpoint{1.857578in}{0.936575in}}%
\pgfpathcurveto{\pgfqpoint{1.868628in}{0.936575in}}{\pgfqpoint{1.879227in}{0.940965in}}{\pgfqpoint{1.887041in}{0.948779in}}%
\pgfpathcurveto{\pgfqpoint{1.894855in}{0.956592in}}{\pgfqpoint{1.899245in}{0.967191in}}{\pgfqpoint{1.899245in}{0.978241in}}%
\pgfpathcurveto{\pgfqpoint{1.899245in}{0.989292in}}{\pgfqpoint{1.894855in}{0.999891in}}{\pgfqpoint{1.887041in}{1.007704in}}%
\pgfpathcurveto{\pgfqpoint{1.879227in}{1.015518in}}{\pgfqpoint{1.868628in}{1.019908in}}{\pgfqpoint{1.857578in}{1.019908in}}%
\pgfpathcurveto{\pgfqpoint{1.846528in}{1.019908in}}{\pgfqpoint{1.835929in}{1.015518in}}{\pgfqpoint{1.828115in}{1.007704in}}%
\pgfpathcurveto{\pgfqpoint{1.820302in}{0.999891in}}{\pgfqpoint{1.815912in}{0.989292in}}{\pgfqpoint{1.815912in}{0.978241in}}%
\pgfpathcurveto{\pgfqpoint{1.815912in}{0.967191in}}{\pgfqpoint{1.820302in}{0.956592in}}{\pgfqpoint{1.828115in}{0.948779in}}%
\pgfpathcurveto{\pgfqpoint{1.835929in}{0.940965in}}{\pgfqpoint{1.846528in}{0.936575in}}{\pgfqpoint{1.857578in}{0.936575in}}%
\pgfpathclose%
\pgfusepath{stroke,fill}%
\end{pgfscope}%
\begin{pgfscope}%
\pgfpathrectangle{\pgfqpoint{0.583026in}{0.320679in}}{\pgfqpoint{4.650000in}{3.020000in}}%
\pgfusepath{clip}%
\pgfsetbuttcap%
\pgfsetroundjoin%
\definecolor{currentfill}{rgb}{0.839216,0.152941,0.156863}%
\pgfsetfillcolor{currentfill}%
\pgfsetlinewidth{1.003750pt}%
\definecolor{currentstroke}{rgb}{0.839216,0.152941,0.156863}%
\pgfsetstrokecolor{currentstroke}%
\pgfsetdash{}{0pt}%
\pgfpathmoveto{\pgfqpoint{2.047833in}{1.038927in}}%
\pgfpathcurveto{\pgfqpoint{2.058883in}{1.038927in}}{\pgfqpoint{2.069482in}{1.043317in}}{\pgfqpoint{2.077296in}{1.051131in}}%
\pgfpathcurveto{\pgfqpoint{2.085109in}{1.058944in}}{\pgfqpoint{2.089500in}{1.069543in}}{\pgfqpoint{2.089500in}{1.080593in}}%
\pgfpathcurveto{\pgfqpoint{2.089500in}{1.091644in}}{\pgfqpoint{2.085109in}{1.102243in}}{\pgfqpoint{2.077296in}{1.110056in}}%
\pgfpathcurveto{\pgfqpoint{2.069482in}{1.117870in}}{\pgfqpoint{2.058883in}{1.122260in}}{\pgfqpoint{2.047833in}{1.122260in}}%
\pgfpathcurveto{\pgfqpoint{2.036783in}{1.122260in}}{\pgfqpoint{2.026184in}{1.117870in}}{\pgfqpoint{2.018370in}{1.110056in}}%
\pgfpathcurveto{\pgfqpoint{2.010557in}{1.102243in}}{\pgfqpoint{2.006166in}{1.091644in}}{\pgfqpoint{2.006166in}{1.080593in}}%
\pgfpathcurveto{\pgfqpoint{2.006166in}{1.069543in}}{\pgfqpoint{2.010557in}{1.058944in}}{\pgfqpoint{2.018370in}{1.051131in}}%
\pgfpathcurveto{\pgfqpoint{2.026184in}{1.043317in}}{\pgfqpoint{2.036783in}{1.038927in}}{\pgfqpoint{2.047833in}{1.038927in}}%
\pgfpathclose%
\pgfusepath{stroke,fill}%
\end{pgfscope}%
\begin{pgfscope}%
\pgfpathrectangle{\pgfqpoint{0.583026in}{0.320679in}}{\pgfqpoint{4.650000in}{3.020000in}}%
\pgfusepath{clip}%
\pgfsetbuttcap%
\pgfsetroundjoin%
\definecolor{currentfill}{rgb}{0.839216,0.152941,0.156863}%
\pgfsetfillcolor{currentfill}%
\pgfsetlinewidth{1.003750pt}%
\definecolor{currentstroke}{rgb}{0.839216,0.152941,0.156863}%
\pgfsetstrokecolor{currentstroke}%
\pgfsetdash{}{0pt}%
\pgfpathmoveto{\pgfqpoint{2.316428in}{1.175396in}}%
\pgfpathcurveto{\pgfqpoint{2.327478in}{1.175396in}}{\pgfqpoint{2.338077in}{1.179787in}}{\pgfqpoint{2.345891in}{1.187600in}}%
\pgfpathcurveto{\pgfqpoint{2.353704in}{1.195414in}}{\pgfqpoint{2.358095in}{1.206013in}}{\pgfqpoint{2.358095in}{1.217063in}}%
\pgfpathcurveto{\pgfqpoint{2.358095in}{1.228113in}}{\pgfqpoint{2.353704in}{1.238712in}}{\pgfqpoint{2.345891in}{1.246526in}}%
\pgfpathcurveto{\pgfqpoint{2.338077in}{1.254339in}}{\pgfqpoint{2.327478in}{1.258730in}}{\pgfqpoint{2.316428in}{1.258730in}}%
\pgfpathcurveto{\pgfqpoint{2.305378in}{1.258730in}}{\pgfqpoint{2.294779in}{1.254339in}}{\pgfqpoint{2.286965in}{1.246526in}}%
\pgfpathcurveto{\pgfqpoint{2.279152in}{1.238712in}}{\pgfqpoint{2.274761in}{1.228113in}}{\pgfqpoint{2.274761in}{1.217063in}}%
\pgfpathcurveto{\pgfqpoint{2.274761in}{1.206013in}}{\pgfqpoint{2.279152in}{1.195414in}}{\pgfqpoint{2.286965in}{1.187600in}}%
\pgfpathcurveto{\pgfqpoint{2.294779in}{1.179787in}}{\pgfqpoint{2.305378in}{1.175396in}}{\pgfqpoint{2.316428in}{1.175396in}}%
\pgfpathclose%
\pgfusepath{stroke,fill}%
\end{pgfscope}%
\begin{pgfscope}%
\pgfpathrectangle{\pgfqpoint{0.583026in}{0.320679in}}{\pgfqpoint{4.650000in}{3.020000in}}%
\pgfusepath{clip}%
\pgfsetbuttcap%
\pgfsetroundjoin%
\definecolor{currentfill}{rgb}{0.839216,0.152941,0.156863}%
\pgfsetfillcolor{currentfill}%
\pgfsetlinewidth{1.003750pt}%
\definecolor{currentstroke}{rgb}{0.839216,0.152941,0.156863}%
\pgfsetstrokecolor{currentstroke}%
\pgfsetdash{}{0pt}%
\pgfpathmoveto{\pgfqpoint{2.562640in}{1.294807in}}%
\pgfpathcurveto{\pgfqpoint{2.573690in}{1.294807in}}{\pgfqpoint{2.584289in}{1.299197in}}{\pgfqpoint{2.592103in}{1.307011in}}%
\pgfpathcurveto{\pgfqpoint{2.599917in}{1.314825in}}{\pgfqpoint{2.604307in}{1.325424in}}{\pgfqpoint{2.604307in}{1.336474in}}%
\pgfpathcurveto{\pgfqpoint{2.604307in}{1.347524in}}{\pgfqpoint{2.599917in}{1.358123in}}{\pgfqpoint{2.592103in}{1.365936in}}%
\pgfpathcurveto{\pgfqpoint{2.584289in}{1.373750in}}{\pgfqpoint{2.573690in}{1.378140in}}{\pgfqpoint{2.562640in}{1.378140in}}%
\pgfpathcurveto{\pgfqpoint{2.551590in}{1.378140in}}{\pgfqpoint{2.540991in}{1.373750in}}{\pgfqpoint{2.533177in}{1.365936in}}%
\pgfpathcurveto{\pgfqpoint{2.525364in}{1.358123in}}{\pgfqpoint{2.520973in}{1.347524in}}{\pgfqpoint{2.520973in}{1.336474in}}%
\pgfpathcurveto{\pgfqpoint{2.520973in}{1.325424in}}{\pgfqpoint{2.525364in}{1.314825in}}{\pgfqpoint{2.533177in}{1.307011in}}%
\pgfpathcurveto{\pgfqpoint{2.540991in}{1.299197in}}{\pgfqpoint{2.551590in}{1.294807in}}{\pgfqpoint{2.562640in}{1.294807in}}%
\pgfpathclose%
\pgfusepath{stroke,fill}%
\end{pgfscope}%
\begin{pgfscope}%
\pgfpathrectangle{\pgfqpoint{0.583026in}{0.320679in}}{\pgfqpoint{4.650000in}{3.020000in}}%
\pgfusepath{clip}%
\pgfsetbuttcap%
\pgfsetroundjoin%
\definecolor{currentfill}{rgb}{0.839216,0.152941,0.156863}%
\pgfsetfillcolor{currentfill}%
\pgfsetlinewidth{1.003750pt}%
\definecolor{currentstroke}{rgb}{0.839216,0.152941,0.156863}%
\pgfsetstrokecolor{currentstroke}%
\pgfsetdash{}{0pt}%
\pgfpathmoveto{\pgfqpoint{2.775278in}{1.397159in}}%
\pgfpathcurveto{\pgfqpoint{2.786328in}{1.397159in}}{\pgfqpoint{2.796927in}{1.401549in}}{\pgfqpoint{2.804741in}{1.409363in}}%
\pgfpathcurveto{\pgfqpoint{2.812554in}{1.417177in}}{\pgfqpoint{2.816945in}{1.427776in}}{\pgfqpoint{2.816945in}{1.438826in}}%
\pgfpathcurveto{\pgfqpoint{2.816945in}{1.449876in}}{\pgfqpoint{2.812554in}{1.460475in}}{\pgfqpoint{2.804741in}{1.468289in}}%
\pgfpathcurveto{\pgfqpoint{2.796927in}{1.476102in}}{\pgfqpoint{2.786328in}{1.480492in}}{\pgfqpoint{2.775278in}{1.480492in}}%
\pgfpathcurveto{\pgfqpoint{2.764228in}{1.480492in}}{\pgfqpoint{2.753629in}{1.476102in}}{\pgfqpoint{2.745815in}{1.468289in}}%
\pgfpathcurveto{\pgfqpoint{2.738001in}{1.460475in}}{\pgfqpoint{2.733611in}{1.449876in}}{\pgfqpoint{2.733611in}{1.438826in}}%
\pgfpathcurveto{\pgfqpoint{2.733611in}{1.427776in}}{\pgfqpoint{2.738001in}{1.417177in}}{\pgfqpoint{2.745815in}{1.409363in}}%
\pgfpathcurveto{\pgfqpoint{2.753629in}{1.401549in}}{\pgfqpoint{2.764228in}{1.397159in}}{\pgfqpoint{2.775278in}{1.397159in}}%
\pgfpathclose%
\pgfusepath{stroke,fill}%
\end{pgfscope}%
\begin{pgfscope}%
\pgfpathrectangle{\pgfqpoint{0.583026in}{0.320679in}}{\pgfqpoint{4.650000in}{3.020000in}}%
\pgfusepath{clip}%
\pgfsetbuttcap%
\pgfsetroundjoin%
\definecolor{currentfill}{rgb}{0.839216,0.152941,0.156863}%
\pgfsetfillcolor{currentfill}%
\pgfsetlinewidth{1.003750pt}%
\definecolor{currentstroke}{rgb}{0.839216,0.152941,0.156863}%
\pgfsetstrokecolor{currentstroke}%
\pgfsetdash{}{0pt}%
\pgfpathmoveto{\pgfqpoint{2.976724in}{1.508041in}}%
\pgfpathcurveto{\pgfqpoint{2.987774in}{1.508041in}}{\pgfqpoint{2.998373in}{1.512431in}}{\pgfqpoint{3.006187in}{1.520244in}}%
\pgfpathcurveto{\pgfqpoint{3.014001in}{1.528058in}}{\pgfqpoint{3.018391in}{1.538657in}}{\pgfqpoint{3.018391in}{1.549707in}}%
\pgfpathcurveto{\pgfqpoint{3.018391in}{1.560757in}}{\pgfqpoint{3.014001in}{1.571356in}}{\pgfqpoint{3.006187in}{1.579170in}}%
\pgfpathcurveto{\pgfqpoint{2.998373in}{1.586984in}}{\pgfqpoint{2.987774in}{1.591374in}}{\pgfqpoint{2.976724in}{1.591374in}}%
\pgfpathcurveto{\pgfqpoint{2.965674in}{1.591374in}}{\pgfqpoint{2.955075in}{1.586984in}}{\pgfqpoint{2.947261in}{1.579170in}}%
\pgfpathcurveto{\pgfqpoint{2.939448in}{1.571356in}}{\pgfqpoint{2.935058in}{1.560757in}}{\pgfqpoint{2.935058in}{1.549707in}}%
\pgfpathcurveto{\pgfqpoint{2.935058in}{1.538657in}}{\pgfqpoint{2.939448in}{1.528058in}}{\pgfqpoint{2.947261in}{1.520244in}}%
\pgfpathcurveto{\pgfqpoint{2.955075in}{1.512431in}}{\pgfqpoint{2.965674in}{1.508041in}}{\pgfqpoint{2.976724in}{1.508041in}}%
\pgfpathclose%
\pgfusepath{stroke,fill}%
\end{pgfscope}%
\begin{pgfscope}%
\pgfpathrectangle{\pgfqpoint{0.583026in}{0.320679in}}{\pgfqpoint{4.650000in}{3.020000in}}%
\pgfusepath{clip}%
\pgfsetbuttcap%
\pgfsetroundjoin%
\definecolor{currentfill}{rgb}{0.839216,0.152941,0.156863}%
\pgfsetfillcolor{currentfill}%
\pgfsetlinewidth{1.003750pt}%
\definecolor{currentstroke}{rgb}{0.839216,0.152941,0.156863}%
\pgfsetstrokecolor{currentstroke}%
\pgfsetdash{}{0pt}%
\pgfpathmoveto{\pgfqpoint{3.211745in}{1.627451in}}%
\pgfpathcurveto{\pgfqpoint{3.222795in}{1.627451in}}{\pgfqpoint{3.233394in}{1.631842in}}{\pgfqpoint{3.241208in}{1.639655in}}%
\pgfpathcurveto{\pgfqpoint{3.249021in}{1.647469in}}{\pgfqpoint{3.253412in}{1.658068in}}{\pgfqpoint{3.253412in}{1.669118in}}%
\pgfpathcurveto{\pgfqpoint{3.253412in}{1.680168in}}{\pgfqpoint{3.249021in}{1.690767in}}{\pgfqpoint{3.241208in}{1.698581in}}%
\pgfpathcurveto{\pgfqpoint{3.233394in}{1.706394in}}{\pgfqpoint{3.222795in}{1.710785in}}{\pgfqpoint{3.211745in}{1.710785in}}%
\pgfpathcurveto{\pgfqpoint{3.200695in}{1.710785in}}{\pgfqpoint{3.190096in}{1.706394in}}{\pgfqpoint{3.182282in}{1.698581in}}%
\pgfpathcurveto{\pgfqpoint{3.174468in}{1.690767in}}{\pgfqpoint{3.170078in}{1.680168in}}{\pgfqpoint{3.170078in}{1.669118in}}%
\pgfpathcurveto{\pgfqpoint{3.170078in}{1.658068in}}{\pgfqpoint{3.174468in}{1.647469in}}{\pgfqpoint{3.182282in}{1.639655in}}%
\pgfpathcurveto{\pgfqpoint{3.190096in}{1.631842in}}{\pgfqpoint{3.200695in}{1.627451in}}{\pgfqpoint{3.211745in}{1.627451in}}%
\pgfpathclose%
\pgfusepath{stroke,fill}%
\end{pgfscope}%
\begin{pgfscope}%
\pgfpathrectangle{\pgfqpoint{0.583026in}{0.320679in}}{\pgfqpoint{4.650000in}{3.020000in}}%
\pgfusepath{clip}%
\pgfsetbuttcap%
\pgfsetroundjoin%
\definecolor{currentfill}{rgb}{0.839216,0.152941,0.156863}%
\pgfsetfillcolor{currentfill}%
\pgfsetlinewidth{1.003750pt}%
\definecolor{currentstroke}{rgb}{0.839216,0.152941,0.156863}%
\pgfsetstrokecolor{currentstroke}%
\pgfsetdash{}{0pt}%
\pgfpathmoveto{\pgfqpoint{3.469148in}{1.755391in}}%
\pgfpathcurveto{\pgfqpoint{3.480199in}{1.755391in}}{\pgfqpoint{3.490798in}{1.759782in}}{\pgfqpoint{3.498611in}{1.767595in}}%
\pgfpathcurveto{\pgfqpoint{3.506425in}{1.775409in}}{\pgfqpoint{3.510815in}{1.786008in}}{\pgfqpoint{3.510815in}{1.797058in}}%
\pgfpathcurveto{\pgfqpoint{3.510815in}{1.808108in}}{\pgfqpoint{3.506425in}{1.818707in}}{\pgfqpoint{3.498611in}{1.826521in}}%
\pgfpathcurveto{\pgfqpoint{3.490798in}{1.834335in}}{\pgfqpoint{3.480199in}{1.838725in}}{\pgfqpoint{3.469148in}{1.838725in}}%
\pgfpathcurveto{\pgfqpoint{3.458098in}{1.838725in}}{\pgfqpoint{3.447499in}{1.834335in}}{\pgfqpoint{3.439686in}{1.826521in}}%
\pgfpathcurveto{\pgfqpoint{3.431872in}{1.818707in}}{\pgfqpoint{3.427482in}{1.808108in}}{\pgfqpoint{3.427482in}{1.797058in}}%
\pgfpathcurveto{\pgfqpoint{3.427482in}{1.786008in}}{\pgfqpoint{3.431872in}{1.775409in}}{\pgfqpoint{3.439686in}{1.767595in}}%
\pgfpathcurveto{\pgfqpoint{3.447499in}{1.759782in}}{\pgfqpoint{3.458098in}{1.755391in}}{\pgfqpoint{3.469148in}{1.755391in}}%
\pgfpathclose%
\pgfusepath{stroke,fill}%
\end{pgfscope}%
\begin{pgfscope}%
\pgfpathrectangle{\pgfqpoint{0.583026in}{0.320679in}}{\pgfqpoint{4.650000in}{3.020000in}}%
\pgfusepath{clip}%
\pgfsetbuttcap%
\pgfsetroundjoin%
\definecolor{currentfill}{rgb}{0.839216,0.152941,0.156863}%
\pgfsetfillcolor{currentfill}%
\pgfsetlinewidth{1.003750pt}%
\definecolor{currentstroke}{rgb}{0.839216,0.152941,0.156863}%
\pgfsetstrokecolor{currentstroke}%
\pgfsetdash{}{0pt}%
\pgfpathmoveto{\pgfqpoint{3.726552in}{1.883332in}}%
\pgfpathcurveto{\pgfqpoint{3.737602in}{1.883332in}}{\pgfqpoint{3.748201in}{1.887722in}}{\pgfqpoint{3.756015in}{1.895535in}}%
\pgfpathcurveto{\pgfqpoint{3.763828in}{1.903349in}}{\pgfqpoint{3.768219in}{1.913948in}}{\pgfqpoint{3.768219in}{1.924998in}}%
\pgfpathcurveto{\pgfqpoint{3.768219in}{1.936048in}}{\pgfqpoint{3.763828in}{1.946647in}}{\pgfqpoint{3.756015in}{1.954461in}}%
\pgfpathcurveto{\pgfqpoint{3.748201in}{1.962275in}}{\pgfqpoint{3.737602in}{1.966665in}}{\pgfqpoint{3.726552in}{1.966665in}}%
\pgfpathcurveto{\pgfqpoint{3.715502in}{1.966665in}}{\pgfqpoint{3.704903in}{1.962275in}}{\pgfqpoint{3.697089in}{1.954461in}}%
\pgfpathcurveto{\pgfqpoint{3.689276in}{1.946647in}}{\pgfqpoint{3.684885in}{1.936048in}}{\pgfqpoint{3.684885in}{1.924998in}}%
\pgfpathcurveto{\pgfqpoint{3.684885in}{1.913948in}}{\pgfqpoint{3.689276in}{1.903349in}}{\pgfqpoint{3.697089in}{1.895535in}}%
\pgfpathcurveto{\pgfqpoint{3.704903in}{1.887722in}}{\pgfqpoint{3.715502in}{1.883332in}}{\pgfqpoint{3.726552in}{1.883332in}}%
\pgfpathclose%
\pgfusepath{stroke,fill}%
\end{pgfscope}%
\begin{pgfscope}%
\pgfpathrectangle{\pgfqpoint{0.583026in}{0.320679in}}{\pgfqpoint{4.650000in}{3.020000in}}%
\pgfusepath{clip}%
\pgfsetbuttcap%
\pgfsetroundjoin%
\definecolor{currentfill}{rgb}{0.839216,0.152941,0.156863}%
\pgfsetfillcolor{currentfill}%
\pgfsetlinewidth{1.003750pt}%
\definecolor{currentstroke}{rgb}{0.839216,0.152941,0.156863}%
\pgfsetstrokecolor{currentstroke}%
\pgfsetdash{}{0pt}%
\pgfpathmoveto{\pgfqpoint{3.838467in}{1.951566in}}%
\pgfpathcurveto{\pgfqpoint{3.849517in}{1.951566in}}{\pgfqpoint{3.860116in}{1.955957in}}{\pgfqpoint{3.867929in}{1.963770in}}%
\pgfpathcurveto{\pgfqpoint{3.875743in}{1.971584in}}{\pgfqpoint{3.880133in}{1.982183in}}{\pgfqpoint{3.880133in}{1.993233in}}%
\pgfpathcurveto{\pgfqpoint{3.880133in}{2.004283in}}{\pgfqpoint{3.875743in}{2.014882in}}{\pgfqpoint{3.867929in}{2.022696in}}%
\pgfpathcurveto{\pgfqpoint{3.860116in}{2.030509in}}{\pgfqpoint{3.849517in}{2.034900in}}{\pgfqpoint{3.838467in}{2.034900in}}%
\pgfpathcurveto{\pgfqpoint{3.827416in}{2.034900in}}{\pgfqpoint{3.816817in}{2.030509in}}{\pgfqpoint{3.809004in}{2.022696in}}%
\pgfpathcurveto{\pgfqpoint{3.801190in}{2.014882in}}{\pgfqpoint{3.796800in}{2.004283in}}{\pgfqpoint{3.796800in}{1.993233in}}%
\pgfpathcurveto{\pgfqpoint{3.796800in}{1.982183in}}{\pgfqpoint{3.801190in}{1.971584in}}{\pgfqpoint{3.809004in}{1.963770in}}%
\pgfpathcurveto{\pgfqpoint{3.816817in}{1.955957in}}{\pgfqpoint{3.827416in}{1.951566in}}{\pgfqpoint{3.838467in}{1.951566in}}%
\pgfpathclose%
\pgfusepath{stroke,fill}%
\end{pgfscope}%
\begin{pgfscope}%
\pgfpathrectangle{\pgfqpoint{0.583026in}{0.320679in}}{\pgfqpoint{4.650000in}{3.020000in}}%
\pgfusepath{clip}%
\pgfsetbuttcap%
\pgfsetroundjoin%
\definecolor{currentfill}{rgb}{0.839216,0.152941,0.156863}%
\pgfsetfillcolor{currentfill}%
\pgfsetlinewidth{1.003750pt}%
\definecolor{currentstroke}{rgb}{0.839216,0.152941,0.156863}%
\pgfsetstrokecolor{currentstroke}%
\pgfsetdash{}{0pt}%
\pgfpathmoveto{\pgfqpoint{4.062296in}{2.062448in}}%
\pgfpathcurveto{\pgfqpoint{4.073346in}{2.062448in}}{\pgfqpoint{4.083945in}{2.066838in}}{\pgfqpoint{4.091759in}{2.074652in}}%
\pgfpathcurveto{\pgfqpoint{4.099572in}{2.082465in}}{\pgfqpoint{4.103962in}{2.093064in}}{\pgfqpoint{4.103962in}{2.104114in}}%
\pgfpathcurveto{\pgfqpoint{4.103962in}{2.115164in}}{\pgfqpoint{4.099572in}{2.125764in}}{\pgfqpoint{4.091759in}{2.133577in}}%
\pgfpathcurveto{\pgfqpoint{4.083945in}{2.141391in}}{\pgfqpoint{4.073346in}{2.145781in}}{\pgfqpoint{4.062296in}{2.145781in}}%
\pgfpathcurveto{\pgfqpoint{4.051246in}{2.145781in}}{\pgfqpoint{4.040647in}{2.141391in}}{\pgfqpoint{4.032833in}{2.133577in}}%
\pgfpathcurveto{\pgfqpoint{4.025019in}{2.125764in}}{\pgfqpoint{4.020629in}{2.115164in}}{\pgfqpoint{4.020629in}{2.104114in}}%
\pgfpathcurveto{\pgfqpoint{4.020629in}{2.093064in}}{\pgfqpoint{4.025019in}{2.082465in}}{\pgfqpoint{4.032833in}{2.074652in}}%
\pgfpathcurveto{\pgfqpoint{4.040647in}{2.066838in}}{\pgfqpoint{4.051246in}{2.062448in}}{\pgfqpoint{4.062296in}{2.062448in}}%
\pgfpathclose%
\pgfusepath{stroke,fill}%
\end{pgfscope}%
\begin{pgfscope}%
\pgfpathrectangle{\pgfqpoint{0.583026in}{0.320679in}}{\pgfqpoint{4.650000in}{3.020000in}}%
\pgfusepath{clip}%
\pgfsetbuttcap%
\pgfsetroundjoin%
\definecolor{currentfill}{rgb}{0.839216,0.152941,0.156863}%
\pgfsetfillcolor{currentfill}%
\pgfsetlinewidth{1.003750pt}%
\definecolor{currentstroke}{rgb}{0.839216,0.152941,0.156863}%
\pgfsetstrokecolor{currentstroke}%
\pgfsetdash{}{0pt}%
\pgfpathmoveto{\pgfqpoint{4.330891in}{2.190388in}}%
\pgfpathcurveto{\pgfqpoint{4.341941in}{2.190388in}}{\pgfqpoint{4.352540in}{2.194778in}}{\pgfqpoint{4.360354in}{2.202592in}}%
\pgfpathcurveto{\pgfqpoint{4.368167in}{2.210405in}}{\pgfqpoint{4.372558in}{2.221004in}}{\pgfqpoint{4.372558in}{2.232054in}}%
\pgfpathcurveto{\pgfqpoint{4.372558in}{2.243105in}}{\pgfqpoint{4.368167in}{2.253704in}}{\pgfqpoint{4.360354in}{2.261517in}}%
\pgfpathcurveto{\pgfqpoint{4.352540in}{2.269331in}}{\pgfqpoint{4.341941in}{2.273721in}}{\pgfqpoint{4.330891in}{2.273721in}}%
\pgfpathcurveto{\pgfqpoint{4.319841in}{2.273721in}}{\pgfqpoint{4.309242in}{2.269331in}}{\pgfqpoint{4.301428in}{2.261517in}}%
\pgfpathcurveto{\pgfqpoint{4.293614in}{2.253704in}}{\pgfqpoint{4.289224in}{2.243105in}}{\pgfqpoint{4.289224in}{2.232054in}}%
\pgfpathcurveto{\pgfqpoint{4.289224in}{2.221004in}}{\pgfqpoint{4.293614in}{2.210405in}}{\pgfqpoint{4.301428in}{2.202592in}}%
\pgfpathcurveto{\pgfqpoint{4.309242in}{2.194778in}}{\pgfqpoint{4.319841in}{2.190388in}}{\pgfqpoint{4.330891in}{2.190388in}}%
\pgfpathclose%
\pgfusepath{stroke,fill}%
\end{pgfscope}%
\begin{pgfscope}%
\pgfpathrectangle{\pgfqpoint{0.583026in}{0.320679in}}{\pgfqpoint{4.650000in}{3.020000in}}%
\pgfusepath{clip}%
\pgfsetbuttcap%
\pgfsetroundjoin%
\definecolor{currentfill}{rgb}{0.839216,0.152941,0.156863}%
\pgfsetfillcolor{currentfill}%
\pgfsetlinewidth{1.003750pt}%
\definecolor{currentstroke}{rgb}{0.839216,0.152941,0.156863}%
\pgfsetstrokecolor{currentstroke}%
\pgfsetdash{}{0pt}%
\pgfpathmoveto{\pgfqpoint{4.565912in}{2.309799in}}%
\pgfpathcurveto{\pgfqpoint{4.576962in}{2.309799in}}{\pgfqpoint{4.587561in}{2.314189in}}{\pgfqpoint{4.595374in}{2.322002in}}%
\pgfpathcurveto{\pgfqpoint{4.603188in}{2.329816in}}{\pgfqpoint{4.607578in}{2.340415in}}{\pgfqpoint{4.607578in}{2.351465in}}%
\pgfpathcurveto{\pgfqpoint{4.607578in}{2.362515in}}{\pgfqpoint{4.603188in}{2.373114in}}{\pgfqpoint{4.595374in}{2.380928in}}%
\pgfpathcurveto{\pgfqpoint{4.587561in}{2.388742in}}{\pgfqpoint{4.576962in}{2.393132in}}{\pgfqpoint{4.565912in}{2.393132in}}%
\pgfpathcurveto{\pgfqpoint{4.554861in}{2.393132in}}{\pgfqpoint{4.544262in}{2.388742in}}{\pgfqpoint{4.536449in}{2.380928in}}%
\pgfpathcurveto{\pgfqpoint{4.528635in}{2.373114in}}{\pgfqpoint{4.524245in}{2.362515in}}{\pgfqpoint{4.524245in}{2.351465in}}%
\pgfpathcurveto{\pgfqpoint{4.524245in}{2.340415in}}{\pgfqpoint{4.528635in}{2.329816in}}{\pgfqpoint{4.536449in}{2.322002in}}%
\pgfpathcurveto{\pgfqpoint{4.544262in}{2.314189in}}{\pgfqpoint{4.554861in}{2.309799in}}{\pgfqpoint{4.565912in}{2.309799in}}%
\pgfpathclose%
\pgfusepath{stroke,fill}%
\end{pgfscope}%
\begin{pgfscope}%
\pgfpathrectangle{\pgfqpoint{0.583026in}{0.320679in}}{\pgfqpoint{4.650000in}{3.020000in}}%
\pgfusepath{clip}%
\pgfsetbuttcap%
\pgfsetroundjoin%
\definecolor{currentfill}{rgb}{0.839216,0.152941,0.156863}%
\pgfsetfillcolor{currentfill}%
\pgfsetlinewidth{1.003750pt}%
\definecolor{currentstroke}{rgb}{0.839216,0.152941,0.156863}%
\pgfsetstrokecolor{currentstroke}%
\pgfsetdash{}{0pt}%
\pgfpathmoveto{\pgfqpoint{4.834507in}{2.446268in}}%
\pgfpathcurveto{\pgfqpoint{4.845557in}{2.446268in}}{\pgfqpoint{4.856156in}{2.450658in}}{\pgfqpoint{4.863969in}{2.458472in}}%
\pgfpathcurveto{\pgfqpoint{4.871783in}{2.466286in}}{\pgfqpoint{4.876173in}{2.476885in}}{\pgfqpoint{4.876173in}{2.487935in}}%
\pgfpathcurveto{\pgfqpoint{4.876173in}{2.498985in}}{\pgfqpoint{4.871783in}{2.509584in}}{\pgfqpoint{4.863969in}{2.517397in}}%
\pgfpathcurveto{\pgfqpoint{4.856156in}{2.525211in}}{\pgfqpoint{4.845557in}{2.529601in}}{\pgfqpoint{4.834507in}{2.529601in}}%
\pgfpathcurveto{\pgfqpoint{4.823456in}{2.529601in}}{\pgfqpoint{4.812857in}{2.525211in}}{\pgfqpoint{4.805044in}{2.517397in}}%
\pgfpathcurveto{\pgfqpoint{4.797230in}{2.509584in}}{\pgfqpoint{4.792840in}{2.498985in}}{\pgfqpoint{4.792840in}{2.487935in}}%
\pgfpathcurveto{\pgfqpoint{4.792840in}{2.476885in}}{\pgfqpoint{4.797230in}{2.466286in}}{\pgfqpoint{4.805044in}{2.458472in}}%
\pgfpathcurveto{\pgfqpoint{4.812857in}{2.450658in}}{\pgfqpoint{4.823456in}{2.446268in}}{\pgfqpoint{4.834507in}{2.446268in}}%
\pgfpathclose%
\pgfusepath{stroke,fill}%
\end{pgfscope}%
\begin{pgfscope}%
\pgfpathrectangle{\pgfqpoint{0.583026in}{0.320679in}}{\pgfqpoint{4.650000in}{3.020000in}}%
\pgfusepath{clip}%
\pgfsetbuttcap%
\pgfsetroundjoin%
\definecolor{currentfill}{rgb}{0.839216,0.152941,0.156863}%
\pgfsetfillcolor{currentfill}%
\pgfsetlinewidth{1.003750pt}%
\definecolor{currentstroke}{rgb}{0.839216,0.152941,0.156863}%
\pgfsetstrokecolor{currentstroke}%
\pgfsetdash{}{0pt}%
\pgfpathmoveto{\pgfqpoint{4.979996in}{2.505973in}}%
\pgfpathcurveto{\pgfqpoint{4.991046in}{2.505973in}}{\pgfqpoint{5.001645in}{2.510364in}}{\pgfqpoint{5.009458in}{2.518177in}}%
\pgfpathcurveto{\pgfqpoint{5.017272in}{2.525991in}}{\pgfqpoint{5.021662in}{2.536590in}}{\pgfqpoint{5.021662in}{2.547640in}}%
\pgfpathcurveto{\pgfqpoint{5.021662in}{2.558690in}}{\pgfqpoint{5.017272in}{2.569289in}}{\pgfqpoint{5.009458in}{2.577103in}}%
\pgfpathcurveto{\pgfqpoint{5.001645in}{2.584916in}}{\pgfqpoint{4.991046in}{2.589307in}}{\pgfqpoint{4.979996in}{2.589307in}}%
\pgfpathcurveto{\pgfqpoint{4.968945in}{2.589307in}}{\pgfqpoint{4.958346in}{2.584916in}}{\pgfqpoint{4.950533in}{2.577103in}}%
\pgfpathcurveto{\pgfqpoint{4.942719in}{2.569289in}}{\pgfqpoint{4.938329in}{2.558690in}}{\pgfqpoint{4.938329in}{2.547640in}}%
\pgfpathcurveto{\pgfqpoint{4.938329in}{2.536590in}}{\pgfqpoint{4.942719in}{2.525991in}}{\pgfqpoint{4.950533in}{2.518177in}}%
\pgfpathcurveto{\pgfqpoint{4.958346in}{2.510364in}}{\pgfqpoint{4.968945in}{2.505973in}}{\pgfqpoint{4.979996in}{2.505973in}}%
\pgfpathclose%
\pgfusepath{stroke,fill}%
\end{pgfscope}%
\begin{pgfscope}%
\pgfpathrectangle{\pgfqpoint{0.583026in}{0.320679in}}{\pgfqpoint{4.650000in}{3.020000in}}%
\pgfusepath{clip}%
\pgfsetbuttcap%
\pgfsetroundjoin%
\definecolor{currentfill}{rgb}{0.580392,0.403922,0.741176}%
\pgfsetfillcolor{currentfill}%
\pgfsetlinewidth{1.003750pt}%
\definecolor{currentstroke}{rgb}{0.580392,0.403922,0.741176}%
\pgfsetstrokecolor{currentstroke}%
\pgfsetdash{}{0pt}%
\pgfpathmoveto{\pgfqpoint{0.962261in}{0.510108in}}%
\pgfpathcurveto{\pgfqpoint{0.973311in}{0.510108in}}{\pgfqpoint{0.983911in}{0.514498in}}{\pgfqpoint{0.991724in}{0.522312in}}%
\pgfpathcurveto{\pgfqpoint{0.999538in}{0.530125in}}{\pgfqpoint{1.003928in}{0.540724in}}{\pgfqpoint{1.003928in}{0.551774in}}%
\pgfpathcurveto{\pgfqpoint{1.003928in}{0.562824in}}{\pgfqpoint{0.999538in}{0.573424in}}{\pgfqpoint{0.991724in}{0.581237in}}%
\pgfpathcurveto{\pgfqpoint{0.983911in}{0.589051in}}{\pgfqpoint{0.973311in}{0.593441in}}{\pgfqpoint{0.962261in}{0.593441in}}%
\pgfpathcurveto{\pgfqpoint{0.951211in}{0.593441in}}{\pgfqpoint{0.940612in}{0.589051in}}{\pgfqpoint{0.932799in}{0.581237in}}%
\pgfpathcurveto{\pgfqpoint{0.924985in}{0.573424in}}{\pgfqpoint{0.920595in}{0.562824in}}{\pgfqpoint{0.920595in}{0.551774in}}%
\pgfpathcurveto{\pgfqpoint{0.920595in}{0.540724in}}{\pgfqpoint{0.924985in}{0.530125in}}{\pgfqpoint{0.932799in}{0.522312in}}%
\pgfpathcurveto{\pgfqpoint{0.940612in}{0.514498in}}{\pgfqpoint{0.951211in}{0.510108in}}{\pgfqpoint{0.962261in}{0.510108in}}%
\pgfpathclose%
\pgfusepath{stroke,fill}%
\end{pgfscope}%
\begin{pgfscope}%
\pgfpathrectangle{\pgfqpoint{0.583026in}{0.320679in}}{\pgfqpoint{4.650000in}{3.020000in}}%
\pgfusepath{clip}%
\pgfsetbuttcap%
\pgfsetroundjoin%
\definecolor{currentfill}{rgb}{0.580392,0.403922,0.741176}%
\pgfsetfillcolor{currentfill}%
\pgfsetlinewidth{1.003750pt}%
\definecolor{currentstroke}{rgb}{0.580392,0.403922,0.741176}%
\pgfsetstrokecolor{currentstroke}%
\pgfsetdash{}{0pt}%
\pgfpathmoveto{\pgfqpoint{1.174899in}{0.646577in}}%
\pgfpathcurveto{\pgfqpoint{1.185949in}{0.646577in}}{\pgfqpoint{1.196548in}{0.650967in}}{\pgfqpoint{1.204362in}{0.658781in}}%
\pgfpathcurveto{\pgfqpoint{1.212176in}{0.666595in}}{\pgfqpoint{1.216566in}{0.677194in}}{\pgfqpoint{1.216566in}{0.688244in}}%
\pgfpathcurveto{\pgfqpoint{1.216566in}{0.699294in}}{\pgfqpoint{1.212176in}{0.709893in}}{\pgfqpoint{1.204362in}{0.717707in}}%
\pgfpathcurveto{\pgfqpoint{1.196548in}{0.725520in}}{\pgfqpoint{1.185949in}{0.729910in}}{\pgfqpoint{1.174899in}{0.729910in}}%
\pgfpathcurveto{\pgfqpoint{1.163849in}{0.729910in}}{\pgfqpoint{1.153250in}{0.725520in}}{\pgfqpoint{1.145436in}{0.717707in}}%
\pgfpathcurveto{\pgfqpoint{1.137623in}{0.709893in}}{\pgfqpoint{1.133232in}{0.699294in}}{\pgfqpoint{1.133232in}{0.688244in}}%
\pgfpathcurveto{\pgfqpoint{1.133232in}{0.677194in}}{\pgfqpoint{1.137623in}{0.666595in}}{\pgfqpoint{1.145436in}{0.658781in}}%
\pgfpathcurveto{\pgfqpoint{1.153250in}{0.650967in}}{\pgfqpoint{1.163849in}{0.646577in}}{\pgfqpoint{1.174899in}{0.646577in}}%
\pgfpathclose%
\pgfusepath{stroke,fill}%
\end{pgfscope}%
\begin{pgfscope}%
\pgfpathrectangle{\pgfqpoint{0.583026in}{0.320679in}}{\pgfqpoint{4.650000in}{3.020000in}}%
\pgfusepath{clip}%
\pgfsetbuttcap%
\pgfsetroundjoin%
\definecolor{currentfill}{rgb}{0.580392,0.403922,0.741176}%
\pgfsetfillcolor{currentfill}%
\pgfsetlinewidth{1.003750pt}%
\definecolor{currentstroke}{rgb}{0.580392,0.403922,0.741176}%
\pgfsetstrokecolor{currentstroke}%
\pgfsetdash{}{0pt}%
\pgfpathmoveto{\pgfqpoint{1.533026in}{0.885399in}}%
\pgfpathcurveto{\pgfqpoint{1.544076in}{0.885399in}}{\pgfqpoint{1.554675in}{0.889789in}}{\pgfqpoint{1.562489in}{0.897603in}}%
\pgfpathcurveto{\pgfqpoint{1.570302in}{0.905416in}}{\pgfqpoint{1.574693in}{0.916015in}}{\pgfqpoint{1.574693in}{0.927065in}}%
\pgfpathcurveto{\pgfqpoint{1.574693in}{0.938115in}}{\pgfqpoint{1.570302in}{0.948715in}}{\pgfqpoint{1.562489in}{0.956528in}}%
\pgfpathcurveto{\pgfqpoint{1.554675in}{0.964342in}}{\pgfqpoint{1.544076in}{0.968732in}}{\pgfqpoint{1.533026in}{0.968732in}}%
\pgfpathcurveto{\pgfqpoint{1.521976in}{0.968732in}}{\pgfqpoint{1.511377in}{0.964342in}}{\pgfqpoint{1.503563in}{0.956528in}}%
\pgfpathcurveto{\pgfqpoint{1.495749in}{0.948715in}}{\pgfqpoint{1.491359in}{0.938115in}}{\pgfqpoint{1.491359in}{0.927065in}}%
\pgfpathcurveto{\pgfqpoint{1.491359in}{0.916015in}}{\pgfqpoint{1.495749in}{0.905416in}}{\pgfqpoint{1.503563in}{0.897603in}}%
\pgfpathcurveto{\pgfqpoint{1.511377in}{0.889789in}}{\pgfqpoint{1.521976in}{0.885399in}}{\pgfqpoint{1.533026in}{0.885399in}}%
\pgfpathclose%
\pgfusepath{stroke,fill}%
\end{pgfscope}%
\begin{pgfscope}%
\pgfpathrectangle{\pgfqpoint{0.583026in}{0.320679in}}{\pgfqpoint{4.650000in}{3.020000in}}%
\pgfusepath{clip}%
\pgfsetbuttcap%
\pgfsetroundjoin%
\definecolor{currentfill}{rgb}{0.580392,0.403922,0.741176}%
\pgfsetfillcolor{currentfill}%
\pgfsetlinewidth{1.003750pt}%
\definecolor{currentstroke}{rgb}{0.580392,0.403922,0.741176}%
\pgfsetstrokecolor{currentstroke}%
\pgfsetdash{}{0pt}%
\pgfpathmoveto{\pgfqpoint{1.846387in}{1.081574in}}%
\pgfpathcurveto{\pgfqpoint{1.857437in}{1.081574in}}{\pgfqpoint{1.868036in}{1.085964in}}{\pgfqpoint{1.875849in}{1.093777in}}%
\pgfpathcurveto{\pgfqpoint{1.883663in}{1.101591in}}{\pgfqpoint{1.888053in}{1.112190in}}{\pgfqpoint{1.888053in}{1.123240in}}%
\pgfpathcurveto{\pgfqpoint{1.888053in}{1.134290in}}{\pgfqpoint{1.883663in}{1.144889in}}{\pgfqpoint{1.875849in}{1.152703in}}%
\pgfpathcurveto{\pgfqpoint{1.868036in}{1.160517in}}{\pgfqpoint{1.857437in}{1.164907in}}{\pgfqpoint{1.846387in}{1.164907in}}%
\pgfpathcurveto{\pgfqpoint{1.835337in}{1.164907in}}{\pgfqpoint{1.824738in}{1.160517in}}{\pgfqpoint{1.816924in}{1.152703in}}%
\pgfpathcurveto{\pgfqpoint{1.809110in}{1.144889in}}{\pgfqpoint{1.804720in}{1.134290in}}{\pgfqpoint{1.804720in}{1.123240in}}%
\pgfpathcurveto{\pgfqpoint{1.804720in}{1.112190in}}{\pgfqpoint{1.809110in}{1.101591in}}{\pgfqpoint{1.816924in}{1.093777in}}%
\pgfpathcurveto{\pgfqpoint{1.824738in}{1.085964in}}{\pgfqpoint{1.835337in}{1.081574in}}{\pgfqpoint{1.846387in}{1.081574in}}%
\pgfpathclose%
\pgfusepath{stroke,fill}%
\end{pgfscope}%
\begin{pgfscope}%
\pgfpathrectangle{\pgfqpoint{0.583026in}{0.320679in}}{\pgfqpoint{4.650000in}{3.020000in}}%
\pgfusepath{clip}%
\pgfsetbuttcap%
\pgfsetroundjoin%
\definecolor{currentfill}{rgb}{0.580392,0.403922,0.741176}%
\pgfsetfillcolor{currentfill}%
\pgfsetlinewidth{1.003750pt}%
\definecolor{currentstroke}{rgb}{0.580392,0.403922,0.741176}%
\pgfsetstrokecolor{currentstroke}%
\pgfsetdash{}{0pt}%
\pgfpathmoveto{\pgfqpoint{2.126173in}{1.260690in}}%
\pgfpathcurveto{\pgfqpoint{2.137223in}{1.260690in}}{\pgfqpoint{2.147822in}{1.265080in}}{\pgfqpoint{2.155636in}{1.272894in}}%
\pgfpathcurveto{\pgfqpoint{2.163450in}{1.280707in}}{\pgfqpoint{2.167840in}{1.291306in}}{\pgfqpoint{2.167840in}{1.302356in}}%
\pgfpathcurveto{\pgfqpoint{2.167840in}{1.313406in}}{\pgfqpoint{2.163450in}{1.324006in}}{\pgfqpoint{2.155636in}{1.331819in}}%
\pgfpathcurveto{\pgfqpoint{2.147822in}{1.339633in}}{\pgfqpoint{2.137223in}{1.344023in}}{\pgfqpoint{2.126173in}{1.344023in}}%
\pgfpathcurveto{\pgfqpoint{2.115123in}{1.344023in}}{\pgfqpoint{2.104524in}{1.339633in}}{\pgfqpoint{2.096710in}{1.331819in}}%
\pgfpathcurveto{\pgfqpoint{2.088897in}{1.324006in}}{\pgfqpoint{2.084507in}{1.313406in}}{\pgfqpoint{2.084507in}{1.302356in}}%
\pgfpathcurveto{\pgfqpoint{2.084507in}{1.291306in}}{\pgfqpoint{2.088897in}{1.280707in}}{\pgfqpoint{2.096710in}{1.272894in}}%
\pgfpathcurveto{\pgfqpoint{2.104524in}{1.265080in}}{\pgfqpoint{2.115123in}{1.260690in}}{\pgfqpoint{2.126173in}{1.260690in}}%
\pgfpathclose%
\pgfusepath{stroke,fill}%
\end{pgfscope}%
\begin{pgfscope}%
\pgfpathrectangle{\pgfqpoint{0.583026in}{0.320679in}}{\pgfqpoint{4.650000in}{3.020000in}}%
\pgfusepath{clip}%
\pgfsetbuttcap%
\pgfsetroundjoin%
\definecolor{currentfill}{rgb}{0.580392,0.403922,0.741176}%
\pgfsetfillcolor{currentfill}%
\pgfsetlinewidth{1.003750pt}%
\definecolor{currentstroke}{rgb}{0.580392,0.403922,0.741176}%
\pgfsetstrokecolor{currentstroke}%
\pgfsetdash{}{0pt}%
\pgfpathmoveto{\pgfqpoint{2.350002in}{1.405688in}}%
\pgfpathcurveto{\pgfqpoint{2.361053in}{1.405688in}}{\pgfqpoint{2.371652in}{1.410079in}}{\pgfqpoint{2.379465in}{1.417892in}}%
\pgfpathcurveto{\pgfqpoint{2.387279in}{1.425706in}}{\pgfqpoint{2.391669in}{1.436305in}}{\pgfqpoint{2.391669in}{1.447355in}}%
\pgfpathcurveto{\pgfqpoint{2.391669in}{1.458405in}}{\pgfqpoint{2.387279in}{1.469004in}}{\pgfqpoint{2.379465in}{1.476818in}}%
\pgfpathcurveto{\pgfqpoint{2.371652in}{1.484632in}}{\pgfqpoint{2.361053in}{1.489022in}}{\pgfqpoint{2.350002in}{1.489022in}}%
\pgfpathcurveto{\pgfqpoint{2.338952in}{1.489022in}}{\pgfqpoint{2.328353in}{1.484632in}}{\pgfqpoint{2.320540in}{1.476818in}}%
\pgfpathcurveto{\pgfqpoint{2.312726in}{1.469004in}}{\pgfqpoint{2.308336in}{1.458405in}}{\pgfqpoint{2.308336in}{1.447355in}}%
\pgfpathcurveto{\pgfqpoint{2.308336in}{1.436305in}}{\pgfqpoint{2.312726in}{1.425706in}}{\pgfqpoint{2.320540in}{1.417892in}}%
\pgfpathcurveto{\pgfqpoint{2.328353in}{1.410079in}}{\pgfqpoint{2.338952in}{1.405688in}}{\pgfqpoint{2.350002in}{1.405688in}}%
\pgfpathclose%
\pgfusepath{stroke,fill}%
\end{pgfscope}%
\begin{pgfscope}%
\pgfpathrectangle{\pgfqpoint{0.583026in}{0.320679in}}{\pgfqpoint{4.650000in}{3.020000in}}%
\pgfusepath{clip}%
\pgfsetbuttcap%
\pgfsetroundjoin%
\definecolor{currentfill}{rgb}{0.580392,0.403922,0.741176}%
\pgfsetfillcolor{currentfill}%
\pgfsetlinewidth{1.003750pt}%
\definecolor{currentstroke}{rgb}{0.580392,0.403922,0.741176}%
\pgfsetstrokecolor{currentstroke}%
\pgfsetdash{}{0pt}%
\pgfpathmoveto{\pgfqpoint{2.607406in}{1.559217in}}%
\pgfpathcurveto{\pgfqpoint{2.618456in}{1.559217in}}{\pgfqpoint{2.629055in}{1.563607in}}{\pgfqpoint{2.636869in}{1.571420in}}%
\pgfpathcurveto{\pgfqpoint{2.644682in}{1.579234in}}{\pgfqpoint{2.649073in}{1.589833in}}{\pgfqpoint{2.649073in}{1.600883in}}%
\pgfpathcurveto{\pgfqpoint{2.649073in}{1.611933in}}{\pgfqpoint{2.644682in}{1.622532in}}{\pgfqpoint{2.636869in}{1.630346in}}%
\pgfpathcurveto{\pgfqpoint{2.629055in}{1.638160in}}{\pgfqpoint{2.618456in}{1.642550in}}{\pgfqpoint{2.607406in}{1.642550in}}%
\pgfpathcurveto{\pgfqpoint{2.596356in}{1.642550in}}{\pgfqpoint{2.585757in}{1.638160in}}{\pgfqpoint{2.577943in}{1.630346in}}%
\pgfpathcurveto{\pgfqpoint{2.570130in}{1.622532in}}{\pgfqpoint{2.565739in}{1.611933in}}{\pgfqpoint{2.565739in}{1.600883in}}%
\pgfpathcurveto{\pgfqpoint{2.565739in}{1.589833in}}{\pgfqpoint{2.570130in}{1.579234in}}{\pgfqpoint{2.577943in}{1.571420in}}%
\pgfpathcurveto{\pgfqpoint{2.585757in}{1.563607in}}{\pgfqpoint{2.596356in}{1.559217in}}{\pgfqpoint{2.607406in}{1.559217in}}%
\pgfpathclose%
\pgfusepath{stroke,fill}%
\end{pgfscope}%
\begin{pgfscope}%
\pgfpathrectangle{\pgfqpoint{0.583026in}{0.320679in}}{\pgfqpoint{4.650000in}{3.020000in}}%
\pgfusepath{clip}%
\pgfsetbuttcap%
\pgfsetroundjoin%
\definecolor{currentfill}{rgb}{0.580392,0.403922,0.741176}%
\pgfsetfillcolor{currentfill}%
\pgfsetlinewidth{1.003750pt}%
\definecolor{currentstroke}{rgb}{0.580392,0.403922,0.741176}%
\pgfsetstrokecolor{currentstroke}%
\pgfsetdash{}{0pt}%
\pgfpathmoveto{\pgfqpoint{2.808852in}{1.704215in}}%
\pgfpathcurveto{\pgfqpoint{2.819902in}{1.704215in}}{\pgfqpoint{2.830501in}{1.708606in}}{\pgfqpoint{2.838315in}{1.716419in}}%
\pgfpathcurveto{\pgfqpoint{2.846129in}{1.724233in}}{\pgfqpoint{2.850519in}{1.734832in}}{\pgfqpoint{2.850519in}{1.745882in}}%
\pgfpathcurveto{\pgfqpoint{2.850519in}{1.756932in}}{\pgfqpoint{2.846129in}{1.767531in}}{\pgfqpoint{2.838315in}{1.775345in}}%
\pgfpathcurveto{\pgfqpoint{2.830501in}{1.783158in}}{\pgfqpoint{2.819902in}{1.787549in}}{\pgfqpoint{2.808852in}{1.787549in}}%
\pgfpathcurveto{\pgfqpoint{2.797802in}{1.787549in}}{\pgfqpoint{2.787203in}{1.783158in}}{\pgfqpoint{2.779389in}{1.775345in}}%
\pgfpathcurveto{\pgfqpoint{2.771576in}{1.767531in}}{\pgfqpoint{2.767186in}{1.756932in}}{\pgfqpoint{2.767186in}{1.745882in}}%
\pgfpathcurveto{\pgfqpoint{2.767186in}{1.734832in}}{\pgfqpoint{2.771576in}{1.724233in}}{\pgfqpoint{2.779389in}{1.716419in}}%
\pgfpathcurveto{\pgfqpoint{2.787203in}{1.708606in}}{\pgfqpoint{2.797802in}{1.704215in}}{\pgfqpoint{2.808852in}{1.704215in}}%
\pgfpathclose%
\pgfusepath{stroke,fill}%
\end{pgfscope}%
\begin{pgfscope}%
\pgfpathrectangle{\pgfqpoint{0.583026in}{0.320679in}}{\pgfqpoint{4.650000in}{3.020000in}}%
\pgfusepath{clip}%
\pgfsetbuttcap%
\pgfsetroundjoin%
\definecolor{currentfill}{rgb}{0.580392,0.403922,0.741176}%
\pgfsetfillcolor{currentfill}%
\pgfsetlinewidth{1.003750pt}%
\definecolor{currentstroke}{rgb}{0.580392,0.403922,0.741176}%
\pgfsetstrokecolor{currentstroke}%
\pgfsetdash{}{0pt}%
\pgfpathmoveto{\pgfqpoint{3.021490in}{1.832155in}}%
\pgfpathcurveto{\pgfqpoint{3.032540in}{1.832155in}}{\pgfqpoint{3.043139in}{1.836546in}}{\pgfqpoint{3.050953in}{1.844359in}}%
\pgfpathcurveto{\pgfqpoint{3.058766in}{1.852173in}}{\pgfqpoint{3.063157in}{1.862772in}}{\pgfqpoint{3.063157in}{1.873822in}}%
\pgfpathcurveto{\pgfqpoint{3.063157in}{1.884872in}}{\pgfqpoint{3.058766in}{1.895471in}}{\pgfqpoint{3.050953in}{1.903285in}}%
\pgfpathcurveto{\pgfqpoint{3.043139in}{1.911099in}}{\pgfqpoint{3.032540in}{1.915489in}}{\pgfqpoint{3.021490in}{1.915489in}}%
\pgfpathcurveto{\pgfqpoint{3.010440in}{1.915489in}}{\pgfqpoint{2.999841in}{1.911099in}}{\pgfqpoint{2.992027in}{1.903285in}}%
\pgfpathcurveto{\pgfqpoint{2.984214in}{1.895471in}}{\pgfqpoint{2.979823in}{1.884872in}}{\pgfqpoint{2.979823in}{1.873822in}}%
\pgfpathcurveto{\pgfqpoint{2.979823in}{1.862772in}}{\pgfqpoint{2.984214in}{1.852173in}}{\pgfqpoint{2.992027in}{1.844359in}}%
\pgfpathcurveto{\pgfqpoint{2.999841in}{1.836546in}}{\pgfqpoint{3.010440in}{1.832155in}}{\pgfqpoint{3.021490in}{1.832155in}}%
\pgfpathclose%
\pgfusepath{stroke,fill}%
\end{pgfscope}%
\begin{pgfscope}%
\pgfpathrectangle{\pgfqpoint{0.583026in}{0.320679in}}{\pgfqpoint{4.650000in}{3.020000in}}%
\pgfusepath{clip}%
\pgfsetbuttcap%
\pgfsetroundjoin%
\definecolor{currentfill}{rgb}{0.580392,0.403922,0.741176}%
\pgfsetfillcolor{currentfill}%
\pgfsetlinewidth{1.003750pt}%
\definecolor{currentstroke}{rgb}{0.580392,0.403922,0.741176}%
\pgfsetstrokecolor{currentstroke}%
\pgfsetdash{}{0pt}%
\pgfpathmoveto{\pgfqpoint{3.211745in}{1.968625in}}%
\pgfpathcurveto{\pgfqpoint{3.222795in}{1.968625in}}{\pgfqpoint{3.233394in}{1.973015in}}{\pgfqpoint{3.241208in}{1.980829in}}%
\pgfpathcurveto{\pgfqpoint{3.249021in}{1.988642in}}{\pgfqpoint{3.253412in}{1.999241in}}{\pgfqpoint{3.253412in}{2.010292in}}%
\pgfpathcurveto{\pgfqpoint{3.253412in}{2.021342in}}{\pgfqpoint{3.249021in}{2.031941in}}{\pgfqpoint{3.241208in}{2.039754in}}%
\pgfpathcurveto{\pgfqpoint{3.233394in}{2.047568in}}{\pgfqpoint{3.222795in}{2.051958in}}{\pgfqpoint{3.211745in}{2.051958in}}%
\pgfpathcurveto{\pgfqpoint{3.200695in}{2.051958in}}{\pgfqpoint{3.190096in}{2.047568in}}{\pgfqpoint{3.182282in}{2.039754in}}%
\pgfpathcurveto{\pgfqpoint{3.174468in}{2.031941in}}{\pgfqpoint{3.170078in}{2.021342in}}{\pgfqpoint{3.170078in}{2.010292in}}%
\pgfpathcurveto{\pgfqpoint{3.170078in}{1.999241in}}{\pgfqpoint{3.174468in}{1.988642in}}{\pgfqpoint{3.182282in}{1.980829in}}%
\pgfpathcurveto{\pgfqpoint{3.190096in}{1.973015in}}{\pgfqpoint{3.200695in}{1.968625in}}{\pgfqpoint{3.211745in}{1.968625in}}%
\pgfpathclose%
\pgfusepath{stroke,fill}%
\end{pgfscope}%
\begin{pgfscope}%
\pgfpathrectangle{\pgfqpoint{0.583026in}{0.320679in}}{\pgfqpoint{4.650000in}{3.020000in}}%
\pgfusepath{clip}%
\pgfsetbuttcap%
\pgfsetroundjoin%
\definecolor{currentfill}{rgb}{0.580392,0.403922,0.741176}%
\pgfsetfillcolor{currentfill}%
\pgfsetlinewidth{1.003750pt}%
\definecolor{currentstroke}{rgb}{0.580392,0.403922,0.741176}%
\pgfsetstrokecolor{currentstroke}%
\pgfsetdash{}{0pt}%
\pgfpathmoveto{\pgfqpoint{3.413191in}{2.088036in}}%
\pgfpathcurveto{\pgfqpoint{3.424241in}{2.088036in}}{\pgfqpoint{3.434840in}{2.092426in}}{\pgfqpoint{3.442654in}{2.100240in}}%
\pgfpathcurveto{\pgfqpoint{3.450468in}{2.108053in}}{\pgfqpoint{3.454858in}{2.118652in}}{\pgfqpoint{3.454858in}{2.129702in}}%
\pgfpathcurveto{\pgfqpoint{3.454858in}{2.140753in}}{\pgfqpoint{3.450468in}{2.151352in}}{\pgfqpoint{3.442654in}{2.159165in}}%
\pgfpathcurveto{\pgfqpoint{3.434840in}{2.166979in}}{\pgfqpoint{3.424241in}{2.171369in}}{\pgfqpoint{3.413191in}{2.171369in}}%
\pgfpathcurveto{\pgfqpoint{3.402141in}{2.171369in}}{\pgfqpoint{3.391542in}{2.166979in}}{\pgfqpoint{3.383728in}{2.159165in}}%
\pgfpathcurveto{\pgfqpoint{3.375915in}{2.151352in}}{\pgfqpoint{3.371524in}{2.140753in}}{\pgfqpoint{3.371524in}{2.129702in}}%
\pgfpathcurveto{\pgfqpoint{3.371524in}{2.118652in}}{\pgfqpoint{3.375915in}{2.108053in}}{\pgfqpoint{3.383728in}{2.100240in}}%
\pgfpathcurveto{\pgfqpoint{3.391542in}{2.092426in}}{\pgfqpoint{3.402141in}{2.088036in}}{\pgfqpoint{3.413191in}{2.088036in}}%
\pgfpathclose%
\pgfusepath{stroke,fill}%
\end{pgfscope}%
\begin{pgfscope}%
\pgfpathrectangle{\pgfqpoint{0.583026in}{0.320679in}}{\pgfqpoint{4.650000in}{3.020000in}}%
\pgfusepath{clip}%
\pgfsetbuttcap%
\pgfsetroundjoin%
\definecolor{currentfill}{rgb}{0.580392,0.403922,0.741176}%
\pgfsetfillcolor{currentfill}%
\pgfsetlinewidth{1.003750pt}%
\definecolor{currentstroke}{rgb}{0.580392,0.403922,0.741176}%
\pgfsetstrokecolor{currentstroke}%
\pgfsetdash{}{0pt}%
\pgfpathmoveto{\pgfqpoint{3.659403in}{2.250093in}}%
\pgfpathcurveto{\pgfqpoint{3.670453in}{2.250093in}}{\pgfqpoint{3.681052in}{2.254483in}}{\pgfqpoint{3.688866in}{2.262297in}}%
\pgfpathcurveto{\pgfqpoint{3.696680in}{2.270111in}}{\pgfqpoint{3.701070in}{2.280710in}}{\pgfqpoint{3.701070in}{2.291760in}}%
\pgfpathcurveto{\pgfqpoint{3.701070in}{2.302810in}}{\pgfqpoint{3.696680in}{2.313409in}}{\pgfqpoint{3.688866in}{2.321223in}}%
\pgfpathcurveto{\pgfqpoint{3.681052in}{2.329036in}}{\pgfqpoint{3.670453in}{2.333427in}}{\pgfqpoint{3.659403in}{2.333427in}}%
\pgfpathcurveto{\pgfqpoint{3.648353in}{2.333427in}}{\pgfqpoint{3.637754in}{2.329036in}}{\pgfqpoint{3.629940in}{2.321223in}}%
\pgfpathcurveto{\pgfqpoint{3.622127in}{2.313409in}}{\pgfqpoint{3.617737in}{2.302810in}}{\pgfqpoint{3.617737in}{2.291760in}}%
\pgfpathcurveto{\pgfqpoint{3.617737in}{2.280710in}}{\pgfqpoint{3.622127in}{2.270111in}}{\pgfqpoint{3.629940in}{2.262297in}}%
\pgfpathcurveto{\pgfqpoint{3.637754in}{2.254483in}}{\pgfqpoint{3.648353in}{2.250093in}}{\pgfqpoint{3.659403in}{2.250093in}}%
\pgfpathclose%
\pgfusepath{stroke,fill}%
\end{pgfscope}%
\begin{pgfscope}%
\pgfpathrectangle{\pgfqpoint{0.583026in}{0.320679in}}{\pgfqpoint{4.650000in}{3.020000in}}%
\pgfusepath{clip}%
\pgfsetbuttcap%
\pgfsetroundjoin%
\definecolor{currentfill}{rgb}{0.580392,0.403922,0.741176}%
\pgfsetfillcolor{currentfill}%
\pgfsetlinewidth{1.003750pt}%
\definecolor{currentstroke}{rgb}{0.580392,0.403922,0.741176}%
\pgfsetstrokecolor{currentstroke}%
\pgfsetdash{}{0pt}%
\pgfpathmoveto{\pgfqpoint{3.872041in}{2.395092in}}%
\pgfpathcurveto{\pgfqpoint{3.883091in}{2.395092in}}{\pgfqpoint{3.893690in}{2.399482in}}{\pgfqpoint{3.901504in}{2.407296in}}%
\pgfpathcurveto{\pgfqpoint{3.909317in}{2.415109in}}{\pgfqpoint{3.913708in}{2.425709in}}{\pgfqpoint{3.913708in}{2.436759in}}%
\pgfpathcurveto{\pgfqpoint{3.913708in}{2.447809in}}{\pgfqpoint{3.909317in}{2.458408in}}{\pgfqpoint{3.901504in}{2.466221in}}%
\pgfpathcurveto{\pgfqpoint{3.893690in}{2.474035in}}{\pgfqpoint{3.883091in}{2.478425in}}{\pgfqpoint{3.872041in}{2.478425in}}%
\pgfpathcurveto{\pgfqpoint{3.860991in}{2.478425in}}{\pgfqpoint{3.850392in}{2.474035in}}{\pgfqpoint{3.842578in}{2.466221in}}%
\pgfpathcurveto{\pgfqpoint{3.834765in}{2.458408in}}{\pgfqpoint{3.830374in}{2.447809in}}{\pgfqpoint{3.830374in}{2.436759in}}%
\pgfpathcurveto{\pgfqpoint{3.830374in}{2.425709in}}{\pgfqpoint{3.834765in}{2.415109in}}{\pgfqpoint{3.842578in}{2.407296in}}%
\pgfpathcurveto{\pgfqpoint{3.850392in}{2.399482in}}{\pgfqpoint{3.860991in}{2.395092in}}{\pgfqpoint{3.872041in}{2.395092in}}%
\pgfpathclose%
\pgfusepath{stroke,fill}%
\end{pgfscope}%
\begin{pgfscope}%
\pgfpathrectangle{\pgfqpoint{0.583026in}{0.320679in}}{\pgfqpoint{4.650000in}{3.020000in}}%
\pgfusepath{clip}%
\pgfsetbuttcap%
\pgfsetroundjoin%
\definecolor{currentfill}{rgb}{0.580392,0.403922,0.741176}%
\pgfsetfillcolor{currentfill}%
\pgfsetlinewidth{1.003750pt}%
\definecolor{currentstroke}{rgb}{0.580392,0.403922,0.741176}%
\pgfsetstrokecolor{currentstroke}%
\pgfsetdash{}{0pt}%
\pgfpathmoveto{\pgfqpoint{4.095870in}{2.540091in}}%
\pgfpathcurveto{\pgfqpoint{4.106920in}{2.540091in}}{\pgfqpoint{4.117519in}{2.544481in}}{\pgfqpoint{4.125333in}{2.552295in}}%
\pgfpathcurveto{\pgfqpoint{4.133147in}{2.560108in}}{\pgfqpoint{4.137537in}{2.570707in}}{\pgfqpoint{4.137537in}{2.581757in}}%
\pgfpathcurveto{\pgfqpoint{4.137537in}{2.592808in}}{\pgfqpoint{4.133147in}{2.603407in}}{\pgfqpoint{4.125333in}{2.611220in}}%
\pgfpathcurveto{\pgfqpoint{4.117519in}{2.619034in}}{\pgfqpoint{4.106920in}{2.623424in}}{\pgfqpoint{4.095870in}{2.623424in}}%
\pgfpathcurveto{\pgfqpoint{4.084820in}{2.623424in}}{\pgfqpoint{4.074221in}{2.619034in}}{\pgfqpoint{4.066407in}{2.611220in}}%
\pgfpathcurveto{\pgfqpoint{4.058594in}{2.603407in}}{\pgfqpoint{4.054204in}{2.592808in}}{\pgfqpoint{4.054204in}{2.581757in}}%
\pgfpathcurveto{\pgfqpoint{4.054204in}{2.570707in}}{\pgfqpoint{4.058594in}{2.560108in}}{\pgfqpoint{4.066407in}{2.552295in}}%
\pgfpathcurveto{\pgfqpoint{4.074221in}{2.544481in}}{\pgfqpoint{4.084820in}{2.540091in}}{\pgfqpoint{4.095870in}{2.540091in}}%
\pgfpathclose%
\pgfusepath{stroke,fill}%
\end{pgfscope}%
\begin{pgfscope}%
\pgfpathrectangle{\pgfqpoint{0.583026in}{0.320679in}}{\pgfqpoint{4.650000in}{3.020000in}}%
\pgfusepath{clip}%
\pgfsetbuttcap%
\pgfsetroundjoin%
\definecolor{currentfill}{rgb}{0.580392,0.403922,0.741176}%
\pgfsetfillcolor{currentfill}%
\pgfsetlinewidth{1.003750pt}%
\definecolor{currentstroke}{rgb}{0.580392,0.403922,0.741176}%
\pgfsetstrokecolor{currentstroke}%
\pgfsetdash{}{0pt}%
\pgfpathmoveto{\pgfqpoint{4.364465in}{2.710678in}}%
\pgfpathcurveto{\pgfqpoint{4.375515in}{2.710678in}}{\pgfqpoint{4.386114in}{2.715068in}}{\pgfqpoint{4.393928in}{2.722881in}}%
\pgfpathcurveto{\pgfqpoint{4.401742in}{2.730695in}}{\pgfqpoint{4.406132in}{2.741294in}}{\pgfqpoint{4.406132in}{2.752344in}}%
\pgfpathcurveto{\pgfqpoint{4.406132in}{2.763394in}}{\pgfqpoint{4.401742in}{2.773993in}}{\pgfqpoint{4.393928in}{2.781807in}}%
\pgfpathcurveto{\pgfqpoint{4.386114in}{2.789621in}}{\pgfqpoint{4.375515in}{2.794011in}}{\pgfqpoint{4.364465in}{2.794011in}}%
\pgfpathcurveto{\pgfqpoint{4.353415in}{2.794011in}}{\pgfqpoint{4.342816in}{2.789621in}}{\pgfqpoint{4.335002in}{2.781807in}}%
\pgfpathcurveto{\pgfqpoint{4.327189in}{2.773993in}}{\pgfqpoint{4.322799in}{2.763394in}}{\pgfqpoint{4.322799in}{2.752344in}}%
\pgfpathcurveto{\pgfqpoint{4.322799in}{2.741294in}}{\pgfqpoint{4.327189in}{2.730695in}}{\pgfqpoint{4.335002in}{2.722881in}}%
\pgfpathcurveto{\pgfqpoint{4.342816in}{2.715068in}}{\pgfqpoint{4.353415in}{2.710678in}}{\pgfqpoint{4.364465in}{2.710678in}}%
\pgfpathclose%
\pgfusepath{stroke,fill}%
\end{pgfscope}%
\begin{pgfscope}%
\pgfpathrectangle{\pgfqpoint{0.583026in}{0.320679in}}{\pgfqpoint{4.650000in}{3.020000in}}%
\pgfusepath{clip}%
\pgfsetbuttcap%
\pgfsetroundjoin%
\definecolor{currentfill}{rgb}{0.580392,0.403922,0.741176}%
\pgfsetfillcolor{currentfill}%
\pgfsetlinewidth{1.003750pt}%
\definecolor{currentstroke}{rgb}{0.580392,0.403922,0.741176}%
\pgfsetstrokecolor{currentstroke}%
\pgfsetdash{}{0pt}%
\pgfpathmoveto{\pgfqpoint{4.621869in}{2.881264in}}%
\pgfpathcurveto{\pgfqpoint{4.632919in}{2.881264in}}{\pgfqpoint{4.643518in}{2.885655in}}{\pgfqpoint{4.651332in}{2.893468in}}%
\pgfpathcurveto{\pgfqpoint{4.659145in}{2.901282in}}{\pgfqpoint{4.663535in}{2.911881in}}{\pgfqpoint{4.663535in}{2.922931in}}%
\pgfpathcurveto{\pgfqpoint{4.663535in}{2.933981in}}{\pgfqpoint{4.659145in}{2.944580in}}{\pgfqpoint{4.651332in}{2.952394in}}%
\pgfpathcurveto{\pgfqpoint{4.643518in}{2.960207in}}{\pgfqpoint{4.632919in}{2.964598in}}{\pgfqpoint{4.621869in}{2.964598in}}%
\pgfpathcurveto{\pgfqpoint{4.610819in}{2.964598in}}{\pgfqpoint{4.600220in}{2.960207in}}{\pgfqpoint{4.592406in}{2.952394in}}%
\pgfpathcurveto{\pgfqpoint{4.584592in}{2.944580in}}{\pgfqpoint{4.580202in}{2.933981in}}{\pgfqpoint{4.580202in}{2.922931in}}%
\pgfpathcurveto{\pgfqpoint{4.580202in}{2.911881in}}{\pgfqpoint{4.584592in}{2.901282in}}{\pgfqpoint{4.592406in}{2.893468in}}%
\pgfpathcurveto{\pgfqpoint{4.600220in}{2.885655in}}{\pgfqpoint{4.610819in}{2.881264in}}{\pgfqpoint{4.621869in}{2.881264in}}%
\pgfpathclose%
\pgfusepath{stroke,fill}%
\end{pgfscope}%
\begin{pgfscope}%
\pgfpathrectangle{\pgfqpoint{0.583026in}{0.320679in}}{\pgfqpoint{4.650000in}{3.020000in}}%
\pgfusepath{clip}%
\pgfsetbuttcap%
\pgfsetroundjoin%
\definecolor{currentfill}{rgb}{0.580392,0.403922,0.741176}%
\pgfsetfillcolor{currentfill}%
\pgfsetlinewidth{1.003750pt}%
\definecolor{currentstroke}{rgb}{0.580392,0.403922,0.741176}%
\pgfsetstrokecolor{currentstroke}%
\pgfsetdash{}{0pt}%
\pgfpathmoveto{\pgfqpoint{4.812124in}{3.009205in}}%
\pgfpathcurveto{\pgfqpoint{4.823174in}{3.009205in}}{\pgfqpoint{4.833773in}{3.013595in}}{\pgfqpoint{4.841586in}{3.021408in}}%
\pgfpathcurveto{\pgfqpoint{4.849400in}{3.029222in}}{\pgfqpoint{4.853790in}{3.039821in}}{\pgfqpoint{4.853790in}{3.050871in}}%
\pgfpathcurveto{\pgfqpoint{4.853790in}{3.061921in}}{\pgfqpoint{4.849400in}{3.072520in}}{\pgfqpoint{4.841586in}{3.080334in}}%
\pgfpathcurveto{\pgfqpoint{4.833773in}{3.088148in}}{\pgfqpoint{4.823174in}{3.092538in}}{\pgfqpoint{4.812124in}{3.092538in}}%
\pgfpathcurveto{\pgfqpoint{4.801074in}{3.092538in}}{\pgfqpoint{4.790474in}{3.088148in}}{\pgfqpoint{4.782661in}{3.080334in}}%
\pgfpathcurveto{\pgfqpoint{4.774847in}{3.072520in}}{\pgfqpoint{4.770457in}{3.061921in}}{\pgfqpoint{4.770457in}{3.050871in}}%
\pgfpathcurveto{\pgfqpoint{4.770457in}{3.039821in}}{\pgfqpoint{4.774847in}{3.029222in}}{\pgfqpoint{4.782661in}{3.021408in}}%
\pgfpathcurveto{\pgfqpoint{4.790474in}{3.013595in}}{\pgfqpoint{4.801074in}{3.009205in}}{\pgfqpoint{4.812124in}{3.009205in}}%
\pgfpathclose%
\pgfusepath{stroke,fill}%
\end{pgfscope}%
\begin{pgfscope}%
\pgfpathrectangle{\pgfqpoint{0.583026in}{0.320679in}}{\pgfqpoint{4.650000in}{3.020000in}}%
\pgfusepath{clip}%
\pgfsetbuttcap%
\pgfsetroundjoin%
\definecolor{currentfill}{rgb}{0.580392,0.403922,0.741176}%
\pgfsetfillcolor{currentfill}%
\pgfsetlinewidth{1.003750pt}%
\definecolor{currentstroke}{rgb}{0.580392,0.403922,0.741176}%
\pgfsetstrokecolor{currentstroke}%
\pgfsetdash{}{0pt}%
\pgfpathmoveto{\pgfqpoint{4.979996in}{3.120086in}}%
\pgfpathcurveto{\pgfqpoint{4.991046in}{3.120086in}}{\pgfqpoint{5.001645in}{3.124476in}}{\pgfqpoint{5.009458in}{3.132290in}}%
\pgfpathcurveto{\pgfqpoint{5.017272in}{3.140103in}}{\pgfqpoint{5.021662in}{3.150702in}}{\pgfqpoint{5.021662in}{3.161753in}}%
\pgfpathcurveto{\pgfqpoint{5.021662in}{3.172803in}}{\pgfqpoint{5.017272in}{3.183402in}}{\pgfqpoint{5.009458in}{3.191215in}}%
\pgfpathcurveto{\pgfqpoint{5.001645in}{3.199029in}}{\pgfqpoint{4.991046in}{3.203419in}}{\pgfqpoint{4.979996in}{3.203419in}}%
\pgfpathcurveto{\pgfqpoint{4.968945in}{3.203419in}}{\pgfqpoint{4.958346in}{3.199029in}}{\pgfqpoint{4.950533in}{3.191215in}}%
\pgfpathcurveto{\pgfqpoint{4.942719in}{3.183402in}}{\pgfqpoint{4.938329in}{3.172803in}}{\pgfqpoint{4.938329in}{3.161753in}}%
\pgfpathcurveto{\pgfqpoint{4.938329in}{3.150702in}}{\pgfqpoint{4.942719in}{3.140103in}}{\pgfqpoint{4.950533in}{3.132290in}}%
\pgfpathcurveto{\pgfqpoint{4.958346in}{3.124476in}}{\pgfqpoint{4.968945in}{3.120086in}}{\pgfqpoint{4.979996in}{3.120086in}}%
\pgfpathclose%
\pgfusepath{stroke,fill}%
\end{pgfscope}%
\begin{pgfscope}%
\pgfpathrectangle{\pgfqpoint{0.583026in}{0.320679in}}{\pgfqpoint{4.650000in}{3.020000in}}%
\pgfusepath{clip}%
\pgfsetbuttcap%
\pgfsetroundjoin%
\definecolor{currentfill}{rgb}{0.549020,0.337255,0.294118}%
\pgfsetfillcolor{currentfill}%
\pgfsetlinewidth{1.003750pt}%
\definecolor{currentstroke}{rgb}{0.549020,0.337255,0.294118}%
\pgfsetstrokecolor{currentstroke}%
\pgfsetdash{}{0pt}%
\pgfpathmoveto{\pgfqpoint{1.107750in}{0.475990in}}%
\pgfpathcurveto{\pgfqpoint{1.118800in}{0.475990in}}{\pgfqpoint{1.129400in}{0.480381in}}{\pgfqpoint{1.137213in}{0.488194in}}%
\pgfpathcurveto{\pgfqpoint{1.145027in}{0.496008in}}{\pgfqpoint{1.149417in}{0.506607in}}{\pgfqpoint{1.149417in}{0.517657in}}%
\pgfpathcurveto{\pgfqpoint{1.149417in}{0.528707in}}{\pgfqpoint{1.145027in}{0.539306in}}{\pgfqpoint{1.137213in}{0.547120in}}%
\pgfpathcurveto{\pgfqpoint{1.129400in}{0.554933in}}{\pgfqpoint{1.118800in}{0.559324in}}{\pgfqpoint{1.107750in}{0.559324in}}%
\pgfpathcurveto{\pgfqpoint{1.096700in}{0.559324in}}{\pgfqpoint{1.086101in}{0.554933in}}{\pgfqpoint{1.078288in}{0.547120in}}%
\pgfpathcurveto{\pgfqpoint{1.070474in}{0.539306in}}{\pgfqpoint{1.066084in}{0.528707in}}{\pgfqpoint{1.066084in}{0.517657in}}%
\pgfpathcurveto{\pgfqpoint{1.066084in}{0.506607in}}{\pgfqpoint{1.070474in}{0.496008in}}{\pgfqpoint{1.078288in}{0.488194in}}%
\pgfpathcurveto{\pgfqpoint{1.086101in}{0.480381in}}{\pgfqpoint{1.096700in}{0.475990in}}{\pgfqpoint{1.107750in}{0.475990in}}%
\pgfpathclose%
\pgfusepath{stroke,fill}%
\end{pgfscope}%
\begin{pgfscope}%
\pgfpathrectangle{\pgfqpoint{0.583026in}{0.320679in}}{\pgfqpoint{4.650000in}{3.020000in}}%
\pgfusepath{clip}%
\pgfsetbuttcap%
\pgfsetroundjoin%
\definecolor{currentfill}{rgb}{0.549020,0.337255,0.294118}%
\pgfsetfillcolor{currentfill}%
\pgfsetlinewidth{1.003750pt}%
\definecolor{currentstroke}{rgb}{0.549020,0.337255,0.294118}%
\pgfsetstrokecolor{currentstroke}%
\pgfsetdash{}{0pt}%
\pgfpathmoveto{\pgfqpoint{1.342771in}{0.535696in}}%
\pgfpathcurveto{\pgfqpoint{1.353821in}{0.535696in}}{\pgfqpoint{1.364420in}{0.540086in}}{\pgfqpoint{1.372234in}{0.547900in}}%
\pgfpathcurveto{\pgfqpoint{1.380047in}{0.555713in}}{\pgfqpoint{1.384438in}{0.566312in}}{\pgfqpoint{1.384438in}{0.577362in}}%
\pgfpathcurveto{\pgfqpoint{1.384438in}{0.588413in}}{\pgfqpoint{1.380047in}{0.599012in}}{\pgfqpoint{1.372234in}{0.606825in}}%
\pgfpathcurveto{\pgfqpoint{1.364420in}{0.614639in}}{\pgfqpoint{1.353821in}{0.619029in}}{\pgfqpoint{1.342771in}{0.619029in}}%
\pgfpathcurveto{\pgfqpoint{1.331721in}{0.619029in}}{\pgfqpoint{1.321122in}{0.614639in}}{\pgfqpoint{1.313308in}{0.606825in}}%
\pgfpathcurveto{\pgfqpoint{1.305495in}{0.599012in}}{\pgfqpoint{1.301104in}{0.588413in}}{\pgfqpoint{1.301104in}{0.577362in}}%
\pgfpathcurveto{\pgfqpoint{1.301104in}{0.566312in}}{\pgfqpoint{1.305495in}{0.555713in}}{\pgfqpoint{1.313308in}{0.547900in}}%
\pgfpathcurveto{\pgfqpoint{1.321122in}{0.540086in}}{\pgfqpoint{1.331721in}{0.535696in}}{\pgfqpoint{1.342771in}{0.535696in}}%
\pgfpathclose%
\pgfusepath{stroke,fill}%
\end{pgfscope}%
\begin{pgfscope}%
\pgfpathrectangle{\pgfqpoint{0.583026in}{0.320679in}}{\pgfqpoint{4.650000in}{3.020000in}}%
\pgfusepath{clip}%
\pgfsetbuttcap%
\pgfsetroundjoin%
\definecolor{currentfill}{rgb}{0.549020,0.337255,0.294118}%
\pgfsetfillcolor{currentfill}%
\pgfsetlinewidth{1.003750pt}%
\definecolor{currentstroke}{rgb}{0.549020,0.337255,0.294118}%
\pgfsetstrokecolor{currentstroke}%
\pgfsetdash{}{0pt}%
\pgfpathmoveto{\pgfqpoint{1.600175in}{0.603930in}}%
\pgfpathcurveto{\pgfqpoint{1.611225in}{0.603930in}}{\pgfqpoint{1.621824in}{0.608321in}}{\pgfqpoint{1.629637in}{0.616134in}}%
\pgfpathcurveto{\pgfqpoint{1.637451in}{0.623948in}}{\pgfqpoint{1.641841in}{0.634547in}}{\pgfqpoint{1.641841in}{0.645597in}}%
\pgfpathcurveto{\pgfqpoint{1.641841in}{0.656647in}}{\pgfqpoint{1.637451in}{0.667246in}}{\pgfqpoint{1.629637in}{0.675060in}}%
\pgfpathcurveto{\pgfqpoint{1.621824in}{0.682874in}}{\pgfqpoint{1.611225in}{0.687264in}}{\pgfqpoint{1.600175in}{0.687264in}}%
\pgfpathcurveto{\pgfqpoint{1.589124in}{0.687264in}}{\pgfqpoint{1.578525in}{0.682874in}}{\pgfqpoint{1.570712in}{0.675060in}}%
\pgfpathcurveto{\pgfqpoint{1.562898in}{0.667246in}}{\pgfqpoint{1.558508in}{0.656647in}}{\pgfqpoint{1.558508in}{0.645597in}}%
\pgfpathcurveto{\pgfqpoint{1.558508in}{0.634547in}}{\pgfqpoint{1.562898in}{0.623948in}}{\pgfqpoint{1.570712in}{0.616134in}}%
\pgfpathcurveto{\pgfqpoint{1.578525in}{0.608321in}}{\pgfqpoint{1.589124in}{0.603930in}}{\pgfqpoint{1.600175in}{0.603930in}}%
\pgfpathclose%
\pgfusepath{stroke,fill}%
\end{pgfscope}%
\begin{pgfscope}%
\pgfpathrectangle{\pgfqpoint{0.583026in}{0.320679in}}{\pgfqpoint{4.650000in}{3.020000in}}%
\pgfusepath{clip}%
\pgfsetbuttcap%
\pgfsetroundjoin%
\definecolor{currentfill}{rgb}{0.549020,0.337255,0.294118}%
\pgfsetfillcolor{currentfill}%
\pgfsetlinewidth{1.003750pt}%
\definecolor{currentstroke}{rgb}{0.549020,0.337255,0.294118}%
\pgfsetstrokecolor{currentstroke}%
\pgfsetdash{}{0pt}%
\pgfpathmoveto{\pgfqpoint{1.891153in}{0.680695in}}%
\pgfpathcurveto{\pgfqpoint{1.902203in}{0.680695in}}{\pgfqpoint{1.912802in}{0.685085in}}{\pgfqpoint{1.920615in}{0.692898in}}%
\pgfpathcurveto{\pgfqpoint{1.928429in}{0.700712in}}{\pgfqpoint{1.932819in}{0.711311in}}{\pgfqpoint{1.932819in}{0.722361in}}%
\pgfpathcurveto{\pgfqpoint{1.932819in}{0.733411in}}{\pgfqpoint{1.928429in}{0.744010in}}{\pgfqpoint{1.920615in}{0.751824in}}%
\pgfpathcurveto{\pgfqpoint{1.912802in}{0.759638in}}{\pgfqpoint{1.902203in}{0.764028in}}{\pgfqpoint{1.891153in}{0.764028in}}%
\pgfpathcurveto{\pgfqpoint{1.880102in}{0.764028in}}{\pgfqpoint{1.869503in}{0.759638in}}{\pgfqpoint{1.861690in}{0.751824in}}%
\pgfpathcurveto{\pgfqpoint{1.853876in}{0.744010in}}{\pgfqpoint{1.849486in}{0.733411in}}{\pgfqpoint{1.849486in}{0.722361in}}%
\pgfpathcurveto{\pgfqpoint{1.849486in}{0.711311in}}{\pgfqpoint{1.853876in}{0.700712in}}{\pgfqpoint{1.861690in}{0.692898in}}%
\pgfpathcurveto{\pgfqpoint{1.869503in}{0.685085in}}{\pgfqpoint{1.880102in}{0.680695in}}{\pgfqpoint{1.891153in}{0.680695in}}%
\pgfpathclose%
\pgfusepath{stroke,fill}%
\end{pgfscope}%
\begin{pgfscope}%
\pgfpathrectangle{\pgfqpoint{0.583026in}{0.320679in}}{\pgfqpoint{4.650000in}{3.020000in}}%
\pgfusepath{clip}%
\pgfsetbuttcap%
\pgfsetroundjoin%
\definecolor{currentfill}{rgb}{0.549020,0.337255,0.294118}%
\pgfsetfillcolor{currentfill}%
\pgfsetlinewidth{1.003750pt}%
\definecolor{currentstroke}{rgb}{0.549020,0.337255,0.294118}%
\pgfsetstrokecolor{currentstroke}%
\pgfsetdash{}{0pt}%
\pgfpathmoveto{\pgfqpoint{2.114982in}{0.740400in}}%
\pgfpathcurveto{\pgfqpoint{2.126032in}{0.740400in}}{\pgfqpoint{2.136631in}{0.744790in}}{\pgfqpoint{2.144445in}{0.752604in}}%
\pgfpathcurveto{\pgfqpoint{2.152258in}{0.760417in}}{\pgfqpoint{2.156648in}{0.771016in}}{\pgfqpoint{2.156648in}{0.782067in}}%
\pgfpathcurveto{\pgfqpoint{2.156648in}{0.793117in}}{\pgfqpoint{2.152258in}{0.803716in}}{\pgfqpoint{2.144445in}{0.811529in}}%
\pgfpathcurveto{\pgfqpoint{2.136631in}{0.819343in}}{\pgfqpoint{2.126032in}{0.823733in}}{\pgfqpoint{2.114982in}{0.823733in}}%
\pgfpathcurveto{\pgfqpoint{2.103932in}{0.823733in}}{\pgfqpoint{2.093333in}{0.819343in}}{\pgfqpoint{2.085519in}{0.811529in}}%
\pgfpathcurveto{\pgfqpoint{2.077705in}{0.803716in}}{\pgfqpoint{2.073315in}{0.793117in}}{\pgfqpoint{2.073315in}{0.782067in}}%
\pgfpathcurveto{\pgfqpoint{2.073315in}{0.771016in}}{\pgfqpoint{2.077705in}{0.760417in}}{\pgfqpoint{2.085519in}{0.752604in}}%
\pgfpathcurveto{\pgfqpoint{2.093333in}{0.744790in}}{\pgfqpoint{2.103932in}{0.740400in}}{\pgfqpoint{2.114982in}{0.740400in}}%
\pgfpathclose%
\pgfusepath{stroke,fill}%
\end{pgfscope}%
\begin{pgfscope}%
\pgfpathrectangle{\pgfqpoint{0.583026in}{0.320679in}}{\pgfqpoint{4.650000in}{3.020000in}}%
\pgfusepath{clip}%
\pgfsetbuttcap%
\pgfsetroundjoin%
\definecolor{currentfill}{rgb}{0.549020,0.337255,0.294118}%
\pgfsetfillcolor{currentfill}%
\pgfsetlinewidth{1.003750pt}%
\definecolor{currentstroke}{rgb}{0.549020,0.337255,0.294118}%
\pgfsetstrokecolor{currentstroke}%
\pgfsetdash{}{0pt}%
\pgfpathmoveto{\pgfqpoint{2.327619in}{0.791576in}}%
\pgfpathcurveto{\pgfqpoint{2.338670in}{0.791576in}}{\pgfqpoint{2.349269in}{0.795966in}}{\pgfqpoint{2.357082in}{0.803780in}}%
\pgfpathcurveto{\pgfqpoint{2.364896in}{0.811593in}}{\pgfqpoint{2.369286in}{0.822192in}}{\pgfqpoint{2.369286in}{0.833243in}}%
\pgfpathcurveto{\pgfqpoint{2.369286in}{0.844293in}}{\pgfqpoint{2.364896in}{0.854892in}}{\pgfqpoint{2.357082in}{0.862705in}}%
\pgfpathcurveto{\pgfqpoint{2.349269in}{0.870519in}}{\pgfqpoint{2.338670in}{0.874909in}}{\pgfqpoint{2.327619in}{0.874909in}}%
\pgfpathcurveto{\pgfqpoint{2.316569in}{0.874909in}}{\pgfqpoint{2.305970in}{0.870519in}}{\pgfqpoint{2.298157in}{0.862705in}}%
\pgfpathcurveto{\pgfqpoint{2.290343in}{0.854892in}}{\pgfqpoint{2.285953in}{0.844293in}}{\pgfqpoint{2.285953in}{0.833243in}}%
\pgfpathcurveto{\pgfqpoint{2.285953in}{0.822192in}}{\pgfqpoint{2.290343in}{0.811593in}}{\pgfqpoint{2.298157in}{0.803780in}}%
\pgfpathcurveto{\pgfqpoint{2.305970in}{0.795966in}}{\pgfqpoint{2.316569in}{0.791576in}}{\pgfqpoint{2.327619in}{0.791576in}}%
\pgfpathclose%
\pgfusepath{stroke,fill}%
\end{pgfscope}%
\begin{pgfscope}%
\pgfpathrectangle{\pgfqpoint{0.583026in}{0.320679in}}{\pgfqpoint{4.650000in}{3.020000in}}%
\pgfusepath{clip}%
\pgfsetbuttcap%
\pgfsetroundjoin%
\definecolor{currentfill}{rgb}{0.549020,0.337255,0.294118}%
\pgfsetfillcolor{currentfill}%
\pgfsetlinewidth{1.003750pt}%
\definecolor{currentstroke}{rgb}{0.549020,0.337255,0.294118}%
\pgfsetstrokecolor{currentstroke}%
\pgfsetdash{}{0pt}%
\pgfpathmoveto{\pgfqpoint{2.573832in}{0.859811in}}%
\pgfpathcurveto{\pgfqpoint{2.584882in}{0.859811in}}{\pgfqpoint{2.595481in}{0.864201in}}{\pgfqpoint{2.603294in}{0.872015in}}%
\pgfpathcurveto{\pgfqpoint{2.611108in}{0.879828in}}{\pgfqpoint{2.615498in}{0.890427in}}{\pgfqpoint{2.615498in}{0.901477in}}%
\pgfpathcurveto{\pgfqpoint{2.615498in}{0.912527in}}{\pgfqpoint{2.611108in}{0.923126in}}{\pgfqpoint{2.603294in}{0.930940in}}%
\pgfpathcurveto{\pgfqpoint{2.595481in}{0.938754in}}{\pgfqpoint{2.584882in}{0.943144in}}{\pgfqpoint{2.573832in}{0.943144in}}%
\pgfpathcurveto{\pgfqpoint{2.562781in}{0.943144in}}{\pgfqpoint{2.552182in}{0.938754in}}{\pgfqpoint{2.544369in}{0.930940in}}%
\pgfpathcurveto{\pgfqpoint{2.536555in}{0.923126in}}{\pgfqpoint{2.532165in}{0.912527in}}{\pgfqpoint{2.532165in}{0.901477in}}%
\pgfpathcurveto{\pgfqpoint{2.532165in}{0.890427in}}{\pgfqpoint{2.536555in}{0.879828in}}{\pgfqpoint{2.544369in}{0.872015in}}%
\pgfpathcurveto{\pgfqpoint{2.552182in}{0.864201in}}{\pgfqpoint{2.562781in}{0.859811in}}{\pgfqpoint{2.573832in}{0.859811in}}%
\pgfpathclose%
\pgfusepath{stroke,fill}%
\end{pgfscope}%
\begin{pgfscope}%
\pgfpathrectangle{\pgfqpoint{0.583026in}{0.320679in}}{\pgfqpoint{4.650000in}{3.020000in}}%
\pgfusepath{clip}%
\pgfsetbuttcap%
\pgfsetroundjoin%
\definecolor{currentfill}{rgb}{0.549020,0.337255,0.294118}%
\pgfsetfillcolor{currentfill}%
\pgfsetlinewidth{1.003750pt}%
\definecolor{currentstroke}{rgb}{0.549020,0.337255,0.294118}%
\pgfsetstrokecolor{currentstroke}%
\pgfsetdash{}{0pt}%
\pgfpathmoveto{\pgfqpoint{2.820044in}{0.919516in}}%
\pgfpathcurveto{\pgfqpoint{2.831094in}{0.919516in}}{\pgfqpoint{2.841693in}{0.923906in}}{\pgfqpoint{2.849507in}{0.931720in}}%
\pgfpathcurveto{\pgfqpoint{2.857320in}{0.939534in}}{\pgfqpoint{2.861710in}{0.950133in}}{\pgfqpoint{2.861710in}{0.961183in}}%
\pgfpathcurveto{\pgfqpoint{2.861710in}{0.972233in}}{\pgfqpoint{2.857320in}{0.982832in}}{\pgfqpoint{2.849507in}{0.990645in}}%
\pgfpathcurveto{\pgfqpoint{2.841693in}{0.998459in}}{\pgfqpoint{2.831094in}{1.002849in}}{\pgfqpoint{2.820044in}{1.002849in}}%
\pgfpathcurveto{\pgfqpoint{2.808994in}{1.002849in}}{\pgfqpoint{2.798395in}{0.998459in}}{\pgfqpoint{2.790581in}{0.990645in}}%
\pgfpathcurveto{\pgfqpoint{2.782767in}{0.982832in}}{\pgfqpoint{2.778377in}{0.972233in}}{\pgfqpoint{2.778377in}{0.961183in}}%
\pgfpathcurveto{\pgfqpoint{2.778377in}{0.950133in}}{\pgfqpoint{2.782767in}{0.939534in}}{\pgfqpoint{2.790581in}{0.931720in}}%
\pgfpathcurveto{\pgfqpoint{2.798395in}{0.923906in}}{\pgfqpoint{2.808994in}{0.919516in}}{\pgfqpoint{2.820044in}{0.919516in}}%
\pgfpathclose%
\pgfusepath{stroke,fill}%
\end{pgfscope}%
\begin{pgfscope}%
\pgfpathrectangle{\pgfqpoint{0.583026in}{0.320679in}}{\pgfqpoint{4.650000in}{3.020000in}}%
\pgfusepath{clip}%
\pgfsetbuttcap%
\pgfsetroundjoin%
\definecolor{currentfill}{rgb}{0.549020,0.337255,0.294118}%
\pgfsetfillcolor{currentfill}%
\pgfsetlinewidth{1.003750pt}%
\definecolor{currentstroke}{rgb}{0.549020,0.337255,0.294118}%
\pgfsetstrokecolor{currentstroke}%
\pgfsetdash{}{0pt}%
\pgfpathmoveto{\pgfqpoint{3.043873in}{0.970692in}}%
\pgfpathcurveto{\pgfqpoint{3.054923in}{0.970692in}}{\pgfqpoint{3.065522in}{0.975082in}}{\pgfqpoint{3.073336in}{0.982896in}}%
\pgfpathcurveto{\pgfqpoint{3.081149in}{0.990710in}}{\pgfqpoint{3.085540in}{1.001309in}}{\pgfqpoint{3.085540in}{1.012359in}}%
\pgfpathcurveto{\pgfqpoint{3.085540in}{1.023409in}}{\pgfqpoint{3.081149in}{1.034008in}}{\pgfqpoint{3.073336in}{1.041822in}}%
\pgfpathcurveto{\pgfqpoint{3.065522in}{1.049635in}}{\pgfqpoint{3.054923in}{1.054025in}}{\pgfqpoint{3.043873in}{1.054025in}}%
\pgfpathcurveto{\pgfqpoint{3.032823in}{1.054025in}}{\pgfqpoint{3.022224in}{1.049635in}}{\pgfqpoint{3.014410in}{1.041822in}}%
\pgfpathcurveto{\pgfqpoint{3.006597in}{1.034008in}}{\pgfqpoint{3.002206in}{1.023409in}}{\pgfqpoint{3.002206in}{1.012359in}}%
\pgfpathcurveto{\pgfqpoint{3.002206in}{1.001309in}}{\pgfqpoint{3.006597in}{0.990710in}}{\pgfqpoint{3.014410in}{0.982896in}}%
\pgfpathcurveto{\pgfqpoint{3.022224in}{0.975082in}}{\pgfqpoint{3.032823in}{0.970692in}}{\pgfqpoint{3.043873in}{0.970692in}}%
\pgfpathclose%
\pgfusepath{stroke,fill}%
\end{pgfscope}%
\begin{pgfscope}%
\pgfpathrectangle{\pgfqpoint{0.583026in}{0.320679in}}{\pgfqpoint{4.650000in}{3.020000in}}%
\pgfusepath{clip}%
\pgfsetbuttcap%
\pgfsetroundjoin%
\definecolor{currentfill}{rgb}{0.549020,0.337255,0.294118}%
\pgfsetfillcolor{currentfill}%
\pgfsetlinewidth{1.003750pt}%
\definecolor{currentstroke}{rgb}{0.549020,0.337255,0.294118}%
\pgfsetstrokecolor{currentstroke}%
\pgfsetdash{}{0pt}%
\pgfpathmoveto{\pgfqpoint{3.290085in}{1.047456in}}%
\pgfpathcurveto{\pgfqpoint{3.301135in}{1.047456in}}{\pgfqpoint{3.311734in}{1.051846in}}{\pgfqpoint{3.319548in}{1.059660in}}%
\pgfpathcurveto{\pgfqpoint{3.327361in}{1.067474in}}{\pgfqpoint{3.331752in}{1.078073in}}{\pgfqpoint{3.331752in}{1.089123in}}%
\pgfpathcurveto{\pgfqpoint{3.331752in}{1.100173in}}{\pgfqpoint{3.327361in}{1.110772in}}{\pgfqpoint{3.319548in}{1.118586in}}%
\pgfpathcurveto{\pgfqpoint{3.311734in}{1.126399in}}{\pgfqpoint{3.301135in}{1.130789in}}{\pgfqpoint{3.290085in}{1.130789in}}%
\pgfpathcurveto{\pgfqpoint{3.279035in}{1.130789in}}{\pgfqpoint{3.268436in}{1.126399in}}{\pgfqpoint{3.260622in}{1.118586in}}%
\pgfpathcurveto{\pgfqpoint{3.252809in}{1.110772in}}{\pgfqpoint{3.248418in}{1.100173in}}{\pgfqpoint{3.248418in}{1.089123in}}%
\pgfpathcurveto{\pgfqpoint{3.248418in}{1.078073in}}{\pgfqpoint{3.252809in}{1.067474in}}{\pgfqpoint{3.260622in}{1.059660in}}%
\pgfpathcurveto{\pgfqpoint{3.268436in}{1.051846in}}{\pgfqpoint{3.279035in}{1.047456in}}{\pgfqpoint{3.290085in}{1.047456in}}%
\pgfpathclose%
\pgfusepath{stroke,fill}%
\end{pgfscope}%
\begin{pgfscope}%
\pgfpathrectangle{\pgfqpoint{0.583026in}{0.320679in}}{\pgfqpoint{4.650000in}{3.020000in}}%
\pgfusepath{clip}%
\pgfsetbuttcap%
\pgfsetroundjoin%
\definecolor{currentfill}{rgb}{0.549020,0.337255,0.294118}%
\pgfsetfillcolor{currentfill}%
\pgfsetlinewidth{1.003750pt}%
\definecolor{currentstroke}{rgb}{0.549020,0.337255,0.294118}%
\pgfsetstrokecolor{currentstroke}%
\pgfsetdash{}{0pt}%
\pgfpathmoveto{\pgfqpoint{3.569872in}{1.098632in}}%
\pgfpathcurveto{\pgfqpoint{3.580922in}{1.098632in}}{\pgfqpoint{3.591521in}{1.103022in}}{\pgfqpoint{3.599334in}{1.110836in}}%
\pgfpathcurveto{\pgfqpoint{3.607148in}{1.118650in}}{\pgfqpoint{3.611538in}{1.129249in}}{\pgfqpoint{3.611538in}{1.140299in}}%
\pgfpathcurveto{\pgfqpoint{3.611538in}{1.151349in}}{\pgfqpoint{3.607148in}{1.161948in}}{\pgfqpoint{3.599334in}{1.169762in}}%
\pgfpathcurveto{\pgfqpoint{3.591521in}{1.177575in}}{\pgfqpoint{3.580922in}{1.181966in}}{\pgfqpoint{3.569872in}{1.181966in}}%
\pgfpathcurveto{\pgfqpoint{3.558821in}{1.181966in}}{\pgfqpoint{3.548222in}{1.177575in}}{\pgfqpoint{3.540409in}{1.169762in}}%
\pgfpathcurveto{\pgfqpoint{3.532595in}{1.161948in}}{\pgfqpoint{3.528205in}{1.151349in}}{\pgfqpoint{3.528205in}{1.140299in}}%
\pgfpathcurveto{\pgfqpoint{3.528205in}{1.129249in}}{\pgfqpoint{3.532595in}{1.118650in}}{\pgfqpoint{3.540409in}{1.110836in}}%
\pgfpathcurveto{\pgfqpoint{3.548222in}{1.103022in}}{\pgfqpoint{3.558821in}{1.098632in}}{\pgfqpoint{3.569872in}{1.098632in}}%
\pgfpathclose%
\pgfusepath{stroke,fill}%
\end{pgfscope}%
\begin{pgfscope}%
\pgfpathrectangle{\pgfqpoint{0.583026in}{0.320679in}}{\pgfqpoint{4.650000in}{3.020000in}}%
\pgfusepath{clip}%
\pgfsetbuttcap%
\pgfsetroundjoin%
\definecolor{currentfill}{rgb}{0.549020,0.337255,0.294118}%
\pgfsetfillcolor{currentfill}%
\pgfsetlinewidth{1.003750pt}%
\definecolor{currentstroke}{rgb}{0.549020,0.337255,0.294118}%
\pgfsetstrokecolor{currentstroke}%
\pgfsetdash{}{0pt}%
\pgfpathmoveto{\pgfqpoint{3.726552in}{1.141279in}}%
\pgfpathcurveto{\pgfqpoint{3.737602in}{1.141279in}}{\pgfqpoint{3.748201in}{1.145669in}}{\pgfqpoint{3.756015in}{1.153483in}}%
\pgfpathcurveto{\pgfqpoint{3.763828in}{1.161296in}}{\pgfqpoint{3.768219in}{1.171895in}}{\pgfqpoint{3.768219in}{1.182946in}}%
\pgfpathcurveto{\pgfqpoint{3.768219in}{1.193996in}}{\pgfqpoint{3.763828in}{1.204595in}}{\pgfqpoint{3.756015in}{1.212408in}}%
\pgfpathcurveto{\pgfqpoint{3.748201in}{1.220222in}}{\pgfqpoint{3.737602in}{1.224612in}}{\pgfqpoint{3.726552in}{1.224612in}}%
\pgfpathcurveto{\pgfqpoint{3.715502in}{1.224612in}}{\pgfqpoint{3.704903in}{1.220222in}}{\pgfqpoint{3.697089in}{1.212408in}}%
\pgfpathcurveto{\pgfqpoint{3.689276in}{1.204595in}}{\pgfqpoint{3.684885in}{1.193996in}}{\pgfqpoint{3.684885in}{1.182946in}}%
\pgfpathcurveto{\pgfqpoint{3.684885in}{1.171895in}}{\pgfqpoint{3.689276in}{1.161296in}}{\pgfqpoint{3.697089in}{1.153483in}}%
\pgfpathcurveto{\pgfqpoint{3.704903in}{1.145669in}}{\pgfqpoint{3.715502in}{1.141279in}}{\pgfqpoint{3.726552in}{1.141279in}}%
\pgfpathclose%
\pgfusepath{stroke,fill}%
\end{pgfscope}%
\begin{pgfscope}%
\pgfpathrectangle{\pgfqpoint{0.583026in}{0.320679in}}{\pgfqpoint{4.650000in}{3.020000in}}%
\pgfusepath{clip}%
\pgfsetbuttcap%
\pgfsetroundjoin%
\definecolor{currentfill}{rgb}{0.549020,0.337255,0.294118}%
\pgfsetfillcolor{currentfill}%
\pgfsetlinewidth{1.003750pt}%
\definecolor{currentstroke}{rgb}{0.549020,0.337255,0.294118}%
\pgfsetstrokecolor{currentstroke}%
\pgfsetdash{}{0pt}%
\pgfpathmoveto{\pgfqpoint{3.894424in}{1.183926in}}%
\pgfpathcurveto{\pgfqpoint{3.905474in}{1.183926in}}{\pgfqpoint{3.916073in}{1.188316in}}{\pgfqpoint{3.923887in}{1.196129in}}%
\pgfpathcurveto{\pgfqpoint{3.931700in}{1.203943in}}{\pgfqpoint{3.936091in}{1.214542in}}{\pgfqpoint{3.936091in}{1.225592in}}%
\pgfpathcurveto{\pgfqpoint{3.936091in}{1.236642in}}{\pgfqpoint{3.931700in}{1.247241in}}{\pgfqpoint{3.923887in}{1.255055in}}%
\pgfpathcurveto{\pgfqpoint{3.916073in}{1.262869in}}{\pgfqpoint{3.905474in}{1.267259in}}{\pgfqpoint{3.894424in}{1.267259in}}%
\pgfpathcurveto{\pgfqpoint{3.883374in}{1.267259in}}{\pgfqpoint{3.872775in}{1.262869in}}{\pgfqpoint{3.864961in}{1.255055in}}%
\pgfpathcurveto{\pgfqpoint{3.857148in}{1.247241in}}{\pgfqpoint{3.852757in}{1.236642in}}{\pgfqpoint{3.852757in}{1.225592in}}%
\pgfpathcurveto{\pgfqpoint{3.852757in}{1.214542in}}{\pgfqpoint{3.857148in}{1.203943in}}{\pgfqpoint{3.864961in}{1.196129in}}%
\pgfpathcurveto{\pgfqpoint{3.872775in}{1.188316in}}{\pgfqpoint{3.883374in}{1.183926in}}{\pgfqpoint{3.894424in}{1.183926in}}%
\pgfpathclose%
\pgfusepath{stroke,fill}%
\end{pgfscope}%
\begin{pgfscope}%
\pgfpathrectangle{\pgfqpoint{0.583026in}{0.320679in}}{\pgfqpoint{4.650000in}{3.020000in}}%
\pgfusepath{clip}%
\pgfsetbuttcap%
\pgfsetroundjoin%
\definecolor{currentfill}{rgb}{0.549020,0.337255,0.294118}%
\pgfsetfillcolor{currentfill}%
\pgfsetlinewidth{1.003750pt}%
\definecolor{currentstroke}{rgb}{0.549020,0.337255,0.294118}%
\pgfsetstrokecolor{currentstroke}%
\pgfsetdash{}{0pt}%
\pgfpathmoveto{\pgfqpoint{4.095870in}{1.243631in}}%
\pgfpathcurveto{\pgfqpoint{4.106920in}{1.243631in}}{\pgfqpoint{4.117519in}{1.248021in}}{\pgfqpoint{4.125333in}{1.255835in}}%
\pgfpathcurveto{\pgfqpoint{4.133147in}{1.263648in}}{\pgfqpoint{4.137537in}{1.274248in}}{\pgfqpoint{4.137537in}{1.285298in}}%
\pgfpathcurveto{\pgfqpoint{4.137537in}{1.296348in}}{\pgfqpoint{4.133147in}{1.306947in}}{\pgfqpoint{4.125333in}{1.314760in}}%
\pgfpathcurveto{\pgfqpoint{4.117519in}{1.322574in}}{\pgfqpoint{4.106920in}{1.326964in}}{\pgfqpoint{4.095870in}{1.326964in}}%
\pgfpathcurveto{\pgfqpoint{4.084820in}{1.326964in}}{\pgfqpoint{4.074221in}{1.322574in}}{\pgfqpoint{4.066407in}{1.314760in}}%
\pgfpathcurveto{\pgfqpoint{4.058594in}{1.306947in}}{\pgfqpoint{4.054204in}{1.296348in}}{\pgfqpoint{4.054204in}{1.285298in}}%
\pgfpathcurveto{\pgfqpoint{4.054204in}{1.274248in}}{\pgfqpoint{4.058594in}{1.263648in}}{\pgfqpoint{4.066407in}{1.255835in}}%
\pgfpathcurveto{\pgfqpoint{4.074221in}{1.248021in}}{\pgfqpoint{4.084820in}{1.243631in}}{\pgfqpoint{4.095870in}{1.243631in}}%
\pgfpathclose%
\pgfusepath{stroke,fill}%
\end{pgfscope}%
\begin{pgfscope}%
\pgfpathrectangle{\pgfqpoint{0.583026in}{0.320679in}}{\pgfqpoint{4.650000in}{3.020000in}}%
\pgfusepath{clip}%
\pgfsetbuttcap%
\pgfsetroundjoin%
\definecolor{currentfill}{rgb}{0.549020,0.337255,0.294118}%
\pgfsetfillcolor{currentfill}%
\pgfsetlinewidth{1.003750pt}%
\definecolor{currentstroke}{rgb}{0.549020,0.337255,0.294118}%
\pgfsetstrokecolor{currentstroke}%
\pgfsetdash{}{0pt}%
\pgfpathmoveto{\pgfqpoint{4.409231in}{1.311866in}}%
\pgfpathcurveto{\pgfqpoint{4.420281in}{1.311866in}}{\pgfqpoint{4.430880in}{1.316256in}}{\pgfqpoint{4.438694in}{1.324070in}}%
\pgfpathcurveto{\pgfqpoint{4.446507in}{1.331883in}}{\pgfqpoint{4.450898in}{1.342482in}}{\pgfqpoint{4.450898in}{1.353532in}}%
\pgfpathcurveto{\pgfqpoint{4.450898in}{1.364583in}}{\pgfqpoint{4.446507in}{1.375182in}}{\pgfqpoint{4.438694in}{1.382995in}}%
\pgfpathcurveto{\pgfqpoint{4.430880in}{1.390809in}}{\pgfqpoint{4.420281in}{1.395199in}}{\pgfqpoint{4.409231in}{1.395199in}}%
\pgfpathcurveto{\pgfqpoint{4.398181in}{1.395199in}}{\pgfqpoint{4.387582in}{1.390809in}}{\pgfqpoint{4.379768in}{1.382995in}}%
\pgfpathcurveto{\pgfqpoint{4.371955in}{1.375182in}}{\pgfqpoint{4.367564in}{1.364583in}}{\pgfqpoint{4.367564in}{1.353532in}}%
\pgfpathcurveto{\pgfqpoint{4.367564in}{1.342482in}}{\pgfqpoint{4.371955in}{1.331883in}}{\pgfqpoint{4.379768in}{1.324070in}}%
\pgfpathcurveto{\pgfqpoint{4.387582in}{1.316256in}}{\pgfqpoint{4.398181in}{1.311866in}}{\pgfqpoint{4.409231in}{1.311866in}}%
\pgfpathclose%
\pgfusepath{stroke,fill}%
\end{pgfscope}%
\begin{pgfscope}%
\pgfpathrectangle{\pgfqpoint{0.583026in}{0.320679in}}{\pgfqpoint{4.650000in}{3.020000in}}%
\pgfusepath{clip}%
\pgfsetbuttcap%
\pgfsetroundjoin%
\definecolor{currentfill}{rgb}{0.549020,0.337255,0.294118}%
\pgfsetfillcolor{currentfill}%
\pgfsetlinewidth{1.003750pt}%
\definecolor{currentstroke}{rgb}{0.549020,0.337255,0.294118}%
\pgfsetstrokecolor{currentstroke}%
\pgfsetdash{}{0pt}%
\pgfpathmoveto{\pgfqpoint{4.677826in}{1.388630in}}%
\pgfpathcurveto{\pgfqpoint{4.688876in}{1.388630in}}{\pgfqpoint{4.699475in}{1.393020in}}{\pgfqpoint{4.707289in}{1.400834in}}%
\pgfpathcurveto{\pgfqpoint{4.715103in}{1.408647in}}{\pgfqpoint{4.719493in}{1.419246in}}{\pgfqpoint{4.719493in}{1.430296in}}%
\pgfpathcurveto{\pgfqpoint{4.719493in}{1.441347in}}{\pgfqpoint{4.715103in}{1.451946in}}{\pgfqpoint{4.707289in}{1.459759in}}%
\pgfpathcurveto{\pgfqpoint{4.699475in}{1.467573in}}{\pgfqpoint{4.688876in}{1.471963in}}{\pgfqpoint{4.677826in}{1.471963in}}%
\pgfpathcurveto{\pgfqpoint{4.666776in}{1.471963in}}{\pgfqpoint{4.656177in}{1.467573in}}{\pgfqpoint{4.648363in}{1.459759in}}%
\pgfpathcurveto{\pgfqpoint{4.640550in}{1.451946in}}{\pgfqpoint{4.636159in}{1.441347in}}{\pgfqpoint{4.636159in}{1.430296in}}%
\pgfpathcurveto{\pgfqpoint{4.636159in}{1.419246in}}{\pgfqpoint{4.640550in}{1.408647in}}{\pgfqpoint{4.648363in}{1.400834in}}%
\pgfpathcurveto{\pgfqpoint{4.656177in}{1.393020in}}{\pgfqpoint{4.666776in}{1.388630in}}{\pgfqpoint{4.677826in}{1.388630in}}%
\pgfpathclose%
\pgfusepath{stroke,fill}%
\end{pgfscope}%
\begin{pgfscope}%
\pgfpathrectangle{\pgfqpoint{0.583026in}{0.320679in}}{\pgfqpoint{4.650000in}{3.020000in}}%
\pgfusepath{clip}%
\pgfsetbuttcap%
\pgfsetroundjoin%
\definecolor{currentfill}{rgb}{0.549020,0.337255,0.294118}%
\pgfsetfillcolor{currentfill}%
\pgfsetlinewidth{1.003750pt}%
\definecolor{currentstroke}{rgb}{0.549020,0.337255,0.294118}%
\pgfsetstrokecolor{currentstroke}%
\pgfsetdash{}{0pt}%
\pgfpathmoveto{\pgfqpoint{4.879272in}{1.448335in}}%
\pgfpathcurveto{\pgfqpoint{4.890323in}{1.448335in}}{\pgfqpoint{4.900922in}{1.452725in}}{\pgfqpoint{4.908735in}{1.460539in}}%
\pgfpathcurveto{\pgfqpoint{4.916549in}{1.468353in}}{\pgfqpoint{4.920939in}{1.478952in}}{\pgfqpoint{4.920939in}{1.490002in}}%
\pgfpathcurveto{\pgfqpoint{4.920939in}{1.501052in}}{\pgfqpoint{4.916549in}{1.511651in}}{\pgfqpoint{4.908735in}{1.519465in}}%
\pgfpathcurveto{\pgfqpoint{4.900922in}{1.527278in}}{\pgfqpoint{4.890323in}{1.531669in}}{\pgfqpoint{4.879272in}{1.531669in}}%
\pgfpathcurveto{\pgfqpoint{4.868222in}{1.531669in}}{\pgfqpoint{4.857623in}{1.527278in}}{\pgfqpoint{4.849810in}{1.519465in}}%
\pgfpathcurveto{\pgfqpoint{4.841996in}{1.511651in}}{\pgfqpoint{4.837606in}{1.501052in}}{\pgfqpoint{4.837606in}{1.490002in}}%
\pgfpathcurveto{\pgfqpoint{4.837606in}{1.478952in}}{\pgfqpoint{4.841996in}{1.468353in}}{\pgfqpoint{4.849810in}{1.460539in}}%
\pgfpathcurveto{\pgfqpoint{4.857623in}{1.452725in}}{\pgfqpoint{4.868222in}{1.448335in}}{\pgfqpoint{4.879272in}{1.448335in}}%
\pgfpathclose%
\pgfusepath{stroke,fill}%
\end{pgfscope}%
\begin{pgfscope}%
\pgfpathrectangle{\pgfqpoint{0.583026in}{0.320679in}}{\pgfqpoint{4.650000in}{3.020000in}}%
\pgfusepath{clip}%
\pgfsetbuttcap%
\pgfsetroundjoin%
\definecolor{currentfill}{rgb}{0.549020,0.337255,0.294118}%
\pgfsetfillcolor{currentfill}%
\pgfsetlinewidth{1.003750pt}%
\definecolor{currentstroke}{rgb}{0.549020,0.337255,0.294118}%
\pgfsetstrokecolor{currentstroke}%
\pgfsetdash{}{0pt}%
\pgfpathmoveto{\pgfqpoint{4.979996in}{1.465394in}}%
\pgfpathcurveto{\pgfqpoint{4.991046in}{1.465394in}}{\pgfqpoint{5.001645in}{1.469784in}}{\pgfqpoint{5.009458in}{1.477598in}}%
\pgfpathcurveto{\pgfqpoint{5.017272in}{1.485411in}}{\pgfqpoint{5.021662in}{1.496010in}}{\pgfqpoint{5.021662in}{1.507061in}}%
\pgfpathcurveto{\pgfqpoint{5.021662in}{1.518111in}}{\pgfqpoint{5.017272in}{1.528710in}}{\pgfqpoint{5.009458in}{1.536523in}}%
\pgfpathcurveto{\pgfqpoint{5.001645in}{1.544337in}}{\pgfqpoint{4.991046in}{1.548727in}}{\pgfqpoint{4.979996in}{1.548727in}}%
\pgfpathcurveto{\pgfqpoint{4.968945in}{1.548727in}}{\pgfqpoint{4.958346in}{1.544337in}}{\pgfqpoint{4.950533in}{1.536523in}}%
\pgfpathcurveto{\pgfqpoint{4.942719in}{1.528710in}}{\pgfqpoint{4.938329in}{1.518111in}}{\pgfqpoint{4.938329in}{1.507061in}}%
\pgfpathcurveto{\pgfqpoint{4.938329in}{1.496010in}}{\pgfqpoint{4.942719in}{1.485411in}}{\pgfqpoint{4.950533in}{1.477598in}}%
\pgfpathcurveto{\pgfqpoint{4.958346in}{1.469784in}}{\pgfqpoint{4.968945in}{1.465394in}}{\pgfqpoint{4.979996in}{1.465394in}}%
\pgfpathclose%
\pgfusepath{stroke,fill}%
\end{pgfscope}%
\begin{pgfscope}%
\pgfsetbuttcap%
\pgfsetroundjoin%
\definecolor{currentfill}{rgb}{0.000000,0.000000,0.000000}%
\pgfsetfillcolor{currentfill}%
\pgfsetlinewidth{0.803000pt}%
\definecolor{currentstroke}{rgb}{0.000000,0.000000,0.000000}%
\pgfsetstrokecolor{currentstroke}%
\pgfsetdash{}{0pt}%
\pgfsys@defobject{currentmarker}{\pgfqpoint{0.000000in}{-0.048611in}}{\pgfqpoint{0.000000in}{0.000000in}}{%
\pgfpathmoveto{\pgfqpoint{0.000000in}{0.000000in}}%
\pgfpathlineto{\pgfqpoint{0.000000in}{-0.048611in}}%
\pgfusepath{stroke,fill}%
}%
\begin{pgfscope}%
\pgfsys@transformshift{0.794389in}{0.320679in}%
\pgfsys@useobject{currentmarker}{}%
\end{pgfscope}%
\end{pgfscope}%
\begin{pgfscope}%
\definecolor{textcolor}{rgb}{0.000000,0.000000,0.000000}%
\pgfsetstrokecolor{textcolor}%
\pgfsetfillcolor{textcolor}%
\pgftext[x=0.794389in,y=0.223457in,,top]{\color{textcolor}\rmfamily\fontsize{10.000000}{12.000000}\selectfont \(\displaystyle 0.0\)}%
\end{pgfscope}%
\begin{pgfscope}%
\pgfsetbuttcap%
\pgfsetroundjoin%
\definecolor{currentfill}{rgb}{0.000000,0.000000,0.000000}%
\pgfsetfillcolor{currentfill}%
\pgfsetlinewidth{0.803000pt}%
\definecolor{currentstroke}{rgb}{0.000000,0.000000,0.000000}%
\pgfsetstrokecolor{currentstroke}%
\pgfsetdash{}{0pt}%
\pgfsys@defobject{currentmarker}{\pgfqpoint{0.000000in}{-0.048611in}}{\pgfqpoint{0.000000in}{0.000000in}}{%
\pgfpathmoveto{\pgfqpoint{0.000000in}{0.000000in}}%
\pgfpathlineto{\pgfqpoint{0.000000in}{-0.048611in}}%
\pgfusepath{stroke,fill}%
}%
\begin{pgfscope}%
\pgfsys@transformshift{1.353962in}{0.320679in}%
\pgfsys@useobject{currentmarker}{}%
\end{pgfscope}%
\end{pgfscope}%
\begin{pgfscope}%
\definecolor{textcolor}{rgb}{0.000000,0.000000,0.000000}%
\pgfsetstrokecolor{textcolor}%
\pgfsetfillcolor{textcolor}%
\pgftext[x=1.353962in,y=0.223457in,,top]{\color{textcolor}\rmfamily\fontsize{10.000000}{12.000000}\selectfont \(\displaystyle 0.5\)}%
\end{pgfscope}%
\begin{pgfscope}%
\pgfsetbuttcap%
\pgfsetroundjoin%
\definecolor{currentfill}{rgb}{0.000000,0.000000,0.000000}%
\pgfsetfillcolor{currentfill}%
\pgfsetlinewidth{0.803000pt}%
\definecolor{currentstroke}{rgb}{0.000000,0.000000,0.000000}%
\pgfsetstrokecolor{currentstroke}%
\pgfsetdash{}{0pt}%
\pgfsys@defobject{currentmarker}{\pgfqpoint{0.000000in}{-0.048611in}}{\pgfqpoint{0.000000in}{0.000000in}}{%
\pgfpathmoveto{\pgfqpoint{0.000000in}{0.000000in}}%
\pgfpathlineto{\pgfqpoint{0.000000in}{-0.048611in}}%
\pgfusepath{stroke,fill}%
}%
\begin{pgfscope}%
\pgfsys@transformshift{1.913535in}{0.320679in}%
\pgfsys@useobject{currentmarker}{}%
\end{pgfscope}%
\end{pgfscope}%
\begin{pgfscope}%
\definecolor{textcolor}{rgb}{0.000000,0.000000,0.000000}%
\pgfsetstrokecolor{textcolor}%
\pgfsetfillcolor{textcolor}%
\pgftext[x=1.913535in,y=0.223457in,,top]{\color{textcolor}\rmfamily\fontsize{10.000000}{12.000000}\selectfont \(\displaystyle 1.0\)}%
\end{pgfscope}%
\begin{pgfscope}%
\pgfsetbuttcap%
\pgfsetroundjoin%
\definecolor{currentfill}{rgb}{0.000000,0.000000,0.000000}%
\pgfsetfillcolor{currentfill}%
\pgfsetlinewidth{0.803000pt}%
\definecolor{currentstroke}{rgb}{0.000000,0.000000,0.000000}%
\pgfsetstrokecolor{currentstroke}%
\pgfsetdash{}{0pt}%
\pgfsys@defobject{currentmarker}{\pgfqpoint{0.000000in}{-0.048611in}}{\pgfqpoint{0.000000in}{0.000000in}}{%
\pgfpathmoveto{\pgfqpoint{0.000000in}{0.000000in}}%
\pgfpathlineto{\pgfqpoint{0.000000in}{-0.048611in}}%
\pgfusepath{stroke,fill}%
}%
\begin{pgfscope}%
\pgfsys@transformshift{2.473108in}{0.320679in}%
\pgfsys@useobject{currentmarker}{}%
\end{pgfscope}%
\end{pgfscope}%
\begin{pgfscope}%
\definecolor{textcolor}{rgb}{0.000000,0.000000,0.000000}%
\pgfsetstrokecolor{textcolor}%
\pgfsetfillcolor{textcolor}%
\pgftext[x=2.473108in,y=0.223457in,,top]{\color{textcolor}\rmfamily\fontsize{10.000000}{12.000000}\selectfont \(\displaystyle 1.5\)}%
\end{pgfscope}%
\begin{pgfscope}%
\pgfsetbuttcap%
\pgfsetroundjoin%
\definecolor{currentfill}{rgb}{0.000000,0.000000,0.000000}%
\pgfsetfillcolor{currentfill}%
\pgfsetlinewidth{0.803000pt}%
\definecolor{currentstroke}{rgb}{0.000000,0.000000,0.000000}%
\pgfsetstrokecolor{currentstroke}%
\pgfsetdash{}{0pt}%
\pgfsys@defobject{currentmarker}{\pgfqpoint{0.000000in}{-0.048611in}}{\pgfqpoint{0.000000in}{0.000000in}}{%
\pgfpathmoveto{\pgfqpoint{0.000000in}{0.000000in}}%
\pgfpathlineto{\pgfqpoint{0.000000in}{-0.048611in}}%
\pgfusepath{stroke,fill}%
}%
\begin{pgfscope}%
\pgfsys@transformshift{3.032681in}{0.320679in}%
\pgfsys@useobject{currentmarker}{}%
\end{pgfscope}%
\end{pgfscope}%
\begin{pgfscope}%
\definecolor{textcolor}{rgb}{0.000000,0.000000,0.000000}%
\pgfsetstrokecolor{textcolor}%
\pgfsetfillcolor{textcolor}%
\pgftext[x=3.032681in,y=0.223457in,,top]{\color{textcolor}\rmfamily\fontsize{10.000000}{12.000000}\selectfont \(\displaystyle 2.0\)}%
\end{pgfscope}%
\begin{pgfscope}%
\pgfsetbuttcap%
\pgfsetroundjoin%
\definecolor{currentfill}{rgb}{0.000000,0.000000,0.000000}%
\pgfsetfillcolor{currentfill}%
\pgfsetlinewidth{0.803000pt}%
\definecolor{currentstroke}{rgb}{0.000000,0.000000,0.000000}%
\pgfsetstrokecolor{currentstroke}%
\pgfsetdash{}{0pt}%
\pgfsys@defobject{currentmarker}{\pgfqpoint{0.000000in}{-0.048611in}}{\pgfqpoint{0.000000in}{0.000000in}}{%
\pgfpathmoveto{\pgfqpoint{0.000000in}{0.000000in}}%
\pgfpathlineto{\pgfqpoint{0.000000in}{-0.048611in}}%
\pgfusepath{stroke,fill}%
}%
\begin{pgfscope}%
\pgfsys@transformshift{3.592254in}{0.320679in}%
\pgfsys@useobject{currentmarker}{}%
\end{pgfscope}%
\end{pgfscope}%
\begin{pgfscope}%
\definecolor{textcolor}{rgb}{0.000000,0.000000,0.000000}%
\pgfsetstrokecolor{textcolor}%
\pgfsetfillcolor{textcolor}%
\pgftext[x=3.592254in,y=0.223457in,,top]{\color{textcolor}\rmfamily\fontsize{10.000000}{12.000000}\selectfont \(\displaystyle 2.5\)}%
\end{pgfscope}%
\begin{pgfscope}%
\pgfsetbuttcap%
\pgfsetroundjoin%
\definecolor{currentfill}{rgb}{0.000000,0.000000,0.000000}%
\pgfsetfillcolor{currentfill}%
\pgfsetlinewidth{0.803000pt}%
\definecolor{currentstroke}{rgb}{0.000000,0.000000,0.000000}%
\pgfsetstrokecolor{currentstroke}%
\pgfsetdash{}{0pt}%
\pgfsys@defobject{currentmarker}{\pgfqpoint{0.000000in}{-0.048611in}}{\pgfqpoint{0.000000in}{0.000000in}}{%
\pgfpathmoveto{\pgfqpoint{0.000000in}{0.000000in}}%
\pgfpathlineto{\pgfqpoint{0.000000in}{-0.048611in}}%
\pgfusepath{stroke,fill}%
}%
\begin{pgfscope}%
\pgfsys@transformshift{4.151827in}{0.320679in}%
\pgfsys@useobject{currentmarker}{}%
\end{pgfscope}%
\end{pgfscope}%
\begin{pgfscope}%
\definecolor{textcolor}{rgb}{0.000000,0.000000,0.000000}%
\pgfsetstrokecolor{textcolor}%
\pgfsetfillcolor{textcolor}%
\pgftext[x=4.151827in,y=0.223457in,,top]{\color{textcolor}\rmfamily\fontsize{10.000000}{12.000000}\selectfont \(\displaystyle 3.0\)}%
\end{pgfscope}%
\begin{pgfscope}%
\pgfsetbuttcap%
\pgfsetroundjoin%
\definecolor{currentfill}{rgb}{0.000000,0.000000,0.000000}%
\pgfsetfillcolor{currentfill}%
\pgfsetlinewidth{0.803000pt}%
\definecolor{currentstroke}{rgb}{0.000000,0.000000,0.000000}%
\pgfsetstrokecolor{currentstroke}%
\pgfsetdash{}{0pt}%
\pgfsys@defobject{currentmarker}{\pgfqpoint{0.000000in}{-0.048611in}}{\pgfqpoint{0.000000in}{0.000000in}}{%
\pgfpathmoveto{\pgfqpoint{0.000000in}{0.000000in}}%
\pgfpathlineto{\pgfqpoint{0.000000in}{-0.048611in}}%
\pgfusepath{stroke,fill}%
}%
\begin{pgfscope}%
\pgfsys@transformshift{4.711400in}{0.320679in}%
\pgfsys@useobject{currentmarker}{}%
\end{pgfscope}%
\end{pgfscope}%
\begin{pgfscope}%
\definecolor{textcolor}{rgb}{0.000000,0.000000,0.000000}%
\pgfsetstrokecolor{textcolor}%
\pgfsetfillcolor{textcolor}%
\pgftext[x=4.711400in,y=0.223457in,,top]{\color{textcolor}\rmfamily\fontsize{10.000000}{12.000000}\selectfont \(\displaystyle 3.5\)}%
\end{pgfscope}%
\begin{pgfscope}%
\pgfsetbuttcap%
\pgfsetroundjoin%
\definecolor{currentfill}{rgb}{0.000000,0.000000,0.000000}%
\pgfsetfillcolor{currentfill}%
\pgfsetlinewidth{0.803000pt}%
\definecolor{currentstroke}{rgb}{0.000000,0.000000,0.000000}%
\pgfsetstrokecolor{currentstroke}%
\pgfsetdash{}{0pt}%
\pgfsys@defobject{currentmarker}{\pgfqpoint{-0.048611in}{0.000000in}}{\pgfqpoint{0.000000in}{0.000000in}}{%
\pgfpathmoveto{\pgfqpoint{0.000000in}{0.000000in}}%
\pgfpathlineto{\pgfqpoint{-0.048611in}{0.000000in}}%
\pgfusepath{stroke,fill}%
}%
\begin{pgfscope}%
\pgfsys@transformshift{0.583026in}{0.457952in}%
\pgfsys@useobject{currentmarker}{}%
\end{pgfscope}%
\end{pgfscope}%
\begin{pgfscope}%
\definecolor{textcolor}{rgb}{0.000000,0.000000,0.000000}%
\pgfsetstrokecolor{textcolor}%
\pgfsetfillcolor{textcolor}%
\pgftext[x=0.100000in,y=0.409726in,left,base]{\color{textcolor}\rmfamily\fontsize{10.000000}{12.000000}\selectfont \(\displaystyle 0.0000\)}%
\end{pgfscope}%
\begin{pgfscope}%
\pgfsetbuttcap%
\pgfsetroundjoin%
\definecolor{currentfill}{rgb}{0.000000,0.000000,0.000000}%
\pgfsetfillcolor{currentfill}%
\pgfsetlinewidth{0.803000pt}%
\definecolor{currentstroke}{rgb}{0.000000,0.000000,0.000000}%
\pgfsetstrokecolor{currentstroke}%
\pgfsetdash{}{0pt}%
\pgfsys@defobject{currentmarker}{\pgfqpoint{-0.048611in}{0.000000in}}{\pgfqpoint{0.000000in}{0.000000in}}{%
\pgfpathmoveto{\pgfqpoint{0.000000in}{0.000000in}}%
\pgfpathlineto{\pgfqpoint{-0.048611in}{0.000000in}}%
\pgfusepath{stroke,fill}%
}%
\begin{pgfscope}%
\pgfsys@transformshift{0.583026in}{0.884419in}%
\pgfsys@useobject{currentmarker}{}%
\end{pgfscope}%
\end{pgfscope}%
\begin{pgfscope}%
\definecolor{textcolor}{rgb}{0.000000,0.000000,0.000000}%
\pgfsetstrokecolor{textcolor}%
\pgfsetfillcolor{textcolor}%
\pgftext[x=0.100000in,y=0.836193in,left,base]{\color{textcolor}\rmfamily\fontsize{10.000000}{12.000000}\selectfont \(\displaystyle 0.0005\)}%
\end{pgfscope}%
\begin{pgfscope}%
\pgfsetbuttcap%
\pgfsetroundjoin%
\definecolor{currentfill}{rgb}{0.000000,0.000000,0.000000}%
\pgfsetfillcolor{currentfill}%
\pgfsetlinewidth{0.803000pt}%
\definecolor{currentstroke}{rgb}{0.000000,0.000000,0.000000}%
\pgfsetstrokecolor{currentstroke}%
\pgfsetdash{}{0pt}%
\pgfsys@defobject{currentmarker}{\pgfqpoint{-0.048611in}{0.000000in}}{\pgfqpoint{0.000000in}{0.000000in}}{%
\pgfpathmoveto{\pgfqpoint{0.000000in}{0.000000in}}%
\pgfpathlineto{\pgfqpoint{-0.048611in}{0.000000in}}%
\pgfusepath{stroke,fill}%
}%
\begin{pgfscope}%
\pgfsys@transformshift{0.583026in}{1.310886in}%
\pgfsys@useobject{currentmarker}{}%
\end{pgfscope}%
\end{pgfscope}%
\begin{pgfscope}%
\definecolor{textcolor}{rgb}{0.000000,0.000000,0.000000}%
\pgfsetstrokecolor{textcolor}%
\pgfsetfillcolor{textcolor}%
\pgftext[x=0.100000in,y=1.262660in,left,base]{\color{textcolor}\rmfamily\fontsize{10.000000}{12.000000}\selectfont \(\displaystyle 0.0010\)}%
\end{pgfscope}%
\begin{pgfscope}%
\pgfsetbuttcap%
\pgfsetroundjoin%
\definecolor{currentfill}{rgb}{0.000000,0.000000,0.000000}%
\pgfsetfillcolor{currentfill}%
\pgfsetlinewidth{0.803000pt}%
\definecolor{currentstroke}{rgb}{0.000000,0.000000,0.000000}%
\pgfsetstrokecolor{currentstroke}%
\pgfsetdash{}{0pt}%
\pgfsys@defobject{currentmarker}{\pgfqpoint{-0.048611in}{0.000000in}}{\pgfqpoint{0.000000in}{0.000000in}}{%
\pgfpathmoveto{\pgfqpoint{0.000000in}{0.000000in}}%
\pgfpathlineto{\pgfqpoint{-0.048611in}{0.000000in}}%
\pgfusepath{stroke,fill}%
}%
\begin{pgfscope}%
\pgfsys@transformshift{0.583026in}{1.737353in}%
\pgfsys@useobject{currentmarker}{}%
\end{pgfscope}%
\end{pgfscope}%
\begin{pgfscope}%
\definecolor{textcolor}{rgb}{0.000000,0.000000,0.000000}%
\pgfsetstrokecolor{textcolor}%
\pgfsetfillcolor{textcolor}%
\pgftext[x=0.100000in,y=1.689127in,left,base]{\color{textcolor}\rmfamily\fontsize{10.000000}{12.000000}\selectfont \(\displaystyle 0.0015\)}%
\end{pgfscope}%
\begin{pgfscope}%
\pgfsetbuttcap%
\pgfsetroundjoin%
\definecolor{currentfill}{rgb}{0.000000,0.000000,0.000000}%
\pgfsetfillcolor{currentfill}%
\pgfsetlinewidth{0.803000pt}%
\definecolor{currentstroke}{rgb}{0.000000,0.000000,0.000000}%
\pgfsetstrokecolor{currentstroke}%
\pgfsetdash{}{0pt}%
\pgfsys@defobject{currentmarker}{\pgfqpoint{-0.048611in}{0.000000in}}{\pgfqpoint{0.000000in}{0.000000in}}{%
\pgfpathmoveto{\pgfqpoint{0.000000in}{0.000000in}}%
\pgfpathlineto{\pgfqpoint{-0.048611in}{0.000000in}}%
\pgfusepath{stroke,fill}%
}%
\begin{pgfscope}%
\pgfsys@transformshift{0.583026in}{2.163820in}%
\pgfsys@useobject{currentmarker}{}%
\end{pgfscope}%
\end{pgfscope}%
\begin{pgfscope}%
\definecolor{textcolor}{rgb}{0.000000,0.000000,0.000000}%
\pgfsetstrokecolor{textcolor}%
\pgfsetfillcolor{textcolor}%
\pgftext[x=0.100000in,y=2.115594in,left,base]{\color{textcolor}\rmfamily\fontsize{10.000000}{12.000000}\selectfont \(\displaystyle 0.0020\)}%
\end{pgfscope}%
\begin{pgfscope}%
\pgfsetbuttcap%
\pgfsetroundjoin%
\definecolor{currentfill}{rgb}{0.000000,0.000000,0.000000}%
\pgfsetfillcolor{currentfill}%
\pgfsetlinewidth{0.803000pt}%
\definecolor{currentstroke}{rgb}{0.000000,0.000000,0.000000}%
\pgfsetstrokecolor{currentstroke}%
\pgfsetdash{}{0pt}%
\pgfsys@defobject{currentmarker}{\pgfqpoint{-0.048611in}{0.000000in}}{\pgfqpoint{0.000000in}{0.000000in}}{%
\pgfpathmoveto{\pgfqpoint{0.000000in}{0.000000in}}%
\pgfpathlineto{\pgfqpoint{-0.048611in}{0.000000in}}%
\pgfusepath{stroke,fill}%
}%
\begin{pgfscope}%
\pgfsys@transformshift{0.583026in}{2.590287in}%
\pgfsys@useobject{currentmarker}{}%
\end{pgfscope}%
\end{pgfscope}%
\begin{pgfscope}%
\definecolor{textcolor}{rgb}{0.000000,0.000000,0.000000}%
\pgfsetstrokecolor{textcolor}%
\pgfsetfillcolor{textcolor}%
\pgftext[x=0.100000in,y=2.542062in,left,base]{\color{textcolor}\rmfamily\fontsize{10.000000}{12.000000}\selectfont \(\displaystyle 0.0025\)}%
\end{pgfscope}%
\begin{pgfscope}%
\pgfsetbuttcap%
\pgfsetroundjoin%
\definecolor{currentfill}{rgb}{0.000000,0.000000,0.000000}%
\pgfsetfillcolor{currentfill}%
\pgfsetlinewidth{0.803000pt}%
\definecolor{currentstroke}{rgb}{0.000000,0.000000,0.000000}%
\pgfsetstrokecolor{currentstroke}%
\pgfsetdash{}{0pt}%
\pgfsys@defobject{currentmarker}{\pgfqpoint{-0.048611in}{0.000000in}}{\pgfqpoint{0.000000in}{0.000000in}}{%
\pgfpathmoveto{\pgfqpoint{0.000000in}{0.000000in}}%
\pgfpathlineto{\pgfqpoint{-0.048611in}{0.000000in}}%
\pgfusepath{stroke,fill}%
}%
\begin{pgfscope}%
\pgfsys@transformshift{0.583026in}{3.016754in}%
\pgfsys@useobject{currentmarker}{}%
\end{pgfscope}%
\end{pgfscope}%
\begin{pgfscope}%
\definecolor{textcolor}{rgb}{0.000000,0.000000,0.000000}%
\pgfsetstrokecolor{textcolor}%
\pgfsetfillcolor{textcolor}%
\pgftext[x=0.100000in,y=2.968529in,left,base]{\color{textcolor}\rmfamily\fontsize{10.000000}{12.000000}\selectfont \(\displaystyle 0.0030\)}%
\end{pgfscope}%
\begin{pgfscope}%
\pgfpathrectangle{\pgfqpoint{0.583026in}{0.320679in}}{\pgfqpoint{4.650000in}{3.020000in}}%
\pgfusepath{clip}%
\pgfsetrectcap%
\pgfsetroundjoin%
\pgfsetlinewidth{1.505625pt}%
\definecolor{currentstroke}{rgb}{0.121569,0.466667,0.705882}%
\pgfsetstrokecolor{currentstroke}%
\pgfsetdash{}{0pt}%
\pgfpathmoveto{\pgfqpoint{0.794389in}{0.457952in}}%
\pgfpathlineto{\pgfqpoint{1.259457in}{0.690897in}}%
\pgfpathlineto{\pgfqpoint{1.724524in}{0.923842in}}%
\pgfpathlineto{\pgfqpoint{2.189591in}{1.156788in}}%
\pgfpathlineto{\pgfqpoint{2.654659in}{1.389733in}}%
\pgfpathlineto{\pgfqpoint{3.119726in}{1.622678in}}%
\pgfpathlineto{\pgfqpoint{3.584794in}{1.855624in}}%
\pgfpathlineto{\pgfqpoint{4.049861in}{2.088569in}}%
\pgfpathlineto{\pgfqpoint{4.514928in}{2.321514in}}%
\pgfpathlineto{\pgfqpoint{4.979996in}{2.554460in}}%
\pgfusepath{stroke}%
\end{pgfscope}%
\begin{pgfscope}%
\pgfpathrectangle{\pgfqpoint{0.583026in}{0.320679in}}{\pgfqpoint{4.650000in}{3.020000in}}%
\pgfusepath{clip}%
\pgfsetrectcap%
\pgfsetroundjoin%
\pgfsetlinewidth{1.505625pt}%
\definecolor{currentstroke}{rgb}{1.000000,0.498039,0.054902}%
\pgfsetstrokecolor{currentstroke}%
\pgfsetdash{}{0pt}%
\pgfpathmoveto{\pgfqpoint{0.794389in}{0.457952in}}%
\pgfpathlineto{\pgfqpoint{1.259457in}{0.756583in}}%
\pgfpathlineto{\pgfqpoint{1.724524in}{1.055214in}}%
\pgfpathlineto{\pgfqpoint{2.189591in}{1.353845in}}%
\pgfpathlineto{\pgfqpoint{2.654659in}{1.652476in}}%
\pgfpathlineto{\pgfqpoint{3.119726in}{1.951107in}}%
\pgfpathlineto{\pgfqpoint{3.584794in}{2.249738in}}%
\pgfpathlineto{\pgfqpoint{4.049861in}{2.548369in}}%
\pgfpathlineto{\pgfqpoint{4.514928in}{2.847000in}}%
\pgfpathlineto{\pgfqpoint{4.979996in}{3.145631in}}%
\pgfusepath{stroke}%
\end{pgfscope}%
\begin{pgfscope}%
\pgfpathrectangle{\pgfqpoint{0.583026in}{0.320679in}}{\pgfqpoint{4.650000in}{3.020000in}}%
\pgfusepath{clip}%
\pgfsetrectcap%
\pgfsetroundjoin%
\pgfsetlinewidth{1.505625pt}%
\definecolor{currentstroke}{rgb}{0.172549,0.627451,0.172549}%
\pgfsetstrokecolor{currentstroke}%
\pgfsetdash{}{0pt}%
\pgfpathmoveto{\pgfqpoint{0.794389in}{0.457952in}}%
\pgfpathlineto{\pgfqpoint{1.259457in}{0.573820in}}%
\pgfpathlineto{\pgfqpoint{1.724524in}{0.689689in}}%
\pgfpathlineto{\pgfqpoint{2.189591in}{0.805557in}}%
\pgfpathlineto{\pgfqpoint{2.654659in}{0.921426in}}%
\pgfpathlineto{\pgfqpoint{3.119726in}{1.037295in}}%
\pgfpathlineto{\pgfqpoint{3.584794in}{1.153163in}}%
\pgfpathlineto{\pgfqpoint{4.049861in}{1.269032in}}%
\pgfpathlineto{\pgfqpoint{4.514928in}{1.384901in}}%
\pgfpathlineto{\pgfqpoint{4.979996in}{1.500769in}}%
\pgfusepath{stroke}%
\end{pgfscope}%
\begin{pgfscope}%
\pgfpathrectangle{\pgfqpoint{0.583026in}{0.320679in}}{\pgfqpoint{4.650000in}{3.020000in}}%
\pgfusepath{clip}%
\pgfsetrectcap%
\pgfsetroundjoin%
\pgfsetlinewidth{1.505625pt}%
\definecolor{currentstroke}{rgb}{0.839216,0.152941,0.156863}%
\pgfsetstrokecolor{currentstroke}%
\pgfsetdash{}{0pt}%
\pgfpathmoveto{\pgfqpoint{0.794389in}{0.457952in}}%
\pgfpathlineto{\pgfqpoint{1.259457in}{0.690897in}}%
\pgfpathlineto{\pgfqpoint{1.724524in}{0.923842in}}%
\pgfpathlineto{\pgfqpoint{2.189591in}{1.156788in}}%
\pgfpathlineto{\pgfqpoint{2.654659in}{1.389733in}}%
\pgfpathlineto{\pgfqpoint{3.119726in}{1.622678in}}%
\pgfpathlineto{\pgfqpoint{3.584794in}{1.855624in}}%
\pgfpathlineto{\pgfqpoint{4.049861in}{2.088569in}}%
\pgfpathlineto{\pgfqpoint{4.514928in}{2.321514in}}%
\pgfpathlineto{\pgfqpoint{4.979996in}{2.554460in}}%
\pgfusepath{stroke}%
\end{pgfscope}%
\begin{pgfscope}%
\pgfpathrectangle{\pgfqpoint{0.583026in}{0.320679in}}{\pgfqpoint{4.650000in}{3.020000in}}%
\pgfusepath{clip}%
\pgfsetrectcap%
\pgfsetroundjoin%
\pgfsetlinewidth{1.505625pt}%
\definecolor{currentstroke}{rgb}{0.580392,0.403922,0.741176}%
\pgfsetstrokecolor{currentstroke}%
\pgfsetdash{}{0pt}%
\pgfpathmoveto{\pgfqpoint{0.794389in}{0.457952in}}%
\pgfpathlineto{\pgfqpoint{1.259457in}{0.756583in}}%
\pgfpathlineto{\pgfqpoint{1.724524in}{1.055214in}}%
\pgfpathlineto{\pgfqpoint{2.189591in}{1.353845in}}%
\pgfpathlineto{\pgfqpoint{2.654659in}{1.652476in}}%
\pgfpathlineto{\pgfqpoint{3.119726in}{1.951107in}}%
\pgfpathlineto{\pgfqpoint{3.584794in}{2.249738in}}%
\pgfpathlineto{\pgfqpoint{4.049861in}{2.548369in}}%
\pgfpathlineto{\pgfqpoint{4.514928in}{2.847000in}}%
\pgfpathlineto{\pgfqpoint{4.979996in}{3.145631in}}%
\pgfusepath{stroke}%
\end{pgfscope}%
\begin{pgfscope}%
\pgfpathrectangle{\pgfqpoint{0.583026in}{0.320679in}}{\pgfqpoint{4.650000in}{3.020000in}}%
\pgfusepath{clip}%
\pgfsetrectcap%
\pgfsetroundjoin%
\pgfsetlinewidth{1.505625pt}%
\definecolor{currentstroke}{rgb}{0.549020,0.337255,0.294118}%
\pgfsetstrokecolor{currentstroke}%
\pgfsetdash{}{0pt}%
\pgfpathmoveto{\pgfqpoint{0.794389in}{0.457952in}}%
\pgfpathlineto{\pgfqpoint{1.259457in}{0.573820in}}%
\pgfpathlineto{\pgfqpoint{1.724524in}{0.689689in}}%
\pgfpathlineto{\pgfqpoint{2.189591in}{0.805557in}}%
\pgfpathlineto{\pgfqpoint{2.654659in}{0.921426in}}%
\pgfpathlineto{\pgfqpoint{3.119726in}{1.037295in}}%
\pgfpathlineto{\pgfqpoint{3.584794in}{1.153163in}}%
\pgfpathlineto{\pgfqpoint{4.049861in}{1.269032in}}%
\pgfpathlineto{\pgfqpoint{4.514928in}{1.384901in}}%
\pgfpathlineto{\pgfqpoint{4.979996in}{1.500769in}}%
\pgfusepath{stroke}%
\end{pgfscope}%
\begin{pgfscope}%
\pgfsetrectcap%
\pgfsetmiterjoin%
\pgfsetlinewidth{0.803000pt}%
\definecolor{currentstroke}{rgb}{0.000000,0.000000,0.000000}%
\pgfsetstrokecolor{currentstroke}%
\pgfsetdash{}{0pt}%
\pgfpathmoveto{\pgfqpoint{0.583026in}{0.320679in}}%
\pgfpathlineto{\pgfqpoint{0.583026in}{3.340679in}}%
\pgfusepath{stroke}%
\end{pgfscope}%
\begin{pgfscope}%
\pgfsetrectcap%
\pgfsetmiterjoin%
\pgfsetlinewidth{0.803000pt}%
\definecolor{currentstroke}{rgb}{0.000000,0.000000,0.000000}%
\pgfsetstrokecolor{currentstroke}%
\pgfsetdash{}{0pt}%
\pgfpathmoveto{\pgfqpoint{5.233026in}{0.320679in}}%
\pgfpathlineto{\pgfqpoint{5.233026in}{3.340679in}}%
\pgfusepath{stroke}%
\end{pgfscope}%
\begin{pgfscope}%
\pgfsetrectcap%
\pgfsetmiterjoin%
\pgfsetlinewidth{0.803000pt}%
\definecolor{currentstroke}{rgb}{0.000000,0.000000,0.000000}%
\pgfsetstrokecolor{currentstroke}%
\pgfsetdash{}{0pt}%
\pgfpathmoveto{\pgfqpoint{0.583026in}{0.320679in}}%
\pgfpathlineto{\pgfqpoint{5.233026in}{0.320679in}}%
\pgfusepath{stroke}%
\end{pgfscope}%
\begin{pgfscope}%
\pgfsetrectcap%
\pgfsetmiterjoin%
\pgfsetlinewidth{0.803000pt}%
\definecolor{currentstroke}{rgb}{0.000000,0.000000,0.000000}%
\pgfsetstrokecolor{currentstroke}%
\pgfsetdash{}{0pt}%
\pgfpathmoveto{\pgfqpoint{0.583026in}{3.340679in}}%
\pgfpathlineto{\pgfqpoint{5.233026in}{3.340679in}}%
\pgfusepath{stroke}%
\end{pgfscope}%
\end{pgfpicture}%
\makeatother%
\endgroup%

  \end{figure}


  \newpage
  \begin{appendices}
    \addtocontents{toc}{\protect\setcounter{tocdepth}{2}}
    \makeatletter
    \addtocontents{toc}{%
    \begingroup
    \let\protect\l@chapter\protect\l@section
    \let\protect\l@section\protect\l@subsection
    }

    \section{Bibliografía}

    (All the info)

    \addtocontents{toc}{\endgroup}
  \end{appendices}


\end{document}
