\documentclass[12pt, a4paper, titlepage]{article}

%INFORMACIÓN
\title{\textbf {TITULO}}
\author{{\Large NOMBRE}\\DNI}
\date{}

%PAQUETES
\usepackage[utf8]{inputenc}
\usepackage[spanish]{babel}
\usepackage[centertags]{amsmath} %Excluir ecuaciones de la enumeración automática
\usepackage{tocloft} %Crear listas (por ejemplo, de ecuaciones)
\usepackage[skip=12pt]{parskip} %Añadir espacio tras los párrafos
\usepackage{csvsimple} %Tablas desde archivos .csv
\usepackage{pgfplots} %Gráficas desde matplotlib con .pgf
\pgfplotsset{compat=1.16}
\usepackage{float} %Controlar el posicionamiento de gráficas y tablas con H
\usepackage{enumitem} %Cambiar los estilos de las listas
\usepackage[toc,page]{appendix} %Anexos
\usepackage{chngcntr} %Numeración de capítulos por partes


%CONFIGURACIÓN
\renewcommand{\contentsname}{Índice}
\renewcommand{\partname}{Experiencia}
\renewcommand{\listtablename}{Lista de Tablas}
\renewcommand{\listfigurename}{Lista de Figuras}
\renewcommand{\appendixpagename}{Anexos}
\renewcommand{\appendixtocname}{\large Anexos}
\renewcommand{\appendixname}{Anexo}

\newcommand{\listecuacionesname}{\Large Lista de Ecuaciones}
\newlistof{ecuaciones}{equ}{\listecuacionesname}
\newcommand{\ecuaciones}[1]{\addcontentsline{equ}{ecuaciones}{\protect\numberline{\theequation}#1}\par}

\newcommand{\ectag}[1]{\tag*{#1}}% Etiquetar ecuación con nombre

\linespread{1.3}
\counterwithin*{section}{part}



%DOCUMENTO
\begin{document}
  \maketitle

  \tableofcontents
  %\listoftables
  %\listoffigures
  %\listofecuaciones

  \newpage
  \part*{Introducción}
  \addcontentsline{toc}{part}{Introducción}
  Prueba de ácídífícácíón


  \newpage
  \part{Corriente Continua}


  \newpage
  \part{Corriente Alterna}

  \section{Fórmulas}

  \subsection{Ecuaciones}

  Ecuación numerada:
  \begin{equation} \label{ec:test}
    f(x) = x^2
  \end{equation}
  \ecuaciones{Ecuación número \ref{ec:test}}

  Varias ecuaciones alineadas:
  \begin{align} %Se alinean en &
    f(x) &= 2x + \int^a_b y^2 dy \nonumber \\
    f(x) &= 2 (x\lambda + \frac{1}{2}y) \label{ec:lambda}
  \end{align}
  \ecuaciones{Lista de ecuaciones}
  Ver~\ref{ec:lambda}


  Una ecuación cómo $y = \sqrt{x}$ dentro del texto.

  \subsection{Matrices}

  $\left(
  \begin{matrix}
    1 & 0 & 0\\
    0 & 1 & 0\\
    0 & 0 & 1
  \end{matrix}
  \right)$

  Una matriz en su propio entorno


  \newpage
  \begin{appendices}
    \addtocontents{toc}{\protect\setcounter{tocdepth}{2}}
    \makeatletter
    \addtocontents{toc}{%
    \begingroup
    \let\protect\l@chapter\protect\l@section
    \let\protect\l@section\protect\l@subsection
    }

    \section{Bibliografía}

    (All the info)

    \addtocontents{toc}{\endgroup}
  \end{appendices}


\end{document}
