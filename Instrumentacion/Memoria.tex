\documentclass[12pt, a4paper, titlepage]{article}

%INFORMACIÓN
\title{\textbf {Instrumentación Electrónica: Medición de circuitos de corriente continua y corriente alterna}}
\author{{\Large Pazos Pérez, José}\\ DNI}
\date{}

%PAQUETES
%\usepackage[utf8]{inputenc} %Añadir acentos y caracteres (Windows y Linux)

\usepackage[dvipsnames]{xcolor} %Colorear texto y colores estándar
\usepackage{colortbl} %Colorear celdas de tablas
%COLOR
\definecolor{Black}{RGB}{31, 31, 31}
\definecolor{Brown}{RGB}{179, 131, 55}
\definecolor{Red}{RGB}{222, 94, 80}
\definecolor{Orange}{RGB}{245, 149, 76}
\definecolor{Yellow}{RGB}{247, 192, 62}
\definecolor{Green}{RGB}{122, 217, 119}
\definecolor{Blue}{RGB}{110, 176, 230}
\definecolor{Violet}{RGB}{176, 110, 230}
\definecolor{Grey}{RGB}{189, 189, 189}
\definecolor{White}{RGB}{245, 245, 245}
\definecolor{Golden}{RGB}{207, 194, 97}
\definecolor{Silver}{RGB}{179, 193, 196}
\definecolor{LinkBlue}{RGB}{20, 88, 224}

\usepackage[centertags]{amsmath} %Excluir ecuaciones de la enumeración automática
\usepackage{csvsimple} %Tablas desde archivos .csv
\usepackage{pgfplots} %Gráficas desde matplotlib con .pgf
\usepackage{graphicx} %Imágenes
\usepackage[siunitx]{circuitikz} %Circuitos
\usepackage{pythonhighlight} %Código de python
\pgfplotsset{compat=1.16}

\usepackage{tocloft} %Crear listas (por ejemplo, de ecuaciones)
\usepackage{enumitem} %Cambiar los estilos de las listas
\usepackage{verbatim} %Comentarios de varias lineas
\usepackage[margin=1.0in]{geometry} %Márgenes
\usepackage[skip=12pt]{parskip} %Añadir espacio tras los párrafos
\usepackage{float} %Controlar el posicionamiento de gráficas y tablas con H
\usepackage[toc,page]{appendix} %Anexos
\usepackage{chngcntr} %Numeración de capítulos por partes
\usepackage{hyperref} %Añadir vínculos
\hypersetup{
    colorlinks=true,
    linkcolor=LinkBlue,
    filecolor=Red,
    urlcolor=Blue,
}

%CONFIGURACIÓN
\renewcommand{\contentsname}{Índice}
\renewcommand{\partname}{Experiencia}
\renewcommand{\listtablename}{Lista de Tablas}
\renewcommand{\listfigurename}{Lista de Figuras}
\renewcommand{\appendixpagename}{Anexos}
\renewcommand{\appendixtocname}{\large Anexos}
\renewcommand{\appendixname}{Anexo}
\renewcommand{\figurename}{Figura}
\renewcommand{\tablename}{Tabla}

\newcommand{\listecuacionesname}{\Large Lista de Ecuaciones}
\newlistof{ecuaciones}{equ}{\listecuacionesname}
\newcommand{\ecuaciones}[1]{\addcontentsline{equ}{ecuaciones}{\protect\numberline{\theequation}#1}\par}

\linespread{1.3}
\counterwithin*{section}{part}

\newcommand{\code}[1]{\texttt{#1}} %Formatear texto como código




%DOCUMENTO
\begin{document}
  \maketitle

  \tableofcontents

  \newpage
  \part*{Introducción}
  \addcontentsline{toc}{part}{Introducción}

  El objetivo de esta memoria es ofrecer un informe detallado de las prácticas de instrumentación, detallando las distintas experiencias llevadas a cabo con circuitos de \textbf{corriente continua} y \textbf{corriente alterna}.

  \section{Material}

  \subsection{Corriente Continua (Experiencia I)}

  \begin{itemize}[label=$-$]
    \item Polímetro
    \item Fuente de alimentación (CC)
    \item Resistencias de $180k\Omega$, $220k\Omega$, $390k\Omega$ y $1M\Omega$
    \item Tablero de conexiones
    \item Cables
  \end{itemize}

  \subsection{Corriente Alterna (Experiencia II)}
  \begin{itemize}[label=$-$]
    \item Osciloscopio
    \item Fuente de alimentación (CA, senosoidal)
    \item Resistencia de $10k\Omega$
    \item Condensador de $12nF$
    \item Tablero de conexiones
    \item Cables
  \end{itemize}


  \section{Tratamiento de datos}

  Para una correcta interpretación de los resultados expuestos en las próximas páginas, explicaremos las distintas metodologías y convenciones sobre el tratamiento de los datos de las mismas. En este apartado se incluyen la explicación de los métodos de \textbf{regresión lineal} y \textbf{propagación de incertidumbres}.

  \subsection{Reglas de redondeo}
  \label{sec:redondeo}

  En todas las mediciones expuestas utilizaremos los siguientes métodos de redondeo, aquellos convenidos en las jornadas de introducción:

  \begin{enumerate}
    \item Si la cifra en la posición n+1 es mayor que 5, la cifra n se incrementa en una unidad.
    \item Si la cifra en la posición n+1 es menor que 5, la cifra n se mantiene igual.
    \item Si la cifra en la posición n+1 es igual a 5, y alguna de las otras cifras suprimidas es distinta de 0, la cifra n se incrementa en una unidad.
    \item Si la cifra en la posición n+1 es igual a 5, y el resto de cifras suprimidas son iguales a 0, la cifra n se mantiene igual si es par y se incrementa en una unidad si es impar.
  \end{enumerate}

  \subsection{Incertidumbre de medidas directas}
  \label{sec:incert}

  En la práctica de corriente continua debemos indicar las incertidumbres de las mediciones realizadas. Como el polímetro utilizado es un aparato digital, consideraremos que una estimación de la incertidumbre sobre el valor real será la resolución ($\Delta x$) del mismo. Por lo tanto, tomaremos una unidad de la última cifra que muestre el aparato cómo $s_B (x)$.

  Como en estas medidas no existe otro tipo de incertidumbre a la que podamos aplicar un tratamiento estadístico, consideraremos $s_B (x)$ la incertidumbre final de la medida.

  \subsection{Múltiples medidas directas}

  Si realizamos una serie de medidas directas, la desviación típica de la media ($s_A (\bar{x})$) representa la incertidumbre de cualquier medida realizada con el mismo instrumento bajo las mismas condiciones. Con ambos datos podemos calcular la incertidumbre combinada:
  \begin{equation} \label{ec:scx}
    s_C (\bar{x}) = \sqrt{\left[ s_A (\bar{x}) \right]^2 + \left[ s_B (x) \right]^2}
  \end{equation}

  \subsection{Regresión lineal simple}
  \label{sec:reglin}

  Para ajustar los una serie de datos que parezcan seguir una relación lineal podemos utilizar el método de \textbf{ajuste por mínimos cuadrados}. Al ser \textit{lineal}, podemos ajustarla por una recta general de la forma $y = \alpha + \beta x$. El problema a resolver es conseguir la mejor aproximación \textit{a}, \textit{b} de los parámetros $\alpha, \beta$ y sus incertidumbres utilizando nuestra serie de parámetros $\left\{ x_i, y_i \right\}$.

  Para ello minimizaremos la suma de los productos del peso estadístico de cada punto, $w_i$, por el cuadrado de la desviación de los datos, $[y_i - (a + bx_i)]^2$. Por lo tanto, las derivadas parciales respecto a \textit{a} y \textit{b} deben de ser 0.
  \begin{equation} \label{ec:chi2}
    \chi^2 = \sum^{n}_{i=1} w_i[y_i - (a + bx_i)]^2
  \end{equation}
  \begin{equation} \label{ec:deriv}
    \frac{\partial \chi^2}{\partial a}=0, \frac{\partial \chi^2}{\partial b}=0
  \end{equation}

  De aquí resultan dos posibles casos:

  \begin{enumerate}
    \item Si las incertidumbres de $x_i$ no son despreciables respecto a las de $y_i$. Obtenemos la siguiente ecuación: $\omega_i=\frac{1}{s^2(y_i)+b^2s^2(x_i)}$, que excede el nivel de este curso.
    \item Si por el contrario podemos despreciar las incertidumbres de $x_i$ respecto a $y_i$, podemos simplificar la ecuación anterior a $\omega_i=[s(y_i)]^{-2}$.
  \end{enumerate}

  En nuestro caso, tanto en la estimación de una resistencia mediante la ley de Ohm como en el circuito en serie se pueden despreciar las incertidumbres. En ambos representaremos intensidad (I) frente a voltaje (V). La incertidumbre de la intensidad es del orden de $10^{-7}$, mientras que la del voltaje es del orden de $10^{-2}$, por lo que podremos aplicar el segundo método.

  Además, podemos considerar que las incertidumbres de $y_i$ permanecen constantes entre las diferentes medidas, por lo que también lo hará el peso estadístico $w = cte$. Podemos sustituír en la fórmula \ref{ec:chi2} y derivar respecto a \textit{a} y \textit{b} para obtener:
  \begin{gather}
    \chi^2 = w\sum^{n}_{i=1}[y_i - (a + bx_i)]^2 \nonumber \\
    a n + b\sum_{i}x_i = \sum_{i}y_i \quad \quad \sum_{i}x_i + b\sum_{i}x_i^2 = \sum_{i}x_i y_i \label{ec:dchi}
  \end{gather}

  Finalmente, conseguimos las expresiones matemáticas de \textit{a} y \textit{b} en base a la serie de medidas $\left\{ x_i, y_i \right\}$:
  \begin{gather}
    a = \frac{\left ( \sum_i y_i \right )\left ( \sum_i x_i^2 \right ) - \left ( \sum_i x_i \right )\left ( \sum_i x_i y_i \right )}{n\left ( \sum_i x_i^2 \right ) - \left ( \sum_i x_i \right )^2} \label{ec:a} \\
    b = \frac{n\left ( \sum_i x_i y_i \right ) - \left ( \sum_i x_i \right )\left ( \sum_i y_i \right )}{n\left ( \sum_i x_i^2 \right )-\left ( \sum_i x_i \right )^2} \label{ec:b}
  \end{gather}

  Además de los coeficientes para ajustar la recta, también podemos obtener otras magnitudes de interés sobre nuestra muestra:
  \begin{figure}[H]
    \begin{equation} \label{ec:s}
      s = \sqrt{\frac{\sum_i \left ( y_i - a - bx_i \right )^2}{1}}
    \end{equation}
    \caption{Desviación típica del ajuste para la muestra}
  \end{figure}
  \begin{figure}[H]
    \begin{equation} \label{ec:sab}
      s(a) = s\sqrt{\frac{\sum_i x_i^2}{n\left ( \sum_i x_i^2 \right ) - \left ( \sum_i x_i \right )^2}} \quad \quad s(b) = s\sqrt{\frac{n}{n\left ( \sum_i x_i^2 \right ) - \left ( \sum_i x_i \right )^2}}
    \end{equation}
    \caption{Incertidumbres de los parámetros \textit{a} y \textit{b}}
  \end{figure}
  \begin{figure}[H]
    \begin{equation} \label{ec:r}
      r = \frac{n\left ( \sum_i x_i y_i \right ) - \left ( \sum_i x_i \right )\left ( \sum_i y_i \right )}{\sqrt{\left [ n\left ( \sum_i x_i^2 \right ) - \left ( \sum_i x_i \right ) ^2\right ]\left [ n\left ( \sum_i y_i^2 \right ) - \left ( \sum_i y_i \right ) ^2 \right ]}}
    \end{equation}
    \caption{Coeficiente de regresión lineal}
  \end{figure}

  \subsection{Propagación de incertidumbres}
  \label{sec:propinc}

  Hay que hayar la incertidumbre correcta para aquellas medidas que sean indirectas (no se miden experimentalmente, si no que se aplican fórmulas a mediciones experimentales). Para ello empleamos el método de propagación de incertidumbres, que nos da la desviación estándar combinada:
  \begin{equation} \label{ec:propinc}
    s(y) = \sqrt{\sum_i\left ( \frac{\partial y}{\partial x_i} \right )^2 s^2(x_i)}
  \end{equation}

  Las magnitudes $x_i$ son aquellas medidas experimentalmente de las que depende la magnitud indirecta $y$. Todas las magnitudes $x_i$ han de ser independientes entre si.



  \newpage
  \part{Corriente Continua}

  \section{Objetivos}
  \begin{itemize}[label=$-$]
    \item Comprobar que el código de colores de las resistencias se corresponde con su valor real.
    \item Verificar el cumplimiento de la ley de Ohm ($V = I\cdot R$) en un circuito simple.
    \item Verificar las leyes de asociación de resistencias en circuitos en serie, paralelo y mixto.
    \item Perfeccionar el manejo del polímetro y demás utensilios del laboratorio.
  \end{itemize}


  \section{Resultados}


  \subsection{Medida de resistencias}
  El material de la práctica incluye cuatro resistencias. Para conocer sus valores utilizaremos dos métodos distintos, uno teórico y uno experimental, y luego contrastaremos los resultados.

  \subsubsection{Código de colores}
  Las resistencias tienen cuatro bandas de colores que indican su \textbf{valor nominal}. Estos colores se rigen por el siguiente código de resistencias estándar para cuatro bandas.

  \begin{figure}[H]
    \centering
    \begin{tabular}{|c|c|c|c|c|}
      \hline
      Color & 1º & 2º & Multiplicador & Tolerancia \\
      \hline
      \rowcolor{Black}
      \color{White} Negro & \color{White} 0 & \color{White} 0 & \color{White} 1$\Omega$ & \color{White} - \\
      \hline
      \rowcolor{Brown}
      Marrón & 1 & 1 & 10$\Omega$ & $\pm$1\% \\
      \hline
      \rowcolor{Red}
      Rojo & 2 & 2 & 100$\Omega$ & $\pm$2\% \\
      \hline
      \rowcolor{Orange}
      Naranja & 3 & 3 & 1k$\Omega$ & - \\
      \hline
      \rowcolor{Yellow}
      Amarillo & 4 & 4 & 10k$\Omega$ & - \\
      \hline
      \rowcolor{Green}
      Verde & 5 & 5 & 100k$\Omega$ & $\pm$0,5\% \\
      \hline
      \rowcolor{Blue}
      Azul & 6 & 6 & 1M$\Omega$ & $\pm$0,25\% \\
      \hline
      \rowcolor{Violet}
      Violeta & 7 & 7 & 10M$\Omega$ & $\pm$0,1\% \\
      \hline
      \rowcolor{Grey}
      Gris & 8 & 8 & 100M$\Omega$ & $\pm$0,05\% \\
      \hline
      \rowcolor{White}
      Blanco & 9 & 9 & 1G$\Omega$ & - \\
      \hline
      \rowcolor{Golden}
      Dorado & - & - & 0,1$\Omega$ & $\pm$5\% \\
      \hline
      \rowcolor{Silver}
      Plateado & - & - & 0,01$\Omega$ & $\pm$10\% \\
      \hline
    \end{tabular}
    \caption{Código de colores para resistencias}
  \end{figure}

  En base a esta tabla podemos calcular el valor teórico de la resistencia y su indeterminación (La tolerancia por el valor nominal). La primera banda indicará la primera cifra (A), la segunda banda será la segunda cifra (B), y la tercera banda la potencia de diez a la que está elevado ($C = 10^n$). Por lo tanto, el número resultante será de la forma $AB\cdot C$. La última banda representa la toleracia de acuerdo a la tabla anterior. Así tenemos los siguientes resultados, ordenados de menor a mayor.

  \begin{table}[H]
  \centering
  \csvreader[
    tabular=|c|c|c|c|c|,
    table head=\hline Resistencia & Color & V.~Nominal~($\Omega$) & Tolerancia~(\%) & s(R)~($\Omega$) \\ \hline,
    late after last line=\\\hline,
    separator=semicolon
    ]{CC51.csv}
    {r=\r, color=\rescolor, vn=\vn, t=\t, sr=\sr}
    {\r & \rescolor & \vn & \t & \sr}
  \caption{Medida del valor nominal de las resistencias}
  \end{table}

  \subsubsection{Medida directa}
  \label{sec:medidaresist}
  Ahora utilizaremos el polímetro para determinar el valor experimental de cada resistencia y comprobar si se corresponde al valor teórico. Configuramos el polímetro para la medición de resistencias, y obtenemos los siguientes resultados:

  \begin{table}[H]
  \centering
  \csvreader[
    tabular=|c|c|c|c|,
    table head=\hline Resistencia & Lectura~($\Omega$) & Resolución~($\Omega$) & $R \pm s(R)~(\Omega)$ \\ \hline,
    late after last line=\\\hline,
    separator=semicolon
    ]{CC52.csv}
    {r=\r, lectura=\lectura, resolucion=\resolucion, sr=\sr}
    {\r & \lectura & \resolucion & \lectura \hspace{4pt}$\pm$ \sr}
  \caption{Medida del valor experimental de las resistencias}
  \end{table}

  Podemos observar que los resultados experimentales entran dentro del umbral de confianza del 5\% de los obtenidos teóricamente, por lo que asumiremos que son correctos.


  \subsection{Ley de Ohm}

  \subsubsection{Explicación teórica}
  Si tomamos un circuito eléctrico de corriente continua, podemos derivar la siguiente relación entre la diferencia de potencial del circuito (\textit{V}), la intensidad que circula por él (\textit{I}) y la resitencia eléctrica del material (\textit{R}). A esta relación la llamamos \textbf{Ley de Ohm}.
  \begin{equation} \label{ec:ohm}
    V = I \cdot R
  \end{equation}
  Siguiendo el sistema internacional (SI), \textit{V} se expresa en Voltios (\textit{V}), \textit{I} en Amperios (\textit{A}) y \textit{R} en Ohmios ($\Omega$).

  \subsubsection{Estimación indirecta}
  Ahora comprobaremos experimentalmente el cumplimiento de la Ley de Ohm. Para ello construiremos un circuito simple, utilizando una fuente de corriente continua y una resistencia ($R_1$: 180k$\Omega$). Lo dispondremos de la siguiente manera:

  \begin{figure}[H]
    \centering
    \begin{circuitikz}[european]
      \draw (0,0) to[voltage source] (0,3) -- (3,3)
      to[R=$R_1$] (3,0) -- (0,0);
    \end{circuitikz}
    \caption{Circuito simple}
  \end{figure}

  Colocaremos el polímetro en dos posiciones: Para medir la intensidad, en serie; para medir el potencial, en paralelo alrededor de la resistencia. Podemos verlo en el diagrama mostrado a continuación.

  \begin{figure}[H]
    \centering
    \begin{circuitikz}[european]
      \draw (0,0) to[voltage source] (0,3) -- (2,3)
      to[R=$R_1$] (2,0) -- (0,0);
      \draw (2,3) -- (4,3)
      to[voltmeter] (4,0) -- (2,0);
    \end{circuitikz}
    \quad \quad
    \begin{circuitikz}[european]
      \draw (0,0) to[voltage source] (0,3)
      to[ammeter](3,3)
      to[R=$R_1$] (3,0) -- (0,0);
    \end{circuitikz}
    \caption{Medición de potenciales y intensidades}
  \end{figure}

  Ahora configuramos la fuente de corriente continua para un voltaje $V_1$, y medimos con el polímetro primero el voltaje (Porque la medida de la fuente no siempre es fiable) y luego la intensidad asociada a ese voltaje, $I_1$. Realizamos sucesivas mediciones aumentando progresivamente el voltaje, y obtendremos unos datos, por ejemplo:

  \begin{table}[H]
  \centering
  \csvreader[
    tabular=|c|c|c|,
    table head=\hline Medida & $V \pm s(V)~(V)$ & $I \pm s(I)~(I)$ \\ \hline,
    late after last line=\\\hline,
    filter test=\ifnumless{\thecsvrow}{2},
    separator=semicolon
    ]{CC53.csv}
    {v=\v, sv=\sv, i=\int, si=\si}
    {\thecsvrow & \v \hspace{4pt}$\pm$ \sv & \int \hspace{4pt}$\pm$ \si}
  \caption{Ejemplo de las mediciones de voltaje e intensidad}
  \end{table}

  Sim embargo, nuestro objetivo es comprobar si se cumple la Ley de Ohm, y para eso debemos de calcular el valor de las resistencias de manera indirecta. Para eso utilizaremos la ecuación \ref{ec:ohm} y despejaremos para \textit{R}.
  \begin{equation*}
    R = \frac{V}{I}
  \end{equation*}
  Ahora debemos de calcular la incertidumbre de \textit{R}, \textit{s(R)}. Para ello utilizamos el método de propagación de incertidumbres descrito en la sección \ref{sec:propinc}, y utilizaremos la ecuación \ref{ec:propinc}. Tomaremos \textit{V} como $x_1$, \textit{I} como $x_2$ y \textit{R} como \textit{y}.
  \begin{gather}
    s(R)) = \sqrt{\left(\frac{\partial R}{\partial V}\right)^2 s^2(V) + \left(\frac{\partial R}{\partial I}\right)^2 s^2(I)} \quad \quad
    \frac{\partial R}{\partial V} = \frac{1}{I} \quad \quad
    \frac{\partial R}{\partial I} = -\frac{V}{I^2} \nonumber \\
    s(R) = \sqrt{\left(\frac{1}{I}\right)^2 s^2(V) + \left(-\frac{V}{I^2}\right)^2 s^2(I)} \label{ec:propincohm}
  \end{gather}
  Aplicamos la fórmula anterior para calcular las incertidumbres de \textit{R}. Redondeamos \textit{s(R)} para que tenga dos cifras significativas (\textit{Por ejemplo: De 2641 a $2,6 \cdot 10^3$}) y ajustamos \textit{R} para que su última cifra significativa coincida con la posición decimal de la última cifra significativa de \textit{s(R)}. (\textit{Por ejemplo: con s(R) = $2,6 \cdot 10^3$, R se redondearía a las centenas ($10^2$)}). Utilizamos las técnicas de redondeo del apartado \ref{sec:redondeo}.

  \begin{table}[H]
  \centering
  \csvreader[
    tabular=|c|c|c|c|,
    table head=\hline Medida & $V \pm s(V)~(V)$ & $I \pm s(I)~(I)$ & $R \pm s(R)~(\Omega)$ \\ \hline,
    late after last line=\\\hline,
    separator=semicolon
    ]{CC53.csv}
    {v=\v, sv=\sv, i=\int, si=\si, r=\r, sr=\sr}
    {\thecsvrow & \v \hspace{4pt}$\pm$ \sv & \int \hspace{4pt}$\pm$ \si & \r \hspace{4pt}$\pm$ \sr}
  \caption{Potenciales e intensidades de un circuito simple con la resistencia $R_1$}
  \label{tb:ohm}
  \end{table}

  Podemos observar que el valor de \textit{R} es prácticamente constanete para cualquier combinación de voltajes e intensidades que escojamos. Además, se adecua mucho al valor experimental que obtuvimos en el apartado \ref{sec:medidaresist} (175500 $\pm$ 100 $\Omega$), por lo que podemos concluir que se cumple la \textbf{Ley de Ohm}.

  \subsubsection{Representación gráfica de V frente a I}

  Hagamos ahora una representación gráfica de los valores del potencial (\textit{V}) frente a los de la intensidad (\textit{I}). Para hacer la gráfica utilizaremos el paquete \code{matplotlib} y para cargar los datos desde un .csv en el que tabulamos los datos (que también usamos para generar las tablas en \LaTeX) añadiremos \code{pandas} (una extensión de \code{numpy}).

  \begin{python}
    import matplotlib.pyplot as plt
    import pandas as pd

    d = pd.read_csv(name + ".csv", sep=';', decimal=',')
    x = d["i"] ; y = d["v"]
    n = d.shape[0]

    def plot(x, y):
        plt.scatter(x, y)
        plt.xlabel('I($\mu$A)')
        plt.ylabel('V(V)', rotation=0, labelpad=20)
        plt.show()

    plot(x, y)
  \end{python}

  Vamos a crear una gráfica \textit{scatter} (de dispersión) para no unir los puntos de manera automática, ya que los valores medidos en el laboratorio son discretos. También cambiaremos la escala del eje x a $\mu$A para una leyenda más compacta. La exportamos a .pgf para poder cargarla en \LaTeX\  sin pérdida de calidad, y la representamos a continuación:

  \begin{figure}[H]
    %\centering
    \hspace{2.5em} %% Creator: Matplotlib, PGF backend
%%
%% To include the figure in your LaTeX document, write
%%   \input{<filename>.pgf}
%%
%% Make sure the required packages are loaded in your preamble
%%   \usepackage{pgf}
%%
%% Figures using additional raster images can only be included by \input if
%% they are in the same directory as the main LaTeX file. For loading figures
%% from other directories you can use the `import` package
%%   \usepackage{import}
%% and then include the figures with
%%   \import{<path to file>}{<filename>.pgf}
%%
%% Matplotlib used the following preamble
%%
\begingroup%
\makeatletter%
\begin{pgfpicture}%
\pgfpathrectangle{\pgfpointorigin}{\pgfqpoint{4.805892in}{3.310123in}}%
\pgfusepath{use as bounding box, clip}%
\begin{pgfscope}%
\pgfsetbuttcap%
\pgfsetmiterjoin%
\definecolor{currentfill}{rgb}{1.000000,1.000000,1.000000}%
\pgfsetfillcolor{currentfill}%
\pgfsetlinewidth{0.000000pt}%
\definecolor{currentstroke}{rgb}{1.000000,1.000000,1.000000}%
\pgfsetstrokecolor{currentstroke}%
\pgfsetdash{}{0pt}%
\pgfpathmoveto{\pgfqpoint{0.000000in}{0.000000in}}%
\pgfpathlineto{\pgfqpoint{4.805892in}{0.000000in}}%
\pgfpathlineto{\pgfqpoint{4.805892in}{3.310123in}}%
\pgfpathlineto{\pgfqpoint{0.000000in}{3.310123in}}%
\pgfpathclose%
\pgfusepath{fill}%
\end{pgfscope}%
\begin{pgfscope}%
\pgfsetbuttcap%
\pgfsetmiterjoin%
\definecolor{currentfill}{rgb}{1.000000,1.000000,1.000000}%
\pgfsetfillcolor{currentfill}%
\pgfsetlinewidth{0.000000pt}%
\definecolor{currentstroke}{rgb}{0.000000,0.000000,0.000000}%
\pgfsetstrokecolor{currentstroke}%
\pgfsetstrokeopacity{0.000000}%
\pgfsetdash{}{0pt}%
\pgfpathmoveto{\pgfqpoint{0.772069in}{0.515123in}}%
\pgfpathlineto{\pgfqpoint{4.647069in}{0.515123in}}%
\pgfpathlineto{\pgfqpoint{4.647069in}{3.210123in}}%
\pgfpathlineto{\pgfqpoint{0.772069in}{3.210123in}}%
\pgfpathclose%
\pgfusepath{fill}%
\end{pgfscope}%
\begin{pgfscope}%
\pgfsetbuttcap%
\pgfsetroundjoin%
\definecolor{currentfill}{rgb}{0.000000,0.000000,0.000000}%
\pgfsetfillcolor{currentfill}%
\pgfsetlinewidth{0.803000pt}%
\definecolor{currentstroke}{rgb}{0.000000,0.000000,0.000000}%
\pgfsetstrokecolor{currentstroke}%
\pgfsetdash{}{0pt}%
\pgfsys@defobject{currentmarker}{\pgfqpoint{0.000000in}{-0.048611in}}{\pgfqpoint{0.000000in}{0.000000in}}{%
\pgfpathmoveto{\pgfqpoint{0.000000in}{0.000000in}}%
\pgfpathlineto{\pgfqpoint{0.000000in}{-0.048611in}}%
\pgfusepath{stroke,fill}%
}%
\begin{pgfscope}%
\pgfsys@transformshift{1.177060in}{0.515123in}%
\pgfsys@useobject{currentmarker}{}%
\end{pgfscope}%
\end{pgfscope}%
\begin{pgfscope}%
\definecolor{textcolor}{rgb}{0.000000,0.000000,0.000000}%
\pgfsetstrokecolor{textcolor}%
\pgfsetfillcolor{textcolor}%
\pgftext[x=1.177060in,y=0.417901in,,top]{\color{textcolor}\rmfamily\fontsize{10.000000}{12.000000}\selectfont \(\displaystyle 10\)}%
\end{pgfscope}%
\begin{pgfscope}%
\pgfsetbuttcap%
\pgfsetroundjoin%
\definecolor{currentfill}{rgb}{0.000000,0.000000,0.000000}%
\pgfsetfillcolor{currentfill}%
\pgfsetlinewidth{0.803000pt}%
\definecolor{currentstroke}{rgb}{0.000000,0.000000,0.000000}%
\pgfsetstrokecolor{currentstroke}%
\pgfsetdash{}{0pt}%
\pgfsys@defobject{currentmarker}{\pgfqpoint{0.000000in}{-0.048611in}}{\pgfqpoint{0.000000in}{0.000000in}}{%
\pgfpathmoveto{\pgfqpoint{0.000000in}{0.000000in}}%
\pgfpathlineto{\pgfqpoint{0.000000in}{-0.048611in}}%
\pgfusepath{stroke,fill}%
}%
\begin{pgfscope}%
\pgfsys@transformshift{1.868938in}{0.515123in}%
\pgfsys@useobject{currentmarker}{}%
\end{pgfscope}%
\end{pgfscope}%
\begin{pgfscope}%
\definecolor{textcolor}{rgb}{0.000000,0.000000,0.000000}%
\pgfsetstrokecolor{textcolor}%
\pgfsetfillcolor{textcolor}%
\pgftext[x=1.868938in,y=0.417901in,,top]{\color{textcolor}\rmfamily\fontsize{10.000000}{12.000000}\selectfont \(\displaystyle 20\)}%
\end{pgfscope}%
\begin{pgfscope}%
\pgfsetbuttcap%
\pgfsetroundjoin%
\definecolor{currentfill}{rgb}{0.000000,0.000000,0.000000}%
\pgfsetfillcolor{currentfill}%
\pgfsetlinewidth{0.803000pt}%
\definecolor{currentstroke}{rgb}{0.000000,0.000000,0.000000}%
\pgfsetstrokecolor{currentstroke}%
\pgfsetdash{}{0pt}%
\pgfsys@defobject{currentmarker}{\pgfqpoint{0.000000in}{-0.048611in}}{\pgfqpoint{0.000000in}{0.000000in}}{%
\pgfpathmoveto{\pgfqpoint{0.000000in}{0.000000in}}%
\pgfpathlineto{\pgfqpoint{0.000000in}{-0.048611in}}%
\pgfusepath{stroke,fill}%
}%
\begin{pgfscope}%
\pgfsys@transformshift{2.560815in}{0.515123in}%
\pgfsys@useobject{currentmarker}{}%
\end{pgfscope}%
\end{pgfscope}%
\begin{pgfscope}%
\definecolor{textcolor}{rgb}{0.000000,0.000000,0.000000}%
\pgfsetstrokecolor{textcolor}%
\pgfsetfillcolor{textcolor}%
\pgftext[x=2.560815in,y=0.417901in,,top]{\color{textcolor}\rmfamily\fontsize{10.000000}{12.000000}\selectfont \(\displaystyle 30\)}%
\end{pgfscope}%
\begin{pgfscope}%
\pgfsetbuttcap%
\pgfsetroundjoin%
\definecolor{currentfill}{rgb}{0.000000,0.000000,0.000000}%
\pgfsetfillcolor{currentfill}%
\pgfsetlinewidth{0.803000pt}%
\definecolor{currentstroke}{rgb}{0.000000,0.000000,0.000000}%
\pgfsetstrokecolor{currentstroke}%
\pgfsetdash{}{0pt}%
\pgfsys@defobject{currentmarker}{\pgfqpoint{0.000000in}{-0.048611in}}{\pgfqpoint{0.000000in}{0.000000in}}{%
\pgfpathmoveto{\pgfqpoint{0.000000in}{0.000000in}}%
\pgfpathlineto{\pgfqpoint{0.000000in}{-0.048611in}}%
\pgfusepath{stroke,fill}%
}%
\begin{pgfscope}%
\pgfsys@transformshift{3.252692in}{0.515123in}%
\pgfsys@useobject{currentmarker}{}%
\end{pgfscope}%
\end{pgfscope}%
\begin{pgfscope}%
\definecolor{textcolor}{rgb}{0.000000,0.000000,0.000000}%
\pgfsetstrokecolor{textcolor}%
\pgfsetfillcolor{textcolor}%
\pgftext[x=3.252692in,y=0.417901in,,top]{\color{textcolor}\rmfamily\fontsize{10.000000}{12.000000}\selectfont \(\displaystyle 40\)}%
\end{pgfscope}%
\begin{pgfscope}%
\pgfsetbuttcap%
\pgfsetroundjoin%
\definecolor{currentfill}{rgb}{0.000000,0.000000,0.000000}%
\pgfsetfillcolor{currentfill}%
\pgfsetlinewidth{0.803000pt}%
\definecolor{currentstroke}{rgb}{0.000000,0.000000,0.000000}%
\pgfsetstrokecolor{currentstroke}%
\pgfsetdash{}{0pt}%
\pgfsys@defobject{currentmarker}{\pgfqpoint{0.000000in}{-0.048611in}}{\pgfqpoint{0.000000in}{0.000000in}}{%
\pgfpathmoveto{\pgfqpoint{0.000000in}{0.000000in}}%
\pgfpathlineto{\pgfqpoint{0.000000in}{-0.048611in}}%
\pgfusepath{stroke,fill}%
}%
\begin{pgfscope}%
\pgfsys@transformshift{3.944570in}{0.515123in}%
\pgfsys@useobject{currentmarker}{}%
\end{pgfscope}%
\end{pgfscope}%
\begin{pgfscope}%
\definecolor{textcolor}{rgb}{0.000000,0.000000,0.000000}%
\pgfsetstrokecolor{textcolor}%
\pgfsetfillcolor{textcolor}%
\pgftext[x=3.944570in,y=0.417901in,,top]{\color{textcolor}\rmfamily\fontsize{10.000000}{12.000000}\selectfont \(\displaystyle 50\)}%
\end{pgfscope}%
\begin{pgfscope}%
\pgfsetbuttcap%
\pgfsetroundjoin%
\definecolor{currentfill}{rgb}{0.000000,0.000000,0.000000}%
\pgfsetfillcolor{currentfill}%
\pgfsetlinewidth{0.803000pt}%
\definecolor{currentstroke}{rgb}{0.000000,0.000000,0.000000}%
\pgfsetstrokecolor{currentstroke}%
\pgfsetdash{}{0pt}%
\pgfsys@defobject{currentmarker}{\pgfqpoint{0.000000in}{-0.048611in}}{\pgfqpoint{0.000000in}{0.000000in}}{%
\pgfpathmoveto{\pgfqpoint{0.000000in}{0.000000in}}%
\pgfpathlineto{\pgfqpoint{0.000000in}{-0.048611in}}%
\pgfusepath{stroke,fill}%
}%
\begin{pgfscope}%
\pgfsys@transformshift{4.636447in}{0.515123in}%
\pgfsys@useobject{currentmarker}{}%
\end{pgfscope}%
\end{pgfscope}%
\begin{pgfscope}%
\definecolor{textcolor}{rgb}{0.000000,0.000000,0.000000}%
\pgfsetstrokecolor{textcolor}%
\pgfsetfillcolor{textcolor}%
\pgftext[x=4.636447in,y=0.417901in,,top]{\color{textcolor}\rmfamily\fontsize{10.000000}{12.000000}\selectfont \(\displaystyle 60\)}%
\end{pgfscope}%
\begin{pgfscope}%
\definecolor{textcolor}{rgb}{0.000000,0.000000,0.000000}%
\pgfsetstrokecolor{textcolor}%
\pgfsetfillcolor{textcolor}%
\pgftext[x=2.709569in,y=0.238889in,,top]{\color{textcolor}\rmfamily\fontsize{10.000000}{12.000000}\selectfont I(\(\displaystyle \mu\)A)}%
\end{pgfscope}%
\begin{pgfscope}%
\pgfsetbuttcap%
\pgfsetroundjoin%
\definecolor{currentfill}{rgb}{0.000000,0.000000,0.000000}%
\pgfsetfillcolor{currentfill}%
\pgfsetlinewidth{0.803000pt}%
\definecolor{currentstroke}{rgb}{0.000000,0.000000,0.000000}%
\pgfsetstrokecolor{currentstroke}%
\pgfsetdash{}{0pt}%
\pgfsys@defobject{currentmarker}{\pgfqpoint{-0.048611in}{0.000000in}}{\pgfqpoint{0.000000in}{0.000000in}}{%
\pgfpathmoveto{\pgfqpoint{0.000000in}{0.000000in}}%
\pgfpathlineto{\pgfqpoint{-0.048611in}{0.000000in}}%
\pgfusepath{stroke,fill}%
}%
\begin{pgfscope}%
\pgfsys@transformshift{0.772069in}{0.863332in}%
\pgfsys@useobject{currentmarker}{}%
\end{pgfscope}%
\end{pgfscope}%
\begin{pgfscope}%
\definecolor{textcolor}{rgb}{0.000000,0.000000,0.000000}%
\pgfsetstrokecolor{textcolor}%
\pgfsetfillcolor{textcolor}%
\pgftext[x=0.605402in,y=0.815107in,left,base]{\color{textcolor}\rmfamily\fontsize{10.000000}{12.000000}\selectfont \(\displaystyle 2\)}%
\end{pgfscope}%
\begin{pgfscope}%
\pgfsetbuttcap%
\pgfsetroundjoin%
\definecolor{currentfill}{rgb}{0.000000,0.000000,0.000000}%
\pgfsetfillcolor{currentfill}%
\pgfsetlinewidth{0.803000pt}%
\definecolor{currentstroke}{rgb}{0.000000,0.000000,0.000000}%
\pgfsetstrokecolor{currentstroke}%
\pgfsetdash{}{0pt}%
\pgfsys@defobject{currentmarker}{\pgfqpoint{-0.048611in}{0.000000in}}{\pgfqpoint{0.000000in}{0.000000in}}{%
\pgfpathmoveto{\pgfqpoint{0.000000in}{0.000000in}}%
\pgfpathlineto{\pgfqpoint{-0.048611in}{0.000000in}}%
\pgfusepath{stroke,fill}%
}%
\begin{pgfscope}%
\pgfsys@transformshift{0.772069in}{1.409095in}%
\pgfsys@useobject{currentmarker}{}%
\end{pgfscope}%
\end{pgfscope}%
\begin{pgfscope}%
\definecolor{textcolor}{rgb}{0.000000,0.000000,0.000000}%
\pgfsetstrokecolor{textcolor}%
\pgfsetfillcolor{textcolor}%
\pgftext[x=0.605402in,y=1.360869in,left,base]{\color{textcolor}\rmfamily\fontsize{10.000000}{12.000000}\selectfont \(\displaystyle 4\)}%
\end{pgfscope}%
\begin{pgfscope}%
\pgfsetbuttcap%
\pgfsetroundjoin%
\definecolor{currentfill}{rgb}{0.000000,0.000000,0.000000}%
\pgfsetfillcolor{currentfill}%
\pgfsetlinewidth{0.803000pt}%
\definecolor{currentstroke}{rgb}{0.000000,0.000000,0.000000}%
\pgfsetstrokecolor{currentstroke}%
\pgfsetdash{}{0pt}%
\pgfsys@defobject{currentmarker}{\pgfqpoint{-0.048611in}{0.000000in}}{\pgfqpoint{0.000000in}{0.000000in}}{%
\pgfpathmoveto{\pgfqpoint{0.000000in}{0.000000in}}%
\pgfpathlineto{\pgfqpoint{-0.048611in}{0.000000in}}%
\pgfusepath{stroke,fill}%
}%
\begin{pgfscope}%
\pgfsys@transformshift{0.772069in}{1.954857in}%
\pgfsys@useobject{currentmarker}{}%
\end{pgfscope}%
\end{pgfscope}%
\begin{pgfscope}%
\definecolor{textcolor}{rgb}{0.000000,0.000000,0.000000}%
\pgfsetstrokecolor{textcolor}%
\pgfsetfillcolor{textcolor}%
\pgftext[x=0.605402in,y=1.906632in,left,base]{\color{textcolor}\rmfamily\fontsize{10.000000}{12.000000}\selectfont \(\displaystyle 6\)}%
\end{pgfscope}%
\begin{pgfscope}%
\pgfsetbuttcap%
\pgfsetroundjoin%
\definecolor{currentfill}{rgb}{0.000000,0.000000,0.000000}%
\pgfsetfillcolor{currentfill}%
\pgfsetlinewidth{0.803000pt}%
\definecolor{currentstroke}{rgb}{0.000000,0.000000,0.000000}%
\pgfsetstrokecolor{currentstroke}%
\pgfsetdash{}{0pt}%
\pgfsys@defobject{currentmarker}{\pgfqpoint{-0.048611in}{0.000000in}}{\pgfqpoint{0.000000in}{0.000000in}}{%
\pgfpathmoveto{\pgfqpoint{0.000000in}{0.000000in}}%
\pgfpathlineto{\pgfqpoint{-0.048611in}{0.000000in}}%
\pgfusepath{stroke,fill}%
}%
\begin{pgfscope}%
\pgfsys@transformshift{0.772069in}{2.500620in}%
\pgfsys@useobject{currentmarker}{}%
\end{pgfscope}%
\end{pgfscope}%
\begin{pgfscope}%
\definecolor{textcolor}{rgb}{0.000000,0.000000,0.000000}%
\pgfsetstrokecolor{textcolor}%
\pgfsetfillcolor{textcolor}%
\pgftext[x=0.605402in,y=2.452394in,left,base]{\color{textcolor}\rmfamily\fontsize{10.000000}{12.000000}\selectfont \(\displaystyle 8\)}%
\end{pgfscope}%
\begin{pgfscope}%
\pgfsetbuttcap%
\pgfsetroundjoin%
\definecolor{currentfill}{rgb}{0.000000,0.000000,0.000000}%
\pgfsetfillcolor{currentfill}%
\pgfsetlinewidth{0.803000pt}%
\definecolor{currentstroke}{rgb}{0.000000,0.000000,0.000000}%
\pgfsetstrokecolor{currentstroke}%
\pgfsetdash{}{0pt}%
\pgfsys@defobject{currentmarker}{\pgfqpoint{-0.048611in}{0.000000in}}{\pgfqpoint{0.000000in}{0.000000in}}{%
\pgfpathmoveto{\pgfqpoint{0.000000in}{0.000000in}}%
\pgfpathlineto{\pgfqpoint{-0.048611in}{0.000000in}}%
\pgfusepath{stroke,fill}%
}%
\begin{pgfscope}%
\pgfsys@transformshift{0.772069in}{3.046382in}%
\pgfsys@useobject{currentmarker}{}%
\end{pgfscope}%
\end{pgfscope}%
\begin{pgfscope}%
\definecolor{textcolor}{rgb}{0.000000,0.000000,0.000000}%
\pgfsetstrokecolor{textcolor}%
\pgfsetfillcolor{textcolor}%
\pgftext[x=0.535957in,y=2.998157in,left,base]{\color{textcolor}\rmfamily\fontsize{10.000000}{12.000000}\selectfont \(\displaystyle 10\)}%
\end{pgfscope}%
\begin{pgfscope}%
\definecolor{textcolor}{rgb}{0.000000,0.000000,0.000000}%
\pgfsetstrokecolor{textcolor}%
\pgfsetfillcolor{textcolor}%
\pgftext[x=0.258179in,y=1.862623in,,bottom]{\color{textcolor}\rmfamily\fontsize{10.000000}{12.000000}\selectfont V(V)}%
\end{pgfscope}%
\begin{pgfscope}%
\pgfpathrectangle{\pgfqpoint{0.772069in}{0.515123in}}{\pgfqpoint{3.875000in}{2.695000in}}%
\pgfusepath{clip}%
\pgfsetbuttcap%
\pgfsetroundjoin%
\definecolor{currentfill}{rgb}{0.121569,0.466667,0.705882}%
\pgfsetfillcolor{currentfill}%
\pgfsetlinewidth{1.003750pt}%
\definecolor{currentstroke}{rgb}{0.121569,0.466667,0.705882}%
\pgfsetstrokecolor{currentstroke}%
\pgfsetdash{}{0pt}%
\pgfpathmoveto{\pgfqpoint{0.948741in}{0.598994in}}%
\pgfpathcurveto{\pgfqpoint{0.959791in}{0.598994in}}{\pgfqpoint{0.970390in}{0.603385in}}{\pgfqpoint{0.978203in}{0.611198in}}%
\pgfpathcurveto{\pgfqpoint{0.986017in}{0.619012in}}{\pgfqpoint{0.990407in}{0.629611in}}{\pgfqpoint{0.990407in}{0.640661in}}%
\pgfpathcurveto{\pgfqpoint{0.990407in}{0.651711in}}{\pgfqpoint{0.986017in}{0.662310in}}{\pgfqpoint{0.978203in}{0.670124in}}%
\pgfpathcurveto{\pgfqpoint{0.970390in}{0.677937in}}{\pgfqpoint{0.959791in}{0.682328in}}{\pgfqpoint{0.948741in}{0.682328in}}%
\pgfpathcurveto{\pgfqpoint{0.937691in}{0.682328in}}{\pgfqpoint{0.927092in}{0.677937in}}{\pgfqpoint{0.919278in}{0.670124in}}%
\pgfpathcurveto{\pgfqpoint{0.911464in}{0.662310in}}{\pgfqpoint{0.907074in}{0.651711in}}{\pgfqpoint{0.907074in}{0.640661in}}%
\pgfpathcurveto{\pgfqpoint{0.907074in}{0.629611in}}{\pgfqpoint{0.911464in}{0.619012in}}{\pgfqpoint{0.919278in}{0.611198in}}%
\pgfpathcurveto{\pgfqpoint{0.927092in}{0.603385in}}{\pgfqpoint{0.937691in}{0.598994in}}{\pgfqpoint{0.948741in}{0.598994in}}%
\pgfpathclose%
\pgfusepath{stroke,fill}%
\end{pgfscope}%
\begin{pgfscope}%
\pgfpathrectangle{\pgfqpoint{0.772069in}{0.515123in}}{\pgfqpoint{3.875000in}{2.695000in}}%
\pgfusepath{clip}%
\pgfsetbuttcap%
\pgfsetroundjoin%
\definecolor{currentfill}{rgb}{0.121569,0.466667,0.705882}%
\pgfsetfillcolor{currentfill}%
\pgfsetlinewidth{1.003750pt}%
\definecolor{currentstroke}{rgb}{0.121569,0.466667,0.705882}%
\pgfsetstrokecolor{currentstroke}%
\pgfsetdash{}{0pt}%
\pgfpathmoveto{\pgfqpoint{1.315436in}{0.851682in}}%
\pgfpathcurveto{\pgfqpoint{1.326486in}{0.851682in}}{\pgfqpoint{1.337085in}{0.856073in}}{\pgfqpoint{1.344898in}{0.863886in}}%
\pgfpathcurveto{\pgfqpoint{1.352712in}{0.871700in}}{\pgfqpoint{1.357102in}{0.882299in}}{\pgfqpoint{1.357102in}{0.893349in}}%
\pgfpathcurveto{\pgfqpoint{1.357102in}{0.904399in}}{\pgfqpoint{1.352712in}{0.914998in}}{\pgfqpoint{1.344898in}{0.922812in}}%
\pgfpathcurveto{\pgfqpoint{1.337085in}{0.930625in}}{\pgfqpoint{1.326486in}{0.935016in}}{\pgfqpoint{1.315436in}{0.935016in}}%
\pgfpathcurveto{\pgfqpoint{1.304386in}{0.935016in}}{\pgfqpoint{1.293787in}{0.930625in}}{\pgfqpoint{1.285973in}{0.922812in}}%
\pgfpathcurveto{\pgfqpoint{1.278159in}{0.914998in}}{\pgfqpoint{1.273769in}{0.904399in}}{\pgfqpoint{1.273769in}{0.893349in}}%
\pgfpathcurveto{\pgfqpoint{1.273769in}{0.882299in}}{\pgfqpoint{1.278159in}{0.871700in}}{\pgfqpoint{1.285973in}{0.863886in}}%
\pgfpathcurveto{\pgfqpoint{1.293787in}{0.856073in}}{\pgfqpoint{1.304386in}{0.851682in}}{\pgfqpoint{1.315436in}{0.851682in}}%
\pgfpathclose%
\pgfusepath{stroke,fill}%
\end{pgfscope}%
\begin{pgfscope}%
\pgfpathrectangle{\pgfqpoint{0.772069in}{0.515123in}}{\pgfqpoint{3.875000in}{2.695000in}}%
\pgfusepath{clip}%
\pgfsetbuttcap%
\pgfsetroundjoin%
\definecolor{currentfill}{rgb}{0.121569,0.466667,0.705882}%
\pgfsetfillcolor{currentfill}%
\pgfsetlinewidth{1.003750pt}%
\definecolor{currentstroke}{rgb}{0.121569,0.466667,0.705882}%
\pgfsetstrokecolor{currentstroke}%
\pgfsetdash{}{0pt}%
\pgfpathmoveto{\pgfqpoint{1.675212in}{1.105462in}}%
\pgfpathcurveto{\pgfqpoint{1.686262in}{1.105462in}}{\pgfqpoint{1.696861in}{1.109852in}}{\pgfqpoint{1.704675in}{1.117666in}}%
\pgfpathcurveto{\pgfqpoint{1.712488in}{1.125479in}}{\pgfqpoint{1.716879in}{1.136078in}}{\pgfqpoint{1.716879in}{1.147129in}}%
\pgfpathcurveto{\pgfqpoint{1.716879in}{1.158179in}}{\pgfqpoint{1.712488in}{1.168778in}}{\pgfqpoint{1.704675in}{1.176591in}}%
\pgfpathcurveto{\pgfqpoint{1.696861in}{1.184405in}}{\pgfqpoint{1.686262in}{1.188795in}}{\pgfqpoint{1.675212in}{1.188795in}}%
\pgfpathcurveto{\pgfqpoint{1.664162in}{1.188795in}}{\pgfqpoint{1.653563in}{1.184405in}}{\pgfqpoint{1.645749in}{1.176591in}}%
\pgfpathcurveto{\pgfqpoint{1.637936in}{1.168778in}}{\pgfqpoint{1.633545in}{1.158179in}}{\pgfqpoint{1.633545in}{1.147129in}}%
\pgfpathcurveto{\pgfqpoint{1.633545in}{1.136078in}}{\pgfqpoint{1.637936in}{1.125479in}}{\pgfqpoint{1.645749in}{1.117666in}}%
\pgfpathcurveto{\pgfqpoint{1.653563in}{1.109852in}}{\pgfqpoint{1.664162in}{1.105462in}}{\pgfqpoint{1.675212in}{1.105462in}}%
\pgfpathclose%
\pgfusepath{stroke,fill}%
\end{pgfscope}%
\begin{pgfscope}%
\pgfpathrectangle{\pgfqpoint{0.772069in}{0.515123in}}{\pgfqpoint{3.875000in}{2.695000in}}%
\pgfusepath{clip}%
\pgfsetbuttcap%
\pgfsetroundjoin%
\definecolor{currentfill}{rgb}{0.121569,0.466667,0.705882}%
\pgfsetfillcolor{currentfill}%
\pgfsetlinewidth{1.003750pt}%
\definecolor{currentstroke}{rgb}{0.121569,0.466667,0.705882}%
\pgfsetstrokecolor{currentstroke}%
\pgfsetdash{}{0pt}%
\pgfpathmoveto{\pgfqpoint{2.076501in}{1.383801in}}%
\pgfpathcurveto{\pgfqpoint{2.087551in}{1.383801in}}{\pgfqpoint{2.098150in}{1.388191in}}{\pgfqpoint{2.105964in}{1.396005in}}%
\pgfpathcurveto{\pgfqpoint{2.113777in}{1.403818in}}{\pgfqpoint{2.118168in}{1.414417in}}{\pgfqpoint{2.118168in}{1.425468in}}%
\pgfpathcurveto{\pgfqpoint{2.118168in}{1.436518in}}{\pgfqpoint{2.113777in}{1.447117in}}{\pgfqpoint{2.105964in}{1.454930in}}%
\pgfpathcurveto{\pgfqpoint{2.098150in}{1.462744in}}{\pgfqpoint{2.087551in}{1.467134in}}{\pgfqpoint{2.076501in}{1.467134in}}%
\pgfpathcurveto{\pgfqpoint{2.065451in}{1.467134in}}{\pgfqpoint{2.054852in}{1.462744in}}{\pgfqpoint{2.047038in}{1.454930in}}%
\pgfpathcurveto{\pgfqpoint{2.039224in}{1.447117in}}{\pgfqpoint{2.034834in}{1.436518in}}{\pgfqpoint{2.034834in}{1.425468in}}%
\pgfpathcurveto{\pgfqpoint{2.034834in}{1.414417in}}{\pgfqpoint{2.039224in}{1.403818in}}{\pgfqpoint{2.047038in}{1.396005in}}%
\pgfpathcurveto{\pgfqpoint{2.054852in}{1.388191in}}{\pgfqpoint{2.065451in}{1.383801in}}{\pgfqpoint{2.076501in}{1.383801in}}%
\pgfpathclose%
\pgfusepath{stroke,fill}%
\end{pgfscope}%
\begin{pgfscope}%
\pgfpathrectangle{\pgfqpoint{0.772069in}{0.515123in}}{\pgfqpoint{3.875000in}{2.695000in}}%
\pgfusepath{clip}%
\pgfsetbuttcap%
\pgfsetroundjoin%
\definecolor{currentfill}{rgb}{0.121569,0.466667,0.705882}%
\pgfsetfillcolor{currentfill}%
\pgfsetlinewidth{1.003750pt}%
\definecolor{currentstroke}{rgb}{0.121569,0.466667,0.705882}%
\pgfsetstrokecolor{currentstroke}%
\pgfsetdash{}{0pt}%
\pgfpathmoveto{\pgfqpoint{2.463952in}{1.651225in}}%
\pgfpathcurveto{\pgfqpoint{2.475002in}{1.651225in}}{\pgfqpoint{2.485601in}{1.655615in}}{\pgfqpoint{2.493415in}{1.663428in}}%
\pgfpathcurveto{\pgfqpoint{2.501229in}{1.671242in}}{\pgfqpoint{2.505619in}{1.681841in}}{\pgfqpoint{2.505619in}{1.692891in}}%
\pgfpathcurveto{\pgfqpoint{2.505619in}{1.703941in}}{\pgfqpoint{2.501229in}{1.714540in}}{\pgfqpoint{2.493415in}{1.722354in}}%
\pgfpathcurveto{\pgfqpoint{2.485601in}{1.730168in}}{\pgfqpoint{2.475002in}{1.734558in}}{\pgfqpoint{2.463952in}{1.734558in}}%
\pgfpathcurveto{\pgfqpoint{2.452902in}{1.734558in}}{\pgfqpoint{2.442303in}{1.730168in}}{\pgfqpoint{2.434489in}{1.722354in}}%
\pgfpathcurveto{\pgfqpoint{2.426676in}{1.714540in}}{\pgfqpoint{2.422286in}{1.703941in}}{\pgfqpoint{2.422286in}{1.692891in}}%
\pgfpathcurveto{\pgfqpoint{2.422286in}{1.681841in}}{\pgfqpoint{2.426676in}{1.671242in}}{\pgfqpoint{2.434489in}{1.663428in}}%
\pgfpathcurveto{\pgfqpoint{2.442303in}{1.655615in}}{\pgfqpoint{2.452902in}{1.651225in}}{\pgfqpoint{2.463952in}{1.651225in}}%
\pgfpathclose%
\pgfusepath{stroke,fill}%
\end{pgfscope}%
\begin{pgfscope}%
\pgfpathrectangle{\pgfqpoint{0.772069in}{0.515123in}}{\pgfqpoint{3.875000in}{2.695000in}}%
\pgfusepath{clip}%
\pgfsetbuttcap%
\pgfsetroundjoin%
\definecolor{currentfill}{rgb}{0.121569,0.466667,0.705882}%
\pgfsetfillcolor{currentfill}%
\pgfsetlinewidth{1.003750pt}%
\definecolor{currentstroke}{rgb}{0.121569,0.466667,0.705882}%
\pgfsetstrokecolor{currentstroke}%
\pgfsetdash{}{0pt}%
\pgfpathmoveto{\pgfqpoint{2.872160in}{1.935021in}}%
\pgfpathcurveto{\pgfqpoint{2.883210in}{1.935021in}}{\pgfqpoint{2.893809in}{1.939411in}}{\pgfqpoint{2.901623in}{1.947225in}}%
\pgfpathcurveto{\pgfqpoint{2.909436in}{1.955039in}}{\pgfqpoint{2.913827in}{1.965638in}}{\pgfqpoint{2.913827in}{1.976688in}}%
\pgfpathcurveto{\pgfqpoint{2.913827in}{1.987738in}}{\pgfqpoint{2.909436in}{1.998337in}}{\pgfqpoint{2.901623in}{2.006150in}}%
\pgfpathcurveto{\pgfqpoint{2.893809in}{2.013964in}}{\pgfqpoint{2.883210in}{2.018354in}}{\pgfqpoint{2.872160in}{2.018354in}}%
\pgfpathcurveto{\pgfqpoint{2.861110in}{2.018354in}}{\pgfqpoint{2.850511in}{2.013964in}}{\pgfqpoint{2.842697in}{2.006150in}}%
\pgfpathcurveto{\pgfqpoint{2.834883in}{1.998337in}}{\pgfqpoint{2.830493in}{1.987738in}}{\pgfqpoint{2.830493in}{1.976688in}}%
\pgfpathcurveto{\pgfqpoint{2.830493in}{1.965638in}}{\pgfqpoint{2.834883in}{1.955039in}}{\pgfqpoint{2.842697in}{1.947225in}}%
\pgfpathcurveto{\pgfqpoint{2.850511in}{1.939411in}}{\pgfqpoint{2.861110in}{1.935021in}}{\pgfqpoint{2.872160in}{1.935021in}}%
\pgfpathclose%
\pgfusepath{stroke,fill}%
\end{pgfscope}%
\begin{pgfscope}%
\pgfpathrectangle{\pgfqpoint{0.772069in}{0.515123in}}{\pgfqpoint{3.875000in}{2.695000in}}%
\pgfusepath{clip}%
\pgfsetbuttcap%
\pgfsetroundjoin%
\definecolor{currentfill}{rgb}{0.121569,0.466667,0.705882}%
\pgfsetfillcolor{currentfill}%
\pgfsetlinewidth{1.003750pt}%
\definecolor{currentstroke}{rgb}{0.121569,0.466667,0.705882}%
\pgfsetstrokecolor{currentstroke}%
\pgfsetdash{}{0pt}%
\pgfpathmoveto{\pgfqpoint{3.273449in}{2.213360in}}%
\pgfpathcurveto{\pgfqpoint{3.284499in}{2.213360in}}{\pgfqpoint{3.295098in}{2.217750in}}{\pgfqpoint{3.302912in}{2.225564in}}%
\pgfpathcurveto{\pgfqpoint{3.310725in}{2.233377in}}{\pgfqpoint{3.315115in}{2.243976in}}{\pgfqpoint{3.315115in}{2.255027in}}%
\pgfpathcurveto{\pgfqpoint{3.315115in}{2.266077in}}{\pgfqpoint{3.310725in}{2.276676in}}{\pgfqpoint{3.302912in}{2.284489in}}%
\pgfpathcurveto{\pgfqpoint{3.295098in}{2.292303in}}{\pgfqpoint{3.284499in}{2.296693in}}{\pgfqpoint{3.273449in}{2.296693in}}%
\pgfpathcurveto{\pgfqpoint{3.262399in}{2.296693in}}{\pgfqpoint{3.251800in}{2.292303in}}{\pgfqpoint{3.243986in}{2.284489in}}%
\pgfpathcurveto{\pgfqpoint{3.236172in}{2.276676in}}{\pgfqpoint{3.231782in}{2.266077in}}{\pgfqpoint{3.231782in}{2.255027in}}%
\pgfpathcurveto{\pgfqpoint{3.231782in}{2.243976in}}{\pgfqpoint{3.236172in}{2.233377in}}{\pgfqpoint{3.243986in}{2.225564in}}%
\pgfpathcurveto{\pgfqpoint{3.251800in}{2.217750in}}{\pgfqpoint{3.262399in}{2.213360in}}{\pgfqpoint{3.273449in}{2.213360in}}%
\pgfpathclose%
\pgfusepath{stroke,fill}%
\end{pgfscope}%
\begin{pgfscope}%
\pgfpathrectangle{\pgfqpoint{0.772069in}{0.515123in}}{\pgfqpoint{3.875000in}{2.695000in}}%
\pgfusepath{clip}%
\pgfsetbuttcap%
\pgfsetroundjoin%
\definecolor{currentfill}{rgb}{0.121569,0.466667,0.705882}%
\pgfsetfillcolor{currentfill}%
\pgfsetlinewidth{1.003750pt}%
\definecolor{currentstroke}{rgb}{0.121569,0.466667,0.705882}%
\pgfsetstrokecolor{currentstroke}%
\pgfsetdash{}{0pt}%
\pgfpathmoveto{\pgfqpoint{3.667819in}{2.486241in}}%
\pgfpathcurveto{\pgfqpoint{3.678869in}{2.486241in}}{\pgfqpoint{3.689468in}{2.490631in}}{\pgfqpoint{3.697282in}{2.498445in}}%
\pgfpathcurveto{\pgfqpoint{3.705095in}{2.506259in}}{\pgfqpoint{3.709486in}{2.516858in}}{\pgfqpoint{3.709486in}{2.527908in}}%
\pgfpathcurveto{\pgfqpoint{3.709486in}{2.538958in}}{\pgfqpoint{3.705095in}{2.549557in}}{\pgfqpoint{3.697282in}{2.557371in}}%
\pgfpathcurveto{\pgfqpoint{3.689468in}{2.565184in}}{\pgfqpoint{3.678869in}{2.569575in}}{\pgfqpoint{3.667819in}{2.569575in}}%
\pgfpathcurveto{\pgfqpoint{3.656769in}{2.569575in}}{\pgfqpoint{3.646170in}{2.565184in}}{\pgfqpoint{3.638356in}{2.557371in}}%
\pgfpathcurveto{\pgfqpoint{3.630542in}{2.549557in}}{\pgfqpoint{3.626152in}{2.538958in}}{\pgfqpoint{3.626152in}{2.527908in}}%
\pgfpathcurveto{\pgfqpoint{3.626152in}{2.516858in}}{\pgfqpoint{3.630542in}{2.506259in}}{\pgfqpoint{3.638356in}{2.498445in}}%
\pgfpathcurveto{\pgfqpoint{3.646170in}{2.490631in}}{\pgfqpoint{3.656769in}{2.486241in}}{\pgfqpoint{3.667819in}{2.486241in}}%
\pgfpathclose%
\pgfusepath{stroke,fill}%
\end{pgfscope}%
\begin{pgfscope}%
\pgfpathrectangle{\pgfqpoint{0.772069in}{0.515123in}}{\pgfqpoint{3.875000in}{2.695000in}}%
\pgfusepath{clip}%
\pgfsetbuttcap%
\pgfsetroundjoin%
\definecolor{currentfill}{rgb}{0.121569,0.466667,0.705882}%
\pgfsetfillcolor{currentfill}%
\pgfsetlinewidth{1.003750pt}%
\definecolor{currentstroke}{rgb}{0.121569,0.466667,0.705882}%
\pgfsetstrokecolor{currentstroke}%
\pgfsetdash{}{0pt}%
\pgfpathmoveto{\pgfqpoint{4.089864in}{2.775495in}}%
\pgfpathcurveto{\pgfqpoint{4.100914in}{2.775495in}}{\pgfqpoint{4.111513in}{2.779886in}}{\pgfqpoint{4.119327in}{2.787699in}}%
\pgfpathcurveto{\pgfqpoint{4.127141in}{2.795513in}}{\pgfqpoint{4.131531in}{2.806112in}}{\pgfqpoint{4.131531in}{2.817162in}}%
\pgfpathcurveto{\pgfqpoint{4.131531in}{2.828212in}}{\pgfqpoint{4.127141in}{2.838811in}}{\pgfqpoint{4.119327in}{2.846625in}}%
\pgfpathcurveto{\pgfqpoint{4.111513in}{2.854438in}}{\pgfqpoint{4.100914in}{2.858829in}}{\pgfqpoint{4.089864in}{2.858829in}}%
\pgfpathcurveto{\pgfqpoint{4.078814in}{2.858829in}}{\pgfqpoint{4.068215in}{2.854438in}}{\pgfqpoint{4.060401in}{2.846625in}}%
\pgfpathcurveto{\pgfqpoint{4.052588in}{2.838811in}}{\pgfqpoint{4.048197in}{2.828212in}}{\pgfqpoint{4.048197in}{2.817162in}}%
\pgfpathcurveto{\pgfqpoint{4.048197in}{2.806112in}}{\pgfqpoint{4.052588in}{2.795513in}}{\pgfqpoint{4.060401in}{2.787699in}}%
\pgfpathcurveto{\pgfqpoint{4.068215in}{2.779886in}}{\pgfqpoint{4.078814in}{2.775495in}}{\pgfqpoint{4.089864in}{2.775495in}}%
\pgfpathclose%
\pgfusepath{stroke,fill}%
\end{pgfscope}%
\begin{pgfscope}%
\pgfpathrectangle{\pgfqpoint{0.772069in}{0.515123in}}{\pgfqpoint{3.875000in}{2.695000in}}%
\pgfusepath{clip}%
\pgfsetbuttcap%
\pgfsetroundjoin%
\definecolor{currentfill}{rgb}{0.121569,0.466667,0.705882}%
\pgfsetfillcolor{currentfill}%
\pgfsetlinewidth{1.003750pt}%
\definecolor{currentstroke}{rgb}{0.121569,0.466667,0.705882}%
\pgfsetstrokecolor{currentstroke}%
\pgfsetdash{}{0pt}%
\pgfpathmoveto{\pgfqpoint{4.470397in}{3.042919in}}%
\pgfpathcurveto{\pgfqpoint{4.481447in}{3.042919in}}{\pgfqpoint{4.492046in}{3.047309in}}{\pgfqpoint{4.499859in}{3.055123in}}%
\pgfpathcurveto{\pgfqpoint{4.507673in}{3.062937in}}{\pgfqpoint{4.512063in}{3.073536in}}{\pgfqpoint{4.512063in}{3.084586in}}%
\pgfpathcurveto{\pgfqpoint{4.512063in}{3.095636in}}{\pgfqpoint{4.507673in}{3.106235in}}{\pgfqpoint{4.499859in}{3.114048in}}%
\pgfpathcurveto{\pgfqpoint{4.492046in}{3.121862in}}{\pgfqpoint{4.481447in}{3.126252in}}{\pgfqpoint{4.470397in}{3.126252in}}%
\pgfpathcurveto{\pgfqpoint{4.459347in}{3.126252in}}{\pgfqpoint{4.448748in}{3.121862in}}{\pgfqpoint{4.440934in}{3.114048in}}%
\pgfpathcurveto{\pgfqpoint{4.433120in}{3.106235in}}{\pgfqpoint{4.428730in}{3.095636in}}{\pgfqpoint{4.428730in}{3.084586in}}%
\pgfpathcurveto{\pgfqpoint{4.428730in}{3.073536in}}{\pgfqpoint{4.433120in}{3.062937in}}{\pgfqpoint{4.440934in}{3.055123in}}%
\pgfpathcurveto{\pgfqpoint{4.448748in}{3.047309in}}{\pgfqpoint{4.459347in}{3.042919in}}{\pgfqpoint{4.470397in}{3.042919in}}%
\pgfpathclose%
\pgfusepath{stroke,fill}%
\end{pgfscope}%
\begin{pgfscope}%
\pgfsetrectcap%
\pgfsetmiterjoin%
\pgfsetlinewidth{0.803000pt}%
\definecolor{currentstroke}{rgb}{0.000000,0.000000,0.000000}%
\pgfsetstrokecolor{currentstroke}%
\pgfsetdash{}{0pt}%
\pgfpathmoveto{\pgfqpoint{0.772069in}{0.515123in}}%
\pgfpathlineto{\pgfqpoint{0.772069in}{3.210123in}}%
\pgfusepath{stroke}%
\end{pgfscope}%
\begin{pgfscope}%
\pgfsetrectcap%
\pgfsetmiterjoin%
\pgfsetlinewidth{0.803000pt}%
\definecolor{currentstroke}{rgb}{0.000000,0.000000,0.000000}%
\pgfsetstrokecolor{currentstroke}%
\pgfsetdash{}{0pt}%
\pgfpathmoveto{\pgfqpoint{4.647069in}{0.515123in}}%
\pgfpathlineto{\pgfqpoint{4.647069in}{3.210123in}}%
\pgfusepath{stroke}%
\end{pgfscope}%
\begin{pgfscope}%
\pgfsetrectcap%
\pgfsetmiterjoin%
\pgfsetlinewidth{0.803000pt}%
\definecolor{currentstroke}{rgb}{0.000000,0.000000,0.000000}%
\pgfsetstrokecolor{currentstroke}%
\pgfsetdash{}{0pt}%
\pgfpathmoveto{\pgfqpoint{0.772069in}{0.515123in}}%
\pgfpathlineto{\pgfqpoint{4.647069in}{0.515123in}}%
\pgfusepath{stroke}%
\end{pgfscope}%
\begin{pgfscope}%
\pgfsetrectcap%
\pgfsetmiterjoin%
\pgfsetlinewidth{0.803000pt}%
\definecolor{currentstroke}{rgb}{0.000000,0.000000,0.000000}%
\pgfsetstrokecolor{currentstroke}%
\pgfsetdash{}{0pt}%
\pgfpathmoveto{\pgfqpoint{0.772069in}{3.210123in}}%
\pgfpathlineto{\pgfqpoint{4.647069in}{3.210123in}}%
\pgfusepath{stroke}%
\end{pgfscope}%
\end{pgfpicture}%
\makeatother%
\endgroup%

    \caption{Voltaje (V) frente a intensidad (I)}
  \end{figure}

  \subsubsection{Ajuste por mínimos cuadrados}
  \label{sec:ajusteminres}

  Como podemos observar, parece que existe una relación lineal entre las dos magnitudes. Procedemos entonces a hacer un ajuste de regresión lineal por mínimos cuadrados. Como explicamos en la sección \ref{sec:reglin}, tenemos que calcular los coeficientes \textit{a} y \textit{b} de la recta $a + bx$. Para ello utilizamos las ecuaciones \ref{ec:a} y \ref{ec:b}. Como vemos, hay que calcular los siguientes términos:
  \begin{gather}
    \sum_i x_i \nonumber \qquad \sum_i y_i \nonumber \qquad \sum_i x_iy_i \nonumber \qquad \sum_i x_i^2 \nonumber
  \end{gather}

  Para ello, utilizaremos la tabla en .csv en la que tenemos los datos, y la cargaremos en python. Ahora definiremos una función \code{reg\_lin()} que calcule estas sumas a partir de las columnas de nuestra tabla y las almacene en variables separadas. Posteriormente, aplicaremos las fórmulas \ref{ec:a} y \ref{ec:b} y calcularemos así los términos \textit{a} y \textit{b}.

  \begin{python}
    def reg_lin(x, y, n):
        sx = x.sum()
        sy = y.sum()
        sxy = (x*y).sum()
        sx2 = (x**2).sum()

        a = (sy*sx2 - sx*sxy) / (n*sx2 - sx**2)
        b = (n*sxy - sx*sy) / (n*sx2 - sx**2)

        return a, b

    a, b = reg_lin(x, y, n)
  \end{python}

  El comando \code{.sum()} de la tabla de cargamos en \code{pandas} simplemente suma todos los términos de la columna que especificamos. En este caso, tomamos la columna x como la intensidad (\textit{I}) y la columna y como el potencial (\textit{V}) de la tabla \ref{tb:ohm}, que cargamos directamente en \code{python}. También podemos realizar operaciones con las columnas antes de sumar sus términos, como multiplicar una por otra o elevar sus términos al cuadrado. Así podemos crear los sumatorios necesarios para resolver las fórmulas de a y b. Obtenemos los siguientes resultados:
  \begin{gather}
    \sum_i x_i = 3,18\cdot10^{-4} \nonumber \qquad \sum_i y_i = 56,0 \nonumber \qquad \sum_i x_iy_i = 2,25\cdot10^{-3} \nonumber \qquad \sum_i x_i^2 = 1,28\cdot10^{-8} \nonumber \\
    a = 9,19\cdot10^{-3} \nonumber \qquad b = 175000 \nonumber
  \end{gather}
  Podemos utilizar ahora estos valores para crear la gráfica con la recta que mejor se ajuste a los datos, añadiendo el siguiente código a nuestra función \code{plot()}:

  \begin{python}
    def plot(x, y, a, b):
      #Codigo anterior
      xr = np.linspace(5, 60, 10)
      yr = (a + b*xr) / (10**6)
      plt.plot(xr, yr, color="skyblue", zorder=1)
  \end{python}

  Y, ahora sí, podemos ver nuestra gráfica completa.

  \begin{figure}[H]
    %\centering
    \hspace{2.5em} %% Creator: Matplotlib, PGF backend
%%
%% To include the figure in your LaTeX document, write
%%   \input{<filename>.pgf}
%%
%% Make sure the required packages are loaded in your preamble
%%   \usepackage{pgf}
%%
%% Figures using additional raster images can only be included by \input if
%% they are in the same directory as the main LaTeX file. For loading figures
%% from other directories you can use the `import` package
%%   \usepackage{import}
%% and then include the figures with
%%   \import{<path to file>}{<filename>.pgf}
%%
%% Matplotlib used the following preamble
%%
\begingroup%
\makeatletter%
\begin{pgfpicture}%
\pgfpathrectangle{\pgfpointorigin}{\pgfqpoint{4.747069in}{3.310123in}}%
\pgfusepath{use as bounding box, clip}%
\begin{pgfscope}%
\pgfsetbuttcap%
\pgfsetmiterjoin%
\definecolor{currentfill}{rgb}{1.000000,1.000000,1.000000}%
\pgfsetfillcolor{currentfill}%
\pgfsetlinewidth{0.000000pt}%
\definecolor{currentstroke}{rgb}{1.000000,1.000000,1.000000}%
\pgfsetstrokecolor{currentstroke}%
\pgfsetdash{}{0pt}%
\pgfpathmoveto{\pgfqpoint{0.000000in}{0.000000in}}%
\pgfpathlineto{\pgfqpoint{4.747069in}{0.000000in}}%
\pgfpathlineto{\pgfqpoint{4.747069in}{3.310123in}}%
\pgfpathlineto{\pgfqpoint{0.000000in}{3.310123in}}%
\pgfpathclose%
\pgfusepath{fill}%
\end{pgfscope}%
\begin{pgfscope}%
\pgfsetbuttcap%
\pgfsetmiterjoin%
\definecolor{currentfill}{rgb}{1.000000,1.000000,1.000000}%
\pgfsetfillcolor{currentfill}%
\pgfsetlinewidth{0.000000pt}%
\definecolor{currentstroke}{rgb}{0.000000,0.000000,0.000000}%
\pgfsetstrokecolor{currentstroke}%
\pgfsetstrokeopacity{0.000000}%
\pgfsetdash{}{0pt}%
\pgfpathmoveto{\pgfqpoint{0.772069in}{0.515123in}}%
\pgfpathlineto{\pgfqpoint{4.647069in}{0.515123in}}%
\pgfpathlineto{\pgfqpoint{4.647069in}{3.210123in}}%
\pgfpathlineto{\pgfqpoint{0.772069in}{3.210123in}}%
\pgfpathclose%
\pgfusepath{fill}%
\end{pgfscope}%
\begin{pgfscope}%
\pgfpathrectangle{\pgfqpoint{0.772069in}{0.515123in}}{\pgfqpoint{3.875000in}{2.695000in}}%
\pgfusepath{clip}%
\pgfsetrectcap%
\pgfsetroundjoin%
\pgfsetlinewidth{1.505625pt}%
\definecolor{currentstroke}{rgb}{0.529412,0.807843,0.921569}%
\pgfsetstrokecolor{currentstroke}%
\pgfsetdash{}{0pt}%
\pgfpathmoveto{\pgfqpoint{0.948205in}{0.637623in}}%
\pgfpathlineto{\pgfqpoint{1.339619in}{0.909846in}}%
\pgfpathlineto{\pgfqpoint{1.731033in}{1.182068in}}%
\pgfpathlineto{\pgfqpoint{2.122447in}{1.454290in}}%
\pgfpathlineto{\pgfqpoint{2.513862in}{1.726512in}}%
\pgfpathlineto{\pgfqpoint{2.905276in}{1.998734in}}%
\pgfpathlineto{\pgfqpoint{3.296690in}{2.270957in}}%
\pgfpathlineto{\pgfqpoint{3.688104in}{2.543179in}}%
\pgfpathlineto{\pgfqpoint{4.079518in}{2.815401in}}%
\pgfpathlineto{\pgfqpoint{4.470932in}{3.087623in}}%
\pgfusepath{stroke}%
\end{pgfscope}%
\begin{pgfscope}%
\pgfsetbuttcap%
\pgfsetroundjoin%
\definecolor{currentfill}{rgb}{0.000000,0.000000,0.000000}%
\pgfsetfillcolor{currentfill}%
\pgfsetlinewidth{0.803000pt}%
\definecolor{currentstroke}{rgb}{0.000000,0.000000,0.000000}%
\pgfsetstrokecolor{currentstroke}%
\pgfsetdash{}{0pt}%
\pgfsys@defobject{currentmarker}{\pgfqpoint{0.000000in}{-0.048611in}}{\pgfqpoint{0.000000in}{0.000000in}}{%
\pgfpathmoveto{\pgfqpoint{0.000000in}{0.000000in}}%
\pgfpathlineto{\pgfqpoint{0.000000in}{-0.048611in}}%
\pgfusepath{stroke,fill}%
}%
\begin{pgfscope}%
\pgfsys@transformshift{1.268453in}{0.515123in}%
\pgfsys@useobject{currentmarker}{}%
\end{pgfscope}%
\end{pgfscope}%
\begin{pgfscope}%
\definecolor{textcolor}{rgb}{0.000000,0.000000,0.000000}%
\pgfsetstrokecolor{textcolor}%
\pgfsetfillcolor{textcolor}%
\pgftext[x=1.268453in,y=0.417901in,,top]{\color{textcolor}\rmfamily\fontsize{10.000000}{12.000000}\selectfont \(\displaystyle 10\)}%
\end{pgfscope}%
\begin{pgfscope}%
\pgfsetbuttcap%
\pgfsetroundjoin%
\definecolor{currentfill}{rgb}{0.000000,0.000000,0.000000}%
\pgfsetfillcolor{currentfill}%
\pgfsetlinewidth{0.803000pt}%
\definecolor{currentstroke}{rgb}{0.000000,0.000000,0.000000}%
\pgfsetstrokecolor{currentstroke}%
\pgfsetdash{}{0pt}%
\pgfsys@defobject{currentmarker}{\pgfqpoint{0.000000in}{-0.048611in}}{\pgfqpoint{0.000000in}{0.000000in}}{%
\pgfpathmoveto{\pgfqpoint{0.000000in}{0.000000in}}%
\pgfpathlineto{\pgfqpoint{0.000000in}{-0.048611in}}%
\pgfusepath{stroke,fill}%
}%
\begin{pgfscope}%
\pgfsys@transformshift{1.908949in}{0.515123in}%
\pgfsys@useobject{currentmarker}{}%
\end{pgfscope}%
\end{pgfscope}%
\begin{pgfscope}%
\definecolor{textcolor}{rgb}{0.000000,0.000000,0.000000}%
\pgfsetstrokecolor{textcolor}%
\pgfsetfillcolor{textcolor}%
\pgftext[x=1.908949in,y=0.417901in,,top]{\color{textcolor}\rmfamily\fontsize{10.000000}{12.000000}\selectfont \(\displaystyle 20\)}%
\end{pgfscope}%
\begin{pgfscope}%
\pgfsetbuttcap%
\pgfsetroundjoin%
\definecolor{currentfill}{rgb}{0.000000,0.000000,0.000000}%
\pgfsetfillcolor{currentfill}%
\pgfsetlinewidth{0.803000pt}%
\definecolor{currentstroke}{rgb}{0.000000,0.000000,0.000000}%
\pgfsetstrokecolor{currentstroke}%
\pgfsetdash{}{0pt}%
\pgfsys@defobject{currentmarker}{\pgfqpoint{0.000000in}{-0.048611in}}{\pgfqpoint{0.000000in}{0.000000in}}{%
\pgfpathmoveto{\pgfqpoint{0.000000in}{0.000000in}}%
\pgfpathlineto{\pgfqpoint{0.000000in}{-0.048611in}}%
\pgfusepath{stroke,fill}%
}%
\begin{pgfscope}%
\pgfsys@transformshift{2.549445in}{0.515123in}%
\pgfsys@useobject{currentmarker}{}%
\end{pgfscope}%
\end{pgfscope}%
\begin{pgfscope}%
\definecolor{textcolor}{rgb}{0.000000,0.000000,0.000000}%
\pgfsetstrokecolor{textcolor}%
\pgfsetfillcolor{textcolor}%
\pgftext[x=2.549445in,y=0.417901in,,top]{\color{textcolor}\rmfamily\fontsize{10.000000}{12.000000}\selectfont \(\displaystyle 30\)}%
\end{pgfscope}%
\begin{pgfscope}%
\pgfsetbuttcap%
\pgfsetroundjoin%
\definecolor{currentfill}{rgb}{0.000000,0.000000,0.000000}%
\pgfsetfillcolor{currentfill}%
\pgfsetlinewidth{0.803000pt}%
\definecolor{currentstroke}{rgb}{0.000000,0.000000,0.000000}%
\pgfsetstrokecolor{currentstroke}%
\pgfsetdash{}{0pt}%
\pgfsys@defobject{currentmarker}{\pgfqpoint{0.000000in}{-0.048611in}}{\pgfqpoint{0.000000in}{0.000000in}}{%
\pgfpathmoveto{\pgfqpoint{0.000000in}{0.000000in}}%
\pgfpathlineto{\pgfqpoint{0.000000in}{-0.048611in}}%
\pgfusepath{stroke,fill}%
}%
\begin{pgfscope}%
\pgfsys@transformshift{3.189941in}{0.515123in}%
\pgfsys@useobject{currentmarker}{}%
\end{pgfscope}%
\end{pgfscope}%
\begin{pgfscope}%
\definecolor{textcolor}{rgb}{0.000000,0.000000,0.000000}%
\pgfsetstrokecolor{textcolor}%
\pgfsetfillcolor{textcolor}%
\pgftext[x=3.189941in,y=0.417901in,,top]{\color{textcolor}\rmfamily\fontsize{10.000000}{12.000000}\selectfont \(\displaystyle 40\)}%
\end{pgfscope}%
\begin{pgfscope}%
\pgfsetbuttcap%
\pgfsetroundjoin%
\definecolor{currentfill}{rgb}{0.000000,0.000000,0.000000}%
\pgfsetfillcolor{currentfill}%
\pgfsetlinewidth{0.803000pt}%
\definecolor{currentstroke}{rgb}{0.000000,0.000000,0.000000}%
\pgfsetstrokecolor{currentstroke}%
\pgfsetdash{}{0pt}%
\pgfsys@defobject{currentmarker}{\pgfqpoint{0.000000in}{-0.048611in}}{\pgfqpoint{0.000000in}{0.000000in}}{%
\pgfpathmoveto{\pgfqpoint{0.000000in}{0.000000in}}%
\pgfpathlineto{\pgfqpoint{0.000000in}{-0.048611in}}%
\pgfusepath{stroke,fill}%
}%
\begin{pgfscope}%
\pgfsys@transformshift{3.830436in}{0.515123in}%
\pgfsys@useobject{currentmarker}{}%
\end{pgfscope}%
\end{pgfscope}%
\begin{pgfscope}%
\definecolor{textcolor}{rgb}{0.000000,0.000000,0.000000}%
\pgfsetstrokecolor{textcolor}%
\pgfsetfillcolor{textcolor}%
\pgftext[x=3.830436in,y=0.417901in,,top]{\color{textcolor}\rmfamily\fontsize{10.000000}{12.000000}\selectfont \(\displaystyle 50\)}%
\end{pgfscope}%
\begin{pgfscope}%
\pgfsetbuttcap%
\pgfsetroundjoin%
\definecolor{currentfill}{rgb}{0.000000,0.000000,0.000000}%
\pgfsetfillcolor{currentfill}%
\pgfsetlinewidth{0.803000pt}%
\definecolor{currentstroke}{rgb}{0.000000,0.000000,0.000000}%
\pgfsetstrokecolor{currentstroke}%
\pgfsetdash{}{0pt}%
\pgfsys@defobject{currentmarker}{\pgfqpoint{0.000000in}{-0.048611in}}{\pgfqpoint{0.000000in}{0.000000in}}{%
\pgfpathmoveto{\pgfqpoint{0.000000in}{0.000000in}}%
\pgfpathlineto{\pgfqpoint{0.000000in}{-0.048611in}}%
\pgfusepath{stroke,fill}%
}%
\begin{pgfscope}%
\pgfsys@transformshift{4.470932in}{0.515123in}%
\pgfsys@useobject{currentmarker}{}%
\end{pgfscope}%
\end{pgfscope}%
\begin{pgfscope}%
\definecolor{textcolor}{rgb}{0.000000,0.000000,0.000000}%
\pgfsetstrokecolor{textcolor}%
\pgfsetfillcolor{textcolor}%
\pgftext[x=4.470932in,y=0.417901in,,top]{\color{textcolor}\rmfamily\fontsize{10.000000}{12.000000}\selectfont \(\displaystyle 60\)}%
\end{pgfscope}%
\begin{pgfscope}%
\definecolor{textcolor}{rgb}{0.000000,0.000000,0.000000}%
\pgfsetstrokecolor{textcolor}%
\pgfsetfillcolor{textcolor}%
\pgftext[x=2.709569in,y=0.238889in,,top]{\color{textcolor}\rmfamily\fontsize{10.000000}{12.000000}\selectfont I(\(\displaystyle \mu\)A)}%
\end{pgfscope}%
\begin{pgfscope}%
\pgfsetbuttcap%
\pgfsetroundjoin%
\definecolor{currentfill}{rgb}{0.000000,0.000000,0.000000}%
\pgfsetfillcolor{currentfill}%
\pgfsetlinewidth{0.803000pt}%
\definecolor{currentstroke}{rgb}{0.000000,0.000000,0.000000}%
\pgfsetstrokecolor{currentstroke}%
\pgfsetdash{}{0pt}%
\pgfsys@defobject{currentmarker}{\pgfqpoint{-0.048611in}{0.000000in}}{\pgfqpoint{0.000000in}{0.000000in}}{%
\pgfpathmoveto{\pgfqpoint{0.000000in}{0.000000in}}%
\pgfpathlineto{\pgfqpoint{-0.048611in}{0.000000in}}%
\pgfusepath{stroke,fill}%
}%
\begin{pgfscope}%
\pgfsys@transformshift{0.772069in}{0.921510in}%
\pgfsys@useobject{currentmarker}{}%
\end{pgfscope}%
\end{pgfscope}%
\begin{pgfscope}%
\definecolor{textcolor}{rgb}{0.000000,0.000000,0.000000}%
\pgfsetstrokecolor{textcolor}%
\pgfsetfillcolor{textcolor}%
\pgftext[x=0.605402in,y=0.873285in,left,base]{\color{textcolor}\rmfamily\fontsize{10.000000}{12.000000}\selectfont \(\displaystyle 2\)}%
\end{pgfscope}%
\begin{pgfscope}%
\pgfsetbuttcap%
\pgfsetroundjoin%
\definecolor{currentfill}{rgb}{0.000000,0.000000,0.000000}%
\pgfsetfillcolor{currentfill}%
\pgfsetlinewidth{0.803000pt}%
\definecolor{currentstroke}{rgb}{0.000000,0.000000,0.000000}%
\pgfsetstrokecolor{currentstroke}%
\pgfsetdash{}{0pt}%
\pgfsys@defobject{currentmarker}{\pgfqpoint{-0.048611in}{0.000000in}}{\pgfqpoint{0.000000in}{0.000000in}}{%
\pgfpathmoveto{\pgfqpoint{0.000000in}{0.000000in}}%
\pgfpathlineto{\pgfqpoint{-0.048611in}{0.000000in}}%
\pgfusepath{stroke,fill}%
}%
\begin{pgfscope}%
\pgfsys@transformshift{0.772069in}{1.428124in}%
\pgfsys@useobject{currentmarker}{}%
\end{pgfscope}%
\end{pgfscope}%
\begin{pgfscope}%
\definecolor{textcolor}{rgb}{0.000000,0.000000,0.000000}%
\pgfsetstrokecolor{textcolor}%
\pgfsetfillcolor{textcolor}%
\pgftext[x=0.605402in,y=1.379899in,left,base]{\color{textcolor}\rmfamily\fontsize{10.000000}{12.000000}\selectfont \(\displaystyle 4\)}%
\end{pgfscope}%
\begin{pgfscope}%
\pgfsetbuttcap%
\pgfsetroundjoin%
\definecolor{currentfill}{rgb}{0.000000,0.000000,0.000000}%
\pgfsetfillcolor{currentfill}%
\pgfsetlinewidth{0.803000pt}%
\definecolor{currentstroke}{rgb}{0.000000,0.000000,0.000000}%
\pgfsetstrokecolor{currentstroke}%
\pgfsetdash{}{0pt}%
\pgfsys@defobject{currentmarker}{\pgfqpoint{-0.048611in}{0.000000in}}{\pgfqpoint{0.000000in}{0.000000in}}{%
\pgfpathmoveto{\pgfqpoint{0.000000in}{0.000000in}}%
\pgfpathlineto{\pgfqpoint{-0.048611in}{0.000000in}}%
\pgfusepath{stroke,fill}%
}%
\begin{pgfscope}%
\pgfsys@transformshift{0.772069in}{1.934738in}%
\pgfsys@useobject{currentmarker}{}%
\end{pgfscope}%
\end{pgfscope}%
\begin{pgfscope}%
\definecolor{textcolor}{rgb}{0.000000,0.000000,0.000000}%
\pgfsetstrokecolor{textcolor}%
\pgfsetfillcolor{textcolor}%
\pgftext[x=0.605402in,y=1.886513in,left,base]{\color{textcolor}\rmfamily\fontsize{10.000000}{12.000000}\selectfont \(\displaystyle 6\)}%
\end{pgfscope}%
\begin{pgfscope}%
\pgfsetbuttcap%
\pgfsetroundjoin%
\definecolor{currentfill}{rgb}{0.000000,0.000000,0.000000}%
\pgfsetfillcolor{currentfill}%
\pgfsetlinewidth{0.803000pt}%
\definecolor{currentstroke}{rgb}{0.000000,0.000000,0.000000}%
\pgfsetstrokecolor{currentstroke}%
\pgfsetdash{}{0pt}%
\pgfsys@defobject{currentmarker}{\pgfqpoint{-0.048611in}{0.000000in}}{\pgfqpoint{0.000000in}{0.000000in}}{%
\pgfpathmoveto{\pgfqpoint{0.000000in}{0.000000in}}%
\pgfpathlineto{\pgfqpoint{-0.048611in}{0.000000in}}%
\pgfusepath{stroke,fill}%
}%
\begin{pgfscope}%
\pgfsys@transformshift{0.772069in}{2.441352in}%
\pgfsys@useobject{currentmarker}{}%
\end{pgfscope}%
\end{pgfscope}%
\begin{pgfscope}%
\definecolor{textcolor}{rgb}{0.000000,0.000000,0.000000}%
\pgfsetstrokecolor{textcolor}%
\pgfsetfillcolor{textcolor}%
\pgftext[x=0.605402in,y=2.393127in,left,base]{\color{textcolor}\rmfamily\fontsize{10.000000}{12.000000}\selectfont \(\displaystyle 8\)}%
\end{pgfscope}%
\begin{pgfscope}%
\pgfsetbuttcap%
\pgfsetroundjoin%
\definecolor{currentfill}{rgb}{0.000000,0.000000,0.000000}%
\pgfsetfillcolor{currentfill}%
\pgfsetlinewidth{0.803000pt}%
\definecolor{currentstroke}{rgb}{0.000000,0.000000,0.000000}%
\pgfsetstrokecolor{currentstroke}%
\pgfsetdash{}{0pt}%
\pgfsys@defobject{currentmarker}{\pgfqpoint{-0.048611in}{0.000000in}}{\pgfqpoint{0.000000in}{0.000000in}}{%
\pgfpathmoveto{\pgfqpoint{0.000000in}{0.000000in}}%
\pgfpathlineto{\pgfqpoint{-0.048611in}{0.000000in}}%
\pgfusepath{stroke,fill}%
}%
\begin{pgfscope}%
\pgfsys@transformshift{0.772069in}{2.947966in}%
\pgfsys@useobject{currentmarker}{}%
\end{pgfscope}%
\end{pgfscope}%
\begin{pgfscope}%
\definecolor{textcolor}{rgb}{0.000000,0.000000,0.000000}%
\pgfsetstrokecolor{textcolor}%
\pgfsetfillcolor{textcolor}%
\pgftext[x=0.535957in,y=2.899741in,left,base]{\color{textcolor}\rmfamily\fontsize{10.000000}{12.000000}\selectfont \(\displaystyle 10\)}%
\end{pgfscope}%
\begin{pgfscope}%
\definecolor{textcolor}{rgb}{0.000000,0.000000,0.000000}%
\pgfsetstrokecolor{textcolor}%
\pgfsetfillcolor{textcolor}%
\pgftext[x=0.258179in,y=1.862623in,,bottom]{\color{textcolor}\rmfamily\fontsize{10.000000}{12.000000}\selectfont V(V)}%
\end{pgfscope}%
\begin{pgfscope}%
\pgfpathrectangle{\pgfqpoint{0.772069in}{0.515123in}}{\pgfqpoint{3.875000in}{2.695000in}}%
\pgfusepath{clip}%
\pgfsetbuttcap%
\pgfsetroundjoin%
\definecolor{currentfill}{rgb}{0.121569,0.466667,0.705882}%
\pgfsetfillcolor{currentfill}%
\pgfsetlinewidth{1.003750pt}%
\definecolor{currentstroke}{rgb}{0.121569,0.466667,0.705882}%
\pgfsetstrokecolor{currentstroke}%
\pgfsetdash{}{0pt}%
\pgfpathmoveto{\pgfqpoint{1.057089in}{0.673145in}}%
\pgfpathcurveto{\pgfqpoint{1.068139in}{0.673145in}}{\pgfqpoint{1.078739in}{0.677535in}}{\pgfqpoint{1.086552in}{0.685349in}}%
\pgfpathcurveto{\pgfqpoint{1.094366in}{0.693162in}}{\pgfqpoint{1.098756in}{0.703761in}}{\pgfqpoint{1.098756in}{0.714812in}}%
\pgfpathcurveto{\pgfqpoint{1.098756in}{0.725862in}}{\pgfqpoint{1.094366in}{0.736461in}}{\pgfqpoint{1.086552in}{0.744274in}}%
\pgfpathcurveto{\pgfqpoint{1.078739in}{0.752088in}}{\pgfqpoint{1.068139in}{0.756478in}}{\pgfqpoint{1.057089in}{0.756478in}}%
\pgfpathcurveto{\pgfqpoint{1.046039in}{0.756478in}}{\pgfqpoint{1.035440in}{0.752088in}}{\pgfqpoint{1.027627in}{0.744274in}}%
\pgfpathcurveto{\pgfqpoint{1.019813in}{0.736461in}}{\pgfqpoint{1.015423in}{0.725862in}}{\pgfqpoint{1.015423in}{0.714812in}}%
\pgfpathcurveto{\pgfqpoint{1.015423in}{0.703761in}}{\pgfqpoint{1.019813in}{0.693162in}}{\pgfqpoint{1.027627in}{0.685349in}}%
\pgfpathcurveto{\pgfqpoint{1.035440in}{0.677535in}}{\pgfqpoint{1.046039in}{0.673145in}}{\pgfqpoint{1.057089in}{0.673145in}}%
\pgfpathclose%
\pgfusepath{stroke,fill}%
\end{pgfscope}%
\begin{pgfscope}%
\pgfpathrectangle{\pgfqpoint{0.772069in}{0.515123in}}{\pgfqpoint{3.875000in}{2.695000in}}%
\pgfusepath{clip}%
\pgfsetbuttcap%
\pgfsetroundjoin%
\definecolor{currentfill}{rgb}{0.121569,0.466667,0.705882}%
\pgfsetfillcolor{currentfill}%
\pgfsetlinewidth{1.003750pt}%
\definecolor{currentstroke}{rgb}{0.121569,0.466667,0.705882}%
\pgfsetstrokecolor{currentstroke}%
\pgfsetdash{}{0pt}%
\pgfpathmoveto{\pgfqpoint{1.396552in}{0.907707in}}%
\pgfpathcurveto{\pgfqpoint{1.407602in}{0.907707in}}{\pgfqpoint{1.418201in}{0.912097in}}{\pgfqpoint{1.426015in}{0.919911in}}%
\pgfpathcurveto{\pgfqpoint{1.433829in}{0.927725in}}{\pgfqpoint{1.438219in}{0.938324in}}{\pgfqpoint{1.438219in}{0.949374in}}%
\pgfpathcurveto{\pgfqpoint{1.438219in}{0.960424in}}{\pgfqpoint{1.433829in}{0.971023in}}{\pgfqpoint{1.426015in}{0.978837in}}%
\pgfpathcurveto{\pgfqpoint{1.418201in}{0.986650in}}{\pgfqpoint{1.407602in}{0.991040in}}{\pgfqpoint{1.396552in}{0.991040in}}%
\pgfpathcurveto{\pgfqpoint{1.385502in}{0.991040in}}{\pgfqpoint{1.374903in}{0.986650in}}{\pgfqpoint{1.367089in}{0.978837in}}%
\pgfpathcurveto{\pgfqpoint{1.359276in}{0.971023in}}{\pgfqpoint{1.354885in}{0.960424in}}{\pgfqpoint{1.354885in}{0.949374in}}%
\pgfpathcurveto{\pgfqpoint{1.354885in}{0.938324in}}{\pgfqpoint{1.359276in}{0.927725in}}{\pgfqpoint{1.367089in}{0.919911in}}%
\pgfpathcurveto{\pgfqpoint{1.374903in}{0.912097in}}{\pgfqpoint{1.385502in}{0.907707in}}{\pgfqpoint{1.396552in}{0.907707in}}%
\pgfpathclose%
\pgfusepath{stroke,fill}%
\end{pgfscope}%
\begin{pgfscope}%
\pgfpathrectangle{\pgfqpoint{0.772069in}{0.515123in}}{\pgfqpoint{3.875000in}{2.695000in}}%
\pgfusepath{clip}%
\pgfsetbuttcap%
\pgfsetroundjoin%
\definecolor{currentfill}{rgb}{0.121569,0.466667,0.705882}%
\pgfsetfillcolor{currentfill}%
\pgfsetlinewidth{1.003750pt}%
\definecolor{currentstroke}{rgb}{0.121569,0.466667,0.705882}%
\pgfsetstrokecolor{currentstroke}%
\pgfsetdash{}{0pt}%
\pgfpathmoveto{\pgfqpoint{1.729610in}{1.143283in}}%
\pgfpathcurveto{\pgfqpoint{1.740660in}{1.143283in}}{\pgfqpoint{1.751259in}{1.147673in}}{\pgfqpoint{1.759073in}{1.155486in}}%
\pgfpathcurveto{\pgfqpoint{1.766886in}{1.163300in}}{\pgfqpoint{1.771277in}{1.173899in}}{\pgfqpoint{1.771277in}{1.184949in}}%
\pgfpathcurveto{\pgfqpoint{1.771277in}{1.195999in}}{\pgfqpoint{1.766886in}{1.206598in}}{\pgfqpoint{1.759073in}{1.214412in}}%
\pgfpathcurveto{\pgfqpoint{1.751259in}{1.222226in}}{\pgfqpoint{1.740660in}{1.226616in}}{\pgfqpoint{1.729610in}{1.226616in}}%
\pgfpathcurveto{\pgfqpoint{1.718560in}{1.226616in}}{\pgfqpoint{1.707961in}{1.222226in}}{\pgfqpoint{1.700147in}{1.214412in}}%
\pgfpathcurveto{\pgfqpoint{1.692334in}{1.206598in}}{\pgfqpoint{1.687943in}{1.195999in}}{\pgfqpoint{1.687943in}{1.184949in}}%
\pgfpathcurveto{\pgfqpoint{1.687943in}{1.173899in}}{\pgfqpoint{1.692334in}{1.163300in}}{\pgfqpoint{1.700147in}{1.155486in}}%
\pgfpathcurveto{\pgfqpoint{1.707961in}{1.147673in}}{\pgfqpoint{1.718560in}{1.143283in}}{\pgfqpoint{1.729610in}{1.143283in}}%
\pgfpathclose%
\pgfusepath{stroke,fill}%
\end{pgfscope}%
\begin{pgfscope}%
\pgfpathrectangle{\pgfqpoint{0.772069in}{0.515123in}}{\pgfqpoint{3.875000in}{2.695000in}}%
\pgfusepath{clip}%
\pgfsetbuttcap%
\pgfsetroundjoin%
\definecolor{currentfill}{rgb}{0.121569,0.466667,0.705882}%
\pgfsetfillcolor{currentfill}%
\pgfsetlinewidth{1.003750pt}%
\definecolor{currentstroke}{rgb}{0.121569,0.466667,0.705882}%
\pgfsetstrokecolor{currentstroke}%
\pgfsetdash{}{0pt}%
\pgfpathmoveto{\pgfqpoint{2.101098in}{1.401656in}}%
\pgfpathcurveto{\pgfqpoint{2.112148in}{1.401656in}}{\pgfqpoint{2.122747in}{1.406046in}}{\pgfqpoint{2.130560in}{1.413860in}}%
\pgfpathcurveto{\pgfqpoint{2.138374in}{1.421673in}}{\pgfqpoint{2.142764in}{1.432272in}}{\pgfqpoint{2.142764in}{1.443322in}}%
\pgfpathcurveto{\pgfqpoint{2.142764in}{1.454372in}}{\pgfqpoint{2.138374in}{1.464972in}}{\pgfqpoint{2.130560in}{1.472785in}}%
\pgfpathcurveto{\pgfqpoint{2.122747in}{1.480599in}}{\pgfqpoint{2.112148in}{1.484989in}}{\pgfqpoint{2.101098in}{1.484989in}}%
\pgfpathcurveto{\pgfqpoint{2.090047in}{1.484989in}}{\pgfqpoint{2.079448in}{1.480599in}}{\pgfqpoint{2.071635in}{1.472785in}}%
\pgfpathcurveto{\pgfqpoint{2.063821in}{1.464972in}}{\pgfqpoint{2.059431in}{1.454372in}}{\pgfqpoint{2.059431in}{1.443322in}}%
\pgfpathcurveto{\pgfqpoint{2.059431in}{1.432272in}}{\pgfqpoint{2.063821in}{1.421673in}}{\pgfqpoint{2.071635in}{1.413860in}}%
\pgfpathcurveto{\pgfqpoint{2.079448in}{1.406046in}}{\pgfqpoint{2.090047in}{1.401656in}}{\pgfqpoint{2.101098in}{1.401656in}}%
\pgfpathclose%
\pgfusepath{stroke,fill}%
\end{pgfscope}%
\begin{pgfscope}%
\pgfpathrectangle{\pgfqpoint{0.772069in}{0.515123in}}{\pgfqpoint{3.875000in}{2.695000in}}%
\pgfusepath{clip}%
\pgfsetbuttcap%
\pgfsetroundjoin%
\definecolor{currentfill}{rgb}{0.121569,0.466667,0.705882}%
\pgfsetfillcolor{currentfill}%
\pgfsetlinewidth{1.003750pt}%
\definecolor{currentstroke}{rgb}{0.121569,0.466667,0.705882}%
\pgfsetstrokecolor{currentstroke}%
\pgfsetdash{}{0pt}%
\pgfpathmoveto{\pgfqpoint{2.459775in}{1.649897in}}%
\pgfpathcurveto{\pgfqpoint{2.470825in}{1.649897in}}{\pgfqpoint{2.481424in}{1.654287in}}{\pgfqpoint{2.489238in}{1.662100in}}%
\pgfpathcurveto{\pgfqpoint{2.497052in}{1.669914in}}{\pgfqpoint{2.501442in}{1.680513in}}{\pgfqpoint{2.501442in}{1.691563in}}%
\pgfpathcurveto{\pgfqpoint{2.501442in}{1.702613in}}{\pgfqpoint{2.497052in}{1.713212in}}{\pgfqpoint{2.489238in}{1.721026in}}%
\pgfpathcurveto{\pgfqpoint{2.481424in}{1.728840in}}{\pgfqpoint{2.470825in}{1.733230in}}{\pgfqpoint{2.459775in}{1.733230in}}%
\pgfpathcurveto{\pgfqpoint{2.448725in}{1.733230in}}{\pgfqpoint{2.438126in}{1.728840in}}{\pgfqpoint{2.430313in}{1.721026in}}%
\pgfpathcurveto{\pgfqpoint{2.422499in}{1.713212in}}{\pgfqpoint{2.418109in}{1.702613in}}{\pgfqpoint{2.418109in}{1.691563in}}%
\pgfpathcurveto{\pgfqpoint{2.418109in}{1.680513in}}{\pgfqpoint{2.422499in}{1.669914in}}{\pgfqpoint{2.430313in}{1.662100in}}%
\pgfpathcurveto{\pgfqpoint{2.438126in}{1.654287in}}{\pgfqpoint{2.448725in}{1.649897in}}{\pgfqpoint{2.459775in}{1.649897in}}%
\pgfpathclose%
\pgfusepath{stroke,fill}%
\end{pgfscope}%
\begin{pgfscope}%
\pgfpathrectangle{\pgfqpoint{0.772069in}{0.515123in}}{\pgfqpoint{3.875000in}{2.695000in}}%
\pgfusepath{clip}%
\pgfsetbuttcap%
\pgfsetroundjoin%
\definecolor{currentfill}{rgb}{0.121569,0.466667,0.705882}%
\pgfsetfillcolor{currentfill}%
\pgfsetlinewidth{1.003750pt}%
\definecolor{currentstroke}{rgb}{0.121569,0.466667,0.705882}%
\pgfsetstrokecolor{currentstroke}%
\pgfsetdash{}{0pt}%
\pgfpathmoveto{\pgfqpoint{2.837668in}{1.913336in}}%
\pgfpathcurveto{\pgfqpoint{2.848718in}{1.913336in}}{\pgfqpoint{2.859317in}{1.917726in}}{\pgfqpoint{2.867131in}{1.925540in}}%
\pgfpathcurveto{\pgfqpoint{2.874944in}{1.933353in}}{\pgfqpoint{2.879335in}{1.943952in}}{\pgfqpoint{2.879335in}{1.955002in}}%
\pgfpathcurveto{\pgfqpoint{2.879335in}{1.966053in}}{\pgfqpoint{2.874944in}{1.976652in}}{\pgfqpoint{2.867131in}{1.984465in}}%
\pgfpathcurveto{\pgfqpoint{2.859317in}{1.992279in}}{\pgfqpoint{2.848718in}{1.996669in}}{\pgfqpoint{2.837668in}{1.996669in}}%
\pgfpathcurveto{\pgfqpoint{2.826618in}{1.996669in}}{\pgfqpoint{2.816019in}{1.992279in}}{\pgfqpoint{2.808205in}{1.984465in}}%
\pgfpathcurveto{\pgfqpoint{2.800391in}{1.976652in}}{\pgfqpoint{2.796001in}{1.966053in}}{\pgfqpoint{2.796001in}{1.955002in}}%
\pgfpathcurveto{\pgfqpoint{2.796001in}{1.943952in}}{\pgfqpoint{2.800391in}{1.933353in}}{\pgfqpoint{2.808205in}{1.925540in}}%
\pgfpathcurveto{\pgfqpoint{2.816019in}{1.917726in}}{\pgfqpoint{2.826618in}{1.913336in}}{\pgfqpoint{2.837668in}{1.913336in}}%
\pgfpathclose%
\pgfusepath{stroke,fill}%
\end{pgfscope}%
\begin{pgfscope}%
\pgfpathrectangle{\pgfqpoint{0.772069in}{0.515123in}}{\pgfqpoint{3.875000in}{2.695000in}}%
\pgfusepath{clip}%
\pgfsetbuttcap%
\pgfsetroundjoin%
\definecolor{currentfill}{rgb}{0.121569,0.466667,0.705882}%
\pgfsetfillcolor{currentfill}%
\pgfsetlinewidth{1.003750pt}%
\definecolor{currentstroke}{rgb}{0.121569,0.466667,0.705882}%
\pgfsetstrokecolor{currentstroke}%
\pgfsetdash{}{0pt}%
\pgfpathmoveto{\pgfqpoint{3.209155in}{2.171709in}}%
\pgfpathcurveto{\pgfqpoint{3.220206in}{2.171709in}}{\pgfqpoint{3.230805in}{2.176099in}}{\pgfqpoint{3.238618in}{2.183913in}}%
\pgfpathcurveto{\pgfqpoint{3.246432in}{2.191726in}}{\pgfqpoint{3.250822in}{2.202325in}}{\pgfqpoint{3.250822in}{2.213376in}}%
\pgfpathcurveto{\pgfqpoint{3.250822in}{2.224426in}}{\pgfqpoint{3.246432in}{2.235025in}}{\pgfqpoint{3.238618in}{2.242838in}}%
\pgfpathcurveto{\pgfqpoint{3.230805in}{2.250652in}}{\pgfqpoint{3.220206in}{2.255042in}}{\pgfqpoint{3.209155in}{2.255042in}}%
\pgfpathcurveto{\pgfqpoint{3.198105in}{2.255042in}}{\pgfqpoint{3.187506in}{2.250652in}}{\pgfqpoint{3.179693in}{2.242838in}}%
\pgfpathcurveto{\pgfqpoint{3.171879in}{2.235025in}}{\pgfqpoint{3.167489in}{2.224426in}}{\pgfqpoint{3.167489in}{2.213376in}}%
\pgfpathcurveto{\pgfqpoint{3.167489in}{2.202325in}}{\pgfqpoint{3.171879in}{2.191726in}}{\pgfqpoint{3.179693in}{2.183913in}}%
\pgfpathcurveto{\pgfqpoint{3.187506in}{2.176099in}}{\pgfqpoint{3.198105in}{2.171709in}}{\pgfqpoint{3.209155in}{2.171709in}}%
\pgfpathclose%
\pgfusepath{stroke,fill}%
\end{pgfscope}%
\begin{pgfscope}%
\pgfpathrectangle{\pgfqpoint{0.772069in}{0.515123in}}{\pgfqpoint{3.875000in}{2.695000in}}%
\pgfusepath{clip}%
\pgfsetbuttcap%
\pgfsetroundjoin%
\definecolor{currentfill}{rgb}{0.121569,0.466667,0.705882}%
\pgfsetfillcolor{currentfill}%
\pgfsetlinewidth{1.003750pt}%
\definecolor{currentstroke}{rgb}{0.121569,0.466667,0.705882}%
\pgfsetstrokecolor{currentstroke}%
\pgfsetdash{}{0pt}%
\pgfpathmoveto{\pgfqpoint{3.574238in}{2.425016in}}%
\pgfpathcurveto{\pgfqpoint{3.585288in}{2.425016in}}{\pgfqpoint{3.595887in}{2.429406in}}{\pgfqpoint{3.603701in}{2.437220in}}%
\pgfpathcurveto{\pgfqpoint{3.611515in}{2.445033in}}{\pgfqpoint{3.615905in}{2.455632in}}{\pgfqpoint{3.615905in}{2.466683in}}%
\pgfpathcurveto{\pgfqpoint{3.615905in}{2.477733in}}{\pgfqpoint{3.611515in}{2.488332in}}{\pgfqpoint{3.603701in}{2.496145in}}%
\pgfpathcurveto{\pgfqpoint{3.595887in}{2.503959in}}{\pgfqpoint{3.585288in}{2.508349in}}{\pgfqpoint{3.574238in}{2.508349in}}%
\pgfpathcurveto{\pgfqpoint{3.563188in}{2.508349in}}{\pgfqpoint{3.552589in}{2.503959in}}{\pgfqpoint{3.544775in}{2.496145in}}%
\pgfpathcurveto{\pgfqpoint{3.536962in}{2.488332in}}{\pgfqpoint{3.532571in}{2.477733in}}{\pgfqpoint{3.532571in}{2.466683in}}%
\pgfpathcurveto{\pgfqpoint{3.532571in}{2.455632in}}{\pgfqpoint{3.536962in}{2.445033in}}{\pgfqpoint{3.544775in}{2.437220in}}%
\pgfpathcurveto{\pgfqpoint{3.552589in}{2.429406in}}{\pgfqpoint{3.563188in}{2.425016in}}{\pgfqpoint{3.574238in}{2.425016in}}%
\pgfpathclose%
\pgfusepath{stroke,fill}%
\end{pgfscope}%
\begin{pgfscope}%
\pgfpathrectangle{\pgfqpoint{0.772069in}{0.515123in}}{\pgfqpoint{3.875000in}{2.695000in}}%
\pgfusepath{clip}%
\pgfsetbuttcap%
\pgfsetroundjoin%
\definecolor{currentfill}{rgb}{0.121569,0.466667,0.705882}%
\pgfsetfillcolor{currentfill}%
\pgfsetlinewidth{1.003750pt}%
\definecolor{currentstroke}{rgb}{0.121569,0.466667,0.705882}%
\pgfsetstrokecolor{currentstroke}%
\pgfsetdash{}{0pt}%
\pgfpathmoveto{\pgfqpoint{3.964941in}{2.693521in}}%
\pgfpathcurveto{\pgfqpoint{3.975991in}{2.693521in}}{\pgfqpoint{3.986590in}{2.697912in}}{\pgfqpoint{3.994403in}{2.705725in}}%
\pgfpathcurveto{\pgfqpoint{4.002217in}{2.713539in}}{\pgfqpoint{4.006607in}{2.724138in}}{\pgfqpoint{4.006607in}{2.735188in}}%
\pgfpathcurveto{\pgfqpoint{4.006607in}{2.746238in}}{\pgfqpoint{4.002217in}{2.756837in}}{\pgfqpoint{3.994403in}{2.764651in}}%
\pgfpathcurveto{\pgfqpoint{3.986590in}{2.772464in}}{\pgfqpoint{3.975991in}{2.776855in}}{\pgfqpoint{3.964941in}{2.776855in}}%
\pgfpathcurveto{\pgfqpoint{3.953890in}{2.776855in}}{\pgfqpoint{3.943291in}{2.772464in}}{\pgfqpoint{3.935478in}{2.764651in}}%
\pgfpathcurveto{\pgfqpoint{3.927664in}{2.756837in}}{\pgfqpoint{3.923274in}{2.746238in}}{\pgfqpoint{3.923274in}{2.735188in}}%
\pgfpathcurveto{\pgfqpoint{3.923274in}{2.724138in}}{\pgfqpoint{3.927664in}{2.713539in}}{\pgfqpoint{3.935478in}{2.705725in}}%
\pgfpathcurveto{\pgfqpoint{3.943291in}{2.697912in}}{\pgfqpoint{3.953890in}{2.693521in}}{\pgfqpoint{3.964941in}{2.693521in}}%
\pgfpathclose%
\pgfusepath{stroke,fill}%
\end{pgfscope}%
\begin{pgfscope}%
\pgfpathrectangle{\pgfqpoint{0.772069in}{0.515123in}}{\pgfqpoint{3.875000in}{2.695000in}}%
\pgfusepath{clip}%
\pgfsetbuttcap%
\pgfsetroundjoin%
\definecolor{currentfill}{rgb}{0.121569,0.466667,0.705882}%
\pgfsetfillcolor{currentfill}%
\pgfsetlinewidth{1.003750pt}%
\definecolor{currentstroke}{rgb}{0.121569,0.466667,0.705882}%
\pgfsetstrokecolor{currentstroke}%
\pgfsetdash{}{0pt}%
\pgfpathmoveto{\pgfqpoint{4.317213in}{2.941762in}}%
\pgfpathcurveto{\pgfqpoint{4.328263in}{2.941762in}}{\pgfqpoint{4.338862in}{2.946152in}}{\pgfqpoint{4.346676in}{2.953966in}}%
\pgfpathcurveto{\pgfqpoint{4.354490in}{2.961780in}}{\pgfqpoint{4.358880in}{2.972379in}}{\pgfqpoint{4.358880in}{2.983429in}}%
\pgfpathcurveto{\pgfqpoint{4.358880in}{2.994479in}}{\pgfqpoint{4.354490in}{3.005078in}}{\pgfqpoint{4.346676in}{3.012892in}}%
\pgfpathcurveto{\pgfqpoint{4.338862in}{3.020705in}}{\pgfqpoint{4.328263in}{3.025095in}}{\pgfqpoint{4.317213in}{3.025095in}}%
\pgfpathcurveto{\pgfqpoint{4.306163in}{3.025095in}}{\pgfqpoint{4.295564in}{3.020705in}}{\pgfqpoint{4.287751in}{3.012892in}}%
\pgfpathcurveto{\pgfqpoint{4.279937in}{3.005078in}}{\pgfqpoint{4.275547in}{2.994479in}}{\pgfqpoint{4.275547in}{2.983429in}}%
\pgfpathcurveto{\pgfqpoint{4.275547in}{2.972379in}}{\pgfqpoint{4.279937in}{2.961780in}}{\pgfqpoint{4.287751in}{2.953966in}}%
\pgfpathcurveto{\pgfqpoint{4.295564in}{2.946152in}}{\pgfqpoint{4.306163in}{2.941762in}}{\pgfqpoint{4.317213in}{2.941762in}}%
\pgfpathclose%
\pgfusepath{stroke,fill}%
\end{pgfscope}%
\begin{pgfscope}%
\pgfsetrectcap%
\pgfsetmiterjoin%
\pgfsetlinewidth{0.803000pt}%
\definecolor{currentstroke}{rgb}{0.000000,0.000000,0.000000}%
\pgfsetstrokecolor{currentstroke}%
\pgfsetdash{}{0pt}%
\pgfpathmoveto{\pgfqpoint{0.772069in}{0.515123in}}%
\pgfpathlineto{\pgfqpoint{0.772069in}{3.210123in}}%
\pgfusepath{stroke}%
\end{pgfscope}%
\begin{pgfscope}%
\pgfsetrectcap%
\pgfsetmiterjoin%
\pgfsetlinewidth{0.803000pt}%
\definecolor{currentstroke}{rgb}{0.000000,0.000000,0.000000}%
\pgfsetstrokecolor{currentstroke}%
\pgfsetdash{}{0pt}%
\pgfpathmoveto{\pgfqpoint{4.647069in}{0.515123in}}%
\pgfpathlineto{\pgfqpoint{4.647069in}{3.210123in}}%
\pgfusepath{stroke}%
\end{pgfscope}%
\begin{pgfscope}%
\pgfsetrectcap%
\pgfsetmiterjoin%
\pgfsetlinewidth{0.803000pt}%
\definecolor{currentstroke}{rgb}{0.000000,0.000000,0.000000}%
\pgfsetstrokecolor{currentstroke}%
\pgfsetdash{}{0pt}%
\pgfpathmoveto{\pgfqpoint{0.772069in}{0.515123in}}%
\pgfpathlineto{\pgfqpoint{4.647069in}{0.515123in}}%
\pgfusepath{stroke}%
\end{pgfscope}%
\begin{pgfscope}%
\pgfsetrectcap%
\pgfsetmiterjoin%
\pgfsetlinewidth{0.803000pt}%
\definecolor{currentstroke}{rgb}{0.000000,0.000000,0.000000}%
\pgfsetstrokecolor{currentstroke}%
\pgfsetdash{}{0pt}%
\pgfpathmoveto{\pgfqpoint{0.772069in}{3.210123in}}%
\pgfpathlineto{\pgfqpoint{4.647069in}{3.210123in}}%
\pgfusepath{stroke}%
\end{pgfscope}%
\end{pgfpicture}%
\makeatother%
\endgroup%

    \caption{Voltaje (V) frente a intensidad (I) con regresión lineal}
  \end{figure}

  A simple vista se puede observar que el ajuste es razonablemente preciso. Sin embargo, podemos obtener una medida fiable de cúan preciso es. Para ello utilizamos el coeficiente de regresión lineal (Ecuación \ref{ec:r}). En este caso, obtenemos que es $r = 0.999997$, un ajuste con cinco nueves, lo cual muestra una precision notable.


  \subsection{Circuito en Serie}

  La siguiente experiencia consiste en crear un circuito con tres resistencias ($R_1$, $R_2$ y $R_3$) en serie con el objetivo de medir la instensidad del circuito, y el potencial total y en cada resistencia. Para ello colocaremos los componentes de la siguiente manera:

  \begin{figure}[H]
    \centering
    \begin{circuitikz}[european]
      \draw (0,0) to[voltage source] (0,3)
      to[R=$R_1$] (3,3)
      to[R=$R_2$] (3,0)
      to[R=$R_3$] (0,0);
    \end{circuitikz}
    \caption{Circuito con tres resistencias en serie}
    \label{circuito:serie}
  \end{figure}

  \subsubsection{Procedimiento de medición}

  Para medir las diferentes magnitudes, colocaremos el polímetro en serie para las intensidades y en paralelo para los voltajes.

  \begin{figure}[H]
    \centering
    \begin{circuitikz}[european]
      \draw (0,0) to[voltage source] (0,2)
      to[R=$R_1$] (2,2)
      to[R=$R_2$] (2,0)
      to[R=$R_3$] (0,0);
      \draw (0,2) -- (0,3.5)
      to[voltmeter, l=$V_1$] (2,3.5) -- (2,2);
    \end{circuitikz}
    \raisebox{0.28in}{
    \begin{circuitikz}[european]
      \draw (0,0) to[voltage source] (0,2)
      to[R=$R_1$] (2,2)
      to[R=$R_2$] (2,0)
      to[R=$R_3$] (0,0);
      \draw (2,2) -- (3.5,2)
      to[voltmeter, l=$V_2$] (3.5,0) -- (2,0);
    \end{circuitikz}}
    \begin{circuitikz}[european]
      \draw (0,0) to[voltage source] (0,2)
      to[R=$R_1$] (2,2)
      to[R=$R_2$] (2,0)
      to[R=$R_3$] (0,0);
      \draw (2,2) -- (2,-1.5)
      to[voltmeter, l=$V_3$] (0,-1.5) -- (0,0);
    \end{circuitikz}
    \caption{Medición de potenciales de $R_1$, $R_2$ y $R_3$ respectivamente}
  \end{figure}

  \begin{figure}[H]
    \centering
    \begin{circuitikz}[european]
      \draw (0,0) to[voltage source] (0,2)
      to[R=$R_1$] (2,2)
      to[R=$R_2$] (4,2) -- (4,0)
      to[R=$R_3$] (2,0)
      to[ammeter] (0,0);
    \end{circuitikz}
    \caption{Medición de la intensidad total del circuito}
  \end{figure}

  \subsubsection{Resistencia equivalente}

  Primero debemos de averiguar cual es la resistencia equivalente de todo el circuito. Para ello utilizamos la siguiente fórmula, que se aplica a resistencias en serie:
  \begin{equation} \label{eq:reseq}
    R_S = \sum_{k=1}^N R_k \qquad R = R_1 + R_2 + R_3
  \end{equation}
  De aquí podemos deducir que la diferencia de potencial para cada resistencia será distinta, y que su suma ha de resultar en la total del circuito. En este caso:
  \begin{equation}
    V = \sum_{k=1}^N V_k \qquad V = V_1 + V_2 + V_3
  \end{equation}
  Por lo tanto, calculamos la resistencia equivalente del circuito aplicando la fórmula \ref{eq:reseq}:
  \begin{gather}
    R_S = 175500 + 216000 + 394000 = 785500 \Omega \qquad s(R_S) = \pm(100 + 1000 + 1000) = \pm2100 \Omega \nonumber \\ R_S = 785500 \pm 2100 \Omega \nonumber
  \end{gather}

  \subsubsection{Medición experimental}

  Siguiendo el procedimiento anterior, realizamos una serie de medidas en el circuito \ref{circuito:serie}. Iremos variando el potencial (\textit{V}) de la fuente y anotando los cambios del resto de magnitudes. Como especificamos en el apartado \ref{sec:incert} tomaremos la resolución de la medida como su incertidumbre.

  \begin{table}[H]
  \centering
  \resizebox{\columnwidth}{!}{
  \csvreader[
    tabular=|c|c|c|c|c|c|,
    table head=\hline Medida & $V~(V) \pm s(V)$ & $V_1~(V) \pm s(V_1)$ & $V_2~(V) \pm s(V_2)$ & $V_3~(V) \pm s(V_3)$ & $I~(A) \pm s(I)$ \\ \hline,
    late after last line=\\\hline,
    separator=semicolon
    ]{CC6.csv}
    {v=\v, v1=\va, v2=\vb, v3=\vc, i=\int, sv=\sv, sv1=\sva, sv2=\svb, sv3=\svc, si=\si}
    {\thecsvrow & \v \hspace{4pt}$\pm$ \sv & \va \hspace{4pt}$\pm$ \sva & \vb \hspace{4pt}$\pm$ \svb & \vc \hspace{4pt}$\pm$ \svc & \int \hspace{4pt}$\pm$ \si}
  }
  \caption{Potenciales e intensidades del circuito en serie}
  \end{table}

  \subsubsection{Representación gráfica de V frente a I}

  Utilizaremos el mismo programa de \code{python} que en el apartado anterior para representar nuestro voltaje (\textit{V}) frente a la intensidad (\textit{I}).

  \begin{figure}[H]
    %\centering
    \hspace{2.5em} %% Creator: Matplotlib, PGF backend
%%
%% To include the figure in your LaTeX document, write
%%   \input{<filename>.pgf}
%%
%% Make sure the required packages are loaded in your preamble
%%   \usepackage{pgf}
%%
%% Figures using additional raster images can only be included by \input if
%% they are in the same directory as the main LaTeX file. For loading figures
%% from other directories you can use the `import` package
%%   \usepackage{import}
%% and then include the figures with
%%   \import{<path to file>}{<filename>.pgf}
%%
%% Matplotlib used the following preamble
%%
\begingroup%
\makeatletter%
\begin{pgfpicture}%
\pgfpathrectangle{\pgfpointorigin}{\pgfqpoint{4.747069in}{3.310123in}}%
\pgfusepath{use as bounding box, clip}%
\begin{pgfscope}%
\pgfsetbuttcap%
\pgfsetmiterjoin%
\definecolor{currentfill}{rgb}{1.000000,1.000000,1.000000}%
\pgfsetfillcolor{currentfill}%
\pgfsetlinewidth{0.000000pt}%
\definecolor{currentstroke}{rgb}{1.000000,1.000000,1.000000}%
\pgfsetstrokecolor{currentstroke}%
\pgfsetdash{}{0pt}%
\pgfpathmoveto{\pgfqpoint{0.000000in}{0.000000in}}%
\pgfpathlineto{\pgfqpoint{4.747069in}{0.000000in}}%
\pgfpathlineto{\pgfqpoint{4.747069in}{3.310123in}}%
\pgfpathlineto{\pgfqpoint{0.000000in}{3.310123in}}%
\pgfpathclose%
\pgfusepath{fill}%
\end{pgfscope}%
\begin{pgfscope}%
\pgfsetbuttcap%
\pgfsetmiterjoin%
\definecolor{currentfill}{rgb}{1.000000,1.000000,1.000000}%
\pgfsetfillcolor{currentfill}%
\pgfsetlinewidth{0.000000pt}%
\definecolor{currentstroke}{rgb}{0.000000,0.000000,0.000000}%
\pgfsetstrokecolor{currentstroke}%
\pgfsetstrokeopacity{0.000000}%
\pgfsetdash{}{0pt}%
\pgfpathmoveto{\pgfqpoint{0.772069in}{0.515123in}}%
\pgfpathlineto{\pgfqpoint{4.647069in}{0.515123in}}%
\pgfpathlineto{\pgfqpoint{4.647069in}{3.210123in}}%
\pgfpathlineto{\pgfqpoint{0.772069in}{3.210123in}}%
\pgfpathclose%
\pgfusepath{fill}%
\end{pgfscope}%
\begin{pgfscope}%
\pgfsetbuttcap%
\pgfsetroundjoin%
\definecolor{currentfill}{rgb}{0.000000,0.000000,0.000000}%
\pgfsetfillcolor{currentfill}%
\pgfsetlinewidth{0.803000pt}%
\definecolor{currentstroke}{rgb}{0.000000,0.000000,0.000000}%
\pgfsetstrokecolor{currentstroke}%
\pgfsetdash{}{0pt}%
\pgfsys@defobject{currentmarker}{\pgfqpoint{0.000000in}{-0.048611in}}{\pgfqpoint{0.000000in}{0.000000in}}{%
\pgfpathmoveto{\pgfqpoint{0.000000in}{0.000000in}}%
\pgfpathlineto{\pgfqpoint{0.000000in}{-0.048611in}}%
\pgfusepath{stroke,fill}%
}%
\begin{pgfscope}%
\pgfsys@transformshift{1.166609in}{0.515123in}%
\pgfsys@useobject{currentmarker}{}%
\end{pgfscope}%
\end{pgfscope}%
\begin{pgfscope}%
\definecolor{textcolor}{rgb}{0.000000,0.000000,0.000000}%
\pgfsetstrokecolor{textcolor}%
\pgfsetfillcolor{textcolor}%
\pgftext[x=1.166609in,y=0.417901in,,top]{\color{textcolor}\rmfamily\fontsize{10.000000}{12.000000}\selectfont \(\displaystyle 2\)}%
\end{pgfscope}%
\begin{pgfscope}%
\pgfsetbuttcap%
\pgfsetroundjoin%
\definecolor{currentfill}{rgb}{0.000000,0.000000,0.000000}%
\pgfsetfillcolor{currentfill}%
\pgfsetlinewidth{0.803000pt}%
\definecolor{currentstroke}{rgb}{0.000000,0.000000,0.000000}%
\pgfsetstrokecolor{currentstroke}%
\pgfsetdash{}{0pt}%
\pgfsys@defobject{currentmarker}{\pgfqpoint{0.000000in}{-0.048611in}}{\pgfqpoint{0.000000in}{0.000000in}}{%
\pgfpathmoveto{\pgfqpoint{0.000000in}{0.000000in}}%
\pgfpathlineto{\pgfqpoint{0.000000in}{-0.048611in}}%
\pgfusepath{stroke,fill}%
}%
\begin{pgfscope}%
\pgfsys@transformshift{1.783793in}{0.515123in}%
\pgfsys@useobject{currentmarker}{}%
\end{pgfscope}%
\end{pgfscope}%
\begin{pgfscope}%
\definecolor{textcolor}{rgb}{0.000000,0.000000,0.000000}%
\pgfsetstrokecolor{textcolor}%
\pgfsetfillcolor{textcolor}%
\pgftext[x=1.783793in,y=0.417901in,,top]{\color{textcolor}\rmfamily\fontsize{10.000000}{12.000000}\selectfont \(\displaystyle 4\)}%
\end{pgfscope}%
\begin{pgfscope}%
\pgfsetbuttcap%
\pgfsetroundjoin%
\definecolor{currentfill}{rgb}{0.000000,0.000000,0.000000}%
\pgfsetfillcolor{currentfill}%
\pgfsetlinewidth{0.803000pt}%
\definecolor{currentstroke}{rgb}{0.000000,0.000000,0.000000}%
\pgfsetstrokecolor{currentstroke}%
\pgfsetdash{}{0pt}%
\pgfsys@defobject{currentmarker}{\pgfqpoint{0.000000in}{-0.048611in}}{\pgfqpoint{0.000000in}{0.000000in}}{%
\pgfpathmoveto{\pgfqpoint{0.000000in}{0.000000in}}%
\pgfpathlineto{\pgfqpoint{0.000000in}{-0.048611in}}%
\pgfusepath{stroke,fill}%
}%
\begin{pgfscope}%
\pgfsys@transformshift{2.400977in}{0.515123in}%
\pgfsys@useobject{currentmarker}{}%
\end{pgfscope}%
\end{pgfscope}%
\begin{pgfscope}%
\definecolor{textcolor}{rgb}{0.000000,0.000000,0.000000}%
\pgfsetstrokecolor{textcolor}%
\pgfsetfillcolor{textcolor}%
\pgftext[x=2.400977in,y=0.417901in,,top]{\color{textcolor}\rmfamily\fontsize{10.000000}{12.000000}\selectfont \(\displaystyle 6\)}%
\end{pgfscope}%
\begin{pgfscope}%
\pgfsetbuttcap%
\pgfsetroundjoin%
\definecolor{currentfill}{rgb}{0.000000,0.000000,0.000000}%
\pgfsetfillcolor{currentfill}%
\pgfsetlinewidth{0.803000pt}%
\definecolor{currentstroke}{rgb}{0.000000,0.000000,0.000000}%
\pgfsetstrokecolor{currentstroke}%
\pgfsetdash{}{0pt}%
\pgfsys@defobject{currentmarker}{\pgfqpoint{0.000000in}{-0.048611in}}{\pgfqpoint{0.000000in}{0.000000in}}{%
\pgfpathmoveto{\pgfqpoint{0.000000in}{0.000000in}}%
\pgfpathlineto{\pgfqpoint{0.000000in}{-0.048611in}}%
\pgfusepath{stroke,fill}%
}%
\begin{pgfscope}%
\pgfsys@transformshift{3.018161in}{0.515123in}%
\pgfsys@useobject{currentmarker}{}%
\end{pgfscope}%
\end{pgfscope}%
\begin{pgfscope}%
\definecolor{textcolor}{rgb}{0.000000,0.000000,0.000000}%
\pgfsetstrokecolor{textcolor}%
\pgfsetfillcolor{textcolor}%
\pgftext[x=3.018161in,y=0.417901in,,top]{\color{textcolor}\rmfamily\fontsize{10.000000}{12.000000}\selectfont \(\displaystyle 8\)}%
\end{pgfscope}%
\begin{pgfscope}%
\pgfsetbuttcap%
\pgfsetroundjoin%
\definecolor{currentfill}{rgb}{0.000000,0.000000,0.000000}%
\pgfsetfillcolor{currentfill}%
\pgfsetlinewidth{0.803000pt}%
\definecolor{currentstroke}{rgb}{0.000000,0.000000,0.000000}%
\pgfsetstrokecolor{currentstroke}%
\pgfsetdash{}{0pt}%
\pgfsys@defobject{currentmarker}{\pgfqpoint{0.000000in}{-0.048611in}}{\pgfqpoint{0.000000in}{0.000000in}}{%
\pgfpathmoveto{\pgfqpoint{0.000000in}{0.000000in}}%
\pgfpathlineto{\pgfqpoint{0.000000in}{-0.048611in}}%
\pgfusepath{stroke,fill}%
}%
\begin{pgfscope}%
\pgfsys@transformshift{3.635345in}{0.515123in}%
\pgfsys@useobject{currentmarker}{}%
\end{pgfscope}%
\end{pgfscope}%
\begin{pgfscope}%
\definecolor{textcolor}{rgb}{0.000000,0.000000,0.000000}%
\pgfsetstrokecolor{textcolor}%
\pgfsetfillcolor{textcolor}%
\pgftext[x=3.635345in,y=0.417901in,,top]{\color{textcolor}\rmfamily\fontsize{10.000000}{12.000000}\selectfont \(\displaystyle 10\)}%
\end{pgfscope}%
\begin{pgfscope}%
\pgfsetbuttcap%
\pgfsetroundjoin%
\definecolor{currentfill}{rgb}{0.000000,0.000000,0.000000}%
\pgfsetfillcolor{currentfill}%
\pgfsetlinewidth{0.803000pt}%
\definecolor{currentstroke}{rgb}{0.000000,0.000000,0.000000}%
\pgfsetstrokecolor{currentstroke}%
\pgfsetdash{}{0pt}%
\pgfsys@defobject{currentmarker}{\pgfqpoint{0.000000in}{-0.048611in}}{\pgfqpoint{0.000000in}{0.000000in}}{%
\pgfpathmoveto{\pgfqpoint{0.000000in}{0.000000in}}%
\pgfpathlineto{\pgfqpoint{0.000000in}{-0.048611in}}%
\pgfusepath{stroke,fill}%
}%
\begin{pgfscope}%
\pgfsys@transformshift{4.252529in}{0.515123in}%
\pgfsys@useobject{currentmarker}{}%
\end{pgfscope}%
\end{pgfscope}%
\begin{pgfscope}%
\definecolor{textcolor}{rgb}{0.000000,0.000000,0.000000}%
\pgfsetstrokecolor{textcolor}%
\pgfsetfillcolor{textcolor}%
\pgftext[x=4.252529in,y=0.417901in,,top]{\color{textcolor}\rmfamily\fontsize{10.000000}{12.000000}\selectfont \(\displaystyle 12\)}%
\end{pgfscope}%
\begin{pgfscope}%
\definecolor{textcolor}{rgb}{0.000000,0.000000,0.000000}%
\pgfsetstrokecolor{textcolor}%
\pgfsetfillcolor{textcolor}%
\pgftext[x=2.709569in,y=0.238889in,,top]{\color{textcolor}\rmfamily\fontsize{10.000000}{12.000000}\selectfont I(\(\displaystyle \mu\)A)}%
\end{pgfscope}%
\begin{pgfscope}%
\pgfsetbuttcap%
\pgfsetroundjoin%
\definecolor{currentfill}{rgb}{0.000000,0.000000,0.000000}%
\pgfsetfillcolor{currentfill}%
\pgfsetlinewidth{0.803000pt}%
\definecolor{currentstroke}{rgb}{0.000000,0.000000,0.000000}%
\pgfsetstrokecolor{currentstroke}%
\pgfsetdash{}{0pt}%
\pgfsys@defobject{currentmarker}{\pgfqpoint{-0.048611in}{0.000000in}}{\pgfqpoint{0.000000in}{0.000000in}}{%
\pgfpathmoveto{\pgfqpoint{0.000000in}{0.000000in}}%
\pgfpathlineto{\pgfqpoint{-0.048611in}{0.000000in}}%
\pgfusepath{stroke,fill}%
}%
\begin{pgfscope}%
\pgfsys@transformshift{0.772069in}{0.894605in}%
\pgfsys@useobject{currentmarker}{}%
\end{pgfscope}%
\end{pgfscope}%
\begin{pgfscope}%
\definecolor{textcolor}{rgb}{0.000000,0.000000,0.000000}%
\pgfsetstrokecolor{textcolor}%
\pgfsetfillcolor{textcolor}%
\pgftext[x=0.605402in,y=0.846379in,left,base]{\color{textcolor}\rmfamily\fontsize{10.000000}{12.000000}\selectfont \(\displaystyle 2\)}%
\end{pgfscope}%
\begin{pgfscope}%
\pgfsetbuttcap%
\pgfsetroundjoin%
\definecolor{currentfill}{rgb}{0.000000,0.000000,0.000000}%
\pgfsetfillcolor{currentfill}%
\pgfsetlinewidth{0.803000pt}%
\definecolor{currentstroke}{rgb}{0.000000,0.000000,0.000000}%
\pgfsetstrokecolor{currentstroke}%
\pgfsetdash{}{0pt}%
\pgfsys@defobject{currentmarker}{\pgfqpoint{-0.048611in}{0.000000in}}{\pgfqpoint{0.000000in}{0.000000in}}{%
\pgfpathmoveto{\pgfqpoint{0.000000in}{0.000000in}}%
\pgfpathlineto{\pgfqpoint{-0.048611in}{0.000000in}}%
\pgfusepath{stroke,fill}%
}%
\begin{pgfscope}%
\pgfsys@transformshift{0.772069in}{1.436684in}%
\pgfsys@useobject{currentmarker}{}%
\end{pgfscope}%
\end{pgfscope}%
\begin{pgfscope}%
\definecolor{textcolor}{rgb}{0.000000,0.000000,0.000000}%
\pgfsetstrokecolor{textcolor}%
\pgfsetfillcolor{textcolor}%
\pgftext[x=0.605402in,y=1.388459in,left,base]{\color{textcolor}\rmfamily\fontsize{10.000000}{12.000000}\selectfont \(\displaystyle 4\)}%
\end{pgfscope}%
\begin{pgfscope}%
\pgfsetbuttcap%
\pgfsetroundjoin%
\definecolor{currentfill}{rgb}{0.000000,0.000000,0.000000}%
\pgfsetfillcolor{currentfill}%
\pgfsetlinewidth{0.803000pt}%
\definecolor{currentstroke}{rgb}{0.000000,0.000000,0.000000}%
\pgfsetstrokecolor{currentstroke}%
\pgfsetdash{}{0pt}%
\pgfsys@defobject{currentmarker}{\pgfqpoint{-0.048611in}{0.000000in}}{\pgfqpoint{0.000000in}{0.000000in}}{%
\pgfpathmoveto{\pgfqpoint{0.000000in}{0.000000in}}%
\pgfpathlineto{\pgfqpoint{-0.048611in}{0.000000in}}%
\pgfusepath{stroke,fill}%
}%
\begin{pgfscope}%
\pgfsys@transformshift{0.772069in}{1.978764in}%
\pgfsys@useobject{currentmarker}{}%
\end{pgfscope}%
\end{pgfscope}%
\begin{pgfscope}%
\definecolor{textcolor}{rgb}{0.000000,0.000000,0.000000}%
\pgfsetstrokecolor{textcolor}%
\pgfsetfillcolor{textcolor}%
\pgftext[x=0.605402in,y=1.930539in,left,base]{\color{textcolor}\rmfamily\fontsize{10.000000}{12.000000}\selectfont \(\displaystyle 6\)}%
\end{pgfscope}%
\begin{pgfscope}%
\pgfsetbuttcap%
\pgfsetroundjoin%
\definecolor{currentfill}{rgb}{0.000000,0.000000,0.000000}%
\pgfsetfillcolor{currentfill}%
\pgfsetlinewidth{0.803000pt}%
\definecolor{currentstroke}{rgb}{0.000000,0.000000,0.000000}%
\pgfsetstrokecolor{currentstroke}%
\pgfsetdash{}{0pt}%
\pgfsys@defobject{currentmarker}{\pgfqpoint{-0.048611in}{0.000000in}}{\pgfqpoint{0.000000in}{0.000000in}}{%
\pgfpathmoveto{\pgfqpoint{0.000000in}{0.000000in}}%
\pgfpathlineto{\pgfqpoint{-0.048611in}{0.000000in}}%
\pgfusepath{stroke,fill}%
}%
\begin{pgfscope}%
\pgfsys@transformshift{0.772069in}{2.520843in}%
\pgfsys@useobject{currentmarker}{}%
\end{pgfscope}%
\end{pgfscope}%
\begin{pgfscope}%
\definecolor{textcolor}{rgb}{0.000000,0.000000,0.000000}%
\pgfsetstrokecolor{textcolor}%
\pgfsetfillcolor{textcolor}%
\pgftext[x=0.605402in,y=2.472618in,left,base]{\color{textcolor}\rmfamily\fontsize{10.000000}{12.000000}\selectfont \(\displaystyle 8\)}%
\end{pgfscope}%
\begin{pgfscope}%
\pgfsetbuttcap%
\pgfsetroundjoin%
\definecolor{currentfill}{rgb}{0.000000,0.000000,0.000000}%
\pgfsetfillcolor{currentfill}%
\pgfsetlinewidth{0.803000pt}%
\definecolor{currentstroke}{rgb}{0.000000,0.000000,0.000000}%
\pgfsetstrokecolor{currentstroke}%
\pgfsetdash{}{0pt}%
\pgfsys@defobject{currentmarker}{\pgfqpoint{-0.048611in}{0.000000in}}{\pgfqpoint{0.000000in}{0.000000in}}{%
\pgfpathmoveto{\pgfqpoint{0.000000in}{0.000000in}}%
\pgfpathlineto{\pgfqpoint{-0.048611in}{0.000000in}}%
\pgfusepath{stroke,fill}%
}%
\begin{pgfscope}%
\pgfsys@transformshift{0.772069in}{3.062923in}%
\pgfsys@useobject{currentmarker}{}%
\end{pgfscope}%
\end{pgfscope}%
\begin{pgfscope}%
\definecolor{textcolor}{rgb}{0.000000,0.000000,0.000000}%
\pgfsetstrokecolor{textcolor}%
\pgfsetfillcolor{textcolor}%
\pgftext[x=0.535957in,y=3.014698in,left,base]{\color{textcolor}\rmfamily\fontsize{10.000000}{12.000000}\selectfont \(\displaystyle 10\)}%
\end{pgfscope}%
\begin{pgfscope}%
\definecolor{textcolor}{rgb}{0.000000,0.000000,0.000000}%
\pgfsetstrokecolor{textcolor}%
\pgfsetfillcolor{textcolor}%
\pgftext[x=0.258179in,y=1.862623in,,bottom]{\color{textcolor}\rmfamily\fontsize{10.000000}{12.000000}\selectfont V(V)}%
\end{pgfscope}%
\begin{pgfscope}%
\pgfpathrectangle{\pgfqpoint{0.772069in}{0.515123in}}{\pgfqpoint{3.875000in}{2.695000in}}%
\pgfusepath{clip}%
\pgfsetbuttcap%
\pgfsetroundjoin%
\definecolor{currentfill}{rgb}{0.121569,0.466667,0.705882}%
\pgfsetfillcolor{currentfill}%
\pgfsetlinewidth{1.003750pt}%
\definecolor{currentstroke}{rgb}{0.121569,0.466667,0.705882}%
\pgfsetstrokecolor{currentstroke}%
\pgfsetdash{}{0pt}%
\pgfpathmoveto{\pgfqpoint{0.950594in}{0.598974in}}%
\pgfpathcurveto{\pgfqpoint{0.961644in}{0.598974in}}{\pgfqpoint{0.972243in}{0.603364in}}{\pgfqpoint{0.980057in}{0.611178in}}%
\pgfpathcurveto{\pgfqpoint{0.987871in}{0.618991in}}{\pgfqpoint{0.992261in}{0.629590in}}{\pgfqpoint{0.992261in}{0.640640in}}%
\pgfpathcurveto{\pgfqpoint{0.992261in}{0.651691in}}{\pgfqpoint{0.987871in}{0.662290in}}{\pgfqpoint{0.980057in}{0.670103in}}%
\pgfpathcurveto{\pgfqpoint{0.972243in}{0.677917in}}{\pgfqpoint{0.961644in}{0.682307in}}{\pgfqpoint{0.950594in}{0.682307in}}%
\pgfpathcurveto{\pgfqpoint{0.939544in}{0.682307in}}{\pgfqpoint{0.928945in}{0.677917in}}{\pgfqpoint{0.921131in}{0.670103in}}%
\pgfpathcurveto{\pgfqpoint{0.913318in}{0.662290in}}{\pgfqpoint{0.908927in}{0.651691in}}{\pgfqpoint{0.908927in}{0.640640in}}%
\pgfpathcurveto{\pgfqpoint{0.908927in}{0.629590in}}{\pgfqpoint{0.913318in}{0.618991in}}{\pgfqpoint{0.921131in}{0.611178in}}%
\pgfpathcurveto{\pgfqpoint{0.928945in}{0.603364in}}{\pgfqpoint{0.939544in}{0.598974in}}{\pgfqpoint{0.950594in}{0.598974in}}%
\pgfpathclose%
\pgfusepath{stroke,fill}%
\end{pgfscope}%
\begin{pgfscope}%
\pgfpathrectangle{\pgfqpoint{0.772069in}{0.515123in}}{\pgfqpoint{3.875000in}{2.695000in}}%
\pgfusepath{clip}%
\pgfsetbuttcap%
\pgfsetroundjoin%
\definecolor{currentfill}{rgb}{0.121569,0.466667,0.705882}%
\pgfsetfillcolor{currentfill}%
\pgfsetlinewidth{1.003750pt}%
\definecolor{currentstroke}{rgb}{0.121569,0.466667,0.705882}%
\pgfsetstrokecolor{currentstroke}%
\pgfsetdash{}{0pt}%
\pgfpathmoveto{\pgfqpoint{1.320905in}{0.866490in}}%
\pgfpathcurveto{\pgfqpoint{1.331955in}{0.866490in}}{\pgfqpoint{1.342554in}{0.870880in}}{\pgfqpoint{1.350367in}{0.878694in}}%
\pgfpathcurveto{\pgfqpoint{1.358181in}{0.886508in}}{\pgfqpoint{1.362571in}{0.897107in}}{\pgfqpoint{1.362571in}{0.908157in}}%
\pgfpathcurveto{\pgfqpoint{1.362571in}{0.919207in}}{\pgfqpoint{1.358181in}{0.929806in}}{\pgfqpoint{1.350367in}{0.937620in}}%
\pgfpathcurveto{\pgfqpoint{1.342554in}{0.945433in}}{\pgfqpoint{1.331955in}{0.949823in}}{\pgfqpoint{1.320905in}{0.949823in}}%
\pgfpathcurveto{\pgfqpoint{1.309854in}{0.949823in}}{\pgfqpoint{1.299255in}{0.945433in}}{\pgfqpoint{1.291442in}{0.937620in}}%
\pgfpathcurveto{\pgfqpoint{1.283628in}{0.929806in}}{\pgfqpoint{1.279238in}{0.919207in}}{\pgfqpoint{1.279238in}{0.908157in}}%
\pgfpathcurveto{\pgfqpoint{1.279238in}{0.897107in}}{\pgfqpoint{1.283628in}{0.886508in}}{\pgfqpoint{1.291442in}{0.878694in}}%
\pgfpathcurveto{\pgfqpoint{1.299255in}{0.870880in}}{\pgfqpoint{1.309854in}{0.866490in}}{\pgfqpoint{1.320905in}{0.866490in}}%
\pgfpathclose%
\pgfusepath{stroke,fill}%
\end{pgfscope}%
\begin{pgfscope}%
\pgfpathrectangle{\pgfqpoint{0.772069in}{0.515123in}}{\pgfqpoint{3.875000in}{2.695000in}}%
\pgfusepath{clip}%
\pgfsetbuttcap%
\pgfsetroundjoin%
\definecolor{currentfill}{rgb}{0.121569,0.466667,0.705882}%
\pgfsetfillcolor{currentfill}%
\pgfsetlinewidth{1.003750pt}%
\definecolor{currentstroke}{rgb}{0.121569,0.466667,0.705882}%
\pgfsetstrokecolor{currentstroke}%
\pgfsetdash{}{0pt}%
\pgfpathmoveto{\pgfqpoint{1.722074in}{1.134819in}}%
\pgfpathcurveto{\pgfqpoint{1.733124in}{1.134819in}}{\pgfqpoint{1.743723in}{1.139210in}}{\pgfqpoint{1.751537in}{1.147023in}}%
\pgfpathcurveto{\pgfqpoint{1.759351in}{1.154837in}}{\pgfqpoint{1.763741in}{1.165436in}}{\pgfqpoint{1.763741in}{1.176486in}}%
\pgfpathcurveto{\pgfqpoint{1.763741in}{1.187536in}}{\pgfqpoint{1.759351in}{1.198135in}}{\pgfqpoint{1.751537in}{1.205949in}}%
\pgfpathcurveto{\pgfqpoint{1.743723in}{1.213763in}}{\pgfqpoint{1.733124in}{1.218153in}}{\pgfqpoint{1.722074in}{1.218153in}}%
\pgfpathcurveto{\pgfqpoint{1.711024in}{1.218153in}}{\pgfqpoint{1.700425in}{1.213763in}}{\pgfqpoint{1.692611in}{1.205949in}}%
\pgfpathcurveto{\pgfqpoint{1.684798in}{1.198135in}}{\pgfqpoint{1.680408in}{1.187536in}}{\pgfqpoint{1.680408in}{1.176486in}}%
\pgfpathcurveto{\pgfqpoint{1.680408in}{1.165436in}}{\pgfqpoint{1.684798in}{1.154837in}}{\pgfqpoint{1.692611in}{1.147023in}}%
\pgfpathcurveto{\pgfqpoint{1.700425in}{1.139210in}}{\pgfqpoint{1.711024in}{1.134819in}}{\pgfqpoint{1.722074in}{1.134819in}}%
\pgfpathclose%
\pgfusepath{stroke,fill}%
\end{pgfscope}%
\begin{pgfscope}%
\pgfpathrectangle{\pgfqpoint{0.772069in}{0.515123in}}{\pgfqpoint{3.875000in}{2.695000in}}%
\pgfusepath{clip}%
\pgfsetbuttcap%
\pgfsetroundjoin%
\definecolor{currentfill}{rgb}{0.121569,0.466667,0.705882}%
\pgfsetfillcolor{currentfill}%
\pgfsetlinewidth{1.003750pt}%
\definecolor{currentstroke}{rgb}{0.121569,0.466667,0.705882}%
\pgfsetstrokecolor{currentstroke}%
\pgfsetdash{}{0pt}%
\pgfpathmoveto{\pgfqpoint{2.123244in}{1.411280in}}%
\pgfpathcurveto{\pgfqpoint{2.134294in}{1.411280in}}{\pgfqpoint{2.144893in}{1.415670in}}{\pgfqpoint{2.152707in}{1.423484in}}%
\pgfpathcurveto{\pgfqpoint{2.160520in}{1.431298in}}{\pgfqpoint{2.164911in}{1.441897in}}{\pgfqpoint{2.164911in}{1.452947in}}%
\pgfpathcurveto{\pgfqpoint{2.164911in}{1.463997in}}{\pgfqpoint{2.160520in}{1.474596in}}{\pgfqpoint{2.152707in}{1.482409in}}%
\pgfpathcurveto{\pgfqpoint{2.144893in}{1.490223in}}{\pgfqpoint{2.134294in}{1.494613in}}{\pgfqpoint{2.123244in}{1.494613in}}%
\pgfpathcurveto{\pgfqpoint{2.112194in}{1.494613in}}{\pgfqpoint{2.101595in}{1.490223in}}{\pgfqpoint{2.093781in}{1.482409in}}%
\pgfpathcurveto{\pgfqpoint{2.085967in}{1.474596in}}{\pgfqpoint{2.081577in}{1.463997in}}{\pgfqpoint{2.081577in}{1.452947in}}%
\pgfpathcurveto{\pgfqpoint{2.081577in}{1.441897in}}{\pgfqpoint{2.085967in}{1.431298in}}{\pgfqpoint{2.093781in}{1.423484in}}%
\pgfpathcurveto{\pgfqpoint{2.101595in}{1.415670in}}{\pgfqpoint{2.112194in}{1.411280in}}{\pgfqpoint{2.123244in}{1.411280in}}%
\pgfpathclose%
\pgfusepath{stroke,fill}%
\end{pgfscope}%
\begin{pgfscope}%
\pgfpathrectangle{\pgfqpoint{0.772069in}{0.515123in}}{\pgfqpoint{3.875000in}{2.695000in}}%
\pgfusepath{clip}%
\pgfsetbuttcap%
\pgfsetroundjoin%
\definecolor{currentfill}{rgb}{0.121569,0.466667,0.705882}%
\pgfsetfillcolor{currentfill}%
\pgfsetlinewidth{1.003750pt}%
\definecolor{currentstroke}{rgb}{0.121569,0.466667,0.705882}%
\pgfsetstrokecolor{currentstroke}%
\pgfsetdash{}{0pt}%
\pgfpathmoveto{\pgfqpoint{2.493554in}{1.676899in}}%
\pgfpathcurveto{\pgfqpoint{2.504604in}{1.676899in}}{\pgfqpoint{2.515203in}{1.681289in}}{\pgfqpoint{2.523017in}{1.689103in}}%
\pgfpathcurveto{\pgfqpoint{2.530831in}{1.696917in}}{\pgfqpoint{2.535221in}{1.707516in}}{\pgfqpoint{2.535221in}{1.718566in}}%
\pgfpathcurveto{\pgfqpoint{2.535221in}{1.729616in}}{\pgfqpoint{2.530831in}{1.740215in}}{\pgfqpoint{2.523017in}{1.748028in}}%
\pgfpathcurveto{\pgfqpoint{2.515203in}{1.755842in}}{\pgfqpoint{2.504604in}{1.760232in}}{\pgfqpoint{2.493554in}{1.760232in}}%
\pgfpathcurveto{\pgfqpoint{2.482504in}{1.760232in}}{\pgfqpoint{2.471905in}{1.755842in}}{\pgfqpoint{2.464091in}{1.748028in}}%
\pgfpathcurveto{\pgfqpoint{2.456278in}{1.740215in}}{\pgfqpoint{2.451888in}{1.729616in}}{\pgfqpoint{2.451888in}{1.718566in}}%
\pgfpathcurveto{\pgfqpoint{2.451888in}{1.707516in}}{\pgfqpoint{2.456278in}{1.696917in}}{\pgfqpoint{2.464091in}{1.689103in}}%
\pgfpathcurveto{\pgfqpoint{2.471905in}{1.681289in}}{\pgfqpoint{2.482504in}{1.676899in}}{\pgfqpoint{2.493554in}{1.676899in}}%
\pgfpathclose%
\pgfusepath{stroke,fill}%
\end{pgfscope}%
\begin{pgfscope}%
\pgfpathrectangle{\pgfqpoint{0.772069in}{0.515123in}}{\pgfqpoint{3.875000in}{2.695000in}}%
\pgfusepath{clip}%
\pgfsetbuttcap%
\pgfsetroundjoin%
\definecolor{currentfill}{rgb}{0.121569,0.466667,0.705882}%
\pgfsetfillcolor{currentfill}%
\pgfsetlinewidth{1.003750pt}%
\definecolor{currentstroke}{rgb}{0.121569,0.466667,0.705882}%
\pgfsetstrokecolor{currentstroke}%
\pgfsetdash{}{0pt}%
\pgfpathmoveto{\pgfqpoint{2.925583in}{1.969622in}}%
\pgfpathcurveto{\pgfqpoint{2.936633in}{1.969622in}}{\pgfqpoint{2.947232in}{1.974012in}}{\pgfqpoint{2.955046in}{1.981826in}}%
\pgfpathcurveto{\pgfqpoint{2.962860in}{1.989639in}}{\pgfqpoint{2.967250in}{2.000239in}}{\pgfqpoint{2.967250in}{2.011289in}}%
\pgfpathcurveto{\pgfqpoint{2.967250in}{2.022339in}}{\pgfqpoint{2.962860in}{2.032938in}}{\pgfqpoint{2.955046in}{2.040751in}}%
\pgfpathcurveto{\pgfqpoint{2.947232in}{2.048565in}}{\pgfqpoint{2.936633in}{2.052955in}}{\pgfqpoint{2.925583in}{2.052955in}}%
\pgfpathcurveto{\pgfqpoint{2.914533in}{2.052955in}}{\pgfqpoint{2.903934in}{2.048565in}}{\pgfqpoint{2.896120in}{2.040751in}}%
\pgfpathcurveto{\pgfqpoint{2.888307in}{2.032938in}}{\pgfqpoint{2.883916in}{2.022339in}}{\pgfqpoint{2.883916in}{2.011289in}}%
\pgfpathcurveto{\pgfqpoint{2.883916in}{2.000239in}}{\pgfqpoint{2.888307in}{1.989639in}}{\pgfqpoint{2.896120in}{1.981826in}}%
\pgfpathcurveto{\pgfqpoint{2.903934in}{1.974012in}}{\pgfqpoint{2.914533in}{1.969622in}}{\pgfqpoint{2.925583in}{1.969622in}}%
\pgfpathclose%
\pgfusepath{stroke,fill}%
\end{pgfscope}%
\begin{pgfscope}%
\pgfpathrectangle{\pgfqpoint{0.772069in}{0.515123in}}{\pgfqpoint{3.875000in}{2.695000in}}%
\pgfusepath{clip}%
\pgfsetbuttcap%
\pgfsetroundjoin%
\definecolor{currentfill}{rgb}{0.121569,0.466667,0.705882}%
\pgfsetfillcolor{currentfill}%
\pgfsetlinewidth{1.003750pt}%
\definecolor{currentstroke}{rgb}{0.121569,0.466667,0.705882}%
\pgfsetstrokecolor{currentstroke}%
\pgfsetdash{}{0pt}%
\pgfpathmoveto{\pgfqpoint{3.295894in}{2.229820in}}%
\pgfpathcurveto{\pgfqpoint{3.306944in}{2.229820in}}{\pgfqpoint{3.317543in}{2.234210in}}{\pgfqpoint{3.325356in}{2.242024in}}%
\pgfpathcurveto{\pgfqpoint{3.333170in}{2.249838in}}{\pgfqpoint{3.337560in}{2.260437in}}{\pgfqpoint{3.337560in}{2.271487in}}%
\pgfpathcurveto{\pgfqpoint{3.337560in}{2.282537in}}{\pgfqpoint{3.333170in}{2.293136in}}{\pgfqpoint{3.325356in}{2.300950in}}%
\pgfpathcurveto{\pgfqpoint{3.317543in}{2.308763in}}{\pgfqpoint{3.306944in}{2.313154in}}{\pgfqpoint{3.295894in}{2.313154in}}%
\pgfpathcurveto{\pgfqpoint{3.284843in}{2.313154in}}{\pgfqpoint{3.274244in}{2.308763in}}{\pgfqpoint{3.266431in}{2.300950in}}%
\pgfpathcurveto{\pgfqpoint{3.258617in}{2.293136in}}{\pgfqpoint{3.254227in}{2.282537in}}{\pgfqpoint{3.254227in}{2.271487in}}%
\pgfpathcurveto{\pgfqpoint{3.254227in}{2.260437in}}{\pgfqpoint{3.258617in}{2.249838in}}{\pgfqpoint{3.266431in}{2.242024in}}%
\pgfpathcurveto{\pgfqpoint{3.274244in}{2.234210in}}{\pgfqpoint{3.284843in}{2.229820in}}{\pgfqpoint{3.295894in}{2.229820in}}%
\pgfpathclose%
\pgfusepath{stroke,fill}%
\end{pgfscope}%
\begin{pgfscope}%
\pgfpathrectangle{\pgfqpoint{0.772069in}{0.515123in}}{\pgfqpoint{3.875000in}{2.695000in}}%
\pgfusepath{clip}%
\pgfsetbuttcap%
\pgfsetroundjoin%
\definecolor{currentfill}{rgb}{0.121569,0.466667,0.705882}%
\pgfsetfillcolor{currentfill}%
\pgfsetlinewidth{1.003750pt}%
\definecolor{currentstroke}{rgb}{0.121569,0.466667,0.705882}%
\pgfsetstrokecolor{currentstroke}%
\pgfsetdash{}{0pt}%
\pgfpathmoveto{\pgfqpoint{3.697063in}{2.498150in}}%
\pgfpathcurveto{\pgfqpoint{3.708113in}{2.498150in}}{\pgfqpoint{3.718712in}{2.502540in}}{\pgfqpoint{3.726526in}{2.510353in}}%
\pgfpathcurveto{\pgfqpoint{3.734340in}{2.518167in}}{\pgfqpoint{3.738730in}{2.528766in}}{\pgfqpoint{3.738730in}{2.539816in}}%
\pgfpathcurveto{\pgfqpoint{3.738730in}{2.550866in}}{\pgfqpoint{3.734340in}{2.561465in}}{\pgfqpoint{3.726526in}{2.569279in}}%
\pgfpathcurveto{\pgfqpoint{3.718712in}{2.577093in}}{\pgfqpoint{3.708113in}{2.581483in}}{\pgfqpoint{3.697063in}{2.581483in}}%
\pgfpathcurveto{\pgfqpoint{3.686013in}{2.581483in}}{\pgfqpoint{3.675414in}{2.577093in}}{\pgfqpoint{3.667600in}{2.569279in}}%
\pgfpathcurveto{\pgfqpoint{3.659787in}{2.561465in}}{\pgfqpoint{3.655396in}{2.550866in}}{\pgfqpoint{3.655396in}{2.539816in}}%
\pgfpathcurveto{\pgfqpoint{3.655396in}{2.528766in}}{\pgfqpoint{3.659787in}{2.518167in}}{\pgfqpoint{3.667600in}{2.510353in}}%
\pgfpathcurveto{\pgfqpoint{3.675414in}{2.502540in}}{\pgfqpoint{3.686013in}{2.498150in}}{\pgfqpoint{3.697063in}{2.498150in}}%
\pgfpathclose%
\pgfusepath{stroke,fill}%
\end{pgfscope}%
\begin{pgfscope}%
\pgfpathrectangle{\pgfqpoint{0.772069in}{0.515123in}}{\pgfqpoint{3.875000in}{2.695000in}}%
\pgfusepath{clip}%
\pgfsetbuttcap%
\pgfsetroundjoin%
\definecolor{currentfill}{rgb}{0.121569,0.466667,0.705882}%
\pgfsetfillcolor{currentfill}%
\pgfsetlinewidth{1.003750pt}%
\definecolor{currentstroke}{rgb}{0.121569,0.466667,0.705882}%
\pgfsetstrokecolor{currentstroke}%
\pgfsetdash{}{0pt}%
\pgfpathmoveto{\pgfqpoint{4.036514in}{2.750217in}}%
\pgfpathcurveto{\pgfqpoint{4.047565in}{2.750217in}}{\pgfqpoint{4.058164in}{2.754607in}}{\pgfqpoint{4.065977in}{2.762420in}}%
\pgfpathcurveto{\pgfqpoint{4.073791in}{2.770234in}}{\pgfqpoint{4.078181in}{2.780833in}}{\pgfqpoint{4.078181in}{2.791883in}}%
\pgfpathcurveto{\pgfqpoint{4.078181in}{2.802933in}}{\pgfqpoint{4.073791in}{2.813532in}}{\pgfqpoint{4.065977in}{2.821346in}}%
\pgfpathcurveto{\pgfqpoint{4.058164in}{2.829160in}}{\pgfqpoint{4.047565in}{2.833550in}}{\pgfqpoint{4.036514in}{2.833550in}}%
\pgfpathcurveto{\pgfqpoint{4.025464in}{2.833550in}}{\pgfqpoint{4.014865in}{2.829160in}}{\pgfqpoint{4.007052in}{2.821346in}}%
\pgfpathcurveto{\pgfqpoint{3.999238in}{2.813532in}}{\pgfqpoint{3.994848in}{2.802933in}}{\pgfqpoint{3.994848in}{2.791883in}}%
\pgfpathcurveto{\pgfqpoint{3.994848in}{2.780833in}}{\pgfqpoint{3.999238in}{2.770234in}}{\pgfqpoint{4.007052in}{2.762420in}}%
\pgfpathcurveto{\pgfqpoint{4.014865in}{2.754607in}}{\pgfqpoint{4.025464in}{2.750217in}}{\pgfqpoint{4.036514in}{2.750217in}}%
\pgfpathclose%
\pgfusepath{stroke,fill}%
\end{pgfscope}%
\begin{pgfscope}%
\pgfpathrectangle{\pgfqpoint{0.772069in}{0.515123in}}{\pgfqpoint{3.875000in}{2.695000in}}%
\pgfusepath{clip}%
\pgfsetbuttcap%
\pgfsetroundjoin%
\definecolor{currentfill}{rgb}{0.121569,0.466667,0.705882}%
\pgfsetfillcolor{currentfill}%
\pgfsetlinewidth{1.003750pt}%
\definecolor{currentstroke}{rgb}{0.121569,0.466667,0.705882}%
\pgfsetstrokecolor{currentstroke}%
\pgfsetdash{}{0pt}%
\pgfpathmoveto{\pgfqpoint{4.468543in}{3.042940in}}%
\pgfpathcurveto{\pgfqpoint{4.479593in}{3.042940in}}{\pgfqpoint{4.490192in}{3.047330in}}{\pgfqpoint{4.498006in}{3.055143in}}%
\pgfpathcurveto{\pgfqpoint{4.505820in}{3.062957in}}{\pgfqpoint{4.510210in}{3.073556in}}{\pgfqpoint{4.510210in}{3.084606in}}%
\pgfpathcurveto{\pgfqpoint{4.510210in}{3.095656in}}{\pgfqpoint{4.505820in}{3.106255in}}{\pgfqpoint{4.498006in}{3.114069in}}%
\pgfpathcurveto{\pgfqpoint{4.490192in}{3.121883in}}{\pgfqpoint{4.479593in}{3.126273in}}{\pgfqpoint{4.468543in}{3.126273in}}%
\pgfpathcurveto{\pgfqpoint{4.457493in}{3.126273in}}{\pgfqpoint{4.446894in}{3.121883in}}{\pgfqpoint{4.439080in}{3.114069in}}%
\pgfpathcurveto{\pgfqpoint{4.431267in}{3.106255in}}{\pgfqpoint{4.426877in}{3.095656in}}{\pgfqpoint{4.426877in}{3.084606in}}%
\pgfpathcurveto{\pgfqpoint{4.426877in}{3.073556in}}{\pgfqpoint{4.431267in}{3.062957in}}{\pgfqpoint{4.439080in}{3.055143in}}%
\pgfpathcurveto{\pgfqpoint{4.446894in}{3.047330in}}{\pgfqpoint{4.457493in}{3.042940in}}{\pgfqpoint{4.468543in}{3.042940in}}%
\pgfpathclose%
\pgfusepath{stroke,fill}%
\end{pgfscope}%
\begin{pgfscope}%
\pgfsetrectcap%
\pgfsetmiterjoin%
\pgfsetlinewidth{0.803000pt}%
\definecolor{currentstroke}{rgb}{0.000000,0.000000,0.000000}%
\pgfsetstrokecolor{currentstroke}%
\pgfsetdash{}{0pt}%
\pgfpathmoveto{\pgfqpoint{0.772069in}{0.515123in}}%
\pgfpathlineto{\pgfqpoint{0.772069in}{3.210123in}}%
\pgfusepath{stroke}%
\end{pgfscope}%
\begin{pgfscope}%
\pgfsetrectcap%
\pgfsetmiterjoin%
\pgfsetlinewidth{0.803000pt}%
\definecolor{currentstroke}{rgb}{0.000000,0.000000,0.000000}%
\pgfsetstrokecolor{currentstroke}%
\pgfsetdash{}{0pt}%
\pgfpathmoveto{\pgfqpoint{4.647069in}{0.515123in}}%
\pgfpathlineto{\pgfqpoint{4.647069in}{3.210123in}}%
\pgfusepath{stroke}%
\end{pgfscope}%
\begin{pgfscope}%
\pgfsetrectcap%
\pgfsetmiterjoin%
\pgfsetlinewidth{0.803000pt}%
\definecolor{currentstroke}{rgb}{0.000000,0.000000,0.000000}%
\pgfsetstrokecolor{currentstroke}%
\pgfsetdash{}{0pt}%
\pgfpathmoveto{\pgfqpoint{0.772069in}{0.515123in}}%
\pgfpathlineto{\pgfqpoint{4.647069in}{0.515123in}}%
\pgfusepath{stroke}%
\end{pgfscope}%
\begin{pgfscope}%
\pgfsetrectcap%
\pgfsetmiterjoin%
\pgfsetlinewidth{0.803000pt}%
\definecolor{currentstroke}{rgb}{0.000000,0.000000,0.000000}%
\pgfsetstrokecolor{currentstroke}%
\pgfsetdash{}{0pt}%
\pgfpathmoveto{\pgfqpoint{0.772069in}{3.210123in}}%
\pgfpathlineto{\pgfqpoint{4.647069in}{3.210123in}}%
\pgfusepath{stroke}%
\end{pgfscope}%
\end{pgfpicture}%
\makeatother%
\endgroup%

    \caption{Voltaje (V) frente a intensidad (I)}
  \end{figure}

  También podemos comparar la diferencia entre los potenciales de las distintas resistencias en serie añadiendo las gráficas de $V_1$, $V_2$ y $V_3$ frente a I.

  \begin{figure}[H]
    %\centering
    \hspace{2.5em} %% Creator: Matplotlib, PGF backend
%%
%% To include the figure in your LaTeX document, write
%%   \input{<filename>.pgf}
%%
%% Make sure the required packages are loaded in your preamble
%%   \usepackage{pgf}
%%
%% Figures using additional raster images can only be included by \input if
%% they are in the same directory as the main LaTeX file. For loading figures
%% from other directories you can use the `import` package
%%   \usepackage{import}
%% and then include the figures with
%%   \import{<path to file>}{<filename>.pgf}
%%
%% Matplotlib used the following preamble
%%
\begingroup%
\makeatletter%
\begin{pgfpicture}%
\pgfpathrectangle{\pgfpointorigin}{\pgfqpoint{4.747069in}{3.310123in}}%
\pgfusepath{use as bounding box, clip}%
\begin{pgfscope}%
\pgfsetbuttcap%
\pgfsetmiterjoin%
\definecolor{currentfill}{rgb}{1.000000,1.000000,1.000000}%
\pgfsetfillcolor{currentfill}%
\pgfsetlinewidth{0.000000pt}%
\definecolor{currentstroke}{rgb}{1.000000,1.000000,1.000000}%
\pgfsetstrokecolor{currentstroke}%
\pgfsetdash{}{0pt}%
\pgfpathmoveto{\pgfqpoint{0.000000in}{0.000000in}}%
\pgfpathlineto{\pgfqpoint{4.747069in}{0.000000in}}%
\pgfpathlineto{\pgfqpoint{4.747069in}{3.310123in}}%
\pgfpathlineto{\pgfqpoint{0.000000in}{3.310123in}}%
\pgfpathclose%
\pgfusepath{fill}%
\end{pgfscope}%
\begin{pgfscope}%
\pgfsetbuttcap%
\pgfsetmiterjoin%
\definecolor{currentfill}{rgb}{1.000000,1.000000,1.000000}%
\pgfsetfillcolor{currentfill}%
\pgfsetlinewidth{0.000000pt}%
\definecolor{currentstroke}{rgb}{0.000000,0.000000,0.000000}%
\pgfsetstrokecolor{currentstroke}%
\pgfsetstrokeopacity{0.000000}%
\pgfsetdash{}{0pt}%
\pgfpathmoveto{\pgfqpoint{0.772069in}{0.515123in}}%
\pgfpathlineto{\pgfqpoint{4.647069in}{0.515123in}}%
\pgfpathlineto{\pgfqpoint{4.647069in}{3.210123in}}%
\pgfpathlineto{\pgfqpoint{0.772069in}{3.210123in}}%
\pgfpathclose%
\pgfusepath{fill}%
\end{pgfscope}%
\begin{pgfscope}%
\pgfsetbuttcap%
\pgfsetroundjoin%
\definecolor{currentfill}{rgb}{0.000000,0.000000,0.000000}%
\pgfsetfillcolor{currentfill}%
\pgfsetlinewidth{0.803000pt}%
\definecolor{currentstroke}{rgb}{0.000000,0.000000,0.000000}%
\pgfsetstrokecolor{currentstroke}%
\pgfsetdash{}{0pt}%
\pgfsys@defobject{currentmarker}{\pgfqpoint{0.000000in}{-0.048611in}}{\pgfqpoint{0.000000in}{0.000000in}}{%
\pgfpathmoveto{\pgfqpoint{0.000000in}{0.000000in}}%
\pgfpathlineto{\pgfqpoint{0.000000in}{-0.048611in}}%
\pgfusepath{stroke,fill}%
}%
\begin{pgfscope}%
\pgfsys@transformshift{1.190829in}{0.515123in}%
\pgfsys@useobject{currentmarker}{}%
\end{pgfscope}%
\end{pgfscope}%
\begin{pgfscope}%
\definecolor{textcolor}{rgb}{0.000000,0.000000,0.000000}%
\pgfsetstrokecolor{textcolor}%
\pgfsetfillcolor{textcolor}%
\pgftext[x=1.190829in,y=0.417901in,,top]{\color{textcolor}\rmfamily\fontsize{10.000000}{12.000000}\selectfont \(\displaystyle 2\)}%
\end{pgfscope}%
\begin{pgfscope}%
\pgfsetbuttcap%
\pgfsetroundjoin%
\definecolor{currentfill}{rgb}{0.000000,0.000000,0.000000}%
\pgfsetfillcolor{currentfill}%
\pgfsetlinewidth{0.803000pt}%
\definecolor{currentstroke}{rgb}{0.000000,0.000000,0.000000}%
\pgfsetstrokecolor{currentstroke}%
\pgfsetdash{}{0pt}%
\pgfsys@defobject{currentmarker}{\pgfqpoint{0.000000in}{-0.048611in}}{\pgfqpoint{0.000000in}{0.000000in}}{%
\pgfpathmoveto{\pgfqpoint{0.000000in}{0.000000in}}%
\pgfpathlineto{\pgfqpoint{0.000000in}{-0.048611in}}%
\pgfusepath{stroke,fill}%
}%
\begin{pgfscope}%
\pgfsys@transformshift{1.798325in}{0.515123in}%
\pgfsys@useobject{currentmarker}{}%
\end{pgfscope}%
\end{pgfscope}%
\begin{pgfscope}%
\definecolor{textcolor}{rgb}{0.000000,0.000000,0.000000}%
\pgfsetstrokecolor{textcolor}%
\pgfsetfillcolor{textcolor}%
\pgftext[x=1.798325in,y=0.417901in,,top]{\color{textcolor}\rmfamily\fontsize{10.000000}{12.000000}\selectfont \(\displaystyle 4\)}%
\end{pgfscope}%
\begin{pgfscope}%
\pgfsetbuttcap%
\pgfsetroundjoin%
\definecolor{currentfill}{rgb}{0.000000,0.000000,0.000000}%
\pgfsetfillcolor{currentfill}%
\pgfsetlinewidth{0.803000pt}%
\definecolor{currentstroke}{rgb}{0.000000,0.000000,0.000000}%
\pgfsetstrokecolor{currentstroke}%
\pgfsetdash{}{0pt}%
\pgfsys@defobject{currentmarker}{\pgfqpoint{0.000000in}{-0.048611in}}{\pgfqpoint{0.000000in}{0.000000in}}{%
\pgfpathmoveto{\pgfqpoint{0.000000in}{0.000000in}}%
\pgfpathlineto{\pgfqpoint{0.000000in}{-0.048611in}}%
\pgfusepath{stroke,fill}%
}%
\begin{pgfscope}%
\pgfsys@transformshift{2.405821in}{0.515123in}%
\pgfsys@useobject{currentmarker}{}%
\end{pgfscope}%
\end{pgfscope}%
\begin{pgfscope}%
\definecolor{textcolor}{rgb}{0.000000,0.000000,0.000000}%
\pgfsetstrokecolor{textcolor}%
\pgfsetfillcolor{textcolor}%
\pgftext[x=2.405821in,y=0.417901in,,top]{\color{textcolor}\rmfamily\fontsize{10.000000}{12.000000}\selectfont \(\displaystyle 6\)}%
\end{pgfscope}%
\begin{pgfscope}%
\pgfsetbuttcap%
\pgfsetroundjoin%
\definecolor{currentfill}{rgb}{0.000000,0.000000,0.000000}%
\pgfsetfillcolor{currentfill}%
\pgfsetlinewidth{0.803000pt}%
\definecolor{currentstroke}{rgb}{0.000000,0.000000,0.000000}%
\pgfsetstrokecolor{currentstroke}%
\pgfsetdash{}{0pt}%
\pgfsys@defobject{currentmarker}{\pgfqpoint{0.000000in}{-0.048611in}}{\pgfqpoint{0.000000in}{0.000000in}}{%
\pgfpathmoveto{\pgfqpoint{0.000000in}{0.000000in}}%
\pgfpathlineto{\pgfqpoint{0.000000in}{-0.048611in}}%
\pgfusepath{stroke,fill}%
}%
\begin{pgfscope}%
\pgfsys@transformshift{3.013317in}{0.515123in}%
\pgfsys@useobject{currentmarker}{}%
\end{pgfscope}%
\end{pgfscope}%
\begin{pgfscope}%
\definecolor{textcolor}{rgb}{0.000000,0.000000,0.000000}%
\pgfsetstrokecolor{textcolor}%
\pgfsetfillcolor{textcolor}%
\pgftext[x=3.013317in,y=0.417901in,,top]{\color{textcolor}\rmfamily\fontsize{10.000000}{12.000000}\selectfont \(\displaystyle 8\)}%
\end{pgfscope}%
\begin{pgfscope}%
\pgfsetbuttcap%
\pgfsetroundjoin%
\definecolor{currentfill}{rgb}{0.000000,0.000000,0.000000}%
\pgfsetfillcolor{currentfill}%
\pgfsetlinewidth{0.803000pt}%
\definecolor{currentstroke}{rgb}{0.000000,0.000000,0.000000}%
\pgfsetstrokecolor{currentstroke}%
\pgfsetdash{}{0pt}%
\pgfsys@defobject{currentmarker}{\pgfqpoint{0.000000in}{-0.048611in}}{\pgfqpoint{0.000000in}{0.000000in}}{%
\pgfpathmoveto{\pgfqpoint{0.000000in}{0.000000in}}%
\pgfpathlineto{\pgfqpoint{0.000000in}{-0.048611in}}%
\pgfusepath{stroke,fill}%
}%
\begin{pgfscope}%
\pgfsys@transformshift{3.620813in}{0.515123in}%
\pgfsys@useobject{currentmarker}{}%
\end{pgfscope}%
\end{pgfscope}%
\begin{pgfscope}%
\definecolor{textcolor}{rgb}{0.000000,0.000000,0.000000}%
\pgfsetstrokecolor{textcolor}%
\pgfsetfillcolor{textcolor}%
\pgftext[x=3.620813in,y=0.417901in,,top]{\color{textcolor}\rmfamily\fontsize{10.000000}{12.000000}\selectfont \(\displaystyle 10\)}%
\end{pgfscope}%
\begin{pgfscope}%
\pgfsetbuttcap%
\pgfsetroundjoin%
\definecolor{currentfill}{rgb}{0.000000,0.000000,0.000000}%
\pgfsetfillcolor{currentfill}%
\pgfsetlinewidth{0.803000pt}%
\definecolor{currentstroke}{rgb}{0.000000,0.000000,0.000000}%
\pgfsetstrokecolor{currentstroke}%
\pgfsetdash{}{0pt}%
\pgfsys@defobject{currentmarker}{\pgfqpoint{0.000000in}{-0.048611in}}{\pgfqpoint{0.000000in}{0.000000in}}{%
\pgfpathmoveto{\pgfqpoint{0.000000in}{0.000000in}}%
\pgfpathlineto{\pgfqpoint{0.000000in}{-0.048611in}}%
\pgfusepath{stroke,fill}%
}%
\begin{pgfscope}%
\pgfsys@transformshift{4.228309in}{0.515123in}%
\pgfsys@useobject{currentmarker}{}%
\end{pgfscope}%
\end{pgfscope}%
\begin{pgfscope}%
\definecolor{textcolor}{rgb}{0.000000,0.000000,0.000000}%
\pgfsetstrokecolor{textcolor}%
\pgfsetfillcolor{textcolor}%
\pgftext[x=4.228309in,y=0.417901in,,top]{\color{textcolor}\rmfamily\fontsize{10.000000}{12.000000}\selectfont \(\displaystyle 12\)}%
\end{pgfscope}%
\begin{pgfscope}%
\definecolor{textcolor}{rgb}{0.000000,0.000000,0.000000}%
\pgfsetstrokecolor{textcolor}%
\pgfsetfillcolor{textcolor}%
\pgftext[x=2.709569in,y=0.238889in,,top]{\color{textcolor}\rmfamily\fontsize{10.000000}{12.000000}\selectfont I(\(\displaystyle \mu\)A)}%
\end{pgfscope}%
\begin{pgfscope}%
\pgfsetbuttcap%
\pgfsetroundjoin%
\definecolor{currentfill}{rgb}{0.000000,0.000000,0.000000}%
\pgfsetfillcolor{currentfill}%
\pgfsetlinewidth{0.803000pt}%
\definecolor{currentstroke}{rgb}{0.000000,0.000000,0.000000}%
\pgfsetstrokecolor{currentstroke}%
\pgfsetdash{}{0pt}%
\pgfsys@defobject{currentmarker}{\pgfqpoint{-0.048611in}{0.000000in}}{\pgfqpoint{0.000000in}{0.000000in}}{%
\pgfpathmoveto{\pgfqpoint{0.000000in}{0.000000in}}%
\pgfpathlineto{\pgfqpoint{-0.048611in}{0.000000in}}%
\pgfusepath{stroke,fill}%
}%
\begin{pgfscope}%
\pgfsys@transformshift{0.772069in}{0.607562in}%
\pgfsys@useobject{currentmarker}{}%
\end{pgfscope}%
\end{pgfscope}%
\begin{pgfscope}%
\definecolor{textcolor}{rgb}{0.000000,0.000000,0.000000}%
\pgfsetstrokecolor{textcolor}%
\pgfsetfillcolor{textcolor}%
\pgftext[x=0.605402in,y=0.559337in,left,base]{\color{textcolor}\rmfamily\fontsize{10.000000}{12.000000}\selectfont \(\displaystyle 0\)}%
\end{pgfscope}%
\begin{pgfscope}%
\pgfsetbuttcap%
\pgfsetroundjoin%
\definecolor{currentfill}{rgb}{0.000000,0.000000,0.000000}%
\pgfsetfillcolor{currentfill}%
\pgfsetlinewidth{0.803000pt}%
\definecolor{currentstroke}{rgb}{0.000000,0.000000,0.000000}%
\pgfsetstrokecolor{currentstroke}%
\pgfsetdash{}{0pt}%
\pgfsys@defobject{currentmarker}{\pgfqpoint{-0.048611in}{0.000000in}}{\pgfqpoint{0.000000in}{0.000000in}}{%
\pgfpathmoveto{\pgfqpoint{0.000000in}{0.000000in}}%
\pgfpathlineto{\pgfqpoint{-0.048611in}{0.000000in}}%
\pgfusepath{stroke,fill}%
}%
\begin{pgfscope}%
\pgfsys@transformshift{0.772069in}{1.099095in}%
\pgfsys@useobject{currentmarker}{}%
\end{pgfscope}%
\end{pgfscope}%
\begin{pgfscope}%
\definecolor{textcolor}{rgb}{0.000000,0.000000,0.000000}%
\pgfsetstrokecolor{textcolor}%
\pgfsetfillcolor{textcolor}%
\pgftext[x=0.605402in,y=1.050870in,left,base]{\color{textcolor}\rmfamily\fontsize{10.000000}{12.000000}\selectfont \(\displaystyle 2\)}%
\end{pgfscope}%
\begin{pgfscope}%
\pgfsetbuttcap%
\pgfsetroundjoin%
\definecolor{currentfill}{rgb}{0.000000,0.000000,0.000000}%
\pgfsetfillcolor{currentfill}%
\pgfsetlinewidth{0.803000pt}%
\definecolor{currentstroke}{rgb}{0.000000,0.000000,0.000000}%
\pgfsetstrokecolor{currentstroke}%
\pgfsetdash{}{0pt}%
\pgfsys@defobject{currentmarker}{\pgfqpoint{-0.048611in}{0.000000in}}{\pgfqpoint{0.000000in}{0.000000in}}{%
\pgfpathmoveto{\pgfqpoint{0.000000in}{0.000000in}}%
\pgfpathlineto{\pgfqpoint{-0.048611in}{0.000000in}}%
\pgfusepath{stroke,fill}%
}%
\begin{pgfscope}%
\pgfsys@transformshift{0.772069in}{1.590628in}%
\pgfsys@useobject{currentmarker}{}%
\end{pgfscope}%
\end{pgfscope}%
\begin{pgfscope}%
\definecolor{textcolor}{rgb}{0.000000,0.000000,0.000000}%
\pgfsetstrokecolor{textcolor}%
\pgfsetfillcolor{textcolor}%
\pgftext[x=0.605402in,y=1.542403in,left,base]{\color{textcolor}\rmfamily\fontsize{10.000000}{12.000000}\selectfont \(\displaystyle 4\)}%
\end{pgfscope}%
\begin{pgfscope}%
\pgfsetbuttcap%
\pgfsetroundjoin%
\definecolor{currentfill}{rgb}{0.000000,0.000000,0.000000}%
\pgfsetfillcolor{currentfill}%
\pgfsetlinewidth{0.803000pt}%
\definecolor{currentstroke}{rgb}{0.000000,0.000000,0.000000}%
\pgfsetstrokecolor{currentstroke}%
\pgfsetdash{}{0pt}%
\pgfsys@defobject{currentmarker}{\pgfqpoint{-0.048611in}{0.000000in}}{\pgfqpoint{0.000000in}{0.000000in}}{%
\pgfpathmoveto{\pgfqpoint{0.000000in}{0.000000in}}%
\pgfpathlineto{\pgfqpoint{-0.048611in}{0.000000in}}%
\pgfusepath{stroke,fill}%
}%
\begin{pgfscope}%
\pgfsys@transformshift{0.772069in}{2.082161in}%
\pgfsys@useobject{currentmarker}{}%
\end{pgfscope}%
\end{pgfscope}%
\begin{pgfscope}%
\definecolor{textcolor}{rgb}{0.000000,0.000000,0.000000}%
\pgfsetstrokecolor{textcolor}%
\pgfsetfillcolor{textcolor}%
\pgftext[x=0.605402in,y=2.033935in,left,base]{\color{textcolor}\rmfamily\fontsize{10.000000}{12.000000}\selectfont \(\displaystyle 6\)}%
\end{pgfscope}%
\begin{pgfscope}%
\pgfsetbuttcap%
\pgfsetroundjoin%
\definecolor{currentfill}{rgb}{0.000000,0.000000,0.000000}%
\pgfsetfillcolor{currentfill}%
\pgfsetlinewidth{0.803000pt}%
\definecolor{currentstroke}{rgb}{0.000000,0.000000,0.000000}%
\pgfsetstrokecolor{currentstroke}%
\pgfsetdash{}{0pt}%
\pgfsys@defobject{currentmarker}{\pgfqpoint{-0.048611in}{0.000000in}}{\pgfqpoint{0.000000in}{0.000000in}}{%
\pgfpathmoveto{\pgfqpoint{0.000000in}{0.000000in}}%
\pgfpathlineto{\pgfqpoint{-0.048611in}{0.000000in}}%
\pgfusepath{stroke,fill}%
}%
\begin{pgfscope}%
\pgfsys@transformshift{0.772069in}{2.573693in}%
\pgfsys@useobject{currentmarker}{}%
\end{pgfscope}%
\end{pgfscope}%
\begin{pgfscope}%
\definecolor{textcolor}{rgb}{0.000000,0.000000,0.000000}%
\pgfsetstrokecolor{textcolor}%
\pgfsetfillcolor{textcolor}%
\pgftext[x=0.605402in,y=2.525468in,left,base]{\color{textcolor}\rmfamily\fontsize{10.000000}{12.000000}\selectfont \(\displaystyle 8\)}%
\end{pgfscope}%
\begin{pgfscope}%
\pgfsetbuttcap%
\pgfsetroundjoin%
\definecolor{currentfill}{rgb}{0.000000,0.000000,0.000000}%
\pgfsetfillcolor{currentfill}%
\pgfsetlinewidth{0.803000pt}%
\definecolor{currentstroke}{rgb}{0.000000,0.000000,0.000000}%
\pgfsetstrokecolor{currentstroke}%
\pgfsetdash{}{0pt}%
\pgfsys@defobject{currentmarker}{\pgfqpoint{-0.048611in}{0.000000in}}{\pgfqpoint{0.000000in}{0.000000in}}{%
\pgfpathmoveto{\pgfqpoint{0.000000in}{0.000000in}}%
\pgfpathlineto{\pgfqpoint{-0.048611in}{0.000000in}}%
\pgfusepath{stroke,fill}%
}%
\begin{pgfscope}%
\pgfsys@transformshift{0.772069in}{3.065226in}%
\pgfsys@useobject{currentmarker}{}%
\end{pgfscope}%
\end{pgfscope}%
\begin{pgfscope}%
\definecolor{textcolor}{rgb}{0.000000,0.000000,0.000000}%
\pgfsetstrokecolor{textcolor}%
\pgfsetfillcolor{textcolor}%
\pgftext[x=0.535957in,y=3.017001in,left,base]{\color{textcolor}\rmfamily\fontsize{10.000000}{12.000000}\selectfont \(\displaystyle 10\)}%
\end{pgfscope}%
\begin{pgfscope}%
\definecolor{textcolor}{rgb}{0.000000,0.000000,0.000000}%
\pgfsetstrokecolor{textcolor}%
\pgfsetfillcolor{textcolor}%
\pgftext[x=0.258179in,y=1.862623in,,bottom]{\color{textcolor}\rmfamily\fontsize{10.000000}{12.000000}\selectfont V(V)}%
\end{pgfscope}%
\begin{pgfscope}%
\pgfpathrectangle{\pgfqpoint{0.772069in}{0.515123in}}{\pgfqpoint{3.875000in}{2.695000in}}%
\pgfusepath{clip}%
\pgfsetbuttcap%
\pgfsetroundjoin%
\definecolor{currentfill}{rgb}{0.121569,0.466667,0.705882}%
\pgfsetfillcolor{currentfill}%
\pgfsetlinewidth{1.003750pt}%
\definecolor{currentstroke}{rgb}{0.121569,0.466667,0.705882}%
\pgfsetstrokecolor{currentstroke}%
\pgfsetdash{}{0pt}%
\pgfpathmoveto{\pgfqpoint{0.978205in}{0.827145in}}%
\pgfpathcurveto{\pgfqpoint{0.989255in}{0.827145in}}{\pgfqpoint{0.999854in}{0.831536in}}{\pgfqpoint{1.007668in}{0.839349in}}%
\pgfpathcurveto{\pgfqpoint{1.015481in}{0.847163in}}{\pgfqpoint{1.019872in}{0.857762in}}{\pgfqpoint{1.019872in}{0.868812in}}%
\pgfpathcurveto{\pgfqpoint{1.019872in}{0.879862in}}{\pgfqpoint{1.015481in}{0.890461in}}{\pgfqpoint{1.007668in}{0.898275in}}%
\pgfpathcurveto{\pgfqpoint{0.999854in}{0.906088in}}{\pgfqpoint{0.989255in}{0.910479in}}{\pgfqpoint{0.978205in}{0.910479in}}%
\pgfpathcurveto{\pgfqpoint{0.967155in}{0.910479in}}{\pgfqpoint{0.956556in}{0.906088in}}{\pgfqpoint{0.948742in}{0.898275in}}%
\pgfpathcurveto{\pgfqpoint{0.940929in}{0.890461in}}{\pgfqpoint{0.936538in}{0.879862in}}{\pgfqpoint{0.936538in}{0.868812in}}%
\pgfpathcurveto{\pgfqpoint{0.936538in}{0.857762in}}{\pgfqpoint{0.940929in}{0.847163in}}{\pgfqpoint{0.948742in}{0.839349in}}%
\pgfpathcurveto{\pgfqpoint{0.956556in}{0.831536in}}{\pgfqpoint{0.967155in}{0.827145in}}{\pgfqpoint{0.978205in}{0.827145in}}%
\pgfpathclose%
\pgfusepath{stroke,fill}%
\end{pgfscope}%
\begin{pgfscope}%
\pgfpathrectangle{\pgfqpoint{0.772069in}{0.515123in}}{\pgfqpoint{3.875000in}{2.695000in}}%
\pgfusepath{clip}%
\pgfsetbuttcap%
\pgfsetroundjoin%
\definecolor{currentfill}{rgb}{0.121569,0.466667,0.705882}%
\pgfsetfillcolor{currentfill}%
\pgfsetlinewidth{1.003750pt}%
\definecolor{currentstroke}{rgb}{0.121569,0.466667,0.705882}%
\pgfsetstrokecolor{currentstroke}%
\pgfsetdash{}{0pt}%
\pgfpathmoveto{\pgfqpoint{1.342703in}{1.069717in}}%
\pgfpathcurveto{\pgfqpoint{1.353753in}{1.069717in}}{\pgfqpoint{1.364352in}{1.074107in}}{\pgfqpoint{1.372165in}{1.081921in}}%
\pgfpathcurveto{\pgfqpoint{1.379979in}{1.089734in}}{\pgfqpoint{1.384369in}{1.100333in}}{\pgfqpoint{1.384369in}{1.111383in}}%
\pgfpathcurveto{\pgfqpoint{1.384369in}{1.122434in}}{\pgfqpoint{1.379979in}{1.133033in}}{\pgfqpoint{1.372165in}{1.140846in}}%
\pgfpathcurveto{\pgfqpoint{1.364352in}{1.148660in}}{\pgfqpoint{1.353753in}{1.153050in}}{\pgfqpoint{1.342703in}{1.153050in}}%
\pgfpathcurveto{\pgfqpoint{1.331652in}{1.153050in}}{\pgfqpoint{1.321053in}{1.148660in}}{\pgfqpoint{1.313240in}{1.140846in}}%
\pgfpathcurveto{\pgfqpoint{1.305426in}{1.133033in}}{\pgfqpoint{1.301036in}{1.122434in}}{\pgfqpoint{1.301036in}{1.111383in}}%
\pgfpathcurveto{\pgfqpoint{1.301036in}{1.100333in}}{\pgfqpoint{1.305426in}{1.089734in}}{\pgfqpoint{1.313240in}{1.081921in}}%
\pgfpathcurveto{\pgfqpoint{1.321053in}{1.074107in}}{\pgfqpoint{1.331652in}{1.069717in}}{\pgfqpoint{1.342703in}{1.069717in}}%
\pgfpathclose%
\pgfusepath{stroke,fill}%
\end{pgfscope}%
\begin{pgfscope}%
\pgfpathrectangle{\pgfqpoint{0.772069in}{0.515123in}}{\pgfqpoint{3.875000in}{2.695000in}}%
\pgfusepath{clip}%
\pgfsetbuttcap%
\pgfsetroundjoin%
\definecolor{currentfill}{rgb}{0.121569,0.466667,0.705882}%
\pgfsetfillcolor{currentfill}%
\pgfsetlinewidth{1.003750pt}%
\definecolor{currentstroke}{rgb}{0.121569,0.466667,0.705882}%
\pgfsetstrokecolor{currentstroke}%
\pgfsetdash{}{0pt}%
\pgfpathmoveto{\pgfqpoint{1.737575in}{1.313026in}}%
\pgfpathcurveto{\pgfqpoint{1.748625in}{1.313026in}}{\pgfqpoint{1.759224in}{1.317416in}}{\pgfqpoint{1.767038in}{1.325229in}}%
\pgfpathcurveto{\pgfqpoint{1.774851in}{1.333043in}}{\pgfqpoint{1.779242in}{1.343642in}}{\pgfqpoint{1.779242in}{1.354692in}}%
\pgfpathcurveto{\pgfqpoint{1.779242in}{1.365742in}}{\pgfqpoint{1.774851in}{1.376341in}}{\pgfqpoint{1.767038in}{1.384155in}}%
\pgfpathcurveto{\pgfqpoint{1.759224in}{1.391969in}}{\pgfqpoint{1.748625in}{1.396359in}}{\pgfqpoint{1.737575in}{1.396359in}}%
\pgfpathcurveto{\pgfqpoint{1.726525in}{1.396359in}}{\pgfqpoint{1.715926in}{1.391969in}}{\pgfqpoint{1.708112in}{1.384155in}}%
\pgfpathcurveto{\pgfqpoint{1.700299in}{1.376341in}}{\pgfqpoint{1.695908in}{1.365742in}}{\pgfqpoint{1.695908in}{1.354692in}}%
\pgfpathcurveto{\pgfqpoint{1.695908in}{1.343642in}}{\pgfqpoint{1.700299in}{1.333043in}}{\pgfqpoint{1.708112in}{1.325229in}}%
\pgfpathcurveto{\pgfqpoint{1.715926in}{1.317416in}}{\pgfqpoint{1.726525in}{1.313026in}}{\pgfqpoint{1.737575in}{1.313026in}}%
\pgfpathclose%
\pgfusepath{stroke,fill}%
\end{pgfscope}%
\begin{pgfscope}%
\pgfpathrectangle{\pgfqpoint{0.772069in}{0.515123in}}{\pgfqpoint{3.875000in}{2.695000in}}%
\pgfusepath{clip}%
\pgfsetbuttcap%
\pgfsetroundjoin%
\definecolor{currentfill}{rgb}{0.121569,0.466667,0.705882}%
\pgfsetfillcolor{currentfill}%
\pgfsetlinewidth{1.003750pt}%
\definecolor{currentstroke}{rgb}{0.121569,0.466667,0.705882}%
\pgfsetstrokecolor{currentstroke}%
\pgfsetdash{}{0pt}%
\pgfpathmoveto{\pgfqpoint{2.132447in}{1.563707in}}%
\pgfpathcurveto{\pgfqpoint{2.143498in}{1.563707in}}{\pgfqpoint{2.154097in}{1.568097in}}{\pgfqpoint{2.161910in}{1.575911in}}%
\pgfpathcurveto{\pgfqpoint{2.169724in}{1.583725in}}{\pgfqpoint{2.174114in}{1.594324in}}{\pgfqpoint{2.174114in}{1.605374in}}%
\pgfpathcurveto{\pgfqpoint{2.174114in}{1.616424in}}{\pgfqpoint{2.169724in}{1.627023in}}{\pgfqpoint{2.161910in}{1.634837in}}%
\pgfpathcurveto{\pgfqpoint{2.154097in}{1.642650in}}{\pgfqpoint{2.143498in}{1.647041in}}{\pgfqpoint{2.132447in}{1.647041in}}%
\pgfpathcurveto{\pgfqpoint{2.121397in}{1.647041in}}{\pgfqpoint{2.110798in}{1.642650in}}{\pgfqpoint{2.102985in}{1.634837in}}%
\pgfpathcurveto{\pgfqpoint{2.095171in}{1.627023in}}{\pgfqpoint{2.090781in}{1.616424in}}{\pgfqpoint{2.090781in}{1.605374in}}%
\pgfpathcurveto{\pgfqpoint{2.090781in}{1.594324in}}{\pgfqpoint{2.095171in}{1.583725in}}{\pgfqpoint{2.102985in}{1.575911in}}%
\pgfpathcurveto{\pgfqpoint{2.110798in}{1.568097in}}{\pgfqpoint{2.121397in}{1.563707in}}{\pgfqpoint{2.132447in}{1.563707in}}%
\pgfpathclose%
\pgfusepath{stroke,fill}%
\end{pgfscope}%
\begin{pgfscope}%
\pgfpathrectangle{\pgfqpoint{0.772069in}{0.515123in}}{\pgfqpoint{3.875000in}{2.695000in}}%
\pgfusepath{clip}%
\pgfsetbuttcap%
\pgfsetroundjoin%
\definecolor{currentfill}{rgb}{0.121569,0.466667,0.705882}%
\pgfsetfillcolor{currentfill}%
\pgfsetlinewidth{1.003750pt}%
\definecolor{currentstroke}{rgb}{0.121569,0.466667,0.705882}%
\pgfsetstrokecolor{currentstroke}%
\pgfsetdash{}{0pt}%
\pgfpathmoveto{\pgfqpoint{2.496945in}{1.804558in}}%
\pgfpathcurveto{\pgfqpoint{2.507995in}{1.804558in}}{\pgfqpoint{2.518594in}{1.808949in}}{\pgfqpoint{2.526408in}{1.816762in}}%
\pgfpathcurveto{\pgfqpoint{2.534221in}{1.824576in}}{\pgfqpoint{2.538612in}{1.835175in}}{\pgfqpoint{2.538612in}{1.846225in}}%
\pgfpathcurveto{\pgfqpoint{2.538612in}{1.857275in}}{\pgfqpoint{2.534221in}{1.867874in}}{\pgfqpoint{2.526408in}{1.875688in}}%
\pgfpathcurveto{\pgfqpoint{2.518594in}{1.883501in}}{\pgfqpoint{2.507995in}{1.887892in}}{\pgfqpoint{2.496945in}{1.887892in}}%
\pgfpathcurveto{\pgfqpoint{2.485895in}{1.887892in}}{\pgfqpoint{2.475296in}{1.883501in}}{\pgfqpoint{2.467482in}{1.875688in}}%
\pgfpathcurveto{\pgfqpoint{2.459669in}{1.867874in}}{\pgfqpoint{2.455278in}{1.857275in}}{\pgfqpoint{2.455278in}{1.846225in}}%
\pgfpathcurveto{\pgfqpoint{2.455278in}{1.835175in}}{\pgfqpoint{2.459669in}{1.824576in}}{\pgfqpoint{2.467482in}{1.816762in}}%
\pgfpathcurveto{\pgfqpoint{2.475296in}{1.808949in}}{\pgfqpoint{2.485895in}{1.804558in}}{\pgfqpoint{2.496945in}{1.804558in}}%
\pgfpathclose%
\pgfusepath{stroke,fill}%
\end{pgfscope}%
\begin{pgfscope}%
\pgfpathrectangle{\pgfqpoint{0.772069in}{0.515123in}}{\pgfqpoint{3.875000in}{2.695000in}}%
\pgfusepath{clip}%
\pgfsetbuttcap%
\pgfsetroundjoin%
\definecolor{currentfill}{rgb}{0.121569,0.466667,0.705882}%
\pgfsetfillcolor{currentfill}%
\pgfsetlinewidth{1.003750pt}%
\definecolor{currentstroke}{rgb}{0.121569,0.466667,0.705882}%
\pgfsetstrokecolor{currentstroke}%
\pgfsetdash{}{0pt}%
\pgfpathmoveto{\pgfqpoint{2.922192in}{2.069986in}}%
\pgfpathcurveto{\pgfqpoint{2.933242in}{2.069986in}}{\pgfqpoint{2.943841in}{2.074376in}}{\pgfqpoint{2.951655in}{2.082190in}}%
\pgfpathcurveto{\pgfqpoint{2.959469in}{2.090003in}}{\pgfqpoint{2.963859in}{2.100603in}}{\pgfqpoint{2.963859in}{2.111653in}}%
\pgfpathcurveto{\pgfqpoint{2.963859in}{2.122703in}}{\pgfqpoint{2.959469in}{2.133302in}}{\pgfqpoint{2.951655in}{2.141115in}}%
\pgfpathcurveto{\pgfqpoint{2.943841in}{2.148929in}}{\pgfqpoint{2.933242in}{2.153319in}}{\pgfqpoint{2.922192in}{2.153319in}}%
\pgfpathcurveto{\pgfqpoint{2.911142in}{2.153319in}}{\pgfqpoint{2.900543in}{2.148929in}}{\pgfqpoint{2.892730in}{2.141115in}}%
\pgfpathcurveto{\pgfqpoint{2.884916in}{2.133302in}}{\pgfqpoint{2.880526in}{2.122703in}}{\pgfqpoint{2.880526in}{2.111653in}}%
\pgfpathcurveto{\pgfqpoint{2.880526in}{2.100603in}}{\pgfqpoint{2.884916in}{2.090003in}}{\pgfqpoint{2.892730in}{2.082190in}}%
\pgfpathcurveto{\pgfqpoint{2.900543in}{2.074376in}}{\pgfqpoint{2.911142in}{2.069986in}}{\pgfqpoint{2.922192in}{2.069986in}}%
\pgfpathclose%
\pgfusepath{stroke,fill}%
\end{pgfscope}%
\begin{pgfscope}%
\pgfpathrectangle{\pgfqpoint{0.772069in}{0.515123in}}{\pgfqpoint{3.875000in}{2.695000in}}%
\pgfusepath{clip}%
\pgfsetbuttcap%
\pgfsetroundjoin%
\definecolor{currentfill}{rgb}{0.121569,0.466667,0.705882}%
\pgfsetfillcolor{currentfill}%
\pgfsetlinewidth{1.003750pt}%
\definecolor{currentstroke}{rgb}{0.121569,0.466667,0.705882}%
\pgfsetstrokecolor{currentstroke}%
\pgfsetdash{}{0pt}%
\pgfpathmoveto{\pgfqpoint{3.286690in}{2.305922in}}%
\pgfpathcurveto{\pgfqpoint{3.297740in}{2.305922in}}{\pgfqpoint{3.308339in}{2.310312in}}{\pgfqpoint{3.316153in}{2.318126in}}%
\pgfpathcurveto{\pgfqpoint{3.323966in}{2.325939in}}{\pgfqpoint{3.328357in}{2.336538in}}{\pgfqpoint{3.328357in}{2.347588in}}%
\pgfpathcurveto{\pgfqpoint{3.328357in}{2.358639in}}{\pgfqpoint{3.323966in}{2.369238in}}{\pgfqpoint{3.316153in}{2.377051in}}%
\pgfpathcurveto{\pgfqpoint{3.308339in}{2.384865in}}{\pgfqpoint{3.297740in}{2.389255in}}{\pgfqpoint{3.286690in}{2.389255in}}%
\pgfpathcurveto{\pgfqpoint{3.275640in}{2.389255in}}{\pgfqpoint{3.265041in}{2.384865in}}{\pgfqpoint{3.257227in}{2.377051in}}%
\pgfpathcurveto{\pgfqpoint{3.249414in}{2.369238in}}{\pgfqpoint{3.245023in}{2.358639in}}{\pgfqpoint{3.245023in}{2.347588in}}%
\pgfpathcurveto{\pgfqpoint{3.245023in}{2.336538in}}{\pgfqpoint{3.249414in}{2.325939in}}{\pgfqpoint{3.257227in}{2.318126in}}%
\pgfpathcurveto{\pgfqpoint{3.265041in}{2.310312in}}{\pgfqpoint{3.275640in}{2.305922in}}{\pgfqpoint{3.286690in}{2.305922in}}%
\pgfpathclose%
\pgfusepath{stroke,fill}%
\end{pgfscope}%
\begin{pgfscope}%
\pgfpathrectangle{\pgfqpoint{0.772069in}{0.515123in}}{\pgfqpoint{3.875000in}{2.695000in}}%
\pgfusepath{clip}%
\pgfsetbuttcap%
\pgfsetroundjoin%
\definecolor{currentfill}{rgb}{0.121569,0.466667,0.705882}%
\pgfsetfillcolor{currentfill}%
\pgfsetlinewidth{1.003750pt}%
\definecolor{currentstroke}{rgb}{0.121569,0.466667,0.705882}%
\pgfsetstrokecolor{currentstroke}%
\pgfsetdash{}{0pt}%
\pgfpathmoveto{\pgfqpoint{3.681562in}{2.549230in}}%
\pgfpathcurveto{\pgfqpoint{3.692613in}{2.549230in}}{\pgfqpoint{3.703212in}{2.553621in}}{\pgfqpoint{3.711025in}{2.561434in}}%
\pgfpathcurveto{\pgfqpoint{3.718839in}{2.569248in}}{\pgfqpoint{3.723229in}{2.579847in}}{\pgfqpoint{3.723229in}{2.590897in}}%
\pgfpathcurveto{\pgfqpoint{3.723229in}{2.601947in}}{\pgfqpoint{3.718839in}{2.612546in}}{\pgfqpoint{3.711025in}{2.620360in}}%
\pgfpathcurveto{\pgfqpoint{3.703212in}{2.628174in}}{\pgfqpoint{3.692613in}{2.632564in}}{\pgfqpoint{3.681562in}{2.632564in}}%
\pgfpathcurveto{\pgfqpoint{3.670512in}{2.632564in}}{\pgfqpoint{3.659913in}{2.628174in}}{\pgfqpoint{3.652100in}{2.620360in}}%
\pgfpathcurveto{\pgfqpoint{3.644286in}{2.612546in}}{\pgfqpoint{3.639896in}{2.601947in}}{\pgfqpoint{3.639896in}{2.590897in}}%
\pgfpathcurveto{\pgfqpoint{3.639896in}{2.579847in}}{\pgfqpoint{3.644286in}{2.569248in}}{\pgfqpoint{3.652100in}{2.561434in}}%
\pgfpathcurveto{\pgfqpoint{3.659913in}{2.553621in}}{\pgfqpoint{3.670512in}{2.549230in}}{\pgfqpoint{3.681562in}{2.549230in}}%
\pgfpathclose%
\pgfusepath{stroke,fill}%
\end{pgfscope}%
\begin{pgfscope}%
\pgfpathrectangle{\pgfqpoint{0.772069in}{0.515123in}}{\pgfqpoint{3.875000in}{2.695000in}}%
\pgfusepath{clip}%
\pgfsetbuttcap%
\pgfsetroundjoin%
\definecolor{currentfill}{rgb}{0.121569,0.466667,0.705882}%
\pgfsetfillcolor{currentfill}%
\pgfsetlinewidth{1.003750pt}%
\definecolor{currentstroke}{rgb}{0.121569,0.466667,0.705882}%
\pgfsetstrokecolor{currentstroke}%
\pgfsetdash{}{0pt}%
\pgfpathmoveto{\pgfqpoint{4.015685in}{2.777793in}}%
\pgfpathcurveto{\pgfqpoint{4.026735in}{2.777793in}}{\pgfqpoint{4.037334in}{2.782183in}}{\pgfqpoint{4.045148in}{2.789997in}}%
\pgfpathcurveto{\pgfqpoint{4.052962in}{2.797811in}}{\pgfqpoint{4.057352in}{2.808410in}}{\pgfqpoint{4.057352in}{2.819460in}}%
\pgfpathcurveto{\pgfqpoint{4.057352in}{2.830510in}}{\pgfqpoint{4.052962in}{2.841109in}}{\pgfqpoint{4.045148in}{2.848923in}}%
\pgfpathcurveto{\pgfqpoint{4.037334in}{2.856736in}}{\pgfqpoint{4.026735in}{2.861127in}}{\pgfqpoint{4.015685in}{2.861127in}}%
\pgfpathcurveto{\pgfqpoint{4.004635in}{2.861127in}}{\pgfqpoint{3.994036in}{2.856736in}}{\pgfqpoint{3.986222in}{2.848923in}}%
\pgfpathcurveto{\pgfqpoint{3.978409in}{2.841109in}}{\pgfqpoint{3.974019in}{2.830510in}}{\pgfqpoint{3.974019in}{2.819460in}}%
\pgfpathcurveto{\pgfqpoint{3.974019in}{2.808410in}}{\pgfqpoint{3.978409in}{2.797811in}}{\pgfqpoint{3.986222in}{2.789997in}}%
\pgfpathcurveto{\pgfqpoint{3.994036in}{2.782183in}}{\pgfqpoint{4.004635in}{2.777793in}}{\pgfqpoint{4.015685in}{2.777793in}}%
\pgfpathclose%
\pgfusepath{stroke,fill}%
\end{pgfscope}%
\begin{pgfscope}%
\pgfpathrectangle{\pgfqpoint{0.772069in}{0.515123in}}{\pgfqpoint{3.875000in}{2.695000in}}%
\pgfusepath{clip}%
\pgfsetbuttcap%
\pgfsetroundjoin%
\definecolor{currentfill}{rgb}{0.121569,0.466667,0.705882}%
\pgfsetfillcolor{currentfill}%
\pgfsetlinewidth{1.003750pt}%
\definecolor{currentstroke}{rgb}{0.121569,0.466667,0.705882}%
\pgfsetstrokecolor{currentstroke}%
\pgfsetdash{}{0pt}%
\pgfpathmoveto{\pgfqpoint{4.440932in}{3.043221in}}%
\pgfpathcurveto{\pgfqpoint{4.451983in}{3.043221in}}{\pgfqpoint{4.462582in}{3.047611in}}{\pgfqpoint{4.470395in}{3.055425in}}%
\pgfpathcurveto{\pgfqpoint{4.478209in}{3.063238in}}{\pgfqpoint{4.482599in}{3.073837in}}{\pgfqpoint{4.482599in}{3.084888in}}%
\pgfpathcurveto{\pgfqpoint{4.482599in}{3.095938in}}{\pgfqpoint{4.478209in}{3.106537in}}{\pgfqpoint{4.470395in}{3.114350in}}%
\pgfpathcurveto{\pgfqpoint{4.462582in}{3.122164in}}{\pgfqpoint{4.451983in}{3.126554in}}{\pgfqpoint{4.440932in}{3.126554in}}%
\pgfpathcurveto{\pgfqpoint{4.429882in}{3.126554in}}{\pgfqpoint{4.419283in}{3.122164in}}{\pgfqpoint{4.411470in}{3.114350in}}%
\pgfpathcurveto{\pgfqpoint{4.403656in}{3.106537in}}{\pgfqpoint{4.399266in}{3.095938in}}{\pgfqpoint{4.399266in}{3.084888in}}%
\pgfpathcurveto{\pgfqpoint{4.399266in}{3.073837in}}{\pgfqpoint{4.403656in}{3.063238in}}{\pgfqpoint{4.411470in}{3.055425in}}%
\pgfpathcurveto{\pgfqpoint{4.419283in}{3.047611in}}{\pgfqpoint{4.429882in}{3.043221in}}{\pgfqpoint{4.440932in}{3.043221in}}%
\pgfpathclose%
\pgfusepath{stroke,fill}%
\end{pgfscope}%
\begin{pgfscope}%
\pgfpathrectangle{\pgfqpoint{0.772069in}{0.515123in}}{\pgfqpoint{3.875000in}{2.695000in}}%
\pgfusepath{clip}%
\pgfsetbuttcap%
\pgfsetroundjoin%
\definecolor{currentfill}{rgb}{1.000000,0.388235,0.278431}%
\pgfsetfillcolor{currentfill}%
\pgfsetlinewidth{1.003750pt}%
\definecolor{currentstroke}{rgb}{1.000000,0.388235,0.278431}%
\pgfsetstrokecolor{currentstroke}%
\pgfsetdash{}{0pt}%
\pgfpathmoveto{\pgfqpoint{0.978205in}{0.623159in}}%
\pgfpathcurveto{\pgfqpoint{0.989255in}{0.623159in}}{\pgfqpoint{0.999854in}{0.627550in}}{\pgfqpoint{1.007668in}{0.635363in}}%
\pgfpathcurveto{\pgfqpoint{1.015481in}{0.643177in}}{\pgfqpoint{1.019872in}{0.653776in}}{\pgfqpoint{1.019872in}{0.664826in}}%
\pgfpathcurveto{\pgfqpoint{1.019872in}{0.675876in}}{\pgfqpoint{1.015481in}{0.686475in}}{\pgfqpoint{1.007668in}{0.694289in}}%
\pgfpathcurveto{\pgfqpoint{0.999854in}{0.702102in}}{\pgfqpoint{0.989255in}{0.706493in}}{\pgfqpoint{0.978205in}{0.706493in}}%
\pgfpathcurveto{\pgfqpoint{0.967155in}{0.706493in}}{\pgfqpoint{0.956556in}{0.702102in}}{\pgfqpoint{0.948742in}{0.694289in}}%
\pgfpathcurveto{\pgfqpoint{0.940929in}{0.686475in}}{\pgfqpoint{0.936538in}{0.675876in}}{\pgfqpoint{0.936538in}{0.664826in}}%
\pgfpathcurveto{\pgfqpoint{0.936538in}{0.653776in}}{\pgfqpoint{0.940929in}{0.643177in}}{\pgfqpoint{0.948742in}{0.635363in}}%
\pgfpathcurveto{\pgfqpoint{0.956556in}{0.627550in}}{\pgfqpoint{0.967155in}{0.623159in}}{\pgfqpoint{0.978205in}{0.623159in}}%
\pgfpathclose%
\pgfusepath{stroke,fill}%
\end{pgfscope}%
\begin{pgfscope}%
\pgfpathrectangle{\pgfqpoint{0.772069in}{0.515123in}}{\pgfqpoint{3.875000in}{2.695000in}}%
\pgfusepath{clip}%
\pgfsetbuttcap%
\pgfsetroundjoin%
\definecolor{currentfill}{rgb}{1.000000,0.388235,0.278431}%
\pgfsetfillcolor{currentfill}%
\pgfsetlinewidth{1.003750pt}%
\definecolor{currentstroke}{rgb}{1.000000,0.388235,0.278431}%
\pgfsetstrokecolor{currentstroke}%
\pgfsetdash{}{0pt}%
\pgfpathmoveto{\pgfqpoint{1.342703in}{0.676736in}}%
\pgfpathcurveto{\pgfqpoint{1.353753in}{0.676736in}}{\pgfqpoint{1.364352in}{0.681127in}}{\pgfqpoint{1.372165in}{0.688940in}}%
\pgfpathcurveto{\pgfqpoint{1.379979in}{0.696754in}}{\pgfqpoint{1.384369in}{0.707353in}}{\pgfqpoint{1.384369in}{0.718403in}}%
\pgfpathcurveto{\pgfqpoint{1.384369in}{0.729453in}}{\pgfqpoint{1.379979in}{0.740052in}}{\pgfqpoint{1.372165in}{0.747866in}}%
\pgfpathcurveto{\pgfqpoint{1.364352in}{0.755679in}}{\pgfqpoint{1.353753in}{0.760070in}}{\pgfqpoint{1.342703in}{0.760070in}}%
\pgfpathcurveto{\pgfqpoint{1.331652in}{0.760070in}}{\pgfqpoint{1.321053in}{0.755679in}}{\pgfqpoint{1.313240in}{0.747866in}}%
\pgfpathcurveto{\pgfqpoint{1.305426in}{0.740052in}}{\pgfqpoint{1.301036in}{0.729453in}}{\pgfqpoint{1.301036in}{0.718403in}}%
\pgfpathcurveto{\pgfqpoint{1.301036in}{0.707353in}}{\pgfqpoint{1.305426in}{0.696754in}}{\pgfqpoint{1.313240in}{0.688940in}}%
\pgfpathcurveto{\pgfqpoint{1.321053in}{0.681127in}}{\pgfqpoint{1.331652in}{0.676736in}}{\pgfqpoint{1.342703in}{0.676736in}}%
\pgfpathclose%
\pgfusepath{stroke,fill}%
\end{pgfscope}%
\begin{pgfscope}%
\pgfpathrectangle{\pgfqpoint{0.772069in}{0.515123in}}{\pgfqpoint{3.875000in}{2.695000in}}%
\pgfusepath{clip}%
\pgfsetbuttcap%
\pgfsetroundjoin%
\definecolor{currentfill}{rgb}{1.000000,0.388235,0.278431}%
\pgfsetfillcolor{currentfill}%
\pgfsetlinewidth{1.003750pt}%
\definecolor{currentstroke}{rgb}{1.000000,0.388235,0.278431}%
\pgfsetstrokecolor{currentstroke}%
\pgfsetdash{}{0pt}%
\pgfpathmoveto{\pgfqpoint{1.737575in}{0.730805in}}%
\pgfpathcurveto{\pgfqpoint{1.748625in}{0.730805in}}{\pgfqpoint{1.759224in}{0.735195in}}{\pgfqpoint{1.767038in}{0.743009in}}%
\pgfpathcurveto{\pgfqpoint{1.774851in}{0.750822in}}{\pgfqpoint{1.779242in}{0.761421in}}{\pgfqpoint{1.779242in}{0.772472in}}%
\pgfpathcurveto{\pgfqpoint{1.779242in}{0.783522in}}{\pgfqpoint{1.774851in}{0.794121in}}{\pgfqpoint{1.767038in}{0.801934in}}%
\pgfpathcurveto{\pgfqpoint{1.759224in}{0.809748in}}{\pgfqpoint{1.748625in}{0.814138in}}{\pgfqpoint{1.737575in}{0.814138in}}%
\pgfpathcurveto{\pgfqpoint{1.726525in}{0.814138in}}{\pgfqpoint{1.715926in}{0.809748in}}{\pgfqpoint{1.708112in}{0.801934in}}%
\pgfpathcurveto{\pgfqpoint{1.700299in}{0.794121in}}{\pgfqpoint{1.695908in}{0.783522in}}{\pgfqpoint{1.695908in}{0.772472in}}%
\pgfpathcurveto{\pgfqpoint{1.695908in}{0.761421in}}{\pgfqpoint{1.700299in}{0.750822in}}{\pgfqpoint{1.708112in}{0.743009in}}%
\pgfpathcurveto{\pgfqpoint{1.715926in}{0.735195in}}{\pgfqpoint{1.726525in}{0.730805in}}{\pgfqpoint{1.737575in}{0.730805in}}%
\pgfpathclose%
\pgfusepath{stroke,fill}%
\end{pgfscope}%
\begin{pgfscope}%
\pgfpathrectangle{\pgfqpoint{0.772069in}{0.515123in}}{\pgfqpoint{3.875000in}{2.695000in}}%
\pgfusepath{clip}%
\pgfsetbuttcap%
\pgfsetroundjoin%
\definecolor{currentfill}{rgb}{1.000000,0.388235,0.278431}%
\pgfsetfillcolor{currentfill}%
\pgfsetlinewidth{1.003750pt}%
\definecolor{currentstroke}{rgb}{1.000000,0.388235,0.278431}%
\pgfsetstrokecolor{currentstroke}%
\pgfsetdash{}{0pt}%
\pgfpathmoveto{\pgfqpoint{2.132447in}{0.785611in}}%
\pgfpathcurveto{\pgfqpoint{2.143498in}{0.785611in}}{\pgfqpoint{2.154097in}{0.790001in}}{\pgfqpoint{2.161910in}{0.797815in}}%
\pgfpathcurveto{\pgfqpoint{2.169724in}{0.805628in}}{\pgfqpoint{2.174114in}{0.816227in}}{\pgfqpoint{2.174114in}{0.827278in}}%
\pgfpathcurveto{\pgfqpoint{2.174114in}{0.838328in}}{\pgfqpoint{2.169724in}{0.848927in}}{\pgfqpoint{2.161910in}{0.856740in}}%
\pgfpathcurveto{\pgfqpoint{2.154097in}{0.864554in}}{\pgfqpoint{2.143498in}{0.868944in}}{\pgfqpoint{2.132447in}{0.868944in}}%
\pgfpathcurveto{\pgfqpoint{2.121397in}{0.868944in}}{\pgfqpoint{2.110798in}{0.864554in}}{\pgfqpoint{2.102985in}{0.856740in}}%
\pgfpathcurveto{\pgfqpoint{2.095171in}{0.848927in}}{\pgfqpoint{2.090781in}{0.838328in}}{\pgfqpoint{2.090781in}{0.827278in}}%
\pgfpathcurveto{\pgfqpoint{2.090781in}{0.816227in}}{\pgfqpoint{2.095171in}{0.805628in}}{\pgfqpoint{2.102985in}{0.797815in}}%
\pgfpathcurveto{\pgfqpoint{2.110798in}{0.790001in}}{\pgfqpoint{2.121397in}{0.785611in}}{\pgfqpoint{2.132447in}{0.785611in}}%
\pgfpathclose%
\pgfusepath{stroke,fill}%
\end{pgfscope}%
\begin{pgfscope}%
\pgfpathrectangle{\pgfqpoint{0.772069in}{0.515123in}}{\pgfqpoint{3.875000in}{2.695000in}}%
\pgfusepath{clip}%
\pgfsetbuttcap%
\pgfsetroundjoin%
\definecolor{currentfill}{rgb}{1.000000,0.388235,0.278431}%
\pgfsetfillcolor{currentfill}%
\pgfsetlinewidth{1.003750pt}%
\definecolor{currentstroke}{rgb}{1.000000,0.388235,0.278431}%
\pgfsetstrokecolor{currentstroke}%
\pgfsetdash{}{0pt}%
\pgfpathmoveto{\pgfqpoint{2.496945in}{0.838696in}}%
\pgfpathcurveto{\pgfqpoint{2.507995in}{0.838696in}}{\pgfqpoint{2.518594in}{0.843087in}}{\pgfqpoint{2.526408in}{0.850900in}}%
\pgfpathcurveto{\pgfqpoint{2.534221in}{0.858714in}}{\pgfqpoint{2.538612in}{0.869313in}}{\pgfqpoint{2.538612in}{0.880363in}}%
\pgfpathcurveto{\pgfqpoint{2.538612in}{0.891413in}}{\pgfqpoint{2.534221in}{0.902012in}}{\pgfqpoint{2.526408in}{0.909826in}}%
\pgfpathcurveto{\pgfqpoint{2.518594in}{0.917639in}}{\pgfqpoint{2.507995in}{0.922030in}}{\pgfqpoint{2.496945in}{0.922030in}}%
\pgfpathcurveto{\pgfqpoint{2.485895in}{0.922030in}}{\pgfqpoint{2.475296in}{0.917639in}}{\pgfqpoint{2.467482in}{0.909826in}}%
\pgfpathcurveto{\pgfqpoint{2.459669in}{0.902012in}}{\pgfqpoint{2.455278in}{0.891413in}}{\pgfqpoint{2.455278in}{0.880363in}}%
\pgfpathcurveto{\pgfqpoint{2.455278in}{0.869313in}}{\pgfqpoint{2.459669in}{0.858714in}}{\pgfqpoint{2.467482in}{0.850900in}}%
\pgfpathcurveto{\pgfqpoint{2.475296in}{0.843087in}}{\pgfqpoint{2.485895in}{0.838696in}}{\pgfqpoint{2.496945in}{0.838696in}}%
\pgfpathclose%
\pgfusepath{stroke,fill}%
\end{pgfscope}%
\begin{pgfscope}%
\pgfpathrectangle{\pgfqpoint{0.772069in}{0.515123in}}{\pgfqpoint{3.875000in}{2.695000in}}%
\pgfusepath{clip}%
\pgfsetbuttcap%
\pgfsetroundjoin%
\definecolor{currentfill}{rgb}{1.000000,0.388235,0.278431}%
\pgfsetfillcolor{currentfill}%
\pgfsetlinewidth{1.003750pt}%
\definecolor{currentstroke}{rgb}{1.000000,0.388235,0.278431}%
\pgfsetstrokecolor{currentstroke}%
\pgfsetdash{}{0pt}%
\pgfpathmoveto{\pgfqpoint{2.922192in}{0.896943in}}%
\pgfpathcurveto{\pgfqpoint{2.933242in}{0.896943in}}{\pgfqpoint{2.943841in}{0.901333in}}{\pgfqpoint{2.951655in}{0.909147in}}%
\pgfpathcurveto{\pgfqpoint{2.959469in}{0.916961in}}{\pgfqpoint{2.963859in}{0.927560in}}{\pgfqpoint{2.963859in}{0.938610in}}%
\pgfpathcurveto{\pgfqpoint{2.963859in}{0.949660in}}{\pgfqpoint{2.959469in}{0.960259in}}{\pgfqpoint{2.951655in}{0.968072in}}%
\pgfpathcurveto{\pgfqpoint{2.943841in}{0.975886in}}{\pgfqpoint{2.933242in}{0.980276in}}{\pgfqpoint{2.922192in}{0.980276in}}%
\pgfpathcurveto{\pgfqpoint{2.911142in}{0.980276in}}{\pgfqpoint{2.900543in}{0.975886in}}{\pgfqpoint{2.892730in}{0.968072in}}%
\pgfpathcurveto{\pgfqpoint{2.884916in}{0.960259in}}{\pgfqpoint{2.880526in}{0.949660in}}{\pgfqpoint{2.880526in}{0.938610in}}%
\pgfpathcurveto{\pgfqpoint{2.880526in}{0.927560in}}{\pgfqpoint{2.884916in}{0.916961in}}{\pgfqpoint{2.892730in}{0.909147in}}%
\pgfpathcurveto{\pgfqpoint{2.900543in}{0.901333in}}{\pgfqpoint{2.911142in}{0.896943in}}{\pgfqpoint{2.922192in}{0.896943in}}%
\pgfpathclose%
\pgfusepath{stroke,fill}%
\end{pgfscope}%
\begin{pgfscope}%
\pgfpathrectangle{\pgfqpoint{0.772069in}{0.515123in}}{\pgfqpoint{3.875000in}{2.695000in}}%
\pgfusepath{clip}%
\pgfsetbuttcap%
\pgfsetroundjoin%
\definecolor{currentfill}{rgb}{1.000000,0.388235,0.278431}%
\pgfsetfillcolor{currentfill}%
\pgfsetlinewidth{1.003750pt}%
\definecolor{currentstroke}{rgb}{1.000000,0.388235,0.278431}%
\pgfsetstrokecolor{currentstroke}%
\pgfsetdash{}{0pt}%
\pgfpathmoveto{\pgfqpoint{3.286690in}{0.948800in}}%
\pgfpathcurveto{\pgfqpoint{3.297740in}{0.948800in}}{\pgfqpoint{3.308339in}{0.953190in}}{\pgfqpoint{3.316153in}{0.961004in}}%
\pgfpathcurveto{\pgfqpoint{3.323966in}{0.968817in}}{\pgfqpoint{3.328357in}{0.979416in}}{\pgfqpoint{3.328357in}{0.990466in}}%
\pgfpathcurveto{\pgfqpoint{3.328357in}{1.001517in}}{\pgfqpoint{3.323966in}{1.012116in}}{\pgfqpoint{3.316153in}{1.019929in}}%
\pgfpathcurveto{\pgfqpoint{3.308339in}{1.027743in}}{\pgfqpoint{3.297740in}{1.032133in}}{\pgfqpoint{3.286690in}{1.032133in}}%
\pgfpathcurveto{\pgfqpoint{3.275640in}{1.032133in}}{\pgfqpoint{3.265041in}{1.027743in}}{\pgfqpoint{3.257227in}{1.019929in}}%
\pgfpathcurveto{\pgfqpoint{3.249414in}{1.012116in}}{\pgfqpoint{3.245023in}{1.001517in}}{\pgfqpoint{3.245023in}{0.990466in}}%
\pgfpathcurveto{\pgfqpoint{3.245023in}{0.979416in}}{\pgfqpoint{3.249414in}{0.968817in}}{\pgfqpoint{3.257227in}{0.961004in}}%
\pgfpathcurveto{\pgfqpoint{3.265041in}{0.953190in}}{\pgfqpoint{3.275640in}{0.948800in}}{\pgfqpoint{3.286690in}{0.948800in}}%
\pgfpathclose%
\pgfusepath{stroke,fill}%
\end{pgfscope}%
\begin{pgfscope}%
\pgfpathrectangle{\pgfqpoint{0.772069in}{0.515123in}}{\pgfqpoint{3.875000in}{2.695000in}}%
\pgfusepath{clip}%
\pgfsetbuttcap%
\pgfsetroundjoin%
\definecolor{currentfill}{rgb}{1.000000,0.388235,0.278431}%
\pgfsetfillcolor{currentfill}%
\pgfsetlinewidth{1.003750pt}%
\definecolor{currentstroke}{rgb}{1.000000,0.388235,0.278431}%
\pgfsetstrokecolor{currentstroke}%
\pgfsetdash{}{0pt}%
\pgfpathmoveto{\pgfqpoint{3.681562in}{1.002623in}}%
\pgfpathcurveto{\pgfqpoint{3.692613in}{1.002623in}}{\pgfqpoint{3.703212in}{1.007013in}}{\pgfqpoint{3.711025in}{1.014826in}}%
\pgfpathcurveto{\pgfqpoint{3.718839in}{1.022640in}}{\pgfqpoint{3.723229in}{1.033239in}}{\pgfqpoint{3.723229in}{1.044289in}}%
\pgfpathcurveto{\pgfqpoint{3.723229in}{1.055339in}}{\pgfqpoint{3.718839in}{1.065938in}}{\pgfqpoint{3.711025in}{1.073752in}}%
\pgfpathcurveto{\pgfqpoint{3.703212in}{1.081566in}}{\pgfqpoint{3.692613in}{1.085956in}}{\pgfqpoint{3.681562in}{1.085956in}}%
\pgfpathcurveto{\pgfqpoint{3.670512in}{1.085956in}}{\pgfqpoint{3.659913in}{1.081566in}}{\pgfqpoint{3.652100in}{1.073752in}}%
\pgfpathcurveto{\pgfqpoint{3.644286in}{1.065938in}}{\pgfqpoint{3.639896in}{1.055339in}}{\pgfqpoint{3.639896in}{1.044289in}}%
\pgfpathcurveto{\pgfqpoint{3.639896in}{1.033239in}}{\pgfqpoint{3.644286in}{1.022640in}}{\pgfqpoint{3.652100in}{1.014826in}}%
\pgfpathcurveto{\pgfqpoint{3.659913in}{1.007013in}}{\pgfqpoint{3.670512in}{1.002623in}}{\pgfqpoint{3.681562in}{1.002623in}}%
\pgfpathclose%
\pgfusepath{stroke,fill}%
\end{pgfscope}%
\begin{pgfscope}%
\pgfpathrectangle{\pgfqpoint{0.772069in}{0.515123in}}{\pgfqpoint{3.875000in}{2.695000in}}%
\pgfusepath{clip}%
\pgfsetbuttcap%
\pgfsetroundjoin%
\definecolor{currentfill}{rgb}{1.000000,0.388235,0.278431}%
\pgfsetfillcolor{currentfill}%
\pgfsetlinewidth{1.003750pt}%
\definecolor{currentstroke}{rgb}{1.000000,0.388235,0.278431}%
\pgfsetstrokecolor{currentstroke}%
\pgfsetdash{}{0pt}%
\pgfpathmoveto{\pgfqpoint{4.015685in}{1.050055in}}%
\pgfpathcurveto{\pgfqpoint{4.026735in}{1.050055in}}{\pgfqpoint{4.037334in}{1.054446in}}{\pgfqpoint{4.045148in}{1.062259in}}%
\pgfpathcurveto{\pgfqpoint{4.052962in}{1.070073in}}{\pgfqpoint{4.057352in}{1.080672in}}{\pgfqpoint{4.057352in}{1.091722in}}%
\pgfpathcurveto{\pgfqpoint{4.057352in}{1.102772in}}{\pgfqpoint{4.052962in}{1.113371in}}{\pgfqpoint{4.045148in}{1.121185in}}%
\pgfpathcurveto{\pgfqpoint{4.037334in}{1.128999in}}{\pgfqpoint{4.026735in}{1.133389in}}{\pgfqpoint{4.015685in}{1.133389in}}%
\pgfpathcurveto{\pgfqpoint{4.004635in}{1.133389in}}{\pgfqpoint{3.994036in}{1.128999in}}{\pgfqpoint{3.986222in}{1.121185in}}%
\pgfpathcurveto{\pgfqpoint{3.978409in}{1.113371in}}{\pgfqpoint{3.974019in}{1.102772in}}{\pgfqpoint{3.974019in}{1.091722in}}%
\pgfpathcurveto{\pgfqpoint{3.974019in}{1.080672in}}{\pgfqpoint{3.978409in}{1.070073in}}{\pgfqpoint{3.986222in}{1.062259in}}%
\pgfpathcurveto{\pgfqpoint{3.994036in}{1.054446in}}{\pgfqpoint{4.004635in}{1.050055in}}{\pgfqpoint{4.015685in}{1.050055in}}%
\pgfpathclose%
\pgfusepath{stroke,fill}%
\end{pgfscope}%
\begin{pgfscope}%
\pgfpathrectangle{\pgfqpoint{0.772069in}{0.515123in}}{\pgfqpoint{3.875000in}{2.695000in}}%
\pgfusepath{clip}%
\pgfsetbuttcap%
\pgfsetroundjoin%
\definecolor{currentfill}{rgb}{1.000000,0.388235,0.278431}%
\pgfsetfillcolor{currentfill}%
\pgfsetlinewidth{1.003750pt}%
\definecolor{currentstroke}{rgb}{1.000000,0.388235,0.278431}%
\pgfsetstrokecolor{currentstroke}%
\pgfsetdash{}{0pt}%
\pgfpathmoveto{\pgfqpoint{4.440932in}{1.109039in}}%
\pgfpathcurveto{\pgfqpoint{4.451983in}{1.109039in}}{\pgfqpoint{4.462582in}{1.113430in}}{\pgfqpoint{4.470395in}{1.121243in}}%
\pgfpathcurveto{\pgfqpoint{4.478209in}{1.129057in}}{\pgfqpoint{4.482599in}{1.139656in}}{\pgfqpoint{4.482599in}{1.150706in}}%
\pgfpathcurveto{\pgfqpoint{4.482599in}{1.161756in}}{\pgfqpoint{4.478209in}{1.172355in}}{\pgfqpoint{4.470395in}{1.180169in}}%
\pgfpathcurveto{\pgfqpoint{4.462582in}{1.187982in}}{\pgfqpoint{4.451983in}{1.192373in}}{\pgfqpoint{4.440932in}{1.192373in}}%
\pgfpathcurveto{\pgfqpoint{4.429882in}{1.192373in}}{\pgfqpoint{4.419283in}{1.187982in}}{\pgfqpoint{4.411470in}{1.180169in}}%
\pgfpathcurveto{\pgfqpoint{4.403656in}{1.172355in}}{\pgfqpoint{4.399266in}{1.161756in}}{\pgfqpoint{4.399266in}{1.150706in}}%
\pgfpathcurveto{\pgfqpoint{4.399266in}{1.139656in}}{\pgfqpoint{4.403656in}{1.129057in}}{\pgfqpoint{4.411470in}{1.121243in}}%
\pgfpathcurveto{\pgfqpoint{4.419283in}{1.113430in}}{\pgfqpoint{4.429882in}{1.109039in}}{\pgfqpoint{4.440932in}{1.109039in}}%
\pgfpathclose%
\pgfusepath{stroke,fill}%
\end{pgfscope}%
\begin{pgfscope}%
\pgfpathrectangle{\pgfqpoint{0.772069in}{0.515123in}}{\pgfqpoint{3.875000in}{2.695000in}}%
\pgfusepath{clip}%
\pgfsetbuttcap%
\pgfsetroundjoin%
\definecolor{currentfill}{rgb}{1.000000,0.843137,0.000000}%
\pgfsetfillcolor{currentfill}%
\pgfsetlinewidth{1.003750pt}%
\definecolor{currentstroke}{rgb}{1.000000,0.843137,0.000000}%
\pgfsetstrokecolor{currentstroke}%
\pgfsetdash{}{0pt}%
\pgfpathmoveto{\pgfqpoint{0.978205in}{0.636922in}}%
\pgfpathcurveto{\pgfqpoint{0.989255in}{0.636922in}}{\pgfqpoint{0.999854in}{0.641312in}}{\pgfqpoint{1.007668in}{0.649126in}}%
\pgfpathcurveto{\pgfqpoint{1.015481in}{0.656940in}}{\pgfqpoint{1.019872in}{0.667539in}}{\pgfqpoint{1.019872in}{0.678589in}}%
\pgfpathcurveto{\pgfqpoint{1.019872in}{0.689639in}}{\pgfqpoint{1.015481in}{0.700238in}}{\pgfqpoint{1.007668in}{0.708052in}}%
\pgfpathcurveto{\pgfqpoint{0.999854in}{0.715865in}}{\pgfqpoint{0.989255in}{0.720256in}}{\pgfqpoint{0.978205in}{0.720256in}}%
\pgfpathcurveto{\pgfqpoint{0.967155in}{0.720256in}}{\pgfqpoint{0.956556in}{0.715865in}}{\pgfqpoint{0.948742in}{0.708052in}}%
\pgfpathcurveto{\pgfqpoint{0.940929in}{0.700238in}}{\pgfqpoint{0.936538in}{0.689639in}}{\pgfqpoint{0.936538in}{0.678589in}}%
\pgfpathcurveto{\pgfqpoint{0.936538in}{0.667539in}}{\pgfqpoint{0.940929in}{0.656940in}}{\pgfqpoint{0.948742in}{0.649126in}}%
\pgfpathcurveto{\pgfqpoint{0.956556in}{0.641312in}}{\pgfqpoint{0.967155in}{0.636922in}}{\pgfqpoint{0.978205in}{0.636922in}}%
\pgfpathclose%
\pgfusepath{stroke,fill}%
\end{pgfscope}%
\begin{pgfscope}%
\pgfpathrectangle{\pgfqpoint{0.772069in}{0.515123in}}{\pgfqpoint{3.875000in}{2.695000in}}%
\pgfusepath{clip}%
\pgfsetbuttcap%
\pgfsetroundjoin%
\definecolor{currentfill}{rgb}{1.000000,0.843137,0.000000}%
\pgfsetfillcolor{currentfill}%
\pgfsetlinewidth{1.003750pt}%
\definecolor{currentstroke}{rgb}{1.000000,0.843137,0.000000}%
\pgfsetstrokecolor{currentstroke}%
\pgfsetdash{}{0pt}%
\pgfpathmoveto{\pgfqpoint{1.342703in}{0.703033in}}%
\pgfpathcurveto{\pgfqpoint{1.353753in}{0.703033in}}{\pgfqpoint{1.364352in}{0.707424in}}{\pgfqpoint{1.372165in}{0.715237in}}%
\pgfpathcurveto{\pgfqpoint{1.379979in}{0.723051in}}{\pgfqpoint{1.384369in}{0.733650in}}{\pgfqpoint{1.384369in}{0.744700in}}%
\pgfpathcurveto{\pgfqpoint{1.384369in}{0.755750in}}{\pgfqpoint{1.379979in}{0.766349in}}{\pgfqpoint{1.372165in}{0.774163in}}%
\pgfpathcurveto{\pgfqpoint{1.364352in}{0.781976in}}{\pgfqpoint{1.353753in}{0.786367in}}{\pgfqpoint{1.342703in}{0.786367in}}%
\pgfpathcurveto{\pgfqpoint{1.331652in}{0.786367in}}{\pgfqpoint{1.321053in}{0.781976in}}{\pgfqpoint{1.313240in}{0.774163in}}%
\pgfpathcurveto{\pgfqpoint{1.305426in}{0.766349in}}{\pgfqpoint{1.301036in}{0.755750in}}{\pgfqpoint{1.301036in}{0.744700in}}%
\pgfpathcurveto{\pgfqpoint{1.301036in}{0.733650in}}{\pgfqpoint{1.305426in}{0.723051in}}{\pgfqpoint{1.313240in}{0.715237in}}%
\pgfpathcurveto{\pgfqpoint{1.321053in}{0.707424in}}{\pgfqpoint{1.331652in}{0.703033in}}{\pgfqpoint{1.342703in}{0.703033in}}%
\pgfpathclose%
\pgfusepath{stroke,fill}%
\end{pgfscope}%
\begin{pgfscope}%
\pgfpathrectangle{\pgfqpoint{0.772069in}{0.515123in}}{\pgfqpoint{3.875000in}{2.695000in}}%
\pgfusepath{clip}%
\pgfsetbuttcap%
\pgfsetroundjoin%
\definecolor{currentfill}{rgb}{1.000000,0.843137,0.000000}%
\pgfsetfillcolor{currentfill}%
\pgfsetlinewidth{1.003750pt}%
\definecolor{currentstroke}{rgb}{1.000000,0.843137,0.000000}%
\pgfsetstrokecolor{currentstroke}%
\pgfsetdash{}{0pt}%
\pgfpathmoveto{\pgfqpoint{1.737575in}{0.769882in}}%
\pgfpathcurveto{\pgfqpoint{1.748625in}{0.769882in}}{\pgfqpoint{1.759224in}{0.774272in}}{\pgfqpoint{1.767038in}{0.782086in}}%
\pgfpathcurveto{\pgfqpoint{1.774851in}{0.789899in}}{\pgfqpoint{1.779242in}{0.800498in}}{\pgfqpoint{1.779242in}{0.811548in}}%
\pgfpathcurveto{\pgfqpoint{1.779242in}{0.822599in}}{\pgfqpoint{1.774851in}{0.833198in}}{\pgfqpoint{1.767038in}{0.841011in}}%
\pgfpathcurveto{\pgfqpoint{1.759224in}{0.848825in}}{\pgfqpoint{1.748625in}{0.853215in}}{\pgfqpoint{1.737575in}{0.853215in}}%
\pgfpathcurveto{\pgfqpoint{1.726525in}{0.853215in}}{\pgfqpoint{1.715926in}{0.848825in}}{\pgfqpoint{1.708112in}{0.841011in}}%
\pgfpathcurveto{\pgfqpoint{1.700299in}{0.833198in}}{\pgfqpoint{1.695908in}{0.822599in}}{\pgfqpoint{1.695908in}{0.811548in}}%
\pgfpathcurveto{\pgfqpoint{1.695908in}{0.800498in}}{\pgfqpoint{1.700299in}{0.789899in}}{\pgfqpoint{1.708112in}{0.782086in}}%
\pgfpathcurveto{\pgfqpoint{1.715926in}{0.774272in}}{\pgfqpoint{1.726525in}{0.769882in}}{\pgfqpoint{1.737575in}{0.769882in}}%
\pgfpathclose%
\pgfusepath{stroke,fill}%
\end{pgfscope}%
\begin{pgfscope}%
\pgfpathrectangle{\pgfqpoint{0.772069in}{0.515123in}}{\pgfqpoint{3.875000in}{2.695000in}}%
\pgfusepath{clip}%
\pgfsetbuttcap%
\pgfsetroundjoin%
\definecolor{currentfill}{rgb}{1.000000,0.843137,0.000000}%
\pgfsetfillcolor{currentfill}%
\pgfsetlinewidth{1.003750pt}%
\definecolor{currentstroke}{rgb}{1.000000,0.843137,0.000000}%
\pgfsetstrokecolor{currentstroke}%
\pgfsetdash{}{0pt}%
\pgfpathmoveto{\pgfqpoint{2.132447in}{0.837713in}}%
\pgfpathcurveto{\pgfqpoint{2.143498in}{0.837713in}}{\pgfqpoint{2.154097in}{0.842104in}}{\pgfqpoint{2.161910in}{0.849917in}}%
\pgfpathcurveto{\pgfqpoint{2.169724in}{0.857731in}}{\pgfqpoint{2.174114in}{0.868330in}}{\pgfqpoint{2.174114in}{0.879380in}}%
\pgfpathcurveto{\pgfqpoint{2.174114in}{0.890430in}}{\pgfqpoint{2.169724in}{0.901029in}}{\pgfqpoint{2.161910in}{0.908843in}}%
\pgfpathcurveto{\pgfqpoint{2.154097in}{0.916656in}}{\pgfqpoint{2.143498in}{0.921047in}}{\pgfqpoint{2.132447in}{0.921047in}}%
\pgfpathcurveto{\pgfqpoint{2.121397in}{0.921047in}}{\pgfqpoint{2.110798in}{0.916656in}}{\pgfqpoint{2.102985in}{0.908843in}}%
\pgfpathcurveto{\pgfqpoint{2.095171in}{0.901029in}}{\pgfqpoint{2.090781in}{0.890430in}}{\pgfqpoint{2.090781in}{0.879380in}}%
\pgfpathcurveto{\pgfqpoint{2.090781in}{0.868330in}}{\pgfqpoint{2.095171in}{0.857731in}}{\pgfqpoint{2.102985in}{0.849917in}}%
\pgfpathcurveto{\pgfqpoint{2.110798in}{0.842104in}}{\pgfqpoint{2.121397in}{0.837713in}}{\pgfqpoint{2.132447in}{0.837713in}}%
\pgfpathclose%
\pgfusepath{stroke,fill}%
\end{pgfscope}%
\begin{pgfscope}%
\pgfpathrectangle{\pgfqpoint{0.772069in}{0.515123in}}{\pgfqpoint{3.875000in}{2.695000in}}%
\pgfusepath{clip}%
\pgfsetbuttcap%
\pgfsetroundjoin%
\definecolor{currentfill}{rgb}{1.000000,0.843137,0.000000}%
\pgfsetfillcolor{currentfill}%
\pgfsetlinewidth{1.003750pt}%
\definecolor{currentstroke}{rgb}{1.000000,0.843137,0.000000}%
\pgfsetstrokecolor{currentstroke}%
\pgfsetdash{}{0pt}%
\pgfpathmoveto{\pgfqpoint{2.496945in}{0.903579in}}%
\pgfpathcurveto{\pgfqpoint{2.507995in}{0.903579in}}{\pgfqpoint{2.518594in}{0.907969in}}{\pgfqpoint{2.526408in}{0.915783in}}%
\pgfpathcurveto{\pgfqpoint{2.534221in}{0.923596in}}{\pgfqpoint{2.538612in}{0.934195in}}{\pgfqpoint{2.538612in}{0.945245in}}%
\pgfpathcurveto{\pgfqpoint{2.538612in}{0.956296in}}{\pgfqpoint{2.534221in}{0.966895in}}{\pgfqpoint{2.526408in}{0.974708in}}%
\pgfpathcurveto{\pgfqpoint{2.518594in}{0.982522in}}{\pgfqpoint{2.507995in}{0.986912in}}{\pgfqpoint{2.496945in}{0.986912in}}%
\pgfpathcurveto{\pgfqpoint{2.485895in}{0.986912in}}{\pgfqpoint{2.475296in}{0.982522in}}{\pgfqpoint{2.467482in}{0.974708in}}%
\pgfpathcurveto{\pgfqpoint{2.459669in}{0.966895in}}{\pgfqpoint{2.455278in}{0.956296in}}{\pgfqpoint{2.455278in}{0.945245in}}%
\pgfpathcurveto{\pgfqpoint{2.455278in}{0.934195in}}{\pgfqpoint{2.459669in}{0.923596in}}{\pgfqpoint{2.467482in}{0.915783in}}%
\pgfpathcurveto{\pgfqpoint{2.475296in}{0.907969in}}{\pgfqpoint{2.485895in}{0.903579in}}{\pgfqpoint{2.496945in}{0.903579in}}%
\pgfpathclose%
\pgfusepath{stroke,fill}%
\end{pgfscope}%
\begin{pgfscope}%
\pgfpathrectangle{\pgfqpoint{0.772069in}{0.515123in}}{\pgfqpoint{3.875000in}{2.695000in}}%
\pgfusepath{clip}%
\pgfsetbuttcap%
\pgfsetroundjoin%
\definecolor{currentfill}{rgb}{1.000000,0.843137,0.000000}%
\pgfsetfillcolor{currentfill}%
\pgfsetlinewidth{1.003750pt}%
\definecolor{currentstroke}{rgb}{1.000000,0.843137,0.000000}%
\pgfsetstrokecolor{currentstroke}%
\pgfsetdash{}{0pt}%
\pgfpathmoveto{\pgfqpoint{2.922192in}{0.975588in}}%
\pgfpathcurveto{\pgfqpoint{2.933242in}{0.975588in}}{\pgfqpoint{2.943841in}{0.979979in}}{\pgfqpoint{2.951655in}{0.987792in}}%
\pgfpathcurveto{\pgfqpoint{2.959469in}{0.995606in}}{\pgfqpoint{2.963859in}{1.006205in}}{\pgfqpoint{2.963859in}{1.017255in}}%
\pgfpathcurveto{\pgfqpoint{2.963859in}{1.028305in}}{\pgfqpoint{2.959469in}{1.038904in}}{\pgfqpoint{2.951655in}{1.046718in}}%
\pgfpathcurveto{\pgfqpoint{2.943841in}{1.054531in}}{\pgfqpoint{2.933242in}{1.058922in}}{\pgfqpoint{2.922192in}{1.058922in}}%
\pgfpathcurveto{\pgfqpoint{2.911142in}{1.058922in}}{\pgfqpoint{2.900543in}{1.054531in}}{\pgfqpoint{2.892730in}{1.046718in}}%
\pgfpathcurveto{\pgfqpoint{2.884916in}{1.038904in}}{\pgfqpoint{2.880526in}{1.028305in}}{\pgfqpoint{2.880526in}{1.017255in}}%
\pgfpathcurveto{\pgfqpoint{2.880526in}{1.006205in}}{\pgfqpoint{2.884916in}{0.995606in}}{\pgfqpoint{2.892730in}{0.987792in}}%
\pgfpathcurveto{\pgfqpoint{2.900543in}{0.979979in}}{\pgfqpoint{2.911142in}{0.975588in}}{\pgfqpoint{2.922192in}{0.975588in}}%
\pgfpathclose%
\pgfusepath{stroke,fill}%
\end{pgfscope}%
\begin{pgfscope}%
\pgfpathrectangle{\pgfqpoint{0.772069in}{0.515123in}}{\pgfqpoint{3.875000in}{2.695000in}}%
\pgfusepath{clip}%
\pgfsetbuttcap%
\pgfsetroundjoin%
\definecolor{currentfill}{rgb}{1.000000,0.843137,0.000000}%
\pgfsetfillcolor{currentfill}%
\pgfsetlinewidth{1.003750pt}%
\definecolor{currentstroke}{rgb}{1.000000,0.843137,0.000000}%
\pgfsetstrokecolor{currentstroke}%
\pgfsetdash{}{0pt}%
\pgfpathmoveto{\pgfqpoint{3.286690in}{1.039733in}}%
\pgfpathcurveto{\pgfqpoint{3.297740in}{1.039733in}}{\pgfqpoint{3.308339in}{1.044124in}}{\pgfqpoint{3.316153in}{1.051937in}}%
\pgfpathcurveto{\pgfqpoint{3.323966in}{1.059751in}}{\pgfqpoint{3.328357in}{1.070350in}}{\pgfqpoint{3.328357in}{1.081400in}}%
\pgfpathcurveto{\pgfqpoint{3.328357in}{1.092450in}}{\pgfqpoint{3.323966in}{1.103049in}}{\pgfqpoint{3.316153in}{1.110863in}}%
\pgfpathcurveto{\pgfqpoint{3.308339in}{1.118676in}}{\pgfqpoint{3.297740in}{1.123067in}}{\pgfqpoint{3.286690in}{1.123067in}}%
\pgfpathcurveto{\pgfqpoint{3.275640in}{1.123067in}}{\pgfqpoint{3.265041in}{1.118676in}}{\pgfqpoint{3.257227in}{1.110863in}}%
\pgfpathcurveto{\pgfqpoint{3.249414in}{1.103049in}}{\pgfqpoint{3.245023in}{1.092450in}}{\pgfqpoint{3.245023in}{1.081400in}}%
\pgfpathcurveto{\pgfqpoint{3.245023in}{1.070350in}}{\pgfqpoint{3.249414in}{1.059751in}}{\pgfqpoint{3.257227in}{1.051937in}}%
\pgfpathcurveto{\pgfqpoint{3.265041in}{1.044124in}}{\pgfqpoint{3.275640in}{1.039733in}}{\pgfqpoint{3.286690in}{1.039733in}}%
\pgfpathclose%
\pgfusepath{stroke,fill}%
\end{pgfscope}%
\begin{pgfscope}%
\pgfpathrectangle{\pgfqpoint{0.772069in}{0.515123in}}{\pgfqpoint{3.875000in}{2.695000in}}%
\pgfusepath{clip}%
\pgfsetbuttcap%
\pgfsetroundjoin%
\definecolor{currentfill}{rgb}{1.000000,0.843137,0.000000}%
\pgfsetfillcolor{currentfill}%
\pgfsetlinewidth{1.003750pt}%
\definecolor{currentstroke}{rgb}{1.000000,0.843137,0.000000}%
\pgfsetstrokecolor{currentstroke}%
\pgfsetdash{}{0pt}%
\pgfpathmoveto{\pgfqpoint{3.681562in}{1.104124in}}%
\pgfpathcurveto{\pgfqpoint{3.692613in}{1.104124in}}{\pgfqpoint{3.703212in}{1.108514in}}{\pgfqpoint{3.711025in}{1.116328in}}%
\pgfpathcurveto{\pgfqpoint{3.718839in}{1.124142in}}{\pgfqpoint{3.723229in}{1.134741in}}{\pgfqpoint{3.723229in}{1.145791in}}%
\pgfpathcurveto{\pgfqpoint{3.723229in}{1.156841in}}{\pgfqpoint{3.718839in}{1.167440in}}{\pgfqpoint{3.711025in}{1.175254in}}%
\pgfpathcurveto{\pgfqpoint{3.703212in}{1.183067in}}{\pgfqpoint{3.692613in}{1.187457in}}{\pgfqpoint{3.681562in}{1.187457in}}%
\pgfpathcurveto{\pgfqpoint{3.670512in}{1.187457in}}{\pgfqpoint{3.659913in}{1.183067in}}{\pgfqpoint{3.652100in}{1.175254in}}%
\pgfpathcurveto{\pgfqpoint{3.644286in}{1.167440in}}{\pgfqpoint{3.639896in}{1.156841in}}{\pgfqpoint{3.639896in}{1.145791in}}%
\pgfpathcurveto{\pgfqpoint{3.639896in}{1.134741in}}{\pgfqpoint{3.644286in}{1.124142in}}{\pgfqpoint{3.652100in}{1.116328in}}%
\pgfpathcurveto{\pgfqpoint{3.659913in}{1.108514in}}{\pgfqpoint{3.670512in}{1.104124in}}{\pgfqpoint{3.681562in}{1.104124in}}%
\pgfpathclose%
\pgfusepath{stroke,fill}%
\end{pgfscope}%
\begin{pgfscope}%
\pgfpathrectangle{\pgfqpoint{0.772069in}{0.515123in}}{\pgfqpoint{3.875000in}{2.695000in}}%
\pgfusepath{clip}%
\pgfsetbuttcap%
\pgfsetroundjoin%
\definecolor{currentfill}{rgb}{1.000000,0.843137,0.000000}%
\pgfsetfillcolor{currentfill}%
\pgfsetlinewidth{1.003750pt}%
\definecolor{currentstroke}{rgb}{1.000000,0.843137,0.000000}%
\pgfsetstrokecolor{currentstroke}%
\pgfsetdash{}{0pt}%
\pgfpathmoveto{\pgfqpoint{4.015685in}{1.165566in}}%
\pgfpathcurveto{\pgfqpoint{4.026735in}{1.165566in}}{\pgfqpoint{4.037334in}{1.169956in}}{\pgfqpoint{4.045148in}{1.177770in}}%
\pgfpathcurveto{\pgfqpoint{4.052962in}{1.185583in}}{\pgfqpoint{4.057352in}{1.196182in}}{\pgfqpoint{4.057352in}{1.207232in}}%
\pgfpathcurveto{\pgfqpoint{4.057352in}{1.218282in}}{\pgfqpoint{4.052962in}{1.228882in}}{\pgfqpoint{4.045148in}{1.236695in}}%
\pgfpathcurveto{\pgfqpoint{4.037334in}{1.244509in}}{\pgfqpoint{4.026735in}{1.248899in}}{\pgfqpoint{4.015685in}{1.248899in}}%
\pgfpathcurveto{\pgfqpoint{4.004635in}{1.248899in}}{\pgfqpoint{3.994036in}{1.244509in}}{\pgfqpoint{3.986222in}{1.236695in}}%
\pgfpathcurveto{\pgfqpoint{3.978409in}{1.228882in}}{\pgfqpoint{3.974019in}{1.218282in}}{\pgfqpoint{3.974019in}{1.207232in}}%
\pgfpathcurveto{\pgfqpoint{3.974019in}{1.196182in}}{\pgfqpoint{3.978409in}{1.185583in}}{\pgfqpoint{3.986222in}{1.177770in}}%
\pgfpathcurveto{\pgfqpoint{3.994036in}{1.169956in}}{\pgfqpoint{4.004635in}{1.165566in}}{\pgfqpoint{4.015685in}{1.165566in}}%
\pgfpathclose%
\pgfusepath{stroke,fill}%
\end{pgfscope}%
\begin{pgfscope}%
\pgfpathrectangle{\pgfqpoint{0.772069in}{0.515123in}}{\pgfqpoint{3.875000in}{2.695000in}}%
\pgfusepath{clip}%
\pgfsetbuttcap%
\pgfsetroundjoin%
\definecolor{currentfill}{rgb}{1.000000,0.843137,0.000000}%
\pgfsetfillcolor{currentfill}%
\pgfsetlinewidth{1.003750pt}%
\definecolor{currentstroke}{rgb}{1.000000,0.843137,0.000000}%
\pgfsetstrokecolor{currentstroke}%
\pgfsetdash{}{0pt}%
\pgfpathmoveto{\pgfqpoint{4.440932in}{1.239296in}}%
\pgfpathcurveto{\pgfqpoint{4.451983in}{1.239296in}}{\pgfqpoint{4.462582in}{1.243686in}}{\pgfqpoint{4.470395in}{1.251499in}}%
\pgfpathcurveto{\pgfqpoint{4.478209in}{1.259313in}}{\pgfqpoint{4.482599in}{1.269912in}}{\pgfqpoint{4.482599in}{1.280962in}}%
\pgfpathcurveto{\pgfqpoint{4.482599in}{1.292012in}}{\pgfqpoint{4.478209in}{1.302611in}}{\pgfqpoint{4.470395in}{1.310425in}}%
\pgfpathcurveto{\pgfqpoint{4.462582in}{1.318239in}}{\pgfqpoint{4.451983in}{1.322629in}}{\pgfqpoint{4.440932in}{1.322629in}}%
\pgfpathcurveto{\pgfqpoint{4.429882in}{1.322629in}}{\pgfqpoint{4.419283in}{1.318239in}}{\pgfqpoint{4.411470in}{1.310425in}}%
\pgfpathcurveto{\pgfqpoint{4.403656in}{1.302611in}}{\pgfqpoint{4.399266in}{1.292012in}}{\pgfqpoint{4.399266in}{1.280962in}}%
\pgfpathcurveto{\pgfqpoint{4.399266in}{1.269912in}}{\pgfqpoint{4.403656in}{1.259313in}}{\pgfqpoint{4.411470in}{1.251499in}}%
\pgfpathcurveto{\pgfqpoint{4.419283in}{1.243686in}}{\pgfqpoint{4.429882in}{1.239296in}}{\pgfqpoint{4.440932in}{1.239296in}}%
\pgfpathclose%
\pgfusepath{stroke,fill}%
\end{pgfscope}%
\begin{pgfscope}%
\pgfpathrectangle{\pgfqpoint{0.772069in}{0.515123in}}{\pgfqpoint{3.875000in}{2.695000in}}%
\pgfusepath{clip}%
\pgfsetbuttcap%
\pgfsetroundjoin%
\definecolor{currentfill}{rgb}{0.196078,0.803922,0.196078}%
\pgfsetfillcolor{currentfill}%
\pgfsetlinewidth{1.003750pt}%
\definecolor{currentstroke}{rgb}{0.196078,0.803922,0.196078}%
\pgfsetstrokecolor{currentstroke}%
\pgfsetdash{}{0pt}%
\pgfpathmoveto{\pgfqpoint{0.978205in}{0.694432in}}%
\pgfpathcurveto{\pgfqpoint{0.989255in}{0.694432in}}{\pgfqpoint{0.999854in}{0.698822in}}{\pgfqpoint{1.007668in}{0.706635in}}%
\pgfpathcurveto{\pgfqpoint{1.015481in}{0.714449in}}{\pgfqpoint{1.019872in}{0.725048in}}{\pgfqpoint{1.019872in}{0.736098in}}%
\pgfpathcurveto{\pgfqpoint{1.019872in}{0.747148in}}{\pgfqpoint{1.015481in}{0.757747in}}{\pgfqpoint{1.007668in}{0.765561in}}%
\pgfpathcurveto{\pgfqpoint{0.999854in}{0.773375in}}{\pgfqpoint{0.989255in}{0.777765in}}{\pgfqpoint{0.978205in}{0.777765in}}%
\pgfpathcurveto{\pgfqpoint{0.967155in}{0.777765in}}{\pgfqpoint{0.956556in}{0.773375in}}{\pgfqpoint{0.948742in}{0.765561in}}%
\pgfpathcurveto{\pgfqpoint{0.940929in}{0.757747in}}{\pgfqpoint{0.936538in}{0.747148in}}{\pgfqpoint{0.936538in}{0.736098in}}%
\pgfpathcurveto{\pgfqpoint{0.936538in}{0.725048in}}{\pgfqpoint{0.940929in}{0.714449in}}{\pgfqpoint{0.948742in}{0.706635in}}%
\pgfpathcurveto{\pgfqpoint{0.956556in}{0.698822in}}{\pgfqpoint{0.967155in}{0.694432in}}{\pgfqpoint{0.978205in}{0.694432in}}%
\pgfpathclose%
\pgfusepath{stroke,fill}%
\end{pgfscope}%
\begin{pgfscope}%
\pgfpathrectangle{\pgfqpoint{0.772069in}{0.515123in}}{\pgfqpoint{3.875000in}{2.695000in}}%
\pgfusepath{clip}%
\pgfsetbuttcap%
\pgfsetroundjoin%
\definecolor{currentfill}{rgb}{0.196078,0.803922,0.196078}%
\pgfsetfillcolor{currentfill}%
\pgfsetlinewidth{1.003750pt}%
\definecolor{currentstroke}{rgb}{0.196078,0.803922,0.196078}%
\pgfsetstrokecolor{currentstroke}%
\pgfsetdash{}{0pt}%
\pgfpathmoveto{\pgfqpoint{1.342703in}{0.814366in}}%
\pgfpathcurveto{\pgfqpoint{1.353753in}{0.814366in}}{\pgfqpoint{1.364352in}{0.818756in}}{\pgfqpoint{1.372165in}{0.826569in}}%
\pgfpathcurveto{\pgfqpoint{1.379979in}{0.834383in}}{\pgfqpoint{1.384369in}{0.844982in}}{\pgfqpoint{1.384369in}{0.856032in}}%
\pgfpathcurveto{\pgfqpoint{1.384369in}{0.867082in}}{\pgfqpoint{1.379979in}{0.877681in}}{\pgfqpoint{1.372165in}{0.885495in}}%
\pgfpathcurveto{\pgfqpoint{1.364352in}{0.893309in}}{\pgfqpoint{1.353753in}{0.897699in}}{\pgfqpoint{1.342703in}{0.897699in}}%
\pgfpathcurveto{\pgfqpoint{1.331652in}{0.897699in}}{\pgfqpoint{1.321053in}{0.893309in}}{\pgfqpoint{1.313240in}{0.885495in}}%
\pgfpathcurveto{\pgfqpoint{1.305426in}{0.877681in}}{\pgfqpoint{1.301036in}{0.867082in}}{\pgfqpoint{1.301036in}{0.856032in}}%
\pgfpathcurveto{\pgfqpoint{1.301036in}{0.844982in}}{\pgfqpoint{1.305426in}{0.834383in}}{\pgfqpoint{1.313240in}{0.826569in}}%
\pgfpathcurveto{\pgfqpoint{1.321053in}{0.818756in}}{\pgfqpoint{1.331652in}{0.814366in}}{\pgfqpoint{1.342703in}{0.814366in}}%
\pgfpathclose%
\pgfusepath{stroke,fill}%
\end{pgfscope}%
\begin{pgfscope}%
\pgfpathrectangle{\pgfqpoint{0.772069in}{0.515123in}}{\pgfqpoint{3.875000in}{2.695000in}}%
\pgfusepath{clip}%
\pgfsetbuttcap%
\pgfsetroundjoin%
\definecolor{currentfill}{rgb}{0.196078,0.803922,0.196078}%
\pgfsetfillcolor{currentfill}%
\pgfsetlinewidth{1.003750pt}%
\definecolor{currentstroke}{rgb}{0.196078,0.803922,0.196078}%
\pgfsetstrokecolor{currentstroke}%
\pgfsetdash{}{0pt}%
\pgfpathmoveto{\pgfqpoint{1.737575in}{0.935037in}}%
\pgfpathcurveto{\pgfqpoint{1.748625in}{0.935037in}}{\pgfqpoint{1.759224in}{0.939427in}}{\pgfqpoint{1.767038in}{0.947241in}}%
\pgfpathcurveto{\pgfqpoint{1.774851in}{0.955054in}}{\pgfqpoint{1.779242in}{0.965653in}}{\pgfqpoint{1.779242in}{0.976703in}}%
\pgfpathcurveto{\pgfqpoint{1.779242in}{0.987754in}}{\pgfqpoint{1.774851in}{0.998353in}}{\pgfqpoint{1.767038in}{1.006166in}}%
\pgfpathcurveto{\pgfqpoint{1.759224in}{1.013980in}}{\pgfqpoint{1.748625in}{1.018370in}}{\pgfqpoint{1.737575in}{1.018370in}}%
\pgfpathcurveto{\pgfqpoint{1.726525in}{1.018370in}}{\pgfqpoint{1.715926in}{1.013980in}}{\pgfqpoint{1.708112in}{1.006166in}}%
\pgfpathcurveto{\pgfqpoint{1.700299in}{0.998353in}}{\pgfqpoint{1.695908in}{0.987754in}}{\pgfqpoint{1.695908in}{0.976703in}}%
\pgfpathcurveto{\pgfqpoint{1.695908in}{0.965653in}}{\pgfqpoint{1.700299in}{0.955054in}}{\pgfqpoint{1.708112in}{0.947241in}}%
\pgfpathcurveto{\pgfqpoint{1.715926in}{0.939427in}}{\pgfqpoint{1.726525in}{0.935037in}}{\pgfqpoint{1.737575in}{0.935037in}}%
\pgfpathclose%
\pgfusepath{stroke,fill}%
\end{pgfscope}%
\begin{pgfscope}%
\pgfpathrectangle{\pgfqpoint{0.772069in}{0.515123in}}{\pgfqpoint{3.875000in}{2.695000in}}%
\pgfusepath{clip}%
\pgfsetbuttcap%
\pgfsetroundjoin%
\definecolor{currentfill}{rgb}{0.196078,0.803922,0.196078}%
\pgfsetfillcolor{currentfill}%
\pgfsetlinewidth{1.003750pt}%
\definecolor{currentstroke}{rgb}{0.196078,0.803922,0.196078}%
\pgfsetstrokecolor{currentstroke}%
\pgfsetdash{}{0pt}%
\pgfpathmoveto{\pgfqpoint{2.132447in}{1.054971in}}%
\pgfpathcurveto{\pgfqpoint{2.143498in}{1.054971in}}{\pgfqpoint{2.154097in}{1.059361in}}{\pgfqpoint{2.161910in}{1.067175in}}%
\pgfpathcurveto{\pgfqpoint{2.169724in}{1.074988in}}{\pgfqpoint{2.174114in}{1.085587in}}{\pgfqpoint{2.174114in}{1.096637in}}%
\pgfpathcurveto{\pgfqpoint{2.174114in}{1.107688in}}{\pgfqpoint{2.169724in}{1.118287in}}{\pgfqpoint{2.161910in}{1.126100in}}%
\pgfpathcurveto{\pgfqpoint{2.154097in}{1.133914in}}{\pgfqpoint{2.143498in}{1.138304in}}{\pgfqpoint{2.132447in}{1.138304in}}%
\pgfpathcurveto{\pgfqpoint{2.121397in}{1.138304in}}{\pgfqpoint{2.110798in}{1.133914in}}{\pgfqpoint{2.102985in}{1.126100in}}%
\pgfpathcurveto{\pgfqpoint{2.095171in}{1.118287in}}{\pgfqpoint{2.090781in}{1.107688in}}{\pgfqpoint{2.090781in}{1.096637in}}%
\pgfpathcurveto{\pgfqpoint{2.090781in}{1.085587in}}{\pgfqpoint{2.095171in}{1.074988in}}{\pgfqpoint{2.102985in}{1.067175in}}%
\pgfpathcurveto{\pgfqpoint{2.110798in}{1.059361in}}{\pgfqpoint{2.121397in}{1.054971in}}{\pgfqpoint{2.132447in}{1.054971in}}%
\pgfpathclose%
\pgfusepath{stroke,fill}%
\end{pgfscope}%
\begin{pgfscope}%
\pgfpathrectangle{\pgfqpoint{0.772069in}{0.515123in}}{\pgfqpoint{3.875000in}{2.695000in}}%
\pgfusepath{clip}%
\pgfsetbuttcap%
\pgfsetroundjoin%
\definecolor{currentfill}{rgb}{0.196078,0.803922,0.196078}%
\pgfsetfillcolor{currentfill}%
\pgfsetlinewidth{1.003750pt}%
\definecolor{currentstroke}{rgb}{0.196078,0.803922,0.196078}%
\pgfsetstrokecolor{currentstroke}%
\pgfsetdash{}{0pt}%
\pgfpathmoveto{\pgfqpoint{2.496945in}{1.172939in}}%
\pgfpathcurveto{\pgfqpoint{2.507995in}{1.172939in}}{\pgfqpoint{2.518594in}{1.177329in}}{\pgfqpoint{2.526408in}{1.185143in}}%
\pgfpathcurveto{\pgfqpoint{2.534221in}{1.192956in}}{\pgfqpoint{2.538612in}{1.203555in}}{\pgfqpoint{2.538612in}{1.214605in}}%
\pgfpathcurveto{\pgfqpoint{2.538612in}{1.225655in}}{\pgfqpoint{2.534221in}{1.236255in}}{\pgfqpoint{2.526408in}{1.244068in}}%
\pgfpathcurveto{\pgfqpoint{2.518594in}{1.251882in}}{\pgfqpoint{2.507995in}{1.256272in}}{\pgfqpoint{2.496945in}{1.256272in}}%
\pgfpathcurveto{\pgfqpoint{2.485895in}{1.256272in}}{\pgfqpoint{2.475296in}{1.251882in}}{\pgfqpoint{2.467482in}{1.244068in}}%
\pgfpathcurveto{\pgfqpoint{2.459669in}{1.236255in}}{\pgfqpoint{2.455278in}{1.225655in}}{\pgfqpoint{2.455278in}{1.214605in}}%
\pgfpathcurveto{\pgfqpoint{2.455278in}{1.203555in}}{\pgfqpoint{2.459669in}{1.192956in}}{\pgfqpoint{2.467482in}{1.185143in}}%
\pgfpathcurveto{\pgfqpoint{2.475296in}{1.177329in}}{\pgfqpoint{2.485895in}{1.172939in}}{\pgfqpoint{2.496945in}{1.172939in}}%
\pgfpathclose%
\pgfusepath{stroke,fill}%
\end{pgfscope}%
\begin{pgfscope}%
\pgfpathrectangle{\pgfqpoint{0.772069in}{0.515123in}}{\pgfqpoint{3.875000in}{2.695000in}}%
\pgfusepath{clip}%
\pgfsetbuttcap%
\pgfsetroundjoin%
\definecolor{currentfill}{rgb}{0.196078,0.803922,0.196078}%
\pgfsetfillcolor{currentfill}%
\pgfsetlinewidth{1.003750pt}%
\definecolor{currentstroke}{rgb}{0.196078,0.803922,0.196078}%
\pgfsetstrokecolor{currentstroke}%
\pgfsetdash{}{0pt}%
\pgfpathmoveto{\pgfqpoint{2.922192in}{1.303195in}}%
\pgfpathcurveto{\pgfqpoint{2.933242in}{1.303195in}}{\pgfqpoint{2.943841in}{1.307585in}}{\pgfqpoint{2.951655in}{1.315399in}}%
\pgfpathcurveto{\pgfqpoint{2.959469in}{1.323212in}}{\pgfqpoint{2.963859in}{1.333811in}}{\pgfqpoint{2.963859in}{1.344862in}}%
\pgfpathcurveto{\pgfqpoint{2.963859in}{1.355912in}}{\pgfqpoint{2.959469in}{1.366511in}}{\pgfqpoint{2.951655in}{1.374324in}}%
\pgfpathcurveto{\pgfqpoint{2.943841in}{1.382138in}}{\pgfqpoint{2.933242in}{1.386528in}}{\pgfqpoint{2.922192in}{1.386528in}}%
\pgfpathcurveto{\pgfqpoint{2.911142in}{1.386528in}}{\pgfqpoint{2.900543in}{1.382138in}}{\pgfqpoint{2.892730in}{1.374324in}}%
\pgfpathcurveto{\pgfqpoint{2.884916in}{1.366511in}}{\pgfqpoint{2.880526in}{1.355912in}}{\pgfqpoint{2.880526in}{1.344862in}}%
\pgfpathcurveto{\pgfqpoint{2.880526in}{1.333811in}}{\pgfqpoint{2.884916in}{1.323212in}}{\pgfqpoint{2.892730in}{1.315399in}}%
\pgfpathcurveto{\pgfqpoint{2.900543in}{1.307585in}}{\pgfqpoint{2.911142in}{1.303195in}}{\pgfqpoint{2.922192in}{1.303195in}}%
\pgfpathclose%
\pgfusepath{stroke,fill}%
\end{pgfscope}%
\begin{pgfscope}%
\pgfpathrectangle{\pgfqpoint{0.772069in}{0.515123in}}{\pgfqpoint{3.875000in}{2.695000in}}%
\pgfusepath{clip}%
\pgfsetbuttcap%
\pgfsetroundjoin%
\definecolor{currentfill}{rgb}{0.196078,0.803922,0.196078}%
\pgfsetfillcolor{currentfill}%
\pgfsetlinewidth{1.003750pt}%
\definecolor{currentstroke}{rgb}{0.196078,0.803922,0.196078}%
\pgfsetstrokecolor{currentstroke}%
\pgfsetdash{}{0pt}%
\pgfpathmoveto{\pgfqpoint{3.286690in}{1.418705in}}%
\pgfpathcurveto{\pgfqpoint{3.297740in}{1.418705in}}{\pgfqpoint{3.308339in}{1.423095in}}{\pgfqpoint{3.316153in}{1.430909in}}%
\pgfpathcurveto{\pgfqpoint{3.323966in}{1.438723in}}{\pgfqpoint{3.328357in}{1.449322in}}{\pgfqpoint{3.328357in}{1.460372in}}%
\pgfpathcurveto{\pgfqpoint{3.328357in}{1.471422in}}{\pgfqpoint{3.323966in}{1.482021in}}{\pgfqpoint{3.316153in}{1.489835in}}%
\pgfpathcurveto{\pgfqpoint{3.308339in}{1.497648in}}{\pgfqpoint{3.297740in}{1.502038in}}{\pgfqpoint{3.286690in}{1.502038in}}%
\pgfpathcurveto{\pgfqpoint{3.275640in}{1.502038in}}{\pgfqpoint{3.265041in}{1.497648in}}{\pgfqpoint{3.257227in}{1.489835in}}%
\pgfpathcurveto{\pgfqpoint{3.249414in}{1.482021in}}{\pgfqpoint{3.245023in}{1.471422in}}{\pgfqpoint{3.245023in}{1.460372in}}%
\pgfpathcurveto{\pgfqpoint{3.245023in}{1.449322in}}{\pgfqpoint{3.249414in}{1.438723in}}{\pgfqpoint{3.257227in}{1.430909in}}%
\pgfpathcurveto{\pgfqpoint{3.265041in}{1.423095in}}{\pgfqpoint{3.275640in}{1.418705in}}{\pgfqpoint{3.286690in}{1.418705in}}%
\pgfpathclose%
\pgfusepath{stroke,fill}%
\end{pgfscope}%
\begin{pgfscope}%
\pgfpathrectangle{\pgfqpoint{0.772069in}{0.515123in}}{\pgfqpoint{3.875000in}{2.695000in}}%
\pgfusepath{clip}%
\pgfsetbuttcap%
\pgfsetroundjoin%
\definecolor{currentfill}{rgb}{0.196078,0.803922,0.196078}%
\pgfsetfillcolor{currentfill}%
\pgfsetlinewidth{1.003750pt}%
\definecolor{currentstroke}{rgb}{0.196078,0.803922,0.196078}%
\pgfsetstrokecolor{currentstroke}%
\pgfsetdash{}{0pt}%
\pgfpathmoveto{\pgfqpoint{3.681562in}{1.539131in}}%
\pgfpathcurveto{\pgfqpoint{3.692613in}{1.539131in}}{\pgfqpoint{3.703212in}{1.543521in}}{\pgfqpoint{3.711025in}{1.551334in}}%
\pgfpathcurveto{\pgfqpoint{3.718839in}{1.559148in}}{\pgfqpoint{3.723229in}{1.569747in}}{\pgfqpoint{3.723229in}{1.580797in}}%
\pgfpathcurveto{\pgfqpoint{3.723229in}{1.591847in}}{\pgfqpoint{3.718839in}{1.602446in}}{\pgfqpoint{3.711025in}{1.610260in}}%
\pgfpathcurveto{\pgfqpoint{3.703212in}{1.618074in}}{\pgfqpoint{3.692613in}{1.622464in}}{\pgfqpoint{3.681562in}{1.622464in}}%
\pgfpathcurveto{\pgfqpoint{3.670512in}{1.622464in}}{\pgfqpoint{3.659913in}{1.618074in}}{\pgfqpoint{3.652100in}{1.610260in}}%
\pgfpathcurveto{\pgfqpoint{3.644286in}{1.602446in}}{\pgfqpoint{3.639896in}{1.591847in}}{\pgfqpoint{3.639896in}{1.580797in}}%
\pgfpathcurveto{\pgfqpoint{3.639896in}{1.569747in}}{\pgfqpoint{3.644286in}{1.559148in}}{\pgfqpoint{3.652100in}{1.551334in}}%
\pgfpathcurveto{\pgfqpoint{3.659913in}{1.543521in}}{\pgfqpoint{3.670512in}{1.539131in}}{\pgfqpoint{3.681562in}{1.539131in}}%
\pgfpathclose%
\pgfusepath{stroke,fill}%
\end{pgfscope}%
\begin{pgfscope}%
\pgfpathrectangle{\pgfqpoint{0.772069in}{0.515123in}}{\pgfqpoint{3.875000in}{2.695000in}}%
\pgfusepath{clip}%
\pgfsetbuttcap%
\pgfsetroundjoin%
\definecolor{currentfill}{rgb}{0.196078,0.803922,0.196078}%
\pgfsetfillcolor{currentfill}%
\pgfsetlinewidth{1.003750pt}%
\definecolor{currentstroke}{rgb}{0.196078,0.803922,0.196078}%
\pgfsetstrokecolor{currentstroke}%
\pgfsetdash{}{0pt}%
\pgfpathmoveto{\pgfqpoint{4.015685in}{1.652183in}}%
\pgfpathcurveto{\pgfqpoint{4.026735in}{1.652183in}}{\pgfqpoint{4.037334in}{1.656573in}}{\pgfqpoint{4.045148in}{1.664387in}}%
\pgfpathcurveto{\pgfqpoint{4.052962in}{1.672201in}}{\pgfqpoint{4.057352in}{1.682800in}}{\pgfqpoint{4.057352in}{1.693850in}}%
\pgfpathcurveto{\pgfqpoint{4.057352in}{1.704900in}}{\pgfqpoint{4.052962in}{1.715499in}}{\pgfqpoint{4.045148in}{1.723313in}}%
\pgfpathcurveto{\pgfqpoint{4.037334in}{1.731126in}}{\pgfqpoint{4.026735in}{1.735516in}}{\pgfqpoint{4.015685in}{1.735516in}}%
\pgfpathcurveto{\pgfqpoint{4.004635in}{1.735516in}}{\pgfqpoint{3.994036in}{1.731126in}}{\pgfqpoint{3.986222in}{1.723313in}}%
\pgfpathcurveto{\pgfqpoint{3.978409in}{1.715499in}}{\pgfqpoint{3.974019in}{1.704900in}}{\pgfqpoint{3.974019in}{1.693850in}}%
\pgfpathcurveto{\pgfqpoint{3.974019in}{1.682800in}}{\pgfqpoint{3.978409in}{1.672201in}}{\pgfqpoint{3.986222in}{1.664387in}}%
\pgfpathcurveto{\pgfqpoint{3.994036in}{1.656573in}}{\pgfqpoint{4.004635in}{1.652183in}}{\pgfqpoint{4.015685in}{1.652183in}}%
\pgfpathclose%
\pgfusepath{stroke,fill}%
\end{pgfscope}%
\begin{pgfscope}%
\pgfpathrectangle{\pgfqpoint{0.772069in}{0.515123in}}{\pgfqpoint{3.875000in}{2.695000in}}%
\pgfusepath{clip}%
\pgfsetbuttcap%
\pgfsetroundjoin%
\definecolor{currentfill}{rgb}{0.196078,0.803922,0.196078}%
\pgfsetfillcolor{currentfill}%
\pgfsetlinewidth{1.003750pt}%
\definecolor{currentstroke}{rgb}{0.196078,0.803922,0.196078}%
\pgfsetstrokecolor{currentstroke}%
\pgfsetdash{}{0pt}%
\pgfpathmoveto{\pgfqpoint{4.440932in}{1.782439in}}%
\pgfpathcurveto{\pgfqpoint{4.451983in}{1.782439in}}{\pgfqpoint{4.462582in}{1.786830in}}{\pgfqpoint{4.470395in}{1.794643in}}%
\pgfpathcurveto{\pgfqpoint{4.478209in}{1.802457in}}{\pgfqpoint{4.482599in}{1.813056in}}{\pgfqpoint{4.482599in}{1.824106in}}%
\pgfpathcurveto{\pgfqpoint{4.482599in}{1.835156in}}{\pgfqpoint{4.478209in}{1.845755in}}{\pgfqpoint{4.470395in}{1.853569in}}%
\pgfpathcurveto{\pgfqpoint{4.462582in}{1.861382in}}{\pgfqpoint{4.451983in}{1.865773in}}{\pgfqpoint{4.440932in}{1.865773in}}%
\pgfpathcurveto{\pgfqpoint{4.429882in}{1.865773in}}{\pgfqpoint{4.419283in}{1.861382in}}{\pgfqpoint{4.411470in}{1.853569in}}%
\pgfpathcurveto{\pgfqpoint{4.403656in}{1.845755in}}{\pgfqpoint{4.399266in}{1.835156in}}{\pgfqpoint{4.399266in}{1.824106in}}%
\pgfpathcurveto{\pgfqpoint{4.399266in}{1.813056in}}{\pgfqpoint{4.403656in}{1.802457in}}{\pgfqpoint{4.411470in}{1.794643in}}%
\pgfpathcurveto{\pgfqpoint{4.419283in}{1.786830in}}{\pgfqpoint{4.429882in}{1.782439in}}{\pgfqpoint{4.440932in}{1.782439in}}%
\pgfpathclose%
\pgfusepath{stroke,fill}%
\end{pgfscope}%
\begin{pgfscope}%
\pgfsetrectcap%
\pgfsetmiterjoin%
\pgfsetlinewidth{0.803000pt}%
\definecolor{currentstroke}{rgb}{0.000000,0.000000,0.000000}%
\pgfsetstrokecolor{currentstroke}%
\pgfsetdash{}{0pt}%
\pgfpathmoveto{\pgfqpoint{0.772069in}{0.515123in}}%
\pgfpathlineto{\pgfqpoint{0.772069in}{3.210123in}}%
\pgfusepath{stroke}%
\end{pgfscope}%
\begin{pgfscope}%
\pgfsetrectcap%
\pgfsetmiterjoin%
\pgfsetlinewidth{0.803000pt}%
\definecolor{currentstroke}{rgb}{0.000000,0.000000,0.000000}%
\pgfsetstrokecolor{currentstroke}%
\pgfsetdash{}{0pt}%
\pgfpathmoveto{\pgfqpoint{4.647069in}{0.515123in}}%
\pgfpathlineto{\pgfqpoint{4.647069in}{3.210123in}}%
\pgfusepath{stroke}%
\end{pgfscope}%
\begin{pgfscope}%
\pgfsetrectcap%
\pgfsetmiterjoin%
\pgfsetlinewidth{0.803000pt}%
\definecolor{currentstroke}{rgb}{0.000000,0.000000,0.000000}%
\pgfsetstrokecolor{currentstroke}%
\pgfsetdash{}{0pt}%
\pgfpathmoveto{\pgfqpoint{0.772069in}{0.515123in}}%
\pgfpathlineto{\pgfqpoint{4.647069in}{0.515123in}}%
\pgfusepath{stroke}%
\end{pgfscope}%
\begin{pgfscope}%
\pgfsetrectcap%
\pgfsetmiterjoin%
\pgfsetlinewidth{0.803000pt}%
\definecolor{currentstroke}{rgb}{0.000000,0.000000,0.000000}%
\pgfsetstrokecolor{currentstroke}%
\pgfsetdash{}{0pt}%
\pgfpathmoveto{\pgfqpoint{0.772069in}{3.210123in}}%
\pgfpathlineto{\pgfqpoint{4.647069in}{3.210123in}}%
\pgfusepath{stroke}%
\end{pgfscope}%
\end{pgfpicture}%
\makeatother%
\endgroup%

    \caption{Voltajes (\textcolor{Blue}{$V$}, \textcolor{Red}{$V_1$}, \textcolor{Yellow}{$V_2$}, \textcolor{Green}{$V_3$}) frente a intensidad (I)}
  \end{figure}

  \subsubsection{Ajuste por mínimos cuadrados}

  Vuelve a aparecer lo que parece una relación lineal en la gráfica, por lo que pasaremos a hacer un ajuste por mínimos cuadrados. Cómo explicamos en el apartado inicial \ref{sec:reglin} y cómo ya aplicamos en la sección anterior, \ref{sec:ajusteminres}, calcularemos los parámetros \textit{a} y \textit{b} utilizando las ecuaciones \ref{ec:a} y \ref{ec:b}, para ello calculando diversos sumatorios utilizando el programa de \code{python} que describimos, cargando la tabla en .csv con nuestros datos y aplicando las fórmulas de los sumatorias, con lo que obtenemos los siguientes resultados:
  \begin{gather}
    \sum_i x_i = 6,98\cdot10^{-5} \nonumber \qquad \sum_i y_i = 55,6 \nonumber \qquad \sum_i x_iy_i = 4,93\cdot10^{-4} \nonumber \qquad \sum_i x_i^2 = 6,20\cdot10^{-10} \nonumber \\
    a = 4,71\cdot10^{-2} \nonumber \qquad b = 790000 \nonumber
  \end{gather}
  El proceso sería el mismo para obtener los datos de las rectas de $V_1$, $V_2$ y $V_3$. Finalmente podríamos dibujar las gráficas, ahora con una recta creada por regresión lineal que se ajusta a los datos que obtuvimos.

  \begin{figure}[H]
    %\centering
    \hspace{2.5em} %% Creator: Matplotlib, PGF backend
%%
%% To include the figure in your LaTeX document, write
%%   \input{<filename>.pgf}
%%
%% Make sure the required packages are loaded in your preamble
%%   \usepackage{pgf}
%%
%% Figures using additional raster images can only be included by \input if
%% they are in the same directory as the main LaTeX file. For loading figures
%% from other directories you can use the `import` package
%%   \usepackage{import}
%% and then include the figures with
%%   \import{<path to file>}{<filename>.pgf}
%%
%% Matplotlib used the following preamble
%%
\begingroup%
\makeatletter%
\begin{pgfpicture}%
\pgfpathrectangle{\pgfpointorigin}{\pgfqpoint{4.747069in}{3.310123in}}%
\pgfusepath{use as bounding box, clip}%
\begin{pgfscope}%
\pgfsetbuttcap%
\pgfsetmiterjoin%
\definecolor{currentfill}{rgb}{1.000000,1.000000,1.000000}%
\pgfsetfillcolor{currentfill}%
\pgfsetlinewidth{0.000000pt}%
\definecolor{currentstroke}{rgb}{1.000000,1.000000,1.000000}%
\pgfsetstrokecolor{currentstroke}%
\pgfsetdash{}{0pt}%
\pgfpathmoveto{\pgfqpoint{0.000000in}{0.000000in}}%
\pgfpathlineto{\pgfqpoint{4.747069in}{0.000000in}}%
\pgfpathlineto{\pgfqpoint{4.747069in}{3.310123in}}%
\pgfpathlineto{\pgfqpoint{0.000000in}{3.310123in}}%
\pgfpathclose%
\pgfusepath{fill}%
\end{pgfscope}%
\begin{pgfscope}%
\pgfsetbuttcap%
\pgfsetmiterjoin%
\definecolor{currentfill}{rgb}{1.000000,1.000000,1.000000}%
\pgfsetfillcolor{currentfill}%
\pgfsetlinewidth{0.000000pt}%
\definecolor{currentstroke}{rgb}{0.000000,0.000000,0.000000}%
\pgfsetstrokecolor{currentstroke}%
\pgfsetstrokeopacity{0.000000}%
\pgfsetdash{}{0pt}%
\pgfpathmoveto{\pgfqpoint{0.772069in}{0.515123in}}%
\pgfpathlineto{\pgfqpoint{4.647069in}{0.515123in}}%
\pgfpathlineto{\pgfqpoint{4.647069in}{3.210123in}}%
\pgfpathlineto{\pgfqpoint{0.772069in}{3.210123in}}%
\pgfpathclose%
\pgfusepath{fill}%
\end{pgfscope}%
\begin{pgfscope}%
\pgfpathrectangle{\pgfqpoint{0.772069in}{0.515123in}}{\pgfqpoint{3.875000in}{2.695000in}}%
\pgfusepath{clip}%
\pgfsetrectcap%
\pgfsetroundjoin%
\pgfsetlinewidth{1.505625pt}%
\definecolor{currentstroke}{rgb}{0.529412,0.807843,0.921569}%
\pgfsetstrokecolor{currentstroke}%
\pgfsetdash{}{0pt}%
\pgfpathmoveto{\pgfqpoint{0.978205in}{0.860239in}}%
\pgfpathlineto{\pgfqpoint{1.362953in}{1.103430in}}%
\pgfpathlineto{\pgfqpoint{1.747700in}{1.346621in}}%
\pgfpathlineto{\pgfqpoint{2.132447in}{1.589812in}}%
\pgfpathlineto{\pgfqpoint{2.517195in}{1.833003in}}%
\pgfpathlineto{\pgfqpoint{2.901942in}{2.076194in}}%
\pgfpathlineto{\pgfqpoint{3.286690in}{2.319385in}}%
\pgfpathlineto{\pgfqpoint{3.671437in}{2.562576in}}%
\pgfpathlineto{\pgfqpoint{4.056185in}{2.805767in}}%
\pgfpathlineto{\pgfqpoint{4.440932in}{3.048958in}}%
\pgfusepath{stroke}%
\end{pgfscope}%
\begin{pgfscope}%
\pgfpathrectangle{\pgfqpoint{0.772069in}{0.515123in}}{\pgfqpoint{3.875000in}{2.695000in}}%
\pgfusepath{clip}%
\pgfsetrectcap%
\pgfsetroundjoin%
\pgfsetlinewidth{1.505625pt}%
\definecolor{currentstroke}{rgb}{1.000000,0.627451,0.478431}%
\pgfsetstrokecolor{currentstroke}%
\pgfsetdash{}{0pt}%
\pgfpathmoveto{\pgfqpoint{0.978205in}{0.665355in}}%
\pgfpathlineto{\pgfqpoint{1.362953in}{0.718660in}}%
\pgfpathlineto{\pgfqpoint{1.747700in}{0.771965in}}%
\pgfpathlineto{\pgfqpoint{2.132447in}{0.825270in}}%
\pgfpathlineto{\pgfqpoint{2.517195in}{0.878575in}}%
\pgfpathlineto{\pgfqpoint{2.901942in}{0.931880in}}%
\pgfpathlineto{\pgfqpoint{3.286690in}{0.985185in}}%
\pgfpathlineto{\pgfqpoint{3.671437in}{1.038490in}}%
\pgfpathlineto{\pgfqpoint{4.056185in}{1.091795in}}%
\pgfpathlineto{\pgfqpoint{4.440932in}{1.145100in}}%
\pgfusepath{stroke}%
\end{pgfscope}%
\begin{pgfscope}%
\pgfpathrectangle{\pgfqpoint{0.772069in}{0.515123in}}{\pgfqpoint{3.875000in}{2.695000in}}%
\pgfusepath{clip}%
\pgfsetrectcap%
\pgfsetroundjoin%
\pgfsetlinewidth{1.505625pt}%
\definecolor{currentstroke}{rgb}{1.000000,0.894118,0.709804}%
\pgfsetstrokecolor{currentstroke}%
\pgfsetdash{}{0pt}%
\pgfpathmoveto{\pgfqpoint{0.978205in}{0.678367in}}%
\pgfpathlineto{\pgfqpoint{1.362953in}{0.744350in}}%
\pgfpathlineto{\pgfqpoint{1.747700in}{0.810332in}}%
\pgfpathlineto{\pgfqpoint{2.132447in}{0.876315in}}%
\pgfpathlineto{\pgfqpoint{2.517195in}{0.942298in}}%
\pgfpathlineto{\pgfqpoint{2.901942in}{1.008280in}}%
\pgfpathlineto{\pgfqpoint{3.286690in}{1.074263in}}%
\pgfpathlineto{\pgfqpoint{3.671437in}{1.140246in}}%
\pgfpathlineto{\pgfqpoint{4.056185in}{1.206228in}}%
\pgfpathlineto{\pgfqpoint{4.440932in}{1.272211in}}%
\pgfusepath{stroke}%
\end{pgfscope}%
\begin{pgfscope}%
\pgfpathrectangle{\pgfqpoint{0.772069in}{0.515123in}}{\pgfqpoint{3.875000in}{2.695000in}}%
\pgfusepath{clip}%
\pgfsetrectcap%
\pgfsetroundjoin%
\pgfsetlinewidth{1.505625pt}%
\definecolor{currentstroke}{rgb}{0.564706,0.933333,0.564706}%
\pgfsetstrokecolor{currentstroke}%
\pgfsetdash{}{0pt}%
\pgfpathmoveto{\pgfqpoint{0.978205in}{0.733052in}}%
\pgfpathlineto{\pgfqpoint{1.362953in}{0.852317in}}%
\pgfpathlineto{\pgfqpoint{1.747700in}{0.971582in}}%
\pgfpathlineto{\pgfqpoint{2.132447in}{1.090847in}}%
\pgfpathlineto{\pgfqpoint{2.517195in}{1.210112in}}%
\pgfpathlineto{\pgfqpoint{2.901942in}{1.329377in}}%
\pgfpathlineto{\pgfqpoint{3.286690in}{1.448643in}}%
\pgfpathlineto{\pgfqpoint{3.671437in}{1.567908in}}%
\pgfpathlineto{\pgfqpoint{4.056185in}{1.687173in}}%
\pgfpathlineto{\pgfqpoint{4.440932in}{1.806438in}}%
\pgfusepath{stroke}%
\end{pgfscope}%
\begin{pgfscope}%
\pgfsetbuttcap%
\pgfsetroundjoin%
\definecolor{currentfill}{rgb}{0.000000,0.000000,0.000000}%
\pgfsetfillcolor{currentfill}%
\pgfsetlinewidth{0.803000pt}%
\definecolor{currentstroke}{rgb}{0.000000,0.000000,0.000000}%
\pgfsetstrokecolor{currentstroke}%
\pgfsetdash{}{0pt}%
\pgfsys@defobject{currentmarker}{\pgfqpoint{0.000000in}{-0.048611in}}{\pgfqpoint{0.000000in}{0.000000in}}{%
\pgfpathmoveto{\pgfqpoint{0.000000in}{0.000000in}}%
\pgfpathlineto{\pgfqpoint{0.000000in}{-0.048611in}}%
\pgfusepath{stroke,fill}%
}%
\begin{pgfscope}%
\pgfsys@transformshift{1.190829in}{0.515123in}%
\pgfsys@useobject{currentmarker}{}%
\end{pgfscope}%
\end{pgfscope}%
\begin{pgfscope}%
\definecolor{textcolor}{rgb}{0.000000,0.000000,0.000000}%
\pgfsetstrokecolor{textcolor}%
\pgfsetfillcolor{textcolor}%
\pgftext[x=1.190829in,y=0.417901in,,top]{\color{textcolor}\rmfamily\fontsize{10.000000}{12.000000}\selectfont \(\displaystyle 2\)}%
\end{pgfscope}%
\begin{pgfscope}%
\pgfsetbuttcap%
\pgfsetroundjoin%
\definecolor{currentfill}{rgb}{0.000000,0.000000,0.000000}%
\pgfsetfillcolor{currentfill}%
\pgfsetlinewidth{0.803000pt}%
\definecolor{currentstroke}{rgb}{0.000000,0.000000,0.000000}%
\pgfsetstrokecolor{currentstroke}%
\pgfsetdash{}{0pt}%
\pgfsys@defobject{currentmarker}{\pgfqpoint{0.000000in}{-0.048611in}}{\pgfqpoint{0.000000in}{0.000000in}}{%
\pgfpathmoveto{\pgfqpoint{0.000000in}{0.000000in}}%
\pgfpathlineto{\pgfqpoint{0.000000in}{-0.048611in}}%
\pgfusepath{stroke,fill}%
}%
\begin{pgfscope}%
\pgfsys@transformshift{1.798325in}{0.515123in}%
\pgfsys@useobject{currentmarker}{}%
\end{pgfscope}%
\end{pgfscope}%
\begin{pgfscope}%
\definecolor{textcolor}{rgb}{0.000000,0.000000,0.000000}%
\pgfsetstrokecolor{textcolor}%
\pgfsetfillcolor{textcolor}%
\pgftext[x=1.798325in,y=0.417901in,,top]{\color{textcolor}\rmfamily\fontsize{10.000000}{12.000000}\selectfont \(\displaystyle 4\)}%
\end{pgfscope}%
\begin{pgfscope}%
\pgfsetbuttcap%
\pgfsetroundjoin%
\definecolor{currentfill}{rgb}{0.000000,0.000000,0.000000}%
\pgfsetfillcolor{currentfill}%
\pgfsetlinewidth{0.803000pt}%
\definecolor{currentstroke}{rgb}{0.000000,0.000000,0.000000}%
\pgfsetstrokecolor{currentstroke}%
\pgfsetdash{}{0pt}%
\pgfsys@defobject{currentmarker}{\pgfqpoint{0.000000in}{-0.048611in}}{\pgfqpoint{0.000000in}{0.000000in}}{%
\pgfpathmoveto{\pgfqpoint{0.000000in}{0.000000in}}%
\pgfpathlineto{\pgfqpoint{0.000000in}{-0.048611in}}%
\pgfusepath{stroke,fill}%
}%
\begin{pgfscope}%
\pgfsys@transformshift{2.405821in}{0.515123in}%
\pgfsys@useobject{currentmarker}{}%
\end{pgfscope}%
\end{pgfscope}%
\begin{pgfscope}%
\definecolor{textcolor}{rgb}{0.000000,0.000000,0.000000}%
\pgfsetstrokecolor{textcolor}%
\pgfsetfillcolor{textcolor}%
\pgftext[x=2.405821in,y=0.417901in,,top]{\color{textcolor}\rmfamily\fontsize{10.000000}{12.000000}\selectfont \(\displaystyle 6\)}%
\end{pgfscope}%
\begin{pgfscope}%
\pgfsetbuttcap%
\pgfsetroundjoin%
\definecolor{currentfill}{rgb}{0.000000,0.000000,0.000000}%
\pgfsetfillcolor{currentfill}%
\pgfsetlinewidth{0.803000pt}%
\definecolor{currentstroke}{rgb}{0.000000,0.000000,0.000000}%
\pgfsetstrokecolor{currentstroke}%
\pgfsetdash{}{0pt}%
\pgfsys@defobject{currentmarker}{\pgfqpoint{0.000000in}{-0.048611in}}{\pgfqpoint{0.000000in}{0.000000in}}{%
\pgfpathmoveto{\pgfqpoint{0.000000in}{0.000000in}}%
\pgfpathlineto{\pgfqpoint{0.000000in}{-0.048611in}}%
\pgfusepath{stroke,fill}%
}%
\begin{pgfscope}%
\pgfsys@transformshift{3.013317in}{0.515123in}%
\pgfsys@useobject{currentmarker}{}%
\end{pgfscope}%
\end{pgfscope}%
\begin{pgfscope}%
\definecolor{textcolor}{rgb}{0.000000,0.000000,0.000000}%
\pgfsetstrokecolor{textcolor}%
\pgfsetfillcolor{textcolor}%
\pgftext[x=3.013317in,y=0.417901in,,top]{\color{textcolor}\rmfamily\fontsize{10.000000}{12.000000}\selectfont \(\displaystyle 8\)}%
\end{pgfscope}%
\begin{pgfscope}%
\pgfsetbuttcap%
\pgfsetroundjoin%
\definecolor{currentfill}{rgb}{0.000000,0.000000,0.000000}%
\pgfsetfillcolor{currentfill}%
\pgfsetlinewidth{0.803000pt}%
\definecolor{currentstroke}{rgb}{0.000000,0.000000,0.000000}%
\pgfsetstrokecolor{currentstroke}%
\pgfsetdash{}{0pt}%
\pgfsys@defobject{currentmarker}{\pgfqpoint{0.000000in}{-0.048611in}}{\pgfqpoint{0.000000in}{0.000000in}}{%
\pgfpathmoveto{\pgfqpoint{0.000000in}{0.000000in}}%
\pgfpathlineto{\pgfqpoint{0.000000in}{-0.048611in}}%
\pgfusepath{stroke,fill}%
}%
\begin{pgfscope}%
\pgfsys@transformshift{3.620813in}{0.515123in}%
\pgfsys@useobject{currentmarker}{}%
\end{pgfscope}%
\end{pgfscope}%
\begin{pgfscope}%
\definecolor{textcolor}{rgb}{0.000000,0.000000,0.000000}%
\pgfsetstrokecolor{textcolor}%
\pgfsetfillcolor{textcolor}%
\pgftext[x=3.620813in,y=0.417901in,,top]{\color{textcolor}\rmfamily\fontsize{10.000000}{12.000000}\selectfont \(\displaystyle 10\)}%
\end{pgfscope}%
\begin{pgfscope}%
\pgfsetbuttcap%
\pgfsetroundjoin%
\definecolor{currentfill}{rgb}{0.000000,0.000000,0.000000}%
\pgfsetfillcolor{currentfill}%
\pgfsetlinewidth{0.803000pt}%
\definecolor{currentstroke}{rgb}{0.000000,0.000000,0.000000}%
\pgfsetstrokecolor{currentstroke}%
\pgfsetdash{}{0pt}%
\pgfsys@defobject{currentmarker}{\pgfqpoint{0.000000in}{-0.048611in}}{\pgfqpoint{0.000000in}{0.000000in}}{%
\pgfpathmoveto{\pgfqpoint{0.000000in}{0.000000in}}%
\pgfpathlineto{\pgfqpoint{0.000000in}{-0.048611in}}%
\pgfusepath{stroke,fill}%
}%
\begin{pgfscope}%
\pgfsys@transformshift{4.228309in}{0.515123in}%
\pgfsys@useobject{currentmarker}{}%
\end{pgfscope}%
\end{pgfscope}%
\begin{pgfscope}%
\definecolor{textcolor}{rgb}{0.000000,0.000000,0.000000}%
\pgfsetstrokecolor{textcolor}%
\pgfsetfillcolor{textcolor}%
\pgftext[x=4.228309in,y=0.417901in,,top]{\color{textcolor}\rmfamily\fontsize{10.000000}{12.000000}\selectfont \(\displaystyle 12\)}%
\end{pgfscope}%
\begin{pgfscope}%
\definecolor{textcolor}{rgb}{0.000000,0.000000,0.000000}%
\pgfsetstrokecolor{textcolor}%
\pgfsetfillcolor{textcolor}%
\pgftext[x=2.709569in,y=0.238889in,,top]{\color{textcolor}\rmfamily\fontsize{10.000000}{12.000000}\selectfont I(\(\displaystyle \mu\)A)}%
\end{pgfscope}%
\begin{pgfscope}%
\pgfsetbuttcap%
\pgfsetroundjoin%
\definecolor{currentfill}{rgb}{0.000000,0.000000,0.000000}%
\pgfsetfillcolor{currentfill}%
\pgfsetlinewidth{0.803000pt}%
\definecolor{currentstroke}{rgb}{0.000000,0.000000,0.000000}%
\pgfsetstrokecolor{currentstroke}%
\pgfsetdash{}{0pt}%
\pgfsys@defobject{currentmarker}{\pgfqpoint{-0.048611in}{0.000000in}}{\pgfqpoint{0.000000in}{0.000000in}}{%
\pgfpathmoveto{\pgfqpoint{0.000000in}{0.000000in}}%
\pgfpathlineto{\pgfqpoint{-0.048611in}{0.000000in}}%
\pgfusepath{stroke,fill}%
}%
\begin{pgfscope}%
\pgfsys@transformshift{0.772069in}{0.610648in}%
\pgfsys@useobject{currentmarker}{}%
\end{pgfscope}%
\end{pgfscope}%
\begin{pgfscope}%
\definecolor{textcolor}{rgb}{0.000000,0.000000,0.000000}%
\pgfsetstrokecolor{textcolor}%
\pgfsetfillcolor{textcolor}%
\pgftext[x=0.605402in,y=0.562422in,left,base]{\color{textcolor}\rmfamily\fontsize{10.000000}{12.000000}\selectfont \(\displaystyle 0\)}%
\end{pgfscope}%
\begin{pgfscope}%
\pgfsetbuttcap%
\pgfsetroundjoin%
\definecolor{currentfill}{rgb}{0.000000,0.000000,0.000000}%
\pgfsetfillcolor{currentfill}%
\pgfsetlinewidth{0.803000pt}%
\definecolor{currentstroke}{rgb}{0.000000,0.000000,0.000000}%
\pgfsetstrokecolor{currentstroke}%
\pgfsetdash{}{0pt}%
\pgfsys@defobject{currentmarker}{\pgfqpoint{-0.048611in}{0.000000in}}{\pgfqpoint{0.000000in}{0.000000in}}{%
\pgfpathmoveto{\pgfqpoint{0.000000in}{0.000000in}}%
\pgfpathlineto{\pgfqpoint{-0.048611in}{0.000000in}}%
\pgfusepath{stroke,fill}%
}%
\begin{pgfscope}%
\pgfsys@transformshift{0.772069in}{1.096794in}%
\pgfsys@useobject{currentmarker}{}%
\end{pgfscope}%
\end{pgfscope}%
\begin{pgfscope}%
\definecolor{textcolor}{rgb}{0.000000,0.000000,0.000000}%
\pgfsetstrokecolor{textcolor}%
\pgfsetfillcolor{textcolor}%
\pgftext[x=0.605402in,y=1.048568in,left,base]{\color{textcolor}\rmfamily\fontsize{10.000000}{12.000000}\selectfont \(\displaystyle 2\)}%
\end{pgfscope}%
\begin{pgfscope}%
\pgfsetbuttcap%
\pgfsetroundjoin%
\definecolor{currentfill}{rgb}{0.000000,0.000000,0.000000}%
\pgfsetfillcolor{currentfill}%
\pgfsetlinewidth{0.803000pt}%
\definecolor{currentstroke}{rgb}{0.000000,0.000000,0.000000}%
\pgfsetstrokecolor{currentstroke}%
\pgfsetdash{}{0pt}%
\pgfsys@defobject{currentmarker}{\pgfqpoint{-0.048611in}{0.000000in}}{\pgfqpoint{0.000000in}{0.000000in}}{%
\pgfpathmoveto{\pgfqpoint{0.000000in}{0.000000in}}%
\pgfpathlineto{\pgfqpoint{-0.048611in}{0.000000in}}%
\pgfusepath{stroke,fill}%
}%
\begin{pgfscope}%
\pgfsys@transformshift{0.772069in}{1.582939in}%
\pgfsys@useobject{currentmarker}{}%
\end{pgfscope}%
\end{pgfscope}%
\begin{pgfscope}%
\definecolor{textcolor}{rgb}{0.000000,0.000000,0.000000}%
\pgfsetstrokecolor{textcolor}%
\pgfsetfillcolor{textcolor}%
\pgftext[x=0.605402in,y=1.534714in,left,base]{\color{textcolor}\rmfamily\fontsize{10.000000}{12.000000}\selectfont \(\displaystyle 4\)}%
\end{pgfscope}%
\begin{pgfscope}%
\pgfsetbuttcap%
\pgfsetroundjoin%
\definecolor{currentfill}{rgb}{0.000000,0.000000,0.000000}%
\pgfsetfillcolor{currentfill}%
\pgfsetlinewidth{0.803000pt}%
\definecolor{currentstroke}{rgb}{0.000000,0.000000,0.000000}%
\pgfsetstrokecolor{currentstroke}%
\pgfsetdash{}{0pt}%
\pgfsys@defobject{currentmarker}{\pgfqpoint{-0.048611in}{0.000000in}}{\pgfqpoint{0.000000in}{0.000000in}}{%
\pgfpathmoveto{\pgfqpoint{0.000000in}{0.000000in}}%
\pgfpathlineto{\pgfqpoint{-0.048611in}{0.000000in}}%
\pgfusepath{stroke,fill}%
}%
\begin{pgfscope}%
\pgfsys@transformshift{0.772069in}{2.069085in}%
\pgfsys@useobject{currentmarker}{}%
\end{pgfscope}%
\end{pgfscope}%
\begin{pgfscope}%
\definecolor{textcolor}{rgb}{0.000000,0.000000,0.000000}%
\pgfsetstrokecolor{textcolor}%
\pgfsetfillcolor{textcolor}%
\pgftext[x=0.605402in,y=2.020860in,left,base]{\color{textcolor}\rmfamily\fontsize{10.000000}{12.000000}\selectfont \(\displaystyle 6\)}%
\end{pgfscope}%
\begin{pgfscope}%
\pgfsetbuttcap%
\pgfsetroundjoin%
\definecolor{currentfill}{rgb}{0.000000,0.000000,0.000000}%
\pgfsetfillcolor{currentfill}%
\pgfsetlinewidth{0.803000pt}%
\definecolor{currentstroke}{rgb}{0.000000,0.000000,0.000000}%
\pgfsetstrokecolor{currentstroke}%
\pgfsetdash{}{0pt}%
\pgfsys@defobject{currentmarker}{\pgfqpoint{-0.048611in}{0.000000in}}{\pgfqpoint{0.000000in}{0.000000in}}{%
\pgfpathmoveto{\pgfqpoint{0.000000in}{0.000000in}}%
\pgfpathlineto{\pgfqpoint{-0.048611in}{0.000000in}}%
\pgfusepath{stroke,fill}%
}%
\begin{pgfscope}%
\pgfsys@transformshift{0.772069in}{2.555231in}%
\pgfsys@useobject{currentmarker}{}%
\end{pgfscope}%
\end{pgfscope}%
\begin{pgfscope}%
\definecolor{textcolor}{rgb}{0.000000,0.000000,0.000000}%
\pgfsetstrokecolor{textcolor}%
\pgfsetfillcolor{textcolor}%
\pgftext[x=0.605402in,y=2.507006in,left,base]{\color{textcolor}\rmfamily\fontsize{10.000000}{12.000000}\selectfont \(\displaystyle 8\)}%
\end{pgfscope}%
\begin{pgfscope}%
\pgfsetbuttcap%
\pgfsetroundjoin%
\definecolor{currentfill}{rgb}{0.000000,0.000000,0.000000}%
\pgfsetfillcolor{currentfill}%
\pgfsetlinewidth{0.803000pt}%
\definecolor{currentstroke}{rgb}{0.000000,0.000000,0.000000}%
\pgfsetstrokecolor{currentstroke}%
\pgfsetdash{}{0pt}%
\pgfsys@defobject{currentmarker}{\pgfqpoint{-0.048611in}{0.000000in}}{\pgfqpoint{0.000000in}{0.000000in}}{%
\pgfpathmoveto{\pgfqpoint{0.000000in}{0.000000in}}%
\pgfpathlineto{\pgfqpoint{-0.048611in}{0.000000in}}%
\pgfusepath{stroke,fill}%
}%
\begin{pgfscope}%
\pgfsys@transformshift{0.772069in}{3.041377in}%
\pgfsys@useobject{currentmarker}{}%
\end{pgfscope}%
\end{pgfscope}%
\begin{pgfscope}%
\definecolor{textcolor}{rgb}{0.000000,0.000000,0.000000}%
\pgfsetstrokecolor{textcolor}%
\pgfsetfillcolor{textcolor}%
\pgftext[x=0.535957in,y=2.993152in,left,base]{\color{textcolor}\rmfamily\fontsize{10.000000}{12.000000}\selectfont \(\displaystyle 10\)}%
\end{pgfscope}%
\begin{pgfscope}%
\definecolor{textcolor}{rgb}{0.000000,0.000000,0.000000}%
\pgfsetstrokecolor{textcolor}%
\pgfsetfillcolor{textcolor}%
\pgftext[x=0.258179in,y=1.862623in,,bottom]{\color{textcolor}\rmfamily\fontsize{10.000000}{12.000000}\selectfont V(V)}%
\end{pgfscope}%
\begin{pgfscope}%
\pgfpathrectangle{\pgfqpoint{0.772069in}{0.515123in}}{\pgfqpoint{3.875000in}{2.695000in}}%
\pgfusepath{clip}%
\pgfsetbuttcap%
\pgfsetroundjoin%
\definecolor{currentfill}{rgb}{0.121569,0.466667,0.705882}%
\pgfsetfillcolor{currentfill}%
\pgfsetlinewidth{1.003750pt}%
\definecolor{currentstroke}{rgb}{0.121569,0.466667,0.705882}%
\pgfsetstrokecolor{currentstroke}%
\pgfsetdash{}{0pt}%
\pgfpathmoveto{\pgfqpoint{0.978205in}{0.827368in}}%
\pgfpathcurveto{\pgfqpoint{0.989255in}{0.827368in}}{\pgfqpoint{0.999854in}{0.831758in}}{\pgfqpoint{1.007668in}{0.839571in}}%
\pgfpathcurveto{\pgfqpoint{1.015481in}{0.847385in}}{\pgfqpoint{1.019872in}{0.857984in}}{\pgfqpoint{1.019872in}{0.869034in}}%
\pgfpathcurveto{\pgfqpoint{1.019872in}{0.880084in}}{\pgfqpoint{1.015481in}{0.890683in}}{\pgfqpoint{1.007668in}{0.898497in}}%
\pgfpathcurveto{\pgfqpoint{0.999854in}{0.906311in}}{\pgfqpoint{0.989255in}{0.910701in}}{\pgfqpoint{0.978205in}{0.910701in}}%
\pgfpathcurveto{\pgfqpoint{0.967155in}{0.910701in}}{\pgfqpoint{0.956556in}{0.906311in}}{\pgfqpoint{0.948742in}{0.898497in}}%
\pgfpathcurveto{\pgfqpoint{0.940929in}{0.890683in}}{\pgfqpoint{0.936538in}{0.880084in}}{\pgfqpoint{0.936538in}{0.869034in}}%
\pgfpathcurveto{\pgfqpoint{0.936538in}{0.857984in}}{\pgfqpoint{0.940929in}{0.847385in}}{\pgfqpoint{0.948742in}{0.839571in}}%
\pgfpathcurveto{\pgfqpoint{0.956556in}{0.831758in}}{\pgfqpoint{0.967155in}{0.827368in}}{\pgfqpoint{0.978205in}{0.827368in}}%
\pgfpathclose%
\pgfusepath{stroke,fill}%
\end{pgfscope}%
\begin{pgfscope}%
\pgfpathrectangle{\pgfqpoint{0.772069in}{0.515123in}}{\pgfqpoint{3.875000in}{2.695000in}}%
\pgfusepath{clip}%
\pgfsetbuttcap%
\pgfsetroundjoin%
\definecolor{currentfill}{rgb}{0.121569,0.466667,0.705882}%
\pgfsetfillcolor{currentfill}%
\pgfsetlinewidth{1.003750pt}%
\definecolor{currentstroke}{rgb}{0.121569,0.466667,0.705882}%
\pgfsetstrokecolor{currentstroke}%
\pgfsetdash{}{0pt}%
\pgfpathmoveto{\pgfqpoint{1.342703in}{1.067281in}}%
\pgfpathcurveto{\pgfqpoint{1.353753in}{1.067281in}}{\pgfqpoint{1.364352in}{1.071671in}}{\pgfqpoint{1.372165in}{1.079484in}}%
\pgfpathcurveto{\pgfqpoint{1.379979in}{1.087298in}}{\pgfqpoint{1.384369in}{1.097897in}}{\pgfqpoint{1.384369in}{1.108947in}}%
\pgfpathcurveto{\pgfqpoint{1.384369in}{1.119997in}}{\pgfqpoint{1.379979in}{1.130596in}}{\pgfqpoint{1.372165in}{1.138410in}}%
\pgfpathcurveto{\pgfqpoint{1.364352in}{1.146224in}}{\pgfqpoint{1.353753in}{1.150614in}}{\pgfqpoint{1.342703in}{1.150614in}}%
\pgfpathcurveto{\pgfqpoint{1.331653in}{1.150614in}}{\pgfqpoint{1.321053in}{1.146224in}}{\pgfqpoint{1.313240in}{1.138410in}}%
\pgfpathcurveto{\pgfqpoint{1.305426in}{1.130596in}}{\pgfqpoint{1.301036in}{1.119997in}}{\pgfqpoint{1.301036in}{1.108947in}}%
\pgfpathcurveto{\pgfqpoint{1.301036in}{1.097897in}}{\pgfqpoint{1.305426in}{1.087298in}}{\pgfqpoint{1.313240in}{1.079484in}}%
\pgfpathcurveto{\pgfqpoint{1.321053in}{1.071671in}}{\pgfqpoint{1.331653in}{1.067281in}}{\pgfqpoint{1.342703in}{1.067281in}}%
\pgfpathclose%
\pgfusepath{stroke,fill}%
\end{pgfscope}%
\begin{pgfscope}%
\pgfpathrectangle{\pgfqpoint{0.772069in}{0.515123in}}{\pgfqpoint{3.875000in}{2.695000in}}%
\pgfusepath{clip}%
\pgfsetbuttcap%
\pgfsetroundjoin%
\definecolor{currentfill}{rgb}{0.121569,0.466667,0.705882}%
\pgfsetfillcolor{currentfill}%
\pgfsetlinewidth{1.003750pt}%
\definecolor{currentstroke}{rgb}{0.121569,0.466667,0.705882}%
\pgfsetstrokecolor{currentstroke}%
\pgfsetdash{}{0pt}%
\pgfpathmoveto{\pgfqpoint{1.737575in}{1.307923in}}%
\pgfpathcurveto{\pgfqpoint{1.748625in}{1.307923in}}{\pgfqpoint{1.759224in}{1.312313in}}{\pgfqpoint{1.767038in}{1.320127in}}%
\pgfpathcurveto{\pgfqpoint{1.774851in}{1.327940in}}{\pgfqpoint{1.779242in}{1.338539in}}{\pgfqpoint{1.779242in}{1.349589in}}%
\pgfpathcurveto{\pgfqpoint{1.779242in}{1.360640in}}{\pgfqpoint{1.774851in}{1.371239in}}{\pgfqpoint{1.767038in}{1.379052in}}%
\pgfpathcurveto{\pgfqpoint{1.759224in}{1.386866in}}{\pgfqpoint{1.748625in}{1.391256in}}{\pgfqpoint{1.737575in}{1.391256in}}%
\pgfpathcurveto{\pgfqpoint{1.726525in}{1.391256in}}{\pgfqpoint{1.715926in}{1.386866in}}{\pgfqpoint{1.708112in}{1.379052in}}%
\pgfpathcurveto{\pgfqpoint{1.700299in}{1.371239in}}{\pgfqpoint{1.695908in}{1.360640in}}{\pgfqpoint{1.695908in}{1.349589in}}%
\pgfpathcurveto{\pgfqpoint{1.695908in}{1.338539in}}{\pgfqpoint{1.700299in}{1.327940in}}{\pgfqpoint{1.708112in}{1.320127in}}%
\pgfpathcurveto{\pgfqpoint{1.715926in}{1.312313in}}{\pgfqpoint{1.726525in}{1.307923in}}{\pgfqpoint{1.737575in}{1.307923in}}%
\pgfpathclose%
\pgfusepath{stroke,fill}%
\end{pgfscope}%
\begin{pgfscope}%
\pgfpathrectangle{\pgfqpoint{0.772069in}{0.515123in}}{\pgfqpoint{3.875000in}{2.695000in}}%
\pgfusepath{clip}%
\pgfsetbuttcap%
\pgfsetroundjoin%
\definecolor{currentfill}{rgb}{0.121569,0.466667,0.705882}%
\pgfsetfillcolor{currentfill}%
\pgfsetlinewidth{1.003750pt}%
\definecolor{currentstroke}{rgb}{0.121569,0.466667,0.705882}%
\pgfsetstrokecolor{currentstroke}%
\pgfsetdash{}{0pt}%
\pgfpathmoveto{\pgfqpoint{2.132447in}{1.555857in}}%
\pgfpathcurveto{\pgfqpoint{2.143498in}{1.555857in}}{\pgfqpoint{2.154097in}{1.560247in}}{\pgfqpoint{2.161910in}{1.568061in}}%
\pgfpathcurveto{\pgfqpoint{2.169724in}{1.575875in}}{\pgfqpoint{2.174114in}{1.586474in}}{\pgfqpoint{2.174114in}{1.597524in}}%
\pgfpathcurveto{\pgfqpoint{2.174114in}{1.608574in}}{\pgfqpoint{2.169724in}{1.619173in}}{\pgfqpoint{2.161910in}{1.626987in}}%
\pgfpathcurveto{\pgfqpoint{2.154097in}{1.634800in}}{\pgfqpoint{2.143498in}{1.639190in}}{\pgfqpoint{2.132447in}{1.639190in}}%
\pgfpathcurveto{\pgfqpoint{2.121397in}{1.639190in}}{\pgfqpoint{2.110798in}{1.634800in}}{\pgfqpoint{2.102985in}{1.626987in}}%
\pgfpathcurveto{\pgfqpoint{2.095171in}{1.619173in}}{\pgfqpoint{2.090781in}{1.608574in}}{\pgfqpoint{2.090781in}{1.597524in}}%
\pgfpathcurveto{\pgfqpoint{2.090781in}{1.586474in}}{\pgfqpoint{2.095171in}{1.575875in}}{\pgfqpoint{2.102985in}{1.568061in}}%
\pgfpathcurveto{\pgfqpoint{2.110798in}{1.560247in}}{\pgfqpoint{2.121397in}{1.555857in}}{\pgfqpoint{2.132447in}{1.555857in}}%
\pgfpathclose%
\pgfusepath{stroke,fill}%
\end{pgfscope}%
\begin{pgfscope}%
\pgfpathrectangle{\pgfqpoint{0.772069in}{0.515123in}}{\pgfqpoint{3.875000in}{2.695000in}}%
\pgfusepath{clip}%
\pgfsetbuttcap%
\pgfsetroundjoin%
\definecolor{currentfill}{rgb}{0.121569,0.466667,0.705882}%
\pgfsetfillcolor{currentfill}%
\pgfsetlinewidth{1.003750pt}%
\definecolor{currentstroke}{rgb}{0.121569,0.466667,0.705882}%
\pgfsetstrokecolor{currentstroke}%
\pgfsetdash{}{0pt}%
\pgfpathmoveto{\pgfqpoint{2.496945in}{1.794069in}}%
\pgfpathcurveto{\pgfqpoint{2.507995in}{1.794069in}}{\pgfqpoint{2.518594in}{1.798459in}}{\pgfqpoint{2.526408in}{1.806272in}}%
\pgfpathcurveto{\pgfqpoint{2.534221in}{1.814086in}}{\pgfqpoint{2.538612in}{1.824685in}}{\pgfqpoint{2.538612in}{1.835735in}}%
\pgfpathcurveto{\pgfqpoint{2.538612in}{1.846785in}}{\pgfqpoint{2.534221in}{1.857384in}}{\pgfqpoint{2.526408in}{1.865198in}}%
\pgfpathcurveto{\pgfqpoint{2.518594in}{1.873012in}}{\pgfqpoint{2.507995in}{1.877402in}}{\pgfqpoint{2.496945in}{1.877402in}}%
\pgfpathcurveto{\pgfqpoint{2.485895in}{1.877402in}}{\pgfqpoint{2.475296in}{1.873012in}}{\pgfqpoint{2.467482in}{1.865198in}}%
\pgfpathcurveto{\pgfqpoint{2.459669in}{1.857384in}}{\pgfqpoint{2.455278in}{1.846785in}}{\pgfqpoint{2.455278in}{1.835735in}}%
\pgfpathcurveto{\pgfqpoint{2.455278in}{1.824685in}}{\pgfqpoint{2.459669in}{1.814086in}}{\pgfqpoint{2.467482in}{1.806272in}}%
\pgfpathcurveto{\pgfqpoint{2.475296in}{1.798459in}}{\pgfqpoint{2.485895in}{1.794069in}}{\pgfqpoint{2.496945in}{1.794069in}}%
\pgfpathclose%
\pgfusepath{stroke,fill}%
\end{pgfscope}%
\begin{pgfscope}%
\pgfpathrectangle{\pgfqpoint{0.772069in}{0.515123in}}{\pgfqpoint{3.875000in}{2.695000in}}%
\pgfusepath{clip}%
\pgfsetbuttcap%
\pgfsetroundjoin%
\definecolor{currentfill}{rgb}{0.121569,0.466667,0.705882}%
\pgfsetfillcolor{currentfill}%
\pgfsetlinewidth{1.003750pt}%
\definecolor{currentstroke}{rgb}{0.121569,0.466667,0.705882}%
\pgfsetstrokecolor{currentstroke}%
\pgfsetdash{}{0pt}%
\pgfpathmoveto{\pgfqpoint{2.922192in}{2.056587in}}%
\pgfpathcurveto{\pgfqpoint{2.933242in}{2.056587in}}{\pgfqpoint{2.943841in}{2.060978in}}{\pgfqpoint{2.951655in}{2.068791in}}%
\pgfpathcurveto{\pgfqpoint{2.959469in}{2.076605in}}{\pgfqpoint{2.963859in}{2.087204in}}{\pgfqpoint{2.963859in}{2.098254in}}%
\pgfpathcurveto{\pgfqpoint{2.963859in}{2.109304in}}{\pgfqpoint{2.959469in}{2.119903in}}{\pgfqpoint{2.951655in}{2.127717in}}%
\pgfpathcurveto{\pgfqpoint{2.943841in}{2.135530in}}{\pgfqpoint{2.933242in}{2.139921in}}{\pgfqpoint{2.922192in}{2.139921in}}%
\pgfpathcurveto{\pgfqpoint{2.911142in}{2.139921in}}{\pgfqpoint{2.900543in}{2.135530in}}{\pgfqpoint{2.892730in}{2.127717in}}%
\pgfpathcurveto{\pgfqpoint{2.884916in}{2.119903in}}{\pgfqpoint{2.880526in}{2.109304in}}{\pgfqpoint{2.880526in}{2.098254in}}%
\pgfpathcurveto{\pgfqpoint{2.880526in}{2.087204in}}{\pgfqpoint{2.884916in}{2.076605in}}{\pgfqpoint{2.892730in}{2.068791in}}%
\pgfpathcurveto{\pgfqpoint{2.900543in}{2.060978in}}{\pgfqpoint{2.911142in}{2.056587in}}{\pgfqpoint{2.922192in}{2.056587in}}%
\pgfpathclose%
\pgfusepath{stroke,fill}%
\end{pgfscope}%
\begin{pgfscope}%
\pgfpathrectangle{\pgfqpoint{0.772069in}{0.515123in}}{\pgfqpoint{3.875000in}{2.695000in}}%
\pgfusepath{clip}%
\pgfsetbuttcap%
\pgfsetroundjoin%
\definecolor{currentfill}{rgb}{0.121569,0.466667,0.705882}%
\pgfsetfillcolor{currentfill}%
\pgfsetlinewidth{1.003750pt}%
\definecolor{currentstroke}{rgb}{0.121569,0.466667,0.705882}%
\pgfsetstrokecolor{currentstroke}%
\pgfsetdash{}{0pt}%
\pgfpathmoveto{\pgfqpoint{3.286690in}{2.289937in}}%
\pgfpathcurveto{\pgfqpoint{3.297740in}{2.289937in}}{\pgfqpoint{3.308339in}{2.294328in}}{\pgfqpoint{3.316153in}{2.302141in}}%
\pgfpathcurveto{\pgfqpoint{3.323966in}{2.309955in}}{\pgfqpoint{3.328357in}{2.320554in}}{\pgfqpoint{3.328357in}{2.331604in}}%
\pgfpathcurveto{\pgfqpoint{3.328357in}{2.342654in}}{\pgfqpoint{3.323966in}{2.353253in}}{\pgfqpoint{3.316153in}{2.361067in}}%
\pgfpathcurveto{\pgfqpoint{3.308339in}{2.368880in}}{\pgfqpoint{3.297740in}{2.373271in}}{\pgfqpoint{3.286690in}{2.373271in}}%
\pgfpathcurveto{\pgfqpoint{3.275640in}{2.373271in}}{\pgfqpoint{3.265041in}{2.368880in}}{\pgfqpoint{3.257227in}{2.361067in}}%
\pgfpathcurveto{\pgfqpoint{3.249413in}{2.353253in}}{\pgfqpoint{3.245023in}{2.342654in}}{\pgfqpoint{3.245023in}{2.331604in}}%
\pgfpathcurveto{\pgfqpoint{3.245023in}{2.320554in}}{\pgfqpoint{3.249413in}{2.309955in}}{\pgfqpoint{3.257227in}{2.302141in}}%
\pgfpathcurveto{\pgfqpoint{3.265041in}{2.294328in}}{\pgfqpoint{3.275640in}{2.289937in}}{\pgfqpoint{3.286690in}{2.289937in}}%
\pgfpathclose%
\pgfusepath{stroke,fill}%
\end{pgfscope}%
\begin{pgfscope}%
\pgfpathrectangle{\pgfqpoint{0.772069in}{0.515123in}}{\pgfqpoint{3.875000in}{2.695000in}}%
\pgfusepath{clip}%
\pgfsetbuttcap%
\pgfsetroundjoin%
\definecolor{currentfill}{rgb}{0.121569,0.466667,0.705882}%
\pgfsetfillcolor{currentfill}%
\pgfsetlinewidth{1.003750pt}%
\definecolor{currentstroke}{rgb}{0.121569,0.466667,0.705882}%
\pgfsetstrokecolor{currentstroke}%
\pgfsetdash{}{0pt}%
\pgfpathmoveto{\pgfqpoint{3.681562in}{2.530579in}}%
\pgfpathcurveto{\pgfqpoint{3.692612in}{2.530579in}}{\pgfqpoint{3.703211in}{2.534970in}}{\pgfqpoint{3.711025in}{2.542783in}}%
\pgfpathcurveto{\pgfqpoint{3.718839in}{2.550597in}}{\pgfqpoint{3.723229in}{2.561196in}}{\pgfqpoint{3.723229in}{2.572246in}}%
\pgfpathcurveto{\pgfqpoint{3.723229in}{2.583296in}}{\pgfqpoint{3.718839in}{2.593895in}}{\pgfqpoint{3.711025in}{2.601709in}}%
\pgfpathcurveto{\pgfqpoint{3.703211in}{2.609523in}}{\pgfqpoint{3.692612in}{2.613913in}}{\pgfqpoint{3.681562in}{2.613913in}}%
\pgfpathcurveto{\pgfqpoint{3.670512in}{2.613913in}}{\pgfqpoint{3.659913in}{2.609523in}}{\pgfqpoint{3.652100in}{2.601709in}}%
\pgfpathcurveto{\pgfqpoint{3.644286in}{2.593895in}}{\pgfqpoint{3.639896in}{2.583296in}}{\pgfqpoint{3.639896in}{2.572246in}}%
\pgfpathcurveto{\pgfqpoint{3.639896in}{2.561196in}}{\pgfqpoint{3.644286in}{2.550597in}}{\pgfqpoint{3.652100in}{2.542783in}}%
\pgfpathcurveto{\pgfqpoint{3.659913in}{2.534970in}}{\pgfqpoint{3.670512in}{2.530579in}}{\pgfqpoint{3.681562in}{2.530579in}}%
\pgfpathclose%
\pgfusepath{stroke,fill}%
\end{pgfscope}%
\begin{pgfscope}%
\pgfpathrectangle{\pgfqpoint{0.772069in}{0.515123in}}{\pgfqpoint{3.875000in}{2.695000in}}%
\pgfusepath{clip}%
\pgfsetbuttcap%
\pgfsetroundjoin%
\definecolor{currentfill}{rgb}{0.121569,0.466667,0.705882}%
\pgfsetfillcolor{currentfill}%
\pgfsetlinewidth{1.003750pt}%
\definecolor{currentstroke}{rgb}{0.121569,0.466667,0.705882}%
\pgfsetstrokecolor{currentstroke}%
\pgfsetdash{}{0pt}%
\pgfpathmoveto{\pgfqpoint{4.015685in}{2.756637in}}%
\pgfpathcurveto{\pgfqpoint{4.026735in}{2.756637in}}{\pgfqpoint{4.037334in}{2.761028in}}{\pgfqpoint{4.045148in}{2.768841in}}%
\pgfpathcurveto{\pgfqpoint{4.052962in}{2.776655in}}{\pgfqpoint{4.057352in}{2.787254in}}{\pgfqpoint{4.057352in}{2.798304in}}%
\pgfpathcurveto{\pgfqpoint{4.057352in}{2.809354in}}{\pgfqpoint{4.052962in}{2.819953in}}{\pgfqpoint{4.045148in}{2.827767in}}%
\pgfpathcurveto{\pgfqpoint{4.037334in}{2.835580in}}{\pgfqpoint{4.026735in}{2.839971in}}{\pgfqpoint{4.015685in}{2.839971in}}%
\pgfpathcurveto{\pgfqpoint{4.004635in}{2.839971in}}{\pgfqpoint{3.994036in}{2.835580in}}{\pgfqpoint{3.986222in}{2.827767in}}%
\pgfpathcurveto{\pgfqpoint{3.978409in}{2.819953in}}{\pgfqpoint{3.974018in}{2.809354in}}{\pgfqpoint{3.974018in}{2.798304in}}%
\pgfpathcurveto{\pgfqpoint{3.974018in}{2.787254in}}{\pgfqpoint{3.978409in}{2.776655in}}{\pgfqpoint{3.986222in}{2.768841in}}%
\pgfpathcurveto{\pgfqpoint{3.994036in}{2.761028in}}{\pgfqpoint{4.004635in}{2.756637in}}{\pgfqpoint{4.015685in}{2.756637in}}%
\pgfpathclose%
\pgfusepath{stroke,fill}%
\end{pgfscope}%
\begin{pgfscope}%
\pgfpathrectangle{\pgfqpoint{0.772069in}{0.515123in}}{\pgfqpoint{3.875000in}{2.695000in}}%
\pgfusepath{clip}%
\pgfsetbuttcap%
\pgfsetroundjoin%
\definecolor{currentfill}{rgb}{0.121569,0.466667,0.705882}%
\pgfsetfillcolor{currentfill}%
\pgfsetlinewidth{1.003750pt}%
\definecolor{currentstroke}{rgb}{0.121569,0.466667,0.705882}%
\pgfsetstrokecolor{currentstroke}%
\pgfsetdash{}{0pt}%
\pgfpathmoveto{\pgfqpoint{4.440932in}{3.019156in}}%
\pgfpathcurveto{\pgfqpoint{4.451982in}{3.019156in}}{\pgfqpoint{4.462581in}{3.023546in}}{\pgfqpoint{4.470395in}{3.031360in}}%
\pgfpathcurveto{\pgfqpoint{4.478209in}{3.039174in}}{\pgfqpoint{4.482599in}{3.049773in}}{\pgfqpoint{4.482599in}{3.060823in}}%
\pgfpathcurveto{\pgfqpoint{4.482599in}{3.071873in}}{\pgfqpoint{4.478209in}{3.082472in}}{\pgfqpoint{4.470395in}{3.090285in}}%
\pgfpathcurveto{\pgfqpoint{4.462581in}{3.098099in}}{\pgfqpoint{4.451982in}{3.102489in}}{\pgfqpoint{4.440932in}{3.102489in}}%
\pgfpathcurveto{\pgfqpoint{4.429882in}{3.102489in}}{\pgfqpoint{4.419283in}{3.098099in}}{\pgfqpoint{4.411470in}{3.090285in}}%
\pgfpathcurveto{\pgfqpoint{4.403656in}{3.082472in}}{\pgfqpoint{4.399266in}{3.071873in}}{\pgfqpoint{4.399266in}{3.060823in}}%
\pgfpathcurveto{\pgfqpoint{4.399266in}{3.049773in}}{\pgfqpoint{4.403656in}{3.039174in}}{\pgfqpoint{4.411470in}{3.031360in}}%
\pgfpathcurveto{\pgfqpoint{4.419283in}{3.023546in}}{\pgfqpoint{4.429882in}{3.019156in}}{\pgfqpoint{4.440932in}{3.019156in}}%
\pgfpathclose%
\pgfusepath{stroke,fill}%
\end{pgfscope}%
\begin{pgfscope}%
\pgfpathrectangle{\pgfqpoint{0.772069in}{0.515123in}}{\pgfqpoint{3.875000in}{2.695000in}}%
\pgfusepath{clip}%
\pgfsetbuttcap%
\pgfsetroundjoin%
\definecolor{currentfill}{rgb}{1.000000,0.388235,0.278431}%
\pgfsetfillcolor{currentfill}%
\pgfsetlinewidth{1.003750pt}%
\definecolor{currentstroke}{rgb}{1.000000,0.388235,0.278431}%
\pgfsetstrokecolor{currentstroke}%
\pgfsetdash{}{0pt}%
\pgfpathmoveto{\pgfqpoint{0.978205in}{0.625617in}}%
\pgfpathcurveto{\pgfqpoint{0.989255in}{0.625617in}}{\pgfqpoint{0.999854in}{0.630007in}}{\pgfqpoint{1.007668in}{0.637821in}}%
\pgfpathcurveto{\pgfqpoint{1.015481in}{0.645635in}}{\pgfqpoint{1.019872in}{0.656234in}}{\pgfqpoint{1.019872in}{0.667284in}}%
\pgfpathcurveto{\pgfqpoint{1.019872in}{0.678334in}}{\pgfqpoint{1.015481in}{0.688933in}}{\pgfqpoint{1.007668in}{0.696747in}}%
\pgfpathcurveto{\pgfqpoint{0.999854in}{0.704560in}}{\pgfqpoint{0.989255in}{0.708950in}}{\pgfqpoint{0.978205in}{0.708950in}}%
\pgfpathcurveto{\pgfqpoint{0.967155in}{0.708950in}}{\pgfqpoint{0.956556in}{0.704560in}}{\pgfqpoint{0.948742in}{0.696747in}}%
\pgfpathcurveto{\pgfqpoint{0.940929in}{0.688933in}}{\pgfqpoint{0.936538in}{0.678334in}}{\pgfqpoint{0.936538in}{0.667284in}}%
\pgfpathcurveto{\pgfqpoint{0.936538in}{0.656234in}}{\pgfqpoint{0.940929in}{0.645635in}}{\pgfqpoint{0.948742in}{0.637821in}}%
\pgfpathcurveto{\pgfqpoint{0.956556in}{0.630007in}}{\pgfqpoint{0.967155in}{0.625617in}}{\pgfqpoint{0.978205in}{0.625617in}}%
\pgfpathclose%
\pgfusepath{stroke,fill}%
\end{pgfscope}%
\begin{pgfscope}%
\pgfpathrectangle{\pgfqpoint{0.772069in}{0.515123in}}{\pgfqpoint{3.875000in}{2.695000in}}%
\pgfusepath{clip}%
\pgfsetbuttcap%
\pgfsetroundjoin%
\definecolor{currentfill}{rgb}{1.000000,0.388235,0.278431}%
\pgfsetfillcolor{currentfill}%
\pgfsetlinewidth{1.003750pt}%
\definecolor{currentstroke}{rgb}{1.000000,0.388235,0.278431}%
\pgfsetstrokecolor{currentstroke}%
\pgfsetdash{}{0pt}%
\pgfpathmoveto{\pgfqpoint{1.342703in}{0.678607in}}%
\pgfpathcurveto{\pgfqpoint{1.353753in}{0.678607in}}{\pgfqpoint{1.364352in}{0.682997in}}{\pgfqpoint{1.372165in}{0.690811in}}%
\pgfpathcurveto{\pgfqpoint{1.379979in}{0.698624in}}{\pgfqpoint{1.384369in}{0.709224in}}{\pgfqpoint{1.384369in}{0.720274in}}%
\pgfpathcurveto{\pgfqpoint{1.384369in}{0.731324in}}{\pgfqpoint{1.379979in}{0.741923in}}{\pgfqpoint{1.372165in}{0.749736in}}%
\pgfpathcurveto{\pgfqpoint{1.364352in}{0.757550in}}{\pgfqpoint{1.353753in}{0.761940in}}{\pgfqpoint{1.342703in}{0.761940in}}%
\pgfpathcurveto{\pgfqpoint{1.331653in}{0.761940in}}{\pgfqpoint{1.321053in}{0.757550in}}{\pgfqpoint{1.313240in}{0.749736in}}%
\pgfpathcurveto{\pgfqpoint{1.305426in}{0.741923in}}{\pgfqpoint{1.301036in}{0.731324in}}{\pgfqpoint{1.301036in}{0.720274in}}%
\pgfpathcurveto{\pgfqpoint{1.301036in}{0.709224in}}{\pgfqpoint{1.305426in}{0.698624in}}{\pgfqpoint{1.313240in}{0.690811in}}%
\pgfpathcurveto{\pgfqpoint{1.321053in}{0.682997in}}{\pgfqpoint{1.331653in}{0.678607in}}{\pgfqpoint{1.342703in}{0.678607in}}%
\pgfpathclose%
\pgfusepath{stroke,fill}%
\end{pgfscope}%
\begin{pgfscope}%
\pgfpathrectangle{\pgfqpoint{0.772069in}{0.515123in}}{\pgfqpoint{3.875000in}{2.695000in}}%
\pgfusepath{clip}%
\pgfsetbuttcap%
\pgfsetroundjoin%
\definecolor{currentfill}{rgb}{1.000000,0.388235,0.278431}%
\pgfsetfillcolor{currentfill}%
\pgfsetlinewidth{1.003750pt}%
\definecolor{currentstroke}{rgb}{1.000000,0.388235,0.278431}%
\pgfsetstrokecolor{currentstroke}%
\pgfsetdash{}{0pt}%
\pgfpathmoveto{\pgfqpoint{1.737575in}{0.732083in}}%
\pgfpathcurveto{\pgfqpoint{1.748625in}{0.732083in}}{\pgfqpoint{1.759224in}{0.736473in}}{\pgfqpoint{1.767038in}{0.744287in}}%
\pgfpathcurveto{\pgfqpoint{1.774851in}{0.752101in}}{\pgfqpoint{1.779242in}{0.762700in}}{\pgfqpoint{1.779242in}{0.773750in}}%
\pgfpathcurveto{\pgfqpoint{1.779242in}{0.784800in}}{\pgfqpoint{1.774851in}{0.795399in}}{\pgfqpoint{1.767038in}{0.803212in}}%
\pgfpathcurveto{\pgfqpoint{1.759224in}{0.811026in}}{\pgfqpoint{1.748625in}{0.815416in}}{\pgfqpoint{1.737575in}{0.815416in}}%
\pgfpathcurveto{\pgfqpoint{1.726525in}{0.815416in}}{\pgfqpoint{1.715926in}{0.811026in}}{\pgfqpoint{1.708112in}{0.803212in}}%
\pgfpathcurveto{\pgfqpoint{1.700299in}{0.795399in}}{\pgfqpoint{1.695908in}{0.784800in}}{\pgfqpoint{1.695908in}{0.773750in}}%
\pgfpathcurveto{\pgfqpoint{1.695908in}{0.762700in}}{\pgfqpoint{1.700299in}{0.752101in}}{\pgfqpoint{1.708112in}{0.744287in}}%
\pgfpathcurveto{\pgfqpoint{1.715926in}{0.736473in}}{\pgfqpoint{1.726525in}{0.732083in}}{\pgfqpoint{1.737575in}{0.732083in}}%
\pgfpathclose%
\pgfusepath{stroke,fill}%
\end{pgfscope}%
\begin{pgfscope}%
\pgfpathrectangle{\pgfqpoint{0.772069in}{0.515123in}}{\pgfqpoint{3.875000in}{2.695000in}}%
\pgfusepath{clip}%
\pgfsetbuttcap%
\pgfsetroundjoin%
\definecolor{currentfill}{rgb}{1.000000,0.388235,0.278431}%
\pgfsetfillcolor{currentfill}%
\pgfsetlinewidth{1.003750pt}%
\definecolor{currentstroke}{rgb}{1.000000,0.388235,0.278431}%
\pgfsetstrokecolor{currentstroke}%
\pgfsetdash{}{0pt}%
\pgfpathmoveto{\pgfqpoint{2.132447in}{0.786288in}}%
\pgfpathcurveto{\pgfqpoint{2.143498in}{0.786288in}}{\pgfqpoint{2.154097in}{0.790679in}}{\pgfqpoint{2.161910in}{0.798492in}}%
\pgfpathcurveto{\pgfqpoint{2.169724in}{0.806306in}}{\pgfqpoint{2.174114in}{0.816905in}}{\pgfqpoint{2.174114in}{0.827955in}}%
\pgfpathcurveto{\pgfqpoint{2.174114in}{0.839005in}}{\pgfqpoint{2.169724in}{0.849604in}}{\pgfqpoint{2.161910in}{0.857418in}}%
\pgfpathcurveto{\pgfqpoint{2.154097in}{0.865231in}}{\pgfqpoint{2.143498in}{0.869622in}}{\pgfqpoint{2.132447in}{0.869622in}}%
\pgfpathcurveto{\pgfqpoint{2.121397in}{0.869622in}}{\pgfqpoint{2.110798in}{0.865231in}}{\pgfqpoint{2.102985in}{0.857418in}}%
\pgfpathcurveto{\pgfqpoint{2.095171in}{0.849604in}}{\pgfqpoint{2.090781in}{0.839005in}}{\pgfqpoint{2.090781in}{0.827955in}}%
\pgfpathcurveto{\pgfqpoint{2.090781in}{0.816905in}}{\pgfqpoint{2.095171in}{0.806306in}}{\pgfqpoint{2.102985in}{0.798492in}}%
\pgfpathcurveto{\pgfqpoint{2.110798in}{0.790679in}}{\pgfqpoint{2.121397in}{0.786288in}}{\pgfqpoint{2.132447in}{0.786288in}}%
\pgfpathclose%
\pgfusepath{stroke,fill}%
\end{pgfscope}%
\begin{pgfscope}%
\pgfpathrectangle{\pgfqpoint{0.772069in}{0.515123in}}{\pgfqpoint{3.875000in}{2.695000in}}%
\pgfusepath{clip}%
\pgfsetbuttcap%
\pgfsetroundjoin%
\definecolor{currentfill}{rgb}{1.000000,0.388235,0.278431}%
\pgfsetfillcolor{currentfill}%
\pgfsetlinewidth{1.003750pt}%
\definecolor{currentstroke}{rgb}{1.000000,0.388235,0.278431}%
\pgfsetstrokecolor{currentstroke}%
\pgfsetdash{}{0pt}%
\pgfpathmoveto{\pgfqpoint{2.496945in}{0.838792in}}%
\pgfpathcurveto{\pgfqpoint{2.507995in}{0.838792in}}{\pgfqpoint{2.518594in}{0.843182in}}{\pgfqpoint{2.526408in}{0.850996in}}%
\pgfpathcurveto{\pgfqpoint{2.534221in}{0.858810in}}{\pgfqpoint{2.538612in}{0.869409in}}{\pgfqpoint{2.538612in}{0.880459in}}%
\pgfpathcurveto{\pgfqpoint{2.538612in}{0.891509in}}{\pgfqpoint{2.534221in}{0.902108in}}{\pgfqpoint{2.526408in}{0.909921in}}%
\pgfpathcurveto{\pgfqpoint{2.518594in}{0.917735in}}{\pgfqpoint{2.507995in}{0.922125in}}{\pgfqpoint{2.496945in}{0.922125in}}%
\pgfpathcurveto{\pgfqpoint{2.485895in}{0.922125in}}{\pgfqpoint{2.475296in}{0.917735in}}{\pgfqpoint{2.467482in}{0.909921in}}%
\pgfpathcurveto{\pgfqpoint{2.459669in}{0.902108in}}{\pgfqpoint{2.455278in}{0.891509in}}{\pgfqpoint{2.455278in}{0.880459in}}%
\pgfpathcurveto{\pgfqpoint{2.455278in}{0.869409in}}{\pgfqpoint{2.459669in}{0.858810in}}{\pgfqpoint{2.467482in}{0.850996in}}%
\pgfpathcurveto{\pgfqpoint{2.475296in}{0.843182in}}{\pgfqpoint{2.485895in}{0.838792in}}{\pgfqpoint{2.496945in}{0.838792in}}%
\pgfpathclose%
\pgfusepath{stroke,fill}%
\end{pgfscope}%
\begin{pgfscope}%
\pgfpathrectangle{\pgfqpoint{0.772069in}{0.515123in}}{\pgfqpoint{3.875000in}{2.695000in}}%
\pgfusepath{clip}%
\pgfsetbuttcap%
\pgfsetroundjoin%
\definecolor{currentfill}{rgb}{1.000000,0.388235,0.278431}%
\pgfsetfillcolor{currentfill}%
\pgfsetlinewidth{1.003750pt}%
\definecolor{currentstroke}{rgb}{1.000000,0.388235,0.278431}%
\pgfsetstrokecolor{currentstroke}%
\pgfsetdash{}{0pt}%
\pgfpathmoveto{\pgfqpoint{2.922192in}{0.896400in}}%
\pgfpathcurveto{\pgfqpoint{2.933242in}{0.896400in}}{\pgfqpoint{2.943841in}{0.900791in}}{\pgfqpoint{2.951655in}{0.908604in}}%
\pgfpathcurveto{\pgfqpoint{2.959469in}{0.916418in}}{\pgfqpoint{2.963859in}{0.927017in}}{\pgfqpoint{2.963859in}{0.938067in}}%
\pgfpathcurveto{\pgfqpoint{2.963859in}{0.949117in}}{\pgfqpoint{2.959469in}{0.959716in}}{\pgfqpoint{2.951655in}{0.967530in}}%
\pgfpathcurveto{\pgfqpoint{2.943841in}{0.975343in}}{\pgfqpoint{2.933242in}{0.979734in}}{\pgfqpoint{2.922192in}{0.979734in}}%
\pgfpathcurveto{\pgfqpoint{2.911142in}{0.979734in}}{\pgfqpoint{2.900543in}{0.975343in}}{\pgfqpoint{2.892730in}{0.967530in}}%
\pgfpathcurveto{\pgfqpoint{2.884916in}{0.959716in}}{\pgfqpoint{2.880526in}{0.949117in}}{\pgfqpoint{2.880526in}{0.938067in}}%
\pgfpathcurveto{\pgfqpoint{2.880526in}{0.927017in}}{\pgfqpoint{2.884916in}{0.916418in}}{\pgfqpoint{2.892730in}{0.908604in}}%
\pgfpathcurveto{\pgfqpoint{2.900543in}{0.900791in}}{\pgfqpoint{2.911142in}{0.896400in}}{\pgfqpoint{2.922192in}{0.896400in}}%
\pgfpathclose%
\pgfusepath{stroke,fill}%
\end{pgfscope}%
\begin{pgfscope}%
\pgfpathrectangle{\pgfqpoint{0.772069in}{0.515123in}}{\pgfqpoint{3.875000in}{2.695000in}}%
\pgfusepath{clip}%
\pgfsetbuttcap%
\pgfsetroundjoin%
\definecolor{currentfill}{rgb}{1.000000,0.388235,0.278431}%
\pgfsetfillcolor{currentfill}%
\pgfsetlinewidth{1.003750pt}%
\definecolor{currentstroke}{rgb}{1.000000,0.388235,0.278431}%
\pgfsetstrokecolor{currentstroke}%
\pgfsetdash{}{0pt}%
\pgfpathmoveto{\pgfqpoint{3.286690in}{0.947689in}}%
\pgfpathcurveto{\pgfqpoint{3.297740in}{0.947689in}}{\pgfqpoint{3.308339in}{0.952079in}}{\pgfqpoint{3.316153in}{0.959893in}}%
\pgfpathcurveto{\pgfqpoint{3.323966in}{0.967706in}}{\pgfqpoint{3.328357in}{0.978305in}}{\pgfqpoint{3.328357in}{0.989355in}}%
\pgfpathcurveto{\pgfqpoint{3.328357in}{1.000405in}}{\pgfqpoint{3.323966in}{1.011005in}}{\pgfqpoint{3.316153in}{1.018818in}}%
\pgfpathcurveto{\pgfqpoint{3.308339in}{1.026632in}}{\pgfqpoint{3.297740in}{1.031022in}}{\pgfqpoint{3.286690in}{1.031022in}}%
\pgfpathcurveto{\pgfqpoint{3.275640in}{1.031022in}}{\pgfqpoint{3.265041in}{1.026632in}}{\pgfqpoint{3.257227in}{1.018818in}}%
\pgfpathcurveto{\pgfqpoint{3.249413in}{1.011005in}}{\pgfqpoint{3.245023in}{1.000405in}}{\pgfqpoint{3.245023in}{0.989355in}}%
\pgfpathcurveto{\pgfqpoint{3.245023in}{0.978305in}}{\pgfqpoint{3.249413in}{0.967706in}}{\pgfqpoint{3.257227in}{0.959893in}}%
\pgfpathcurveto{\pgfqpoint{3.265041in}{0.952079in}}{\pgfqpoint{3.275640in}{0.947689in}}{\pgfqpoint{3.286690in}{0.947689in}}%
\pgfpathclose%
\pgfusepath{stroke,fill}%
\end{pgfscope}%
\begin{pgfscope}%
\pgfpathrectangle{\pgfqpoint{0.772069in}{0.515123in}}{\pgfqpoint{3.875000in}{2.695000in}}%
\pgfusepath{clip}%
\pgfsetbuttcap%
\pgfsetroundjoin%
\definecolor{currentfill}{rgb}{1.000000,0.388235,0.278431}%
\pgfsetfillcolor{currentfill}%
\pgfsetlinewidth{1.003750pt}%
\definecolor{currentstroke}{rgb}{1.000000,0.388235,0.278431}%
\pgfsetstrokecolor{currentstroke}%
\pgfsetdash{}{0pt}%
\pgfpathmoveto{\pgfqpoint{3.681562in}{1.000922in}}%
\pgfpathcurveto{\pgfqpoint{3.692612in}{1.000922in}}{\pgfqpoint{3.703211in}{1.005312in}}{\pgfqpoint{3.711025in}{1.013126in}}%
\pgfpathcurveto{\pgfqpoint{3.718839in}{1.020939in}}{\pgfqpoint{3.723229in}{1.031538in}}{\pgfqpoint{3.723229in}{1.042588in}}%
\pgfpathcurveto{\pgfqpoint{3.723229in}{1.053638in}}{\pgfqpoint{3.718839in}{1.064237in}}{\pgfqpoint{3.711025in}{1.072051in}}%
\pgfpathcurveto{\pgfqpoint{3.703211in}{1.079865in}}{\pgfqpoint{3.692612in}{1.084255in}}{\pgfqpoint{3.681562in}{1.084255in}}%
\pgfpathcurveto{\pgfqpoint{3.670512in}{1.084255in}}{\pgfqpoint{3.659913in}{1.079865in}}{\pgfqpoint{3.652100in}{1.072051in}}%
\pgfpathcurveto{\pgfqpoint{3.644286in}{1.064237in}}{\pgfqpoint{3.639896in}{1.053638in}}{\pgfqpoint{3.639896in}{1.042588in}}%
\pgfpathcurveto{\pgfqpoint{3.639896in}{1.031538in}}{\pgfqpoint{3.644286in}{1.020939in}}{\pgfqpoint{3.652100in}{1.013126in}}%
\pgfpathcurveto{\pgfqpoint{3.659913in}{1.005312in}}{\pgfqpoint{3.670512in}{1.000922in}}{\pgfqpoint{3.681562in}{1.000922in}}%
\pgfpathclose%
\pgfusepath{stroke,fill}%
\end{pgfscope}%
\begin{pgfscope}%
\pgfpathrectangle{\pgfqpoint{0.772069in}{0.515123in}}{\pgfqpoint{3.875000in}{2.695000in}}%
\pgfusepath{clip}%
\pgfsetbuttcap%
\pgfsetroundjoin%
\definecolor{currentfill}{rgb}{1.000000,0.388235,0.278431}%
\pgfsetfillcolor{currentfill}%
\pgfsetlinewidth{1.003750pt}%
\definecolor{currentstroke}{rgb}{1.000000,0.388235,0.278431}%
\pgfsetstrokecolor{currentstroke}%
\pgfsetdash{}{0pt}%
\pgfpathmoveto{\pgfqpoint{4.015685in}{1.047835in}}%
\pgfpathcurveto{\pgfqpoint{4.026735in}{1.047835in}}{\pgfqpoint{4.037334in}{1.052225in}}{\pgfqpoint{4.045148in}{1.060039in}}%
\pgfpathcurveto{\pgfqpoint{4.052962in}{1.067852in}}{\pgfqpoint{4.057352in}{1.078451in}}{\pgfqpoint{4.057352in}{1.089501in}}%
\pgfpathcurveto{\pgfqpoint{4.057352in}{1.100552in}}{\pgfqpoint{4.052962in}{1.111151in}}{\pgfqpoint{4.045148in}{1.118964in}}%
\pgfpathcurveto{\pgfqpoint{4.037334in}{1.126778in}}{\pgfqpoint{4.026735in}{1.131168in}}{\pgfqpoint{4.015685in}{1.131168in}}%
\pgfpathcurveto{\pgfqpoint{4.004635in}{1.131168in}}{\pgfqpoint{3.994036in}{1.126778in}}{\pgfqpoint{3.986222in}{1.118964in}}%
\pgfpathcurveto{\pgfqpoint{3.978409in}{1.111151in}}{\pgfqpoint{3.974018in}{1.100552in}}{\pgfqpoint{3.974018in}{1.089501in}}%
\pgfpathcurveto{\pgfqpoint{3.974018in}{1.078451in}}{\pgfqpoint{3.978409in}{1.067852in}}{\pgfqpoint{3.986222in}{1.060039in}}%
\pgfpathcurveto{\pgfqpoint{3.994036in}{1.052225in}}{\pgfqpoint{4.004635in}{1.047835in}}{\pgfqpoint{4.015685in}{1.047835in}}%
\pgfpathclose%
\pgfusepath{stroke,fill}%
\end{pgfscope}%
\begin{pgfscope}%
\pgfpathrectangle{\pgfqpoint{0.772069in}{0.515123in}}{\pgfqpoint{3.875000in}{2.695000in}}%
\pgfusepath{clip}%
\pgfsetbuttcap%
\pgfsetroundjoin%
\definecolor{currentfill}{rgb}{1.000000,0.388235,0.278431}%
\pgfsetfillcolor{currentfill}%
\pgfsetlinewidth{1.003750pt}%
\definecolor{currentstroke}{rgb}{1.000000,0.388235,0.278431}%
\pgfsetstrokecolor{currentstroke}%
\pgfsetdash{}{0pt}%
\pgfpathmoveto{\pgfqpoint{4.440932in}{1.106172in}}%
\pgfpathcurveto{\pgfqpoint{4.451982in}{1.106172in}}{\pgfqpoint{4.462581in}{1.110562in}}{\pgfqpoint{4.470395in}{1.118376in}}%
\pgfpathcurveto{\pgfqpoint{4.478209in}{1.126190in}}{\pgfqpoint{4.482599in}{1.136789in}}{\pgfqpoint{4.482599in}{1.147839in}}%
\pgfpathcurveto{\pgfqpoint{4.482599in}{1.158889in}}{\pgfqpoint{4.478209in}{1.169488in}}{\pgfqpoint{4.470395in}{1.177302in}}%
\pgfpathcurveto{\pgfqpoint{4.462581in}{1.185115in}}{\pgfqpoint{4.451982in}{1.189506in}}{\pgfqpoint{4.440932in}{1.189506in}}%
\pgfpathcurveto{\pgfqpoint{4.429882in}{1.189506in}}{\pgfqpoint{4.419283in}{1.185115in}}{\pgfqpoint{4.411470in}{1.177302in}}%
\pgfpathcurveto{\pgfqpoint{4.403656in}{1.169488in}}{\pgfqpoint{4.399266in}{1.158889in}}{\pgfqpoint{4.399266in}{1.147839in}}%
\pgfpathcurveto{\pgfqpoint{4.399266in}{1.136789in}}{\pgfqpoint{4.403656in}{1.126190in}}{\pgfqpoint{4.411470in}{1.118376in}}%
\pgfpathcurveto{\pgfqpoint{4.419283in}{1.110562in}}{\pgfqpoint{4.429882in}{1.106172in}}{\pgfqpoint{4.440932in}{1.106172in}}%
\pgfpathclose%
\pgfusepath{stroke,fill}%
\end{pgfscope}%
\begin{pgfscope}%
\pgfpathrectangle{\pgfqpoint{0.772069in}{0.515123in}}{\pgfqpoint{3.875000in}{2.695000in}}%
\pgfusepath{clip}%
\pgfsetbuttcap%
\pgfsetroundjoin%
\definecolor{currentfill}{rgb}{1.000000,0.843137,0.000000}%
\pgfsetfillcolor{currentfill}%
\pgfsetlinewidth{1.003750pt}%
\definecolor{currentstroke}{rgb}{1.000000,0.843137,0.000000}%
\pgfsetstrokecolor{currentstroke}%
\pgfsetdash{}{0pt}%
\pgfpathmoveto{\pgfqpoint{0.978205in}{0.639229in}}%
\pgfpathcurveto{\pgfqpoint{0.989255in}{0.639229in}}{\pgfqpoint{0.999854in}{0.643619in}}{\pgfqpoint{1.007668in}{0.651433in}}%
\pgfpathcurveto{\pgfqpoint{1.015481in}{0.659247in}}{\pgfqpoint{1.019872in}{0.669846in}}{\pgfqpoint{1.019872in}{0.680896in}}%
\pgfpathcurveto{\pgfqpoint{1.019872in}{0.691946in}}{\pgfqpoint{1.015481in}{0.702545in}}{\pgfqpoint{1.007668in}{0.710359in}}%
\pgfpathcurveto{\pgfqpoint{0.999854in}{0.718172in}}{\pgfqpoint{0.989255in}{0.722563in}}{\pgfqpoint{0.978205in}{0.722563in}}%
\pgfpathcurveto{\pgfqpoint{0.967155in}{0.722563in}}{\pgfqpoint{0.956556in}{0.718172in}}{\pgfqpoint{0.948742in}{0.710359in}}%
\pgfpathcurveto{\pgfqpoint{0.940929in}{0.702545in}}{\pgfqpoint{0.936538in}{0.691946in}}{\pgfqpoint{0.936538in}{0.680896in}}%
\pgfpathcurveto{\pgfqpoint{0.936538in}{0.669846in}}{\pgfqpoint{0.940929in}{0.659247in}}{\pgfqpoint{0.948742in}{0.651433in}}%
\pgfpathcurveto{\pgfqpoint{0.956556in}{0.643619in}}{\pgfqpoint{0.967155in}{0.639229in}}{\pgfqpoint{0.978205in}{0.639229in}}%
\pgfpathclose%
\pgfusepath{stroke,fill}%
\end{pgfscope}%
\begin{pgfscope}%
\pgfpathrectangle{\pgfqpoint{0.772069in}{0.515123in}}{\pgfqpoint{3.875000in}{2.695000in}}%
\pgfusepath{clip}%
\pgfsetbuttcap%
\pgfsetroundjoin%
\definecolor{currentfill}{rgb}{1.000000,0.843137,0.000000}%
\pgfsetfillcolor{currentfill}%
\pgfsetlinewidth{1.003750pt}%
\definecolor{currentstroke}{rgb}{1.000000,0.843137,0.000000}%
\pgfsetstrokecolor{currentstroke}%
\pgfsetdash{}{0pt}%
\pgfpathmoveto{\pgfqpoint{1.342703in}{0.704616in}}%
\pgfpathcurveto{\pgfqpoint{1.353753in}{0.704616in}}{\pgfqpoint{1.364352in}{0.709006in}}{\pgfqpoint{1.372165in}{0.716820in}}%
\pgfpathcurveto{\pgfqpoint{1.379979in}{0.724633in}}{\pgfqpoint{1.384369in}{0.735232in}}{\pgfqpoint{1.384369in}{0.746282in}}%
\pgfpathcurveto{\pgfqpoint{1.384369in}{0.757333in}}{\pgfqpoint{1.379979in}{0.767932in}}{\pgfqpoint{1.372165in}{0.775745in}}%
\pgfpathcurveto{\pgfqpoint{1.364352in}{0.783559in}}{\pgfqpoint{1.353753in}{0.787949in}}{\pgfqpoint{1.342703in}{0.787949in}}%
\pgfpathcurveto{\pgfqpoint{1.331653in}{0.787949in}}{\pgfqpoint{1.321053in}{0.783559in}}{\pgfqpoint{1.313240in}{0.775745in}}%
\pgfpathcurveto{\pgfqpoint{1.305426in}{0.767932in}}{\pgfqpoint{1.301036in}{0.757333in}}{\pgfqpoint{1.301036in}{0.746282in}}%
\pgfpathcurveto{\pgfqpoint{1.301036in}{0.735232in}}{\pgfqpoint{1.305426in}{0.724633in}}{\pgfqpoint{1.313240in}{0.716820in}}%
\pgfpathcurveto{\pgfqpoint{1.321053in}{0.709006in}}{\pgfqpoint{1.331653in}{0.704616in}}{\pgfqpoint{1.342703in}{0.704616in}}%
\pgfpathclose%
\pgfusepath{stroke,fill}%
\end{pgfscope}%
\begin{pgfscope}%
\pgfpathrectangle{\pgfqpoint{0.772069in}{0.515123in}}{\pgfqpoint{3.875000in}{2.695000in}}%
\pgfusepath{clip}%
\pgfsetbuttcap%
\pgfsetroundjoin%
\definecolor{currentfill}{rgb}{1.000000,0.843137,0.000000}%
\pgfsetfillcolor{currentfill}%
\pgfsetlinewidth{1.003750pt}%
\definecolor{currentstroke}{rgb}{1.000000,0.843137,0.000000}%
\pgfsetstrokecolor{currentstroke}%
\pgfsetdash{}{0pt}%
\pgfpathmoveto{\pgfqpoint{1.737575in}{0.770732in}}%
\pgfpathcurveto{\pgfqpoint{1.748625in}{0.770732in}}{\pgfqpoint{1.759224in}{0.775122in}}{\pgfqpoint{1.767038in}{0.782935in}}%
\pgfpathcurveto{\pgfqpoint{1.774851in}{0.790749in}}{\pgfqpoint{1.779242in}{0.801348in}}{\pgfqpoint{1.779242in}{0.812398in}}%
\pgfpathcurveto{\pgfqpoint{1.779242in}{0.823448in}}{\pgfqpoint{1.774851in}{0.834047in}}{\pgfqpoint{1.767038in}{0.841861in}}%
\pgfpathcurveto{\pgfqpoint{1.759224in}{0.849675in}}{\pgfqpoint{1.748625in}{0.854065in}}{\pgfqpoint{1.737575in}{0.854065in}}%
\pgfpathcurveto{\pgfqpoint{1.726525in}{0.854065in}}{\pgfqpoint{1.715926in}{0.849675in}}{\pgfqpoint{1.708112in}{0.841861in}}%
\pgfpathcurveto{\pgfqpoint{1.700299in}{0.834047in}}{\pgfqpoint{1.695908in}{0.823448in}}{\pgfqpoint{1.695908in}{0.812398in}}%
\pgfpathcurveto{\pgfqpoint{1.695908in}{0.801348in}}{\pgfqpoint{1.700299in}{0.790749in}}{\pgfqpoint{1.708112in}{0.782935in}}%
\pgfpathcurveto{\pgfqpoint{1.715926in}{0.775122in}}{\pgfqpoint{1.726525in}{0.770732in}}{\pgfqpoint{1.737575in}{0.770732in}}%
\pgfpathclose%
\pgfusepath{stroke,fill}%
\end{pgfscope}%
\begin{pgfscope}%
\pgfpathrectangle{\pgfqpoint{0.772069in}{0.515123in}}{\pgfqpoint{3.875000in}{2.695000in}}%
\pgfusepath{clip}%
\pgfsetbuttcap%
\pgfsetroundjoin%
\definecolor{currentfill}{rgb}{1.000000,0.843137,0.000000}%
\pgfsetfillcolor{currentfill}%
\pgfsetlinewidth{1.003750pt}%
\definecolor{currentstroke}{rgb}{1.000000,0.843137,0.000000}%
\pgfsetstrokecolor{currentstroke}%
\pgfsetdash{}{0pt}%
\pgfpathmoveto{\pgfqpoint{2.132447in}{0.837820in}}%
\pgfpathcurveto{\pgfqpoint{2.143498in}{0.837820in}}{\pgfqpoint{2.154097in}{0.842210in}}{\pgfqpoint{2.161910in}{0.850024in}}%
\pgfpathcurveto{\pgfqpoint{2.169724in}{0.857837in}}{\pgfqpoint{2.174114in}{0.868436in}}{\pgfqpoint{2.174114in}{0.879486in}}%
\pgfpathcurveto{\pgfqpoint{2.174114in}{0.890537in}}{\pgfqpoint{2.169724in}{0.901136in}}{\pgfqpoint{2.161910in}{0.908949in}}%
\pgfpathcurveto{\pgfqpoint{2.154097in}{0.916763in}}{\pgfqpoint{2.143498in}{0.921153in}}{\pgfqpoint{2.132447in}{0.921153in}}%
\pgfpathcurveto{\pgfqpoint{2.121397in}{0.921153in}}{\pgfqpoint{2.110798in}{0.916763in}}{\pgfqpoint{2.102985in}{0.908949in}}%
\pgfpathcurveto{\pgfqpoint{2.095171in}{0.901136in}}{\pgfqpoint{2.090781in}{0.890537in}}{\pgfqpoint{2.090781in}{0.879486in}}%
\pgfpathcurveto{\pgfqpoint{2.090781in}{0.868436in}}{\pgfqpoint{2.095171in}{0.857837in}}{\pgfqpoint{2.102985in}{0.850024in}}%
\pgfpathcurveto{\pgfqpoint{2.110798in}{0.842210in}}{\pgfqpoint{2.121397in}{0.837820in}}{\pgfqpoint{2.132447in}{0.837820in}}%
\pgfpathclose%
\pgfusepath{stroke,fill}%
\end{pgfscope}%
\begin{pgfscope}%
\pgfpathrectangle{\pgfqpoint{0.772069in}{0.515123in}}{\pgfqpoint{3.875000in}{2.695000in}}%
\pgfusepath{clip}%
\pgfsetbuttcap%
\pgfsetroundjoin%
\definecolor{currentfill}{rgb}{1.000000,0.843137,0.000000}%
\pgfsetfillcolor{currentfill}%
\pgfsetlinewidth{1.003750pt}%
\definecolor{currentstroke}{rgb}{1.000000,0.843137,0.000000}%
\pgfsetstrokecolor{currentstroke}%
\pgfsetdash{}{0pt}%
\pgfpathmoveto{\pgfqpoint{2.496945in}{0.902963in}}%
\pgfpathcurveto{\pgfqpoint{2.507995in}{0.902963in}}{\pgfqpoint{2.518594in}{0.907354in}}{\pgfqpoint{2.526408in}{0.915167in}}%
\pgfpathcurveto{\pgfqpoint{2.534221in}{0.922981in}}{\pgfqpoint{2.538612in}{0.933580in}}{\pgfqpoint{2.538612in}{0.944630in}}%
\pgfpathcurveto{\pgfqpoint{2.538612in}{0.955680in}}{\pgfqpoint{2.534221in}{0.966279in}}{\pgfqpoint{2.526408in}{0.974093in}}%
\pgfpathcurveto{\pgfqpoint{2.518594in}{0.981906in}}{\pgfqpoint{2.507995in}{0.986297in}}{\pgfqpoint{2.496945in}{0.986297in}}%
\pgfpathcurveto{\pgfqpoint{2.485895in}{0.986297in}}{\pgfqpoint{2.475296in}{0.981906in}}{\pgfqpoint{2.467482in}{0.974093in}}%
\pgfpathcurveto{\pgfqpoint{2.459669in}{0.966279in}}{\pgfqpoint{2.455278in}{0.955680in}}{\pgfqpoint{2.455278in}{0.944630in}}%
\pgfpathcurveto{\pgfqpoint{2.455278in}{0.933580in}}{\pgfqpoint{2.459669in}{0.922981in}}{\pgfqpoint{2.467482in}{0.915167in}}%
\pgfpathcurveto{\pgfqpoint{2.475296in}{0.907354in}}{\pgfqpoint{2.485895in}{0.902963in}}{\pgfqpoint{2.496945in}{0.902963in}}%
\pgfpathclose%
\pgfusepath{stroke,fill}%
\end{pgfscope}%
\begin{pgfscope}%
\pgfpathrectangle{\pgfqpoint{0.772069in}{0.515123in}}{\pgfqpoint{3.875000in}{2.695000in}}%
\pgfusepath{clip}%
\pgfsetbuttcap%
\pgfsetroundjoin%
\definecolor{currentfill}{rgb}{1.000000,0.843137,0.000000}%
\pgfsetfillcolor{currentfill}%
\pgfsetlinewidth{1.003750pt}%
\definecolor{currentstroke}{rgb}{1.000000,0.843137,0.000000}%
\pgfsetstrokecolor{currentstroke}%
\pgfsetdash{}{0pt}%
\pgfpathmoveto{\pgfqpoint{2.922192in}{0.974184in}}%
\pgfpathcurveto{\pgfqpoint{2.933242in}{0.974184in}}{\pgfqpoint{2.943841in}{0.978574in}}{\pgfqpoint{2.951655in}{0.986388in}}%
\pgfpathcurveto{\pgfqpoint{2.959469in}{0.994201in}}{\pgfqpoint{2.963859in}{1.004800in}}{\pgfqpoint{2.963859in}{1.015850in}}%
\pgfpathcurveto{\pgfqpoint{2.963859in}{1.026900in}}{\pgfqpoint{2.959469in}{1.037499in}}{\pgfqpoint{2.951655in}{1.045313in}}%
\pgfpathcurveto{\pgfqpoint{2.943841in}{1.053127in}}{\pgfqpoint{2.933242in}{1.057517in}}{\pgfqpoint{2.922192in}{1.057517in}}%
\pgfpathcurveto{\pgfqpoint{2.911142in}{1.057517in}}{\pgfqpoint{2.900543in}{1.053127in}}{\pgfqpoint{2.892730in}{1.045313in}}%
\pgfpathcurveto{\pgfqpoint{2.884916in}{1.037499in}}{\pgfqpoint{2.880526in}{1.026900in}}{\pgfqpoint{2.880526in}{1.015850in}}%
\pgfpathcurveto{\pgfqpoint{2.880526in}{1.004800in}}{\pgfqpoint{2.884916in}{0.994201in}}{\pgfqpoint{2.892730in}{0.986388in}}%
\pgfpathcurveto{\pgfqpoint{2.900543in}{0.978574in}}{\pgfqpoint{2.911142in}{0.974184in}}{\pgfqpoint{2.922192in}{0.974184in}}%
\pgfpathclose%
\pgfusepath{stroke,fill}%
\end{pgfscope}%
\begin{pgfscope}%
\pgfpathrectangle{\pgfqpoint{0.772069in}{0.515123in}}{\pgfqpoint{3.875000in}{2.695000in}}%
\pgfusepath{clip}%
\pgfsetbuttcap%
\pgfsetroundjoin%
\definecolor{currentfill}{rgb}{1.000000,0.843137,0.000000}%
\pgfsetfillcolor{currentfill}%
\pgfsetlinewidth{1.003750pt}%
\definecolor{currentstroke}{rgb}{1.000000,0.843137,0.000000}%
\pgfsetstrokecolor{currentstroke}%
\pgfsetdash{}{0pt}%
\pgfpathmoveto{\pgfqpoint{3.286690in}{1.037626in}}%
\pgfpathcurveto{\pgfqpoint{3.297740in}{1.037626in}}{\pgfqpoint{3.308339in}{1.042016in}}{\pgfqpoint{3.316153in}{1.049830in}}%
\pgfpathcurveto{\pgfqpoint{3.323966in}{1.057643in}}{\pgfqpoint{3.328357in}{1.068242in}}{\pgfqpoint{3.328357in}{1.079292in}}%
\pgfpathcurveto{\pgfqpoint{3.328357in}{1.090342in}}{\pgfqpoint{3.323966in}{1.100941in}}{\pgfqpoint{3.316153in}{1.108755in}}%
\pgfpathcurveto{\pgfqpoint{3.308339in}{1.116569in}}{\pgfqpoint{3.297740in}{1.120959in}}{\pgfqpoint{3.286690in}{1.120959in}}%
\pgfpathcurveto{\pgfqpoint{3.275640in}{1.120959in}}{\pgfqpoint{3.265041in}{1.116569in}}{\pgfqpoint{3.257227in}{1.108755in}}%
\pgfpathcurveto{\pgfqpoint{3.249413in}{1.100941in}}{\pgfqpoint{3.245023in}{1.090342in}}{\pgfqpoint{3.245023in}{1.079292in}}%
\pgfpathcurveto{\pgfqpoint{3.245023in}{1.068242in}}{\pgfqpoint{3.249413in}{1.057643in}}{\pgfqpoint{3.257227in}{1.049830in}}%
\pgfpathcurveto{\pgfqpoint{3.265041in}{1.042016in}}{\pgfqpoint{3.275640in}{1.037626in}}{\pgfqpoint{3.286690in}{1.037626in}}%
\pgfpathclose%
\pgfusepath{stroke,fill}%
\end{pgfscope}%
\begin{pgfscope}%
\pgfpathrectangle{\pgfqpoint{0.772069in}{0.515123in}}{\pgfqpoint{3.875000in}{2.695000in}}%
\pgfusepath{clip}%
\pgfsetbuttcap%
\pgfsetroundjoin%
\definecolor{currentfill}{rgb}{1.000000,0.843137,0.000000}%
\pgfsetfillcolor{currentfill}%
\pgfsetlinewidth{1.003750pt}%
\definecolor{currentstroke}{rgb}{1.000000,0.843137,0.000000}%
\pgfsetstrokecolor{currentstroke}%
\pgfsetdash{}{0pt}%
\pgfpathmoveto{\pgfqpoint{3.681562in}{1.101311in}}%
\pgfpathcurveto{\pgfqpoint{3.692612in}{1.101311in}}{\pgfqpoint{3.703211in}{1.105701in}}{\pgfqpoint{3.711025in}{1.113515in}}%
\pgfpathcurveto{\pgfqpoint{3.718839in}{1.121328in}}{\pgfqpoint{3.723229in}{1.131927in}}{\pgfqpoint{3.723229in}{1.142977in}}%
\pgfpathcurveto{\pgfqpoint{3.723229in}{1.154028in}}{\pgfqpoint{3.718839in}{1.164627in}}{\pgfqpoint{3.711025in}{1.172440in}}%
\pgfpathcurveto{\pgfqpoint{3.703211in}{1.180254in}}{\pgfqpoint{3.692612in}{1.184644in}}{\pgfqpoint{3.681562in}{1.184644in}}%
\pgfpathcurveto{\pgfqpoint{3.670512in}{1.184644in}}{\pgfqpoint{3.659913in}{1.180254in}}{\pgfqpoint{3.652100in}{1.172440in}}%
\pgfpathcurveto{\pgfqpoint{3.644286in}{1.164627in}}{\pgfqpoint{3.639896in}{1.154028in}}{\pgfqpoint{3.639896in}{1.142977in}}%
\pgfpathcurveto{\pgfqpoint{3.639896in}{1.131927in}}{\pgfqpoint{3.644286in}{1.121328in}}{\pgfqpoint{3.652100in}{1.113515in}}%
\pgfpathcurveto{\pgfqpoint{3.659913in}{1.105701in}}{\pgfqpoint{3.670512in}{1.101311in}}{\pgfqpoint{3.681562in}{1.101311in}}%
\pgfpathclose%
\pgfusepath{stroke,fill}%
\end{pgfscope}%
\begin{pgfscope}%
\pgfpathrectangle{\pgfqpoint{0.772069in}{0.515123in}}{\pgfqpoint{3.875000in}{2.695000in}}%
\pgfusepath{clip}%
\pgfsetbuttcap%
\pgfsetroundjoin%
\definecolor{currentfill}{rgb}{1.000000,0.843137,0.000000}%
\pgfsetfillcolor{currentfill}%
\pgfsetlinewidth{1.003750pt}%
\definecolor{currentstroke}{rgb}{1.000000,0.843137,0.000000}%
\pgfsetstrokecolor{currentstroke}%
\pgfsetdash{}{0pt}%
\pgfpathmoveto{\pgfqpoint{4.015685in}{1.162079in}}%
\pgfpathcurveto{\pgfqpoint{4.026735in}{1.162079in}}{\pgfqpoint{4.037334in}{1.166469in}}{\pgfqpoint{4.045148in}{1.174283in}}%
\pgfpathcurveto{\pgfqpoint{4.052962in}{1.182097in}}{\pgfqpoint{4.057352in}{1.192696in}}{\pgfqpoint{4.057352in}{1.203746in}}%
\pgfpathcurveto{\pgfqpoint{4.057352in}{1.214796in}}{\pgfqpoint{4.052962in}{1.225395in}}{\pgfqpoint{4.045148in}{1.233208in}}%
\pgfpathcurveto{\pgfqpoint{4.037334in}{1.241022in}}{\pgfqpoint{4.026735in}{1.245412in}}{\pgfqpoint{4.015685in}{1.245412in}}%
\pgfpathcurveto{\pgfqpoint{4.004635in}{1.245412in}}{\pgfqpoint{3.994036in}{1.241022in}}{\pgfqpoint{3.986222in}{1.233208in}}%
\pgfpathcurveto{\pgfqpoint{3.978409in}{1.225395in}}{\pgfqpoint{3.974018in}{1.214796in}}{\pgfqpoint{3.974018in}{1.203746in}}%
\pgfpathcurveto{\pgfqpoint{3.974018in}{1.192696in}}{\pgfqpoint{3.978409in}{1.182097in}}{\pgfqpoint{3.986222in}{1.174283in}}%
\pgfpathcurveto{\pgfqpoint{3.994036in}{1.166469in}}{\pgfqpoint{4.004635in}{1.162079in}}{\pgfqpoint{4.015685in}{1.162079in}}%
\pgfpathclose%
\pgfusepath{stroke,fill}%
\end{pgfscope}%
\begin{pgfscope}%
\pgfpathrectangle{\pgfqpoint{0.772069in}{0.515123in}}{\pgfqpoint{3.875000in}{2.695000in}}%
\pgfusepath{clip}%
\pgfsetbuttcap%
\pgfsetroundjoin%
\definecolor{currentfill}{rgb}{1.000000,0.843137,0.000000}%
\pgfsetfillcolor{currentfill}%
\pgfsetlinewidth{1.003750pt}%
\definecolor{currentstroke}{rgb}{1.000000,0.843137,0.000000}%
\pgfsetstrokecolor{currentstroke}%
\pgfsetdash{}{0pt}%
\pgfpathmoveto{\pgfqpoint{4.440932in}{1.235001in}}%
\pgfpathcurveto{\pgfqpoint{4.451982in}{1.235001in}}{\pgfqpoint{4.462581in}{1.239391in}}{\pgfqpoint{4.470395in}{1.247205in}}%
\pgfpathcurveto{\pgfqpoint{4.478209in}{1.255018in}}{\pgfqpoint{4.482599in}{1.265617in}}{\pgfqpoint{4.482599in}{1.276668in}}%
\pgfpathcurveto{\pgfqpoint{4.482599in}{1.287718in}}{\pgfqpoint{4.478209in}{1.298317in}}{\pgfqpoint{4.470395in}{1.306130in}}%
\pgfpathcurveto{\pgfqpoint{4.462581in}{1.313944in}}{\pgfqpoint{4.451982in}{1.318334in}}{\pgfqpoint{4.440932in}{1.318334in}}%
\pgfpathcurveto{\pgfqpoint{4.429882in}{1.318334in}}{\pgfqpoint{4.419283in}{1.313944in}}{\pgfqpoint{4.411470in}{1.306130in}}%
\pgfpathcurveto{\pgfqpoint{4.403656in}{1.298317in}}{\pgfqpoint{4.399266in}{1.287718in}}{\pgfqpoint{4.399266in}{1.276668in}}%
\pgfpathcurveto{\pgfqpoint{4.399266in}{1.265617in}}{\pgfqpoint{4.403656in}{1.255018in}}{\pgfqpoint{4.411470in}{1.247205in}}%
\pgfpathcurveto{\pgfqpoint{4.419283in}{1.239391in}}{\pgfqpoint{4.429882in}{1.235001in}}{\pgfqpoint{4.440932in}{1.235001in}}%
\pgfpathclose%
\pgfusepath{stroke,fill}%
\end{pgfscope}%
\begin{pgfscope}%
\pgfpathrectangle{\pgfqpoint{0.772069in}{0.515123in}}{\pgfqpoint{3.875000in}{2.695000in}}%
\pgfusepath{clip}%
\pgfsetbuttcap%
\pgfsetroundjoin%
\definecolor{currentfill}{rgb}{0.196078,0.803922,0.196078}%
\pgfsetfillcolor{currentfill}%
\pgfsetlinewidth{1.003750pt}%
\definecolor{currentstroke}{rgb}{0.196078,0.803922,0.196078}%
\pgfsetstrokecolor{currentstroke}%
\pgfsetdash{}{0pt}%
\pgfpathmoveto{\pgfqpoint{0.978205in}{0.696108in}}%
\pgfpathcurveto{\pgfqpoint{0.989255in}{0.696108in}}{\pgfqpoint{0.999854in}{0.700498in}}{\pgfqpoint{1.007668in}{0.708312in}}%
\pgfpathcurveto{\pgfqpoint{1.015481in}{0.716126in}}{\pgfqpoint{1.019872in}{0.726725in}}{\pgfqpoint{1.019872in}{0.737775in}}%
\pgfpathcurveto{\pgfqpoint{1.019872in}{0.748825in}}{\pgfqpoint{1.015481in}{0.759424in}}{\pgfqpoint{1.007668in}{0.767238in}}%
\pgfpathcurveto{\pgfqpoint{0.999854in}{0.775051in}}{\pgfqpoint{0.989255in}{0.779442in}}{\pgfqpoint{0.978205in}{0.779442in}}%
\pgfpathcurveto{\pgfqpoint{0.967155in}{0.779442in}}{\pgfqpoint{0.956556in}{0.775051in}}{\pgfqpoint{0.948742in}{0.767238in}}%
\pgfpathcurveto{\pgfqpoint{0.940929in}{0.759424in}}{\pgfqpoint{0.936538in}{0.748825in}}{\pgfqpoint{0.936538in}{0.737775in}}%
\pgfpathcurveto{\pgfqpoint{0.936538in}{0.726725in}}{\pgfqpoint{0.940929in}{0.716126in}}{\pgfqpoint{0.948742in}{0.708312in}}%
\pgfpathcurveto{\pgfqpoint{0.956556in}{0.700498in}}{\pgfqpoint{0.967155in}{0.696108in}}{\pgfqpoint{0.978205in}{0.696108in}}%
\pgfpathclose%
\pgfusepath{stroke,fill}%
\end{pgfscope}%
\begin{pgfscope}%
\pgfpathrectangle{\pgfqpoint{0.772069in}{0.515123in}}{\pgfqpoint{3.875000in}{2.695000in}}%
\pgfusepath{clip}%
\pgfsetbuttcap%
\pgfsetroundjoin%
\definecolor{currentfill}{rgb}{0.196078,0.803922,0.196078}%
\pgfsetfillcolor{currentfill}%
\pgfsetlinewidth{1.003750pt}%
\definecolor{currentstroke}{rgb}{0.196078,0.803922,0.196078}%
\pgfsetstrokecolor{currentstroke}%
\pgfsetdash{}{0pt}%
\pgfpathmoveto{\pgfqpoint{1.342703in}{0.814728in}}%
\pgfpathcurveto{\pgfqpoint{1.353753in}{0.814728in}}{\pgfqpoint{1.364352in}{0.819118in}}{\pgfqpoint{1.372165in}{0.826932in}}%
\pgfpathcurveto{\pgfqpoint{1.379979in}{0.834745in}}{\pgfqpoint{1.384369in}{0.845344in}}{\pgfqpoint{1.384369in}{0.856394in}}%
\pgfpathcurveto{\pgfqpoint{1.384369in}{0.867445in}}{\pgfqpoint{1.379979in}{0.878044in}}{\pgfqpoint{1.372165in}{0.885857in}}%
\pgfpathcurveto{\pgfqpoint{1.364352in}{0.893671in}}{\pgfqpoint{1.353753in}{0.898061in}}{\pgfqpoint{1.342703in}{0.898061in}}%
\pgfpathcurveto{\pgfqpoint{1.331653in}{0.898061in}}{\pgfqpoint{1.321053in}{0.893671in}}{\pgfqpoint{1.313240in}{0.885857in}}%
\pgfpathcurveto{\pgfqpoint{1.305426in}{0.878044in}}{\pgfqpoint{1.301036in}{0.867445in}}{\pgfqpoint{1.301036in}{0.856394in}}%
\pgfpathcurveto{\pgfqpoint{1.301036in}{0.845344in}}{\pgfqpoint{1.305426in}{0.834745in}}{\pgfqpoint{1.313240in}{0.826932in}}%
\pgfpathcurveto{\pgfqpoint{1.321053in}{0.819118in}}{\pgfqpoint{1.331653in}{0.814728in}}{\pgfqpoint{1.342703in}{0.814728in}}%
\pgfpathclose%
\pgfusepath{stroke,fill}%
\end{pgfscope}%
\begin{pgfscope}%
\pgfpathrectangle{\pgfqpoint{0.772069in}{0.515123in}}{\pgfqpoint{3.875000in}{2.695000in}}%
\pgfusepath{clip}%
\pgfsetbuttcap%
\pgfsetroundjoin%
\definecolor{currentfill}{rgb}{0.196078,0.803922,0.196078}%
\pgfsetfillcolor{currentfill}%
\pgfsetlinewidth{1.003750pt}%
\definecolor{currentstroke}{rgb}{0.196078,0.803922,0.196078}%
\pgfsetstrokecolor{currentstroke}%
\pgfsetdash{}{0pt}%
\pgfpathmoveto{\pgfqpoint{1.737575in}{0.934077in}}%
\pgfpathcurveto{\pgfqpoint{1.748625in}{0.934077in}}{\pgfqpoint{1.759224in}{0.938467in}}{\pgfqpoint{1.767038in}{0.946280in}}%
\pgfpathcurveto{\pgfqpoint{1.774851in}{0.954094in}}{\pgfqpoint{1.779242in}{0.964693in}}{\pgfqpoint{1.779242in}{0.975743in}}%
\pgfpathcurveto{\pgfqpoint{1.779242in}{0.986793in}}{\pgfqpoint{1.774851in}{0.997392in}}{\pgfqpoint{1.767038in}{1.005206in}}%
\pgfpathcurveto{\pgfqpoint{1.759224in}{1.013020in}}{\pgfqpoint{1.748625in}{1.017410in}}{\pgfqpoint{1.737575in}{1.017410in}}%
\pgfpathcurveto{\pgfqpoint{1.726525in}{1.017410in}}{\pgfqpoint{1.715926in}{1.013020in}}{\pgfqpoint{1.708112in}{1.005206in}}%
\pgfpathcurveto{\pgfqpoint{1.700299in}{0.997392in}}{\pgfqpoint{1.695908in}{0.986793in}}{\pgfqpoint{1.695908in}{0.975743in}}%
\pgfpathcurveto{\pgfqpoint{1.695908in}{0.964693in}}{\pgfqpoint{1.700299in}{0.954094in}}{\pgfqpoint{1.708112in}{0.946280in}}%
\pgfpathcurveto{\pgfqpoint{1.715926in}{0.938467in}}{\pgfqpoint{1.726525in}{0.934077in}}{\pgfqpoint{1.737575in}{0.934077in}}%
\pgfpathclose%
\pgfusepath{stroke,fill}%
\end{pgfscope}%
\begin{pgfscope}%
\pgfpathrectangle{\pgfqpoint{0.772069in}{0.515123in}}{\pgfqpoint{3.875000in}{2.695000in}}%
\pgfusepath{clip}%
\pgfsetbuttcap%
\pgfsetroundjoin%
\definecolor{currentfill}{rgb}{0.196078,0.803922,0.196078}%
\pgfsetfillcolor{currentfill}%
\pgfsetlinewidth{1.003750pt}%
\definecolor{currentstroke}{rgb}{0.196078,0.803922,0.196078}%
\pgfsetstrokecolor{currentstroke}%
\pgfsetdash{}{0pt}%
\pgfpathmoveto{\pgfqpoint{2.132447in}{1.052696in}}%
\pgfpathcurveto{\pgfqpoint{2.143498in}{1.052696in}}{\pgfqpoint{2.154097in}{1.057086in}}{\pgfqpoint{2.161910in}{1.064900in}}%
\pgfpathcurveto{\pgfqpoint{2.169724in}{1.072714in}}{\pgfqpoint{2.174114in}{1.083313in}}{\pgfqpoint{2.174114in}{1.094363in}}%
\pgfpathcurveto{\pgfqpoint{2.174114in}{1.105413in}}{\pgfqpoint{2.169724in}{1.116012in}}{\pgfqpoint{2.161910in}{1.123826in}}%
\pgfpathcurveto{\pgfqpoint{2.154097in}{1.131639in}}{\pgfqpoint{2.143498in}{1.136030in}}{\pgfqpoint{2.132447in}{1.136030in}}%
\pgfpathcurveto{\pgfqpoint{2.121397in}{1.136030in}}{\pgfqpoint{2.110798in}{1.131639in}}{\pgfqpoint{2.102985in}{1.123826in}}%
\pgfpathcurveto{\pgfqpoint{2.095171in}{1.116012in}}{\pgfqpoint{2.090781in}{1.105413in}}{\pgfqpoint{2.090781in}{1.094363in}}%
\pgfpathcurveto{\pgfqpoint{2.090781in}{1.083313in}}{\pgfqpoint{2.095171in}{1.072714in}}{\pgfqpoint{2.102985in}{1.064900in}}%
\pgfpathcurveto{\pgfqpoint{2.110798in}{1.057086in}}{\pgfqpoint{2.121397in}{1.052696in}}{\pgfqpoint{2.132447in}{1.052696in}}%
\pgfpathclose%
\pgfusepath{stroke,fill}%
\end{pgfscope}%
\begin{pgfscope}%
\pgfpathrectangle{\pgfqpoint{0.772069in}{0.515123in}}{\pgfqpoint{3.875000in}{2.695000in}}%
\pgfusepath{clip}%
\pgfsetbuttcap%
\pgfsetroundjoin%
\definecolor{currentfill}{rgb}{0.196078,0.803922,0.196078}%
\pgfsetfillcolor{currentfill}%
\pgfsetlinewidth{1.003750pt}%
\definecolor{currentstroke}{rgb}{0.196078,0.803922,0.196078}%
\pgfsetstrokecolor{currentstroke}%
\pgfsetdash{}{0pt}%
\pgfpathmoveto{\pgfqpoint{2.496945in}{1.169371in}}%
\pgfpathcurveto{\pgfqpoint{2.507995in}{1.169371in}}{\pgfqpoint{2.518594in}{1.173761in}}{\pgfqpoint{2.526408in}{1.181575in}}%
\pgfpathcurveto{\pgfqpoint{2.534221in}{1.189389in}}{\pgfqpoint{2.538612in}{1.199988in}}{\pgfqpoint{2.538612in}{1.211038in}}%
\pgfpathcurveto{\pgfqpoint{2.538612in}{1.222088in}}{\pgfqpoint{2.534221in}{1.232687in}}{\pgfqpoint{2.526408in}{1.240501in}}%
\pgfpathcurveto{\pgfqpoint{2.518594in}{1.248314in}}{\pgfqpoint{2.507995in}{1.252705in}}{\pgfqpoint{2.496945in}{1.252705in}}%
\pgfpathcurveto{\pgfqpoint{2.485895in}{1.252705in}}{\pgfqpoint{2.475296in}{1.248314in}}{\pgfqpoint{2.467482in}{1.240501in}}%
\pgfpathcurveto{\pgfqpoint{2.459669in}{1.232687in}}{\pgfqpoint{2.455278in}{1.222088in}}{\pgfqpoint{2.455278in}{1.211038in}}%
\pgfpathcurveto{\pgfqpoint{2.455278in}{1.199988in}}{\pgfqpoint{2.459669in}{1.189389in}}{\pgfqpoint{2.467482in}{1.181575in}}%
\pgfpathcurveto{\pgfqpoint{2.475296in}{1.173761in}}{\pgfqpoint{2.485895in}{1.169371in}}{\pgfqpoint{2.496945in}{1.169371in}}%
\pgfpathclose%
\pgfusepath{stroke,fill}%
\end{pgfscope}%
\begin{pgfscope}%
\pgfpathrectangle{\pgfqpoint{0.772069in}{0.515123in}}{\pgfqpoint{3.875000in}{2.695000in}}%
\pgfusepath{clip}%
\pgfsetbuttcap%
\pgfsetroundjoin%
\definecolor{currentfill}{rgb}{0.196078,0.803922,0.196078}%
\pgfsetfillcolor{currentfill}%
\pgfsetlinewidth{1.003750pt}%
\definecolor{currentstroke}{rgb}{0.196078,0.803922,0.196078}%
\pgfsetstrokecolor{currentstroke}%
\pgfsetdash{}{0pt}%
\pgfpathmoveto{\pgfqpoint{2.922192in}{1.298200in}}%
\pgfpathcurveto{\pgfqpoint{2.933242in}{1.298200in}}{\pgfqpoint{2.943841in}{1.302590in}}{\pgfqpoint{2.951655in}{1.310404in}}%
\pgfpathcurveto{\pgfqpoint{2.959469in}{1.318217in}}{\pgfqpoint{2.963859in}{1.328816in}}{\pgfqpoint{2.963859in}{1.339866in}}%
\pgfpathcurveto{\pgfqpoint{2.963859in}{1.350917in}}{\pgfqpoint{2.959469in}{1.361516in}}{\pgfqpoint{2.951655in}{1.369329in}}%
\pgfpathcurveto{\pgfqpoint{2.943841in}{1.377143in}}{\pgfqpoint{2.933242in}{1.381533in}}{\pgfqpoint{2.922192in}{1.381533in}}%
\pgfpathcurveto{\pgfqpoint{2.911142in}{1.381533in}}{\pgfqpoint{2.900543in}{1.377143in}}{\pgfqpoint{2.892730in}{1.369329in}}%
\pgfpathcurveto{\pgfqpoint{2.884916in}{1.361516in}}{\pgfqpoint{2.880526in}{1.350917in}}{\pgfqpoint{2.880526in}{1.339866in}}%
\pgfpathcurveto{\pgfqpoint{2.880526in}{1.328816in}}{\pgfqpoint{2.884916in}{1.318217in}}{\pgfqpoint{2.892730in}{1.310404in}}%
\pgfpathcurveto{\pgfqpoint{2.900543in}{1.302590in}}{\pgfqpoint{2.911142in}{1.298200in}}{\pgfqpoint{2.922192in}{1.298200in}}%
\pgfpathclose%
\pgfusepath{stroke,fill}%
\end{pgfscope}%
\begin{pgfscope}%
\pgfpathrectangle{\pgfqpoint{0.772069in}{0.515123in}}{\pgfqpoint{3.875000in}{2.695000in}}%
\pgfusepath{clip}%
\pgfsetbuttcap%
\pgfsetroundjoin%
\definecolor{currentfill}{rgb}{0.196078,0.803922,0.196078}%
\pgfsetfillcolor{currentfill}%
\pgfsetlinewidth{1.003750pt}%
\definecolor{currentstroke}{rgb}{0.196078,0.803922,0.196078}%
\pgfsetstrokecolor{currentstroke}%
\pgfsetdash{}{0pt}%
\pgfpathmoveto{\pgfqpoint{3.286690in}{1.412444in}}%
\pgfpathcurveto{\pgfqpoint{3.297740in}{1.412444in}}{\pgfqpoint{3.308339in}{1.416834in}}{\pgfqpoint{3.316153in}{1.424648in}}%
\pgfpathcurveto{\pgfqpoint{3.323966in}{1.432462in}}{\pgfqpoint{3.328357in}{1.443061in}}{\pgfqpoint{3.328357in}{1.454111in}}%
\pgfpathcurveto{\pgfqpoint{3.328357in}{1.465161in}}{\pgfqpoint{3.323966in}{1.475760in}}{\pgfqpoint{3.316153in}{1.483574in}}%
\pgfpathcurveto{\pgfqpoint{3.308339in}{1.491387in}}{\pgfqpoint{3.297740in}{1.495777in}}{\pgfqpoint{3.286690in}{1.495777in}}%
\pgfpathcurveto{\pgfqpoint{3.275640in}{1.495777in}}{\pgfqpoint{3.265041in}{1.491387in}}{\pgfqpoint{3.257227in}{1.483574in}}%
\pgfpathcurveto{\pgfqpoint{3.249413in}{1.475760in}}{\pgfqpoint{3.245023in}{1.465161in}}{\pgfqpoint{3.245023in}{1.454111in}}%
\pgfpathcurveto{\pgfqpoint{3.245023in}{1.443061in}}{\pgfqpoint{3.249413in}{1.432462in}}{\pgfqpoint{3.257227in}{1.424648in}}%
\pgfpathcurveto{\pgfqpoint{3.265041in}{1.416834in}}{\pgfqpoint{3.275640in}{1.412444in}}{\pgfqpoint{3.286690in}{1.412444in}}%
\pgfpathclose%
\pgfusepath{stroke,fill}%
\end{pgfscope}%
\begin{pgfscope}%
\pgfpathrectangle{\pgfqpoint{0.772069in}{0.515123in}}{\pgfqpoint{3.875000in}{2.695000in}}%
\pgfusepath{clip}%
\pgfsetbuttcap%
\pgfsetroundjoin%
\definecolor{currentfill}{rgb}{0.196078,0.803922,0.196078}%
\pgfsetfillcolor{currentfill}%
\pgfsetlinewidth{1.003750pt}%
\definecolor{currentstroke}{rgb}{0.196078,0.803922,0.196078}%
\pgfsetstrokecolor{currentstroke}%
\pgfsetdash{}{0pt}%
\pgfpathmoveto{\pgfqpoint{3.681562in}{1.531550in}}%
\pgfpathcurveto{\pgfqpoint{3.692612in}{1.531550in}}{\pgfqpoint{3.703211in}{1.535940in}}{\pgfqpoint{3.711025in}{1.543754in}}%
\pgfpathcurveto{\pgfqpoint{3.718839in}{1.551567in}}{\pgfqpoint{3.723229in}{1.562166in}}{\pgfqpoint{3.723229in}{1.573216in}}%
\pgfpathcurveto{\pgfqpoint{3.723229in}{1.584267in}}{\pgfqpoint{3.718839in}{1.594866in}}{\pgfqpoint{3.711025in}{1.602679in}}%
\pgfpathcurveto{\pgfqpoint{3.703211in}{1.610493in}}{\pgfqpoint{3.692612in}{1.614883in}}{\pgfqpoint{3.681562in}{1.614883in}}%
\pgfpathcurveto{\pgfqpoint{3.670512in}{1.614883in}}{\pgfqpoint{3.659913in}{1.610493in}}{\pgfqpoint{3.652100in}{1.602679in}}%
\pgfpathcurveto{\pgfqpoint{3.644286in}{1.594866in}}{\pgfqpoint{3.639896in}{1.584267in}}{\pgfqpoint{3.639896in}{1.573216in}}%
\pgfpathcurveto{\pgfqpoint{3.639896in}{1.562166in}}{\pgfqpoint{3.644286in}{1.551567in}}{\pgfqpoint{3.652100in}{1.543754in}}%
\pgfpathcurveto{\pgfqpoint{3.659913in}{1.535940in}}{\pgfqpoint{3.670512in}{1.531550in}}{\pgfqpoint{3.681562in}{1.531550in}}%
\pgfpathclose%
\pgfusepath{stroke,fill}%
\end{pgfscope}%
\begin{pgfscope}%
\pgfpathrectangle{\pgfqpoint{0.772069in}{0.515123in}}{\pgfqpoint{3.875000in}{2.695000in}}%
\pgfusepath{clip}%
\pgfsetbuttcap%
\pgfsetroundjoin%
\definecolor{currentfill}{rgb}{0.196078,0.803922,0.196078}%
\pgfsetfillcolor{currentfill}%
\pgfsetlinewidth{1.003750pt}%
\definecolor{currentstroke}{rgb}{0.196078,0.803922,0.196078}%
\pgfsetstrokecolor{currentstroke}%
\pgfsetdash{}{0pt}%
\pgfpathmoveto{\pgfqpoint{4.015685in}{1.643363in}}%
\pgfpathcurveto{\pgfqpoint{4.026735in}{1.643363in}}{\pgfqpoint{4.037334in}{1.647754in}}{\pgfqpoint{4.045148in}{1.655567in}}%
\pgfpathcurveto{\pgfqpoint{4.052962in}{1.663381in}}{\pgfqpoint{4.057352in}{1.673980in}}{\pgfqpoint{4.057352in}{1.685030in}}%
\pgfpathcurveto{\pgfqpoint{4.057352in}{1.696080in}}{\pgfqpoint{4.052962in}{1.706679in}}{\pgfqpoint{4.045148in}{1.714493in}}%
\pgfpathcurveto{\pgfqpoint{4.037334in}{1.722306in}}{\pgfqpoint{4.026735in}{1.726697in}}{\pgfqpoint{4.015685in}{1.726697in}}%
\pgfpathcurveto{\pgfqpoint{4.004635in}{1.726697in}}{\pgfqpoint{3.994036in}{1.722306in}}{\pgfqpoint{3.986222in}{1.714493in}}%
\pgfpathcurveto{\pgfqpoint{3.978409in}{1.706679in}}{\pgfqpoint{3.974018in}{1.696080in}}{\pgfqpoint{3.974018in}{1.685030in}}%
\pgfpathcurveto{\pgfqpoint{3.974018in}{1.673980in}}{\pgfqpoint{3.978409in}{1.663381in}}{\pgfqpoint{3.986222in}{1.655567in}}%
\pgfpathcurveto{\pgfqpoint{3.994036in}{1.647754in}}{\pgfqpoint{4.004635in}{1.643363in}}{\pgfqpoint{4.015685in}{1.643363in}}%
\pgfpathclose%
\pgfusepath{stroke,fill}%
\end{pgfscope}%
\begin{pgfscope}%
\pgfpathrectangle{\pgfqpoint{0.772069in}{0.515123in}}{\pgfqpoint{3.875000in}{2.695000in}}%
\pgfusepath{clip}%
\pgfsetbuttcap%
\pgfsetroundjoin%
\definecolor{currentfill}{rgb}{0.196078,0.803922,0.196078}%
\pgfsetfillcolor{currentfill}%
\pgfsetlinewidth{1.003750pt}%
\definecolor{currentstroke}{rgb}{0.196078,0.803922,0.196078}%
\pgfsetstrokecolor{currentstroke}%
\pgfsetdash{}{0pt}%
\pgfpathmoveto{\pgfqpoint{4.440932in}{1.772192in}}%
\pgfpathcurveto{\pgfqpoint{4.451982in}{1.772192in}}{\pgfqpoint{4.462581in}{1.776582in}}{\pgfqpoint{4.470395in}{1.784396in}}%
\pgfpathcurveto{\pgfqpoint{4.478209in}{1.792210in}}{\pgfqpoint{4.482599in}{1.802809in}}{\pgfqpoint{4.482599in}{1.813859in}}%
\pgfpathcurveto{\pgfqpoint{4.482599in}{1.824909in}}{\pgfqpoint{4.478209in}{1.835508in}}{\pgfqpoint{4.470395in}{1.843321in}}%
\pgfpathcurveto{\pgfqpoint{4.462581in}{1.851135in}}{\pgfqpoint{4.451982in}{1.855525in}}{\pgfqpoint{4.440932in}{1.855525in}}%
\pgfpathcurveto{\pgfqpoint{4.429882in}{1.855525in}}{\pgfqpoint{4.419283in}{1.851135in}}{\pgfqpoint{4.411470in}{1.843321in}}%
\pgfpathcurveto{\pgfqpoint{4.403656in}{1.835508in}}{\pgfqpoint{4.399266in}{1.824909in}}{\pgfqpoint{4.399266in}{1.813859in}}%
\pgfpathcurveto{\pgfqpoint{4.399266in}{1.802809in}}{\pgfqpoint{4.403656in}{1.792210in}}{\pgfqpoint{4.411470in}{1.784396in}}%
\pgfpathcurveto{\pgfqpoint{4.419283in}{1.776582in}}{\pgfqpoint{4.429882in}{1.772192in}}{\pgfqpoint{4.440932in}{1.772192in}}%
\pgfpathclose%
\pgfusepath{stroke,fill}%
\end{pgfscope}%
\begin{pgfscope}%
\pgfsetrectcap%
\pgfsetmiterjoin%
\pgfsetlinewidth{0.803000pt}%
\definecolor{currentstroke}{rgb}{0.000000,0.000000,0.000000}%
\pgfsetstrokecolor{currentstroke}%
\pgfsetdash{}{0pt}%
\pgfpathmoveto{\pgfqpoint{0.772069in}{0.515123in}}%
\pgfpathlineto{\pgfqpoint{0.772069in}{3.210123in}}%
\pgfusepath{stroke}%
\end{pgfscope}%
\begin{pgfscope}%
\pgfsetrectcap%
\pgfsetmiterjoin%
\pgfsetlinewidth{0.803000pt}%
\definecolor{currentstroke}{rgb}{0.000000,0.000000,0.000000}%
\pgfsetstrokecolor{currentstroke}%
\pgfsetdash{}{0pt}%
\pgfpathmoveto{\pgfqpoint{4.647069in}{0.515123in}}%
\pgfpathlineto{\pgfqpoint{4.647069in}{3.210123in}}%
\pgfusepath{stroke}%
\end{pgfscope}%
\begin{pgfscope}%
\pgfsetrectcap%
\pgfsetmiterjoin%
\pgfsetlinewidth{0.803000pt}%
\definecolor{currentstroke}{rgb}{0.000000,0.000000,0.000000}%
\pgfsetstrokecolor{currentstroke}%
\pgfsetdash{}{0pt}%
\pgfpathmoveto{\pgfqpoint{0.772069in}{0.515123in}}%
\pgfpathlineto{\pgfqpoint{4.647069in}{0.515123in}}%
\pgfusepath{stroke}%
\end{pgfscope}%
\begin{pgfscope}%
\pgfsetrectcap%
\pgfsetmiterjoin%
\pgfsetlinewidth{0.803000pt}%
\definecolor{currentstroke}{rgb}{0.000000,0.000000,0.000000}%
\pgfsetstrokecolor{currentstroke}%
\pgfsetdash{}{0pt}%
\pgfpathmoveto{\pgfqpoint{0.772069in}{3.210123in}}%
\pgfpathlineto{\pgfqpoint{4.647069in}{3.210123in}}%
\pgfusepath{stroke}%
\end{pgfscope}%
\end{pgfpicture}%
\makeatother%
\endgroup%

    \caption{Voltajes (\textcolor{Blue}{$V$}, \textcolor{Red}{$V_1$}, \textcolor{Yellow}{$V_2$}, \textcolor{Green}{$V_3$}) frente a intensidad (I) con regresión lineal}
  \end{figure}

  En cuanto a los coeficientes de regresión lineal, de nuevo obtenemos resultados satisfactorios: $r = 0,99997$, $r_1 = 0,99998$, $r_2 = 0,99997$ y $r_3 = 0,99997$, un ajuste de cuatro nueves en ambos casos.


  \subsection{Circuito en Paralelo}

  En este apartado construiremos un circuito utilizando tres resistencias ($R_1$, $R_2$ y $R_3$) colocadas en paralelo. Ahora mediremos la intensidad en cada resistencia y total, además del voltaje del circuito. Colocamos los componentes siguiendo el diagrama:

  \begin{figure}[H]
    \centering
    \begin{circuitikz}[european]
      \draw (0,0) to[voltage source] (0,3) -- (3.25,3)
      to[R=$R_2$] (3.25,0) -- (0,0);
      \draw (3.25, 2.75) -- (2, 2.75)
      to[R=$R_1$] (2, 0.25) -- (4, 0.25);
      \draw (3.25, 2.75) -- (4.5, 2.75)
      to[R=$R_3$] (4.5, 0.25) -- (4, 0.25);
    \end{circuitikz}
    \caption{Circuito con tres resistencias en paralelo}
    \label{circuito:paralelo}
  \end{figure}

  \subsubsection{Procedimiento de medición}

  Con el objetivo de medir las distintas magnitudes, conectaremos el polímetro en serie para las itensidades y en paralelo para los voltajes.

  \begin{figure}[H]
    \centering
    \raisebox{0.53in}{
    \begin{circuitikz}[european]
      \draw (0,0) to[voltage source] (0,3) -- (3.25,3)
      to[R=$R_2$] (3.25,0) -- (0,0);
      \draw (3.25, 2.75) -- (2, 2.75)
      to[R=$R_1$] (2, 0.25) -- (4, 0.25);
      \draw (3.25, 2.75) -- (4.5, 2.75)
      to[R=$R_3$] (4.5, 0.25) -- (4, 0.25);
      \draw (3.25, 3) -- (6.5, 3)
      to[voltmeter] (6.5, 0) -- (3.25, 0);
    \end{circuitikz}} \qquad
    \begin{circuitikz}[european]
      \draw (0,0) to[voltage source] (0,5) -- (4.5,5) -- (4.5, 4.75)
      to[R=$R_2$] (4.5, 2.25) to[ammeter, l=$I_2$] (4.5,0.25) -- (4.5, 0)
      to[ammeter] (0,0);
      \draw (4.5, 4.75) -- (3, 4.75)
      to[R=$R_1$] (3, 2.25) to[ammeter, l=$I_1$] (3, 0.25) -- (4.5, 0.25);
      \draw (4.5, 4.75) -- (6, 4.75)
      to[R=$R_3$] (6, 2.25) to[ammeter, l=$I_3$] (6, 0.25) -- (4.5, 0.25);
    \end{circuitikz}
    \caption{Medición de potencial del circuito ($V$) y de intensidades, total ($I$) y de cada resistencia ($I_1$, $I_2$ y $I_3$ respectivamente)}
  \end{figure}

  \subsubsection{Resistencia equivalente}

  Calculemos la resistencia equivalente al circuito aplicando la fórmula para resistencias en paralelo:
  \begin{equation}
      \label{eq:resparalelo}
      \frac{1}{R_P} = \sum^N_{k=1} \frac{1}{R_k} \qquad \frac{1}{R_P} = \frac{1}{R_1} + \frac{1}{R_2} + \frac{1}{R_3}
  \end{equation}
  Además de esto sabemos que la intensidad total del circuito se distribuirá entre las tres resistencias, verificándose que:
  \begin{equation}
      I = \sum^N_{k=1} I_k \qquad I = I_1 + I_2 + I_3
  \end{equation}
  Con esta información podemos calcular la resistencia equivalente al circuito aplicando la fórmula \ref{eq:resparalelo}:
  \begin{gather}
      \frac{1}{R_P} = \frac{1}{175500} + \frac{1}{216000} + \frac{1}{394000} \qquad R_P = \frac{1}{\frac{1}{175500} + \frac{1}{216000} + \frac{1}{394000}} = 7,77 \cdot 10^4 \Omega \nonumber \\
      s(R_P) = \frac{1}{\frac{1}{\pm100} + \frac{1}{\pm1000} + \frac{1}{\pm1000}} = \pm83,3 \Omega \qquad R_P = 7,77 \cdot 10^4 \pm 83,3 \Omega \nonumber
  \end{gather}

  \subsubsection{Medición experimental}

  Utilizando el procedimiento descrito con anterioridad, realizamos una serie de mediciones en el circuito \ref{circuito:paralelo}. Variaremos el voltaje para obtener diferentes medidas de los voltajes (\textit{V}) e intensidades (\textit{I}). El resultado se expone en la siguiente tabla (Por cuestiones de presentación se omitirá $\pm s(I)$ en las columnas correspondientes, ya que quedaría redundante y sólo dificultaría la lectura sin aportar información. $s(I) = s(I_1) = s(I_2) = s(I_3) = 1 \cdot 10^-7 A$):

  \begin{table}[H]
  \centering
  \csvreader[
    tabular=|c|c|c|c|c|c|,
    table head=\hline Medida & $V~(V) \pm s(V)$ & $I_1~(V)$ & $I_2~(V)$ & $I_3~(V)$ & $I~(A)$ \\ \hline,
    late after last line=\\\hline,
    separator=semicolon
    ]{CC7.csv}
    {v=\v, i1=\ia, i2=\ib, i3=\ic, i=\int, sv=\sv}
    {\thecsvrow & \v \hspace{4pt}$\pm$ \sv & \ia & \ib & \ic & \int }
  \caption{Potenciales e intensidades del circuito en paralelo}
  \end{table}









  \section{Conclusiones}


  \newpage
  \part{Corriente Alterna}

  \newpage
  \begin{appendices}
    \addtocontents{toc}{\protect\setcounter{tocdepth}{2}}
    \makeatletter
    \addtocontents{toc}{%
    \begingroup
    \let\protect\l@chapter\protect\l@section
    \let\protect\l@section\protect\l@subsection
    }

    \section{Bibliografía}

    (All the info)

    \addtocontents{toc}{\endgroup}
  \end{appendices}


\end{document}
