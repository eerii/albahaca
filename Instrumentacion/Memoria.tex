\documentclass[12pt, a4paper, titlepage]{article}

%INFORMACIÓN
\title{\textbf {Instrumentación Electrónica: Medición de circuitos de corriente continua y corriente alterna}}
\author{{\Large Pazos Pérez, José}\\DNI}
\date{}

%PAQUETES
%\usepackage[utf8]{inputenc} %Añadir acentos y caracteres (Windows y Linux)

\usepackage[dvipsnames]{xcolor} %Colorear texto y colores estándar
\usepackage{colortbl} %Colorear celdas de tablas
%COLOR
\definecolor{Black}{RGB}{31, 31, 31}
\definecolor{Brown}{RGB}{179, 131, 55}
\definecolor{Red}{RGB}{222, 94, 80}
\definecolor{Orange}{RGB}{245, 149, 76}
\definecolor{Yellow}{RGB}{247, 192, 62}
\definecolor{Green}{RGB}{122, 217, 119}
\definecolor{Blue}{RGB}{110, 176, 230}
\definecolor{Violet}{RGB}{176, 110, 230}
\definecolor{Grey}{RGB}{189, 189, 189}
\definecolor{DarkGrey}{RGB}{100, 100, 100}
\definecolor{White}{RGB}{245, 245, 245}
\definecolor{Golden}{RGB}{207, 194, 97}
\definecolor{Silver}{RGB}{179, 193, 196}
\definecolor{LinkBlue}{RGB}{20, 88, 224}

\usepackage[centertags]{amsmath} %Excluir ecuaciones de la enumeración automática
\usepackage{csvsimple} %Tablas desde archivos .csv
\usepackage{pgfplots} %Gráficas desde matplotlib con .pgf
\usepackage{graphicx} %Imágenes
\usepackage[siunitx]{circuitikz} %Circuitos
\usepackage{pythonhighlight} %Código de python
\pgfplotsset{compat=1.16}

\usepackage{tocloft} %Crear listas (por ejemplo, de ecuaciones)
\usepackage{enumitem} %Cambiar los estilos de las listas
\usepackage{verbatim} %Comentarios de varias lineas
\usepackage[margin=0.8in]{geometry} %Márgenes
\usepackage[skip=12pt]{parskip} %Añadir espacio tras los párrafos
\usepackage{float} %Controlar el posicionamiento de gráficas y tablas con H
\usepackage[toc,page]{appendix} %Anexos
\usepackage{chngcntr} %Numeración de capítulos por partes
\usepackage{hyperref} %Añadir vínculos
\hypersetup{
    colorlinks=true,
    linkcolor=LinkBlue,
    filecolor=Red,
    urlcolor=Blue,
}

%CONFIGURACIÓN
\renewcommand{\contentsname}{Índice}
\renewcommand{\partname}{Experiencia}
\renewcommand{\listtablename}{Lista de Tablas}
\renewcommand{\listfigurename}{Lista de Figuras}
\renewcommand{\appendixpagename}{Anexos}
\renewcommand{\appendixtocname}{\large Anexos}
\renewcommand{\appendixname}{Anexo}
\renewcommand{\figurename}{Figura}
\renewcommand{\tablename}{Tabla}

\newcommand{\listecuacionesname}{\Large Lista de Ecuaciones}
\newlistof{ecuaciones}{equ}{\listecuacionesname}
\newcommand{\ecuaciones}[1]{\addcontentsline{equ}{ecuaciones}{\protect\numberline{\theequation}#1}\par}

\linespread{1.3}
\counterwithin*{section}{part}

\newcommand{\code}[1]{\texttt{#1}} %Formatear texto como código




%DOCUMENTO
\begin{document}
  \maketitle

  \tableofcontents

  \newpage
  \part*{Introducción}
  \addcontentsline{toc}{part}{Introducción}

  El objetivo de esta memoria es ofrecer un informe detallado de las prácticas de instrumentación, detallando las distintas experiencias llevadas a cabo con circuitos de \textbf{corriente continua} y \textbf{corriente alterna}.

  \section{Material}

  \subsection{Corriente Continua (Experiencia I)}

  \begin{itemize}[label=$-$]
    \item Polímetro
    \item Fuente de alimentación (CC)
    \item Resistencias de $180k\Omega$, $220k\Omega$, $390k\Omega$ y $1M\Omega$
    \item Tablero de conexiones
    \item Cables
  \end{itemize}

  \subsection{Corriente Alterna (Experiencia II)}
  \begin{itemize}[label=$-$]
    \item Osciloscopio
    \item Fuente de alimentación (CA, senosoidal)
    \item Resistencia de $10k\Omega$
    \item Condensador de $12nF$
    \item Tablero de conexiones
    \item Cables
  \end{itemize}


  \section{Tratamiento de datos}

  Para una correcta interpretación de los resultados expuestos en las próximas páginas, explicaremos las distintas metodologías y convenciones sobre el tratamiento de los datos de las mismas. En este apartado se incluyen la explicación de los métodos de \textbf{regresión lineal} y \textbf{propagación de incertidumbres}.

  \subsection{Reglas de redondeo}
  \label{sec:redondeo}

  En todas las mediciones expuestas utilizaremos los siguientes métodos de redondeo, aquellos convenidos en las jornadas de introducción:

  \begin{enumerate}
    \item Si la cifra en la posición n+1 es mayor que 5, la cifra n se incrementa en una unidad.
    \item Si la cifra en la posición n+1 es menor que 5, la cifra n se mantiene igual.
    \item Si la cifra en la posición n+1 es igual a 5, y alguna de las otras cifras suprimidas es distinta de 0, la cifra n se incrementa en una unidad.
    \item Si la cifra en la posición n+1 es igual a 5, y el resto de cifras suprimidas son iguales a 0, la cifra n se mantiene igual si es par y se incrementa en una unidad si es impar.
  \end{enumerate}

  \subsection{Incertidumbre de medidas directas}
  \label{sec:incert}

  En la práctica de corriente continua debemos indicar las incertidumbres de las mediciones realizadas. Como el polímetro utilizado es un aparato digital, consideraremos que una estimación de la incertidumbre sobre el valor real será la resolución ($\Delta x$) del mismo. Por lo tanto, tomaremos una unidad de la última cifra que muestre el aparato cómo $s_B (x)$.

  Como en estas medidas no existe otro tipo de incertidumbre a la que podamos aplicar un tratamiento estadístico, consideraremos $s_B (x)$ la incertidumbre final de la medida.

  \subsection{Múltiples medidas directas}

  Si realizamos una serie de medidas directas, la desviación típica de la media ($s_A (\bar{x})$) representa la incertidumbre de cualquier medida realizada con el mismo instrumento bajo las mismas condiciones. Con ambos datos podemos calcular la incertidumbre combinada:
  \begin{equation} \label{ec:scx}
    s_C (\bar{x}) = \sqrt{\left[ s_A (\bar{x}) \right]^2 + \left[ s_B (x) \right]^2}
  \end{equation}

  \subsection{Regresión lineal simple}
  \label{sec:reglin}

  Para ajustar los una serie de datos que parezcan seguir una relación lineal podemos utilizar el método de \textbf{ajuste por mínimos cuadrados}. Al ser \textit{lineal}, podemos ajustarla por una recta general de la forma $y = \alpha + \beta x$. El problema a resolver es conseguir la mejor aproximación \textit{a}, \textit{b} de los parámetros $\alpha, \beta$ y sus incertidumbres utilizando nuestra serie de parámetros $\left\{ x_i, y_i \right\}$.

  Para ello minimizaremos la suma de los productos del peso estadístico de cada punto, $w_i$, por el cuadrado de la desviación de los datos, $[y_i - (a + bx_i)]^2$. Por lo tanto, las derivadas parciales respecto a \textit{a} y \textit{b} deben de ser 0.
  \begin{equation} \label{ec:chi2}
    \chi^2 = \sum^{n}_{i=1} w_i[y_i - (a + bx_i)]^2
  \end{equation}
  \begin{equation} \label{ec:deriv}
    \frac{\partial \chi^2}{\partial a}=0, \frac{\partial \chi^2}{\partial b}=0
  \end{equation}

  De aquí resultan dos posibles casos:

  \begin{enumerate}
    \item Si las incertidumbres de $x_i$ no son despreciables respecto a las de $y_i$. Obtenemos la siguiente ecuación: $\omega_i=\frac{1}{s^2(y_i)+b^2s^2(x_i)}$, que excede el nivel de este curso.
    \item Si por el contrario podemos despreciar las incertidumbres de $x_i$ respecto a $y_i$, podemos simplificar la ecuación anterior a $\omega_i=[s(y_i)]^{-2}$.
  \end{enumerate}

  En nuestro caso, tanto en la estimación de una resistencia mediante la ley de Ohm como en el circuito en serie se pueden despreciar las incertidumbres. En ambos representaremos voltaje (V) frente a intensidad (I). La incertidumbre de la intensidad es del orden de $s(x_i)=10^{-7}$, mientras que la del voltaje es del orden de $s(y_i)=10^{-2}$, por lo que podremos aplicar el segundo método.

  Además, podemos considerar que las incertidumbres de $y_i$ permanecen constantes entre las diferentes medidas, por lo que también lo hará el peso estadístico $w = cte$. Podemos sustituír en la fórmula \ref{ec:chi2} y derivar respecto a \textit{a} y \textit{b} para obtener:
  \begin{gather}
    \chi^2 = w\sum^{n}_{i=1}[y_i - (a + bx_i)]^2 \nonumber \\
    a n + b\sum_{i}x_i = \sum_{i}y_i \quad \quad \sum_{i}x_i + b\sum_{i}x_i^2 = \sum_{i}x_i y_i \label{ec:dchi}
  \end{gather}

  Finalmente, conseguimos las expresiones matemáticas de \textit{a} y \textit{b} en base a la serie de medidas $\left\{ x_i, y_i \right\}$:
  \begin{gather}
    a = \frac{\left ( \sum_i y_i \right )\left ( \sum_i x_i^2 \right ) - \left ( \sum_i x_i \right )\left ( \sum_i x_i y_i \right )}{n\left ( \sum_i x_i^2 \right ) - \left ( \sum_i x_i \right )^2} \label{ec:a} \\
    b = \frac{n\left ( \sum_i x_i y_i \right ) - \left ( \sum_i x_i \right )\left ( \sum_i y_i \right )}{n\left ( \sum_i x_i^2 \right )-\left ( \sum_i x_i \right )^2} \label{ec:b}
  \end{gather}

  Además de los coeficientes para ajustar la recta, también podemos obtener otras magnitudes de interés sobre nuestra muestra:
  \begin{figure}[H]
    \begin{equation} \label{ec:s}
      s = \sqrt{\frac{\sum_i \left ( y_i - a - bx_i \right )^2}{1}}
    \end{equation}
    \caption{Desviación típica del ajuste para la muestra}
  \end{figure}
  \begin{figure}[H]
    \begin{equation} \label{ec:sab}
      s(a) = s\sqrt{\frac{\sum_i x_i^2}{n\left ( \sum_i x_i^2 \right ) - \left ( \sum_i x_i \right )^2}} \quad \quad s(b) = s\sqrt{\frac{n}{n\left ( \sum_i x_i^2 \right ) - \left ( \sum_i x_i \right )^2}}
    \end{equation}
    \caption{Incertidumbres de los parámetros \textit{a} y \textit{b}}
  \end{figure}
  \begin{figure}[H]
    \begin{equation} \label{ec:r}
      r = \frac{n\left ( \sum_i x_i y_i \right ) - \left ( \sum_i x_i \right )\left ( \sum_i y_i \right )}{\sqrt{\left [ n\left ( \sum_i x_i^2 \right ) - \left ( \sum_i x_i \right ) ^2\right ]\left [ n\left ( \sum_i y_i^2 \right ) - \left ( \sum_i y_i \right ) ^2 \right ]}}
    \end{equation}
    \caption{Coeficiente de regresión lineal}
  \end{figure}

  \subsection{Propagación de incertidumbres}
  \label{sec:propinc}

  Hay que hayar la incertidumbre correcta para aquellas medidas que sean indirectas (no se miden experimentalmente, si no que se aplican fórmulas a mediciones experimentales). Para ello empleamos el método de propagación de incertidumbres, que nos da la desviación estándar combinada:
  \begin{equation} \label{ec:propinc}
    s(y) = \sqrt{\sum_i\left ( \frac{\partial y}{\partial x_i} \right )^2 s^2(x_i)}
  \end{equation}

  Las magnitudes $x_i$ son aquellas medidas experimentalmente de las que depende la magnitud indirecta $y$. Todas las magnitudes $x_i$ han de ser independientes entre si.



  \newpage
  \part{Corriente Continua}

  \section{Objetivos}
  \begin{itemize}[label=$-$]
    \item Comprobar que el código de colores de las resistencias se corresponde con su valor real.
    \item Verificar el cumplimiento de la ley de Ohm ($V = I\cdot R$) en un circuito simple.
    \item Verificar las leyes de asociación de resistencias (ley de Kirchhoff) en circuitos en serie, paralelo y mixto.
    \item Perfeccionar el manejo del polímetro y demás utensilios del laboratorio.
  \end{itemize}

  \section{Medida de resistencias}
  El material de la práctica incluye cuatro resistencias. Para conocer sus valores utilizaremos dos métodos distintos, uno teórico y uno experimental, y luego contrastaremos los resultados.

  \subsection{Código de colores}
  Las resistencias tienen cuatro bandas de colores que indican su \textbf{valor nominal}. Estos colores se rigen por el siguiente código de resistencias estándar para cuatro bandas.

  \begin{figure}[H]
    \centering
    \begin{tabular}{|c|c|c|c|c|}
      \hline
      Color & 1º & 2º & Multiplicador & Tolerancia \\
      \hline
      \rowcolor{Black}
      \color{White} Negro & \color{White} 0 & \color{White} 0 & \color{White} 1$\Omega$ & \color{White} - \\
      \hline
      \rowcolor{Brown}
      Marrón & 1 & 1 & 10$\Omega$ & $\pm$1\% \\
      \hline
      \rowcolor{Red}
      Rojo & 2 & 2 & 100$\Omega$ & $\pm$2\% \\
      \hline
      \rowcolor{Orange}
      Naranja & 3 & 3 & 1k$\Omega$ & - \\
      \hline
      \rowcolor{Yellow}
      Amarillo & 4 & 4 & 10k$\Omega$ & - \\
      \hline
      \rowcolor{Green}
      Verde & 5 & 5 & 100k$\Omega$ & $\pm$0,5\% \\
      \hline
      \rowcolor{Blue}
      Azul & 6 & 6 & 1M$\Omega$ & $\pm$0,25\% \\
      \hline
      \rowcolor{Violet}
      Violeta & 7 & 7 & 10M$\Omega$ & $\pm$0,1\% \\
      \hline
      \rowcolor{Grey}
      Gris & 8 & 8 & 100M$\Omega$ & $\pm$0,05\% \\
      \hline
      \rowcolor{White}
      Blanco & 9 & 9 & 1G$\Omega$ & - \\
      \hline
      \rowcolor{Golden}
      Dorado & - & - & 0,1$\Omega$ & $\pm$5\% \\
      \hline
      \rowcolor{Silver}
      Plateado & - & - & 0,01$\Omega$ & $\pm$10\% \\
      \hline
    \end{tabular}
    \caption{Código de colores para resistencias}
  \end{figure}

  En base a esta tabla podemos calcular el valor teórico de la resistencia y su indeterminación (La tolerancia por el valor nominal). La primera banda indicará la primera cifra (A), la segunda banda será la segunda cifra (B), y la tercera banda la potencia de diez a la que está elevado ($C = 10^n$). Por lo tanto, el número resultante será de la forma $AB\cdot C$. La última banda representa la toleracia de acuerdo a la tabla anterior. Así tenemos los siguientes resultados, ordenados de menor a mayor.

  \begin{table}[H]
  \centering
  \csvreader[
    tabular=|c|c|c|c|c|,
    table head=\hline Resistencia & Color & V.~Nominal~($\Omega$) & Tolerancia~(\%) & s(R)~($\Omega$) \\ \hline,
    late after last line=\\\hline,
    separator=semicolon
    ]{CC51.csv}
    {r=\r, color=\rescolor, vn=\vn, t=\t, sr=\sr}
    {\r & \rescolor & \vn & \t & \sr}
  \caption{Medida del valor nominal de las resistencias}
  \label{tb:valnomres}
  \end{table}

  \subsection{Medida directa}
  \label{sec:medidaresist}
  Ahora utilizaremos el polímetro para determinar el valor experimental de cada resistencia y comprobar si se corresponde al valor teórico. Configuramos el polímetro para la medición de resistencias, y obtenemos los siguientes resultados:

  \begin{table}[H]
  \centering
  \csvreader[
    tabular=|c|c|c|c|,
    table head=\hline Resistencia & Lectura~($\Omega$) & Resolución~($\Omega$) & $R \pm s(R)~(\Omega)$ \\ \hline,
    late after last line=\\\hline,
    separator=semicolon
    ]{CC52.csv}
    {r=\r, lectura=\lectura, resolucion=\resolucion, sr=\sr}
    {\r & \lectura & \resolucion & \lectura \hspace{4pt}$\pm$ \sr}
  \caption{Medida del valor experimental de las resistencias}
  \label{tab:res}
  \end{table}

  Podemos observar que los resultados experimentales entran dentro del umbral de confianza del 5\% de los obtenidos teóricamente, por lo que asumiremos que son correctos.


  \newpage
  \section{Ley de Ohm}

  \subsection{Explicación teórica}
  Si tomamos un circuito eléctrico de corriente continua, podemos derivar la siguiente relación entre la diferencia de potencial del circuito (\textit{V}), la intensidad que circula por él (\textit{I}) y la resitencia eléctrica del material (\textit{R}). A esta relación la llamamos \textbf{Ley de Ohm}.
  \begin{equation} \label{ec:ohm}
    V = I \cdot R
  \end{equation}
  Siguiendo el sistema internacional (SI), \textit{V} se expresa en Voltios (\textit{V}), \textit{I} en Amperios (\textit{A}) y \textit{R} en Ohmios ($\Omega$).

  \subsection{Estimación indirecta}
  Ahora comprobaremos experimentalmente el cumplimiento de la Ley de Ohm. Para ello construiremos un circuito simple, utilizando una fuente de corriente continua y una resistencia ($R_1$: 180k$\Omega$). Lo dispondremos de la siguiente manera:

  \begin{figure}[H]
    \centering
    \begin{circuitikz}[european]
      \draw (0,0) to[voltage source] (0,3) -- (3,3)
      to[R=$R_1$] (3,0) -- (0,0);
    \end{circuitikz}
    \caption{Circuito simple}
  \end{figure}

  Colocaremos el polímetro en dos posiciones: Para medir la intensidad, en serie; para medir el potencial, en paralelo alrededor de la resistencia. Podemos verlo en el diagrama mostrado a continuación.

  \begin{figure}[H]
    \centering
    \begin{circuitikz}[european]
      \draw (0,0) to[voltage source] (0,3) -- (2,3)
      to[R=$R_1$] (2,0) -- (0,0);
      \draw (2,3) -- (4,3)
      to[voltmeter] (4,0) -- (2,0);
    \end{circuitikz}
    \quad \quad
    \begin{circuitikz}[european]
      \draw (0,0) to[voltage source] (0,3)
      to[ammeter](3,3)
      to[R=$R_1$] (3,0) -- (0,0);
    \end{circuitikz}
    \caption{Medición de potenciales y intensidades}
  \end{figure}

  Ahora configuramos la fuente de corriente continua para un voltaje $V_1$, y medimos con el polímetro primero el voltaje (Porque la medida de la fuente no siempre es fiable) y luego la intensidad asociada a ese voltaje, $I_1$. Realizamos sucesivas mediciones aumentando progresivamente el voltaje, y obtendremos unos datos, por ejemplo:

  \begin{table}[H]
  \centering
  \csvreader[
    tabular=|c|c|c|,
    table head=\hline Medida & $V \pm s(V)~(V)$ & $I \pm s(I)~(I)$ \\ \hline,
    late after last line=\\\hline,
    filter test=\ifnumless{\thecsvrow}{2},
    separator=semicolon
    ]{CC53.csv}
    {v=\v, sv=\sv, i=\int, si=\si}
    {\thecsvrow & \v \hspace{4pt}$\pm$ \sv & \int \hspace{4pt}$\pm$ \si}
  \caption{Ejemplo de las mediciones de voltaje e intensidad}
  \end{table}

  Sim embargo, nuestro objetivo es comprobar si se cumple la Ley de Ohm, y para eso debemos de calcular el valor de las resistencias de manera indirecta. Para eso utilizaremos la ecuación \ref{ec:ohm} y despejaremos para \textit{R}.
  \begin{equation*}
    R = \frac{V}{I}
  \end{equation*}
  Ahora debemos de calcular la incertidumbre de \textit{R}, \textit{s(R)}. Para ello utilizamos el método de propagación de incertidumbres descrito en la sección \ref{sec:propinc}, y utilizaremos la ecuación \ref{ec:propinc}. Tomaremos \textit{V} como $x_1$, \textit{I} como $x_2$ y \textit{R} como \textit{y}.
  \begin{gather}
    s(R)) = \sqrt{\left(\frac{\partial R}{\partial V}\right)^2 s^2(V) + \left(\frac{\partial R}{\partial I}\right)^2 s^2(I)} \quad \quad
    \frac{\partial R}{\partial V} = \frac{1}{I} \quad \quad
    \frac{\partial R}{\partial I} = -\frac{V}{I^2} \nonumber \\
    s(R) = \sqrt{\left(\frac{1}{I}\right)^2 s^2(V) + \left(-\frac{V}{I^2}\right)^2 s^2(I)} \label{ec:propincohm}
  \end{gather}
  Aplicamos la fórmula anterior para calcular las incertidumbres de \textit{R}. Redondeamos \textit{s(R)} para que tenga dos cifras significativas (\textit{Por ejemplo: De 2641 a $2,6 \cdot 10^3$}) y ajustamos \textit{R} para que su última cifra significativa coincida con la posición decimal de la última cifra significativa de \textit{s(R)}. (\textit{Por ejemplo: con s(R) = $2,6 \cdot 10^3$, R se redondearía a las centenas ($10^2$)}). Utilizamos las técnicas de redondeo del apartado \ref{sec:redondeo}.

  \begin{table}[H]
  \centering
  \csvreader[
    tabular=|c|c|c|c|,
    table head=\hline Medida & $V \pm s(V)~(V)$ & $I \pm s(I)~(I)$ & $R \pm s(R)~(\Omega)$ \\ \hline,
    late after last line=\\\hline,
    separator=semicolon
    ]{CC53.csv}
    {v=\v, sv=\sv, i=\int, si=\si, r=\r, sr=\sr}
    {\thecsvrow & \v \hspace{4pt}$\pm$ \sv & \int \hspace{4pt}$\pm$ \si & \r \hspace{4pt}$\pm$ \sr}
  \caption{Potenciales e intensidades de un circuito simple con la resistencia $R_1$}
  \label{tb:ohm}
  \end{table}

  Podemos observar que el valor de \textit{R} es prácticamente constante para cualquier combinación de voltajes e intensidades que escojamos. Además, se adecua mucho al valor experimental que obtuvimos en el apartado \ref{sec:medidaresist} ($1,755 \cdot 10^5 \pm 10^2 \Omega$), por lo que podemos concluir que se cumple la \textbf{Ley de Ohm}.

  \subsection{Representación gráfica de V frente a I}

  Hagamos ahora una representación gráfica de los valores del potencial (\textit{V}) frente a los de la intensidad (\textit{I}). Para hacer la gráfica utilizaremos el paquete \code{matplotlib} y para cargar los datos desde un .csv en el que tabulamos los datos (que también usamos para generar las tablas en \LaTeX) añadiremos \code{pandas} (una extensión de \code{numpy}).

  \begin{python}
    import matplotlib.pyplot as plt
    import pandas as pd

    d = pd.read_csv(name + ".csv", sep=';', decimal=',')
    x = d["i"] ; y = d["v"]
    n = d.shape[0]

    plt.scatter(x * (10**6), y)
    plt.xlabel('I($\mu$A)')
    plt.ylabel('V(V)', rotation=0, labelpad=20)
  \end{python}

  Vamos a crear una gráfica \textit{scatter} (de dispersión) para no unir los puntos de manera automática, ya que los valores medidos en el laboratorio son discretos. También cambiaremos la escala del eje x a $\mu$A para una leyenda más compacta. La exportamos a .pgf para poder cargarla en \LaTeX\  sin pérdida de calidad, y la representamos a continuación:

  \begin{figure}[H]
    %\centering
    \hspace{2.5em} %% Creator: Matplotlib, PGF backend
%%
%% To include the figure in your LaTeX document, write
%%   \input{<filename>.pgf}
%%
%% Make sure the required packages are loaded in your preamble
%%   \usepackage{pgf}
%%
%% Figures using additional raster images can only be included by \input if
%% they are in the same directory as the main LaTeX file. For loading figures
%% from other directories you can use the `import` package
%%   \usepackage{import}
%% and then include the figures with
%%   \import{<path to file>}{<filename>.pgf}
%%
%% Matplotlib used the following preamble
%%
\begingroup%
\makeatletter%
\begin{pgfpicture}%
\pgfpathrectangle{\pgfpointorigin}{\pgfqpoint{4.805892in}{3.310123in}}%
\pgfusepath{use as bounding box, clip}%
\begin{pgfscope}%
\pgfsetbuttcap%
\pgfsetmiterjoin%
\definecolor{currentfill}{rgb}{1.000000,1.000000,1.000000}%
\pgfsetfillcolor{currentfill}%
\pgfsetlinewidth{0.000000pt}%
\definecolor{currentstroke}{rgb}{1.000000,1.000000,1.000000}%
\pgfsetstrokecolor{currentstroke}%
\pgfsetdash{}{0pt}%
\pgfpathmoveto{\pgfqpoint{0.000000in}{0.000000in}}%
\pgfpathlineto{\pgfqpoint{4.805892in}{0.000000in}}%
\pgfpathlineto{\pgfqpoint{4.805892in}{3.310123in}}%
\pgfpathlineto{\pgfqpoint{0.000000in}{3.310123in}}%
\pgfpathclose%
\pgfusepath{fill}%
\end{pgfscope}%
\begin{pgfscope}%
\pgfsetbuttcap%
\pgfsetmiterjoin%
\definecolor{currentfill}{rgb}{1.000000,1.000000,1.000000}%
\pgfsetfillcolor{currentfill}%
\pgfsetlinewidth{0.000000pt}%
\definecolor{currentstroke}{rgb}{0.000000,0.000000,0.000000}%
\pgfsetstrokecolor{currentstroke}%
\pgfsetstrokeopacity{0.000000}%
\pgfsetdash{}{0pt}%
\pgfpathmoveto{\pgfqpoint{0.772069in}{0.515123in}}%
\pgfpathlineto{\pgfqpoint{4.647069in}{0.515123in}}%
\pgfpathlineto{\pgfqpoint{4.647069in}{3.210123in}}%
\pgfpathlineto{\pgfqpoint{0.772069in}{3.210123in}}%
\pgfpathclose%
\pgfusepath{fill}%
\end{pgfscope}%
\begin{pgfscope}%
\pgfsetbuttcap%
\pgfsetroundjoin%
\definecolor{currentfill}{rgb}{0.000000,0.000000,0.000000}%
\pgfsetfillcolor{currentfill}%
\pgfsetlinewidth{0.803000pt}%
\definecolor{currentstroke}{rgb}{0.000000,0.000000,0.000000}%
\pgfsetstrokecolor{currentstroke}%
\pgfsetdash{}{0pt}%
\pgfsys@defobject{currentmarker}{\pgfqpoint{0.000000in}{-0.048611in}}{\pgfqpoint{0.000000in}{0.000000in}}{%
\pgfpathmoveto{\pgfqpoint{0.000000in}{0.000000in}}%
\pgfpathlineto{\pgfqpoint{0.000000in}{-0.048611in}}%
\pgfusepath{stroke,fill}%
}%
\begin{pgfscope}%
\pgfsys@transformshift{1.177060in}{0.515123in}%
\pgfsys@useobject{currentmarker}{}%
\end{pgfscope}%
\end{pgfscope}%
\begin{pgfscope}%
\definecolor{textcolor}{rgb}{0.000000,0.000000,0.000000}%
\pgfsetstrokecolor{textcolor}%
\pgfsetfillcolor{textcolor}%
\pgftext[x=1.177060in,y=0.417901in,,top]{\color{textcolor}\rmfamily\fontsize{10.000000}{12.000000}\selectfont \(\displaystyle 10\)}%
\end{pgfscope}%
\begin{pgfscope}%
\pgfsetbuttcap%
\pgfsetroundjoin%
\definecolor{currentfill}{rgb}{0.000000,0.000000,0.000000}%
\pgfsetfillcolor{currentfill}%
\pgfsetlinewidth{0.803000pt}%
\definecolor{currentstroke}{rgb}{0.000000,0.000000,0.000000}%
\pgfsetstrokecolor{currentstroke}%
\pgfsetdash{}{0pt}%
\pgfsys@defobject{currentmarker}{\pgfqpoint{0.000000in}{-0.048611in}}{\pgfqpoint{0.000000in}{0.000000in}}{%
\pgfpathmoveto{\pgfqpoint{0.000000in}{0.000000in}}%
\pgfpathlineto{\pgfqpoint{0.000000in}{-0.048611in}}%
\pgfusepath{stroke,fill}%
}%
\begin{pgfscope}%
\pgfsys@transformshift{1.868938in}{0.515123in}%
\pgfsys@useobject{currentmarker}{}%
\end{pgfscope}%
\end{pgfscope}%
\begin{pgfscope}%
\definecolor{textcolor}{rgb}{0.000000,0.000000,0.000000}%
\pgfsetstrokecolor{textcolor}%
\pgfsetfillcolor{textcolor}%
\pgftext[x=1.868938in,y=0.417901in,,top]{\color{textcolor}\rmfamily\fontsize{10.000000}{12.000000}\selectfont \(\displaystyle 20\)}%
\end{pgfscope}%
\begin{pgfscope}%
\pgfsetbuttcap%
\pgfsetroundjoin%
\definecolor{currentfill}{rgb}{0.000000,0.000000,0.000000}%
\pgfsetfillcolor{currentfill}%
\pgfsetlinewidth{0.803000pt}%
\definecolor{currentstroke}{rgb}{0.000000,0.000000,0.000000}%
\pgfsetstrokecolor{currentstroke}%
\pgfsetdash{}{0pt}%
\pgfsys@defobject{currentmarker}{\pgfqpoint{0.000000in}{-0.048611in}}{\pgfqpoint{0.000000in}{0.000000in}}{%
\pgfpathmoveto{\pgfqpoint{0.000000in}{0.000000in}}%
\pgfpathlineto{\pgfqpoint{0.000000in}{-0.048611in}}%
\pgfusepath{stroke,fill}%
}%
\begin{pgfscope}%
\pgfsys@transformshift{2.560815in}{0.515123in}%
\pgfsys@useobject{currentmarker}{}%
\end{pgfscope}%
\end{pgfscope}%
\begin{pgfscope}%
\definecolor{textcolor}{rgb}{0.000000,0.000000,0.000000}%
\pgfsetstrokecolor{textcolor}%
\pgfsetfillcolor{textcolor}%
\pgftext[x=2.560815in,y=0.417901in,,top]{\color{textcolor}\rmfamily\fontsize{10.000000}{12.000000}\selectfont \(\displaystyle 30\)}%
\end{pgfscope}%
\begin{pgfscope}%
\pgfsetbuttcap%
\pgfsetroundjoin%
\definecolor{currentfill}{rgb}{0.000000,0.000000,0.000000}%
\pgfsetfillcolor{currentfill}%
\pgfsetlinewidth{0.803000pt}%
\definecolor{currentstroke}{rgb}{0.000000,0.000000,0.000000}%
\pgfsetstrokecolor{currentstroke}%
\pgfsetdash{}{0pt}%
\pgfsys@defobject{currentmarker}{\pgfqpoint{0.000000in}{-0.048611in}}{\pgfqpoint{0.000000in}{0.000000in}}{%
\pgfpathmoveto{\pgfqpoint{0.000000in}{0.000000in}}%
\pgfpathlineto{\pgfqpoint{0.000000in}{-0.048611in}}%
\pgfusepath{stroke,fill}%
}%
\begin{pgfscope}%
\pgfsys@transformshift{3.252692in}{0.515123in}%
\pgfsys@useobject{currentmarker}{}%
\end{pgfscope}%
\end{pgfscope}%
\begin{pgfscope}%
\definecolor{textcolor}{rgb}{0.000000,0.000000,0.000000}%
\pgfsetstrokecolor{textcolor}%
\pgfsetfillcolor{textcolor}%
\pgftext[x=3.252692in,y=0.417901in,,top]{\color{textcolor}\rmfamily\fontsize{10.000000}{12.000000}\selectfont \(\displaystyle 40\)}%
\end{pgfscope}%
\begin{pgfscope}%
\pgfsetbuttcap%
\pgfsetroundjoin%
\definecolor{currentfill}{rgb}{0.000000,0.000000,0.000000}%
\pgfsetfillcolor{currentfill}%
\pgfsetlinewidth{0.803000pt}%
\definecolor{currentstroke}{rgb}{0.000000,0.000000,0.000000}%
\pgfsetstrokecolor{currentstroke}%
\pgfsetdash{}{0pt}%
\pgfsys@defobject{currentmarker}{\pgfqpoint{0.000000in}{-0.048611in}}{\pgfqpoint{0.000000in}{0.000000in}}{%
\pgfpathmoveto{\pgfqpoint{0.000000in}{0.000000in}}%
\pgfpathlineto{\pgfqpoint{0.000000in}{-0.048611in}}%
\pgfusepath{stroke,fill}%
}%
\begin{pgfscope}%
\pgfsys@transformshift{3.944570in}{0.515123in}%
\pgfsys@useobject{currentmarker}{}%
\end{pgfscope}%
\end{pgfscope}%
\begin{pgfscope}%
\definecolor{textcolor}{rgb}{0.000000,0.000000,0.000000}%
\pgfsetstrokecolor{textcolor}%
\pgfsetfillcolor{textcolor}%
\pgftext[x=3.944570in,y=0.417901in,,top]{\color{textcolor}\rmfamily\fontsize{10.000000}{12.000000}\selectfont \(\displaystyle 50\)}%
\end{pgfscope}%
\begin{pgfscope}%
\pgfsetbuttcap%
\pgfsetroundjoin%
\definecolor{currentfill}{rgb}{0.000000,0.000000,0.000000}%
\pgfsetfillcolor{currentfill}%
\pgfsetlinewidth{0.803000pt}%
\definecolor{currentstroke}{rgb}{0.000000,0.000000,0.000000}%
\pgfsetstrokecolor{currentstroke}%
\pgfsetdash{}{0pt}%
\pgfsys@defobject{currentmarker}{\pgfqpoint{0.000000in}{-0.048611in}}{\pgfqpoint{0.000000in}{0.000000in}}{%
\pgfpathmoveto{\pgfqpoint{0.000000in}{0.000000in}}%
\pgfpathlineto{\pgfqpoint{0.000000in}{-0.048611in}}%
\pgfusepath{stroke,fill}%
}%
\begin{pgfscope}%
\pgfsys@transformshift{4.636447in}{0.515123in}%
\pgfsys@useobject{currentmarker}{}%
\end{pgfscope}%
\end{pgfscope}%
\begin{pgfscope}%
\definecolor{textcolor}{rgb}{0.000000,0.000000,0.000000}%
\pgfsetstrokecolor{textcolor}%
\pgfsetfillcolor{textcolor}%
\pgftext[x=4.636447in,y=0.417901in,,top]{\color{textcolor}\rmfamily\fontsize{10.000000}{12.000000}\selectfont \(\displaystyle 60\)}%
\end{pgfscope}%
\begin{pgfscope}%
\definecolor{textcolor}{rgb}{0.000000,0.000000,0.000000}%
\pgfsetstrokecolor{textcolor}%
\pgfsetfillcolor{textcolor}%
\pgftext[x=2.709569in,y=0.238889in,,top]{\color{textcolor}\rmfamily\fontsize{10.000000}{12.000000}\selectfont I(\(\displaystyle \mu\)A)}%
\end{pgfscope}%
\begin{pgfscope}%
\pgfsetbuttcap%
\pgfsetroundjoin%
\definecolor{currentfill}{rgb}{0.000000,0.000000,0.000000}%
\pgfsetfillcolor{currentfill}%
\pgfsetlinewidth{0.803000pt}%
\definecolor{currentstroke}{rgb}{0.000000,0.000000,0.000000}%
\pgfsetstrokecolor{currentstroke}%
\pgfsetdash{}{0pt}%
\pgfsys@defobject{currentmarker}{\pgfqpoint{-0.048611in}{0.000000in}}{\pgfqpoint{0.000000in}{0.000000in}}{%
\pgfpathmoveto{\pgfqpoint{0.000000in}{0.000000in}}%
\pgfpathlineto{\pgfqpoint{-0.048611in}{0.000000in}}%
\pgfusepath{stroke,fill}%
}%
\begin{pgfscope}%
\pgfsys@transformshift{0.772069in}{0.863332in}%
\pgfsys@useobject{currentmarker}{}%
\end{pgfscope}%
\end{pgfscope}%
\begin{pgfscope}%
\definecolor{textcolor}{rgb}{0.000000,0.000000,0.000000}%
\pgfsetstrokecolor{textcolor}%
\pgfsetfillcolor{textcolor}%
\pgftext[x=0.605402in,y=0.815107in,left,base]{\color{textcolor}\rmfamily\fontsize{10.000000}{12.000000}\selectfont \(\displaystyle 2\)}%
\end{pgfscope}%
\begin{pgfscope}%
\pgfsetbuttcap%
\pgfsetroundjoin%
\definecolor{currentfill}{rgb}{0.000000,0.000000,0.000000}%
\pgfsetfillcolor{currentfill}%
\pgfsetlinewidth{0.803000pt}%
\definecolor{currentstroke}{rgb}{0.000000,0.000000,0.000000}%
\pgfsetstrokecolor{currentstroke}%
\pgfsetdash{}{0pt}%
\pgfsys@defobject{currentmarker}{\pgfqpoint{-0.048611in}{0.000000in}}{\pgfqpoint{0.000000in}{0.000000in}}{%
\pgfpathmoveto{\pgfqpoint{0.000000in}{0.000000in}}%
\pgfpathlineto{\pgfqpoint{-0.048611in}{0.000000in}}%
\pgfusepath{stroke,fill}%
}%
\begin{pgfscope}%
\pgfsys@transformshift{0.772069in}{1.409095in}%
\pgfsys@useobject{currentmarker}{}%
\end{pgfscope}%
\end{pgfscope}%
\begin{pgfscope}%
\definecolor{textcolor}{rgb}{0.000000,0.000000,0.000000}%
\pgfsetstrokecolor{textcolor}%
\pgfsetfillcolor{textcolor}%
\pgftext[x=0.605402in,y=1.360869in,left,base]{\color{textcolor}\rmfamily\fontsize{10.000000}{12.000000}\selectfont \(\displaystyle 4\)}%
\end{pgfscope}%
\begin{pgfscope}%
\pgfsetbuttcap%
\pgfsetroundjoin%
\definecolor{currentfill}{rgb}{0.000000,0.000000,0.000000}%
\pgfsetfillcolor{currentfill}%
\pgfsetlinewidth{0.803000pt}%
\definecolor{currentstroke}{rgb}{0.000000,0.000000,0.000000}%
\pgfsetstrokecolor{currentstroke}%
\pgfsetdash{}{0pt}%
\pgfsys@defobject{currentmarker}{\pgfqpoint{-0.048611in}{0.000000in}}{\pgfqpoint{0.000000in}{0.000000in}}{%
\pgfpathmoveto{\pgfqpoint{0.000000in}{0.000000in}}%
\pgfpathlineto{\pgfqpoint{-0.048611in}{0.000000in}}%
\pgfusepath{stroke,fill}%
}%
\begin{pgfscope}%
\pgfsys@transformshift{0.772069in}{1.954857in}%
\pgfsys@useobject{currentmarker}{}%
\end{pgfscope}%
\end{pgfscope}%
\begin{pgfscope}%
\definecolor{textcolor}{rgb}{0.000000,0.000000,0.000000}%
\pgfsetstrokecolor{textcolor}%
\pgfsetfillcolor{textcolor}%
\pgftext[x=0.605402in,y=1.906632in,left,base]{\color{textcolor}\rmfamily\fontsize{10.000000}{12.000000}\selectfont \(\displaystyle 6\)}%
\end{pgfscope}%
\begin{pgfscope}%
\pgfsetbuttcap%
\pgfsetroundjoin%
\definecolor{currentfill}{rgb}{0.000000,0.000000,0.000000}%
\pgfsetfillcolor{currentfill}%
\pgfsetlinewidth{0.803000pt}%
\definecolor{currentstroke}{rgb}{0.000000,0.000000,0.000000}%
\pgfsetstrokecolor{currentstroke}%
\pgfsetdash{}{0pt}%
\pgfsys@defobject{currentmarker}{\pgfqpoint{-0.048611in}{0.000000in}}{\pgfqpoint{0.000000in}{0.000000in}}{%
\pgfpathmoveto{\pgfqpoint{0.000000in}{0.000000in}}%
\pgfpathlineto{\pgfqpoint{-0.048611in}{0.000000in}}%
\pgfusepath{stroke,fill}%
}%
\begin{pgfscope}%
\pgfsys@transformshift{0.772069in}{2.500620in}%
\pgfsys@useobject{currentmarker}{}%
\end{pgfscope}%
\end{pgfscope}%
\begin{pgfscope}%
\definecolor{textcolor}{rgb}{0.000000,0.000000,0.000000}%
\pgfsetstrokecolor{textcolor}%
\pgfsetfillcolor{textcolor}%
\pgftext[x=0.605402in,y=2.452394in,left,base]{\color{textcolor}\rmfamily\fontsize{10.000000}{12.000000}\selectfont \(\displaystyle 8\)}%
\end{pgfscope}%
\begin{pgfscope}%
\pgfsetbuttcap%
\pgfsetroundjoin%
\definecolor{currentfill}{rgb}{0.000000,0.000000,0.000000}%
\pgfsetfillcolor{currentfill}%
\pgfsetlinewidth{0.803000pt}%
\definecolor{currentstroke}{rgb}{0.000000,0.000000,0.000000}%
\pgfsetstrokecolor{currentstroke}%
\pgfsetdash{}{0pt}%
\pgfsys@defobject{currentmarker}{\pgfqpoint{-0.048611in}{0.000000in}}{\pgfqpoint{0.000000in}{0.000000in}}{%
\pgfpathmoveto{\pgfqpoint{0.000000in}{0.000000in}}%
\pgfpathlineto{\pgfqpoint{-0.048611in}{0.000000in}}%
\pgfusepath{stroke,fill}%
}%
\begin{pgfscope}%
\pgfsys@transformshift{0.772069in}{3.046382in}%
\pgfsys@useobject{currentmarker}{}%
\end{pgfscope}%
\end{pgfscope}%
\begin{pgfscope}%
\definecolor{textcolor}{rgb}{0.000000,0.000000,0.000000}%
\pgfsetstrokecolor{textcolor}%
\pgfsetfillcolor{textcolor}%
\pgftext[x=0.535957in,y=2.998157in,left,base]{\color{textcolor}\rmfamily\fontsize{10.000000}{12.000000}\selectfont \(\displaystyle 10\)}%
\end{pgfscope}%
\begin{pgfscope}%
\definecolor{textcolor}{rgb}{0.000000,0.000000,0.000000}%
\pgfsetstrokecolor{textcolor}%
\pgfsetfillcolor{textcolor}%
\pgftext[x=0.258179in,y=1.862623in,,bottom]{\color{textcolor}\rmfamily\fontsize{10.000000}{12.000000}\selectfont V(V)}%
\end{pgfscope}%
\begin{pgfscope}%
\pgfpathrectangle{\pgfqpoint{0.772069in}{0.515123in}}{\pgfqpoint{3.875000in}{2.695000in}}%
\pgfusepath{clip}%
\pgfsetbuttcap%
\pgfsetroundjoin%
\definecolor{currentfill}{rgb}{0.121569,0.466667,0.705882}%
\pgfsetfillcolor{currentfill}%
\pgfsetlinewidth{1.003750pt}%
\definecolor{currentstroke}{rgb}{0.121569,0.466667,0.705882}%
\pgfsetstrokecolor{currentstroke}%
\pgfsetdash{}{0pt}%
\pgfpathmoveto{\pgfqpoint{0.948741in}{0.598994in}}%
\pgfpathcurveto{\pgfqpoint{0.959791in}{0.598994in}}{\pgfqpoint{0.970390in}{0.603385in}}{\pgfqpoint{0.978203in}{0.611198in}}%
\pgfpathcurveto{\pgfqpoint{0.986017in}{0.619012in}}{\pgfqpoint{0.990407in}{0.629611in}}{\pgfqpoint{0.990407in}{0.640661in}}%
\pgfpathcurveto{\pgfqpoint{0.990407in}{0.651711in}}{\pgfqpoint{0.986017in}{0.662310in}}{\pgfqpoint{0.978203in}{0.670124in}}%
\pgfpathcurveto{\pgfqpoint{0.970390in}{0.677937in}}{\pgfqpoint{0.959791in}{0.682328in}}{\pgfqpoint{0.948741in}{0.682328in}}%
\pgfpathcurveto{\pgfqpoint{0.937691in}{0.682328in}}{\pgfqpoint{0.927092in}{0.677937in}}{\pgfqpoint{0.919278in}{0.670124in}}%
\pgfpathcurveto{\pgfqpoint{0.911464in}{0.662310in}}{\pgfqpoint{0.907074in}{0.651711in}}{\pgfqpoint{0.907074in}{0.640661in}}%
\pgfpathcurveto{\pgfqpoint{0.907074in}{0.629611in}}{\pgfqpoint{0.911464in}{0.619012in}}{\pgfqpoint{0.919278in}{0.611198in}}%
\pgfpathcurveto{\pgfqpoint{0.927092in}{0.603385in}}{\pgfqpoint{0.937691in}{0.598994in}}{\pgfqpoint{0.948741in}{0.598994in}}%
\pgfpathclose%
\pgfusepath{stroke,fill}%
\end{pgfscope}%
\begin{pgfscope}%
\pgfpathrectangle{\pgfqpoint{0.772069in}{0.515123in}}{\pgfqpoint{3.875000in}{2.695000in}}%
\pgfusepath{clip}%
\pgfsetbuttcap%
\pgfsetroundjoin%
\definecolor{currentfill}{rgb}{0.121569,0.466667,0.705882}%
\pgfsetfillcolor{currentfill}%
\pgfsetlinewidth{1.003750pt}%
\definecolor{currentstroke}{rgb}{0.121569,0.466667,0.705882}%
\pgfsetstrokecolor{currentstroke}%
\pgfsetdash{}{0pt}%
\pgfpathmoveto{\pgfqpoint{1.315436in}{0.851682in}}%
\pgfpathcurveto{\pgfqpoint{1.326486in}{0.851682in}}{\pgfqpoint{1.337085in}{0.856073in}}{\pgfqpoint{1.344898in}{0.863886in}}%
\pgfpathcurveto{\pgfqpoint{1.352712in}{0.871700in}}{\pgfqpoint{1.357102in}{0.882299in}}{\pgfqpoint{1.357102in}{0.893349in}}%
\pgfpathcurveto{\pgfqpoint{1.357102in}{0.904399in}}{\pgfqpoint{1.352712in}{0.914998in}}{\pgfqpoint{1.344898in}{0.922812in}}%
\pgfpathcurveto{\pgfqpoint{1.337085in}{0.930625in}}{\pgfqpoint{1.326486in}{0.935016in}}{\pgfqpoint{1.315436in}{0.935016in}}%
\pgfpathcurveto{\pgfqpoint{1.304386in}{0.935016in}}{\pgfqpoint{1.293787in}{0.930625in}}{\pgfqpoint{1.285973in}{0.922812in}}%
\pgfpathcurveto{\pgfqpoint{1.278159in}{0.914998in}}{\pgfqpoint{1.273769in}{0.904399in}}{\pgfqpoint{1.273769in}{0.893349in}}%
\pgfpathcurveto{\pgfqpoint{1.273769in}{0.882299in}}{\pgfqpoint{1.278159in}{0.871700in}}{\pgfqpoint{1.285973in}{0.863886in}}%
\pgfpathcurveto{\pgfqpoint{1.293787in}{0.856073in}}{\pgfqpoint{1.304386in}{0.851682in}}{\pgfqpoint{1.315436in}{0.851682in}}%
\pgfpathclose%
\pgfusepath{stroke,fill}%
\end{pgfscope}%
\begin{pgfscope}%
\pgfpathrectangle{\pgfqpoint{0.772069in}{0.515123in}}{\pgfqpoint{3.875000in}{2.695000in}}%
\pgfusepath{clip}%
\pgfsetbuttcap%
\pgfsetroundjoin%
\definecolor{currentfill}{rgb}{0.121569,0.466667,0.705882}%
\pgfsetfillcolor{currentfill}%
\pgfsetlinewidth{1.003750pt}%
\definecolor{currentstroke}{rgb}{0.121569,0.466667,0.705882}%
\pgfsetstrokecolor{currentstroke}%
\pgfsetdash{}{0pt}%
\pgfpathmoveto{\pgfqpoint{1.675212in}{1.105462in}}%
\pgfpathcurveto{\pgfqpoint{1.686262in}{1.105462in}}{\pgfqpoint{1.696861in}{1.109852in}}{\pgfqpoint{1.704675in}{1.117666in}}%
\pgfpathcurveto{\pgfqpoint{1.712488in}{1.125479in}}{\pgfqpoint{1.716879in}{1.136078in}}{\pgfqpoint{1.716879in}{1.147129in}}%
\pgfpathcurveto{\pgfqpoint{1.716879in}{1.158179in}}{\pgfqpoint{1.712488in}{1.168778in}}{\pgfqpoint{1.704675in}{1.176591in}}%
\pgfpathcurveto{\pgfqpoint{1.696861in}{1.184405in}}{\pgfqpoint{1.686262in}{1.188795in}}{\pgfqpoint{1.675212in}{1.188795in}}%
\pgfpathcurveto{\pgfqpoint{1.664162in}{1.188795in}}{\pgfqpoint{1.653563in}{1.184405in}}{\pgfqpoint{1.645749in}{1.176591in}}%
\pgfpathcurveto{\pgfqpoint{1.637936in}{1.168778in}}{\pgfqpoint{1.633545in}{1.158179in}}{\pgfqpoint{1.633545in}{1.147129in}}%
\pgfpathcurveto{\pgfqpoint{1.633545in}{1.136078in}}{\pgfqpoint{1.637936in}{1.125479in}}{\pgfqpoint{1.645749in}{1.117666in}}%
\pgfpathcurveto{\pgfqpoint{1.653563in}{1.109852in}}{\pgfqpoint{1.664162in}{1.105462in}}{\pgfqpoint{1.675212in}{1.105462in}}%
\pgfpathclose%
\pgfusepath{stroke,fill}%
\end{pgfscope}%
\begin{pgfscope}%
\pgfpathrectangle{\pgfqpoint{0.772069in}{0.515123in}}{\pgfqpoint{3.875000in}{2.695000in}}%
\pgfusepath{clip}%
\pgfsetbuttcap%
\pgfsetroundjoin%
\definecolor{currentfill}{rgb}{0.121569,0.466667,0.705882}%
\pgfsetfillcolor{currentfill}%
\pgfsetlinewidth{1.003750pt}%
\definecolor{currentstroke}{rgb}{0.121569,0.466667,0.705882}%
\pgfsetstrokecolor{currentstroke}%
\pgfsetdash{}{0pt}%
\pgfpathmoveto{\pgfqpoint{2.076501in}{1.383801in}}%
\pgfpathcurveto{\pgfqpoint{2.087551in}{1.383801in}}{\pgfqpoint{2.098150in}{1.388191in}}{\pgfqpoint{2.105964in}{1.396005in}}%
\pgfpathcurveto{\pgfqpoint{2.113777in}{1.403818in}}{\pgfqpoint{2.118168in}{1.414417in}}{\pgfqpoint{2.118168in}{1.425468in}}%
\pgfpathcurveto{\pgfqpoint{2.118168in}{1.436518in}}{\pgfqpoint{2.113777in}{1.447117in}}{\pgfqpoint{2.105964in}{1.454930in}}%
\pgfpathcurveto{\pgfqpoint{2.098150in}{1.462744in}}{\pgfqpoint{2.087551in}{1.467134in}}{\pgfqpoint{2.076501in}{1.467134in}}%
\pgfpathcurveto{\pgfqpoint{2.065451in}{1.467134in}}{\pgfqpoint{2.054852in}{1.462744in}}{\pgfqpoint{2.047038in}{1.454930in}}%
\pgfpathcurveto{\pgfqpoint{2.039224in}{1.447117in}}{\pgfqpoint{2.034834in}{1.436518in}}{\pgfqpoint{2.034834in}{1.425468in}}%
\pgfpathcurveto{\pgfqpoint{2.034834in}{1.414417in}}{\pgfqpoint{2.039224in}{1.403818in}}{\pgfqpoint{2.047038in}{1.396005in}}%
\pgfpathcurveto{\pgfqpoint{2.054852in}{1.388191in}}{\pgfqpoint{2.065451in}{1.383801in}}{\pgfqpoint{2.076501in}{1.383801in}}%
\pgfpathclose%
\pgfusepath{stroke,fill}%
\end{pgfscope}%
\begin{pgfscope}%
\pgfpathrectangle{\pgfqpoint{0.772069in}{0.515123in}}{\pgfqpoint{3.875000in}{2.695000in}}%
\pgfusepath{clip}%
\pgfsetbuttcap%
\pgfsetroundjoin%
\definecolor{currentfill}{rgb}{0.121569,0.466667,0.705882}%
\pgfsetfillcolor{currentfill}%
\pgfsetlinewidth{1.003750pt}%
\definecolor{currentstroke}{rgb}{0.121569,0.466667,0.705882}%
\pgfsetstrokecolor{currentstroke}%
\pgfsetdash{}{0pt}%
\pgfpathmoveto{\pgfqpoint{2.463952in}{1.651225in}}%
\pgfpathcurveto{\pgfqpoint{2.475002in}{1.651225in}}{\pgfqpoint{2.485601in}{1.655615in}}{\pgfqpoint{2.493415in}{1.663428in}}%
\pgfpathcurveto{\pgfqpoint{2.501229in}{1.671242in}}{\pgfqpoint{2.505619in}{1.681841in}}{\pgfqpoint{2.505619in}{1.692891in}}%
\pgfpathcurveto{\pgfqpoint{2.505619in}{1.703941in}}{\pgfqpoint{2.501229in}{1.714540in}}{\pgfqpoint{2.493415in}{1.722354in}}%
\pgfpathcurveto{\pgfqpoint{2.485601in}{1.730168in}}{\pgfqpoint{2.475002in}{1.734558in}}{\pgfqpoint{2.463952in}{1.734558in}}%
\pgfpathcurveto{\pgfqpoint{2.452902in}{1.734558in}}{\pgfqpoint{2.442303in}{1.730168in}}{\pgfqpoint{2.434489in}{1.722354in}}%
\pgfpathcurveto{\pgfqpoint{2.426676in}{1.714540in}}{\pgfqpoint{2.422286in}{1.703941in}}{\pgfqpoint{2.422286in}{1.692891in}}%
\pgfpathcurveto{\pgfqpoint{2.422286in}{1.681841in}}{\pgfqpoint{2.426676in}{1.671242in}}{\pgfqpoint{2.434489in}{1.663428in}}%
\pgfpathcurveto{\pgfqpoint{2.442303in}{1.655615in}}{\pgfqpoint{2.452902in}{1.651225in}}{\pgfqpoint{2.463952in}{1.651225in}}%
\pgfpathclose%
\pgfusepath{stroke,fill}%
\end{pgfscope}%
\begin{pgfscope}%
\pgfpathrectangle{\pgfqpoint{0.772069in}{0.515123in}}{\pgfqpoint{3.875000in}{2.695000in}}%
\pgfusepath{clip}%
\pgfsetbuttcap%
\pgfsetroundjoin%
\definecolor{currentfill}{rgb}{0.121569,0.466667,0.705882}%
\pgfsetfillcolor{currentfill}%
\pgfsetlinewidth{1.003750pt}%
\definecolor{currentstroke}{rgb}{0.121569,0.466667,0.705882}%
\pgfsetstrokecolor{currentstroke}%
\pgfsetdash{}{0pt}%
\pgfpathmoveto{\pgfqpoint{2.872160in}{1.935021in}}%
\pgfpathcurveto{\pgfqpoint{2.883210in}{1.935021in}}{\pgfqpoint{2.893809in}{1.939411in}}{\pgfqpoint{2.901623in}{1.947225in}}%
\pgfpathcurveto{\pgfqpoint{2.909436in}{1.955039in}}{\pgfqpoint{2.913827in}{1.965638in}}{\pgfqpoint{2.913827in}{1.976688in}}%
\pgfpathcurveto{\pgfqpoint{2.913827in}{1.987738in}}{\pgfqpoint{2.909436in}{1.998337in}}{\pgfqpoint{2.901623in}{2.006150in}}%
\pgfpathcurveto{\pgfqpoint{2.893809in}{2.013964in}}{\pgfqpoint{2.883210in}{2.018354in}}{\pgfqpoint{2.872160in}{2.018354in}}%
\pgfpathcurveto{\pgfqpoint{2.861110in}{2.018354in}}{\pgfqpoint{2.850511in}{2.013964in}}{\pgfqpoint{2.842697in}{2.006150in}}%
\pgfpathcurveto{\pgfqpoint{2.834883in}{1.998337in}}{\pgfqpoint{2.830493in}{1.987738in}}{\pgfqpoint{2.830493in}{1.976688in}}%
\pgfpathcurveto{\pgfqpoint{2.830493in}{1.965638in}}{\pgfqpoint{2.834883in}{1.955039in}}{\pgfqpoint{2.842697in}{1.947225in}}%
\pgfpathcurveto{\pgfqpoint{2.850511in}{1.939411in}}{\pgfqpoint{2.861110in}{1.935021in}}{\pgfqpoint{2.872160in}{1.935021in}}%
\pgfpathclose%
\pgfusepath{stroke,fill}%
\end{pgfscope}%
\begin{pgfscope}%
\pgfpathrectangle{\pgfqpoint{0.772069in}{0.515123in}}{\pgfqpoint{3.875000in}{2.695000in}}%
\pgfusepath{clip}%
\pgfsetbuttcap%
\pgfsetroundjoin%
\definecolor{currentfill}{rgb}{0.121569,0.466667,0.705882}%
\pgfsetfillcolor{currentfill}%
\pgfsetlinewidth{1.003750pt}%
\definecolor{currentstroke}{rgb}{0.121569,0.466667,0.705882}%
\pgfsetstrokecolor{currentstroke}%
\pgfsetdash{}{0pt}%
\pgfpathmoveto{\pgfqpoint{3.273449in}{2.213360in}}%
\pgfpathcurveto{\pgfqpoint{3.284499in}{2.213360in}}{\pgfqpoint{3.295098in}{2.217750in}}{\pgfqpoint{3.302912in}{2.225564in}}%
\pgfpathcurveto{\pgfqpoint{3.310725in}{2.233377in}}{\pgfqpoint{3.315115in}{2.243976in}}{\pgfqpoint{3.315115in}{2.255027in}}%
\pgfpathcurveto{\pgfqpoint{3.315115in}{2.266077in}}{\pgfqpoint{3.310725in}{2.276676in}}{\pgfqpoint{3.302912in}{2.284489in}}%
\pgfpathcurveto{\pgfqpoint{3.295098in}{2.292303in}}{\pgfqpoint{3.284499in}{2.296693in}}{\pgfqpoint{3.273449in}{2.296693in}}%
\pgfpathcurveto{\pgfqpoint{3.262399in}{2.296693in}}{\pgfqpoint{3.251800in}{2.292303in}}{\pgfqpoint{3.243986in}{2.284489in}}%
\pgfpathcurveto{\pgfqpoint{3.236172in}{2.276676in}}{\pgfqpoint{3.231782in}{2.266077in}}{\pgfqpoint{3.231782in}{2.255027in}}%
\pgfpathcurveto{\pgfqpoint{3.231782in}{2.243976in}}{\pgfqpoint{3.236172in}{2.233377in}}{\pgfqpoint{3.243986in}{2.225564in}}%
\pgfpathcurveto{\pgfqpoint{3.251800in}{2.217750in}}{\pgfqpoint{3.262399in}{2.213360in}}{\pgfqpoint{3.273449in}{2.213360in}}%
\pgfpathclose%
\pgfusepath{stroke,fill}%
\end{pgfscope}%
\begin{pgfscope}%
\pgfpathrectangle{\pgfqpoint{0.772069in}{0.515123in}}{\pgfqpoint{3.875000in}{2.695000in}}%
\pgfusepath{clip}%
\pgfsetbuttcap%
\pgfsetroundjoin%
\definecolor{currentfill}{rgb}{0.121569,0.466667,0.705882}%
\pgfsetfillcolor{currentfill}%
\pgfsetlinewidth{1.003750pt}%
\definecolor{currentstroke}{rgb}{0.121569,0.466667,0.705882}%
\pgfsetstrokecolor{currentstroke}%
\pgfsetdash{}{0pt}%
\pgfpathmoveto{\pgfqpoint{3.667819in}{2.486241in}}%
\pgfpathcurveto{\pgfqpoint{3.678869in}{2.486241in}}{\pgfqpoint{3.689468in}{2.490631in}}{\pgfqpoint{3.697282in}{2.498445in}}%
\pgfpathcurveto{\pgfqpoint{3.705095in}{2.506259in}}{\pgfqpoint{3.709486in}{2.516858in}}{\pgfqpoint{3.709486in}{2.527908in}}%
\pgfpathcurveto{\pgfqpoint{3.709486in}{2.538958in}}{\pgfqpoint{3.705095in}{2.549557in}}{\pgfqpoint{3.697282in}{2.557371in}}%
\pgfpathcurveto{\pgfqpoint{3.689468in}{2.565184in}}{\pgfqpoint{3.678869in}{2.569575in}}{\pgfqpoint{3.667819in}{2.569575in}}%
\pgfpathcurveto{\pgfqpoint{3.656769in}{2.569575in}}{\pgfqpoint{3.646170in}{2.565184in}}{\pgfqpoint{3.638356in}{2.557371in}}%
\pgfpathcurveto{\pgfqpoint{3.630542in}{2.549557in}}{\pgfqpoint{3.626152in}{2.538958in}}{\pgfqpoint{3.626152in}{2.527908in}}%
\pgfpathcurveto{\pgfqpoint{3.626152in}{2.516858in}}{\pgfqpoint{3.630542in}{2.506259in}}{\pgfqpoint{3.638356in}{2.498445in}}%
\pgfpathcurveto{\pgfqpoint{3.646170in}{2.490631in}}{\pgfqpoint{3.656769in}{2.486241in}}{\pgfqpoint{3.667819in}{2.486241in}}%
\pgfpathclose%
\pgfusepath{stroke,fill}%
\end{pgfscope}%
\begin{pgfscope}%
\pgfpathrectangle{\pgfqpoint{0.772069in}{0.515123in}}{\pgfqpoint{3.875000in}{2.695000in}}%
\pgfusepath{clip}%
\pgfsetbuttcap%
\pgfsetroundjoin%
\definecolor{currentfill}{rgb}{0.121569,0.466667,0.705882}%
\pgfsetfillcolor{currentfill}%
\pgfsetlinewidth{1.003750pt}%
\definecolor{currentstroke}{rgb}{0.121569,0.466667,0.705882}%
\pgfsetstrokecolor{currentstroke}%
\pgfsetdash{}{0pt}%
\pgfpathmoveto{\pgfqpoint{4.089864in}{2.775495in}}%
\pgfpathcurveto{\pgfqpoint{4.100914in}{2.775495in}}{\pgfqpoint{4.111513in}{2.779886in}}{\pgfqpoint{4.119327in}{2.787699in}}%
\pgfpathcurveto{\pgfqpoint{4.127141in}{2.795513in}}{\pgfqpoint{4.131531in}{2.806112in}}{\pgfqpoint{4.131531in}{2.817162in}}%
\pgfpathcurveto{\pgfqpoint{4.131531in}{2.828212in}}{\pgfqpoint{4.127141in}{2.838811in}}{\pgfqpoint{4.119327in}{2.846625in}}%
\pgfpathcurveto{\pgfqpoint{4.111513in}{2.854438in}}{\pgfqpoint{4.100914in}{2.858829in}}{\pgfqpoint{4.089864in}{2.858829in}}%
\pgfpathcurveto{\pgfqpoint{4.078814in}{2.858829in}}{\pgfqpoint{4.068215in}{2.854438in}}{\pgfqpoint{4.060401in}{2.846625in}}%
\pgfpathcurveto{\pgfqpoint{4.052588in}{2.838811in}}{\pgfqpoint{4.048197in}{2.828212in}}{\pgfqpoint{4.048197in}{2.817162in}}%
\pgfpathcurveto{\pgfqpoint{4.048197in}{2.806112in}}{\pgfqpoint{4.052588in}{2.795513in}}{\pgfqpoint{4.060401in}{2.787699in}}%
\pgfpathcurveto{\pgfqpoint{4.068215in}{2.779886in}}{\pgfqpoint{4.078814in}{2.775495in}}{\pgfqpoint{4.089864in}{2.775495in}}%
\pgfpathclose%
\pgfusepath{stroke,fill}%
\end{pgfscope}%
\begin{pgfscope}%
\pgfpathrectangle{\pgfqpoint{0.772069in}{0.515123in}}{\pgfqpoint{3.875000in}{2.695000in}}%
\pgfusepath{clip}%
\pgfsetbuttcap%
\pgfsetroundjoin%
\definecolor{currentfill}{rgb}{0.121569,0.466667,0.705882}%
\pgfsetfillcolor{currentfill}%
\pgfsetlinewidth{1.003750pt}%
\definecolor{currentstroke}{rgb}{0.121569,0.466667,0.705882}%
\pgfsetstrokecolor{currentstroke}%
\pgfsetdash{}{0pt}%
\pgfpathmoveto{\pgfqpoint{4.470397in}{3.042919in}}%
\pgfpathcurveto{\pgfqpoint{4.481447in}{3.042919in}}{\pgfqpoint{4.492046in}{3.047309in}}{\pgfqpoint{4.499859in}{3.055123in}}%
\pgfpathcurveto{\pgfqpoint{4.507673in}{3.062937in}}{\pgfqpoint{4.512063in}{3.073536in}}{\pgfqpoint{4.512063in}{3.084586in}}%
\pgfpathcurveto{\pgfqpoint{4.512063in}{3.095636in}}{\pgfqpoint{4.507673in}{3.106235in}}{\pgfqpoint{4.499859in}{3.114048in}}%
\pgfpathcurveto{\pgfqpoint{4.492046in}{3.121862in}}{\pgfqpoint{4.481447in}{3.126252in}}{\pgfqpoint{4.470397in}{3.126252in}}%
\pgfpathcurveto{\pgfqpoint{4.459347in}{3.126252in}}{\pgfqpoint{4.448748in}{3.121862in}}{\pgfqpoint{4.440934in}{3.114048in}}%
\pgfpathcurveto{\pgfqpoint{4.433120in}{3.106235in}}{\pgfqpoint{4.428730in}{3.095636in}}{\pgfqpoint{4.428730in}{3.084586in}}%
\pgfpathcurveto{\pgfqpoint{4.428730in}{3.073536in}}{\pgfqpoint{4.433120in}{3.062937in}}{\pgfqpoint{4.440934in}{3.055123in}}%
\pgfpathcurveto{\pgfqpoint{4.448748in}{3.047309in}}{\pgfqpoint{4.459347in}{3.042919in}}{\pgfqpoint{4.470397in}{3.042919in}}%
\pgfpathclose%
\pgfusepath{stroke,fill}%
\end{pgfscope}%
\begin{pgfscope}%
\pgfsetrectcap%
\pgfsetmiterjoin%
\pgfsetlinewidth{0.803000pt}%
\definecolor{currentstroke}{rgb}{0.000000,0.000000,0.000000}%
\pgfsetstrokecolor{currentstroke}%
\pgfsetdash{}{0pt}%
\pgfpathmoveto{\pgfqpoint{0.772069in}{0.515123in}}%
\pgfpathlineto{\pgfqpoint{0.772069in}{3.210123in}}%
\pgfusepath{stroke}%
\end{pgfscope}%
\begin{pgfscope}%
\pgfsetrectcap%
\pgfsetmiterjoin%
\pgfsetlinewidth{0.803000pt}%
\definecolor{currentstroke}{rgb}{0.000000,0.000000,0.000000}%
\pgfsetstrokecolor{currentstroke}%
\pgfsetdash{}{0pt}%
\pgfpathmoveto{\pgfqpoint{4.647069in}{0.515123in}}%
\pgfpathlineto{\pgfqpoint{4.647069in}{3.210123in}}%
\pgfusepath{stroke}%
\end{pgfscope}%
\begin{pgfscope}%
\pgfsetrectcap%
\pgfsetmiterjoin%
\pgfsetlinewidth{0.803000pt}%
\definecolor{currentstroke}{rgb}{0.000000,0.000000,0.000000}%
\pgfsetstrokecolor{currentstroke}%
\pgfsetdash{}{0pt}%
\pgfpathmoveto{\pgfqpoint{0.772069in}{0.515123in}}%
\pgfpathlineto{\pgfqpoint{4.647069in}{0.515123in}}%
\pgfusepath{stroke}%
\end{pgfscope}%
\begin{pgfscope}%
\pgfsetrectcap%
\pgfsetmiterjoin%
\pgfsetlinewidth{0.803000pt}%
\definecolor{currentstroke}{rgb}{0.000000,0.000000,0.000000}%
\pgfsetstrokecolor{currentstroke}%
\pgfsetdash{}{0pt}%
\pgfpathmoveto{\pgfqpoint{0.772069in}{3.210123in}}%
\pgfpathlineto{\pgfqpoint{4.647069in}{3.210123in}}%
\pgfusepath{stroke}%
\end{pgfscope}%
\end{pgfpicture}%
\makeatother%
\endgroup%

    \caption{Voltaje (V) frente a intensidad (I)}
  \end{figure}

  \subsection{Ajuste por mínimos cuadrados}
  \label{sec:ajusteminres}

  Como podemos observar, parece que existe una relación lineal entre las dos magnitudes. Procedemos entonces a hacer un ajuste de regresión lineal por mínimos cuadrados. Como explicamos en la sección \ref{sec:reglin}, tenemos que calcular los coeficientes \textit{a} y \textit{b} de la recta $a + bx$. Para ello utilizamos las ecuaciones \ref{ec:a} y \ref{ec:b}. Como vemos, hay que calcular los siguientes términos:
  \begin{gather}
    \sum_i x_i \nonumber \qquad \sum_i y_i \nonumber \qquad \sum_i x_iy_i \nonumber \qquad \sum_i x_i^2 \nonumber
  \end{gather}

  Para ello, utilizaremos la tabla en .csv en la que tenemos los datos, y la cargaremos en python. Ahora definiremos una función \code{reg\_lin()} que calcule estas sumas a partir de las columnas de nuestra tabla y las almacene en variables separadas. Posteriormente, aplicaremos las fórmulas \ref{ec:a} y \ref{ec:b} y calcularemos así los términos \textit{a} y \textit{b}. Tenemos que decir que en estos casos, el término \textit{a} tiene que aproximarse mucho a 0 para ser despreciables, ya que sabemos que la recta tiene que pasar por el origen. Calcularemos \textit{a} para comprobar que es pequeño y que no influye en el ajuste. Al final será cómo sumar "0" en la fórmula $y = a + bx$.

  \begin{python}
    def reg_lin(x, y, n):
        sx = x.sum()
        sy = y.sum()
        sxy = (x*y).sum()
        sx2 = (x**2).sum()

        a = (sy*sx2 - sx*sxy) / (n*sx2 - sx**2)
        b = (n*sxy - sx*sy) / (n*sx2 - sx**2)

        sy2 = (y**2).sum()
        r = (n*sxy - sx*sy)/(((n*sx2 - sx**2)*(n*sy2 - sy**2))**0.5)

        return a, b, r

    a, b, r = reg_lin(x, y, n)
  \end{python}

  El comando \code{.sum()} de la tabla de cargamos en \code{pandas} simplemente suma todos los términos de la columna que especificamos. En este caso, tomamos la columna x como la intensidad (\textit{I}) y la columna y como el potencial (\textit{V}) de la tabla \ref{tb:ohm}, que cargamos directamente en \code{python}. También podemos realizar operaciones con las columnas antes de sumar sus términos, como multiplicar una por otra o elevar sus términos al cuadrado. Así podemos crear los sumatorios necesarios para resolver las fórmulas de a y b. Obtenemos los siguientes resultados:
  \begin{gather}
    \sum_i x_i = 3,18\cdot10^{-4} \nonumber \qquad \sum_i y_i = 56,0 \nonumber \qquad \sum_i x_iy_i = 2,25\cdot10^{-3} \nonumber \qquad \sum_i x_i^2 = 1,28\cdot10^{-8} \nonumber \\
    a = 9,19\cdot10^{-3} \nonumber \qquad b = 1,75 \cdot 10^5 \label{v:ohm}
  \end{gather}
  Podemos utilizar ahora estos valores para crear la gráfica con la recta que mejor se ajuste a los datos, creando una función \code{plot()} que permita crear gráficas con y sin regresión lineal:

  \begin{python}
    def plot(x, y, n, reg, cc, cr):
      if reg:
          a, b, r = reg_lin(x, y, n)
          xr = np.linspace(min(x) * (10**6), max(x) * (10**6), 10)
          yr = a + ((b*xr) / (10**6))
          plt.plot(xr, yr, color=cr, zorder=1)
      plt.scatter(x * (10**6), y, color=cc, zorder=2)

    plot(x, y, n, True, "color1", "color2")
  \end{python}

  Y, ahora sí, podemos ver nuestra gráfica completa.

  \begin{figure}[H]
    %\centering
    \hspace{2.5em} %% Creator: Matplotlib, PGF backend
%%
%% To include the figure in your LaTeX document, write
%%   \input{<filename>.pgf}
%%
%% Make sure the required packages are loaded in your preamble
%%   \usepackage{pgf}
%%
%% Figures using additional raster images can only be included by \input if
%% they are in the same directory as the main LaTeX file. For loading figures
%% from other directories you can use the `import` package
%%   \usepackage{import}
%% and then include the figures with
%%   \import{<path to file>}{<filename>.pgf}
%%
%% Matplotlib used the following preamble
%%
\begingroup%
\makeatletter%
\begin{pgfpicture}%
\pgfpathrectangle{\pgfpointorigin}{\pgfqpoint{4.747069in}{3.310123in}}%
\pgfusepath{use as bounding box, clip}%
\begin{pgfscope}%
\pgfsetbuttcap%
\pgfsetmiterjoin%
\definecolor{currentfill}{rgb}{1.000000,1.000000,1.000000}%
\pgfsetfillcolor{currentfill}%
\pgfsetlinewidth{0.000000pt}%
\definecolor{currentstroke}{rgb}{1.000000,1.000000,1.000000}%
\pgfsetstrokecolor{currentstroke}%
\pgfsetdash{}{0pt}%
\pgfpathmoveto{\pgfqpoint{0.000000in}{0.000000in}}%
\pgfpathlineto{\pgfqpoint{4.747069in}{0.000000in}}%
\pgfpathlineto{\pgfqpoint{4.747069in}{3.310123in}}%
\pgfpathlineto{\pgfqpoint{0.000000in}{3.310123in}}%
\pgfpathclose%
\pgfusepath{fill}%
\end{pgfscope}%
\begin{pgfscope}%
\pgfsetbuttcap%
\pgfsetmiterjoin%
\definecolor{currentfill}{rgb}{1.000000,1.000000,1.000000}%
\pgfsetfillcolor{currentfill}%
\pgfsetlinewidth{0.000000pt}%
\definecolor{currentstroke}{rgb}{0.000000,0.000000,0.000000}%
\pgfsetstrokecolor{currentstroke}%
\pgfsetstrokeopacity{0.000000}%
\pgfsetdash{}{0pt}%
\pgfpathmoveto{\pgfqpoint{0.772069in}{0.515123in}}%
\pgfpathlineto{\pgfqpoint{4.647069in}{0.515123in}}%
\pgfpathlineto{\pgfqpoint{4.647069in}{3.210123in}}%
\pgfpathlineto{\pgfqpoint{0.772069in}{3.210123in}}%
\pgfpathclose%
\pgfusepath{fill}%
\end{pgfscope}%
\begin{pgfscope}%
\pgfpathrectangle{\pgfqpoint{0.772069in}{0.515123in}}{\pgfqpoint{3.875000in}{2.695000in}}%
\pgfusepath{clip}%
\pgfsetrectcap%
\pgfsetroundjoin%
\pgfsetlinewidth{1.505625pt}%
\definecolor{currentstroke}{rgb}{0.529412,0.807843,0.921569}%
\pgfsetstrokecolor{currentstroke}%
\pgfsetdash{}{0pt}%
\pgfpathmoveto{\pgfqpoint{0.948205in}{0.637623in}}%
\pgfpathlineto{\pgfqpoint{1.339619in}{0.909846in}}%
\pgfpathlineto{\pgfqpoint{1.731033in}{1.182068in}}%
\pgfpathlineto{\pgfqpoint{2.122447in}{1.454290in}}%
\pgfpathlineto{\pgfqpoint{2.513862in}{1.726512in}}%
\pgfpathlineto{\pgfqpoint{2.905276in}{1.998734in}}%
\pgfpathlineto{\pgfqpoint{3.296690in}{2.270957in}}%
\pgfpathlineto{\pgfqpoint{3.688104in}{2.543179in}}%
\pgfpathlineto{\pgfqpoint{4.079518in}{2.815401in}}%
\pgfpathlineto{\pgfqpoint{4.470932in}{3.087623in}}%
\pgfusepath{stroke}%
\end{pgfscope}%
\begin{pgfscope}%
\pgfsetbuttcap%
\pgfsetroundjoin%
\definecolor{currentfill}{rgb}{0.000000,0.000000,0.000000}%
\pgfsetfillcolor{currentfill}%
\pgfsetlinewidth{0.803000pt}%
\definecolor{currentstroke}{rgb}{0.000000,0.000000,0.000000}%
\pgfsetstrokecolor{currentstroke}%
\pgfsetdash{}{0pt}%
\pgfsys@defobject{currentmarker}{\pgfqpoint{0.000000in}{-0.048611in}}{\pgfqpoint{0.000000in}{0.000000in}}{%
\pgfpathmoveto{\pgfqpoint{0.000000in}{0.000000in}}%
\pgfpathlineto{\pgfqpoint{0.000000in}{-0.048611in}}%
\pgfusepath{stroke,fill}%
}%
\begin{pgfscope}%
\pgfsys@transformshift{1.268453in}{0.515123in}%
\pgfsys@useobject{currentmarker}{}%
\end{pgfscope}%
\end{pgfscope}%
\begin{pgfscope}%
\definecolor{textcolor}{rgb}{0.000000,0.000000,0.000000}%
\pgfsetstrokecolor{textcolor}%
\pgfsetfillcolor{textcolor}%
\pgftext[x=1.268453in,y=0.417901in,,top]{\color{textcolor}\rmfamily\fontsize{10.000000}{12.000000}\selectfont \(\displaystyle 10\)}%
\end{pgfscope}%
\begin{pgfscope}%
\pgfsetbuttcap%
\pgfsetroundjoin%
\definecolor{currentfill}{rgb}{0.000000,0.000000,0.000000}%
\pgfsetfillcolor{currentfill}%
\pgfsetlinewidth{0.803000pt}%
\definecolor{currentstroke}{rgb}{0.000000,0.000000,0.000000}%
\pgfsetstrokecolor{currentstroke}%
\pgfsetdash{}{0pt}%
\pgfsys@defobject{currentmarker}{\pgfqpoint{0.000000in}{-0.048611in}}{\pgfqpoint{0.000000in}{0.000000in}}{%
\pgfpathmoveto{\pgfqpoint{0.000000in}{0.000000in}}%
\pgfpathlineto{\pgfqpoint{0.000000in}{-0.048611in}}%
\pgfusepath{stroke,fill}%
}%
\begin{pgfscope}%
\pgfsys@transformshift{1.908949in}{0.515123in}%
\pgfsys@useobject{currentmarker}{}%
\end{pgfscope}%
\end{pgfscope}%
\begin{pgfscope}%
\definecolor{textcolor}{rgb}{0.000000,0.000000,0.000000}%
\pgfsetstrokecolor{textcolor}%
\pgfsetfillcolor{textcolor}%
\pgftext[x=1.908949in,y=0.417901in,,top]{\color{textcolor}\rmfamily\fontsize{10.000000}{12.000000}\selectfont \(\displaystyle 20\)}%
\end{pgfscope}%
\begin{pgfscope}%
\pgfsetbuttcap%
\pgfsetroundjoin%
\definecolor{currentfill}{rgb}{0.000000,0.000000,0.000000}%
\pgfsetfillcolor{currentfill}%
\pgfsetlinewidth{0.803000pt}%
\definecolor{currentstroke}{rgb}{0.000000,0.000000,0.000000}%
\pgfsetstrokecolor{currentstroke}%
\pgfsetdash{}{0pt}%
\pgfsys@defobject{currentmarker}{\pgfqpoint{0.000000in}{-0.048611in}}{\pgfqpoint{0.000000in}{0.000000in}}{%
\pgfpathmoveto{\pgfqpoint{0.000000in}{0.000000in}}%
\pgfpathlineto{\pgfqpoint{0.000000in}{-0.048611in}}%
\pgfusepath{stroke,fill}%
}%
\begin{pgfscope}%
\pgfsys@transformshift{2.549445in}{0.515123in}%
\pgfsys@useobject{currentmarker}{}%
\end{pgfscope}%
\end{pgfscope}%
\begin{pgfscope}%
\definecolor{textcolor}{rgb}{0.000000,0.000000,0.000000}%
\pgfsetstrokecolor{textcolor}%
\pgfsetfillcolor{textcolor}%
\pgftext[x=2.549445in,y=0.417901in,,top]{\color{textcolor}\rmfamily\fontsize{10.000000}{12.000000}\selectfont \(\displaystyle 30\)}%
\end{pgfscope}%
\begin{pgfscope}%
\pgfsetbuttcap%
\pgfsetroundjoin%
\definecolor{currentfill}{rgb}{0.000000,0.000000,0.000000}%
\pgfsetfillcolor{currentfill}%
\pgfsetlinewidth{0.803000pt}%
\definecolor{currentstroke}{rgb}{0.000000,0.000000,0.000000}%
\pgfsetstrokecolor{currentstroke}%
\pgfsetdash{}{0pt}%
\pgfsys@defobject{currentmarker}{\pgfqpoint{0.000000in}{-0.048611in}}{\pgfqpoint{0.000000in}{0.000000in}}{%
\pgfpathmoveto{\pgfqpoint{0.000000in}{0.000000in}}%
\pgfpathlineto{\pgfqpoint{0.000000in}{-0.048611in}}%
\pgfusepath{stroke,fill}%
}%
\begin{pgfscope}%
\pgfsys@transformshift{3.189941in}{0.515123in}%
\pgfsys@useobject{currentmarker}{}%
\end{pgfscope}%
\end{pgfscope}%
\begin{pgfscope}%
\definecolor{textcolor}{rgb}{0.000000,0.000000,0.000000}%
\pgfsetstrokecolor{textcolor}%
\pgfsetfillcolor{textcolor}%
\pgftext[x=3.189941in,y=0.417901in,,top]{\color{textcolor}\rmfamily\fontsize{10.000000}{12.000000}\selectfont \(\displaystyle 40\)}%
\end{pgfscope}%
\begin{pgfscope}%
\pgfsetbuttcap%
\pgfsetroundjoin%
\definecolor{currentfill}{rgb}{0.000000,0.000000,0.000000}%
\pgfsetfillcolor{currentfill}%
\pgfsetlinewidth{0.803000pt}%
\definecolor{currentstroke}{rgb}{0.000000,0.000000,0.000000}%
\pgfsetstrokecolor{currentstroke}%
\pgfsetdash{}{0pt}%
\pgfsys@defobject{currentmarker}{\pgfqpoint{0.000000in}{-0.048611in}}{\pgfqpoint{0.000000in}{0.000000in}}{%
\pgfpathmoveto{\pgfqpoint{0.000000in}{0.000000in}}%
\pgfpathlineto{\pgfqpoint{0.000000in}{-0.048611in}}%
\pgfusepath{stroke,fill}%
}%
\begin{pgfscope}%
\pgfsys@transformshift{3.830436in}{0.515123in}%
\pgfsys@useobject{currentmarker}{}%
\end{pgfscope}%
\end{pgfscope}%
\begin{pgfscope}%
\definecolor{textcolor}{rgb}{0.000000,0.000000,0.000000}%
\pgfsetstrokecolor{textcolor}%
\pgfsetfillcolor{textcolor}%
\pgftext[x=3.830436in,y=0.417901in,,top]{\color{textcolor}\rmfamily\fontsize{10.000000}{12.000000}\selectfont \(\displaystyle 50\)}%
\end{pgfscope}%
\begin{pgfscope}%
\pgfsetbuttcap%
\pgfsetroundjoin%
\definecolor{currentfill}{rgb}{0.000000,0.000000,0.000000}%
\pgfsetfillcolor{currentfill}%
\pgfsetlinewidth{0.803000pt}%
\definecolor{currentstroke}{rgb}{0.000000,0.000000,0.000000}%
\pgfsetstrokecolor{currentstroke}%
\pgfsetdash{}{0pt}%
\pgfsys@defobject{currentmarker}{\pgfqpoint{0.000000in}{-0.048611in}}{\pgfqpoint{0.000000in}{0.000000in}}{%
\pgfpathmoveto{\pgfqpoint{0.000000in}{0.000000in}}%
\pgfpathlineto{\pgfqpoint{0.000000in}{-0.048611in}}%
\pgfusepath{stroke,fill}%
}%
\begin{pgfscope}%
\pgfsys@transformshift{4.470932in}{0.515123in}%
\pgfsys@useobject{currentmarker}{}%
\end{pgfscope}%
\end{pgfscope}%
\begin{pgfscope}%
\definecolor{textcolor}{rgb}{0.000000,0.000000,0.000000}%
\pgfsetstrokecolor{textcolor}%
\pgfsetfillcolor{textcolor}%
\pgftext[x=4.470932in,y=0.417901in,,top]{\color{textcolor}\rmfamily\fontsize{10.000000}{12.000000}\selectfont \(\displaystyle 60\)}%
\end{pgfscope}%
\begin{pgfscope}%
\definecolor{textcolor}{rgb}{0.000000,0.000000,0.000000}%
\pgfsetstrokecolor{textcolor}%
\pgfsetfillcolor{textcolor}%
\pgftext[x=2.709569in,y=0.238889in,,top]{\color{textcolor}\rmfamily\fontsize{10.000000}{12.000000}\selectfont I(\(\displaystyle \mu\)A)}%
\end{pgfscope}%
\begin{pgfscope}%
\pgfsetbuttcap%
\pgfsetroundjoin%
\definecolor{currentfill}{rgb}{0.000000,0.000000,0.000000}%
\pgfsetfillcolor{currentfill}%
\pgfsetlinewidth{0.803000pt}%
\definecolor{currentstroke}{rgb}{0.000000,0.000000,0.000000}%
\pgfsetstrokecolor{currentstroke}%
\pgfsetdash{}{0pt}%
\pgfsys@defobject{currentmarker}{\pgfqpoint{-0.048611in}{0.000000in}}{\pgfqpoint{0.000000in}{0.000000in}}{%
\pgfpathmoveto{\pgfqpoint{0.000000in}{0.000000in}}%
\pgfpathlineto{\pgfqpoint{-0.048611in}{0.000000in}}%
\pgfusepath{stroke,fill}%
}%
\begin{pgfscope}%
\pgfsys@transformshift{0.772069in}{0.921510in}%
\pgfsys@useobject{currentmarker}{}%
\end{pgfscope}%
\end{pgfscope}%
\begin{pgfscope}%
\definecolor{textcolor}{rgb}{0.000000,0.000000,0.000000}%
\pgfsetstrokecolor{textcolor}%
\pgfsetfillcolor{textcolor}%
\pgftext[x=0.605402in,y=0.873285in,left,base]{\color{textcolor}\rmfamily\fontsize{10.000000}{12.000000}\selectfont \(\displaystyle 2\)}%
\end{pgfscope}%
\begin{pgfscope}%
\pgfsetbuttcap%
\pgfsetroundjoin%
\definecolor{currentfill}{rgb}{0.000000,0.000000,0.000000}%
\pgfsetfillcolor{currentfill}%
\pgfsetlinewidth{0.803000pt}%
\definecolor{currentstroke}{rgb}{0.000000,0.000000,0.000000}%
\pgfsetstrokecolor{currentstroke}%
\pgfsetdash{}{0pt}%
\pgfsys@defobject{currentmarker}{\pgfqpoint{-0.048611in}{0.000000in}}{\pgfqpoint{0.000000in}{0.000000in}}{%
\pgfpathmoveto{\pgfqpoint{0.000000in}{0.000000in}}%
\pgfpathlineto{\pgfqpoint{-0.048611in}{0.000000in}}%
\pgfusepath{stroke,fill}%
}%
\begin{pgfscope}%
\pgfsys@transformshift{0.772069in}{1.428124in}%
\pgfsys@useobject{currentmarker}{}%
\end{pgfscope}%
\end{pgfscope}%
\begin{pgfscope}%
\definecolor{textcolor}{rgb}{0.000000,0.000000,0.000000}%
\pgfsetstrokecolor{textcolor}%
\pgfsetfillcolor{textcolor}%
\pgftext[x=0.605402in,y=1.379899in,left,base]{\color{textcolor}\rmfamily\fontsize{10.000000}{12.000000}\selectfont \(\displaystyle 4\)}%
\end{pgfscope}%
\begin{pgfscope}%
\pgfsetbuttcap%
\pgfsetroundjoin%
\definecolor{currentfill}{rgb}{0.000000,0.000000,0.000000}%
\pgfsetfillcolor{currentfill}%
\pgfsetlinewidth{0.803000pt}%
\definecolor{currentstroke}{rgb}{0.000000,0.000000,0.000000}%
\pgfsetstrokecolor{currentstroke}%
\pgfsetdash{}{0pt}%
\pgfsys@defobject{currentmarker}{\pgfqpoint{-0.048611in}{0.000000in}}{\pgfqpoint{0.000000in}{0.000000in}}{%
\pgfpathmoveto{\pgfqpoint{0.000000in}{0.000000in}}%
\pgfpathlineto{\pgfqpoint{-0.048611in}{0.000000in}}%
\pgfusepath{stroke,fill}%
}%
\begin{pgfscope}%
\pgfsys@transformshift{0.772069in}{1.934738in}%
\pgfsys@useobject{currentmarker}{}%
\end{pgfscope}%
\end{pgfscope}%
\begin{pgfscope}%
\definecolor{textcolor}{rgb}{0.000000,0.000000,0.000000}%
\pgfsetstrokecolor{textcolor}%
\pgfsetfillcolor{textcolor}%
\pgftext[x=0.605402in,y=1.886513in,left,base]{\color{textcolor}\rmfamily\fontsize{10.000000}{12.000000}\selectfont \(\displaystyle 6\)}%
\end{pgfscope}%
\begin{pgfscope}%
\pgfsetbuttcap%
\pgfsetroundjoin%
\definecolor{currentfill}{rgb}{0.000000,0.000000,0.000000}%
\pgfsetfillcolor{currentfill}%
\pgfsetlinewidth{0.803000pt}%
\definecolor{currentstroke}{rgb}{0.000000,0.000000,0.000000}%
\pgfsetstrokecolor{currentstroke}%
\pgfsetdash{}{0pt}%
\pgfsys@defobject{currentmarker}{\pgfqpoint{-0.048611in}{0.000000in}}{\pgfqpoint{0.000000in}{0.000000in}}{%
\pgfpathmoveto{\pgfqpoint{0.000000in}{0.000000in}}%
\pgfpathlineto{\pgfqpoint{-0.048611in}{0.000000in}}%
\pgfusepath{stroke,fill}%
}%
\begin{pgfscope}%
\pgfsys@transformshift{0.772069in}{2.441352in}%
\pgfsys@useobject{currentmarker}{}%
\end{pgfscope}%
\end{pgfscope}%
\begin{pgfscope}%
\definecolor{textcolor}{rgb}{0.000000,0.000000,0.000000}%
\pgfsetstrokecolor{textcolor}%
\pgfsetfillcolor{textcolor}%
\pgftext[x=0.605402in,y=2.393127in,left,base]{\color{textcolor}\rmfamily\fontsize{10.000000}{12.000000}\selectfont \(\displaystyle 8\)}%
\end{pgfscope}%
\begin{pgfscope}%
\pgfsetbuttcap%
\pgfsetroundjoin%
\definecolor{currentfill}{rgb}{0.000000,0.000000,0.000000}%
\pgfsetfillcolor{currentfill}%
\pgfsetlinewidth{0.803000pt}%
\definecolor{currentstroke}{rgb}{0.000000,0.000000,0.000000}%
\pgfsetstrokecolor{currentstroke}%
\pgfsetdash{}{0pt}%
\pgfsys@defobject{currentmarker}{\pgfqpoint{-0.048611in}{0.000000in}}{\pgfqpoint{0.000000in}{0.000000in}}{%
\pgfpathmoveto{\pgfqpoint{0.000000in}{0.000000in}}%
\pgfpathlineto{\pgfqpoint{-0.048611in}{0.000000in}}%
\pgfusepath{stroke,fill}%
}%
\begin{pgfscope}%
\pgfsys@transformshift{0.772069in}{2.947966in}%
\pgfsys@useobject{currentmarker}{}%
\end{pgfscope}%
\end{pgfscope}%
\begin{pgfscope}%
\definecolor{textcolor}{rgb}{0.000000,0.000000,0.000000}%
\pgfsetstrokecolor{textcolor}%
\pgfsetfillcolor{textcolor}%
\pgftext[x=0.535957in,y=2.899741in,left,base]{\color{textcolor}\rmfamily\fontsize{10.000000}{12.000000}\selectfont \(\displaystyle 10\)}%
\end{pgfscope}%
\begin{pgfscope}%
\definecolor{textcolor}{rgb}{0.000000,0.000000,0.000000}%
\pgfsetstrokecolor{textcolor}%
\pgfsetfillcolor{textcolor}%
\pgftext[x=0.258179in,y=1.862623in,,bottom]{\color{textcolor}\rmfamily\fontsize{10.000000}{12.000000}\selectfont V(V)}%
\end{pgfscope}%
\begin{pgfscope}%
\pgfpathrectangle{\pgfqpoint{0.772069in}{0.515123in}}{\pgfqpoint{3.875000in}{2.695000in}}%
\pgfusepath{clip}%
\pgfsetbuttcap%
\pgfsetroundjoin%
\definecolor{currentfill}{rgb}{0.121569,0.466667,0.705882}%
\pgfsetfillcolor{currentfill}%
\pgfsetlinewidth{1.003750pt}%
\definecolor{currentstroke}{rgb}{0.121569,0.466667,0.705882}%
\pgfsetstrokecolor{currentstroke}%
\pgfsetdash{}{0pt}%
\pgfpathmoveto{\pgfqpoint{1.057089in}{0.673145in}}%
\pgfpathcurveto{\pgfqpoint{1.068139in}{0.673145in}}{\pgfqpoint{1.078739in}{0.677535in}}{\pgfqpoint{1.086552in}{0.685349in}}%
\pgfpathcurveto{\pgfqpoint{1.094366in}{0.693162in}}{\pgfqpoint{1.098756in}{0.703761in}}{\pgfqpoint{1.098756in}{0.714812in}}%
\pgfpathcurveto{\pgfqpoint{1.098756in}{0.725862in}}{\pgfqpoint{1.094366in}{0.736461in}}{\pgfqpoint{1.086552in}{0.744274in}}%
\pgfpathcurveto{\pgfqpoint{1.078739in}{0.752088in}}{\pgfqpoint{1.068139in}{0.756478in}}{\pgfqpoint{1.057089in}{0.756478in}}%
\pgfpathcurveto{\pgfqpoint{1.046039in}{0.756478in}}{\pgfqpoint{1.035440in}{0.752088in}}{\pgfqpoint{1.027627in}{0.744274in}}%
\pgfpathcurveto{\pgfqpoint{1.019813in}{0.736461in}}{\pgfqpoint{1.015423in}{0.725862in}}{\pgfqpoint{1.015423in}{0.714812in}}%
\pgfpathcurveto{\pgfqpoint{1.015423in}{0.703761in}}{\pgfqpoint{1.019813in}{0.693162in}}{\pgfqpoint{1.027627in}{0.685349in}}%
\pgfpathcurveto{\pgfqpoint{1.035440in}{0.677535in}}{\pgfqpoint{1.046039in}{0.673145in}}{\pgfqpoint{1.057089in}{0.673145in}}%
\pgfpathclose%
\pgfusepath{stroke,fill}%
\end{pgfscope}%
\begin{pgfscope}%
\pgfpathrectangle{\pgfqpoint{0.772069in}{0.515123in}}{\pgfqpoint{3.875000in}{2.695000in}}%
\pgfusepath{clip}%
\pgfsetbuttcap%
\pgfsetroundjoin%
\definecolor{currentfill}{rgb}{0.121569,0.466667,0.705882}%
\pgfsetfillcolor{currentfill}%
\pgfsetlinewidth{1.003750pt}%
\definecolor{currentstroke}{rgb}{0.121569,0.466667,0.705882}%
\pgfsetstrokecolor{currentstroke}%
\pgfsetdash{}{0pt}%
\pgfpathmoveto{\pgfqpoint{1.396552in}{0.907707in}}%
\pgfpathcurveto{\pgfqpoint{1.407602in}{0.907707in}}{\pgfqpoint{1.418201in}{0.912097in}}{\pgfqpoint{1.426015in}{0.919911in}}%
\pgfpathcurveto{\pgfqpoint{1.433829in}{0.927725in}}{\pgfqpoint{1.438219in}{0.938324in}}{\pgfqpoint{1.438219in}{0.949374in}}%
\pgfpathcurveto{\pgfqpoint{1.438219in}{0.960424in}}{\pgfqpoint{1.433829in}{0.971023in}}{\pgfqpoint{1.426015in}{0.978837in}}%
\pgfpathcurveto{\pgfqpoint{1.418201in}{0.986650in}}{\pgfqpoint{1.407602in}{0.991040in}}{\pgfqpoint{1.396552in}{0.991040in}}%
\pgfpathcurveto{\pgfqpoint{1.385502in}{0.991040in}}{\pgfqpoint{1.374903in}{0.986650in}}{\pgfqpoint{1.367089in}{0.978837in}}%
\pgfpathcurveto{\pgfqpoint{1.359276in}{0.971023in}}{\pgfqpoint{1.354885in}{0.960424in}}{\pgfqpoint{1.354885in}{0.949374in}}%
\pgfpathcurveto{\pgfqpoint{1.354885in}{0.938324in}}{\pgfqpoint{1.359276in}{0.927725in}}{\pgfqpoint{1.367089in}{0.919911in}}%
\pgfpathcurveto{\pgfqpoint{1.374903in}{0.912097in}}{\pgfqpoint{1.385502in}{0.907707in}}{\pgfqpoint{1.396552in}{0.907707in}}%
\pgfpathclose%
\pgfusepath{stroke,fill}%
\end{pgfscope}%
\begin{pgfscope}%
\pgfpathrectangle{\pgfqpoint{0.772069in}{0.515123in}}{\pgfqpoint{3.875000in}{2.695000in}}%
\pgfusepath{clip}%
\pgfsetbuttcap%
\pgfsetroundjoin%
\definecolor{currentfill}{rgb}{0.121569,0.466667,0.705882}%
\pgfsetfillcolor{currentfill}%
\pgfsetlinewidth{1.003750pt}%
\definecolor{currentstroke}{rgb}{0.121569,0.466667,0.705882}%
\pgfsetstrokecolor{currentstroke}%
\pgfsetdash{}{0pt}%
\pgfpathmoveto{\pgfqpoint{1.729610in}{1.143283in}}%
\pgfpathcurveto{\pgfqpoint{1.740660in}{1.143283in}}{\pgfqpoint{1.751259in}{1.147673in}}{\pgfqpoint{1.759073in}{1.155486in}}%
\pgfpathcurveto{\pgfqpoint{1.766886in}{1.163300in}}{\pgfqpoint{1.771277in}{1.173899in}}{\pgfqpoint{1.771277in}{1.184949in}}%
\pgfpathcurveto{\pgfqpoint{1.771277in}{1.195999in}}{\pgfqpoint{1.766886in}{1.206598in}}{\pgfqpoint{1.759073in}{1.214412in}}%
\pgfpathcurveto{\pgfqpoint{1.751259in}{1.222226in}}{\pgfqpoint{1.740660in}{1.226616in}}{\pgfqpoint{1.729610in}{1.226616in}}%
\pgfpathcurveto{\pgfqpoint{1.718560in}{1.226616in}}{\pgfqpoint{1.707961in}{1.222226in}}{\pgfqpoint{1.700147in}{1.214412in}}%
\pgfpathcurveto{\pgfqpoint{1.692334in}{1.206598in}}{\pgfqpoint{1.687943in}{1.195999in}}{\pgfqpoint{1.687943in}{1.184949in}}%
\pgfpathcurveto{\pgfqpoint{1.687943in}{1.173899in}}{\pgfqpoint{1.692334in}{1.163300in}}{\pgfqpoint{1.700147in}{1.155486in}}%
\pgfpathcurveto{\pgfqpoint{1.707961in}{1.147673in}}{\pgfqpoint{1.718560in}{1.143283in}}{\pgfqpoint{1.729610in}{1.143283in}}%
\pgfpathclose%
\pgfusepath{stroke,fill}%
\end{pgfscope}%
\begin{pgfscope}%
\pgfpathrectangle{\pgfqpoint{0.772069in}{0.515123in}}{\pgfqpoint{3.875000in}{2.695000in}}%
\pgfusepath{clip}%
\pgfsetbuttcap%
\pgfsetroundjoin%
\definecolor{currentfill}{rgb}{0.121569,0.466667,0.705882}%
\pgfsetfillcolor{currentfill}%
\pgfsetlinewidth{1.003750pt}%
\definecolor{currentstroke}{rgb}{0.121569,0.466667,0.705882}%
\pgfsetstrokecolor{currentstroke}%
\pgfsetdash{}{0pt}%
\pgfpathmoveto{\pgfqpoint{2.101098in}{1.401656in}}%
\pgfpathcurveto{\pgfqpoint{2.112148in}{1.401656in}}{\pgfqpoint{2.122747in}{1.406046in}}{\pgfqpoint{2.130560in}{1.413860in}}%
\pgfpathcurveto{\pgfqpoint{2.138374in}{1.421673in}}{\pgfqpoint{2.142764in}{1.432272in}}{\pgfqpoint{2.142764in}{1.443322in}}%
\pgfpathcurveto{\pgfqpoint{2.142764in}{1.454372in}}{\pgfqpoint{2.138374in}{1.464972in}}{\pgfqpoint{2.130560in}{1.472785in}}%
\pgfpathcurveto{\pgfqpoint{2.122747in}{1.480599in}}{\pgfqpoint{2.112148in}{1.484989in}}{\pgfqpoint{2.101098in}{1.484989in}}%
\pgfpathcurveto{\pgfqpoint{2.090047in}{1.484989in}}{\pgfqpoint{2.079448in}{1.480599in}}{\pgfqpoint{2.071635in}{1.472785in}}%
\pgfpathcurveto{\pgfqpoint{2.063821in}{1.464972in}}{\pgfqpoint{2.059431in}{1.454372in}}{\pgfqpoint{2.059431in}{1.443322in}}%
\pgfpathcurveto{\pgfqpoint{2.059431in}{1.432272in}}{\pgfqpoint{2.063821in}{1.421673in}}{\pgfqpoint{2.071635in}{1.413860in}}%
\pgfpathcurveto{\pgfqpoint{2.079448in}{1.406046in}}{\pgfqpoint{2.090047in}{1.401656in}}{\pgfqpoint{2.101098in}{1.401656in}}%
\pgfpathclose%
\pgfusepath{stroke,fill}%
\end{pgfscope}%
\begin{pgfscope}%
\pgfpathrectangle{\pgfqpoint{0.772069in}{0.515123in}}{\pgfqpoint{3.875000in}{2.695000in}}%
\pgfusepath{clip}%
\pgfsetbuttcap%
\pgfsetroundjoin%
\definecolor{currentfill}{rgb}{0.121569,0.466667,0.705882}%
\pgfsetfillcolor{currentfill}%
\pgfsetlinewidth{1.003750pt}%
\definecolor{currentstroke}{rgb}{0.121569,0.466667,0.705882}%
\pgfsetstrokecolor{currentstroke}%
\pgfsetdash{}{0pt}%
\pgfpathmoveto{\pgfqpoint{2.459775in}{1.649897in}}%
\pgfpathcurveto{\pgfqpoint{2.470825in}{1.649897in}}{\pgfqpoint{2.481424in}{1.654287in}}{\pgfqpoint{2.489238in}{1.662100in}}%
\pgfpathcurveto{\pgfqpoint{2.497052in}{1.669914in}}{\pgfqpoint{2.501442in}{1.680513in}}{\pgfqpoint{2.501442in}{1.691563in}}%
\pgfpathcurveto{\pgfqpoint{2.501442in}{1.702613in}}{\pgfqpoint{2.497052in}{1.713212in}}{\pgfqpoint{2.489238in}{1.721026in}}%
\pgfpathcurveto{\pgfqpoint{2.481424in}{1.728840in}}{\pgfqpoint{2.470825in}{1.733230in}}{\pgfqpoint{2.459775in}{1.733230in}}%
\pgfpathcurveto{\pgfqpoint{2.448725in}{1.733230in}}{\pgfqpoint{2.438126in}{1.728840in}}{\pgfqpoint{2.430313in}{1.721026in}}%
\pgfpathcurveto{\pgfqpoint{2.422499in}{1.713212in}}{\pgfqpoint{2.418109in}{1.702613in}}{\pgfqpoint{2.418109in}{1.691563in}}%
\pgfpathcurveto{\pgfqpoint{2.418109in}{1.680513in}}{\pgfqpoint{2.422499in}{1.669914in}}{\pgfqpoint{2.430313in}{1.662100in}}%
\pgfpathcurveto{\pgfqpoint{2.438126in}{1.654287in}}{\pgfqpoint{2.448725in}{1.649897in}}{\pgfqpoint{2.459775in}{1.649897in}}%
\pgfpathclose%
\pgfusepath{stroke,fill}%
\end{pgfscope}%
\begin{pgfscope}%
\pgfpathrectangle{\pgfqpoint{0.772069in}{0.515123in}}{\pgfqpoint{3.875000in}{2.695000in}}%
\pgfusepath{clip}%
\pgfsetbuttcap%
\pgfsetroundjoin%
\definecolor{currentfill}{rgb}{0.121569,0.466667,0.705882}%
\pgfsetfillcolor{currentfill}%
\pgfsetlinewidth{1.003750pt}%
\definecolor{currentstroke}{rgb}{0.121569,0.466667,0.705882}%
\pgfsetstrokecolor{currentstroke}%
\pgfsetdash{}{0pt}%
\pgfpathmoveto{\pgfqpoint{2.837668in}{1.913336in}}%
\pgfpathcurveto{\pgfqpoint{2.848718in}{1.913336in}}{\pgfqpoint{2.859317in}{1.917726in}}{\pgfqpoint{2.867131in}{1.925540in}}%
\pgfpathcurveto{\pgfqpoint{2.874944in}{1.933353in}}{\pgfqpoint{2.879335in}{1.943952in}}{\pgfqpoint{2.879335in}{1.955002in}}%
\pgfpathcurveto{\pgfqpoint{2.879335in}{1.966053in}}{\pgfqpoint{2.874944in}{1.976652in}}{\pgfqpoint{2.867131in}{1.984465in}}%
\pgfpathcurveto{\pgfqpoint{2.859317in}{1.992279in}}{\pgfqpoint{2.848718in}{1.996669in}}{\pgfqpoint{2.837668in}{1.996669in}}%
\pgfpathcurveto{\pgfqpoint{2.826618in}{1.996669in}}{\pgfqpoint{2.816019in}{1.992279in}}{\pgfqpoint{2.808205in}{1.984465in}}%
\pgfpathcurveto{\pgfqpoint{2.800391in}{1.976652in}}{\pgfqpoint{2.796001in}{1.966053in}}{\pgfqpoint{2.796001in}{1.955002in}}%
\pgfpathcurveto{\pgfqpoint{2.796001in}{1.943952in}}{\pgfqpoint{2.800391in}{1.933353in}}{\pgfqpoint{2.808205in}{1.925540in}}%
\pgfpathcurveto{\pgfqpoint{2.816019in}{1.917726in}}{\pgfqpoint{2.826618in}{1.913336in}}{\pgfqpoint{2.837668in}{1.913336in}}%
\pgfpathclose%
\pgfusepath{stroke,fill}%
\end{pgfscope}%
\begin{pgfscope}%
\pgfpathrectangle{\pgfqpoint{0.772069in}{0.515123in}}{\pgfqpoint{3.875000in}{2.695000in}}%
\pgfusepath{clip}%
\pgfsetbuttcap%
\pgfsetroundjoin%
\definecolor{currentfill}{rgb}{0.121569,0.466667,0.705882}%
\pgfsetfillcolor{currentfill}%
\pgfsetlinewidth{1.003750pt}%
\definecolor{currentstroke}{rgb}{0.121569,0.466667,0.705882}%
\pgfsetstrokecolor{currentstroke}%
\pgfsetdash{}{0pt}%
\pgfpathmoveto{\pgfqpoint{3.209155in}{2.171709in}}%
\pgfpathcurveto{\pgfqpoint{3.220206in}{2.171709in}}{\pgfqpoint{3.230805in}{2.176099in}}{\pgfqpoint{3.238618in}{2.183913in}}%
\pgfpathcurveto{\pgfqpoint{3.246432in}{2.191726in}}{\pgfqpoint{3.250822in}{2.202325in}}{\pgfqpoint{3.250822in}{2.213376in}}%
\pgfpathcurveto{\pgfqpoint{3.250822in}{2.224426in}}{\pgfqpoint{3.246432in}{2.235025in}}{\pgfqpoint{3.238618in}{2.242838in}}%
\pgfpathcurveto{\pgfqpoint{3.230805in}{2.250652in}}{\pgfqpoint{3.220206in}{2.255042in}}{\pgfqpoint{3.209155in}{2.255042in}}%
\pgfpathcurveto{\pgfqpoint{3.198105in}{2.255042in}}{\pgfqpoint{3.187506in}{2.250652in}}{\pgfqpoint{3.179693in}{2.242838in}}%
\pgfpathcurveto{\pgfqpoint{3.171879in}{2.235025in}}{\pgfqpoint{3.167489in}{2.224426in}}{\pgfqpoint{3.167489in}{2.213376in}}%
\pgfpathcurveto{\pgfqpoint{3.167489in}{2.202325in}}{\pgfqpoint{3.171879in}{2.191726in}}{\pgfqpoint{3.179693in}{2.183913in}}%
\pgfpathcurveto{\pgfqpoint{3.187506in}{2.176099in}}{\pgfqpoint{3.198105in}{2.171709in}}{\pgfqpoint{3.209155in}{2.171709in}}%
\pgfpathclose%
\pgfusepath{stroke,fill}%
\end{pgfscope}%
\begin{pgfscope}%
\pgfpathrectangle{\pgfqpoint{0.772069in}{0.515123in}}{\pgfqpoint{3.875000in}{2.695000in}}%
\pgfusepath{clip}%
\pgfsetbuttcap%
\pgfsetroundjoin%
\definecolor{currentfill}{rgb}{0.121569,0.466667,0.705882}%
\pgfsetfillcolor{currentfill}%
\pgfsetlinewidth{1.003750pt}%
\definecolor{currentstroke}{rgb}{0.121569,0.466667,0.705882}%
\pgfsetstrokecolor{currentstroke}%
\pgfsetdash{}{0pt}%
\pgfpathmoveto{\pgfqpoint{3.574238in}{2.425016in}}%
\pgfpathcurveto{\pgfqpoint{3.585288in}{2.425016in}}{\pgfqpoint{3.595887in}{2.429406in}}{\pgfqpoint{3.603701in}{2.437220in}}%
\pgfpathcurveto{\pgfqpoint{3.611515in}{2.445033in}}{\pgfqpoint{3.615905in}{2.455632in}}{\pgfqpoint{3.615905in}{2.466683in}}%
\pgfpathcurveto{\pgfqpoint{3.615905in}{2.477733in}}{\pgfqpoint{3.611515in}{2.488332in}}{\pgfqpoint{3.603701in}{2.496145in}}%
\pgfpathcurveto{\pgfqpoint{3.595887in}{2.503959in}}{\pgfqpoint{3.585288in}{2.508349in}}{\pgfqpoint{3.574238in}{2.508349in}}%
\pgfpathcurveto{\pgfqpoint{3.563188in}{2.508349in}}{\pgfqpoint{3.552589in}{2.503959in}}{\pgfqpoint{3.544775in}{2.496145in}}%
\pgfpathcurveto{\pgfqpoint{3.536962in}{2.488332in}}{\pgfqpoint{3.532571in}{2.477733in}}{\pgfqpoint{3.532571in}{2.466683in}}%
\pgfpathcurveto{\pgfqpoint{3.532571in}{2.455632in}}{\pgfqpoint{3.536962in}{2.445033in}}{\pgfqpoint{3.544775in}{2.437220in}}%
\pgfpathcurveto{\pgfqpoint{3.552589in}{2.429406in}}{\pgfqpoint{3.563188in}{2.425016in}}{\pgfqpoint{3.574238in}{2.425016in}}%
\pgfpathclose%
\pgfusepath{stroke,fill}%
\end{pgfscope}%
\begin{pgfscope}%
\pgfpathrectangle{\pgfqpoint{0.772069in}{0.515123in}}{\pgfqpoint{3.875000in}{2.695000in}}%
\pgfusepath{clip}%
\pgfsetbuttcap%
\pgfsetroundjoin%
\definecolor{currentfill}{rgb}{0.121569,0.466667,0.705882}%
\pgfsetfillcolor{currentfill}%
\pgfsetlinewidth{1.003750pt}%
\definecolor{currentstroke}{rgb}{0.121569,0.466667,0.705882}%
\pgfsetstrokecolor{currentstroke}%
\pgfsetdash{}{0pt}%
\pgfpathmoveto{\pgfqpoint{3.964941in}{2.693521in}}%
\pgfpathcurveto{\pgfqpoint{3.975991in}{2.693521in}}{\pgfqpoint{3.986590in}{2.697912in}}{\pgfqpoint{3.994403in}{2.705725in}}%
\pgfpathcurveto{\pgfqpoint{4.002217in}{2.713539in}}{\pgfqpoint{4.006607in}{2.724138in}}{\pgfqpoint{4.006607in}{2.735188in}}%
\pgfpathcurveto{\pgfqpoint{4.006607in}{2.746238in}}{\pgfqpoint{4.002217in}{2.756837in}}{\pgfqpoint{3.994403in}{2.764651in}}%
\pgfpathcurveto{\pgfqpoint{3.986590in}{2.772464in}}{\pgfqpoint{3.975991in}{2.776855in}}{\pgfqpoint{3.964941in}{2.776855in}}%
\pgfpathcurveto{\pgfqpoint{3.953890in}{2.776855in}}{\pgfqpoint{3.943291in}{2.772464in}}{\pgfqpoint{3.935478in}{2.764651in}}%
\pgfpathcurveto{\pgfqpoint{3.927664in}{2.756837in}}{\pgfqpoint{3.923274in}{2.746238in}}{\pgfqpoint{3.923274in}{2.735188in}}%
\pgfpathcurveto{\pgfqpoint{3.923274in}{2.724138in}}{\pgfqpoint{3.927664in}{2.713539in}}{\pgfqpoint{3.935478in}{2.705725in}}%
\pgfpathcurveto{\pgfqpoint{3.943291in}{2.697912in}}{\pgfqpoint{3.953890in}{2.693521in}}{\pgfqpoint{3.964941in}{2.693521in}}%
\pgfpathclose%
\pgfusepath{stroke,fill}%
\end{pgfscope}%
\begin{pgfscope}%
\pgfpathrectangle{\pgfqpoint{0.772069in}{0.515123in}}{\pgfqpoint{3.875000in}{2.695000in}}%
\pgfusepath{clip}%
\pgfsetbuttcap%
\pgfsetroundjoin%
\definecolor{currentfill}{rgb}{0.121569,0.466667,0.705882}%
\pgfsetfillcolor{currentfill}%
\pgfsetlinewidth{1.003750pt}%
\definecolor{currentstroke}{rgb}{0.121569,0.466667,0.705882}%
\pgfsetstrokecolor{currentstroke}%
\pgfsetdash{}{0pt}%
\pgfpathmoveto{\pgfqpoint{4.317213in}{2.941762in}}%
\pgfpathcurveto{\pgfqpoint{4.328263in}{2.941762in}}{\pgfqpoint{4.338862in}{2.946152in}}{\pgfqpoint{4.346676in}{2.953966in}}%
\pgfpathcurveto{\pgfqpoint{4.354490in}{2.961780in}}{\pgfqpoint{4.358880in}{2.972379in}}{\pgfqpoint{4.358880in}{2.983429in}}%
\pgfpathcurveto{\pgfqpoint{4.358880in}{2.994479in}}{\pgfqpoint{4.354490in}{3.005078in}}{\pgfqpoint{4.346676in}{3.012892in}}%
\pgfpathcurveto{\pgfqpoint{4.338862in}{3.020705in}}{\pgfqpoint{4.328263in}{3.025095in}}{\pgfqpoint{4.317213in}{3.025095in}}%
\pgfpathcurveto{\pgfqpoint{4.306163in}{3.025095in}}{\pgfqpoint{4.295564in}{3.020705in}}{\pgfqpoint{4.287751in}{3.012892in}}%
\pgfpathcurveto{\pgfqpoint{4.279937in}{3.005078in}}{\pgfqpoint{4.275547in}{2.994479in}}{\pgfqpoint{4.275547in}{2.983429in}}%
\pgfpathcurveto{\pgfqpoint{4.275547in}{2.972379in}}{\pgfqpoint{4.279937in}{2.961780in}}{\pgfqpoint{4.287751in}{2.953966in}}%
\pgfpathcurveto{\pgfqpoint{4.295564in}{2.946152in}}{\pgfqpoint{4.306163in}{2.941762in}}{\pgfqpoint{4.317213in}{2.941762in}}%
\pgfpathclose%
\pgfusepath{stroke,fill}%
\end{pgfscope}%
\begin{pgfscope}%
\pgfsetrectcap%
\pgfsetmiterjoin%
\pgfsetlinewidth{0.803000pt}%
\definecolor{currentstroke}{rgb}{0.000000,0.000000,0.000000}%
\pgfsetstrokecolor{currentstroke}%
\pgfsetdash{}{0pt}%
\pgfpathmoveto{\pgfqpoint{0.772069in}{0.515123in}}%
\pgfpathlineto{\pgfqpoint{0.772069in}{3.210123in}}%
\pgfusepath{stroke}%
\end{pgfscope}%
\begin{pgfscope}%
\pgfsetrectcap%
\pgfsetmiterjoin%
\pgfsetlinewidth{0.803000pt}%
\definecolor{currentstroke}{rgb}{0.000000,0.000000,0.000000}%
\pgfsetstrokecolor{currentstroke}%
\pgfsetdash{}{0pt}%
\pgfpathmoveto{\pgfqpoint{4.647069in}{0.515123in}}%
\pgfpathlineto{\pgfqpoint{4.647069in}{3.210123in}}%
\pgfusepath{stroke}%
\end{pgfscope}%
\begin{pgfscope}%
\pgfsetrectcap%
\pgfsetmiterjoin%
\pgfsetlinewidth{0.803000pt}%
\definecolor{currentstroke}{rgb}{0.000000,0.000000,0.000000}%
\pgfsetstrokecolor{currentstroke}%
\pgfsetdash{}{0pt}%
\pgfpathmoveto{\pgfqpoint{0.772069in}{0.515123in}}%
\pgfpathlineto{\pgfqpoint{4.647069in}{0.515123in}}%
\pgfusepath{stroke}%
\end{pgfscope}%
\begin{pgfscope}%
\pgfsetrectcap%
\pgfsetmiterjoin%
\pgfsetlinewidth{0.803000pt}%
\definecolor{currentstroke}{rgb}{0.000000,0.000000,0.000000}%
\pgfsetstrokecolor{currentstroke}%
\pgfsetdash{}{0pt}%
\pgfpathmoveto{\pgfqpoint{0.772069in}{3.210123in}}%
\pgfpathlineto{\pgfqpoint{4.647069in}{3.210123in}}%
\pgfusepath{stroke}%
\end{pgfscope}%
\end{pgfpicture}%
\makeatother%
\endgroup%

    \caption{Voltaje (V) frente a intensidad (I) con regresión lineal}
  \end{figure}

  A simple vista se puede observar que el ajuste es razonablemente preciso. Sin embargo, podemos obtener una medida fiable de cúan preciso es. Para ello utilizamos el coeficiente de regresión lineal (Ecuación \ref{ec:r}). En este caso, obtenemos que es $r = 0.999997$, un ajuste con cinco nueves, lo cual muestra una precision notable.

  Una propiedad interesante de esta gráfica se puede derivar de la Ley de Ohm \ref{ec:ohm}. Si despejamos \textit{R} en función de \textit{I} y \textit{V}, obtenemos $R = \frac{V}{I}$. De esto se puede deducir que la pendiente de la gráfica ($\frac{\Delta V}{\Delta I}$) será constante e igual a la resistencia. Este término no es otro que nuestra constante de regresión lineal $b = 1,75 \cdot 10^5$. Si nos fijamos, es prácticamente igual al valor experimental que obtuvimos al medir las resistencias (\ref{tab:res}) y del que conseguimos al aplicar la Ley de Ohm (\ref{tb:ohm}).


  \newpage
  \section{Circuito en Serie}

  La siguiente experiencia consiste en crear un circuito con tres resistencias ($R_1$, $R_2$ y $R_3$) en serie con el objetivo de medir la instensidad del circuito, y el potencial total y en cada resistencia. Para ello colocaremos los componentes de la siguiente manera:

  \begin{figure}[H]
    \centering
    \begin{circuitikz}[european]
      \draw (0,0) to[voltage source] (0,3)
      to[R=$R_1$] (3,3)
      to[R=$R_2$] (3,0)
      to[R=$R_3$] (0,0);
    \end{circuitikz}
    \caption{Circuito con tres resistencias en serie}
    \label{circuito:serie}
  \end{figure}

  \subsection{Procedimiento de medición}

  Para medir las diferentes magnitudes, colocaremos el polímetro en serie para las intensidades y en paralelo para los voltajes.

  \begin{figure}[H]
    \centering
    \begin{circuitikz}[european]
      \draw (0,0) to[voltage source] (0,2)
      to[R=$R_1$] (2,2)
      to[R=$R_2$] (2,0)
      to[R=$R_3$] (0,0);
      \draw (0,2) -- (0,3.5)
      to[voltmeter, l=$V_1$] (2,3.5) -- (2,2);
    \end{circuitikz}
    \raisebox{0.28in}{
    \begin{circuitikz}[european]
      \draw (0,0) to[voltage source] (0,2)
      to[R=$R_1$] (2,2)
      to[R=$R_2$] (2,0)
      to[R=$R_3$] (0,0);
      \draw (2,2) -- (3.5,2)
      to[voltmeter, l=$V_2$] (3.5,0) -- (2,0);
    \end{circuitikz}}
    \begin{circuitikz}[european]
      \draw (0,0) to[voltage source] (0,2)
      to[R=$R_1$] (2,2)
      to[R=$R_2$] (2,0)
      to[R=$R_3$] (0,0);
      \draw (2,2) -- (2,-1.5)
      to[voltmeter, l=$V_3$] (0,-1.5) -- (0,0);
    \end{circuitikz}
    \caption{Medición de potenciales de $R_1$, $R_2$ y $R_3$ respectivamente}
  \end{figure}

  \begin{figure}[H]
    \centering
    \begin{circuitikz}[european]
      \draw (0,0) to[voltage source] (0,2)
      to[R=$R_1$] (2,2)
      to[R=$R_2$] (4,2) -- (4,0)
      to[R=$R_3$] (2,0)
      to[ammeter, l=$I$] (0,0);
    \end{circuitikz}
    \caption{Medición de la intensidad total del circuito}
  \end{figure}

  \subsection{Resistencia equivalente}
  \label{sec:reseqserie}

  Primero debemos de averiguar cual es la resistencia equivalente de todo el circuito. Para ello utilizamos la siguiente fórmula, que se aplica a resistencias en serie:
  \begin{equation} \label{ec:reseq}
    R_S = \sum_{k=1}^N R_k \qquad R = R_1 + R_2 + R_3
  \end{equation}
  De aquí podemos deducir que la diferencia de potencial para cada resistencia será distinta, y que su suma ha de resultar en la total del circuito, y que la intensidad será la misma para todas las resistencias (\textbf{Leyes de Kirchhoff}) En este caso:
  \begin{gather}
    V = \sum_{k=1}^N V_k \qquad V = V_1 + V_2 + V_3 \nonumber \\
    I = I_k \qquad I = I_1 = I_2 = I_3 \label{ec:kriser}
  \end{gather}
  Por lo tanto, calculamos la resistencia equivalente del circuito aplicando la fórmula \ref{ec:reseq} y propagación de incertidumbres (\ref{ec:propinc}):
  \begin{gather}
    R_S = 175500 + 216000 + 394000 = 7,855 \cdot 10^5 \Omega \nonumber \\
    s(R_S) = \sqrt{\left ( \frac{\partial R_S}{\partial R_1} \right )^2 s^2(R_1) + \left ( \frac{\partial R_S}{\partial R_2} \right )^2 s^2(R_2) + \left ( \frac{\partial R_S}{\partial R_3} \right )^2 s^2(R_3)} \nonumber \\ s(R_S) = s(R_1) + s(R_2) + s(R_3) = \pm(100 + 1000 + 1000) = \pm2100 \Omega \nonumber \\ R_S = 7,855 \cdot 10^5 \pm 2,1 \cdot 10^3 \Omega \nonumber
  \end{gather}

  Si medimos experimentalmente el valor de la resistencia total del cirtuito obtenemos un valor $R_E = 7,89 \cdot 10^5 \pm 10^3$. No entra por completo dentro de nuestro intervalo de confianza ($R_T + 2,1 \cdot 10^3 = 7,876 \cdot 10^5$ y $R_E - 10^3 = 7,88 \cdot 10^5$), pero queda razonablemente cerca, difieren en una cantidad del orden de $10^2$, por lo que podemos considerar que se debe a condiciones del experimento (Humedad, temperatura...).

  \subsection{Medición experimental}

  Siguiendo el procedimiento anterior, realizamos una serie de medidas en el circuito \ref{circuito:serie}. Iremos variando el potencial (\textit{V}) de la fuente y anotando los cambios del resto de magnitudes. Como especificamos en el apartado \ref{sec:incert} tomaremos la resolución de la medida como su incertidumbre.

  \begin{table}[H]
  \centering
  \resizebox{\columnwidth}{!}{
  \csvreader[
    tabular=|c|c|c|c|c|c|,
    table head=\hline Medida & $V~(V) \pm s(V)$ & $V_1~(V) \pm s(V_1)$ & $V_2~(V) \pm s(V_2)$ & $V_3~(V) \pm s(V_3)$ & $I~(A) \pm s(I)$ \\ \hline,
    late after last line=\\\hline,
    separator=semicolon
    ]{CC6.csv}
    {v=\v, v1=\va, v2=\vb, v3=\vc, i=\int, sv=\sv, sv1=\sva, sv2=\svb, sv3=\svc, si=\si}
    {\thecsvrow & \v \hspace{4pt}$\pm$ \sv & \va \hspace{4pt}$\pm$ \sva & \vb \hspace{4pt}$\pm$ \svb & \vc \hspace{4pt}$\pm$ \svc & \int \hspace{4pt}$\pm$ \si}
  }
  \caption{Potenciales e intensidades del circuito en serie}
  \end{table}

  \subsection{Representación gráfica de V frente a I}

  Utilizaremos el mismo programa de \code{python} que en el apartado anterior para representar nuestro voltaje (\textit{V}) frente a la intensidad (\textit{I}).

  \begin{figure}[H]
    %\centering
    \hspace{2.5em} %% Creator: Matplotlib, PGF backend
%%
%% To include the figure in your LaTeX document, write
%%   \input{<filename>.pgf}
%%
%% Make sure the required packages are loaded in your preamble
%%   \usepackage{pgf}
%%
%% Figures using additional raster images can only be included by \input if
%% they are in the same directory as the main LaTeX file. For loading figures
%% from other directories you can use the `import` package
%%   \usepackage{import}
%% and then include the figures with
%%   \import{<path to file>}{<filename>.pgf}
%%
%% Matplotlib used the following preamble
%%
\begingroup%
\makeatletter%
\begin{pgfpicture}%
\pgfpathrectangle{\pgfpointorigin}{\pgfqpoint{4.747069in}{3.310123in}}%
\pgfusepath{use as bounding box, clip}%
\begin{pgfscope}%
\pgfsetbuttcap%
\pgfsetmiterjoin%
\definecolor{currentfill}{rgb}{1.000000,1.000000,1.000000}%
\pgfsetfillcolor{currentfill}%
\pgfsetlinewidth{0.000000pt}%
\definecolor{currentstroke}{rgb}{1.000000,1.000000,1.000000}%
\pgfsetstrokecolor{currentstroke}%
\pgfsetdash{}{0pt}%
\pgfpathmoveto{\pgfqpoint{0.000000in}{0.000000in}}%
\pgfpathlineto{\pgfqpoint{4.747069in}{0.000000in}}%
\pgfpathlineto{\pgfqpoint{4.747069in}{3.310123in}}%
\pgfpathlineto{\pgfqpoint{0.000000in}{3.310123in}}%
\pgfpathclose%
\pgfusepath{fill}%
\end{pgfscope}%
\begin{pgfscope}%
\pgfsetbuttcap%
\pgfsetmiterjoin%
\definecolor{currentfill}{rgb}{1.000000,1.000000,1.000000}%
\pgfsetfillcolor{currentfill}%
\pgfsetlinewidth{0.000000pt}%
\definecolor{currentstroke}{rgb}{0.000000,0.000000,0.000000}%
\pgfsetstrokecolor{currentstroke}%
\pgfsetstrokeopacity{0.000000}%
\pgfsetdash{}{0pt}%
\pgfpathmoveto{\pgfqpoint{0.772069in}{0.515123in}}%
\pgfpathlineto{\pgfqpoint{4.647069in}{0.515123in}}%
\pgfpathlineto{\pgfqpoint{4.647069in}{3.210123in}}%
\pgfpathlineto{\pgfqpoint{0.772069in}{3.210123in}}%
\pgfpathclose%
\pgfusepath{fill}%
\end{pgfscope}%
\begin{pgfscope}%
\pgfsetbuttcap%
\pgfsetroundjoin%
\definecolor{currentfill}{rgb}{0.000000,0.000000,0.000000}%
\pgfsetfillcolor{currentfill}%
\pgfsetlinewidth{0.803000pt}%
\definecolor{currentstroke}{rgb}{0.000000,0.000000,0.000000}%
\pgfsetstrokecolor{currentstroke}%
\pgfsetdash{}{0pt}%
\pgfsys@defobject{currentmarker}{\pgfqpoint{0.000000in}{-0.048611in}}{\pgfqpoint{0.000000in}{0.000000in}}{%
\pgfpathmoveto{\pgfqpoint{0.000000in}{0.000000in}}%
\pgfpathlineto{\pgfqpoint{0.000000in}{-0.048611in}}%
\pgfusepath{stroke,fill}%
}%
\begin{pgfscope}%
\pgfsys@transformshift{1.166609in}{0.515123in}%
\pgfsys@useobject{currentmarker}{}%
\end{pgfscope}%
\end{pgfscope}%
\begin{pgfscope}%
\definecolor{textcolor}{rgb}{0.000000,0.000000,0.000000}%
\pgfsetstrokecolor{textcolor}%
\pgfsetfillcolor{textcolor}%
\pgftext[x=1.166609in,y=0.417901in,,top]{\color{textcolor}\rmfamily\fontsize{10.000000}{12.000000}\selectfont \(\displaystyle 2\)}%
\end{pgfscope}%
\begin{pgfscope}%
\pgfsetbuttcap%
\pgfsetroundjoin%
\definecolor{currentfill}{rgb}{0.000000,0.000000,0.000000}%
\pgfsetfillcolor{currentfill}%
\pgfsetlinewidth{0.803000pt}%
\definecolor{currentstroke}{rgb}{0.000000,0.000000,0.000000}%
\pgfsetstrokecolor{currentstroke}%
\pgfsetdash{}{0pt}%
\pgfsys@defobject{currentmarker}{\pgfqpoint{0.000000in}{-0.048611in}}{\pgfqpoint{0.000000in}{0.000000in}}{%
\pgfpathmoveto{\pgfqpoint{0.000000in}{0.000000in}}%
\pgfpathlineto{\pgfqpoint{0.000000in}{-0.048611in}}%
\pgfusepath{stroke,fill}%
}%
\begin{pgfscope}%
\pgfsys@transformshift{1.783793in}{0.515123in}%
\pgfsys@useobject{currentmarker}{}%
\end{pgfscope}%
\end{pgfscope}%
\begin{pgfscope}%
\definecolor{textcolor}{rgb}{0.000000,0.000000,0.000000}%
\pgfsetstrokecolor{textcolor}%
\pgfsetfillcolor{textcolor}%
\pgftext[x=1.783793in,y=0.417901in,,top]{\color{textcolor}\rmfamily\fontsize{10.000000}{12.000000}\selectfont \(\displaystyle 4\)}%
\end{pgfscope}%
\begin{pgfscope}%
\pgfsetbuttcap%
\pgfsetroundjoin%
\definecolor{currentfill}{rgb}{0.000000,0.000000,0.000000}%
\pgfsetfillcolor{currentfill}%
\pgfsetlinewidth{0.803000pt}%
\definecolor{currentstroke}{rgb}{0.000000,0.000000,0.000000}%
\pgfsetstrokecolor{currentstroke}%
\pgfsetdash{}{0pt}%
\pgfsys@defobject{currentmarker}{\pgfqpoint{0.000000in}{-0.048611in}}{\pgfqpoint{0.000000in}{0.000000in}}{%
\pgfpathmoveto{\pgfqpoint{0.000000in}{0.000000in}}%
\pgfpathlineto{\pgfqpoint{0.000000in}{-0.048611in}}%
\pgfusepath{stroke,fill}%
}%
\begin{pgfscope}%
\pgfsys@transformshift{2.400977in}{0.515123in}%
\pgfsys@useobject{currentmarker}{}%
\end{pgfscope}%
\end{pgfscope}%
\begin{pgfscope}%
\definecolor{textcolor}{rgb}{0.000000,0.000000,0.000000}%
\pgfsetstrokecolor{textcolor}%
\pgfsetfillcolor{textcolor}%
\pgftext[x=2.400977in,y=0.417901in,,top]{\color{textcolor}\rmfamily\fontsize{10.000000}{12.000000}\selectfont \(\displaystyle 6\)}%
\end{pgfscope}%
\begin{pgfscope}%
\pgfsetbuttcap%
\pgfsetroundjoin%
\definecolor{currentfill}{rgb}{0.000000,0.000000,0.000000}%
\pgfsetfillcolor{currentfill}%
\pgfsetlinewidth{0.803000pt}%
\definecolor{currentstroke}{rgb}{0.000000,0.000000,0.000000}%
\pgfsetstrokecolor{currentstroke}%
\pgfsetdash{}{0pt}%
\pgfsys@defobject{currentmarker}{\pgfqpoint{0.000000in}{-0.048611in}}{\pgfqpoint{0.000000in}{0.000000in}}{%
\pgfpathmoveto{\pgfqpoint{0.000000in}{0.000000in}}%
\pgfpathlineto{\pgfqpoint{0.000000in}{-0.048611in}}%
\pgfusepath{stroke,fill}%
}%
\begin{pgfscope}%
\pgfsys@transformshift{3.018161in}{0.515123in}%
\pgfsys@useobject{currentmarker}{}%
\end{pgfscope}%
\end{pgfscope}%
\begin{pgfscope}%
\definecolor{textcolor}{rgb}{0.000000,0.000000,0.000000}%
\pgfsetstrokecolor{textcolor}%
\pgfsetfillcolor{textcolor}%
\pgftext[x=3.018161in,y=0.417901in,,top]{\color{textcolor}\rmfamily\fontsize{10.000000}{12.000000}\selectfont \(\displaystyle 8\)}%
\end{pgfscope}%
\begin{pgfscope}%
\pgfsetbuttcap%
\pgfsetroundjoin%
\definecolor{currentfill}{rgb}{0.000000,0.000000,0.000000}%
\pgfsetfillcolor{currentfill}%
\pgfsetlinewidth{0.803000pt}%
\definecolor{currentstroke}{rgb}{0.000000,0.000000,0.000000}%
\pgfsetstrokecolor{currentstroke}%
\pgfsetdash{}{0pt}%
\pgfsys@defobject{currentmarker}{\pgfqpoint{0.000000in}{-0.048611in}}{\pgfqpoint{0.000000in}{0.000000in}}{%
\pgfpathmoveto{\pgfqpoint{0.000000in}{0.000000in}}%
\pgfpathlineto{\pgfqpoint{0.000000in}{-0.048611in}}%
\pgfusepath{stroke,fill}%
}%
\begin{pgfscope}%
\pgfsys@transformshift{3.635345in}{0.515123in}%
\pgfsys@useobject{currentmarker}{}%
\end{pgfscope}%
\end{pgfscope}%
\begin{pgfscope}%
\definecolor{textcolor}{rgb}{0.000000,0.000000,0.000000}%
\pgfsetstrokecolor{textcolor}%
\pgfsetfillcolor{textcolor}%
\pgftext[x=3.635345in,y=0.417901in,,top]{\color{textcolor}\rmfamily\fontsize{10.000000}{12.000000}\selectfont \(\displaystyle 10\)}%
\end{pgfscope}%
\begin{pgfscope}%
\pgfsetbuttcap%
\pgfsetroundjoin%
\definecolor{currentfill}{rgb}{0.000000,0.000000,0.000000}%
\pgfsetfillcolor{currentfill}%
\pgfsetlinewidth{0.803000pt}%
\definecolor{currentstroke}{rgb}{0.000000,0.000000,0.000000}%
\pgfsetstrokecolor{currentstroke}%
\pgfsetdash{}{0pt}%
\pgfsys@defobject{currentmarker}{\pgfqpoint{0.000000in}{-0.048611in}}{\pgfqpoint{0.000000in}{0.000000in}}{%
\pgfpathmoveto{\pgfqpoint{0.000000in}{0.000000in}}%
\pgfpathlineto{\pgfqpoint{0.000000in}{-0.048611in}}%
\pgfusepath{stroke,fill}%
}%
\begin{pgfscope}%
\pgfsys@transformshift{4.252529in}{0.515123in}%
\pgfsys@useobject{currentmarker}{}%
\end{pgfscope}%
\end{pgfscope}%
\begin{pgfscope}%
\definecolor{textcolor}{rgb}{0.000000,0.000000,0.000000}%
\pgfsetstrokecolor{textcolor}%
\pgfsetfillcolor{textcolor}%
\pgftext[x=4.252529in,y=0.417901in,,top]{\color{textcolor}\rmfamily\fontsize{10.000000}{12.000000}\selectfont \(\displaystyle 12\)}%
\end{pgfscope}%
\begin{pgfscope}%
\definecolor{textcolor}{rgb}{0.000000,0.000000,0.000000}%
\pgfsetstrokecolor{textcolor}%
\pgfsetfillcolor{textcolor}%
\pgftext[x=2.709569in,y=0.238889in,,top]{\color{textcolor}\rmfamily\fontsize{10.000000}{12.000000}\selectfont I(\(\displaystyle \mu\)A)}%
\end{pgfscope}%
\begin{pgfscope}%
\pgfsetbuttcap%
\pgfsetroundjoin%
\definecolor{currentfill}{rgb}{0.000000,0.000000,0.000000}%
\pgfsetfillcolor{currentfill}%
\pgfsetlinewidth{0.803000pt}%
\definecolor{currentstroke}{rgb}{0.000000,0.000000,0.000000}%
\pgfsetstrokecolor{currentstroke}%
\pgfsetdash{}{0pt}%
\pgfsys@defobject{currentmarker}{\pgfqpoint{-0.048611in}{0.000000in}}{\pgfqpoint{0.000000in}{0.000000in}}{%
\pgfpathmoveto{\pgfqpoint{0.000000in}{0.000000in}}%
\pgfpathlineto{\pgfqpoint{-0.048611in}{0.000000in}}%
\pgfusepath{stroke,fill}%
}%
\begin{pgfscope}%
\pgfsys@transformshift{0.772069in}{0.894605in}%
\pgfsys@useobject{currentmarker}{}%
\end{pgfscope}%
\end{pgfscope}%
\begin{pgfscope}%
\definecolor{textcolor}{rgb}{0.000000,0.000000,0.000000}%
\pgfsetstrokecolor{textcolor}%
\pgfsetfillcolor{textcolor}%
\pgftext[x=0.605402in,y=0.846379in,left,base]{\color{textcolor}\rmfamily\fontsize{10.000000}{12.000000}\selectfont \(\displaystyle 2\)}%
\end{pgfscope}%
\begin{pgfscope}%
\pgfsetbuttcap%
\pgfsetroundjoin%
\definecolor{currentfill}{rgb}{0.000000,0.000000,0.000000}%
\pgfsetfillcolor{currentfill}%
\pgfsetlinewidth{0.803000pt}%
\definecolor{currentstroke}{rgb}{0.000000,0.000000,0.000000}%
\pgfsetstrokecolor{currentstroke}%
\pgfsetdash{}{0pt}%
\pgfsys@defobject{currentmarker}{\pgfqpoint{-0.048611in}{0.000000in}}{\pgfqpoint{0.000000in}{0.000000in}}{%
\pgfpathmoveto{\pgfqpoint{0.000000in}{0.000000in}}%
\pgfpathlineto{\pgfqpoint{-0.048611in}{0.000000in}}%
\pgfusepath{stroke,fill}%
}%
\begin{pgfscope}%
\pgfsys@transformshift{0.772069in}{1.436684in}%
\pgfsys@useobject{currentmarker}{}%
\end{pgfscope}%
\end{pgfscope}%
\begin{pgfscope}%
\definecolor{textcolor}{rgb}{0.000000,0.000000,0.000000}%
\pgfsetstrokecolor{textcolor}%
\pgfsetfillcolor{textcolor}%
\pgftext[x=0.605402in,y=1.388459in,left,base]{\color{textcolor}\rmfamily\fontsize{10.000000}{12.000000}\selectfont \(\displaystyle 4\)}%
\end{pgfscope}%
\begin{pgfscope}%
\pgfsetbuttcap%
\pgfsetroundjoin%
\definecolor{currentfill}{rgb}{0.000000,0.000000,0.000000}%
\pgfsetfillcolor{currentfill}%
\pgfsetlinewidth{0.803000pt}%
\definecolor{currentstroke}{rgb}{0.000000,0.000000,0.000000}%
\pgfsetstrokecolor{currentstroke}%
\pgfsetdash{}{0pt}%
\pgfsys@defobject{currentmarker}{\pgfqpoint{-0.048611in}{0.000000in}}{\pgfqpoint{0.000000in}{0.000000in}}{%
\pgfpathmoveto{\pgfqpoint{0.000000in}{0.000000in}}%
\pgfpathlineto{\pgfqpoint{-0.048611in}{0.000000in}}%
\pgfusepath{stroke,fill}%
}%
\begin{pgfscope}%
\pgfsys@transformshift{0.772069in}{1.978764in}%
\pgfsys@useobject{currentmarker}{}%
\end{pgfscope}%
\end{pgfscope}%
\begin{pgfscope}%
\definecolor{textcolor}{rgb}{0.000000,0.000000,0.000000}%
\pgfsetstrokecolor{textcolor}%
\pgfsetfillcolor{textcolor}%
\pgftext[x=0.605402in,y=1.930539in,left,base]{\color{textcolor}\rmfamily\fontsize{10.000000}{12.000000}\selectfont \(\displaystyle 6\)}%
\end{pgfscope}%
\begin{pgfscope}%
\pgfsetbuttcap%
\pgfsetroundjoin%
\definecolor{currentfill}{rgb}{0.000000,0.000000,0.000000}%
\pgfsetfillcolor{currentfill}%
\pgfsetlinewidth{0.803000pt}%
\definecolor{currentstroke}{rgb}{0.000000,0.000000,0.000000}%
\pgfsetstrokecolor{currentstroke}%
\pgfsetdash{}{0pt}%
\pgfsys@defobject{currentmarker}{\pgfqpoint{-0.048611in}{0.000000in}}{\pgfqpoint{0.000000in}{0.000000in}}{%
\pgfpathmoveto{\pgfqpoint{0.000000in}{0.000000in}}%
\pgfpathlineto{\pgfqpoint{-0.048611in}{0.000000in}}%
\pgfusepath{stroke,fill}%
}%
\begin{pgfscope}%
\pgfsys@transformshift{0.772069in}{2.520843in}%
\pgfsys@useobject{currentmarker}{}%
\end{pgfscope}%
\end{pgfscope}%
\begin{pgfscope}%
\definecolor{textcolor}{rgb}{0.000000,0.000000,0.000000}%
\pgfsetstrokecolor{textcolor}%
\pgfsetfillcolor{textcolor}%
\pgftext[x=0.605402in,y=2.472618in,left,base]{\color{textcolor}\rmfamily\fontsize{10.000000}{12.000000}\selectfont \(\displaystyle 8\)}%
\end{pgfscope}%
\begin{pgfscope}%
\pgfsetbuttcap%
\pgfsetroundjoin%
\definecolor{currentfill}{rgb}{0.000000,0.000000,0.000000}%
\pgfsetfillcolor{currentfill}%
\pgfsetlinewidth{0.803000pt}%
\definecolor{currentstroke}{rgb}{0.000000,0.000000,0.000000}%
\pgfsetstrokecolor{currentstroke}%
\pgfsetdash{}{0pt}%
\pgfsys@defobject{currentmarker}{\pgfqpoint{-0.048611in}{0.000000in}}{\pgfqpoint{0.000000in}{0.000000in}}{%
\pgfpathmoveto{\pgfqpoint{0.000000in}{0.000000in}}%
\pgfpathlineto{\pgfqpoint{-0.048611in}{0.000000in}}%
\pgfusepath{stroke,fill}%
}%
\begin{pgfscope}%
\pgfsys@transformshift{0.772069in}{3.062923in}%
\pgfsys@useobject{currentmarker}{}%
\end{pgfscope}%
\end{pgfscope}%
\begin{pgfscope}%
\definecolor{textcolor}{rgb}{0.000000,0.000000,0.000000}%
\pgfsetstrokecolor{textcolor}%
\pgfsetfillcolor{textcolor}%
\pgftext[x=0.535957in,y=3.014698in,left,base]{\color{textcolor}\rmfamily\fontsize{10.000000}{12.000000}\selectfont \(\displaystyle 10\)}%
\end{pgfscope}%
\begin{pgfscope}%
\definecolor{textcolor}{rgb}{0.000000,0.000000,0.000000}%
\pgfsetstrokecolor{textcolor}%
\pgfsetfillcolor{textcolor}%
\pgftext[x=0.258179in,y=1.862623in,,bottom]{\color{textcolor}\rmfamily\fontsize{10.000000}{12.000000}\selectfont V(V)}%
\end{pgfscope}%
\begin{pgfscope}%
\pgfpathrectangle{\pgfqpoint{0.772069in}{0.515123in}}{\pgfqpoint{3.875000in}{2.695000in}}%
\pgfusepath{clip}%
\pgfsetbuttcap%
\pgfsetroundjoin%
\definecolor{currentfill}{rgb}{0.121569,0.466667,0.705882}%
\pgfsetfillcolor{currentfill}%
\pgfsetlinewidth{1.003750pt}%
\definecolor{currentstroke}{rgb}{0.121569,0.466667,0.705882}%
\pgfsetstrokecolor{currentstroke}%
\pgfsetdash{}{0pt}%
\pgfpathmoveto{\pgfqpoint{0.950594in}{0.598974in}}%
\pgfpathcurveto{\pgfqpoint{0.961644in}{0.598974in}}{\pgfqpoint{0.972243in}{0.603364in}}{\pgfqpoint{0.980057in}{0.611178in}}%
\pgfpathcurveto{\pgfqpoint{0.987871in}{0.618991in}}{\pgfqpoint{0.992261in}{0.629590in}}{\pgfqpoint{0.992261in}{0.640640in}}%
\pgfpathcurveto{\pgfqpoint{0.992261in}{0.651691in}}{\pgfqpoint{0.987871in}{0.662290in}}{\pgfqpoint{0.980057in}{0.670103in}}%
\pgfpathcurveto{\pgfqpoint{0.972243in}{0.677917in}}{\pgfqpoint{0.961644in}{0.682307in}}{\pgfqpoint{0.950594in}{0.682307in}}%
\pgfpathcurveto{\pgfqpoint{0.939544in}{0.682307in}}{\pgfqpoint{0.928945in}{0.677917in}}{\pgfqpoint{0.921131in}{0.670103in}}%
\pgfpathcurveto{\pgfqpoint{0.913318in}{0.662290in}}{\pgfqpoint{0.908927in}{0.651691in}}{\pgfqpoint{0.908927in}{0.640640in}}%
\pgfpathcurveto{\pgfqpoint{0.908927in}{0.629590in}}{\pgfqpoint{0.913318in}{0.618991in}}{\pgfqpoint{0.921131in}{0.611178in}}%
\pgfpathcurveto{\pgfqpoint{0.928945in}{0.603364in}}{\pgfqpoint{0.939544in}{0.598974in}}{\pgfqpoint{0.950594in}{0.598974in}}%
\pgfpathclose%
\pgfusepath{stroke,fill}%
\end{pgfscope}%
\begin{pgfscope}%
\pgfpathrectangle{\pgfqpoint{0.772069in}{0.515123in}}{\pgfqpoint{3.875000in}{2.695000in}}%
\pgfusepath{clip}%
\pgfsetbuttcap%
\pgfsetroundjoin%
\definecolor{currentfill}{rgb}{0.121569,0.466667,0.705882}%
\pgfsetfillcolor{currentfill}%
\pgfsetlinewidth{1.003750pt}%
\definecolor{currentstroke}{rgb}{0.121569,0.466667,0.705882}%
\pgfsetstrokecolor{currentstroke}%
\pgfsetdash{}{0pt}%
\pgfpathmoveto{\pgfqpoint{1.320905in}{0.866490in}}%
\pgfpathcurveto{\pgfqpoint{1.331955in}{0.866490in}}{\pgfqpoint{1.342554in}{0.870880in}}{\pgfqpoint{1.350367in}{0.878694in}}%
\pgfpathcurveto{\pgfqpoint{1.358181in}{0.886508in}}{\pgfqpoint{1.362571in}{0.897107in}}{\pgfqpoint{1.362571in}{0.908157in}}%
\pgfpathcurveto{\pgfqpoint{1.362571in}{0.919207in}}{\pgfqpoint{1.358181in}{0.929806in}}{\pgfqpoint{1.350367in}{0.937620in}}%
\pgfpathcurveto{\pgfqpoint{1.342554in}{0.945433in}}{\pgfqpoint{1.331955in}{0.949823in}}{\pgfqpoint{1.320905in}{0.949823in}}%
\pgfpathcurveto{\pgfqpoint{1.309854in}{0.949823in}}{\pgfqpoint{1.299255in}{0.945433in}}{\pgfqpoint{1.291442in}{0.937620in}}%
\pgfpathcurveto{\pgfqpoint{1.283628in}{0.929806in}}{\pgfqpoint{1.279238in}{0.919207in}}{\pgfqpoint{1.279238in}{0.908157in}}%
\pgfpathcurveto{\pgfqpoint{1.279238in}{0.897107in}}{\pgfqpoint{1.283628in}{0.886508in}}{\pgfqpoint{1.291442in}{0.878694in}}%
\pgfpathcurveto{\pgfqpoint{1.299255in}{0.870880in}}{\pgfqpoint{1.309854in}{0.866490in}}{\pgfqpoint{1.320905in}{0.866490in}}%
\pgfpathclose%
\pgfusepath{stroke,fill}%
\end{pgfscope}%
\begin{pgfscope}%
\pgfpathrectangle{\pgfqpoint{0.772069in}{0.515123in}}{\pgfqpoint{3.875000in}{2.695000in}}%
\pgfusepath{clip}%
\pgfsetbuttcap%
\pgfsetroundjoin%
\definecolor{currentfill}{rgb}{0.121569,0.466667,0.705882}%
\pgfsetfillcolor{currentfill}%
\pgfsetlinewidth{1.003750pt}%
\definecolor{currentstroke}{rgb}{0.121569,0.466667,0.705882}%
\pgfsetstrokecolor{currentstroke}%
\pgfsetdash{}{0pt}%
\pgfpathmoveto{\pgfqpoint{1.722074in}{1.134819in}}%
\pgfpathcurveto{\pgfqpoint{1.733124in}{1.134819in}}{\pgfqpoint{1.743723in}{1.139210in}}{\pgfqpoint{1.751537in}{1.147023in}}%
\pgfpathcurveto{\pgfqpoint{1.759351in}{1.154837in}}{\pgfqpoint{1.763741in}{1.165436in}}{\pgfqpoint{1.763741in}{1.176486in}}%
\pgfpathcurveto{\pgfqpoint{1.763741in}{1.187536in}}{\pgfqpoint{1.759351in}{1.198135in}}{\pgfqpoint{1.751537in}{1.205949in}}%
\pgfpathcurveto{\pgfqpoint{1.743723in}{1.213763in}}{\pgfqpoint{1.733124in}{1.218153in}}{\pgfqpoint{1.722074in}{1.218153in}}%
\pgfpathcurveto{\pgfqpoint{1.711024in}{1.218153in}}{\pgfqpoint{1.700425in}{1.213763in}}{\pgfqpoint{1.692611in}{1.205949in}}%
\pgfpathcurveto{\pgfqpoint{1.684798in}{1.198135in}}{\pgfqpoint{1.680408in}{1.187536in}}{\pgfqpoint{1.680408in}{1.176486in}}%
\pgfpathcurveto{\pgfqpoint{1.680408in}{1.165436in}}{\pgfqpoint{1.684798in}{1.154837in}}{\pgfqpoint{1.692611in}{1.147023in}}%
\pgfpathcurveto{\pgfqpoint{1.700425in}{1.139210in}}{\pgfqpoint{1.711024in}{1.134819in}}{\pgfqpoint{1.722074in}{1.134819in}}%
\pgfpathclose%
\pgfusepath{stroke,fill}%
\end{pgfscope}%
\begin{pgfscope}%
\pgfpathrectangle{\pgfqpoint{0.772069in}{0.515123in}}{\pgfqpoint{3.875000in}{2.695000in}}%
\pgfusepath{clip}%
\pgfsetbuttcap%
\pgfsetroundjoin%
\definecolor{currentfill}{rgb}{0.121569,0.466667,0.705882}%
\pgfsetfillcolor{currentfill}%
\pgfsetlinewidth{1.003750pt}%
\definecolor{currentstroke}{rgb}{0.121569,0.466667,0.705882}%
\pgfsetstrokecolor{currentstroke}%
\pgfsetdash{}{0pt}%
\pgfpathmoveto{\pgfqpoint{2.123244in}{1.411280in}}%
\pgfpathcurveto{\pgfqpoint{2.134294in}{1.411280in}}{\pgfqpoint{2.144893in}{1.415670in}}{\pgfqpoint{2.152707in}{1.423484in}}%
\pgfpathcurveto{\pgfqpoint{2.160520in}{1.431298in}}{\pgfqpoint{2.164911in}{1.441897in}}{\pgfqpoint{2.164911in}{1.452947in}}%
\pgfpathcurveto{\pgfqpoint{2.164911in}{1.463997in}}{\pgfqpoint{2.160520in}{1.474596in}}{\pgfqpoint{2.152707in}{1.482409in}}%
\pgfpathcurveto{\pgfqpoint{2.144893in}{1.490223in}}{\pgfqpoint{2.134294in}{1.494613in}}{\pgfqpoint{2.123244in}{1.494613in}}%
\pgfpathcurveto{\pgfqpoint{2.112194in}{1.494613in}}{\pgfqpoint{2.101595in}{1.490223in}}{\pgfqpoint{2.093781in}{1.482409in}}%
\pgfpathcurveto{\pgfqpoint{2.085967in}{1.474596in}}{\pgfqpoint{2.081577in}{1.463997in}}{\pgfqpoint{2.081577in}{1.452947in}}%
\pgfpathcurveto{\pgfqpoint{2.081577in}{1.441897in}}{\pgfqpoint{2.085967in}{1.431298in}}{\pgfqpoint{2.093781in}{1.423484in}}%
\pgfpathcurveto{\pgfqpoint{2.101595in}{1.415670in}}{\pgfqpoint{2.112194in}{1.411280in}}{\pgfqpoint{2.123244in}{1.411280in}}%
\pgfpathclose%
\pgfusepath{stroke,fill}%
\end{pgfscope}%
\begin{pgfscope}%
\pgfpathrectangle{\pgfqpoint{0.772069in}{0.515123in}}{\pgfqpoint{3.875000in}{2.695000in}}%
\pgfusepath{clip}%
\pgfsetbuttcap%
\pgfsetroundjoin%
\definecolor{currentfill}{rgb}{0.121569,0.466667,0.705882}%
\pgfsetfillcolor{currentfill}%
\pgfsetlinewidth{1.003750pt}%
\definecolor{currentstroke}{rgb}{0.121569,0.466667,0.705882}%
\pgfsetstrokecolor{currentstroke}%
\pgfsetdash{}{0pt}%
\pgfpathmoveto{\pgfqpoint{2.493554in}{1.676899in}}%
\pgfpathcurveto{\pgfqpoint{2.504604in}{1.676899in}}{\pgfqpoint{2.515203in}{1.681289in}}{\pgfqpoint{2.523017in}{1.689103in}}%
\pgfpathcurveto{\pgfqpoint{2.530831in}{1.696917in}}{\pgfqpoint{2.535221in}{1.707516in}}{\pgfqpoint{2.535221in}{1.718566in}}%
\pgfpathcurveto{\pgfqpoint{2.535221in}{1.729616in}}{\pgfqpoint{2.530831in}{1.740215in}}{\pgfqpoint{2.523017in}{1.748028in}}%
\pgfpathcurveto{\pgfqpoint{2.515203in}{1.755842in}}{\pgfqpoint{2.504604in}{1.760232in}}{\pgfqpoint{2.493554in}{1.760232in}}%
\pgfpathcurveto{\pgfqpoint{2.482504in}{1.760232in}}{\pgfqpoint{2.471905in}{1.755842in}}{\pgfqpoint{2.464091in}{1.748028in}}%
\pgfpathcurveto{\pgfqpoint{2.456278in}{1.740215in}}{\pgfqpoint{2.451888in}{1.729616in}}{\pgfqpoint{2.451888in}{1.718566in}}%
\pgfpathcurveto{\pgfqpoint{2.451888in}{1.707516in}}{\pgfqpoint{2.456278in}{1.696917in}}{\pgfqpoint{2.464091in}{1.689103in}}%
\pgfpathcurveto{\pgfqpoint{2.471905in}{1.681289in}}{\pgfqpoint{2.482504in}{1.676899in}}{\pgfqpoint{2.493554in}{1.676899in}}%
\pgfpathclose%
\pgfusepath{stroke,fill}%
\end{pgfscope}%
\begin{pgfscope}%
\pgfpathrectangle{\pgfqpoint{0.772069in}{0.515123in}}{\pgfqpoint{3.875000in}{2.695000in}}%
\pgfusepath{clip}%
\pgfsetbuttcap%
\pgfsetroundjoin%
\definecolor{currentfill}{rgb}{0.121569,0.466667,0.705882}%
\pgfsetfillcolor{currentfill}%
\pgfsetlinewidth{1.003750pt}%
\definecolor{currentstroke}{rgb}{0.121569,0.466667,0.705882}%
\pgfsetstrokecolor{currentstroke}%
\pgfsetdash{}{0pt}%
\pgfpathmoveto{\pgfqpoint{2.925583in}{1.969622in}}%
\pgfpathcurveto{\pgfqpoint{2.936633in}{1.969622in}}{\pgfqpoint{2.947232in}{1.974012in}}{\pgfqpoint{2.955046in}{1.981826in}}%
\pgfpathcurveto{\pgfqpoint{2.962860in}{1.989639in}}{\pgfqpoint{2.967250in}{2.000239in}}{\pgfqpoint{2.967250in}{2.011289in}}%
\pgfpathcurveto{\pgfqpoint{2.967250in}{2.022339in}}{\pgfqpoint{2.962860in}{2.032938in}}{\pgfqpoint{2.955046in}{2.040751in}}%
\pgfpathcurveto{\pgfqpoint{2.947232in}{2.048565in}}{\pgfqpoint{2.936633in}{2.052955in}}{\pgfqpoint{2.925583in}{2.052955in}}%
\pgfpathcurveto{\pgfqpoint{2.914533in}{2.052955in}}{\pgfqpoint{2.903934in}{2.048565in}}{\pgfqpoint{2.896120in}{2.040751in}}%
\pgfpathcurveto{\pgfqpoint{2.888307in}{2.032938in}}{\pgfqpoint{2.883916in}{2.022339in}}{\pgfqpoint{2.883916in}{2.011289in}}%
\pgfpathcurveto{\pgfqpoint{2.883916in}{2.000239in}}{\pgfqpoint{2.888307in}{1.989639in}}{\pgfqpoint{2.896120in}{1.981826in}}%
\pgfpathcurveto{\pgfqpoint{2.903934in}{1.974012in}}{\pgfqpoint{2.914533in}{1.969622in}}{\pgfqpoint{2.925583in}{1.969622in}}%
\pgfpathclose%
\pgfusepath{stroke,fill}%
\end{pgfscope}%
\begin{pgfscope}%
\pgfpathrectangle{\pgfqpoint{0.772069in}{0.515123in}}{\pgfqpoint{3.875000in}{2.695000in}}%
\pgfusepath{clip}%
\pgfsetbuttcap%
\pgfsetroundjoin%
\definecolor{currentfill}{rgb}{0.121569,0.466667,0.705882}%
\pgfsetfillcolor{currentfill}%
\pgfsetlinewidth{1.003750pt}%
\definecolor{currentstroke}{rgb}{0.121569,0.466667,0.705882}%
\pgfsetstrokecolor{currentstroke}%
\pgfsetdash{}{0pt}%
\pgfpathmoveto{\pgfqpoint{3.295894in}{2.229820in}}%
\pgfpathcurveto{\pgfqpoint{3.306944in}{2.229820in}}{\pgfqpoint{3.317543in}{2.234210in}}{\pgfqpoint{3.325356in}{2.242024in}}%
\pgfpathcurveto{\pgfqpoint{3.333170in}{2.249838in}}{\pgfqpoint{3.337560in}{2.260437in}}{\pgfqpoint{3.337560in}{2.271487in}}%
\pgfpathcurveto{\pgfqpoint{3.337560in}{2.282537in}}{\pgfqpoint{3.333170in}{2.293136in}}{\pgfqpoint{3.325356in}{2.300950in}}%
\pgfpathcurveto{\pgfqpoint{3.317543in}{2.308763in}}{\pgfqpoint{3.306944in}{2.313154in}}{\pgfqpoint{3.295894in}{2.313154in}}%
\pgfpathcurveto{\pgfqpoint{3.284843in}{2.313154in}}{\pgfqpoint{3.274244in}{2.308763in}}{\pgfqpoint{3.266431in}{2.300950in}}%
\pgfpathcurveto{\pgfqpoint{3.258617in}{2.293136in}}{\pgfqpoint{3.254227in}{2.282537in}}{\pgfqpoint{3.254227in}{2.271487in}}%
\pgfpathcurveto{\pgfqpoint{3.254227in}{2.260437in}}{\pgfqpoint{3.258617in}{2.249838in}}{\pgfqpoint{3.266431in}{2.242024in}}%
\pgfpathcurveto{\pgfqpoint{3.274244in}{2.234210in}}{\pgfqpoint{3.284843in}{2.229820in}}{\pgfqpoint{3.295894in}{2.229820in}}%
\pgfpathclose%
\pgfusepath{stroke,fill}%
\end{pgfscope}%
\begin{pgfscope}%
\pgfpathrectangle{\pgfqpoint{0.772069in}{0.515123in}}{\pgfqpoint{3.875000in}{2.695000in}}%
\pgfusepath{clip}%
\pgfsetbuttcap%
\pgfsetroundjoin%
\definecolor{currentfill}{rgb}{0.121569,0.466667,0.705882}%
\pgfsetfillcolor{currentfill}%
\pgfsetlinewidth{1.003750pt}%
\definecolor{currentstroke}{rgb}{0.121569,0.466667,0.705882}%
\pgfsetstrokecolor{currentstroke}%
\pgfsetdash{}{0pt}%
\pgfpathmoveto{\pgfqpoint{3.697063in}{2.498150in}}%
\pgfpathcurveto{\pgfqpoint{3.708113in}{2.498150in}}{\pgfqpoint{3.718712in}{2.502540in}}{\pgfqpoint{3.726526in}{2.510353in}}%
\pgfpathcurveto{\pgfqpoint{3.734340in}{2.518167in}}{\pgfqpoint{3.738730in}{2.528766in}}{\pgfqpoint{3.738730in}{2.539816in}}%
\pgfpathcurveto{\pgfqpoint{3.738730in}{2.550866in}}{\pgfqpoint{3.734340in}{2.561465in}}{\pgfqpoint{3.726526in}{2.569279in}}%
\pgfpathcurveto{\pgfqpoint{3.718712in}{2.577093in}}{\pgfqpoint{3.708113in}{2.581483in}}{\pgfqpoint{3.697063in}{2.581483in}}%
\pgfpathcurveto{\pgfqpoint{3.686013in}{2.581483in}}{\pgfqpoint{3.675414in}{2.577093in}}{\pgfqpoint{3.667600in}{2.569279in}}%
\pgfpathcurveto{\pgfqpoint{3.659787in}{2.561465in}}{\pgfqpoint{3.655396in}{2.550866in}}{\pgfqpoint{3.655396in}{2.539816in}}%
\pgfpathcurveto{\pgfqpoint{3.655396in}{2.528766in}}{\pgfqpoint{3.659787in}{2.518167in}}{\pgfqpoint{3.667600in}{2.510353in}}%
\pgfpathcurveto{\pgfqpoint{3.675414in}{2.502540in}}{\pgfqpoint{3.686013in}{2.498150in}}{\pgfqpoint{3.697063in}{2.498150in}}%
\pgfpathclose%
\pgfusepath{stroke,fill}%
\end{pgfscope}%
\begin{pgfscope}%
\pgfpathrectangle{\pgfqpoint{0.772069in}{0.515123in}}{\pgfqpoint{3.875000in}{2.695000in}}%
\pgfusepath{clip}%
\pgfsetbuttcap%
\pgfsetroundjoin%
\definecolor{currentfill}{rgb}{0.121569,0.466667,0.705882}%
\pgfsetfillcolor{currentfill}%
\pgfsetlinewidth{1.003750pt}%
\definecolor{currentstroke}{rgb}{0.121569,0.466667,0.705882}%
\pgfsetstrokecolor{currentstroke}%
\pgfsetdash{}{0pt}%
\pgfpathmoveto{\pgfqpoint{4.036514in}{2.750217in}}%
\pgfpathcurveto{\pgfqpoint{4.047565in}{2.750217in}}{\pgfqpoint{4.058164in}{2.754607in}}{\pgfqpoint{4.065977in}{2.762420in}}%
\pgfpathcurveto{\pgfqpoint{4.073791in}{2.770234in}}{\pgfqpoint{4.078181in}{2.780833in}}{\pgfqpoint{4.078181in}{2.791883in}}%
\pgfpathcurveto{\pgfqpoint{4.078181in}{2.802933in}}{\pgfqpoint{4.073791in}{2.813532in}}{\pgfqpoint{4.065977in}{2.821346in}}%
\pgfpathcurveto{\pgfqpoint{4.058164in}{2.829160in}}{\pgfqpoint{4.047565in}{2.833550in}}{\pgfqpoint{4.036514in}{2.833550in}}%
\pgfpathcurveto{\pgfqpoint{4.025464in}{2.833550in}}{\pgfqpoint{4.014865in}{2.829160in}}{\pgfqpoint{4.007052in}{2.821346in}}%
\pgfpathcurveto{\pgfqpoint{3.999238in}{2.813532in}}{\pgfqpoint{3.994848in}{2.802933in}}{\pgfqpoint{3.994848in}{2.791883in}}%
\pgfpathcurveto{\pgfqpoint{3.994848in}{2.780833in}}{\pgfqpoint{3.999238in}{2.770234in}}{\pgfqpoint{4.007052in}{2.762420in}}%
\pgfpathcurveto{\pgfqpoint{4.014865in}{2.754607in}}{\pgfqpoint{4.025464in}{2.750217in}}{\pgfqpoint{4.036514in}{2.750217in}}%
\pgfpathclose%
\pgfusepath{stroke,fill}%
\end{pgfscope}%
\begin{pgfscope}%
\pgfpathrectangle{\pgfqpoint{0.772069in}{0.515123in}}{\pgfqpoint{3.875000in}{2.695000in}}%
\pgfusepath{clip}%
\pgfsetbuttcap%
\pgfsetroundjoin%
\definecolor{currentfill}{rgb}{0.121569,0.466667,0.705882}%
\pgfsetfillcolor{currentfill}%
\pgfsetlinewidth{1.003750pt}%
\definecolor{currentstroke}{rgb}{0.121569,0.466667,0.705882}%
\pgfsetstrokecolor{currentstroke}%
\pgfsetdash{}{0pt}%
\pgfpathmoveto{\pgfqpoint{4.468543in}{3.042940in}}%
\pgfpathcurveto{\pgfqpoint{4.479593in}{3.042940in}}{\pgfqpoint{4.490192in}{3.047330in}}{\pgfqpoint{4.498006in}{3.055143in}}%
\pgfpathcurveto{\pgfqpoint{4.505820in}{3.062957in}}{\pgfqpoint{4.510210in}{3.073556in}}{\pgfqpoint{4.510210in}{3.084606in}}%
\pgfpathcurveto{\pgfqpoint{4.510210in}{3.095656in}}{\pgfqpoint{4.505820in}{3.106255in}}{\pgfqpoint{4.498006in}{3.114069in}}%
\pgfpathcurveto{\pgfqpoint{4.490192in}{3.121883in}}{\pgfqpoint{4.479593in}{3.126273in}}{\pgfqpoint{4.468543in}{3.126273in}}%
\pgfpathcurveto{\pgfqpoint{4.457493in}{3.126273in}}{\pgfqpoint{4.446894in}{3.121883in}}{\pgfqpoint{4.439080in}{3.114069in}}%
\pgfpathcurveto{\pgfqpoint{4.431267in}{3.106255in}}{\pgfqpoint{4.426877in}{3.095656in}}{\pgfqpoint{4.426877in}{3.084606in}}%
\pgfpathcurveto{\pgfqpoint{4.426877in}{3.073556in}}{\pgfqpoint{4.431267in}{3.062957in}}{\pgfqpoint{4.439080in}{3.055143in}}%
\pgfpathcurveto{\pgfqpoint{4.446894in}{3.047330in}}{\pgfqpoint{4.457493in}{3.042940in}}{\pgfqpoint{4.468543in}{3.042940in}}%
\pgfpathclose%
\pgfusepath{stroke,fill}%
\end{pgfscope}%
\begin{pgfscope}%
\pgfsetrectcap%
\pgfsetmiterjoin%
\pgfsetlinewidth{0.803000pt}%
\definecolor{currentstroke}{rgb}{0.000000,0.000000,0.000000}%
\pgfsetstrokecolor{currentstroke}%
\pgfsetdash{}{0pt}%
\pgfpathmoveto{\pgfqpoint{0.772069in}{0.515123in}}%
\pgfpathlineto{\pgfqpoint{0.772069in}{3.210123in}}%
\pgfusepath{stroke}%
\end{pgfscope}%
\begin{pgfscope}%
\pgfsetrectcap%
\pgfsetmiterjoin%
\pgfsetlinewidth{0.803000pt}%
\definecolor{currentstroke}{rgb}{0.000000,0.000000,0.000000}%
\pgfsetstrokecolor{currentstroke}%
\pgfsetdash{}{0pt}%
\pgfpathmoveto{\pgfqpoint{4.647069in}{0.515123in}}%
\pgfpathlineto{\pgfqpoint{4.647069in}{3.210123in}}%
\pgfusepath{stroke}%
\end{pgfscope}%
\begin{pgfscope}%
\pgfsetrectcap%
\pgfsetmiterjoin%
\pgfsetlinewidth{0.803000pt}%
\definecolor{currentstroke}{rgb}{0.000000,0.000000,0.000000}%
\pgfsetstrokecolor{currentstroke}%
\pgfsetdash{}{0pt}%
\pgfpathmoveto{\pgfqpoint{0.772069in}{0.515123in}}%
\pgfpathlineto{\pgfqpoint{4.647069in}{0.515123in}}%
\pgfusepath{stroke}%
\end{pgfscope}%
\begin{pgfscope}%
\pgfsetrectcap%
\pgfsetmiterjoin%
\pgfsetlinewidth{0.803000pt}%
\definecolor{currentstroke}{rgb}{0.000000,0.000000,0.000000}%
\pgfsetstrokecolor{currentstroke}%
\pgfsetdash{}{0pt}%
\pgfpathmoveto{\pgfqpoint{0.772069in}{3.210123in}}%
\pgfpathlineto{\pgfqpoint{4.647069in}{3.210123in}}%
\pgfusepath{stroke}%
\end{pgfscope}%
\end{pgfpicture}%
\makeatother%
\endgroup%

    \caption{Voltaje (V) frente a intensidad (I)}
  \end{figure}

  También podemos comparar la diferencia entre los potenciales de las distintas resistencias en serie añadiendo las gráficas de $V_1$, $V_2$ y $V_3$ frente a I.

  \begin{figure}[H]
    %\centering
    \hspace{2.5em} %% Creator: Matplotlib, PGF backend
%%
%% To include the figure in your LaTeX document, write
%%   \input{<filename>.pgf}
%%
%% Make sure the required packages are loaded in your preamble
%%   \usepackage{pgf}
%%
%% Figures using additional raster images can only be included by \input if
%% they are in the same directory as the main LaTeX file. For loading figures
%% from other directories you can use the `import` package
%%   \usepackage{import}
%% and then include the figures with
%%   \import{<path to file>}{<filename>.pgf}
%%
%% Matplotlib used the following preamble
%%
\begingroup%
\makeatletter%
\begin{pgfpicture}%
\pgfpathrectangle{\pgfpointorigin}{\pgfqpoint{4.747069in}{3.310123in}}%
\pgfusepath{use as bounding box, clip}%
\begin{pgfscope}%
\pgfsetbuttcap%
\pgfsetmiterjoin%
\definecolor{currentfill}{rgb}{1.000000,1.000000,1.000000}%
\pgfsetfillcolor{currentfill}%
\pgfsetlinewidth{0.000000pt}%
\definecolor{currentstroke}{rgb}{1.000000,1.000000,1.000000}%
\pgfsetstrokecolor{currentstroke}%
\pgfsetdash{}{0pt}%
\pgfpathmoveto{\pgfqpoint{0.000000in}{0.000000in}}%
\pgfpathlineto{\pgfqpoint{4.747069in}{0.000000in}}%
\pgfpathlineto{\pgfqpoint{4.747069in}{3.310123in}}%
\pgfpathlineto{\pgfqpoint{0.000000in}{3.310123in}}%
\pgfpathclose%
\pgfusepath{fill}%
\end{pgfscope}%
\begin{pgfscope}%
\pgfsetbuttcap%
\pgfsetmiterjoin%
\definecolor{currentfill}{rgb}{1.000000,1.000000,1.000000}%
\pgfsetfillcolor{currentfill}%
\pgfsetlinewidth{0.000000pt}%
\definecolor{currentstroke}{rgb}{0.000000,0.000000,0.000000}%
\pgfsetstrokecolor{currentstroke}%
\pgfsetstrokeopacity{0.000000}%
\pgfsetdash{}{0pt}%
\pgfpathmoveto{\pgfqpoint{0.772069in}{0.515123in}}%
\pgfpathlineto{\pgfqpoint{4.647069in}{0.515123in}}%
\pgfpathlineto{\pgfqpoint{4.647069in}{3.210123in}}%
\pgfpathlineto{\pgfqpoint{0.772069in}{3.210123in}}%
\pgfpathclose%
\pgfusepath{fill}%
\end{pgfscope}%
\begin{pgfscope}%
\pgfsetbuttcap%
\pgfsetroundjoin%
\definecolor{currentfill}{rgb}{0.000000,0.000000,0.000000}%
\pgfsetfillcolor{currentfill}%
\pgfsetlinewidth{0.803000pt}%
\definecolor{currentstroke}{rgb}{0.000000,0.000000,0.000000}%
\pgfsetstrokecolor{currentstroke}%
\pgfsetdash{}{0pt}%
\pgfsys@defobject{currentmarker}{\pgfqpoint{0.000000in}{-0.048611in}}{\pgfqpoint{0.000000in}{0.000000in}}{%
\pgfpathmoveto{\pgfqpoint{0.000000in}{0.000000in}}%
\pgfpathlineto{\pgfqpoint{0.000000in}{-0.048611in}}%
\pgfusepath{stroke,fill}%
}%
\begin{pgfscope}%
\pgfsys@transformshift{1.190829in}{0.515123in}%
\pgfsys@useobject{currentmarker}{}%
\end{pgfscope}%
\end{pgfscope}%
\begin{pgfscope}%
\definecolor{textcolor}{rgb}{0.000000,0.000000,0.000000}%
\pgfsetstrokecolor{textcolor}%
\pgfsetfillcolor{textcolor}%
\pgftext[x=1.190829in,y=0.417901in,,top]{\color{textcolor}\rmfamily\fontsize{10.000000}{12.000000}\selectfont \(\displaystyle 2\)}%
\end{pgfscope}%
\begin{pgfscope}%
\pgfsetbuttcap%
\pgfsetroundjoin%
\definecolor{currentfill}{rgb}{0.000000,0.000000,0.000000}%
\pgfsetfillcolor{currentfill}%
\pgfsetlinewidth{0.803000pt}%
\definecolor{currentstroke}{rgb}{0.000000,0.000000,0.000000}%
\pgfsetstrokecolor{currentstroke}%
\pgfsetdash{}{0pt}%
\pgfsys@defobject{currentmarker}{\pgfqpoint{0.000000in}{-0.048611in}}{\pgfqpoint{0.000000in}{0.000000in}}{%
\pgfpathmoveto{\pgfqpoint{0.000000in}{0.000000in}}%
\pgfpathlineto{\pgfqpoint{0.000000in}{-0.048611in}}%
\pgfusepath{stroke,fill}%
}%
\begin{pgfscope}%
\pgfsys@transformshift{1.798325in}{0.515123in}%
\pgfsys@useobject{currentmarker}{}%
\end{pgfscope}%
\end{pgfscope}%
\begin{pgfscope}%
\definecolor{textcolor}{rgb}{0.000000,0.000000,0.000000}%
\pgfsetstrokecolor{textcolor}%
\pgfsetfillcolor{textcolor}%
\pgftext[x=1.798325in,y=0.417901in,,top]{\color{textcolor}\rmfamily\fontsize{10.000000}{12.000000}\selectfont \(\displaystyle 4\)}%
\end{pgfscope}%
\begin{pgfscope}%
\pgfsetbuttcap%
\pgfsetroundjoin%
\definecolor{currentfill}{rgb}{0.000000,0.000000,0.000000}%
\pgfsetfillcolor{currentfill}%
\pgfsetlinewidth{0.803000pt}%
\definecolor{currentstroke}{rgb}{0.000000,0.000000,0.000000}%
\pgfsetstrokecolor{currentstroke}%
\pgfsetdash{}{0pt}%
\pgfsys@defobject{currentmarker}{\pgfqpoint{0.000000in}{-0.048611in}}{\pgfqpoint{0.000000in}{0.000000in}}{%
\pgfpathmoveto{\pgfqpoint{0.000000in}{0.000000in}}%
\pgfpathlineto{\pgfqpoint{0.000000in}{-0.048611in}}%
\pgfusepath{stroke,fill}%
}%
\begin{pgfscope}%
\pgfsys@transformshift{2.405821in}{0.515123in}%
\pgfsys@useobject{currentmarker}{}%
\end{pgfscope}%
\end{pgfscope}%
\begin{pgfscope}%
\definecolor{textcolor}{rgb}{0.000000,0.000000,0.000000}%
\pgfsetstrokecolor{textcolor}%
\pgfsetfillcolor{textcolor}%
\pgftext[x=2.405821in,y=0.417901in,,top]{\color{textcolor}\rmfamily\fontsize{10.000000}{12.000000}\selectfont \(\displaystyle 6\)}%
\end{pgfscope}%
\begin{pgfscope}%
\pgfsetbuttcap%
\pgfsetroundjoin%
\definecolor{currentfill}{rgb}{0.000000,0.000000,0.000000}%
\pgfsetfillcolor{currentfill}%
\pgfsetlinewidth{0.803000pt}%
\definecolor{currentstroke}{rgb}{0.000000,0.000000,0.000000}%
\pgfsetstrokecolor{currentstroke}%
\pgfsetdash{}{0pt}%
\pgfsys@defobject{currentmarker}{\pgfqpoint{0.000000in}{-0.048611in}}{\pgfqpoint{0.000000in}{0.000000in}}{%
\pgfpathmoveto{\pgfqpoint{0.000000in}{0.000000in}}%
\pgfpathlineto{\pgfqpoint{0.000000in}{-0.048611in}}%
\pgfusepath{stroke,fill}%
}%
\begin{pgfscope}%
\pgfsys@transformshift{3.013317in}{0.515123in}%
\pgfsys@useobject{currentmarker}{}%
\end{pgfscope}%
\end{pgfscope}%
\begin{pgfscope}%
\definecolor{textcolor}{rgb}{0.000000,0.000000,0.000000}%
\pgfsetstrokecolor{textcolor}%
\pgfsetfillcolor{textcolor}%
\pgftext[x=3.013317in,y=0.417901in,,top]{\color{textcolor}\rmfamily\fontsize{10.000000}{12.000000}\selectfont \(\displaystyle 8\)}%
\end{pgfscope}%
\begin{pgfscope}%
\pgfsetbuttcap%
\pgfsetroundjoin%
\definecolor{currentfill}{rgb}{0.000000,0.000000,0.000000}%
\pgfsetfillcolor{currentfill}%
\pgfsetlinewidth{0.803000pt}%
\definecolor{currentstroke}{rgb}{0.000000,0.000000,0.000000}%
\pgfsetstrokecolor{currentstroke}%
\pgfsetdash{}{0pt}%
\pgfsys@defobject{currentmarker}{\pgfqpoint{0.000000in}{-0.048611in}}{\pgfqpoint{0.000000in}{0.000000in}}{%
\pgfpathmoveto{\pgfqpoint{0.000000in}{0.000000in}}%
\pgfpathlineto{\pgfqpoint{0.000000in}{-0.048611in}}%
\pgfusepath{stroke,fill}%
}%
\begin{pgfscope}%
\pgfsys@transformshift{3.620813in}{0.515123in}%
\pgfsys@useobject{currentmarker}{}%
\end{pgfscope}%
\end{pgfscope}%
\begin{pgfscope}%
\definecolor{textcolor}{rgb}{0.000000,0.000000,0.000000}%
\pgfsetstrokecolor{textcolor}%
\pgfsetfillcolor{textcolor}%
\pgftext[x=3.620813in,y=0.417901in,,top]{\color{textcolor}\rmfamily\fontsize{10.000000}{12.000000}\selectfont \(\displaystyle 10\)}%
\end{pgfscope}%
\begin{pgfscope}%
\pgfsetbuttcap%
\pgfsetroundjoin%
\definecolor{currentfill}{rgb}{0.000000,0.000000,0.000000}%
\pgfsetfillcolor{currentfill}%
\pgfsetlinewidth{0.803000pt}%
\definecolor{currentstroke}{rgb}{0.000000,0.000000,0.000000}%
\pgfsetstrokecolor{currentstroke}%
\pgfsetdash{}{0pt}%
\pgfsys@defobject{currentmarker}{\pgfqpoint{0.000000in}{-0.048611in}}{\pgfqpoint{0.000000in}{0.000000in}}{%
\pgfpathmoveto{\pgfqpoint{0.000000in}{0.000000in}}%
\pgfpathlineto{\pgfqpoint{0.000000in}{-0.048611in}}%
\pgfusepath{stroke,fill}%
}%
\begin{pgfscope}%
\pgfsys@transformshift{4.228309in}{0.515123in}%
\pgfsys@useobject{currentmarker}{}%
\end{pgfscope}%
\end{pgfscope}%
\begin{pgfscope}%
\definecolor{textcolor}{rgb}{0.000000,0.000000,0.000000}%
\pgfsetstrokecolor{textcolor}%
\pgfsetfillcolor{textcolor}%
\pgftext[x=4.228309in,y=0.417901in,,top]{\color{textcolor}\rmfamily\fontsize{10.000000}{12.000000}\selectfont \(\displaystyle 12\)}%
\end{pgfscope}%
\begin{pgfscope}%
\definecolor{textcolor}{rgb}{0.000000,0.000000,0.000000}%
\pgfsetstrokecolor{textcolor}%
\pgfsetfillcolor{textcolor}%
\pgftext[x=2.709569in,y=0.238889in,,top]{\color{textcolor}\rmfamily\fontsize{10.000000}{12.000000}\selectfont I(\(\displaystyle \mu\)A)}%
\end{pgfscope}%
\begin{pgfscope}%
\pgfsetbuttcap%
\pgfsetroundjoin%
\definecolor{currentfill}{rgb}{0.000000,0.000000,0.000000}%
\pgfsetfillcolor{currentfill}%
\pgfsetlinewidth{0.803000pt}%
\definecolor{currentstroke}{rgb}{0.000000,0.000000,0.000000}%
\pgfsetstrokecolor{currentstroke}%
\pgfsetdash{}{0pt}%
\pgfsys@defobject{currentmarker}{\pgfqpoint{-0.048611in}{0.000000in}}{\pgfqpoint{0.000000in}{0.000000in}}{%
\pgfpathmoveto{\pgfqpoint{0.000000in}{0.000000in}}%
\pgfpathlineto{\pgfqpoint{-0.048611in}{0.000000in}}%
\pgfusepath{stroke,fill}%
}%
\begin{pgfscope}%
\pgfsys@transformshift{0.772069in}{0.607562in}%
\pgfsys@useobject{currentmarker}{}%
\end{pgfscope}%
\end{pgfscope}%
\begin{pgfscope}%
\definecolor{textcolor}{rgb}{0.000000,0.000000,0.000000}%
\pgfsetstrokecolor{textcolor}%
\pgfsetfillcolor{textcolor}%
\pgftext[x=0.605402in,y=0.559337in,left,base]{\color{textcolor}\rmfamily\fontsize{10.000000}{12.000000}\selectfont \(\displaystyle 0\)}%
\end{pgfscope}%
\begin{pgfscope}%
\pgfsetbuttcap%
\pgfsetroundjoin%
\definecolor{currentfill}{rgb}{0.000000,0.000000,0.000000}%
\pgfsetfillcolor{currentfill}%
\pgfsetlinewidth{0.803000pt}%
\definecolor{currentstroke}{rgb}{0.000000,0.000000,0.000000}%
\pgfsetstrokecolor{currentstroke}%
\pgfsetdash{}{0pt}%
\pgfsys@defobject{currentmarker}{\pgfqpoint{-0.048611in}{0.000000in}}{\pgfqpoint{0.000000in}{0.000000in}}{%
\pgfpathmoveto{\pgfqpoint{0.000000in}{0.000000in}}%
\pgfpathlineto{\pgfqpoint{-0.048611in}{0.000000in}}%
\pgfusepath{stroke,fill}%
}%
\begin{pgfscope}%
\pgfsys@transformshift{0.772069in}{1.099095in}%
\pgfsys@useobject{currentmarker}{}%
\end{pgfscope}%
\end{pgfscope}%
\begin{pgfscope}%
\definecolor{textcolor}{rgb}{0.000000,0.000000,0.000000}%
\pgfsetstrokecolor{textcolor}%
\pgfsetfillcolor{textcolor}%
\pgftext[x=0.605402in,y=1.050870in,left,base]{\color{textcolor}\rmfamily\fontsize{10.000000}{12.000000}\selectfont \(\displaystyle 2\)}%
\end{pgfscope}%
\begin{pgfscope}%
\pgfsetbuttcap%
\pgfsetroundjoin%
\definecolor{currentfill}{rgb}{0.000000,0.000000,0.000000}%
\pgfsetfillcolor{currentfill}%
\pgfsetlinewidth{0.803000pt}%
\definecolor{currentstroke}{rgb}{0.000000,0.000000,0.000000}%
\pgfsetstrokecolor{currentstroke}%
\pgfsetdash{}{0pt}%
\pgfsys@defobject{currentmarker}{\pgfqpoint{-0.048611in}{0.000000in}}{\pgfqpoint{0.000000in}{0.000000in}}{%
\pgfpathmoveto{\pgfqpoint{0.000000in}{0.000000in}}%
\pgfpathlineto{\pgfqpoint{-0.048611in}{0.000000in}}%
\pgfusepath{stroke,fill}%
}%
\begin{pgfscope}%
\pgfsys@transformshift{0.772069in}{1.590628in}%
\pgfsys@useobject{currentmarker}{}%
\end{pgfscope}%
\end{pgfscope}%
\begin{pgfscope}%
\definecolor{textcolor}{rgb}{0.000000,0.000000,0.000000}%
\pgfsetstrokecolor{textcolor}%
\pgfsetfillcolor{textcolor}%
\pgftext[x=0.605402in,y=1.542403in,left,base]{\color{textcolor}\rmfamily\fontsize{10.000000}{12.000000}\selectfont \(\displaystyle 4\)}%
\end{pgfscope}%
\begin{pgfscope}%
\pgfsetbuttcap%
\pgfsetroundjoin%
\definecolor{currentfill}{rgb}{0.000000,0.000000,0.000000}%
\pgfsetfillcolor{currentfill}%
\pgfsetlinewidth{0.803000pt}%
\definecolor{currentstroke}{rgb}{0.000000,0.000000,0.000000}%
\pgfsetstrokecolor{currentstroke}%
\pgfsetdash{}{0pt}%
\pgfsys@defobject{currentmarker}{\pgfqpoint{-0.048611in}{0.000000in}}{\pgfqpoint{0.000000in}{0.000000in}}{%
\pgfpathmoveto{\pgfqpoint{0.000000in}{0.000000in}}%
\pgfpathlineto{\pgfqpoint{-0.048611in}{0.000000in}}%
\pgfusepath{stroke,fill}%
}%
\begin{pgfscope}%
\pgfsys@transformshift{0.772069in}{2.082161in}%
\pgfsys@useobject{currentmarker}{}%
\end{pgfscope}%
\end{pgfscope}%
\begin{pgfscope}%
\definecolor{textcolor}{rgb}{0.000000,0.000000,0.000000}%
\pgfsetstrokecolor{textcolor}%
\pgfsetfillcolor{textcolor}%
\pgftext[x=0.605402in,y=2.033935in,left,base]{\color{textcolor}\rmfamily\fontsize{10.000000}{12.000000}\selectfont \(\displaystyle 6\)}%
\end{pgfscope}%
\begin{pgfscope}%
\pgfsetbuttcap%
\pgfsetroundjoin%
\definecolor{currentfill}{rgb}{0.000000,0.000000,0.000000}%
\pgfsetfillcolor{currentfill}%
\pgfsetlinewidth{0.803000pt}%
\definecolor{currentstroke}{rgb}{0.000000,0.000000,0.000000}%
\pgfsetstrokecolor{currentstroke}%
\pgfsetdash{}{0pt}%
\pgfsys@defobject{currentmarker}{\pgfqpoint{-0.048611in}{0.000000in}}{\pgfqpoint{0.000000in}{0.000000in}}{%
\pgfpathmoveto{\pgfqpoint{0.000000in}{0.000000in}}%
\pgfpathlineto{\pgfqpoint{-0.048611in}{0.000000in}}%
\pgfusepath{stroke,fill}%
}%
\begin{pgfscope}%
\pgfsys@transformshift{0.772069in}{2.573693in}%
\pgfsys@useobject{currentmarker}{}%
\end{pgfscope}%
\end{pgfscope}%
\begin{pgfscope}%
\definecolor{textcolor}{rgb}{0.000000,0.000000,0.000000}%
\pgfsetstrokecolor{textcolor}%
\pgfsetfillcolor{textcolor}%
\pgftext[x=0.605402in,y=2.525468in,left,base]{\color{textcolor}\rmfamily\fontsize{10.000000}{12.000000}\selectfont \(\displaystyle 8\)}%
\end{pgfscope}%
\begin{pgfscope}%
\pgfsetbuttcap%
\pgfsetroundjoin%
\definecolor{currentfill}{rgb}{0.000000,0.000000,0.000000}%
\pgfsetfillcolor{currentfill}%
\pgfsetlinewidth{0.803000pt}%
\definecolor{currentstroke}{rgb}{0.000000,0.000000,0.000000}%
\pgfsetstrokecolor{currentstroke}%
\pgfsetdash{}{0pt}%
\pgfsys@defobject{currentmarker}{\pgfqpoint{-0.048611in}{0.000000in}}{\pgfqpoint{0.000000in}{0.000000in}}{%
\pgfpathmoveto{\pgfqpoint{0.000000in}{0.000000in}}%
\pgfpathlineto{\pgfqpoint{-0.048611in}{0.000000in}}%
\pgfusepath{stroke,fill}%
}%
\begin{pgfscope}%
\pgfsys@transformshift{0.772069in}{3.065226in}%
\pgfsys@useobject{currentmarker}{}%
\end{pgfscope}%
\end{pgfscope}%
\begin{pgfscope}%
\definecolor{textcolor}{rgb}{0.000000,0.000000,0.000000}%
\pgfsetstrokecolor{textcolor}%
\pgfsetfillcolor{textcolor}%
\pgftext[x=0.535957in,y=3.017001in,left,base]{\color{textcolor}\rmfamily\fontsize{10.000000}{12.000000}\selectfont \(\displaystyle 10\)}%
\end{pgfscope}%
\begin{pgfscope}%
\definecolor{textcolor}{rgb}{0.000000,0.000000,0.000000}%
\pgfsetstrokecolor{textcolor}%
\pgfsetfillcolor{textcolor}%
\pgftext[x=0.258179in,y=1.862623in,,bottom]{\color{textcolor}\rmfamily\fontsize{10.000000}{12.000000}\selectfont V(V)}%
\end{pgfscope}%
\begin{pgfscope}%
\pgfpathrectangle{\pgfqpoint{0.772069in}{0.515123in}}{\pgfqpoint{3.875000in}{2.695000in}}%
\pgfusepath{clip}%
\pgfsetbuttcap%
\pgfsetroundjoin%
\definecolor{currentfill}{rgb}{0.121569,0.466667,0.705882}%
\pgfsetfillcolor{currentfill}%
\pgfsetlinewidth{1.003750pt}%
\definecolor{currentstroke}{rgb}{0.121569,0.466667,0.705882}%
\pgfsetstrokecolor{currentstroke}%
\pgfsetdash{}{0pt}%
\pgfpathmoveto{\pgfqpoint{0.978205in}{0.827145in}}%
\pgfpathcurveto{\pgfqpoint{0.989255in}{0.827145in}}{\pgfqpoint{0.999854in}{0.831536in}}{\pgfqpoint{1.007668in}{0.839349in}}%
\pgfpathcurveto{\pgfqpoint{1.015481in}{0.847163in}}{\pgfqpoint{1.019872in}{0.857762in}}{\pgfqpoint{1.019872in}{0.868812in}}%
\pgfpathcurveto{\pgfqpoint{1.019872in}{0.879862in}}{\pgfqpoint{1.015481in}{0.890461in}}{\pgfqpoint{1.007668in}{0.898275in}}%
\pgfpathcurveto{\pgfqpoint{0.999854in}{0.906088in}}{\pgfqpoint{0.989255in}{0.910479in}}{\pgfqpoint{0.978205in}{0.910479in}}%
\pgfpathcurveto{\pgfqpoint{0.967155in}{0.910479in}}{\pgfqpoint{0.956556in}{0.906088in}}{\pgfqpoint{0.948742in}{0.898275in}}%
\pgfpathcurveto{\pgfqpoint{0.940929in}{0.890461in}}{\pgfqpoint{0.936538in}{0.879862in}}{\pgfqpoint{0.936538in}{0.868812in}}%
\pgfpathcurveto{\pgfqpoint{0.936538in}{0.857762in}}{\pgfqpoint{0.940929in}{0.847163in}}{\pgfqpoint{0.948742in}{0.839349in}}%
\pgfpathcurveto{\pgfqpoint{0.956556in}{0.831536in}}{\pgfqpoint{0.967155in}{0.827145in}}{\pgfqpoint{0.978205in}{0.827145in}}%
\pgfpathclose%
\pgfusepath{stroke,fill}%
\end{pgfscope}%
\begin{pgfscope}%
\pgfpathrectangle{\pgfqpoint{0.772069in}{0.515123in}}{\pgfqpoint{3.875000in}{2.695000in}}%
\pgfusepath{clip}%
\pgfsetbuttcap%
\pgfsetroundjoin%
\definecolor{currentfill}{rgb}{0.121569,0.466667,0.705882}%
\pgfsetfillcolor{currentfill}%
\pgfsetlinewidth{1.003750pt}%
\definecolor{currentstroke}{rgb}{0.121569,0.466667,0.705882}%
\pgfsetstrokecolor{currentstroke}%
\pgfsetdash{}{0pt}%
\pgfpathmoveto{\pgfqpoint{1.342703in}{1.069717in}}%
\pgfpathcurveto{\pgfqpoint{1.353753in}{1.069717in}}{\pgfqpoint{1.364352in}{1.074107in}}{\pgfqpoint{1.372165in}{1.081921in}}%
\pgfpathcurveto{\pgfqpoint{1.379979in}{1.089734in}}{\pgfqpoint{1.384369in}{1.100333in}}{\pgfqpoint{1.384369in}{1.111383in}}%
\pgfpathcurveto{\pgfqpoint{1.384369in}{1.122434in}}{\pgfqpoint{1.379979in}{1.133033in}}{\pgfqpoint{1.372165in}{1.140846in}}%
\pgfpathcurveto{\pgfqpoint{1.364352in}{1.148660in}}{\pgfqpoint{1.353753in}{1.153050in}}{\pgfqpoint{1.342703in}{1.153050in}}%
\pgfpathcurveto{\pgfqpoint{1.331652in}{1.153050in}}{\pgfqpoint{1.321053in}{1.148660in}}{\pgfqpoint{1.313240in}{1.140846in}}%
\pgfpathcurveto{\pgfqpoint{1.305426in}{1.133033in}}{\pgfqpoint{1.301036in}{1.122434in}}{\pgfqpoint{1.301036in}{1.111383in}}%
\pgfpathcurveto{\pgfqpoint{1.301036in}{1.100333in}}{\pgfqpoint{1.305426in}{1.089734in}}{\pgfqpoint{1.313240in}{1.081921in}}%
\pgfpathcurveto{\pgfqpoint{1.321053in}{1.074107in}}{\pgfqpoint{1.331652in}{1.069717in}}{\pgfqpoint{1.342703in}{1.069717in}}%
\pgfpathclose%
\pgfusepath{stroke,fill}%
\end{pgfscope}%
\begin{pgfscope}%
\pgfpathrectangle{\pgfqpoint{0.772069in}{0.515123in}}{\pgfqpoint{3.875000in}{2.695000in}}%
\pgfusepath{clip}%
\pgfsetbuttcap%
\pgfsetroundjoin%
\definecolor{currentfill}{rgb}{0.121569,0.466667,0.705882}%
\pgfsetfillcolor{currentfill}%
\pgfsetlinewidth{1.003750pt}%
\definecolor{currentstroke}{rgb}{0.121569,0.466667,0.705882}%
\pgfsetstrokecolor{currentstroke}%
\pgfsetdash{}{0pt}%
\pgfpathmoveto{\pgfqpoint{1.737575in}{1.313026in}}%
\pgfpathcurveto{\pgfqpoint{1.748625in}{1.313026in}}{\pgfqpoint{1.759224in}{1.317416in}}{\pgfqpoint{1.767038in}{1.325229in}}%
\pgfpathcurveto{\pgfqpoint{1.774851in}{1.333043in}}{\pgfqpoint{1.779242in}{1.343642in}}{\pgfqpoint{1.779242in}{1.354692in}}%
\pgfpathcurveto{\pgfqpoint{1.779242in}{1.365742in}}{\pgfqpoint{1.774851in}{1.376341in}}{\pgfqpoint{1.767038in}{1.384155in}}%
\pgfpathcurveto{\pgfqpoint{1.759224in}{1.391969in}}{\pgfqpoint{1.748625in}{1.396359in}}{\pgfqpoint{1.737575in}{1.396359in}}%
\pgfpathcurveto{\pgfqpoint{1.726525in}{1.396359in}}{\pgfqpoint{1.715926in}{1.391969in}}{\pgfqpoint{1.708112in}{1.384155in}}%
\pgfpathcurveto{\pgfqpoint{1.700299in}{1.376341in}}{\pgfqpoint{1.695908in}{1.365742in}}{\pgfqpoint{1.695908in}{1.354692in}}%
\pgfpathcurveto{\pgfqpoint{1.695908in}{1.343642in}}{\pgfqpoint{1.700299in}{1.333043in}}{\pgfqpoint{1.708112in}{1.325229in}}%
\pgfpathcurveto{\pgfqpoint{1.715926in}{1.317416in}}{\pgfqpoint{1.726525in}{1.313026in}}{\pgfqpoint{1.737575in}{1.313026in}}%
\pgfpathclose%
\pgfusepath{stroke,fill}%
\end{pgfscope}%
\begin{pgfscope}%
\pgfpathrectangle{\pgfqpoint{0.772069in}{0.515123in}}{\pgfqpoint{3.875000in}{2.695000in}}%
\pgfusepath{clip}%
\pgfsetbuttcap%
\pgfsetroundjoin%
\definecolor{currentfill}{rgb}{0.121569,0.466667,0.705882}%
\pgfsetfillcolor{currentfill}%
\pgfsetlinewidth{1.003750pt}%
\definecolor{currentstroke}{rgb}{0.121569,0.466667,0.705882}%
\pgfsetstrokecolor{currentstroke}%
\pgfsetdash{}{0pt}%
\pgfpathmoveto{\pgfqpoint{2.132447in}{1.563707in}}%
\pgfpathcurveto{\pgfqpoint{2.143498in}{1.563707in}}{\pgfqpoint{2.154097in}{1.568097in}}{\pgfqpoint{2.161910in}{1.575911in}}%
\pgfpathcurveto{\pgfqpoint{2.169724in}{1.583725in}}{\pgfqpoint{2.174114in}{1.594324in}}{\pgfqpoint{2.174114in}{1.605374in}}%
\pgfpathcurveto{\pgfqpoint{2.174114in}{1.616424in}}{\pgfqpoint{2.169724in}{1.627023in}}{\pgfqpoint{2.161910in}{1.634837in}}%
\pgfpathcurveto{\pgfqpoint{2.154097in}{1.642650in}}{\pgfqpoint{2.143498in}{1.647041in}}{\pgfqpoint{2.132447in}{1.647041in}}%
\pgfpathcurveto{\pgfqpoint{2.121397in}{1.647041in}}{\pgfqpoint{2.110798in}{1.642650in}}{\pgfqpoint{2.102985in}{1.634837in}}%
\pgfpathcurveto{\pgfqpoint{2.095171in}{1.627023in}}{\pgfqpoint{2.090781in}{1.616424in}}{\pgfqpoint{2.090781in}{1.605374in}}%
\pgfpathcurveto{\pgfqpoint{2.090781in}{1.594324in}}{\pgfqpoint{2.095171in}{1.583725in}}{\pgfqpoint{2.102985in}{1.575911in}}%
\pgfpathcurveto{\pgfqpoint{2.110798in}{1.568097in}}{\pgfqpoint{2.121397in}{1.563707in}}{\pgfqpoint{2.132447in}{1.563707in}}%
\pgfpathclose%
\pgfusepath{stroke,fill}%
\end{pgfscope}%
\begin{pgfscope}%
\pgfpathrectangle{\pgfqpoint{0.772069in}{0.515123in}}{\pgfqpoint{3.875000in}{2.695000in}}%
\pgfusepath{clip}%
\pgfsetbuttcap%
\pgfsetroundjoin%
\definecolor{currentfill}{rgb}{0.121569,0.466667,0.705882}%
\pgfsetfillcolor{currentfill}%
\pgfsetlinewidth{1.003750pt}%
\definecolor{currentstroke}{rgb}{0.121569,0.466667,0.705882}%
\pgfsetstrokecolor{currentstroke}%
\pgfsetdash{}{0pt}%
\pgfpathmoveto{\pgfqpoint{2.496945in}{1.804558in}}%
\pgfpathcurveto{\pgfqpoint{2.507995in}{1.804558in}}{\pgfqpoint{2.518594in}{1.808949in}}{\pgfqpoint{2.526408in}{1.816762in}}%
\pgfpathcurveto{\pgfqpoint{2.534221in}{1.824576in}}{\pgfqpoint{2.538612in}{1.835175in}}{\pgfqpoint{2.538612in}{1.846225in}}%
\pgfpathcurveto{\pgfqpoint{2.538612in}{1.857275in}}{\pgfqpoint{2.534221in}{1.867874in}}{\pgfqpoint{2.526408in}{1.875688in}}%
\pgfpathcurveto{\pgfqpoint{2.518594in}{1.883501in}}{\pgfqpoint{2.507995in}{1.887892in}}{\pgfqpoint{2.496945in}{1.887892in}}%
\pgfpathcurveto{\pgfqpoint{2.485895in}{1.887892in}}{\pgfqpoint{2.475296in}{1.883501in}}{\pgfqpoint{2.467482in}{1.875688in}}%
\pgfpathcurveto{\pgfqpoint{2.459669in}{1.867874in}}{\pgfqpoint{2.455278in}{1.857275in}}{\pgfqpoint{2.455278in}{1.846225in}}%
\pgfpathcurveto{\pgfqpoint{2.455278in}{1.835175in}}{\pgfqpoint{2.459669in}{1.824576in}}{\pgfqpoint{2.467482in}{1.816762in}}%
\pgfpathcurveto{\pgfqpoint{2.475296in}{1.808949in}}{\pgfqpoint{2.485895in}{1.804558in}}{\pgfqpoint{2.496945in}{1.804558in}}%
\pgfpathclose%
\pgfusepath{stroke,fill}%
\end{pgfscope}%
\begin{pgfscope}%
\pgfpathrectangle{\pgfqpoint{0.772069in}{0.515123in}}{\pgfqpoint{3.875000in}{2.695000in}}%
\pgfusepath{clip}%
\pgfsetbuttcap%
\pgfsetroundjoin%
\definecolor{currentfill}{rgb}{0.121569,0.466667,0.705882}%
\pgfsetfillcolor{currentfill}%
\pgfsetlinewidth{1.003750pt}%
\definecolor{currentstroke}{rgb}{0.121569,0.466667,0.705882}%
\pgfsetstrokecolor{currentstroke}%
\pgfsetdash{}{0pt}%
\pgfpathmoveto{\pgfqpoint{2.922192in}{2.069986in}}%
\pgfpathcurveto{\pgfqpoint{2.933242in}{2.069986in}}{\pgfqpoint{2.943841in}{2.074376in}}{\pgfqpoint{2.951655in}{2.082190in}}%
\pgfpathcurveto{\pgfqpoint{2.959469in}{2.090003in}}{\pgfqpoint{2.963859in}{2.100603in}}{\pgfqpoint{2.963859in}{2.111653in}}%
\pgfpathcurveto{\pgfqpoint{2.963859in}{2.122703in}}{\pgfqpoint{2.959469in}{2.133302in}}{\pgfqpoint{2.951655in}{2.141115in}}%
\pgfpathcurveto{\pgfqpoint{2.943841in}{2.148929in}}{\pgfqpoint{2.933242in}{2.153319in}}{\pgfqpoint{2.922192in}{2.153319in}}%
\pgfpathcurveto{\pgfqpoint{2.911142in}{2.153319in}}{\pgfqpoint{2.900543in}{2.148929in}}{\pgfqpoint{2.892730in}{2.141115in}}%
\pgfpathcurveto{\pgfqpoint{2.884916in}{2.133302in}}{\pgfqpoint{2.880526in}{2.122703in}}{\pgfqpoint{2.880526in}{2.111653in}}%
\pgfpathcurveto{\pgfqpoint{2.880526in}{2.100603in}}{\pgfqpoint{2.884916in}{2.090003in}}{\pgfqpoint{2.892730in}{2.082190in}}%
\pgfpathcurveto{\pgfqpoint{2.900543in}{2.074376in}}{\pgfqpoint{2.911142in}{2.069986in}}{\pgfqpoint{2.922192in}{2.069986in}}%
\pgfpathclose%
\pgfusepath{stroke,fill}%
\end{pgfscope}%
\begin{pgfscope}%
\pgfpathrectangle{\pgfqpoint{0.772069in}{0.515123in}}{\pgfqpoint{3.875000in}{2.695000in}}%
\pgfusepath{clip}%
\pgfsetbuttcap%
\pgfsetroundjoin%
\definecolor{currentfill}{rgb}{0.121569,0.466667,0.705882}%
\pgfsetfillcolor{currentfill}%
\pgfsetlinewidth{1.003750pt}%
\definecolor{currentstroke}{rgb}{0.121569,0.466667,0.705882}%
\pgfsetstrokecolor{currentstroke}%
\pgfsetdash{}{0pt}%
\pgfpathmoveto{\pgfqpoint{3.286690in}{2.305922in}}%
\pgfpathcurveto{\pgfqpoint{3.297740in}{2.305922in}}{\pgfqpoint{3.308339in}{2.310312in}}{\pgfqpoint{3.316153in}{2.318126in}}%
\pgfpathcurveto{\pgfqpoint{3.323966in}{2.325939in}}{\pgfqpoint{3.328357in}{2.336538in}}{\pgfqpoint{3.328357in}{2.347588in}}%
\pgfpathcurveto{\pgfqpoint{3.328357in}{2.358639in}}{\pgfqpoint{3.323966in}{2.369238in}}{\pgfqpoint{3.316153in}{2.377051in}}%
\pgfpathcurveto{\pgfqpoint{3.308339in}{2.384865in}}{\pgfqpoint{3.297740in}{2.389255in}}{\pgfqpoint{3.286690in}{2.389255in}}%
\pgfpathcurveto{\pgfqpoint{3.275640in}{2.389255in}}{\pgfqpoint{3.265041in}{2.384865in}}{\pgfqpoint{3.257227in}{2.377051in}}%
\pgfpathcurveto{\pgfqpoint{3.249414in}{2.369238in}}{\pgfqpoint{3.245023in}{2.358639in}}{\pgfqpoint{3.245023in}{2.347588in}}%
\pgfpathcurveto{\pgfqpoint{3.245023in}{2.336538in}}{\pgfqpoint{3.249414in}{2.325939in}}{\pgfqpoint{3.257227in}{2.318126in}}%
\pgfpathcurveto{\pgfqpoint{3.265041in}{2.310312in}}{\pgfqpoint{3.275640in}{2.305922in}}{\pgfqpoint{3.286690in}{2.305922in}}%
\pgfpathclose%
\pgfusepath{stroke,fill}%
\end{pgfscope}%
\begin{pgfscope}%
\pgfpathrectangle{\pgfqpoint{0.772069in}{0.515123in}}{\pgfqpoint{3.875000in}{2.695000in}}%
\pgfusepath{clip}%
\pgfsetbuttcap%
\pgfsetroundjoin%
\definecolor{currentfill}{rgb}{0.121569,0.466667,0.705882}%
\pgfsetfillcolor{currentfill}%
\pgfsetlinewidth{1.003750pt}%
\definecolor{currentstroke}{rgb}{0.121569,0.466667,0.705882}%
\pgfsetstrokecolor{currentstroke}%
\pgfsetdash{}{0pt}%
\pgfpathmoveto{\pgfqpoint{3.681562in}{2.549230in}}%
\pgfpathcurveto{\pgfqpoint{3.692613in}{2.549230in}}{\pgfqpoint{3.703212in}{2.553621in}}{\pgfqpoint{3.711025in}{2.561434in}}%
\pgfpathcurveto{\pgfqpoint{3.718839in}{2.569248in}}{\pgfqpoint{3.723229in}{2.579847in}}{\pgfqpoint{3.723229in}{2.590897in}}%
\pgfpathcurveto{\pgfqpoint{3.723229in}{2.601947in}}{\pgfqpoint{3.718839in}{2.612546in}}{\pgfqpoint{3.711025in}{2.620360in}}%
\pgfpathcurveto{\pgfqpoint{3.703212in}{2.628174in}}{\pgfqpoint{3.692613in}{2.632564in}}{\pgfqpoint{3.681562in}{2.632564in}}%
\pgfpathcurveto{\pgfqpoint{3.670512in}{2.632564in}}{\pgfqpoint{3.659913in}{2.628174in}}{\pgfqpoint{3.652100in}{2.620360in}}%
\pgfpathcurveto{\pgfqpoint{3.644286in}{2.612546in}}{\pgfqpoint{3.639896in}{2.601947in}}{\pgfqpoint{3.639896in}{2.590897in}}%
\pgfpathcurveto{\pgfqpoint{3.639896in}{2.579847in}}{\pgfqpoint{3.644286in}{2.569248in}}{\pgfqpoint{3.652100in}{2.561434in}}%
\pgfpathcurveto{\pgfqpoint{3.659913in}{2.553621in}}{\pgfqpoint{3.670512in}{2.549230in}}{\pgfqpoint{3.681562in}{2.549230in}}%
\pgfpathclose%
\pgfusepath{stroke,fill}%
\end{pgfscope}%
\begin{pgfscope}%
\pgfpathrectangle{\pgfqpoint{0.772069in}{0.515123in}}{\pgfqpoint{3.875000in}{2.695000in}}%
\pgfusepath{clip}%
\pgfsetbuttcap%
\pgfsetroundjoin%
\definecolor{currentfill}{rgb}{0.121569,0.466667,0.705882}%
\pgfsetfillcolor{currentfill}%
\pgfsetlinewidth{1.003750pt}%
\definecolor{currentstroke}{rgb}{0.121569,0.466667,0.705882}%
\pgfsetstrokecolor{currentstroke}%
\pgfsetdash{}{0pt}%
\pgfpathmoveto{\pgfqpoint{4.015685in}{2.777793in}}%
\pgfpathcurveto{\pgfqpoint{4.026735in}{2.777793in}}{\pgfqpoint{4.037334in}{2.782183in}}{\pgfqpoint{4.045148in}{2.789997in}}%
\pgfpathcurveto{\pgfqpoint{4.052962in}{2.797811in}}{\pgfqpoint{4.057352in}{2.808410in}}{\pgfqpoint{4.057352in}{2.819460in}}%
\pgfpathcurveto{\pgfqpoint{4.057352in}{2.830510in}}{\pgfqpoint{4.052962in}{2.841109in}}{\pgfqpoint{4.045148in}{2.848923in}}%
\pgfpathcurveto{\pgfqpoint{4.037334in}{2.856736in}}{\pgfqpoint{4.026735in}{2.861127in}}{\pgfqpoint{4.015685in}{2.861127in}}%
\pgfpathcurveto{\pgfqpoint{4.004635in}{2.861127in}}{\pgfqpoint{3.994036in}{2.856736in}}{\pgfqpoint{3.986222in}{2.848923in}}%
\pgfpathcurveto{\pgfqpoint{3.978409in}{2.841109in}}{\pgfqpoint{3.974019in}{2.830510in}}{\pgfqpoint{3.974019in}{2.819460in}}%
\pgfpathcurveto{\pgfqpoint{3.974019in}{2.808410in}}{\pgfqpoint{3.978409in}{2.797811in}}{\pgfqpoint{3.986222in}{2.789997in}}%
\pgfpathcurveto{\pgfqpoint{3.994036in}{2.782183in}}{\pgfqpoint{4.004635in}{2.777793in}}{\pgfqpoint{4.015685in}{2.777793in}}%
\pgfpathclose%
\pgfusepath{stroke,fill}%
\end{pgfscope}%
\begin{pgfscope}%
\pgfpathrectangle{\pgfqpoint{0.772069in}{0.515123in}}{\pgfqpoint{3.875000in}{2.695000in}}%
\pgfusepath{clip}%
\pgfsetbuttcap%
\pgfsetroundjoin%
\definecolor{currentfill}{rgb}{0.121569,0.466667,0.705882}%
\pgfsetfillcolor{currentfill}%
\pgfsetlinewidth{1.003750pt}%
\definecolor{currentstroke}{rgb}{0.121569,0.466667,0.705882}%
\pgfsetstrokecolor{currentstroke}%
\pgfsetdash{}{0pt}%
\pgfpathmoveto{\pgfqpoint{4.440932in}{3.043221in}}%
\pgfpathcurveto{\pgfqpoint{4.451983in}{3.043221in}}{\pgfqpoint{4.462582in}{3.047611in}}{\pgfqpoint{4.470395in}{3.055425in}}%
\pgfpathcurveto{\pgfqpoint{4.478209in}{3.063238in}}{\pgfqpoint{4.482599in}{3.073837in}}{\pgfqpoint{4.482599in}{3.084888in}}%
\pgfpathcurveto{\pgfqpoint{4.482599in}{3.095938in}}{\pgfqpoint{4.478209in}{3.106537in}}{\pgfqpoint{4.470395in}{3.114350in}}%
\pgfpathcurveto{\pgfqpoint{4.462582in}{3.122164in}}{\pgfqpoint{4.451983in}{3.126554in}}{\pgfqpoint{4.440932in}{3.126554in}}%
\pgfpathcurveto{\pgfqpoint{4.429882in}{3.126554in}}{\pgfqpoint{4.419283in}{3.122164in}}{\pgfqpoint{4.411470in}{3.114350in}}%
\pgfpathcurveto{\pgfqpoint{4.403656in}{3.106537in}}{\pgfqpoint{4.399266in}{3.095938in}}{\pgfqpoint{4.399266in}{3.084888in}}%
\pgfpathcurveto{\pgfqpoint{4.399266in}{3.073837in}}{\pgfqpoint{4.403656in}{3.063238in}}{\pgfqpoint{4.411470in}{3.055425in}}%
\pgfpathcurveto{\pgfqpoint{4.419283in}{3.047611in}}{\pgfqpoint{4.429882in}{3.043221in}}{\pgfqpoint{4.440932in}{3.043221in}}%
\pgfpathclose%
\pgfusepath{stroke,fill}%
\end{pgfscope}%
\begin{pgfscope}%
\pgfpathrectangle{\pgfqpoint{0.772069in}{0.515123in}}{\pgfqpoint{3.875000in}{2.695000in}}%
\pgfusepath{clip}%
\pgfsetbuttcap%
\pgfsetroundjoin%
\definecolor{currentfill}{rgb}{1.000000,0.388235,0.278431}%
\pgfsetfillcolor{currentfill}%
\pgfsetlinewidth{1.003750pt}%
\definecolor{currentstroke}{rgb}{1.000000,0.388235,0.278431}%
\pgfsetstrokecolor{currentstroke}%
\pgfsetdash{}{0pt}%
\pgfpathmoveto{\pgfqpoint{0.978205in}{0.623159in}}%
\pgfpathcurveto{\pgfqpoint{0.989255in}{0.623159in}}{\pgfqpoint{0.999854in}{0.627550in}}{\pgfqpoint{1.007668in}{0.635363in}}%
\pgfpathcurveto{\pgfqpoint{1.015481in}{0.643177in}}{\pgfqpoint{1.019872in}{0.653776in}}{\pgfqpoint{1.019872in}{0.664826in}}%
\pgfpathcurveto{\pgfqpoint{1.019872in}{0.675876in}}{\pgfqpoint{1.015481in}{0.686475in}}{\pgfqpoint{1.007668in}{0.694289in}}%
\pgfpathcurveto{\pgfqpoint{0.999854in}{0.702102in}}{\pgfqpoint{0.989255in}{0.706493in}}{\pgfqpoint{0.978205in}{0.706493in}}%
\pgfpathcurveto{\pgfqpoint{0.967155in}{0.706493in}}{\pgfqpoint{0.956556in}{0.702102in}}{\pgfqpoint{0.948742in}{0.694289in}}%
\pgfpathcurveto{\pgfqpoint{0.940929in}{0.686475in}}{\pgfqpoint{0.936538in}{0.675876in}}{\pgfqpoint{0.936538in}{0.664826in}}%
\pgfpathcurveto{\pgfqpoint{0.936538in}{0.653776in}}{\pgfqpoint{0.940929in}{0.643177in}}{\pgfqpoint{0.948742in}{0.635363in}}%
\pgfpathcurveto{\pgfqpoint{0.956556in}{0.627550in}}{\pgfqpoint{0.967155in}{0.623159in}}{\pgfqpoint{0.978205in}{0.623159in}}%
\pgfpathclose%
\pgfusepath{stroke,fill}%
\end{pgfscope}%
\begin{pgfscope}%
\pgfpathrectangle{\pgfqpoint{0.772069in}{0.515123in}}{\pgfqpoint{3.875000in}{2.695000in}}%
\pgfusepath{clip}%
\pgfsetbuttcap%
\pgfsetroundjoin%
\definecolor{currentfill}{rgb}{1.000000,0.388235,0.278431}%
\pgfsetfillcolor{currentfill}%
\pgfsetlinewidth{1.003750pt}%
\definecolor{currentstroke}{rgb}{1.000000,0.388235,0.278431}%
\pgfsetstrokecolor{currentstroke}%
\pgfsetdash{}{0pt}%
\pgfpathmoveto{\pgfqpoint{1.342703in}{0.676736in}}%
\pgfpathcurveto{\pgfqpoint{1.353753in}{0.676736in}}{\pgfqpoint{1.364352in}{0.681127in}}{\pgfqpoint{1.372165in}{0.688940in}}%
\pgfpathcurveto{\pgfqpoint{1.379979in}{0.696754in}}{\pgfqpoint{1.384369in}{0.707353in}}{\pgfqpoint{1.384369in}{0.718403in}}%
\pgfpathcurveto{\pgfqpoint{1.384369in}{0.729453in}}{\pgfqpoint{1.379979in}{0.740052in}}{\pgfqpoint{1.372165in}{0.747866in}}%
\pgfpathcurveto{\pgfqpoint{1.364352in}{0.755679in}}{\pgfqpoint{1.353753in}{0.760070in}}{\pgfqpoint{1.342703in}{0.760070in}}%
\pgfpathcurveto{\pgfqpoint{1.331652in}{0.760070in}}{\pgfqpoint{1.321053in}{0.755679in}}{\pgfqpoint{1.313240in}{0.747866in}}%
\pgfpathcurveto{\pgfqpoint{1.305426in}{0.740052in}}{\pgfqpoint{1.301036in}{0.729453in}}{\pgfqpoint{1.301036in}{0.718403in}}%
\pgfpathcurveto{\pgfqpoint{1.301036in}{0.707353in}}{\pgfqpoint{1.305426in}{0.696754in}}{\pgfqpoint{1.313240in}{0.688940in}}%
\pgfpathcurveto{\pgfqpoint{1.321053in}{0.681127in}}{\pgfqpoint{1.331652in}{0.676736in}}{\pgfqpoint{1.342703in}{0.676736in}}%
\pgfpathclose%
\pgfusepath{stroke,fill}%
\end{pgfscope}%
\begin{pgfscope}%
\pgfpathrectangle{\pgfqpoint{0.772069in}{0.515123in}}{\pgfqpoint{3.875000in}{2.695000in}}%
\pgfusepath{clip}%
\pgfsetbuttcap%
\pgfsetroundjoin%
\definecolor{currentfill}{rgb}{1.000000,0.388235,0.278431}%
\pgfsetfillcolor{currentfill}%
\pgfsetlinewidth{1.003750pt}%
\definecolor{currentstroke}{rgb}{1.000000,0.388235,0.278431}%
\pgfsetstrokecolor{currentstroke}%
\pgfsetdash{}{0pt}%
\pgfpathmoveto{\pgfqpoint{1.737575in}{0.730805in}}%
\pgfpathcurveto{\pgfqpoint{1.748625in}{0.730805in}}{\pgfqpoint{1.759224in}{0.735195in}}{\pgfqpoint{1.767038in}{0.743009in}}%
\pgfpathcurveto{\pgfqpoint{1.774851in}{0.750822in}}{\pgfqpoint{1.779242in}{0.761421in}}{\pgfqpoint{1.779242in}{0.772472in}}%
\pgfpathcurveto{\pgfqpoint{1.779242in}{0.783522in}}{\pgfqpoint{1.774851in}{0.794121in}}{\pgfqpoint{1.767038in}{0.801934in}}%
\pgfpathcurveto{\pgfqpoint{1.759224in}{0.809748in}}{\pgfqpoint{1.748625in}{0.814138in}}{\pgfqpoint{1.737575in}{0.814138in}}%
\pgfpathcurveto{\pgfqpoint{1.726525in}{0.814138in}}{\pgfqpoint{1.715926in}{0.809748in}}{\pgfqpoint{1.708112in}{0.801934in}}%
\pgfpathcurveto{\pgfqpoint{1.700299in}{0.794121in}}{\pgfqpoint{1.695908in}{0.783522in}}{\pgfqpoint{1.695908in}{0.772472in}}%
\pgfpathcurveto{\pgfqpoint{1.695908in}{0.761421in}}{\pgfqpoint{1.700299in}{0.750822in}}{\pgfqpoint{1.708112in}{0.743009in}}%
\pgfpathcurveto{\pgfqpoint{1.715926in}{0.735195in}}{\pgfqpoint{1.726525in}{0.730805in}}{\pgfqpoint{1.737575in}{0.730805in}}%
\pgfpathclose%
\pgfusepath{stroke,fill}%
\end{pgfscope}%
\begin{pgfscope}%
\pgfpathrectangle{\pgfqpoint{0.772069in}{0.515123in}}{\pgfqpoint{3.875000in}{2.695000in}}%
\pgfusepath{clip}%
\pgfsetbuttcap%
\pgfsetroundjoin%
\definecolor{currentfill}{rgb}{1.000000,0.388235,0.278431}%
\pgfsetfillcolor{currentfill}%
\pgfsetlinewidth{1.003750pt}%
\definecolor{currentstroke}{rgb}{1.000000,0.388235,0.278431}%
\pgfsetstrokecolor{currentstroke}%
\pgfsetdash{}{0pt}%
\pgfpathmoveto{\pgfqpoint{2.132447in}{0.785611in}}%
\pgfpathcurveto{\pgfqpoint{2.143498in}{0.785611in}}{\pgfqpoint{2.154097in}{0.790001in}}{\pgfqpoint{2.161910in}{0.797815in}}%
\pgfpathcurveto{\pgfqpoint{2.169724in}{0.805628in}}{\pgfqpoint{2.174114in}{0.816227in}}{\pgfqpoint{2.174114in}{0.827278in}}%
\pgfpathcurveto{\pgfqpoint{2.174114in}{0.838328in}}{\pgfqpoint{2.169724in}{0.848927in}}{\pgfqpoint{2.161910in}{0.856740in}}%
\pgfpathcurveto{\pgfqpoint{2.154097in}{0.864554in}}{\pgfqpoint{2.143498in}{0.868944in}}{\pgfqpoint{2.132447in}{0.868944in}}%
\pgfpathcurveto{\pgfqpoint{2.121397in}{0.868944in}}{\pgfqpoint{2.110798in}{0.864554in}}{\pgfqpoint{2.102985in}{0.856740in}}%
\pgfpathcurveto{\pgfqpoint{2.095171in}{0.848927in}}{\pgfqpoint{2.090781in}{0.838328in}}{\pgfqpoint{2.090781in}{0.827278in}}%
\pgfpathcurveto{\pgfqpoint{2.090781in}{0.816227in}}{\pgfqpoint{2.095171in}{0.805628in}}{\pgfqpoint{2.102985in}{0.797815in}}%
\pgfpathcurveto{\pgfqpoint{2.110798in}{0.790001in}}{\pgfqpoint{2.121397in}{0.785611in}}{\pgfqpoint{2.132447in}{0.785611in}}%
\pgfpathclose%
\pgfusepath{stroke,fill}%
\end{pgfscope}%
\begin{pgfscope}%
\pgfpathrectangle{\pgfqpoint{0.772069in}{0.515123in}}{\pgfqpoint{3.875000in}{2.695000in}}%
\pgfusepath{clip}%
\pgfsetbuttcap%
\pgfsetroundjoin%
\definecolor{currentfill}{rgb}{1.000000,0.388235,0.278431}%
\pgfsetfillcolor{currentfill}%
\pgfsetlinewidth{1.003750pt}%
\definecolor{currentstroke}{rgb}{1.000000,0.388235,0.278431}%
\pgfsetstrokecolor{currentstroke}%
\pgfsetdash{}{0pt}%
\pgfpathmoveto{\pgfqpoint{2.496945in}{0.838696in}}%
\pgfpathcurveto{\pgfqpoint{2.507995in}{0.838696in}}{\pgfqpoint{2.518594in}{0.843087in}}{\pgfqpoint{2.526408in}{0.850900in}}%
\pgfpathcurveto{\pgfqpoint{2.534221in}{0.858714in}}{\pgfqpoint{2.538612in}{0.869313in}}{\pgfqpoint{2.538612in}{0.880363in}}%
\pgfpathcurveto{\pgfqpoint{2.538612in}{0.891413in}}{\pgfqpoint{2.534221in}{0.902012in}}{\pgfqpoint{2.526408in}{0.909826in}}%
\pgfpathcurveto{\pgfqpoint{2.518594in}{0.917639in}}{\pgfqpoint{2.507995in}{0.922030in}}{\pgfqpoint{2.496945in}{0.922030in}}%
\pgfpathcurveto{\pgfqpoint{2.485895in}{0.922030in}}{\pgfqpoint{2.475296in}{0.917639in}}{\pgfqpoint{2.467482in}{0.909826in}}%
\pgfpathcurveto{\pgfqpoint{2.459669in}{0.902012in}}{\pgfqpoint{2.455278in}{0.891413in}}{\pgfqpoint{2.455278in}{0.880363in}}%
\pgfpathcurveto{\pgfqpoint{2.455278in}{0.869313in}}{\pgfqpoint{2.459669in}{0.858714in}}{\pgfqpoint{2.467482in}{0.850900in}}%
\pgfpathcurveto{\pgfqpoint{2.475296in}{0.843087in}}{\pgfqpoint{2.485895in}{0.838696in}}{\pgfqpoint{2.496945in}{0.838696in}}%
\pgfpathclose%
\pgfusepath{stroke,fill}%
\end{pgfscope}%
\begin{pgfscope}%
\pgfpathrectangle{\pgfqpoint{0.772069in}{0.515123in}}{\pgfqpoint{3.875000in}{2.695000in}}%
\pgfusepath{clip}%
\pgfsetbuttcap%
\pgfsetroundjoin%
\definecolor{currentfill}{rgb}{1.000000,0.388235,0.278431}%
\pgfsetfillcolor{currentfill}%
\pgfsetlinewidth{1.003750pt}%
\definecolor{currentstroke}{rgb}{1.000000,0.388235,0.278431}%
\pgfsetstrokecolor{currentstroke}%
\pgfsetdash{}{0pt}%
\pgfpathmoveto{\pgfqpoint{2.922192in}{0.896943in}}%
\pgfpathcurveto{\pgfqpoint{2.933242in}{0.896943in}}{\pgfqpoint{2.943841in}{0.901333in}}{\pgfqpoint{2.951655in}{0.909147in}}%
\pgfpathcurveto{\pgfqpoint{2.959469in}{0.916961in}}{\pgfqpoint{2.963859in}{0.927560in}}{\pgfqpoint{2.963859in}{0.938610in}}%
\pgfpathcurveto{\pgfqpoint{2.963859in}{0.949660in}}{\pgfqpoint{2.959469in}{0.960259in}}{\pgfqpoint{2.951655in}{0.968072in}}%
\pgfpathcurveto{\pgfqpoint{2.943841in}{0.975886in}}{\pgfqpoint{2.933242in}{0.980276in}}{\pgfqpoint{2.922192in}{0.980276in}}%
\pgfpathcurveto{\pgfqpoint{2.911142in}{0.980276in}}{\pgfqpoint{2.900543in}{0.975886in}}{\pgfqpoint{2.892730in}{0.968072in}}%
\pgfpathcurveto{\pgfqpoint{2.884916in}{0.960259in}}{\pgfqpoint{2.880526in}{0.949660in}}{\pgfqpoint{2.880526in}{0.938610in}}%
\pgfpathcurveto{\pgfqpoint{2.880526in}{0.927560in}}{\pgfqpoint{2.884916in}{0.916961in}}{\pgfqpoint{2.892730in}{0.909147in}}%
\pgfpathcurveto{\pgfqpoint{2.900543in}{0.901333in}}{\pgfqpoint{2.911142in}{0.896943in}}{\pgfqpoint{2.922192in}{0.896943in}}%
\pgfpathclose%
\pgfusepath{stroke,fill}%
\end{pgfscope}%
\begin{pgfscope}%
\pgfpathrectangle{\pgfqpoint{0.772069in}{0.515123in}}{\pgfqpoint{3.875000in}{2.695000in}}%
\pgfusepath{clip}%
\pgfsetbuttcap%
\pgfsetroundjoin%
\definecolor{currentfill}{rgb}{1.000000,0.388235,0.278431}%
\pgfsetfillcolor{currentfill}%
\pgfsetlinewidth{1.003750pt}%
\definecolor{currentstroke}{rgb}{1.000000,0.388235,0.278431}%
\pgfsetstrokecolor{currentstroke}%
\pgfsetdash{}{0pt}%
\pgfpathmoveto{\pgfqpoint{3.286690in}{0.948800in}}%
\pgfpathcurveto{\pgfqpoint{3.297740in}{0.948800in}}{\pgfqpoint{3.308339in}{0.953190in}}{\pgfqpoint{3.316153in}{0.961004in}}%
\pgfpathcurveto{\pgfqpoint{3.323966in}{0.968817in}}{\pgfqpoint{3.328357in}{0.979416in}}{\pgfqpoint{3.328357in}{0.990466in}}%
\pgfpathcurveto{\pgfqpoint{3.328357in}{1.001517in}}{\pgfqpoint{3.323966in}{1.012116in}}{\pgfqpoint{3.316153in}{1.019929in}}%
\pgfpathcurveto{\pgfqpoint{3.308339in}{1.027743in}}{\pgfqpoint{3.297740in}{1.032133in}}{\pgfqpoint{3.286690in}{1.032133in}}%
\pgfpathcurveto{\pgfqpoint{3.275640in}{1.032133in}}{\pgfqpoint{3.265041in}{1.027743in}}{\pgfqpoint{3.257227in}{1.019929in}}%
\pgfpathcurveto{\pgfqpoint{3.249414in}{1.012116in}}{\pgfqpoint{3.245023in}{1.001517in}}{\pgfqpoint{3.245023in}{0.990466in}}%
\pgfpathcurveto{\pgfqpoint{3.245023in}{0.979416in}}{\pgfqpoint{3.249414in}{0.968817in}}{\pgfqpoint{3.257227in}{0.961004in}}%
\pgfpathcurveto{\pgfqpoint{3.265041in}{0.953190in}}{\pgfqpoint{3.275640in}{0.948800in}}{\pgfqpoint{3.286690in}{0.948800in}}%
\pgfpathclose%
\pgfusepath{stroke,fill}%
\end{pgfscope}%
\begin{pgfscope}%
\pgfpathrectangle{\pgfqpoint{0.772069in}{0.515123in}}{\pgfqpoint{3.875000in}{2.695000in}}%
\pgfusepath{clip}%
\pgfsetbuttcap%
\pgfsetroundjoin%
\definecolor{currentfill}{rgb}{1.000000,0.388235,0.278431}%
\pgfsetfillcolor{currentfill}%
\pgfsetlinewidth{1.003750pt}%
\definecolor{currentstroke}{rgb}{1.000000,0.388235,0.278431}%
\pgfsetstrokecolor{currentstroke}%
\pgfsetdash{}{0pt}%
\pgfpathmoveto{\pgfqpoint{3.681562in}{1.002623in}}%
\pgfpathcurveto{\pgfqpoint{3.692613in}{1.002623in}}{\pgfqpoint{3.703212in}{1.007013in}}{\pgfqpoint{3.711025in}{1.014826in}}%
\pgfpathcurveto{\pgfqpoint{3.718839in}{1.022640in}}{\pgfqpoint{3.723229in}{1.033239in}}{\pgfqpoint{3.723229in}{1.044289in}}%
\pgfpathcurveto{\pgfqpoint{3.723229in}{1.055339in}}{\pgfqpoint{3.718839in}{1.065938in}}{\pgfqpoint{3.711025in}{1.073752in}}%
\pgfpathcurveto{\pgfqpoint{3.703212in}{1.081566in}}{\pgfqpoint{3.692613in}{1.085956in}}{\pgfqpoint{3.681562in}{1.085956in}}%
\pgfpathcurveto{\pgfqpoint{3.670512in}{1.085956in}}{\pgfqpoint{3.659913in}{1.081566in}}{\pgfqpoint{3.652100in}{1.073752in}}%
\pgfpathcurveto{\pgfqpoint{3.644286in}{1.065938in}}{\pgfqpoint{3.639896in}{1.055339in}}{\pgfqpoint{3.639896in}{1.044289in}}%
\pgfpathcurveto{\pgfqpoint{3.639896in}{1.033239in}}{\pgfqpoint{3.644286in}{1.022640in}}{\pgfqpoint{3.652100in}{1.014826in}}%
\pgfpathcurveto{\pgfqpoint{3.659913in}{1.007013in}}{\pgfqpoint{3.670512in}{1.002623in}}{\pgfqpoint{3.681562in}{1.002623in}}%
\pgfpathclose%
\pgfusepath{stroke,fill}%
\end{pgfscope}%
\begin{pgfscope}%
\pgfpathrectangle{\pgfqpoint{0.772069in}{0.515123in}}{\pgfqpoint{3.875000in}{2.695000in}}%
\pgfusepath{clip}%
\pgfsetbuttcap%
\pgfsetroundjoin%
\definecolor{currentfill}{rgb}{1.000000,0.388235,0.278431}%
\pgfsetfillcolor{currentfill}%
\pgfsetlinewidth{1.003750pt}%
\definecolor{currentstroke}{rgb}{1.000000,0.388235,0.278431}%
\pgfsetstrokecolor{currentstroke}%
\pgfsetdash{}{0pt}%
\pgfpathmoveto{\pgfqpoint{4.015685in}{1.050055in}}%
\pgfpathcurveto{\pgfqpoint{4.026735in}{1.050055in}}{\pgfqpoint{4.037334in}{1.054446in}}{\pgfqpoint{4.045148in}{1.062259in}}%
\pgfpathcurveto{\pgfqpoint{4.052962in}{1.070073in}}{\pgfqpoint{4.057352in}{1.080672in}}{\pgfqpoint{4.057352in}{1.091722in}}%
\pgfpathcurveto{\pgfqpoint{4.057352in}{1.102772in}}{\pgfqpoint{4.052962in}{1.113371in}}{\pgfqpoint{4.045148in}{1.121185in}}%
\pgfpathcurveto{\pgfqpoint{4.037334in}{1.128999in}}{\pgfqpoint{4.026735in}{1.133389in}}{\pgfqpoint{4.015685in}{1.133389in}}%
\pgfpathcurveto{\pgfqpoint{4.004635in}{1.133389in}}{\pgfqpoint{3.994036in}{1.128999in}}{\pgfqpoint{3.986222in}{1.121185in}}%
\pgfpathcurveto{\pgfqpoint{3.978409in}{1.113371in}}{\pgfqpoint{3.974019in}{1.102772in}}{\pgfqpoint{3.974019in}{1.091722in}}%
\pgfpathcurveto{\pgfqpoint{3.974019in}{1.080672in}}{\pgfqpoint{3.978409in}{1.070073in}}{\pgfqpoint{3.986222in}{1.062259in}}%
\pgfpathcurveto{\pgfqpoint{3.994036in}{1.054446in}}{\pgfqpoint{4.004635in}{1.050055in}}{\pgfqpoint{4.015685in}{1.050055in}}%
\pgfpathclose%
\pgfusepath{stroke,fill}%
\end{pgfscope}%
\begin{pgfscope}%
\pgfpathrectangle{\pgfqpoint{0.772069in}{0.515123in}}{\pgfqpoint{3.875000in}{2.695000in}}%
\pgfusepath{clip}%
\pgfsetbuttcap%
\pgfsetroundjoin%
\definecolor{currentfill}{rgb}{1.000000,0.388235,0.278431}%
\pgfsetfillcolor{currentfill}%
\pgfsetlinewidth{1.003750pt}%
\definecolor{currentstroke}{rgb}{1.000000,0.388235,0.278431}%
\pgfsetstrokecolor{currentstroke}%
\pgfsetdash{}{0pt}%
\pgfpathmoveto{\pgfqpoint{4.440932in}{1.109039in}}%
\pgfpathcurveto{\pgfqpoint{4.451983in}{1.109039in}}{\pgfqpoint{4.462582in}{1.113430in}}{\pgfqpoint{4.470395in}{1.121243in}}%
\pgfpathcurveto{\pgfqpoint{4.478209in}{1.129057in}}{\pgfqpoint{4.482599in}{1.139656in}}{\pgfqpoint{4.482599in}{1.150706in}}%
\pgfpathcurveto{\pgfqpoint{4.482599in}{1.161756in}}{\pgfqpoint{4.478209in}{1.172355in}}{\pgfqpoint{4.470395in}{1.180169in}}%
\pgfpathcurveto{\pgfqpoint{4.462582in}{1.187982in}}{\pgfqpoint{4.451983in}{1.192373in}}{\pgfqpoint{4.440932in}{1.192373in}}%
\pgfpathcurveto{\pgfqpoint{4.429882in}{1.192373in}}{\pgfqpoint{4.419283in}{1.187982in}}{\pgfqpoint{4.411470in}{1.180169in}}%
\pgfpathcurveto{\pgfqpoint{4.403656in}{1.172355in}}{\pgfqpoint{4.399266in}{1.161756in}}{\pgfqpoint{4.399266in}{1.150706in}}%
\pgfpathcurveto{\pgfqpoint{4.399266in}{1.139656in}}{\pgfqpoint{4.403656in}{1.129057in}}{\pgfqpoint{4.411470in}{1.121243in}}%
\pgfpathcurveto{\pgfqpoint{4.419283in}{1.113430in}}{\pgfqpoint{4.429882in}{1.109039in}}{\pgfqpoint{4.440932in}{1.109039in}}%
\pgfpathclose%
\pgfusepath{stroke,fill}%
\end{pgfscope}%
\begin{pgfscope}%
\pgfpathrectangle{\pgfqpoint{0.772069in}{0.515123in}}{\pgfqpoint{3.875000in}{2.695000in}}%
\pgfusepath{clip}%
\pgfsetbuttcap%
\pgfsetroundjoin%
\definecolor{currentfill}{rgb}{1.000000,0.843137,0.000000}%
\pgfsetfillcolor{currentfill}%
\pgfsetlinewidth{1.003750pt}%
\definecolor{currentstroke}{rgb}{1.000000,0.843137,0.000000}%
\pgfsetstrokecolor{currentstroke}%
\pgfsetdash{}{0pt}%
\pgfpathmoveto{\pgfqpoint{0.978205in}{0.636922in}}%
\pgfpathcurveto{\pgfqpoint{0.989255in}{0.636922in}}{\pgfqpoint{0.999854in}{0.641312in}}{\pgfqpoint{1.007668in}{0.649126in}}%
\pgfpathcurveto{\pgfqpoint{1.015481in}{0.656940in}}{\pgfqpoint{1.019872in}{0.667539in}}{\pgfqpoint{1.019872in}{0.678589in}}%
\pgfpathcurveto{\pgfqpoint{1.019872in}{0.689639in}}{\pgfqpoint{1.015481in}{0.700238in}}{\pgfqpoint{1.007668in}{0.708052in}}%
\pgfpathcurveto{\pgfqpoint{0.999854in}{0.715865in}}{\pgfqpoint{0.989255in}{0.720256in}}{\pgfqpoint{0.978205in}{0.720256in}}%
\pgfpathcurveto{\pgfqpoint{0.967155in}{0.720256in}}{\pgfqpoint{0.956556in}{0.715865in}}{\pgfqpoint{0.948742in}{0.708052in}}%
\pgfpathcurveto{\pgfqpoint{0.940929in}{0.700238in}}{\pgfqpoint{0.936538in}{0.689639in}}{\pgfqpoint{0.936538in}{0.678589in}}%
\pgfpathcurveto{\pgfqpoint{0.936538in}{0.667539in}}{\pgfqpoint{0.940929in}{0.656940in}}{\pgfqpoint{0.948742in}{0.649126in}}%
\pgfpathcurveto{\pgfqpoint{0.956556in}{0.641312in}}{\pgfqpoint{0.967155in}{0.636922in}}{\pgfqpoint{0.978205in}{0.636922in}}%
\pgfpathclose%
\pgfusepath{stroke,fill}%
\end{pgfscope}%
\begin{pgfscope}%
\pgfpathrectangle{\pgfqpoint{0.772069in}{0.515123in}}{\pgfqpoint{3.875000in}{2.695000in}}%
\pgfusepath{clip}%
\pgfsetbuttcap%
\pgfsetroundjoin%
\definecolor{currentfill}{rgb}{1.000000,0.843137,0.000000}%
\pgfsetfillcolor{currentfill}%
\pgfsetlinewidth{1.003750pt}%
\definecolor{currentstroke}{rgb}{1.000000,0.843137,0.000000}%
\pgfsetstrokecolor{currentstroke}%
\pgfsetdash{}{0pt}%
\pgfpathmoveto{\pgfqpoint{1.342703in}{0.703033in}}%
\pgfpathcurveto{\pgfqpoint{1.353753in}{0.703033in}}{\pgfqpoint{1.364352in}{0.707424in}}{\pgfqpoint{1.372165in}{0.715237in}}%
\pgfpathcurveto{\pgfqpoint{1.379979in}{0.723051in}}{\pgfqpoint{1.384369in}{0.733650in}}{\pgfqpoint{1.384369in}{0.744700in}}%
\pgfpathcurveto{\pgfqpoint{1.384369in}{0.755750in}}{\pgfqpoint{1.379979in}{0.766349in}}{\pgfqpoint{1.372165in}{0.774163in}}%
\pgfpathcurveto{\pgfqpoint{1.364352in}{0.781976in}}{\pgfqpoint{1.353753in}{0.786367in}}{\pgfqpoint{1.342703in}{0.786367in}}%
\pgfpathcurveto{\pgfqpoint{1.331652in}{0.786367in}}{\pgfqpoint{1.321053in}{0.781976in}}{\pgfqpoint{1.313240in}{0.774163in}}%
\pgfpathcurveto{\pgfqpoint{1.305426in}{0.766349in}}{\pgfqpoint{1.301036in}{0.755750in}}{\pgfqpoint{1.301036in}{0.744700in}}%
\pgfpathcurveto{\pgfqpoint{1.301036in}{0.733650in}}{\pgfqpoint{1.305426in}{0.723051in}}{\pgfqpoint{1.313240in}{0.715237in}}%
\pgfpathcurveto{\pgfqpoint{1.321053in}{0.707424in}}{\pgfqpoint{1.331652in}{0.703033in}}{\pgfqpoint{1.342703in}{0.703033in}}%
\pgfpathclose%
\pgfusepath{stroke,fill}%
\end{pgfscope}%
\begin{pgfscope}%
\pgfpathrectangle{\pgfqpoint{0.772069in}{0.515123in}}{\pgfqpoint{3.875000in}{2.695000in}}%
\pgfusepath{clip}%
\pgfsetbuttcap%
\pgfsetroundjoin%
\definecolor{currentfill}{rgb}{1.000000,0.843137,0.000000}%
\pgfsetfillcolor{currentfill}%
\pgfsetlinewidth{1.003750pt}%
\definecolor{currentstroke}{rgb}{1.000000,0.843137,0.000000}%
\pgfsetstrokecolor{currentstroke}%
\pgfsetdash{}{0pt}%
\pgfpathmoveto{\pgfqpoint{1.737575in}{0.769882in}}%
\pgfpathcurveto{\pgfqpoint{1.748625in}{0.769882in}}{\pgfqpoint{1.759224in}{0.774272in}}{\pgfqpoint{1.767038in}{0.782086in}}%
\pgfpathcurveto{\pgfqpoint{1.774851in}{0.789899in}}{\pgfqpoint{1.779242in}{0.800498in}}{\pgfqpoint{1.779242in}{0.811548in}}%
\pgfpathcurveto{\pgfqpoint{1.779242in}{0.822599in}}{\pgfqpoint{1.774851in}{0.833198in}}{\pgfqpoint{1.767038in}{0.841011in}}%
\pgfpathcurveto{\pgfqpoint{1.759224in}{0.848825in}}{\pgfqpoint{1.748625in}{0.853215in}}{\pgfqpoint{1.737575in}{0.853215in}}%
\pgfpathcurveto{\pgfqpoint{1.726525in}{0.853215in}}{\pgfqpoint{1.715926in}{0.848825in}}{\pgfqpoint{1.708112in}{0.841011in}}%
\pgfpathcurveto{\pgfqpoint{1.700299in}{0.833198in}}{\pgfqpoint{1.695908in}{0.822599in}}{\pgfqpoint{1.695908in}{0.811548in}}%
\pgfpathcurveto{\pgfqpoint{1.695908in}{0.800498in}}{\pgfqpoint{1.700299in}{0.789899in}}{\pgfqpoint{1.708112in}{0.782086in}}%
\pgfpathcurveto{\pgfqpoint{1.715926in}{0.774272in}}{\pgfqpoint{1.726525in}{0.769882in}}{\pgfqpoint{1.737575in}{0.769882in}}%
\pgfpathclose%
\pgfusepath{stroke,fill}%
\end{pgfscope}%
\begin{pgfscope}%
\pgfpathrectangle{\pgfqpoint{0.772069in}{0.515123in}}{\pgfqpoint{3.875000in}{2.695000in}}%
\pgfusepath{clip}%
\pgfsetbuttcap%
\pgfsetroundjoin%
\definecolor{currentfill}{rgb}{1.000000,0.843137,0.000000}%
\pgfsetfillcolor{currentfill}%
\pgfsetlinewidth{1.003750pt}%
\definecolor{currentstroke}{rgb}{1.000000,0.843137,0.000000}%
\pgfsetstrokecolor{currentstroke}%
\pgfsetdash{}{0pt}%
\pgfpathmoveto{\pgfqpoint{2.132447in}{0.837713in}}%
\pgfpathcurveto{\pgfqpoint{2.143498in}{0.837713in}}{\pgfqpoint{2.154097in}{0.842104in}}{\pgfqpoint{2.161910in}{0.849917in}}%
\pgfpathcurveto{\pgfqpoint{2.169724in}{0.857731in}}{\pgfqpoint{2.174114in}{0.868330in}}{\pgfqpoint{2.174114in}{0.879380in}}%
\pgfpathcurveto{\pgfqpoint{2.174114in}{0.890430in}}{\pgfqpoint{2.169724in}{0.901029in}}{\pgfqpoint{2.161910in}{0.908843in}}%
\pgfpathcurveto{\pgfqpoint{2.154097in}{0.916656in}}{\pgfqpoint{2.143498in}{0.921047in}}{\pgfqpoint{2.132447in}{0.921047in}}%
\pgfpathcurveto{\pgfqpoint{2.121397in}{0.921047in}}{\pgfqpoint{2.110798in}{0.916656in}}{\pgfqpoint{2.102985in}{0.908843in}}%
\pgfpathcurveto{\pgfqpoint{2.095171in}{0.901029in}}{\pgfqpoint{2.090781in}{0.890430in}}{\pgfqpoint{2.090781in}{0.879380in}}%
\pgfpathcurveto{\pgfqpoint{2.090781in}{0.868330in}}{\pgfqpoint{2.095171in}{0.857731in}}{\pgfqpoint{2.102985in}{0.849917in}}%
\pgfpathcurveto{\pgfqpoint{2.110798in}{0.842104in}}{\pgfqpoint{2.121397in}{0.837713in}}{\pgfqpoint{2.132447in}{0.837713in}}%
\pgfpathclose%
\pgfusepath{stroke,fill}%
\end{pgfscope}%
\begin{pgfscope}%
\pgfpathrectangle{\pgfqpoint{0.772069in}{0.515123in}}{\pgfqpoint{3.875000in}{2.695000in}}%
\pgfusepath{clip}%
\pgfsetbuttcap%
\pgfsetroundjoin%
\definecolor{currentfill}{rgb}{1.000000,0.843137,0.000000}%
\pgfsetfillcolor{currentfill}%
\pgfsetlinewidth{1.003750pt}%
\definecolor{currentstroke}{rgb}{1.000000,0.843137,0.000000}%
\pgfsetstrokecolor{currentstroke}%
\pgfsetdash{}{0pt}%
\pgfpathmoveto{\pgfqpoint{2.496945in}{0.903579in}}%
\pgfpathcurveto{\pgfqpoint{2.507995in}{0.903579in}}{\pgfqpoint{2.518594in}{0.907969in}}{\pgfqpoint{2.526408in}{0.915783in}}%
\pgfpathcurveto{\pgfqpoint{2.534221in}{0.923596in}}{\pgfqpoint{2.538612in}{0.934195in}}{\pgfqpoint{2.538612in}{0.945245in}}%
\pgfpathcurveto{\pgfqpoint{2.538612in}{0.956296in}}{\pgfqpoint{2.534221in}{0.966895in}}{\pgfqpoint{2.526408in}{0.974708in}}%
\pgfpathcurveto{\pgfqpoint{2.518594in}{0.982522in}}{\pgfqpoint{2.507995in}{0.986912in}}{\pgfqpoint{2.496945in}{0.986912in}}%
\pgfpathcurveto{\pgfqpoint{2.485895in}{0.986912in}}{\pgfqpoint{2.475296in}{0.982522in}}{\pgfqpoint{2.467482in}{0.974708in}}%
\pgfpathcurveto{\pgfqpoint{2.459669in}{0.966895in}}{\pgfqpoint{2.455278in}{0.956296in}}{\pgfqpoint{2.455278in}{0.945245in}}%
\pgfpathcurveto{\pgfqpoint{2.455278in}{0.934195in}}{\pgfqpoint{2.459669in}{0.923596in}}{\pgfqpoint{2.467482in}{0.915783in}}%
\pgfpathcurveto{\pgfqpoint{2.475296in}{0.907969in}}{\pgfqpoint{2.485895in}{0.903579in}}{\pgfqpoint{2.496945in}{0.903579in}}%
\pgfpathclose%
\pgfusepath{stroke,fill}%
\end{pgfscope}%
\begin{pgfscope}%
\pgfpathrectangle{\pgfqpoint{0.772069in}{0.515123in}}{\pgfqpoint{3.875000in}{2.695000in}}%
\pgfusepath{clip}%
\pgfsetbuttcap%
\pgfsetroundjoin%
\definecolor{currentfill}{rgb}{1.000000,0.843137,0.000000}%
\pgfsetfillcolor{currentfill}%
\pgfsetlinewidth{1.003750pt}%
\definecolor{currentstroke}{rgb}{1.000000,0.843137,0.000000}%
\pgfsetstrokecolor{currentstroke}%
\pgfsetdash{}{0pt}%
\pgfpathmoveto{\pgfqpoint{2.922192in}{0.975588in}}%
\pgfpathcurveto{\pgfqpoint{2.933242in}{0.975588in}}{\pgfqpoint{2.943841in}{0.979979in}}{\pgfqpoint{2.951655in}{0.987792in}}%
\pgfpathcurveto{\pgfqpoint{2.959469in}{0.995606in}}{\pgfqpoint{2.963859in}{1.006205in}}{\pgfqpoint{2.963859in}{1.017255in}}%
\pgfpathcurveto{\pgfqpoint{2.963859in}{1.028305in}}{\pgfqpoint{2.959469in}{1.038904in}}{\pgfqpoint{2.951655in}{1.046718in}}%
\pgfpathcurveto{\pgfqpoint{2.943841in}{1.054531in}}{\pgfqpoint{2.933242in}{1.058922in}}{\pgfqpoint{2.922192in}{1.058922in}}%
\pgfpathcurveto{\pgfqpoint{2.911142in}{1.058922in}}{\pgfqpoint{2.900543in}{1.054531in}}{\pgfqpoint{2.892730in}{1.046718in}}%
\pgfpathcurveto{\pgfqpoint{2.884916in}{1.038904in}}{\pgfqpoint{2.880526in}{1.028305in}}{\pgfqpoint{2.880526in}{1.017255in}}%
\pgfpathcurveto{\pgfqpoint{2.880526in}{1.006205in}}{\pgfqpoint{2.884916in}{0.995606in}}{\pgfqpoint{2.892730in}{0.987792in}}%
\pgfpathcurveto{\pgfqpoint{2.900543in}{0.979979in}}{\pgfqpoint{2.911142in}{0.975588in}}{\pgfqpoint{2.922192in}{0.975588in}}%
\pgfpathclose%
\pgfusepath{stroke,fill}%
\end{pgfscope}%
\begin{pgfscope}%
\pgfpathrectangle{\pgfqpoint{0.772069in}{0.515123in}}{\pgfqpoint{3.875000in}{2.695000in}}%
\pgfusepath{clip}%
\pgfsetbuttcap%
\pgfsetroundjoin%
\definecolor{currentfill}{rgb}{1.000000,0.843137,0.000000}%
\pgfsetfillcolor{currentfill}%
\pgfsetlinewidth{1.003750pt}%
\definecolor{currentstroke}{rgb}{1.000000,0.843137,0.000000}%
\pgfsetstrokecolor{currentstroke}%
\pgfsetdash{}{0pt}%
\pgfpathmoveto{\pgfqpoint{3.286690in}{1.039733in}}%
\pgfpathcurveto{\pgfqpoint{3.297740in}{1.039733in}}{\pgfqpoint{3.308339in}{1.044124in}}{\pgfqpoint{3.316153in}{1.051937in}}%
\pgfpathcurveto{\pgfqpoint{3.323966in}{1.059751in}}{\pgfqpoint{3.328357in}{1.070350in}}{\pgfqpoint{3.328357in}{1.081400in}}%
\pgfpathcurveto{\pgfqpoint{3.328357in}{1.092450in}}{\pgfqpoint{3.323966in}{1.103049in}}{\pgfqpoint{3.316153in}{1.110863in}}%
\pgfpathcurveto{\pgfqpoint{3.308339in}{1.118676in}}{\pgfqpoint{3.297740in}{1.123067in}}{\pgfqpoint{3.286690in}{1.123067in}}%
\pgfpathcurveto{\pgfqpoint{3.275640in}{1.123067in}}{\pgfqpoint{3.265041in}{1.118676in}}{\pgfqpoint{3.257227in}{1.110863in}}%
\pgfpathcurveto{\pgfqpoint{3.249414in}{1.103049in}}{\pgfqpoint{3.245023in}{1.092450in}}{\pgfqpoint{3.245023in}{1.081400in}}%
\pgfpathcurveto{\pgfqpoint{3.245023in}{1.070350in}}{\pgfqpoint{3.249414in}{1.059751in}}{\pgfqpoint{3.257227in}{1.051937in}}%
\pgfpathcurveto{\pgfqpoint{3.265041in}{1.044124in}}{\pgfqpoint{3.275640in}{1.039733in}}{\pgfqpoint{3.286690in}{1.039733in}}%
\pgfpathclose%
\pgfusepath{stroke,fill}%
\end{pgfscope}%
\begin{pgfscope}%
\pgfpathrectangle{\pgfqpoint{0.772069in}{0.515123in}}{\pgfqpoint{3.875000in}{2.695000in}}%
\pgfusepath{clip}%
\pgfsetbuttcap%
\pgfsetroundjoin%
\definecolor{currentfill}{rgb}{1.000000,0.843137,0.000000}%
\pgfsetfillcolor{currentfill}%
\pgfsetlinewidth{1.003750pt}%
\definecolor{currentstroke}{rgb}{1.000000,0.843137,0.000000}%
\pgfsetstrokecolor{currentstroke}%
\pgfsetdash{}{0pt}%
\pgfpathmoveto{\pgfqpoint{3.681562in}{1.104124in}}%
\pgfpathcurveto{\pgfqpoint{3.692613in}{1.104124in}}{\pgfqpoint{3.703212in}{1.108514in}}{\pgfqpoint{3.711025in}{1.116328in}}%
\pgfpathcurveto{\pgfqpoint{3.718839in}{1.124142in}}{\pgfqpoint{3.723229in}{1.134741in}}{\pgfqpoint{3.723229in}{1.145791in}}%
\pgfpathcurveto{\pgfqpoint{3.723229in}{1.156841in}}{\pgfqpoint{3.718839in}{1.167440in}}{\pgfqpoint{3.711025in}{1.175254in}}%
\pgfpathcurveto{\pgfqpoint{3.703212in}{1.183067in}}{\pgfqpoint{3.692613in}{1.187457in}}{\pgfqpoint{3.681562in}{1.187457in}}%
\pgfpathcurveto{\pgfqpoint{3.670512in}{1.187457in}}{\pgfqpoint{3.659913in}{1.183067in}}{\pgfqpoint{3.652100in}{1.175254in}}%
\pgfpathcurveto{\pgfqpoint{3.644286in}{1.167440in}}{\pgfqpoint{3.639896in}{1.156841in}}{\pgfqpoint{3.639896in}{1.145791in}}%
\pgfpathcurveto{\pgfqpoint{3.639896in}{1.134741in}}{\pgfqpoint{3.644286in}{1.124142in}}{\pgfqpoint{3.652100in}{1.116328in}}%
\pgfpathcurveto{\pgfqpoint{3.659913in}{1.108514in}}{\pgfqpoint{3.670512in}{1.104124in}}{\pgfqpoint{3.681562in}{1.104124in}}%
\pgfpathclose%
\pgfusepath{stroke,fill}%
\end{pgfscope}%
\begin{pgfscope}%
\pgfpathrectangle{\pgfqpoint{0.772069in}{0.515123in}}{\pgfqpoint{3.875000in}{2.695000in}}%
\pgfusepath{clip}%
\pgfsetbuttcap%
\pgfsetroundjoin%
\definecolor{currentfill}{rgb}{1.000000,0.843137,0.000000}%
\pgfsetfillcolor{currentfill}%
\pgfsetlinewidth{1.003750pt}%
\definecolor{currentstroke}{rgb}{1.000000,0.843137,0.000000}%
\pgfsetstrokecolor{currentstroke}%
\pgfsetdash{}{0pt}%
\pgfpathmoveto{\pgfqpoint{4.015685in}{1.165566in}}%
\pgfpathcurveto{\pgfqpoint{4.026735in}{1.165566in}}{\pgfqpoint{4.037334in}{1.169956in}}{\pgfqpoint{4.045148in}{1.177770in}}%
\pgfpathcurveto{\pgfqpoint{4.052962in}{1.185583in}}{\pgfqpoint{4.057352in}{1.196182in}}{\pgfqpoint{4.057352in}{1.207232in}}%
\pgfpathcurveto{\pgfqpoint{4.057352in}{1.218282in}}{\pgfqpoint{4.052962in}{1.228882in}}{\pgfqpoint{4.045148in}{1.236695in}}%
\pgfpathcurveto{\pgfqpoint{4.037334in}{1.244509in}}{\pgfqpoint{4.026735in}{1.248899in}}{\pgfqpoint{4.015685in}{1.248899in}}%
\pgfpathcurveto{\pgfqpoint{4.004635in}{1.248899in}}{\pgfqpoint{3.994036in}{1.244509in}}{\pgfqpoint{3.986222in}{1.236695in}}%
\pgfpathcurveto{\pgfqpoint{3.978409in}{1.228882in}}{\pgfqpoint{3.974019in}{1.218282in}}{\pgfqpoint{3.974019in}{1.207232in}}%
\pgfpathcurveto{\pgfqpoint{3.974019in}{1.196182in}}{\pgfqpoint{3.978409in}{1.185583in}}{\pgfqpoint{3.986222in}{1.177770in}}%
\pgfpathcurveto{\pgfqpoint{3.994036in}{1.169956in}}{\pgfqpoint{4.004635in}{1.165566in}}{\pgfqpoint{4.015685in}{1.165566in}}%
\pgfpathclose%
\pgfusepath{stroke,fill}%
\end{pgfscope}%
\begin{pgfscope}%
\pgfpathrectangle{\pgfqpoint{0.772069in}{0.515123in}}{\pgfqpoint{3.875000in}{2.695000in}}%
\pgfusepath{clip}%
\pgfsetbuttcap%
\pgfsetroundjoin%
\definecolor{currentfill}{rgb}{1.000000,0.843137,0.000000}%
\pgfsetfillcolor{currentfill}%
\pgfsetlinewidth{1.003750pt}%
\definecolor{currentstroke}{rgb}{1.000000,0.843137,0.000000}%
\pgfsetstrokecolor{currentstroke}%
\pgfsetdash{}{0pt}%
\pgfpathmoveto{\pgfqpoint{4.440932in}{1.239296in}}%
\pgfpathcurveto{\pgfqpoint{4.451983in}{1.239296in}}{\pgfqpoint{4.462582in}{1.243686in}}{\pgfqpoint{4.470395in}{1.251499in}}%
\pgfpathcurveto{\pgfqpoint{4.478209in}{1.259313in}}{\pgfqpoint{4.482599in}{1.269912in}}{\pgfqpoint{4.482599in}{1.280962in}}%
\pgfpathcurveto{\pgfqpoint{4.482599in}{1.292012in}}{\pgfqpoint{4.478209in}{1.302611in}}{\pgfqpoint{4.470395in}{1.310425in}}%
\pgfpathcurveto{\pgfqpoint{4.462582in}{1.318239in}}{\pgfqpoint{4.451983in}{1.322629in}}{\pgfqpoint{4.440932in}{1.322629in}}%
\pgfpathcurveto{\pgfqpoint{4.429882in}{1.322629in}}{\pgfqpoint{4.419283in}{1.318239in}}{\pgfqpoint{4.411470in}{1.310425in}}%
\pgfpathcurveto{\pgfqpoint{4.403656in}{1.302611in}}{\pgfqpoint{4.399266in}{1.292012in}}{\pgfqpoint{4.399266in}{1.280962in}}%
\pgfpathcurveto{\pgfqpoint{4.399266in}{1.269912in}}{\pgfqpoint{4.403656in}{1.259313in}}{\pgfqpoint{4.411470in}{1.251499in}}%
\pgfpathcurveto{\pgfqpoint{4.419283in}{1.243686in}}{\pgfqpoint{4.429882in}{1.239296in}}{\pgfqpoint{4.440932in}{1.239296in}}%
\pgfpathclose%
\pgfusepath{stroke,fill}%
\end{pgfscope}%
\begin{pgfscope}%
\pgfpathrectangle{\pgfqpoint{0.772069in}{0.515123in}}{\pgfqpoint{3.875000in}{2.695000in}}%
\pgfusepath{clip}%
\pgfsetbuttcap%
\pgfsetroundjoin%
\definecolor{currentfill}{rgb}{0.196078,0.803922,0.196078}%
\pgfsetfillcolor{currentfill}%
\pgfsetlinewidth{1.003750pt}%
\definecolor{currentstroke}{rgb}{0.196078,0.803922,0.196078}%
\pgfsetstrokecolor{currentstroke}%
\pgfsetdash{}{0pt}%
\pgfpathmoveto{\pgfqpoint{0.978205in}{0.694432in}}%
\pgfpathcurveto{\pgfqpoint{0.989255in}{0.694432in}}{\pgfqpoint{0.999854in}{0.698822in}}{\pgfqpoint{1.007668in}{0.706635in}}%
\pgfpathcurveto{\pgfqpoint{1.015481in}{0.714449in}}{\pgfqpoint{1.019872in}{0.725048in}}{\pgfqpoint{1.019872in}{0.736098in}}%
\pgfpathcurveto{\pgfqpoint{1.019872in}{0.747148in}}{\pgfqpoint{1.015481in}{0.757747in}}{\pgfqpoint{1.007668in}{0.765561in}}%
\pgfpathcurveto{\pgfqpoint{0.999854in}{0.773375in}}{\pgfqpoint{0.989255in}{0.777765in}}{\pgfqpoint{0.978205in}{0.777765in}}%
\pgfpathcurveto{\pgfqpoint{0.967155in}{0.777765in}}{\pgfqpoint{0.956556in}{0.773375in}}{\pgfqpoint{0.948742in}{0.765561in}}%
\pgfpathcurveto{\pgfqpoint{0.940929in}{0.757747in}}{\pgfqpoint{0.936538in}{0.747148in}}{\pgfqpoint{0.936538in}{0.736098in}}%
\pgfpathcurveto{\pgfqpoint{0.936538in}{0.725048in}}{\pgfqpoint{0.940929in}{0.714449in}}{\pgfqpoint{0.948742in}{0.706635in}}%
\pgfpathcurveto{\pgfqpoint{0.956556in}{0.698822in}}{\pgfqpoint{0.967155in}{0.694432in}}{\pgfqpoint{0.978205in}{0.694432in}}%
\pgfpathclose%
\pgfusepath{stroke,fill}%
\end{pgfscope}%
\begin{pgfscope}%
\pgfpathrectangle{\pgfqpoint{0.772069in}{0.515123in}}{\pgfqpoint{3.875000in}{2.695000in}}%
\pgfusepath{clip}%
\pgfsetbuttcap%
\pgfsetroundjoin%
\definecolor{currentfill}{rgb}{0.196078,0.803922,0.196078}%
\pgfsetfillcolor{currentfill}%
\pgfsetlinewidth{1.003750pt}%
\definecolor{currentstroke}{rgb}{0.196078,0.803922,0.196078}%
\pgfsetstrokecolor{currentstroke}%
\pgfsetdash{}{0pt}%
\pgfpathmoveto{\pgfqpoint{1.342703in}{0.814366in}}%
\pgfpathcurveto{\pgfqpoint{1.353753in}{0.814366in}}{\pgfqpoint{1.364352in}{0.818756in}}{\pgfqpoint{1.372165in}{0.826569in}}%
\pgfpathcurveto{\pgfqpoint{1.379979in}{0.834383in}}{\pgfqpoint{1.384369in}{0.844982in}}{\pgfqpoint{1.384369in}{0.856032in}}%
\pgfpathcurveto{\pgfqpoint{1.384369in}{0.867082in}}{\pgfqpoint{1.379979in}{0.877681in}}{\pgfqpoint{1.372165in}{0.885495in}}%
\pgfpathcurveto{\pgfqpoint{1.364352in}{0.893309in}}{\pgfqpoint{1.353753in}{0.897699in}}{\pgfqpoint{1.342703in}{0.897699in}}%
\pgfpathcurveto{\pgfqpoint{1.331652in}{0.897699in}}{\pgfqpoint{1.321053in}{0.893309in}}{\pgfqpoint{1.313240in}{0.885495in}}%
\pgfpathcurveto{\pgfqpoint{1.305426in}{0.877681in}}{\pgfqpoint{1.301036in}{0.867082in}}{\pgfqpoint{1.301036in}{0.856032in}}%
\pgfpathcurveto{\pgfqpoint{1.301036in}{0.844982in}}{\pgfqpoint{1.305426in}{0.834383in}}{\pgfqpoint{1.313240in}{0.826569in}}%
\pgfpathcurveto{\pgfqpoint{1.321053in}{0.818756in}}{\pgfqpoint{1.331652in}{0.814366in}}{\pgfqpoint{1.342703in}{0.814366in}}%
\pgfpathclose%
\pgfusepath{stroke,fill}%
\end{pgfscope}%
\begin{pgfscope}%
\pgfpathrectangle{\pgfqpoint{0.772069in}{0.515123in}}{\pgfqpoint{3.875000in}{2.695000in}}%
\pgfusepath{clip}%
\pgfsetbuttcap%
\pgfsetroundjoin%
\definecolor{currentfill}{rgb}{0.196078,0.803922,0.196078}%
\pgfsetfillcolor{currentfill}%
\pgfsetlinewidth{1.003750pt}%
\definecolor{currentstroke}{rgb}{0.196078,0.803922,0.196078}%
\pgfsetstrokecolor{currentstroke}%
\pgfsetdash{}{0pt}%
\pgfpathmoveto{\pgfqpoint{1.737575in}{0.935037in}}%
\pgfpathcurveto{\pgfqpoint{1.748625in}{0.935037in}}{\pgfqpoint{1.759224in}{0.939427in}}{\pgfqpoint{1.767038in}{0.947241in}}%
\pgfpathcurveto{\pgfqpoint{1.774851in}{0.955054in}}{\pgfqpoint{1.779242in}{0.965653in}}{\pgfqpoint{1.779242in}{0.976703in}}%
\pgfpathcurveto{\pgfqpoint{1.779242in}{0.987754in}}{\pgfqpoint{1.774851in}{0.998353in}}{\pgfqpoint{1.767038in}{1.006166in}}%
\pgfpathcurveto{\pgfqpoint{1.759224in}{1.013980in}}{\pgfqpoint{1.748625in}{1.018370in}}{\pgfqpoint{1.737575in}{1.018370in}}%
\pgfpathcurveto{\pgfqpoint{1.726525in}{1.018370in}}{\pgfqpoint{1.715926in}{1.013980in}}{\pgfqpoint{1.708112in}{1.006166in}}%
\pgfpathcurveto{\pgfqpoint{1.700299in}{0.998353in}}{\pgfqpoint{1.695908in}{0.987754in}}{\pgfqpoint{1.695908in}{0.976703in}}%
\pgfpathcurveto{\pgfqpoint{1.695908in}{0.965653in}}{\pgfqpoint{1.700299in}{0.955054in}}{\pgfqpoint{1.708112in}{0.947241in}}%
\pgfpathcurveto{\pgfqpoint{1.715926in}{0.939427in}}{\pgfqpoint{1.726525in}{0.935037in}}{\pgfqpoint{1.737575in}{0.935037in}}%
\pgfpathclose%
\pgfusepath{stroke,fill}%
\end{pgfscope}%
\begin{pgfscope}%
\pgfpathrectangle{\pgfqpoint{0.772069in}{0.515123in}}{\pgfqpoint{3.875000in}{2.695000in}}%
\pgfusepath{clip}%
\pgfsetbuttcap%
\pgfsetroundjoin%
\definecolor{currentfill}{rgb}{0.196078,0.803922,0.196078}%
\pgfsetfillcolor{currentfill}%
\pgfsetlinewidth{1.003750pt}%
\definecolor{currentstroke}{rgb}{0.196078,0.803922,0.196078}%
\pgfsetstrokecolor{currentstroke}%
\pgfsetdash{}{0pt}%
\pgfpathmoveto{\pgfqpoint{2.132447in}{1.054971in}}%
\pgfpathcurveto{\pgfqpoint{2.143498in}{1.054971in}}{\pgfqpoint{2.154097in}{1.059361in}}{\pgfqpoint{2.161910in}{1.067175in}}%
\pgfpathcurveto{\pgfqpoint{2.169724in}{1.074988in}}{\pgfqpoint{2.174114in}{1.085587in}}{\pgfqpoint{2.174114in}{1.096637in}}%
\pgfpathcurveto{\pgfqpoint{2.174114in}{1.107688in}}{\pgfqpoint{2.169724in}{1.118287in}}{\pgfqpoint{2.161910in}{1.126100in}}%
\pgfpathcurveto{\pgfqpoint{2.154097in}{1.133914in}}{\pgfqpoint{2.143498in}{1.138304in}}{\pgfqpoint{2.132447in}{1.138304in}}%
\pgfpathcurveto{\pgfqpoint{2.121397in}{1.138304in}}{\pgfqpoint{2.110798in}{1.133914in}}{\pgfqpoint{2.102985in}{1.126100in}}%
\pgfpathcurveto{\pgfqpoint{2.095171in}{1.118287in}}{\pgfqpoint{2.090781in}{1.107688in}}{\pgfqpoint{2.090781in}{1.096637in}}%
\pgfpathcurveto{\pgfqpoint{2.090781in}{1.085587in}}{\pgfqpoint{2.095171in}{1.074988in}}{\pgfqpoint{2.102985in}{1.067175in}}%
\pgfpathcurveto{\pgfqpoint{2.110798in}{1.059361in}}{\pgfqpoint{2.121397in}{1.054971in}}{\pgfqpoint{2.132447in}{1.054971in}}%
\pgfpathclose%
\pgfusepath{stroke,fill}%
\end{pgfscope}%
\begin{pgfscope}%
\pgfpathrectangle{\pgfqpoint{0.772069in}{0.515123in}}{\pgfqpoint{3.875000in}{2.695000in}}%
\pgfusepath{clip}%
\pgfsetbuttcap%
\pgfsetroundjoin%
\definecolor{currentfill}{rgb}{0.196078,0.803922,0.196078}%
\pgfsetfillcolor{currentfill}%
\pgfsetlinewidth{1.003750pt}%
\definecolor{currentstroke}{rgb}{0.196078,0.803922,0.196078}%
\pgfsetstrokecolor{currentstroke}%
\pgfsetdash{}{0pt}%
\pgfpathmoveto{\pgfqpoint{2.496945in}{1.172939in}}%
\pgfpathcurveto{\pgfqpoint{2.507995in}{1.172939in}}{\pgfqpoint{2.518594in}{1.177329in}}{\pgfqpoint{2.526408in}{1.185143in}}%
\pgfpathcurveto{\pgfqpoint{2.534221in}{1.192956in}}{\pgfqpoint{2.538612in}{1.203555in}}{\pgfqpoint{2.538612in}{1.214605in}}%
\pgfpathcurveto{\pgfqpoint{2.538612in}{1.225655in}}{\pgfqpoint{2.534221in}{1.236255in}}{\pgfqpoint{2.526408in}{1.244068in}}%
\pgfpathcurveto{\pgfqpoint{2.518594in}{1.251882in}}{\pgfqpoint{2.507995in}{1.256272in}}{\pgfqpoint{2.496945in}{1.256272in}}%
\pgfpathcurveto{\pgfqpoint{2.485895in}{1.256272in}}{\pgfqpoint{2.475296in}{1.251882in}}{\pgfqpoint{2.467482in}{1.244068in}}%
\pgfpathcurveto{\pgfqpoint{2.459669in}{1.236255in}}{\pgfqpoint{2.455278in}{1.225655in}}{\pgfqpoint{2.455278in}{1.214605in}}%
\pgfpathcurveto{\pgfqpoint{2.455278in}{1.203555in}}{\pgfqpoint{2.459669in}{1.192956in}}{\pgfqpoint{2.467482in}{1.185143in}}%
\pgfpathcurveto{\pgfqpoint{2.475296in}{1.177329in}}{\pgfqpoint{2.485895in}{1.172939in}}{\pgfqpoint{2.496945in}{1.172939in}}%
\pgfpathclose%
\pgfusepath{stroke,fill}%
\end{pgfscope}%
\begin{pgfscope}%
\pgfpathrectangle{\pgfqpoint{0.772069in}{0.515123in}}{\pgfqpoint{3.875000in}{2.695000in}}%
\pgfusepath{clip}%
\pgfsetbuttcap%
\pgfsetroundjoin%
\definecolor{currentfill}{rgb}{0.196078,0.803922,0.196078}%
\pgfsetfillcolor{currentfill}%
\pgfsetlinewidth{1.003750pt}%
\definecolor{currentstroke}{rgb}{0.196078,0.803922,0.196078}%
\pgfsetstrokecolor{currentstroke}%
\pgfsetdash{}{0pt}%
\pgfpathmoveto{\pgfqpoint{2.922192in}{1.303195in}}%
\pgfpathcurveto{\pgfqpoint{2.933242in}{1.303195in}}{\pgfqpoint{2.943841in}{1.307585in}}{\pgfqpoint{2.951655in}{1.315399in}}%
\pgfpathcurveto{\pgfqpoint{2.959469in}{1.323212in}}{\pgfqpoint{2.963859in}{1.333811in}}{\pgfqpoint{2.963859in}{1.344862in}}%
\pgfpathcurveto{\pgfqpoint{2.963859in}{1.355912in}}{\pgfqpoint{2.959469in}{1.366511in}}{\pgfqpoint{2.951655in}{1.374324in}}%
\pgfpathcurveto{\pgfqpoint{2.943841in}{1.382138in}}{\pgfqpoint{2.933242in}{1.386528in}}{\pgfqpoint{2.922192in}{1.386528in}}%
\pgfpathcurveto{\pgfqpoint{2.911142in}{1.386528in}}{\pgfqpoint{2.900543in}{1.382138in}}{\pgfqpoint{2.892730in}{1.374324in}}%
\pgfpathcurveto{\pgfqpoint{2.884916in}{1.366511in}}{\pgfqpoint{2.880526in}{1.355912in}}{\pgfqpoint{2.880526in}{1.344862in}}%
\pgfpathcurveto{\pgfqpoint{2.880526in}{1.333811in}}{\pgfqpoint{2.884916in}{1.323212in}}{\pgfqpoint{2.892730in}{1.315399in}}%
\pgfpathcurveto{\pgfqpoint{2.900543in}{1.307585in}}{\pgfqpoint{2.911142in}{1.303195in}}{\pgfqpoint{2.922192in}{1.303195in}}%
\pgfpathclose%
\pgfusepath{stroke,fill}%
\end{pgfscope}%
\begin{pgfscope}%
\pgfpathrectangle{\pgfqpoint{0.772069in}{0.515123in}}{\pgfqpoint{3.875000in}{2.695000in}}%
\pgfusepath{clip}%
\pgfsetbuttcap%
\pgfsetroundjoin%
\definecolor{currentfill}{rgb}{0.196078,0.803922,0.196078}%
\pgfsetfillcolor{currentfill}%
\pgfsetlinewidth{1.003750pt}%
\definecolor{currentstroke}{rgb}{0.196078,0.803922,0.196078}%
\pgfsetstrokecolor{currentstroke}%
\pgfsetdash{}{0pt}%
\pgfpathmoveto{\pgfqpoint{3.286690in}{1.418705in}}%
\pgfpathcurveto{\pgfqpoint{3.297740in}{1.418705in}}{\pgfqpoint{3.308339in}{1.423095in}}{\pgfqpoint{3.316153in}{1.430909in}}%
\pgfpathcurveto{\pgfqpoint{3.323966in}{1.438723in}}{\pgfqpoint{3.328357in}{1.449322in}}{\pgfqpoint{3.328357in}{1.460372in}}%
\pgfpathcurveto{\pgfqpoint{3.328357in}{1.471422in}}{\pgfqpoint{3.323966in}{1.482021in}}{\pgfqpoint{3.316153in}{1.489835in}}%
\pgfpathcurveto{\pgfqpoint{3.308339in}{1.497648in}}{\pgfqpoint{3.297740in}{1.502038in}}{\pgfqpoint{3.286690in}{1.502038in}}%
\pgfpathcurveto{\pgfqpoint{3.275640in}{1.502038in}}{\pgfqpoint{3.265041in}{1.497648in}}{\pgfqpoint{3.257227in}{1.489835in}}%
\pgfpathcurveto{\pgfqpoint{3.249414in}{1.482021in}}{\pgfqpoint{3.245023in}{1.471422in}}{\pgfqpoint{3.245023in}{1.460372in}}%
\pgfpathcurveto{\pgfqpoint{3.245023in}{1.449322in}}{\pgfqpoint{3.249414in}{1.438723in}}{\pgfqpoint{3.257227in}{1.430909in}}%
\pgfpathcurveto{\pgfqpoint{3.265041in}{1.423095in}}{\pgfqpoint{3.275640in}{1.418705in}}{\pgfqpoint{3.286690in}{1.418705in}}%
\pgfpathclose%
\pgfusepath{stroke,fill}%
\end{pgfscope}%
\begin{pgfscope}%
\pgfpathrectangle{\pgfqpoint{0.772069in}{0.515123in}}{\pgfqpoint{3.875000in}{2.695000in}}%
\pgfusepath{clip}%
\pgfsetbuttcap%
\pgfsetroundjoin%
\definecolor{currentfill}{rgb}{0.196078,0.803922,0.196078}%
\pgfsetfillcolor{currentfill}%
\pgfsetlinewidth{1.003750pt}%
\definecolor{currentstroke}{rgb}{0.196078,0.803922,0.196078}%
\pgfsetstrokecolor{currentstroke}%
\pgfsetdash{}{0pt}%
\pgfpathmoveto{\pgfqpoint{3.681562in}{1.539131in}}%
\pgfpathcurveto{\pgfqpoint{3.692613in}{1.539131in}}{\pgfqpoint{3.703212in}{1.543521in}}{\pgfqpoint{3.711025in}{1.551334in}}%
\pgfpathcurveto{\pgfqpoint{3.718839in}{1.559148in}}{\pgfqpoint{3.723229in}{1.569747in}}{\pgfqpoint{3.723229in}{1.580797in}}%
\pgfpathcurveto{\pgfqpoint{3.723229in}{1.591847in}}{\pgfqpoint{3.718839in}{1.602446in}}{\pgfqpoint{3.711025in}{1.610260in}}%
\pgfpathcurveto{\pgfqpoint{3.703212in}{1.618074in}}{\pgfqpoint{3.692613in}{1.622464in}}{\pgfqpoint{3.681562in}{1.622464in}}%
\pgfpathcurveto{\pgfqpoint{3.670512in}{1.622464in}}{\pgfqpoint{3.659913in}{1.618074in}}{\pgfqpoint{3.652100in}{1.610260in}}%
\pgfpathcurveto{\pgfqpoint{3.644286in}{1.602446in}}{\pgfqpoint{3.639896in}{1.591847in}}{\pgfqpoint{3.639896in}{1.580797in}}%
\pgfpathcurveto{\pgfqpoint{3.639896in}{1.569747in}}{\pgfqpoint{3.644286in}{1.559148in}}{\pgfqpoint{3.652100in}{1.551334in}}%
\pgfpathcurveto{\pgfqpoint{3.659913in}{1.543521in}}{\pgfqpoint{3.670512in}{1.539131in}}{\pgfqpoint{3.681562in}{1.539131in}}%
\pgfpathclose%
\pgfusepath{stroke,fill}%
\end{pgfscope}%
\begin{pgfscope}%
\pgfpathrectangle{\pgfqpoint{0.772069in}{0.515123in}}{\pgfqpoint{3.875000in}{2.695000in}}%
\pgfusepath{clip}%
\pgfsetbuttcap%
\pgfsetroundjoin%
\definecolor{currentfill}{rgb}{0.196078,0.803922,0.196078}%
\pgfsetfillcolor{currentfill}%
\pgfsetlinewidth{1.003750pt}%
\definecolor{currentstroke}{rgb}{0.196078,0.803922,0.196078}%
\pgfsetstrokecolor{currentstroke}%
\pgfsetdash{}{0pt}%
\pgfpathmoveto{\pgfqpoint{4.015685in}{1.652183in}}%
\pgfpathcurveto{\pgfqpoint{4.026735in}{1.652183in}}{\pgfqpoint{4.037334in}{1.656573in}}{\pgfqpoint{4.045148in}{1.664387in}}%
\pgfpathcurveto{\pgfqpoint{4.052962in}{1.672201in}}{\pgfqpoint{4.057352in}{1.682800in}}{\pgfqpoint{4.057352in}{1.693850in}}%
\pgfpathcurveto{\pgfqpoint{4.057352in}{1.704900in}}{\pgfqpoint{4.052962in}{1.715499in}}{\pgfqpoint{4.045148in}{1.723313in}}%
\pgfpathcurveto{\pgfqpoint{4.037334in}{1.731126in}}{\pgfqpoint{4.026735in}{1.735516in}}{\pgfqpoint{4.015685in}{1.735516in}}%
\pgfpathcurveto{\pgfqpoint{4.004635in}{1.735516in}}{\pgfqpoint{3.994036in}{1.731126in}}{\pgfqpoint{3.986222in}{1.723313in}}%
\pgfpathcurveto{\pgfqpoint{3.978409in}{1.715499in}}{\pgfqpoint{3.974019in}{1.704900in}}{\pgfqpoint{3.974019in}{1.693850in}}%
\pgfpathcurveto{\pgfqpoint{3.974019in}{1.682800in}}{\pgfqpoint{3.978409in}{1.672201in}}{\pgfqpoint{3.986222in}{1.664387in}}%
\pgfpathcurveto{\pgfqpoint{3.994036in}{1.656573in}}{\pgfqpoint{4.004635in}{1.652183in}}{\pgfqpoint{4.015685in}{1.652183in}}%
\pgfpathclose%
\pgfusepath{stroke,fill}%
\end{pgfscope}%
\begin{pgfscope}%
\pgfpathrectangle{\pgfqpoint{0.772069in}{0.515123in}}{\pgfqpoint{3.875000in}{2.695000in}}%
\pgfusepath{clip}%
\pgfsetbuttcap%
\pgfsetroundjoin%
\definecolor{currentfill}{rgb}{0.196078,0.803922,0.196078}%
\pgfsetfillcolor{currentfill}%
\pgfsetlinewidth{1.003750pt}%
\definecolor{currentstroke}{rgb}{0.196078,0.803922,0.196078}%
\pgfsetstrokecolor{currentstroke}%
\pgfsetdash{}{0pt}%
\pgfpathmoveto{\pgfqpoint{4.440932in}{1.782439in}}%
\pgfpathcurveto{\pgfqpoint{4.451983in}{1.782439in}}{\pgfqpoint{4.462582in}{1.786830in}}{\pgfqpoint{4.470395in}{1.794643in}}%
\pgfpathcurveto{\pgfqpoint{4.478209in}{1.802457in}}{\pgfqpoint{4.482599in}{1.813056in}}{\pgfqpoint{4.482599in}{1.824106in}}%
\pgfpathcurveto{\pgfqpoint{4.482599in}{1.835156in}}{\pgfqpoint{4.478209in}{1.845755in}}{\pgfqpoint{4.470395in}{1.853569in}}%
\pgfpathcurveto{\pgfqpoint{4.462582in}{1.861382in}}{\pgfqpoint{4.451983in}{1.865773in}}{\pgfqpoint{4.440932in}{1.865773in}}%
\pgfpathcurveto{\pgfqpoint{4.429882in}{1.865773in}}{\pgfqpoint{4.419283in}{1.861382in}}{\pgfqpoint{4.411470in}{1.853569in}}%
\pgfpathcurveto{\pgfqpoint{4.403656in}{1.845755in}}{\pgfqpoint{4.399266in}{1.835156in}}{\pgfqpoint{4.399266in}{1.824106in}}%
\pgfpathcurveto{\pgfqpoint{4.399266in}{1.813056in}}{\pgfqpoint{4.403656in}{1.802457in}}{\pgfqpoint{4.411470in}{1.794643in}}%
\pgfpathcurveto{\pgfqpoint{4.419283in}{1.786830in}}{\pgfqpoint{4.429882in}{1.782439in}}{\pgfqpoint{4.440932in}{1.782439in}}%
\pgfpathclose%
\pgfusepath{stroke,fill}%
\end{pgfscope}%
\begin{pgfscope}%
\pgfsetrectcap%
\pgfsetmiterjoin%
\pgfsetlinewidth{0.803000pt}%
\definecolor{currentstroke}{rgb}{0.000000,0.000000,0.000000}%
\pgfsetstrokecolor{currentstroke}%
\pgfsetdash{}{0pt}%
\pgfpathmoveto{\pgfqpoint{0.772069in}{0.515123in}}%
\pgfpathlineto{\pgfqpoint{0.772069in}{3.210123in}}%
\pgfusepath{stroke}%
\end{pgfscope}%
\begin{pgfscope}%
\pgfsetrectcap%
\pgfsetmiterjoin%
\pgfsetlinewidth{0.803000pt}%
\definecolor{currentstroke}{rgb}{0.000000,0.000000,0.000000}%
\pgfsetstrokecolor{currentstroke}%
\pgfsetdash{}{0pt}%
\pgfpathmoveto{\pgfqpoint{4.647069in}{0.515123in}}%
\pgfpathlineto{\pgfqpoint{4.647069in}{3.210123in}}%
\pgfusepath{stroke}%
\end{pgfscope}%
\begin{pgfscope}%
\pgfsetrectcap%
\pgfsetmiterjoin%
\pgfsetlinewidth{0.803000pt}%
\definecolor{currentstroke}{rgb}{0.000000,0.000000,0.000000}%
\pgfsetstrokecolor{currentstroke}%
\pgfsetdash{}{0pt}%
\pgfpathmoveto{\pgfqpoint{0.772069in}{0.515123in}}%
\pgfpathlineto{\pgfqpoint{4.647069in}{0.515123in}}%
\pgfusepath{stroke}%
\end{pgfscope}%
\begin{pgfscope}%
\pgfsetrectcap%
\pgfsetmiterjoin%
\pgfsetlinewidth{0.803000pt}%
\definecolor{currentstroke}{rgb}{0.000000,0.000000,0.000000}%
\pgfsetstrokecolor{currentstroke}%
\pgfsetdash{}{0pt}%
\pgfpathmoveto{\pgfqpoint{0.772069in}{3.210123in}}%
\pgfpathlineto{\pgfqpoint{4.647069in}{3.210123in}}%
\pgfusepath{stroke}%
\end{pgfscope}%
\end{pgfpicture}%
\makeatother%
\endgroup%

    \caption{Voltajes (\textcolor{Blue}{$V$}, \textcolor{Red}{$V_1$}, \textcolor{Yellow}{$V_2$}, \textcolor{Green}{$V_3$}) frente a intensidad (I)}
  \end{figure}

  \subsection{Ajuste por mínimos cuadrados}
  \label{sec:ajusteminimoscuadradosserie}

  Vuelve a aparecer lo que parece una relación lineal en la gráfica, por lo que pasaremos a hacer un ajuste por mínimos cuadrados. Cómo explicamos en el apartado inicial \ref{sec:reglin} y cómo ya aplicamos en la sección anterior, \ref{sec:ajusteminres}, calcularemos los parámetros \textit{a} y \textit{b} utilizando las ecuaciones \ref{ec:a} y \ref{ec:b}, para ello calculando diversos sumatorios utilizando el programa de \code{python} que describimos, cargando la tabla en .csv con nuestros datos y aplicando las fórmulas de los sumatorias, con lo que obtenemos los siguientes resultados:
  \begin{gather}
    \sum_i x_i = 6,98\cdot10^{-5} \nonumber \qquad \sum_i y_i = 55,6 \nonumber \qquad \sum_i x_iy_i = 4,93\cdot10^{-4} \nonumber \qquad \sum_i x_i^2 = 6,20\cdot10^{-10} \nonumber \\
    a = 4,71\cdot10^{-2} \nonumber \qquad b = 790000 \label{v:serie}
  \end{gather}
  El proceso sería el mismo para obtener los datos de las rectas de $V_1$, $V_2$ y $V_3$. Finalmente podríamos dibujar las gráficas, ahora con una recta creada por regresión lineal que se ajusta a los datos que obtuvimos.

  \begin{figure}[H]
    %\centering
    \hspace{2.5em} %% Creator: Matplotlib, PGF backend
%%
%% To include the figure in your LaTeX document, write
%%   \input{<filename>.pgf}
%%
%% Make sure the required packages are loaded in your preamble
%%   \usepackage{pgf}
%%
%% Figures using additional raster images can only be included by \input if
%% they are in the same directory as the main LaTeX file. For loading figures
%% from other directories you can use the `import` package
%%   \usepackage{import}
%% and then include the figures with
%%   \import{<path to file>}{<filename>.pgf}
%%
%% Matplotlib used the following preamble
%%
\begingroup%
\makeatletter%
\begin{pgfpicture}%
\pgfpathrectangle{\pgfpointorigin}{\pgfqpoint{4.747069in}{3.310123in}}%
\pgfusepath{use as bounding box, clip}%
\begin{pgfscope}%
\pgfsetbuttcap%
\pgfsetmiterjoin%
\definecolor{currentfill}{rgb}{1.000000,1.000000,1.000000}%
\pgfsetfillcolor{currentfill}%
\pgfsetlinewidth{0.000000pt}%
\definecolor{currentstroke}{rgb}{1.000000,1.000000,1.000000}%
\pgfsetstrokecolor{currentstroke}%
\pgfsetdash{}{0pt}%
\pgfpathmoveto{\pgfqpoint{0.000000in}{0.000000in}}%
\pgfpathlineto{\pgfqpoint{4.747069in}{0.000000in}}%
\pgfpathlineto{\pgfqpoint{4.747069in}{3.310123in}}%
\pgfpathlineto{\pgfqpoint{0.000000in}{3.310123in}}%
\pgfpathclose%
\pgfusepath{fill}%
\end{pgfscope}%
\begin{pgfscope}%
\pgfsetbuttcap%
\pgfsetmiterjoin%
\definecolor{currentfill}{rgb}{1.000000,1.000000,1.000000}%
\pgfsetfillcolor{currentfill}%
\pgfsetlinewidth{0.000000pt}%
\definecolor{currentstroke}{rgb}{0.000000,0.000000,0.000000}%
\pgfsetstrokecolor{currentstroke}%
\pgfsetstrokeopacity{0.000000}%
\pgfsetdash{}{0pt}%
\pgfpathmoveto{\pgfqpoint{0.772069in}{0.515123in}}%
\pgfpathlineto{\pgfqpoint{4.647069in}{0.515123in}}%
\pgfpathlineto{\pgfqpoint{4.647069in}{3.210123in}}%
\pgfpathlineto{\pgfqpoint{0.772069in}{3.210123in}}%
\pgfpathclose%
\pgfusepath{fill}%
\end{pgfscope}%
\begin{pgfscope}%
\pgfpathrectangle{\pgfqpoint{0.772069in}{0.515123in}}{\pgfqpoint{3.875000in}{2.695000in}}%
\pgfusepath{clip}%
\pgfsetrectcap%
\pgfsetroundjoin%
\pgfsetlinewidth{1.505625pt}%
\definecolor{currentstroke}{rgb}{0.529412,0.807843,0.921569}%
\pgfsetstrokecolor{currentstroke}%
\pgfsetdash{}{0pt}%
\pgfpathmoveto{\pgfqpoint{0.978205in}{0.860239in}}%
\pgfpathlineto{\pgfqpoint{1.362953in}{1.103430in}}%
\pgfpathlineto{\pgfqpoint{1.747700in}{1.346621in}}%
\pgfpathlineto{\pgfqpoint{2.132447in}{1.589812in}}%
\pgfpathlineto{\pgfqpoint{2.517195in}{1.833003in}}%
\pgfpathlineto{\pgfqpoint{2.901942in}{2.076194in}}%
\pgfpathlineto{\pgfqpoint{3.286690in}{2.319385in}}%
\pgfpathlineto{\pgfqpoint{3.671437in}{2.562576in}}%
\pgfpathlineto{\pgfqpoint{4.056185in}{2.805767in}}%
\pgfpathlineto{\pgfqpoint{4.440932in}{3.048958in}}%
\pgfusepath{stroke}%
\end{pgfscope}%
\begin{pgfscope}%
\pgfpathrectangle{\pgfqpoint{0.772069in}{0.515123in}}{\pgfqpoint{3.875000in}{2.695000in}}%
\pgfusepath{clip}%
\pgfsetrectcap%
\pgfsetroundjoin%
\pgfsetlinewidth{1.505625pt}%
\definecolor{currentstroke}{rgb}{1.000000,0.627451,0.478431}%
\pgfsetstrokecolor{currentstroke}%
\pgfsetdash{}{0pt}%
\pgfpathmoveto{\pgfqpoint{0.978205in}{0.665355in}}%
\pgfpathlineto{\pgfqpoint{1.362953in}{0.718660in}}%
\pgfpathlineto{\pgfqpoint{1.747700in}{0.771965in}}%
\pgfpathlineto{\pgfqpoint{2.132447in}{0.825270in}}%
\pgfpathlineto{\pgfqpoint{2.517195in}{0.878575in}}%
\pgfpathlineto{\pgfqpoint{2.901942in}{0.931880in}}%
\pgfpathlineto{\pgfqpoint{3.286690in}{0.985185in}}%
\pgfpathlineto{\pgfqpoint{3.671437in}{1.038490in}}%
\pgfpathlineto{\pgfqpoint{4.056185in}{1.091795in}}%
\pgfpathlineto{\pgfqpoint{4.440932in}{1.145100in}}%
\pgfusepath{stroke}%
\end{pgfscope}%
\begin{pgfscope}%
\pgfpathrectangle{\pgfqpoint{0.772069in}{0.515123in}}{\pgfqpoint{3.875000in}{2.695000in}}%
\pgfusepath{clip}%
\pgfsetrectcap%
\pgfsetroundjoin%
\pgfsetlinewidth{1.505625pt}%
\definecolor{currentstroke}{rgb}{1.000000,0.894118,0.709804}%
\pgfsetstrokecolor{currentstroke}%
\pgfsetdash{}{0pt}%
\pgfpathmoveto{\pgfqpoint{0.978205in}{0.678367in}}%
\pgfpathlineto{\pgfqpoint{1.362953in}{0.744350in}}%
\pgfpathlineto{\pgfqpoint{1.747700in}{0.810332in}}%
\pgfpathlineto{\pgfqpoint{2.132447in}{0.876315in}}%
\pgfpathlineto{\pgfqpoint{2.517195in}{0.942298in}}%
\pgfpathlineto{\pgfqpoint{2.901942in}{1.008280in}}%
\pgfpathlineto{\pgfqpoint{3.286690in}{1.074263in}}%
\pgfpathlineto{\pgfqpoint{3.671437in}{1.140246in}}%
\pgfpathlineto{\pgfqpoint{4.056185in}{1.206228in}}%
\pgfpathlineto{\pgfqpoint{4.440932in}{1.272211in}}%
\pgfusepath{stroke}%
\end{pgfscope}%
\begin{pgfscope}%
\pgfpathrectangle{\pgfqpoint{0.772069in}{0.515123in}}{\pgfqpoint{3.875000in}{2.695000in}}%
\pgfusepath{clip}%
\pgfsetrectcap%
\pgfsetroundjoin%
\pgfsetlinewidth{1.505625pt}%
\definecolor{currentstroke}{rgb}{0.564706,0.933333,0.564706}%
\pgfsetstrokecolor{currentstroke}%
\pgfsetdash{}{0pt}%
\pgfpathmoveto{\pgfqpoint{0.978205in}{0.733052in}}%
\pgfpathlineto{\pgfqpoint{1.362953in}{0.852317in}}%
\pgfpathlineto{\pgfqpoint{1.747700in}{0.971582in}}%
\pgfpathlineto{\pgfqpoint{2.132447in}{1.090847in}}%
\pgfpathlineto{\pgfqpoint{2.517195in}{1.210112in}}%
\pgfpathlineto{\pgfqpoint{2.901942in}{1.329377in}}%
\pgfpathlineto{\pgfqpoint{3.286690in}{1.448643in}}%
\pgfpathlineto{\pgfqpoint{3.671437in}{1.567908in}}%
\pgfpathlineto{\pgfqpoint{4.056185in}{1.687173in}}%
\pgfpathlineto{\pgfqpoint{4.440932in}{1.806438in}}%
\pgfusepath{stroke}%
\end{pgfscope}%
\begin{pgfscope}%
\pgfsetbuttcap%
\pgfsetroundjoin%
\definecolor{currentfill}{rgb}{0.000000,0.000000,0.000000}%
\pgfsetfillcolor{currentfill}%
\pgfsetlinewidth{0.803000pt}%
\definecolor{currentstroke}{rgb}{0.000000,0.000000,0.000000}%
\pgfsetstrokecolor{currentstroke}%
\pgfsetdash{}{0pt}%
\pgfsys@defobject{currentmarker}{\pgfqpoint{0.000000in}{-0.048611in}}{\pgfqpoint{0.000000in}{0.000000in}}{%
\pgfpathmoveto{\pgfqpoint{0.000000in}{0.000000in}}%
\pgfpathlineto{\pgfqpoint{0.000000in}{-0.048611in}}%
\pgfusepath{stroke,fill}%
}%
\begin{pgfscope}%
\pgfsys@transformshift{1.190829in}{0.515123in}%
\pgfsys@useobject{currentmarker}{}%
\end{pgfscope}%
\end{pgfscope}%
\begin{pgfscope}%
\definecolor{textcolor}{rgb}{0.000000,0.000000,0.000000}%
\pgfsetstrokecolor{textcolor}%
\pgfsetfillcolor{textcolor}%
\pgftext[x=1.190829in,y=0.417901in,,top]{\color{textcolor}\rmfamily\fontsize{10.000000}{12.000000}\selectfont \(\displaystyle 2\)}%
\end{pgfscope}%
\begin{pgfscope}%
\pgfsetbuttcap%
\pgfsetroundjoin%
\definecolor{currentfill}{rgb}{0.000000,0.000000,0.000000}%
\pgfsetfillcolor{currentfill}%
\pgfsetlinewidth{0.803000pt}%
\definecolor{currentstroke}{rgb}{0.000000,0.000000,0.000000}%
\pgfsetstrokecolor{currentstroke}%
\pgfsetdash{}{0pt}%
\pgfsys@defobject{currentmarker}{\pgfqpoint{0.000000in}{-0.048611in}}{\pgfqpoint{0.000000in}{0.000000in}}{%
\pgfpathmoveto{\pgfqpoint{0.000000in}{0.000000in}}%
\pgfpathlineto{\pgfqpoint{0.000000in}{-0.048611in}}%
\pgfusepath{stroke,fill}%
}%
\begin{pgfscope}%
\pgfsys@transformshift{1.798325in}{0.515123in}%
\pgfsys@useobject{currentmarker}{}%
\end{pgfscope}%
\end{pgfscope}%
\begin{pgfscope}%
\definecolor{textcolor}{rgb}{0.000000,0.000000,0.000000}%
\pgfsetstrokecolor{textcolor}%
\pgfsetfillcolor{textcolor}%
\pgftext[x=1.798325in,y=0.417901in,,top]{\color{textcolor}\rmfamily\fontsize{10.000000}{12.000000}\selectfont \(\displaystyle 4\)}%
\end{pgfscope}%
\begin{pgfscope}%
\pgfsetbuttcap%
\pgfsetroundjoin%
\definecolor{currentfill}{rgb}{0.000000,0.000000,0.000000}%
\pgfsetfillcolor{currentfill}%
\pgfsetlinewidth{0.803000pt}%
\definecolor{currentstroke}{rgb}{0.000000,0.000000,0.000000}%
\pgfsetstrokecolor{currentstroke}%
\pgfsetdash{}{0pt}%
\pgfsys@defobject{currentmarker}{\pgfqpoint{0.000000in}{-0.048611in}}{\pgfqpoint{0.000000in}{0.000000in}}{%
\pgfpathmoveto{\pgfqpoint{0.000000in}{0.000000in}}%
\pgfpathlineto{\pgfqpoint{0.000000in}{-0.048611in}}%
\pgfusepath{stroke,fill}%
}%
\begin{pgfscope}%
\pgfsys@transformshift{2.405821in}{0.515123in}%
\pgfsys@useobject{currentmarker}{}%
\end{pgfscope}%
\end{pgfscope}%
\begin{pgfscope}%
\definecolor{textcolor}{rgb}{0.000000,0.000000,0.000000}%
\pgfsetstrokecolor{textcolor}%
\pgfsetfillcolor{textcolor}%
\pgftext[x=2.405821in,y=0.417901in,,top]{\color{textcolor}\rmfamily\fontsize{10.000000}{12.000000}\selectfont \(\displaystyle 6\)}%
\end{pgfscope}%
\begin{pgfscope}%
\pgfsetbuttcap%
\pgfsetroundjoin%
\definecolor{currentfill}{rgb}{0.000000,0.000000,0.000000}%
\pgfsetfillcolor{currentfill}%
\pgfsetlinewidth{0.803000pt}%
\definecolor{currentstroke}{rgb}{0.000000,0.000000,0.000000}%
\pgfsetstrokecolor{currentstroke}%
\pgfsetdash{}{0pt}%
\pgfsys@defobject{currentmarker}{\pgfqpoint{0.000000in}{-0.048611in}}{\pgfqpoint{0.000000in}{0.000000in}}{%
\pgfpathmoveto{\pgfqpoint{0.000000in}{0.000000in}}%
\pgfpathlineto{\pgfqpoint{0.000000in}{-0.048611in}}%
\pgfusepath{stroke,fill}%
}%
\begin{pgfscope}%
\pgfsys@transformshift{3.013317in}{0.515123in}%
\pgfsys@useobject{currentmarker}{}%
\end{pgfscope}%
\end{pgfscope}%
\begin{pgfscope}%
\definecolor{textcolor}{rgb}{0.000000,0.000000,0.000000}%
\pgfsetstrokecolor{textcolor}%
\pgfsetfillcolor{textcolor}%
\pgftext[x=3.013317in,y=0.417901in,,top]{\color{textcolor}\rmfamily\fontsize{10.000000}{12.000000}\selectfont \(\displaystyle 8\)}%
\end{pgfscope}%
\begin{pgfscope}%
\pgfsetbuttcap%
\pgfsetroundjoin%
\definecolor{currentfill}{rgb}{0.000000,0.000000,0.000000}%
\pgfsetfillcolor{currentfill}%
\pgfsetlinewidth{0.803000pt}%
\definecolor{currentstroke}{rgb}{0.000000,0.000000,0.000000}%
\pgfsetstrokecolor{currentstroke}%
\pgfsetdash{}{0pt}%
\pgfsys@defobject{currentmarker}{\pgfqpoint{0.000000in}{-0.048611in}}{\pgfqpoint{0.000000in}{0.000000in}}{%
\pgfpathmoveto{\pgfqpoint{0.000000in}{0.000000in}}%
\pgfpathlineto{\pgfqpoint{0.000000in}{-0.048611in}}%
\pgfusepath{stroke,fill}%
}%
\begin{pgfscope}%
\pgfsys@transformshift{3.620813in}{0.515123in}%
\pgfsys@useobject{currentmarker}{}%
\end{pgfscope}%
\end{pgfscope}%
\begin{pgfscope}%
\definecolor{textcolor}{rgb}{0.000000,0.000000,0.000000}%
\pgfsetstrokecolor{textcolor}%
\pgfsetfillcolor{textcolor}%
\pgftext[x=3.620813in,y=0.417901in,,top]{\color{textcolor}\rmfamily\fontsize{10.000000}{12.000000}\selectfont \(\displaystyle 10\)}%
\end{pgfscope}%
\begin{pgfscope}%
\pgfsetbuttcap%
\pgfsetroundjoin%
\definecolor{currentfill}{rgb}{0.000000,0.000000,0.000000}%
\pgfsetfillcolor{currentfill}%
\pgfsetlinewidth{0.803000pt}%
\definecolor{currentstroke}{rgb}{0.000000,0.000000,0.000000}%
\pgfsetstrokecolor{currentstroke}%
\pgfsetdash{}{0pt}%
\pgfsys@defobject{currentmarker}{\pgfqpoint{0.000000in}{-0.048611in}}{\pgfqpoint{0.000000in}{0.000000in}}{%
\pgfpathmoveto{\pgfqpoint{0.000000in}{0.000000in}}%
\pgfpathlineto{\pgfqpoint{0.000000in}{-0.048611in}}%
\pgfusepath{stroke,fill}%
}%
\begin{pgfscope}%
\pgfsys@transformshift{4.228309in}{0.515123in}%
\pgfsys@useobject{currentmarker}{}%
\end{pgfscope}%
\end{pgfscope}%
\begin{pgfscope}%
\definecolor{textcolor}{rgb}{0.000000,0.000000,0.000000}%
\pgfsetstrokecolor{textcolor}%
\pgfsetfillcolor{textcolor}%
\pgftext[x=4.228309in,y=0.417901in,,top]{\color{textcolor}\rmfamily\fontsize{10.000000}{12.000000}\selectfont \(\displaystyle 12\)}%
\end{pgfscope}%
\begin{pgfscope}%
\definecolor{textcolor}{rgb}{0.000000,0.000000,0.000000}%
\pgfsetstrokecolor{textcolor}%
\pgfsetfillcolor{textcolor}%
\pgftext[x=2.709569in,y=0.238889in,,top]{\color{textcolor}\rmfamily\fontsize{10.000000}{12.000000}\selectfont I(\(\displaystyle \mu\)A)}%
\end{pgfscope}%
\begin{pgfscope}%
\pgfsetbuttcap%
\pgfsetroundjoin%
\definecolor{currentfill}{rgb}{0.000000,0.000000,0.000000}%
\pgfsetfillcolor{currentfill}%
\pgfsetlinewidth{0.803000pt}%
\definecolor{currentstroke}{rgb}{0.000000,0.000000,0.000000}%
\pgfsetstrokecolor{currentstroke}%
\pgfsetdash{}{0pt}%
\pgfsys@defobject{currentmarker}{\pgfqpoint{-0.048611in}{0.000000in}}{\pgfqpoint{0.000000in}{0.000000in}}{%
\pgfpathmoveto{\pgfqpoint{0.000000in}{0.000000in}}%
\pgfpathlineto{\pgfqpoint{-0.048611in}{0.000000in}}%
\pgfusepath{stroke,fill}%
}%
\begin{pgfscope}%
\pgfsys@transformshift{0.772069in}{0.610648in}%
\pgfsys@useobject{currentmarker}{}%
\end{pgfscope}%
\end{pgfscope}%
\begin{pgfscope}%
\definecolor{textcolor}{rgb}{0.000000,0.000000,0.000000}%
\pgfsetstrokecolor{textcolor}%
\pgfsetfillcolor{textcolor}%
\pgftext[x=0.605402in,y=0.562422in,left,base]{\color{textcolor}\rmfamily\fontsize{10.000000}{12.000000}\selectfont \(\displaystyle 0\)}%
\end{pgfscope}%
\begin{pgfscope}%
\pgfsetbuttcap%
\pgfsetroundjoin%
\definecolor{currentfill}{rgb}{0.000000,0.000000,0.000000}%
\pgfsetfillcolor{currentfill}%
\pgfsetlinewidth{0.803000pt}%
\definecolor{currentstroke}{rgb}{0.000000,0.000000,0.000000}%
\pgfsetstrokecolor{currentstroke}%
\pgfsetdash{}{0pt}%
\pgfsys@defobject{currentmarker}{\pgfqpoint{-0.048611in}{0.000000in}}{\pgfqpoint{0.000000in}{0.000000in}}{%
\pgfpathmoveto{\pgfqpoint{0.000000in}{0.000000in}}%
\pgfpathlineto{\pgfqpoint{-0.048611in}{0.000000in}}%
\pgfusepath{stroke,fill}%
}%
\begin{pgfscope}%
\pgfsys@transformshift{0.772069in}{1.096794in}%
\pgfsys@useobject{currentmarker}{}%
\end{pgfscope}%
\end{pgfscope}%
\begin{pgfscope}%
\definecolor{textcolor}{rgb}{0.000000,0.000000,0.000000}%
\pgfsetstrokecolor{textcolor}%
\pgfsetfillcolor{textcolor}%
\pgftext[x=0.605402in,y=1.048568in,left,base]{\color{textcolor}\rmfamily\fontsize{10.000000}{12.000000}\selectfont \(\displaystyle 2\)}%
\end{pgfscope}%
\begin{pgfscope}%
\pgfsetbuttcap%
\pgfsetroundjoin%
\definecolor{currentfill}{rgb}{0.000000,0.000000,0.000000}%
\pgfsetfillcolor{currentfill}%
\pgfsetlinewidth{0.803000pt}%
\definecolor{currentstroke}{rgb}{0.000000,0.000000,0.000000}%
\pgfsetstrokecolor{currentstroke}%
\pgfsetdash{}{0pt}%
\pgfsys@defobject{currentmarker}{\pgfqpoint{-0.048611in}{0.000000in}}{\pgfqpoint{0.000000in}{0.000000in}}{%
\pgfpathmoveto{\pgfqpoint{0.000000in}{0.000000in}}%
\pgfpathlineto{\pgfqpoint{-0.048611in}{0.000000in}}%
\pgfusepath{stroke,fill}%
}%
\begin{pgfscope}%
\pgfsys@transformshift{0.772069in}{1.582939in}%
\pgfsys@useobject{currentmarker}{}%
\end{pgfscope}%
\end{pgfscope}%
\begin{pgfscope}%
\definecolor{textcolor}{rgb}{0.000000,0.000000,0.000000}%
\pgfsetstrokecolor{textcolor}%
\pgfsetfillcolor{textcolor}%
\pgftext[x=0.605402in,y=1.534714in,left,base]{\color{textcolor}\rmfamily\fontsize{10.000000}{12.000000}\selectfont \(\displaystyle 4\)}%
\end{pgfscope}%
\begin{pgfscope}%
\pgfsetbuttcap%
\pgfsetroundjoin%
\definecolor{currentfill}{rgb}{0.000000,0.000000,0.000000}%
\pgfsetfillcolor{currentfill}%
\pgfsetlinewidth{0.803000pt}%
\definecolor{currentstroke}{rgb}{0.000000,0.000000,0.000000}%
\pgfsetstrokecolor{currentstroke}%
\pgfsetdash{}{0pt}%
\pgfsys@defobject{currentmarker}{\pgfqpoint{-0.048611in}{0.000000in}}{\pgfqpoint{0.000000in}{0.000000in}}{%
\pgfpathmoveto{\pgfqpoint{0.000000in}{0.000000in}}%
\pgfpathlineto{\pgfqpoint{-0.048611in}{0.000000in}}%
\pgfusepath{stroke,fill}%
}%
\begin{pgfscope}%
\pgfsys@transformshift{0.772069in}{2.069085in}%
\pgfsys@useobject{currentmarker}{}%
\end{pgfscope}%
\end{pgfscope}%
\begin{pgfscope}%
\definecolor{textcolor}{rgb}{0.000000,0.000000,0.000000}%
\pgfsetstrokecolor{textcolor}%
\pgfsetfillcolor{textcolor}%
\pgftext[x=0.605402in,y=2.020860in,left,base]{\color{textcolor}\rmfamily\fontsize{10.000000}{12.000000}\selectfont \(\displaystyle 6\)}%
\end{pgfscope}%
\begin{pgfscope}%
\pgfsetbuttcap%
\pgfsetroundjoin%
\definecolor{currentfill}{rgb}{0.000000,0.000000,0.000000}%
\pgfsetfillcolor{currentfill}%
\pgfsetlinewidth{0.803000pt}%
\definecolor{currentstroke}{rgb}{0.000000,0.000000,0.000000}%
\pgfsetstrokecolor{currentstroke}%
\pgfsetdash{}{0pt}%
\pgfsys@defobject{currentmarker}{\pgfqpoint{-0.048611in}{0.000000in}}{\pgfqpoint{0.000000in}{0.000000in}}{%
\pgfpathmoveto{\pgfqpoint{0.000000in}{0.000000in}}%
\pgfpathlineto{\pgfqpoint{-0.048611in}{0.000000in}}%
\pgfusepath{stroke,fill}%
}%
\begin{pgfscope}%
\pgfsys@transformshift{0.772069in}{2.555231in}%
\pgfsys@useobject{currentmarker}{}%
\end{pgfscope}%
\end{pgfscope}%
\begin{pgfscope}%
\definecolor{textcolor}{rgb}{0.000000,0.000000,0.000000}%
\pgfsetstrokecolor{textcolor}%
\pgfsetfillcolor{textcolor}%
\pgftext[x=0.605402in,y=2.507006in,left,base]{\color{textcolor}\rmfamily\fontsize{10.000000}{12.000000}\selectfont \(\displaystyle 8\)}%
\end{pgfscope}%
\begin{pgfscope}%
\pgfsetbuttcap%
\pgfsetroundjoin%
\definecolor{currentfill}{rgb}{0.000000,0.000000,0.000000}%
\pgfsetfillcolor{currentfill}%
\pgfsetlinewidth{0.803000pt}%
\definecolor{currentstroke}{rgb}{0.000000,0.000000,0.000000}%
\pgfsetstrokecolor{currentstroke}%
\pgfsetdash{}{0pt}%
\pgfsys@defobject{currentmarker}{\pgfqpoint{-0.048611in}{0.000000in}}{\pgfqpoint{0.000000in}{0.000000in}}{%
\pgfpathmoveto{\pgfqpoint{0.000000in}{0.000000in}}%
\pgfpathlineto{\pgfqpoint{-0.048611in}{0.000000in}}%
\pgfusepath{stroke,fill}%
}%
\begin{pgfscope}%
\pgfsys@transformshift{0.772069in}{3.041377in}%
\pgfsys@useobject{currentmarker}{}%
\end{pgfscope}%
\end{pgfscope}%
\begin{pgfscope}%
\definecolor{textcolor}{rgb}{0.000000,0.000000,0.000000}%
\pgfsetstrokecolor{textcolor}%
\pgfsetfillcolor{textcolor}%
\pgftext[x=0.535957in,y=2.993152in,left,base]{\color{textcolor}\rmfamily\fontsize{10.000000}{12.000000}\selectfont \(\displaystyle 10\)}%
\end{pgfscope}%
\begin{pgfscope}%
\definecolor{textcolor}{rgb}{0.000000,0.000000,0.000000}%
\pgfsetstrokecolor{textcolor}%
\pgfsetfillcolor{textcolor}%
\pgftext[x=0.258179in,y=1.862623in,,bottom]{\color{textcolor}\rmfamily\fontsize{10.000000}{12.000000}\selectfont V(V)}%
\end{pgfscope}%
\begin{pgfscope}%
\pgfpathrectangle{\pgfqpoint{0.772069in}{0.515123in}}{\pgfqpoint{3.875000in}{2.695000in}}%
\pgfusepath{clip}%
\pgfsetbuttcap%
\pgfsetroundjoin%
\definecolor{currentfill}{rgb}{0.121569,0.466667,0.705882}%
\pgfsetfillcolor{currentfill}%
\pgfsetlinewidth{1.003750pt}%
\definecolor{currentstroke}{rgb}{0.121569,0.466667,0.705882}%
\pgfsetstrokecolor{currentstroke}%
\pgfsetdash{}{0pt}%
\pgfpathmoveto{\pgfqpoint{0.978205in}{0.827368in}}%
\pgfpathcurveto{\pgfqpoint{0.989255in}{0.827368in}}{\pgfqpoint{0.999854in}{0.831758in}}{\pgfqpoint{1.007668in}{0.839571in}}%
\pgfpathcurveto{\pgfqpoint{1.015481in}{0.847385in}}{\pgfqpoint{1.019872in}{0.857984in}}{\pgfqpoint{1.019872in}{0.869034in}}%
\pgfpathcurveto{\pgfqpoint{1.019872in}{0.880084in}}{\pgfqpoint{1.015481in}{0.890683in}}{\pgfqpoint{1.007668in}{0.898497in}}%
\pgfpathcurveto{\pgfqpoint{0.999854in}{0.906311in}}{\pgfqpoint{0.989255in}{0.910701in}}{\pgfqpoint{0.978205in}{0.910701in}}%
\pgfpathcurveto{\pgfqpoint{0.967155in}{0.910701in}}{\pgfqpoint{0.956556in}{0.906311in}}{\pgfqpoint{0.948742in}{0.898497in}}%
\pgfpathcurveto{\pgfqpoint{0.940929in}{0.890683in}}{\pgfqpoint{0.936538in}{0.880084in}}{\pgfqpoint{0.936538in}{0.869034in}}%
\pgfpathcurveto{\pgfqpoint{0.936538in}{0.857984in}}{\pgfqpoint{0.940929in}{0.847385in}}{\pgfqpoint{0.948742in}{0.839571in}}%
\pgfpathcurveto{\pgfqpoint{0.956556in}{0.831758in}}{\pgfqpoint{0.967155in}{0.827368in}}{\pgfqpoint{0.978205in}{0.827368in}}%
\pgfpathclose%
\pgfusepath{stroke,fill}%
\end{pgfscope}%
\begin{pgfscope}%
\pgfpathrectangle{\pgfqpoint{0.772069in}{0.515123in}}{\pgfqpoint{3.875000in}{2.695000in}}%
\pgfusepath{clip}%
\pgfsetbuttcap%
\pgfsetroundjoin%
\definecolor{currentfill}{rgb}{0.121569,0.466667,0.705882}%
\pgfsetfillcolor{currentfill}%
\pgfsetlinewidth{1.003750pt}%
\definecolor{currentstroke}{rgb}{0.121569,0.466667,0.705882}%
\pgfsetstrokecolor{currentstroke}%
\pgfsetdash{}{0pt}%
\pgfpathmoveto{\pgfqpoint{1.342703in}{1.067281in}}%
\pgfpathcurveto{\pgfqpoint{1.353753in}{1.067281in}}{\pgfqpoint{1.364352in}{1.071671in}}{\pgfqpoint{1.372165in}{1.079484in}}%
\pgfpathcurveto{\pgfqpoint{1.379979in}{1.087298in}}{\pgfqpoint{1.384369in}{1.097897in}}{\pgfqpoint{1.384369in}{1.108947in}}%
\pgfpathcurveto{\pgfqpoint{1.384369in}{1.119997in}}{\pgfqpoint{1.379979in}{1.130596in}}{\pgfqpoint{1.372165in}{1.138410in}}%
\pgfpathcurveto{\pgfqpoint{1.364352in}{1.146224in}}{\pgfqpoint{1.353753in}{1.150614in}}{\pgfqpoint{1.342703in}{1.150614in}}%
\pgfpathcurveto{\pgfqpoint{1.331653in}{1.150614in}}{\pgfqpoint{1.321053in}{1.146224in}}{\pgfqpoint{1.313240in}{1.138410in}}%
\pgfpathcurveto{\pgfqpoint{1.305426in}{1.130596in}}{\pgfqpoint{1.301036in}{1.119997in}}{\pgfqpoint{1.301036in}{1.108947in}}%
\pgfpathcurveto{\pgfqpoint{1.301036in}{1.097897in}}{\pgfqpoint{1.305426in}{1.087298in}}{\pgfqpoint{1.313240in}{1.079484in}}%
\pgfpathcurveto{\pgfqpoint{1.321053in}{1.071671in}}{\pgfqpoint{1.331653in}{1.067281in}}{\pgfqpoint{1.342703in}{1.067281in}}%
\pgfpathclose%
\pgfusepath{stroke,fill}%
\end{pgfscope}%
\begin{pgfscope}%
\pgfpathrectangle{\pgfqpoint{0.772069in}{0.515123in}}{\pgfqpoint{3.875000in}{2.695000in}}%
\pgfusepath{clip}%
\pgfsetbuttcap%
\pgfsetroundjoin%
\definecolor{currentfill}{rgb}{0.121569,0.466667,0.705882}%
\pgfsetfillcolor{currentfill}%
\pgfsetlinewidth{1.003750pt}%
\definecolor{currentstroke}{rgb}{0.121569,0.466667,0.705882}%
\pgfsetstrokecolor{currentstroke}%
\pgfsetdash{}{0pt}%
\pgfpathmoveto{\pgfqpoint{1.737575in}{1.307923in}}%
\pgfpathcurveto{\pgfqpoint{1.748625in}{1.307923in}}{\pgfqpoint{1.759224in}{1.312313in}}{\pgfqpoint{1.767038in}{1.320127in}}%
\pgfpathcurveto{\pgfqpoint{1.774851in}{1.327940in}}{\pgfqpoint{1.779242in}{1.338539in}}{\pgfqpoint{1.779242in}{1.349589in}}%
\pgfpathcurveto{\pgfqpoint{1.779242in}{1.360640in}}{\pgfqpoint{1.774851in}{1.371239in}}{\pgfqpoint{1.767038in}{1.379052in}}%
\pgfpathcurveto{\pgfqpoint{1.759224in}{1.386866in}}{\pgfqpoint{1.748625in}{1.391256in}}{\pgfqpoint{1.737575in}{1.391256in}}%
\pgfpathcurveto{\pgfqpoint{1.726525in}{1.391256in}}{\pgfqpoint{1.715926in}{1.386866in}}{\pgfqpoint{1.708112in}{1.379052in}}%
\pgfpathcurveto{\pgfqpoint{1.700299in}{1.371239in}}{\pgfqpoint{1.695908in}{1.360640in}}{\pgfqpoint{1.695908in}{1.349589in}}%
\pgfpathcurveto{\pgfqpoint{1.695908in}{1.338539in}}{\pgfqpoint{1.700299in}{1.327940in}}{\pgfqpoint{1.708112in}{1.320127in}}%
\pgfpathcurveto{\pgfqpoint{1.715926in}{1.312313in}}{\pgfqpoint{1.726525in}{1.307923in}}{\pgfqpoint{1.737575in}{1.307923in}}%
\pgfpathclose%
\pgfusepath{stroke,fill}%
\end{pgfscope}%
\begin{pgfscope}%
\pgfpathrectangle{\pgfqpoint{0.772069in}{0.515123in}}{\pgfqpoint{3.875000in}{2.695000in}}%
\pgfusepath{clip}%
\pgfsetbuttcap%
\pgfsetroundjoin%
\definecolor{currentfill}{rgb}{0.121569,0.466667,0.705882}%
\pgfsetfillcolor{currentfill}%
\pgfsetlinewidth{1.003750pt}%
\definecolor{currentstroke}{rgb}{0.121569,0.466667,0.705882}%
\pgfsetstrokecolor{currentstroke}%
\pgfsetdash{}{0pt}%
\pgfpathmoveto{\pgfqpoint{2.132447in}{1.555857in}}%
\pgfpathcurveto{\pgfqpoint{2.143498in}{1.555857in}}{\pgfqpoint{2.154097in}{1.560247in}}{\pgfqpoint{2.161910in}{1.568061in}}%
\pgfpathcurveto{\pgfqpoint{2.169724in}{1.575875in}}{\pgfqpoint{2.174114in}{1.586474in}}{\pgfqpoint{2.174114in}{1.597524in}}%
\pgfpathcurveto{\pgfqpoint{2.174114in}{1.608574in}}{\pgfqpoint{2.169724in}{1.619173in}}{\pgfqpoint{2.161910in}{1.626987in}}%
\pgfpathcurveto{\pgfqpoint{2.154097in}{1.634800in}}{\pgfqpoint{2.143498in}{1.639190in}}{\pgfqpoint{2.132447in}{1.639190in}}%
\pgfpathcurveto{\pgfqpoint{2.121397in}{1.639190in}}{\pgfqpoint{2.110798in}{1.634800in}}{\pgfqpoint{2.102985in}{1.626987in}}%
\pgfpathcurveto{\pgfqpoint{2.095171in}{1.619173in}}{\pgfqpoint{2.090781in}{1.608574in}}{\pgfqpoint{2.090781in}{1.597524in}}%
\pgfpathcurveto{\pgfqpoint{2.090781in}{1.586474in}}{\pgfqpoint{2.095171in}{1.575875in}}{\pgfqpoint{2.102985in}{1.568061in}}%
\pgfpathcurveto{\pgfqpoint{2.110798in}{1.560247in}}{\pgfqpoint{2.121397in}{1.555857in}}{\pgfqpoint{2.132447in}{1.555857in}}%
\pgfpathclose%
\pgfusepath{stroke,fill}%
\end{pgfscope}%
\begin{pgfscope}%
\pgfpathrectangle{\pgfqpoint{0.772069in}{0.515123in}}{\pgfqpoint{3.875000in}{2.695000in}}%
\pgfusepath{clip}%
\pgfsetbuttcap%
\pgfsetroundjoin%
\definecolor{currentfill}{rgb}{0.121569,0.466667,0.705882}%
\pgfsetfillcolor{currentfill}%
\pgfsetlinewidth{1.003750pt}%
\definecolor{currentstroke}{rgb}{0.121569,0.466667,0.705882}%
\pgfsetstrokecolor{currentstroke}%
\pgfsetdash{}{0pt}%
\pgfpathmoveto{\pgfqpoint{2.496945in}{1.794069in}}%
\pgfpathcurveto{\pgfqpoint{2.507995in}{1.794069in}}{\pgfqpoint{2.518594in}{1.798459in}}{\pgfqpoint{2.526408in}{1.806272in}}%
\pgfpathcurveto{\pgfqpoint{2.534221in}{1.814086in}}{\pgfqpoint{2.538612in}{1.824685in}}{\pgfqpoint{2.538612in}{1.835735in}}%
\pgfpathcurveto{\pgfqpoint{2.538612in}{1.846785in}}{\pgfqpoint{2.534221in}{1.857384in}}{\pgfqpoint{2.526408in}{1.865198in}}%
\pgfpathcurveto{\pgfqpoint{2.518594in}{1.873012in}}{\pgfqpoint{2.507995in}{1.877402in}}{\pgfqpoint{2.496945in}{1.877402in}}%
\pgfpathcurveto{\pgfqpoint{2.485895in}{1.877402in}}{\pgfqpoint{2.475296in}{1.873012in}}{\pgfqpoint{2.467482in}{1.865198in}}%
\pgfpathcurveto{\pgfqpoint{2.459669in}{1.857384in}}{\pgfqpoint{2.455278in}{1.846785in}}{\pgfqpoint{2.455278in}{1.835735in}}%
\pgfpathcurveto{\pgfqpoint{2.455278in}{1.824685in}}{\pgfqpoint{2.459669in}{1.814086in}}{\pgfqpoint{2.467482in}{1.806272in}}%
\pgfpathcurveto{\pgfqpoint{2.475296in}{1.798459in}}{\pgfqpoint{2.485895in}{1.794069in}}{\pgfqpoint{2.496945in}{1.794069in}}%
\pgfpathclose%
\pgfusepath{stroke,fill}%
\end{pgfscope}%
\begin{pgfscope}%
\pgfpathrectangle{\pgfqpoint{0.772069in}{0.515123in}}{\pgfqpoint{3.875000in}{2.695000in}}%
\pgfusepath{clip}%
\pgfsetbuttcap%
\pgfsetroundjoin%
\definecolor{currentfill}{rgb}{0.121569,0.466667,0.705882}%
\pgfsetfillcolor{currentfill}%
\pgfsetlinewidth{1.003750pt}%
\definecolor{currentstroke}{rgb}{0.121569,0.466667,0.705882}%
\pgfsetstrokecolor{currentstroke}%
\pgfsetdash{}{0pt}%
\pgfpathmoveto{\pgfqpoint{2.922192in}{2.056587in}}%
\pgfpathcurveto{\pgfqpoint{2.933242in}{2.056587in}}{\pgfqpoint{2.943841in}{2.060978in}}{\pgfqpoint{2.951655in}{2.068791in}}%
\pgfpathcurveto{\pgfqpoint{2.959469in}{2.076605in}}{\pgfqpoint{2.963859in}{2.087204in}}{\pgfqpoint{2.963859in}{2.098254in}}%
\pgfpathcurveto{\pgfqpoint{2.963859in}{2.109304in}}{\pgfqpoint{2.959469in}{2.119903in}}{\pgfqpoint{2.951655in}{2.127717in}}%
\pgfpathcurveto{\pgfqpoint{2.943841in}{2.135530in}}{\pgfqpoint{2.933242in}{2.139921in}}{\pgfqpoint{2.922192in}{2.139921in}}%
\pgfpathcurveto{\pgfqpoint{2.911142in}{2.139921in}}{\pgfqpoint{2.900543in}{2.135530in}}{\pgfqpoint{2.892730in}{2.127717in}}%
\pgfpathcurveto{\pgfqpoint{2.884916in}{2.119903in}}{\pgfqpoint{2.880526in}{2.109304in}}{\pgfqpoint{2.880526in}{2.098254in}}%
\pgfpathcurveto{\pgfqpoint{2.880526in}{2.087204in}}{\pgfqpoint{2.884916in}{2.076605in}}{\pgfqpoint{2.892730in}{2.068791in}}%
\pgfpathcurveto{\pgfqpoint{2.900543in}{2.060978in}}{\pgfqpoint{2.911142in}{2.056587in}}{\pgfqpoint{2.922192in}{2.056587in}}%
\pgfpathclose%
\pgfusepath{stroke,fill}%
\end{pgfscope}%
\begin{pgfscope}%
\pgfpathrectangle{\pgfqpoint{0.772069in}{0.515123in}}{\pgfqpoint{3.875000in}{2.695000in}}%
\pgfusepath{clip}%
\pgfsetbuttcap%
\pgfsetroundjoin%
\definecolor{currentfill}{rgb}{0.121569,0.466667,0.705882}%
\pgfsetfillcolor{currentfill}%
\pgfsetlinewidth{1.003750pt}%
\definecolor{currentstroke}{rgb}{0.121569,0.466667,0.705882}%
\pgfsetstrokecolor{currentstroke}%
\pgfsetdash{}{0pt}%
\pgfpathmoveto{\pgfqpoint{3.286690in}{2.289937in}}%
\pgfpathcurveto{\pgfqpoint{3.297740in}{2.289937in}}{\pgfqpoint{3.308339in}{2.294328in}}{\pgfqpoint{3.316153in}{2.302141in}}%
\pgfpathcurveto{\pgfqpoint{3.323966in}{2.309955in}}{\pgfqpoint{3.328357in}{2.320554in}}{\pgfqpoint{3.328357in}{2.331604in}}%
\pgfpathcurveto{\pgfqpoint{3.328357in}{2.342654in}}{\pgfqpoint{3.323966in}{2.353253in}}{\pgfqpoint{3.316153in}{2.361067in}}%
\pgfpathcurveto{\pgfqpoint{3.308339in}{2.368880in}}{\pgfqpoint{3.297740in}{2.373271in}}{\pgfqpoint{3.286690in}{2.373271in}}%
\pgfpathcurveto{\pgfqpoint{3.275640in}{2.373271in}}{\pgfqpoint{3.265041in}{2.368880in}}{\pgfqpoint{3.257227in}{2.361067in}}%
\pgfpathcurveto{\pgfqpoint{3.249413in}{2.353253in}}{\pgfqpoint{3.245023in}{2.342654in}}{\pgfqpoint{3.245023in}{2.331604in}}%
\pgfpathcurveto{\pgfqpoint{3.245023in}{2.320554in}}{\pgfqpoint{3.249413in}{2.309955in}}{\pgfqpoint{3.257227in}{2.302141in}}%
\pgfpathcurveto{\pgfqpoint{3.265041in}{2.294328in}}{\pgfqpoint{3.275640in}{2.289937in}}{\pgfqpoint{3.286690in}{2.289937in}}%
\pgfpathclose%
\pgfusepath{stroke,fill}%
\end{pgfscope}%
\begin{pgfscope}%
\pgfpathrectangle{\pgfqpoint{0.772069in}{0.515123in}}{\pgfqpoint{3.875000in}{2.695000in}}%
\pgfusepath{clip}%
\pgfsetbuttcap%
\pgfsetroundjoin%
\definecolor{currentfill}{rgb}{0.121569,0.466667,0.705882}%
\pgfsetfillcolor{currentfill}%
\pgfsetlinewidth{1.003750pt}%
\definecolor{currentstroke}{rgb}{0.121569,0.466667,0.705882}%
\pgfsetstrokecolor{currentstroke}%
\pgfsetdash{}{0pt}%
\pgfpathmoveto{\pgfqpoint{3.681562in}{2.530579in}}%
\pgfpathcurveto{\pgfqpoint{3.692612in}{2.530579in}}{\pgfqpoint{3.703211in}{2.534970in}}{\pgfqpoint{3.711025in}{2.542783in}}%
\pgfpathcurveto{\pgfqpoint{3.718839in}{2.550597in}}{\pgfqpoint{3.723229in}{2.561196in}}{\pgfqpoint{3.723229in}{2.572246in}}%
\pgfpathcurveto{\pgfqpoint{3.723229in}{2.583296in}}{\pgfqpoint{3.718839in}{2.593895in}}{\pgfqpoint{3.711025in}{2.601709in}}%
\pgfpathcurveto{\pgfqpoint{3.703211in}{2.609523in}}{\pgfqpoint{3.692612in}{2.613913in}}{\pgfqpoint{3.681562in}{2.613913in}}%
\pgfpathcurveto{\pgfqpoint{3.670512in}{2.613913in}}{\pgfqpoint{3.659913in}{2.609523in}}{\pgfqpoint{3.652100in}{2.601709in}}%
\pgfpathcurveto{\pgfqpoint{3.644286in}{2.593895in}}{\pgfqpoint{3.639896in}{2.583296in}}{\pgfqpoint{3.639896in}{2.572246in}}%
\pgfpathcurveto{\pgfqpoint{3.639896in}{2.561196in}}{\pgfqpoint{3.644286in}{2.550597in}}{\pgfqpoint{3.652100in}{2.542783in}}%
\pgfpathcurveto{\pgfqpoint{3.659913in}{2.534970in}}{\pgfqpoint{3.670512in}{2.530579in}}{\pgfqpoint{3.681562in}{2.530579in}}%
\pgfpathclose%
\pgfusepath{stroke,fill}%
\end{pgfscope}%
\begin{pgfscope}%
\pgfpathrectangle{\pgfqpoint{0.772069in}{0.515123in}}{\pgfqpoint{3.875000in}{2.695000in}}%
\pgfusepath{clip}%
\pgfsetbuttcap%
\pgfsetroundjoin%
\definecolor{currentfill}{rgb}{0.121569,0.466667,0.705882}%
\pgfsetfillcolor{currentfill}%
\pgfsetlinewidth{1.003750pt}%
\definecolor{currentstroke}{rgb}{0.121569,0.466667,0.705882}%
\pgfsetstrokecolor{currentstroke}%
\pgfsetdash{}{0pt}%
\pgfpathmoveto{\pgfqpoint{4.015685in}{2.756637in}}%
\pgfpathcurveto{\pgfqpoint{4.026735in}{2.756637in}}{\pgfqpoint{4.037334in}{2.761028in}}{\pgfqpoint{4.045148in}{2.768841in}}%
\pgfpathcurveto{\pgfqpoint{4.052962in}{2.776655in}}{\pgfqpoint{4.057352in}{2.787254in}}{\pgfqpoint{4.057352in}{2.798304in}}%
\pgfpathcurveto{\pgfqpoint{4.057352in}{2.809354in}}{\pgfqpoint{4.052962in}{2.819953in}}{\pgfqpoint{4.045148in}{2.827767in}}%
\pgfpathcurveto{\pgfqpoint{4.037334in}{2.835580in}}{\pgfqpoint{4.026735in}{2.839971in}}{\pgfqpoint{4.015685in}{2.839971in}}%
\pgfpathcurveto{\pgfqpoint{4.004635in}{2.839971in}}{\pgfqpoint{3.994036in}{2.835580in}}{\pgfqpoint{3.986222in}{2.827767in}}%
\pgfpathcurveto{\pgfqpoint{3.978409in}{2.819953in}}{\pgfqpoint{3.974018in}{2.809354in}}{\pgfqpoint{3.974018in}{2.798304in}}%
\pgfpathcurveto{\pgfqpoint{3.974018in}{2.787254in}}{\pgfqpoint{3.978409in}{2.776655in}}{\pgfqpoint{3.986222in}{2.768841in}}%
\pgfpathcurveto{\pgfqpoint{3.994036in}{2.761028in}}{\pgfqpoint{4.004635in}{2.756637in}}{\pgfqpoint{4.015685in}{2.756637in}}%
\pgfpathclose%
\pgfusepath{stroke,fill}%
\end{pgfscope}%
\begin{pgfscope}%
\pgfpathrectangle{\pgfqpoint{0.772069in}{0.515123in}}{\pgfqpoint{3.875000in}{2.695000in}}%
\pgfusepath{clip}%
\pgfsetbuttcap%
\pgfsetroundjoin%
\definecolor{currentfill}{rgb}{0.121569,0.466667,0.705882}%
\pgfsetfillcolor{currentfill}%
\pgfsetlinewidth{1.003750pt}%
\definecolor{currentstroke}{rgb}{0.121569,0.466667,0.705882}%
\pgfsetstrokecolor{currentstroke}%
\pgfsetdash{}{0pt}%
\pgfpathmoveto{\pgfqpoint{4.440932in}{3.019156in}}%
\pgfpathcurveto{\pgfqpoint{4.451982in}{3.019156in}}{\pgfqpoint{4.462581in}{3.023546in}}{\pgfqpoint{4.470395in}{3.031360in}}%
\pgfpathcurveto{\pgfqpoint{4.478209in}{3.039174in}}{\pgfqpoint{4.482599in}{3.049773in}}{\pgfqpoint{4.482599in}{3.060823in}}%
\pgfpathcurveto{\pgfqpoint{4.482599in}{3.071873in}}{\pgfqpoint{4.478209in}{3.082472in}}{\pgfqpoint{4.470395in}{3.090285in}}%
\pgfpathcurveto{\pgfqpoint{4.462581in}{3.098099in}}{\pgfqpoint{4.451982in}{3.102489in}}{\pgfqpoint{4.440932in}{3.102489in}}%
\pgfpathcurveto{\pgfqpoint{4.429882in}{3.102489in}}{\pgfqpoint{4.419283in}{3.098099in}}{\pgfqpoint{4.411470in}{3.090285in}}%
\pgfpathcurveto{\pgfqpoint{4.403656in}{3.082472in}}{\pgfqpoint{4.399266in}{3.071873in}}{\pgfqpoint{4.399266in}{3.060823in}}%
\pgfpathcurveto{\pgfqpoint{4.399266in}{3.049773in}}{\pgfqpoint{4.403656in}{3.039174in}}{\pgfqpoint{4.411470in}{3.031360in}}%
\pgfpathcurveto{\pgfqpoint{4.419283in}{3.023546in}}{\pgfqpoint{4.429882in}{3.019156in}}{\pgfqpoint{4.440932in}{3.019156in}}%
\pgfpathclose%
\pgfusepath{stroke,fill}%
\end{pgfscope}%
\begin{pgfscope}%
\pgfpathrectangle{\pgfqpoint{0.772069in}{0.515123in}}{\pgfqpoint{3.875000in}{2.695000in}}%
\pgfusepath{clip}%
\pgfsetbuttcap%
\pgfsetroundjoin%
\definecolor{currentfill}{rgb}{1.000000,0.388235,0.278431}%
\pgfsetfillcolor{currentfill}%
\pgfsetlinewidth{1.003750pt}%
\definecolor{currentstroke}{rgb}{1.000000,0.388235,0.278431}%
\pgfsetstrokecolor{currentstroke}%
\pgfsetdash{}{0pt}%
\pgfpathmoveto{\pgfqpoint{0.978205in}{0.625617in}}%
\pgfpathcurveto{\pgfqpoint{0.989255in}{0.625617in}}{\pgfqpoint{0.999854in}{0.630007in}}{\pgfqpoint{1.007668in}{0.637821in}}%
\pgfpathcurveto{\pgfqpoint{1.015481in}{0.645635in}}{\pgfqpoint{1.019872in}{0.656234in}}{\pgfqpoint{1.019872in}{0.667284in}}%
\pgfpathcurveto{\pgfqpoint{1.019872in}{0.678334in}}{\pgfqpoint{1.015481in}{0.688933in}}{\pgfqpoint{1.007668in}{0.696747in}}%
\pgfpathcurveto{\pgfqpoint{0.999854in}{0.704560in}}{\pgfqpoint{0.989255in}{0.708950in}}{\pgfqpoint{0.978205in}{0.708950in}}%
\pgfpathcurveto{\pgfqpoint{0.967155in}{0.708950in}}{\pgfqpoint{0.956556in}{0.704560in}}{\pgfqpoint{0.948742in}{0.696747in}}%
\pgfpathcurveto{\pgfqpoint{0.940929in}{0.688933in}}{\pgfqpoint{0.936538in}{0.678334in}}{\pgfqpoint{0.936538in}{0.667284in}}%
\pgfpathcurveto{\pgfqpoint{0.936538in}{0.656234in}}{\pgfqpoint{0.940929in}{0.645635in}}{\pgfqpoint{0.948742in}{0.637821in}}%
\pgfpathcurveto{\pgfqpoint{0.956556in}{0.630007in}}{\pgfqpoint{0.967155in}{0.625617in}}{\pgfqpoint{0.978205in}{0.625617in}}%
\pgfpathclose%
\pgfusepath{stroke,fill}%
\end{pgfscope}%
\begin{pgfscope}%
\pgfpathrectangle{\pgfqpoint{0.772069in}{0.515123in}}{\pgfqpoint{3.875000in}{2.695000in}}%
\pgfusepath{clip}%
\pgfsetbuttcap%
\pgfsetroundjoin%
\definecolor{currentfill}{rgb}{1.000000,0.388235,0.278431}%
\pgfsetfillcolor{currentfill}%
\pgfsetlinewidth{1.003750pt}%
\definecolor{currentstroke}{rgb}{1.000000,0.388235,0.278431}%
\pgfsetstrokecolor{currentstroke}%
\pgfsetdash{}{0pt}%
\pgfpathmoveto{\pgfqpoint{1.342703in}{0.678607in}}%
\pgfpathcurveto{\pgfqpoint{1.353753in}{0.678607in}}{\pgfqpoint{1.364352in}{0.682997in}}{\pgfqpoint{1.372165in}{0.690811in}}%
\pgfpathcurveto{\pgfqpoint{1.379979in}{0.698624in}}{\pgfqpoint{1.384369in}{0.709224in}}{\pgfqpoint{1.384369in}{0.720274in}}%
\pgfpathcurveto{\pgfqpoint{1.384369in}{0.731324in}}{\pgfqpoint{1.379979in}{0.741923in}}{\pgfqpoint{1.372165in}{0.749736in}}%
\pgfpathcurveto{\pgfqpoint{1.364352in}{0.757550in}}{\pgfqpoint{1.353753in}{0.761940in}}{\pgfqpoint{1.342703in}{0.761940in}}%
\pgfpathcurveto{\pgfqpoint{1.331653in}{0.761940in}}{\pgfqpoint{1.321053in}{0.757550in}}{\pgfqpoint{1.313240in}{0.749736in}}%
\pgfpathcurveto{\pgfqpoint{1.305426in}{0.741923in}}{\pgfqpoint{1.301036in}{0.731324in}}{\pgfqpoint{1.301036in}{0.720274in}}%
\pgfpathcurveto{\pgfqpoint{1.301036in}{0.709224in}}{\pgfqpoint{1.305426in}{0.698624in}}{\pgfqpoint{1.313240in}{0.690811in}}%
\pgfpathcurveto{\pgfqpoint{1.321053in}{0.682997in}}{\pgfqpoint{1.331653in}{0.678607in}}{\pgfqpoint{1.342703in}{0.678607in}}%
\pgfpathclose%
\pgfusepath{stroke,fill}%
\end{pgfscope}%
\begin{pgfscope}%
\pgfpathrectangle{\pgfqpoint{0.772069in}{0.515123in}}{\pgfqpoint{3.875000in}{2.695000in}}%
\pgfusepath{clip}%
\pgfsetbuttcap%
\pgfsetroundjoin%
\definecolor{currentfill}{rgb}{1.000000,0.388235,0.278431}%
\pgfsetfillcolor{currentfill}%
\pgfsetlinewidth{1.003750pt}%
\definecolor{currentstroke}{rgb}{1.000000,0.388235,0.278431}%
\pgfsetstrokecolor{currentstroke}%
\pgfsetdash{}{0pt}%
\pgfpathmoveto{\pgfqpoint{1.737575in}{0.732083in}}%
\pgfpathcurveto{\pgfqpoint{1.748625in}{0.732083in}}{\pgfqpoint{1.759224in}{0.736473in}}{\pgfqpoint{1.767038in}{0.744287in}}%
\pgfpathcurveto{\pgfqpoint{1.774851in}{0.752101in}}{\pgfqpoint{1.779242in}{0.762700in}}{\pgfqpoint{1.779242in}{0.773750in}}%
\pgfpathcurveto{\pgfqpoint{1.779242in}{0.784800in}}{\pgfqpoint{1.774851in}{0.795399in}}{\pgfqpoint{1.767038in}{0.803212in}}%
\pgfpathcurveto{\pgfqpoint{1.759224in}{0.811026in}}{\pgfqpoint{1.748625in}{0.815416in}}{\pgfqpoint{1.737575in}{0.815416in}}%
\pgfpathcurveto{\pgfqpoint{1.726525in}{0.815416in}}{\pgfqpoint{1.715926in}{0.811026in}}{\pgfqpoint{1.708112in}{0.803212in}}%
\pgfpathcurveto{\pgfqpoint{1.700299in}{0.795399in}}{\pgfqpoint{1.695908in}{0.784800in}}{\pgfqpoint{1.695908in}{0.773750in}}%
\pgfpathcurveto{\pgfqpoint{1.695908in}{0.762700in}}{\pgfqpoint{1.700299in}{0.752101in}}{\pgfqpoint{1.708112in}{0.744287in}}%
\pgfpathcurveto{\pgfqpoint{1.715926in}{0.736473in}}{\pgfqpoint{1.726525in}{0.732083in}}{\pgfqpoint{1.737575in}{0.732083in}}%
\pgfpathclose%
\pgfusepath{stroke,fill}%
\end{pgfscope}%
\begin{pgfscope}%
\pgfpathrectangle{\pgfqpoint{0.772069in}{0.515123in}}{\pgfqpoint{3.875000in}{2.695000in}}%
\pgfusepath{clip}%
\pgfsetbuttcap%
\pgfsetroundjoin%
\definecolor{currentfill}{rgb}{1.000000,0.388235,0.278431}%
\pgfsetfillcolor{currentfill}%
\pgfsetlinewidth{1.003750pt}%
\definecolor{currentstroke}{rgb}{1.000000,0.388235,0.278431}%
\pgfsetstrokecolor{currentstroke}%
\pgfsetdash{}{0pt}%
\pgfpathmoveto{\pgfqpoint{2.132447in}{0.786288in}}%
\pgfpathcurveto{\pgfqpoint{2.143498in}{0.786288in}}{\pgfqpoint{2.154097in}{0.790679in}}{\pgfqpoint{2.161910in}{0.798492in}}%
\pgfpathcurveto{\pgfqpoint{2.169724in}{0.806306in}}{\pgfqpoint{2.174114in}{0.816905in}}{\pgfqpoint{2.174114in}{0.827955in}}%
\pgfpathcurveto{\pgfqpoint{2.174114in}{0.839005in}}{\pgfqpoint{2.169724in}{0.849604in}}{\pgfqpoint{2.161910in}{0.857418in}}%
\pgfpathcurveto{\pgfqpoint{2.154097in}{0.865231in}}{\pgfqpoint{2.143498in}{0.869622in}}{\pgfqpoint{2.132447in}{0.869622in}}%
\pgfpathcurveto{\pgfqpoint{2.121397in}{0.869622in}}{\pgfqpoint{2.110798in}{0.865231in}}{\pgfqpoint{2.102985in}{0.857418in}}%
\pgfpathcurveto{\pgfqpoint{2.095171in}{0.849604in}}{\pgfqpoint{2.090781in}{0.839005in}}{\pgfqpoint{2.090781in}{0.827955in}}%
\pgfpathcurveto{\pgfqpoint{2.090781in}{0.816905in}}{\pgfqpoint{2.095171in}{0.806306in}}{\pgfqpoint{2.102985in}{0.798492in}}%
\pgfpathcurveto{\pgfqpoint{2.110798in}{0.790679in}}{\pgfqpoint{2.121397in}{0.786288in}}{\pgfqpoint{2.132447in}{0.786288in}}%
\pgfpathclose%
\pgfusepath{stroke,fill}%
\end{pgfscope}%
\begin{pgfscope}%
\pgfpathrectangle{\pgfqpoint{0.772069in}{0.515123in}}{\pgfqpoint{3.875000in}{2.695000in}}%
\pgfusepath{clip}%
\pgfsetbuttcap%
\pgfsetroundjoin%
\definecolor{currentfill}{rgb}{1.000000,0.388235,0.278431}%
\pgfsetfillcolor{currentfill}%
\pgfsetlinewidth{1.003750pt}%
\definecolor{currentstroke}{rgb}{1.000000,0.388235,0.278431}%
\pgfsetstrokecolor{currentstroke}%
\pgfsetdash{}{0pt}%
\pgfpathmoveto{\pgfqpoint{2.496945in}{0.838792in}}%
\pgfpathcurveto{\pgfqpoint{2.507995in}{0.838792in}}{\pgfqpoint{2.518594in}{0.843182in}}{\pgfqpoint{2.526408in}{0.850996in}}%
\pgfpathcurveto{\pgfqpoint{2.534221in}{0.858810in}}{\pgfqpoint{2.538612in}{0.869409in}}{\pgfqpoint{2.538612in}{0.880459in}}%
\pgfpathcurveto{\pgfqpoint{2.538612in}{0.891509in}}{\pgfqpoint{2.534221in}{0.902108in}}{\pgfqpoint{2.526408in}{0.909921in}}%
\pgfpathcurveto{\pgfqpoint{2.518594in}{0.917735in}}{\pgfqpoint{2.507995in}{0.922125in}}{\pgfqpoint{2.496945in}{0.922125in}}%
\pgfpathcurveto{\pgfqpoint{2.485895in}{0.922125in}}{\pgfqpoint{2.475296in}{0.917735in}}{\pgfqpoint{2.467482in}{0.909921in}}%
\pgfpathcurveto{\pgfqpoint{2.459669in}{0.902108in}}{\pgfqpoint{2.455278in}{0.891509in}}{\pgfqpoint{2.455278in}{0.880459in}}%
\pgfpathcurveto{\pgfqpoint{2.455278in}{0.869409in}}{\pgfqpoint{2.459669in}{0.858810in}}{\pgfqpoint{2.467482in}{0.850996in}}%
\pgfpathcurveto{\pgfqpoint{2.475296in}{0.843182in}}{\pgfqpoint{2.485895in}{0.838792in}}{\pgfqpoint{2.496945in}{0.838792in}}%
\pgfpathclose%
\pgfusepath{stroke,fill}%
\end{pgfscope}%
\begin{pgfscope}%
\pgfpathrectangle{\pgfqpoint{0.772069in}{0.515123in}}{\pgfqpoint{3.875000in}{2.695000in}}%
\pgfusepath{clip}%
\pgfsetbuttcap%
\pgfsetroundjoin%
\definecolor{currentfill}{rgb}{1.000000,0.388235,0.278431}%
\pgfsetfillcolor{currentfill}%
\pgfsetlinewidth{1.003750pt}%
\definecolor{currentstroke}{rgb}{1.000000,0.388235,0.278431}%
\pgfsetstrokecolor{currentstroke}%
\pgfsetdash{}{0pt}%
\pgfpathmoveto{\pgfqpoint{2.922192in}{0.896400in}}%
\pgfpathcurveto{\pgfqpoint{2.933242in}{0.896400in}}{\pgfqpoint{2.943841in}{0.900791in}}{\pgfqpoint{2.951655in}{0.908604in}}%
\pgfpathcurveto{\pgfqpoint{2.959469in}{0.916418in}}{\pgfqpoint{2.963859in}{0.927017in}}{\pgfqpoint{2.963859in}{0.938067in}}%
\pgfpathcurveto{\pgfqpoint{2.963859in}{0.949117in}}{\pgfqpoint{2.959469in}{0.959716in}}{\pgfqpoint{2.951655in}{0.967530in}}%
\pgfpathcurveto{\pgfqpoint{2.943841in}{0.975343in}}{\pgfqpoint{2.933242in}{0.979734in}}{\pgfqpoint{2.922192in}{0.979734in}}%
\pgfpathcurveto{\pgfqpoint{2.911142in}{0.979734in}}{\pgfqpoint{2.900543in}{0.975343in}}{\pgfqpoint{2.892730in}{0.967530in}}%
\pgfpathcurveto{\pgfqpoint{2.884916in}{0.959716in}}{\pgfqpoint{2.880526in}{0.949117in}}{\pgfqpoint{2.880526in}{0.938067in}}%
\pgfpathcurveto{\pgfqpoint{2.880526in}{0.927017in}}{\pgfqpoint{2.884916in}{0.916418in}}{\pgfqpoint{2.892730in}{0.908604in}}%
\pgfpathcurveto{\pgfqpoint{2.900543in}{0.900791in}}{\pgfqpoint{2.911142in}{0.896400in}}{\pgfqpoint{2.922192in}{0.896400in}}%
\pgfpathclose%
\pgfusepath{stroke,fill}%
\end{pgfscope}%
\begin{pgfscope}%
\pgfpathrectangle{\pgfqpoint{0.772069in}{0.515123in}}{\pgfqpoint{3.875000in}{2.695000in}}%
\pgfusepath{clip}%
\pgfsetbuttcap%
\pgfsetroundjoin%
\definecolor{currentfill}{rgb}{1.000000,0.388235,0.278431}%
\pgfsetfillcolor{currentfill}%
\pgfsetlinewidth{1.003750pt}%
\definecolor{currentstroke}{rgb}{1.000000,0.388235,0.278431}%
\pgfsetstrokecolor{currentstroke}%
\pgfsetdash{}{0pt}%
\pgfpathmoveto{\pgfqpoint{3.286690in}{0.947689in}}%
\pgfpathcurveto{\pgfqpoint{3.297740in}{0.947689in}}{\pgfqpoint{3.308339in}{0.952079in}}{\pgfqpoint{3.316153in}{0.959893in}}%
\pgfpathcurveto{\pgfqpoint{3.323966in}{0.967706in}}{\pgfqpoint{3.328357in}{0.978305in}}{\pgfqpoint{3.328357in}{0.989355in}}%
\pgfpathcurveto{\pgfqpoint{3.328357in}{1.000405in}}{\pgfqpoint{3.323966in}{1.011005in}}{\pgfqpoint{3.316153in}{1.018818in}}%
\pgfpathcurveto{\pgfqpoint{3.308339in}{1.026632in}}{\pgfqpoint{3.297740in}{1.031022in}}{\pgfqpoint{3.286690in}{1.031022in}}%
\pgfpathcurveto{\pgfqpoint{3.275640in}{1.031022in}}{\pgfqpoint{3.265041in}{1.026632in}}{\pgfqpoint{3.257227in}{1.018818in}}%
\pgfpathcurveto{\pgfqpoint{3.249413in}{1.011005in}}{\pgfqpoint{3.245023in}{1.000405in}}{\pgfqpoint{3.245023in}{0.989355in}}%
\pgfpathcurveto{\pgfqpoint{3.245023in}{0.978305in}}{\pgfqpoint{3.249413in}{0.967706in}}{\pgfqpoint{3.257227in}{0.959893in}}%
\pgfpathcurveto{\pgfqpoint{3.265041in}{0.952079in}}{\pgfqpoint{3.275640in}{0.947689in}}{\pgfqpoint{3.286690in}{0.947689in}}%
\pgfpathclose%
\pgfusepath{stroke,fill}%
\end{pgfscope}%
\begin{pgfscope}%
\pgfpathrectangle{\pgfqpoint{0.772069in}{0.515123in}}{\pgfqpoint{3.875000in}{2.695000in}}%
\pgfusepath{clip}%
\pgfsetbuttcap%
\pgfsetroundjoin%
\definecolor{currentfill}{rgb}{1.000000,0.388235,0.278431}%
\pgfsetfillcolor{currentfill}%
\pgfsetlinewidth{1.003750pt}%
\definecolor{currentstroke}{rgb}{1.000000,0.388235,0.278431}%
\pgfsetstrokecolor{currentstroke}%
\pgfsetdash{}{0pt}%
\pgfpathmoveto{\pgfqpoint{3.681562in}{1.000922in}}%
\pgfpathcurveto{\pgfqpoint{3.692612in}{1.000922in}}{\pgfqpoint{3.703211in}{1.005312in}}{\pgfqpoint{3.711025in}{1.013126in}}%
\pgfpathcurveto{\pgfqpoint{3.718839in}{1.020939in}}{\pgfqpoint{3.723229in}{1.031538in}}{\pgfqpoint{3.723229in}{1.042588in}}%
\pgfpathcurveto{\pgfqpoint{3.723229in}{1.053638in}}{\pgfqpoint{3.718839in}{1.064237in}}{\pgfqpoint{3.711025in}{1.072051in}}%
\pgfpathcurveto{\pgfqpoint{3.703211in}{1.079865in}}{\pgfqpoint{3.692612in}{1.084255in}}{\pgfqpoint{3.681562in}{1.084255in}}%
\pgfpathcurveto{\pgfqpoint{3.670512in}{1.084255in}}{\pgfqpoint{3.659913in}{1.079865in}}{\pgfqpoint{3.652100in}{1.072051in}}%
\pgfpathcurveto{\pgfqpoint{3.644286in}{1.064237in}}{\pgfqpoint{3.639896in}{1.053638in}}{\pgfqpoint{3.639896in}{1.042588in}}%
\pgfpathcurveto{\pgfqpoint{3.639896in}{1.031538in}}{\pgfqpoint{3.644286in}{1.020939in}}{\pgfqpoint{3.652100in}{1.013126in}}%
\pgfpathcurveto{\pgfqpoint{3.659913in}{1.005312in}}{\pgfqpoint{3.670512in}{1.000922in}}{\pgfqpoint{3.681562in}{1.000922in}}%
\pgfpathclose%
\pgfusepath{stroke,fill}%
\end{pgfscope}%
\begin{pgfscope}%
\pgfpathrectangle{\pgfqpoint{0.772069in}{0.515123in}}{\pgfqpoint{3.875000in}{2.695000in}}%
\pgfusepath{clip}%
\pgfsetbuttcap%
\pgfsetroundjoin%
\definecolor{currentfill}{rgb}{1.000000,0.388235,0.278431}%
\pgfsetfillcolor{currentfill}%
\pgfsetlinewidth{1.003750pt}%
\definecolor{currentstroke}{rgb}{1.000000,0.388235,0.278431}%
\pgfsetstrokecolor{currentstroke}%
\pgfsetdash{}{0pt}%
\pgfpathmoveto{\pgfqpoint{4.015685in}{1.047835in}}%
\pgfpathcurveto{\pgfqpoint{4.026735in}{1.047835in}}{\pgfqpoint{4.037334in}{1.052225in}}{\pgfqpoint{4.045148in}{1.060039in}}%
\pgfpathcurveto{\pgfqpoint{4.052962in}{1.067852in}}{\pgfqpoint{4.057352in}{1.078451in}}{\pgfqpoint{4.057352in}{1.089501in}}%
\pgfpathcurveto{\pgfqpoint{4.057352in}{1.100552in}}{\pgfqpoint{4.052962in}{1.111151in}}{\pgfqpoint{4.045148in}{1.118964in}}%
\pgfpathcurveto{\pgfqpoint{4.037334in}{1.126778in}}{\pgfqpoint{4.026735in}{1.131168in}}{\pgfqpoint{4.015685in}{1.131168in}}%
\pgfpathcurveto{\pgfqpoint{4.004635in}{1.131168in}}{\pgfqpoint{3.994036in}{1.126778in}}{\pgfqpoint{3.986222in}{1.118964in}}%
\pgfpathcurveto{\pgfqpoint{3.978409in}{1.111151in}}{\pgfqpoint{3.974018in}{1.100552in}}{\pgfqpoint{3.974018in}{1.089501in}}%
\pgfpathcurveto{\pgfqpoint{3.974018in}{1.078451in}}{\pgfqpoint{3.978409in}{1.067852in}}{\pgfqpoint{3.986222in}{1.060039in}}%
\pgfpathcurveto{\pgfqpoint{3.994036in}{1.052225in}}{\pgfqpoint{4.004635in}{1.047835in}}{\pgfqpoint{4.015685in}{1.047835in}}%
\pgfpathclose%
\pgfusepath{stroke,fill}%
\end{pgfscope}%
\begin{pgfscope}%
\pgfpathrectangle{\pgfqpoint{0.772069in}{0.515123in}}{\pgfqpoint{3.875000in}{2.695000in}}%
\pgfusepath{clip}%
\pgfsetbuttcap%
\pgfsetroundjoin%
\definecolor{currentfill}{rgb}{1.000000,0.388235,0.278431}%
\pgfsetfillcolor{currentfill}%
\pgfsetlinewidth{1.003750pt}%
\definecolor{currentstroke}{rgb}{1.000000,0.388235,0.278431}%
\pgfsetstrokecolor{currentstroke}%
\pgfsetdash{}{0pt}%
\pgfpathmoveto{\pgfqpoint{4.440932in}{1.106172in}}%
\pgfpathcurveto{\pgfqpoint{4.451982in}{1.106172in}}{\pgfqpoint{4.462581in}{1.110562in}}{\pgfqpoint{4.470395in}{1.118376in}}%
\pgfpathcurveto{\pgfqpoint{4.478209in}{1.126190in}}{\pgfqpoint{4.482599in}{1.136789in}}{\pgfqpoint{4.482599in}{1.147839in}}%
\pgfpathcurveto{\pgfqpoint{4.482599in}{1.158889in}}{\pgfqpoint{4.478209in}{1.169488in}}{\pgfqpoint{4.470395in}{1.177302in}}%
\pgfpathcurveto{\pgfqpoint{4.462581in}{1.185115in}}{\pgfqpoint{4.451982in}{1.189506in}}{\pgfqpoint{4.440932in}{1.189506in}}%
\pgfpathcurveto{\pgfqpoint{4.429882in}{1.189506in}}{\pgfqpoint{4.419283in}{1.185115in}}{\pgfqpoint{4.411470in}{1.177302in}}%
\pgfpathcurveto{\pgfqpoint{4.403656in}{1.169488in}}{\pgfqpoint{4.399266in}{1.158889in}}{\pgfqpoint{4.399266in}{1.147839in}}%
\pgfpathcurveto{\pgfqpoint{4.399266in}{1.136789in}}{\pgfqpoint{4.403656in}{1.126190in}}{\pgfqpoint{4.411470in}{1.118376in}}%
\pgfpathcurveto{\pgfqpoint{4.419283in}{1.110562in}}{\pgfqpoint{4.429882in}{1.106172in}}{\pgfqpoint{4.440932in}{1.106172in}}%
\pgfpathclose%
\pgfusepath{stroke,fill}%
\end{pgfscope}%
\begin{pgfscope}%
\pgfpathrectangle{\pgfqpoint{0.772069in}{0.515123in}}{\pgfqpoint{3.875000in}{2.695000in}}%
\pgfusepath{clip}%
\pgfsetbuttcap%
\pgfsetroundjoin%
\definecolor{currentfill}{rgb}{1.000000,0.843137,0.000000}%
\pgfsetfillcolor{currentfill}%
\pgfsetlinewidth{1.003750pt}%
\definecolor{currentstroke}{rgb}{1.000000,0.843137,0.000000}%
\pgfsetstrokecolor{currentstroke}%
\pgfsetdash{}{0pt}%
\pgfpathmoveto{\pgfqpoint{0.978205in}{0.639229in}}%
\pgfpathcurveto{\pgfqpoint{0.989255in}{0.639229in}}{\pgfqpoint{0.999854in}{0.643619in}}{\pgfqpoint{1.007668in}{0.651433in}}%
\pgfpathcurveto{\pgfqpoint{1.015481in}{0.659247in}}{\pgfqpoint{1.019872in}{0.669846in}}{\pgfqpoint{1.019872in}{0.680896in}}%
\pgfpathcurveto{\pgfqpoint{1.019872in}{0.691946in}}{\pgfqpoint{1.015481in}{0.702545in}}{\pgfqpoint{1.007668in}{0.710359in}}%
\pgfpathcurveto{\pgfqpoint{0.999854in}{0.718172in}}{\pgfqpoint{0.989255in}{0.722563in}}{\pgfqpoint{0.978205in}{0.722563in}}%
\pgfpathcurveto{\pgfqpoint{0.967155in}{0.722563in}}{\pgfqpoint{0.956556in}{0.718172in}}{\pgfqpoint{0.948742in}{0.710359in}}%
\pgfpathcurveto{\pgfqpoint{0.940929in}{0.702545in}}{\pgfqpoint{0.936538in}{0.691946in}}{\pgfqpoint{0.936538in}{0.680896in}}%
\pgfpathcurveto{\pgfqpoint{0.936538in}{0.669846in}}{\pgfqpoint{0.940929in}{0.659247in}}{\pgfqpoint{0.948742in}{0.651433in}}%
\pgfpathcurveto{\pgfqpoint{0.956556in}{0.643619in}}{\pgfqpoint{0.967155in}{0.639229in}}{\pgfqpoint{0.978205in}{0.639229in}}%
\pgfpathclose%
\pgfusepath{stroke,fill}%
\end{pgfscope}%
\begin{pgfscope}%
\pgfpathrectangle{\pgfqpoint{0.772069in}{0.515123in}}{\pgfqpoint{3.875000in}{2.695000in}}%
\pgfusepath{clip}%
\pgfsetbuttcap%
\pgfsetroundjoin%
\definecolor{currentfill}{rgb}{1.000000,0.843137,0.000000}%
\pgfsetfillcolor{currentfill}%
\pgfsetlinewidth{1.003750pt}%
\definecolor{currentstroke}{rgb}{1.000000,0.843137,0.000000}%
\pgfsetstrokecolor{currentstroke}%
\pgfsetdash{}{0pt}%
\pgfpathmoveto{\pgfqpoint{1.342703in}{0.704616in}}%
\pgfpathcurveto{\pgfqpoint{1.353753in}{0.704616in}}{\pgfqpoint{1.364352in}{0.709006in}}{\pgfqpoint{1.372165in}{0.716820in}}%
\pgfpathcurveto{\pgfqpoint{1.379979in}{0.724633in}}{\pgfqpoint{1.384369in}{0.735232in}}{\pgfqpoint{1.384369in}{0.746282in}}%
\pgfpathcurveto{\pgfqpoint{1.384369in}{0.757333in}}{\pgfqpoint{1.379979in}{0.767932in}}{\pgfqpoint{1.372165in}{0.775745in}}%
\pgfpathcurveto{\pgfqpoint{1.364352in}{0.783559in}}{\pgfqpoint{1.353753in}{0.787949in}}{\pgfqpoint{1.342703in}{0.787949in}}%
\pgfpathcurveto{\pgfqpoint{1.331653in}{0.787949in}}{\pgfqpoint{1.321053in}{0.783559in}}{\pgfqpoint{1.313240in}{0.775745in}}%
\pgfpathcurveto{\pgfqpoint{1.305426in}{0.767932in}}{\pgfqpoint{1.301036in}{0.757333in}}{\pgfqpoint{1.301036in}{0.746282in}}%
\pgfpathcurveto{\pgfqpoint{1.301036in}{0.735232in}}{\pgfqpoint{1.305426in}{0.724633in}}{\pgfqpoint{1.313240in}{0.716820in}}%
\pgfpathcurveto{\pgfqpoint{1.321053in}{0.709006in}}{\pgfqpoint{1.331653in}{0.704616in}}{\pgfqpoint{1.342703in}{0.704616in}}%
\pgfpathclose%
\pgfusepath{stroke,fill}%
\end{pgfscope}%
\begin{pgfscope}%
\pgfpathrectangle{\pgfqpoint{0.772069in}{0.515123in}}{\pgfqpoint{3.875000in}{2.695000in}}%
\pgfusepath{clip}%
\pgfsetbuttcap%
\pgfsetroundjoin%
\definecolor{currentfill}{rgb}{1.000000,0.843137,0.000000}%
\pgfsetfillcolor{currentfill}%
\pgfsetlinewidth{1.003750pt}%
\definecolor{currentstroke}{rgb}{1.000000,0.843137,0.000000}%
\pgfsetstrokecolor{currentstroke}%
\pgfsetdash{}{0pt}%
\pgfpathmoveto{\pgfqpoint{1.737575in}{0.770732in}}%
\pgfpathcurveto{\pgfqpoint{1.748625in}{0.770732in}}{\pgfqpoint{1.759224in}{0.775122in}}{\pgfqpoint{1.767038in}{0.782935in}}%
\pgfpathcurveto{\pgfqpoint{1.774851in}{0.790749in}}{\pgfqpoint{1.779242in}{0.801348in}}{\pgfqpoint{1.779242in}{0.812398in}}%
\pgfpathcurveto{\pgfqpoint{1.779242in}{0.823448in}}{\pgfqpoint{1.774851in}{0.834047in}}{\pgfqpoint{1.767038in}{0.841861in}}%
\pgfpathcurveto{\pgfqpoint{1.759224in}{0.849675in}}{\pgfqpoint{1.748625in}{0.854065in}}{\pgfqpoint{1.737575in}{0.854065in}}%
\pgfpathcurveto{\pgfqpoint{1.726525in}{0.854065in}}{\pgfqpoint{1.715926in}{0.849675in}}{\pgfqpoint{1.708112in}{0.841861in}}%
\pgfpathcurveto{\pgfqpoint{1.700299in}{0.834047in}}{\pgfqpoint{1.695908in}{0.823448in}}{\pgfqpoint{1.695908in}{0.812398in}}%
\pgfpathcurveto{\pgfqpoint{1.695908in}{0.801348in}}{\pgfqpoint{1.700299in}{0.790749in}}{\pgfqpoint{1.708112in}{0.782935in}}%
\pgfpathcurveto{\pgfqpoint{1.715926in}{0.775122in}}{\pgfqpoint{1.726525in}{0.770732in}}{\pgfqpoint{1.737575in}{0.770732in}}%
\pgfpathclose%
\pgfusepath{stroke,fill}%
\end{pgfscope}%
\begin{pgfscope}%
\pgfpathrectangle{\pgfqpoint{0.772069in}{0.515123in}}{\pgfqpoint{3.875000in}{2.695000in}}%
\pgfusepath{clip}%
\pgfsetbuttcap%
\pgfsetroundjoin%
\definecolor{currentfill}{rgb}{1.000000,0.843137,0.000000}%
\pgfsetfillcolor{currentfill}%
\pgfsetlinewidth{1.003750pt}%
\definecolor{currentstroke}{rgb}{1.000000,0.843137,0.000000}%
\pgfsetstrokecolor{currentstroke}%
\pgfsetdash{}{0pt}%
\pgfpathmoveto{\pgfqpoint{2.132447in}{0.837820in}}%
\pgfpathcurveto{\pgfqpoint{2.143498in}{0.837820in}}{\pgfqpoint{2.154097in}{0.842210in}}{\pgfqpoint{2.161910in}{0.850024in}}%
\pgfpathcurveto{\pgfqpoint{2.169724in}{0.857837in}}{\pgfqpoint{2.174114in}{0.868436in}}{\pgfqpoint{2.174114in}{0.879486in}}%
\pgfpathcurveto{\pgfqpoint{2.174114in}{0.890537in}}{\pgfqpoint{2.169724in}{0.901136in}}{\pgfqpoint{2.161910in}{0.908949in}}%
\pgfpathcurveto{\pgfqpoint{2.154097in}{0.916763in}}{\pgfqpoint{2.143498in}{0.921153in}}{\pgfqpoint{2.132447in}{0.921153in}}%
\pgfpathcurveto{\pgfqpoint{2.121397in}{0.921153in}}{\pgfqpoint{2.110798in}{0.916763in}}{\pgfqpoint{2.102985in}{0.908949in}}%
\pgfpathcurveto{\pgfqpoint{2.095171in}{0.901136in}}{\pgfqpoint{2.090781in}{0.890537in}}{\pgfqpoint{2.090781in}{0.879486in}}%
\pgfpathcurveto{\pgfqpoint{2.090781in}{0.868436in}}{\pgfqpoint{2.095171in}{0.857837in}}{\pgfqpoint{2.102985in}{0.850024in}}%
\pgfpathcurveto{\pgfqpoint{2.110798in}{0.842210in}}{\pgfqpoint{2.121397in}{0.837820in}}{\pgfqpoint{2.132447in}{0.837820in}}%
\pgfpathclose%
\pgfusepath{stroke,fill}%
\end{pgfscope}%
\begin{pgfscope}%
\pgfpathrectangle{\pgfqpoint{0.772069in}{0.515123in}}{\pgfqpoint{3.875000in}{2.695000in}}%
\pgfusepath{clip}%
\pgfsetbuttcap%
\pgfsetroundjoin%
\definecolor{currentfill}{rgb}{1.000000,0.843137,0.000000}%
\pgfsetfillcolor{currentfill}%
\pgfsetlinewidth{1.003750pt}%
\definecolor{currentstroke}{rgb}{1.000000,0.843137,0.000000}%
\pgfsetstrokecolor{currentstroke}%
\pgfsetdash{}{0pt}%
\pgfpathmoveto{\pgfqpoint{2.496945in}{0.902963in}}%
\pgfpathcurveto{\pgfqpoint{2.507995in}{0.902963in}}{\pgfqpoint{2.518594in}{0.907354in}}{\pgfqpoint{2.526408in}{0.915167in}}%
\pgfpathcurveto{\pgfqpoint{2.534221in}{0.922981in}}{\pgfqpoint{2.538612in}{0.933580in}}{\pgfqpoint{2.538612in}{0.944630in}}%
\pgfpathcurveto{\pgfqpoint{2.538612in}{0.955680in}}{\pgfqpoint{2.534221in}{0.966279in}}{\pgfqpoint{2.526408in}{0.974093in}}%
\pgfpathcurveto{\pgfqpoint{2.518594in}{0.981906in}}{\pgfqpoint{2.507995in}{0.986297in}}{\pgfqpoint{2.496945in}{0.986297in}}%
\pgfpathcurveto{\pgfqpoint{2.485895in}{0.986297in}}{\pgfqpoint{2.475296in}{0.981906in}}{\pgfqpoint{2.467482in}{0.974093in}}%
\pgfpathcurveto{\pgfqpoint{2.459669in}{0.966279in}}{\pgfqpoint{2.455278in}{0.955680in}}{\pgfqpoint{2.455278in}{0.944630in}}%
\pgfpathcurveto{\pgfqpoint{2.455278in}{0.933580in}}{\pgfqpoint{2.459669in}{0.922981in}}{\pgfqpoint{2.467482in}{0.915167in}}%
\pgfpathcurveto{\pgfqpoint{2.475296in}{0.907354in}}{\pgfqpoint{2.485895in}{0.902963in}}{\pgfqpoint{2.496945in}{0.902963in}}%
\pgfpathclose%
\pgfusepath{stroke,fill}%
\end{pgfscope}%
\begin{pgfscope}%
\pgfpathrectangle{\pgfqpoint{0.772069in}{0.515123in}}{\pgfqpoint{3.875000in}{2.695000in}}%
\pgfusepath{clip}%
\pgfsetbuttcap%
\pgfsetroundjoin%
\definecolor{currentfill}{rgb}{1.000000,0.843137,0.000000}%
\pgfsetfillcolor{currentfill}%
\pgfsetlinewidth{1.003750pt}%
\definecolor{currentstroke}{rgb}{1.000000,0.843137,0.000000}%
\pgfsetstrokecolor{currentstroke}%
\pgfsetdash{}{0pt}%
\pgfpathmoveto{\pgfqpoint{2.922192in}{0.974184in}}%
\pgfpathcurveto{\pgfqpoint{2.933242in}{0.974184in}}{\pgfqpoint{2.943841in}{0.978574in}}{\pgfqpoint{2.951655in}{0.986388in}}%
\pgfpathcurveto{\pgfqpoint{2.959469in}{0.994201in}}{\pgfqpoint{2.963859in}{1.004800in}}{\pgfqpoint{2.963859in}{1.015850in}}%
\pgfpathcurveto{\pgfqpoint{2.963859in}{1.026900in}}{\pgfqpoint{2.959469in}{1.037499in}}{\pgfqpoint{2.951655in}{1.045313in}}%
\pgfpathcurveto{\pgfqpoint{2.943841in}{1.053127in}}{\pgfqpoint{2.933242in}{1.057517in}}{\pgfqpoint{2.922192in}{1.057517in}}%
\pgfpathcurveto{\pgfqpoint{2.911142in}{1.057517in}}{\pgfqpoint{2.900543in}{1.053127in}}{\pgfqpoint{2.892730in}{1.045313in}}%
\pgfpathcurveto{\pgfqpoint{2.884916in}{1.037499in}}{\pgfqpoint{2.880526in}{1.026900in}}{\pgfqpoint{2.880526in}{1.015850in}}%
\pgfpathcurveto{\pgfqpoint{2.880526in}{1.004800in}}{\pgfqpoint{2.884916in}{0.994201in}}{\pgfqpoint{2.892730in}{0.986388in}}%
\pgfpathcurveto{\pgfqpoint{2.900543in}{0.978574in}}{\pgfqpoint{2.911142in}{0.974184in}}{\pgfqpoint{2.922192in}{0.974184in}}%
\pgfpathclose%
\pgfusepath{stroke,fill}%
\end{pgfscope}%
\begin{pgfscope}%
\pgfpathrectangle{\pgfqpoint{0.772069in}{0.515123in}}{\pgfqpoint{3.875000in}{2.695000in}}%
\pgfusepath{clip}%
\pgfsetbuttcap%
\pgfsetroundjoin%
\definecolor{currentfill}{rgb}{1.000000,0.843137,0.000000}%
\pgfsetfillcolor{currentfill}%
\pgfsetlinewidth{1.003750pt}%
\definecolor{currentstroke}{rgb}{1.000000,0.843137,0.000000}%
\pgfsetstrokecolor{currentstroke}%
\pgfsetdash{}{0pt}%
\pgfpathmoveto{\pgfqpoint{3.286690in}{1.037626in}}%
\pgfpathcurveto{\pgfqpoint{3.297740in}{1.037626in}}{\pgfqpoint{3.308339in}{1.042016in}}{\pgfqpoint{3.316153in}{1.049830in}}%
\pgfpathcurveto{\pgfqpoint{3.323966in}{1.057643in}}{\pgfqpoint{3.328357in}{1.068242in}}{\pgfqpoint{3.328357in}{1.079292in}}%
\pgfpathcurveto{\pgfqpoint{3.328357in}{1.090342in}}{\pgfqpoint{3.323966in}{1.100941in}}{\pgfqpoint{3.316153in}{1.108755in}}%
\pgfpathcurveto{\pgfqpoint{3.308339in}{1.116569in}}{\pgfqpoint{3.297740in}{1.120959in}}{\pgfqpoint{3.286690in}{1.120959in}}%
\pgfpathcurveto{\pgfqpoint{3.275640in}{1.120959in}}{\pgfqpoint{3.265041in}{1.116569in}}{\pgfqpoint{3.257227in}{1.108755in}}%
\pgfpathcurveto{\pgfqpoint{3.249413in}{1.100941in}}{\pgfqpoint{3.245023in}{1.090342in}}{\pgfqpoint{3.245023in}{1.079292in}}%
\pgfpathcurveto{\pgfqpoint{3.245023in}{1.068242in}}{\pgfqpoint{3.249413in}{1.057643in}}{\pgfqpoint{3.257227in}{1.049830in}}%
\pgfpathcurveto{\pgfqpoint{3.265041in}{1.042016in}}{\pgfqpoint{3.275640in}{1.037626in}}{\pgfqpoint{3.286690in}{1.037626in}}%
\pgfpathclose%
\pgfusepath{stroke,fill}%
\end{pgfscope}%
\begin{pgfscope}%
\pgfpathrectangle{\pgfqpoint{0.772069in}{0.515123in}}{\pgfqpoint{3.875000in}{2.695000in}}%
\pgfusepath{clip}%
\pgfsetbuttcap%
\pgfsetroundjoin%
\definecolor{currentfill}{rgb}{1.000000,0.843137,0.000000}%
\pgfsetfillcolor{currentfill}%
\pgfsetlinewidth{1.003750pt}%
\definecolor{currentstroke}{rgb}{1.000000,0.843137,0.000000}%
\pgfsetstrokecolor{currentstroke}%
\pgfsetdash{}{0pt}%
\pgfpathmoveto{\pgfqpoint{3.681562in}{1.101311in}}%
\pgfpathcurveto{\pgfqpoint{3.692612in}{1.101311in}}{\pgfqpoint{3.703211in}{1.105701in}}{\pgfqpoint{3.711025in}{1.113515in}}%
\pgfpathcurveto{\pgfqpoint{3.718839in}{1.121328in}}{\pgfqpoint{3.723229in}{1.131927in}}{\pgfqpoint{3.723229in}{1.142977in}}%
\pgfpathcurveto{\pgfqpoint{3.723229in}{1.154028in}}{\pgfqpoint{3.718839in}{1.164627in}}{\pgfqpoint{3.711025in}{1.172440in}}%
\pgfpathcurveto{\pgfqpoint{3.703211in}{1.180254in}}{\pgfqpoint{3.692612in}{1.184644in}}{\pgfqpoint{3.681562in}{1.184644in}}%
\pgfpathcurveto{\pgfqpoint{3.670512in}{1.184644in}}{\pgfqpoint{3.659913in}{1.180254in}}{\pgfqpoint{3.652100in}{1.172440in}}%
\pgfpathcurveto{\pgfqpoint{3.644286in}{1.164627in}}{\pgfqpoint{3.639896in}{1.154028in}}{\pgfqpoint{3.639896in}{1.142977in}}%
\pgfpathcurveto{\pgfqpoint{3.639896in}{1.131927in}}{\pgfqpoint{3.644286in}{1.121328in}}{\pgfqpoint{3.652100in}{1.113515in}}%
\pgfpathcurveto{\pgfqpoint{3.659913in}{1.105701in}}{\pgfqpoint{3.670512in}{1.101311in}}{\pgfqpoint{3.681562in}{1.101311in}}%
\pgfpathclose%
\pgfusepath{stroke,fill}%
\end{pgfscope}%
\begin{pgfscope}%
\pgfpathrectangle{\pgfqpoint{0.772069in}{0.515123in}}{\pgfqpoint{3.875000in}{2.695000in}}%
\pgfusepath{clip}%
\pgfsetbuttcap%
\pgfsetroundjoin%
\definecolor{currentfill}{rgb}{1.000000,0.843137,0.000000}%
\pgfsetfillcolor{currentfill}%
\pgfsetlinewidth{1.003750pt}%
\definecolor{currentstroke}{rgb}{1.000000,0.843137,0.000000}%
\pgfsetstrokecolor{currentstroke}%
\pgfsetdash{}{0pt}%
\pgfpathmoveto{\pgfqpoint{4.015685in}{1.162079in}}%
\pgfpathcurveto{\pgfqpoint{4.026735in}{1.162079in}}{\pgfqpoint{4.037334in}{1.166469in}}{\pgfqpoint{4.045148in}{1.174283in}}%
\pgfpathcurveto{\pgfqpoint{4.052962in}{1.182097in}}{\pgfqpoint{4.057352in}{1.192696in}}{\pgfqpoint{4.057352in}{1.203746in}}%
\pgfpathcurveto{\pgfqpoint{4.057352in}{1.214796in}}{\pgfqpoint{4.052962in}{1.225395in}}{\pgfqpoint{4.045148in}{1.233208in}}%
\pgfpathcurveto{\pgfqpoint{4.037334in}{1.241022in}}{\pgfqpoint{4.026735in}{1.245412in}}{\pgfqpoint{4.015685in}{1.245412in}}%
\pgfpathcurveto{\pgfqpoint{4.004635in}{1.245412in}}{\pgfqpoint{3.994036in}{1.241022in}}{\pgfqpoint{3.986222in}{1.233208in}}%
\pgfpathcurveto{\pgfqpoint{3.978409in}{1.225395in}}{\pgfqpoint{3.974018in}{1.214796in}}{\pgfqpoint{3.974018in}{1.203746in}}%
\pgfpathcurveto{\pgfqpoint{3.974018in}{1.192696in}}{\pgfqpoint{3.978409in}{1.182097in}}{\pgfqpoint{3.986222in}{1.174283in}}%
\pgfpathcurveto{\pgfqpoint{3.994036in}{1.166469in}}{\pgfqpoint{4.004635in}{1.162079in}}{\pgfqpoint{4.015685in}{1.162079in}}%
\pgfpathclose%
\pgfusepath{stroke,fill}%
\end{pgfscope}%
\begin{pgfscope}%
\pgfpathrectangle{\pgfqpoint{0.772069in}{0.515123in}}{\pgfqpoint{3.875000in}{2.695000in}}%
\pgfusepath{clip}%
\pgfsetbuttcap%
\pgfsetroundjoin%
\definecolor{currentfill}{rgb}{1.000000,0.843137,0.000000}%
\pgfsetfillcolor{currentfill}%
\pgfsetlinewidth{1.003750pt}%
\definecolor{currentstroke}{rgb}{1.000000,0.843137,0.000000}%
\pgfsetstrokecolor{currentstroke}%
\pgfsetdash{}{0pt}%
\pgfpathmoveto{\pgfqpoint{4.440932in}{1.235001in}}%
\pgfpathcurveto{\pgfqpoint{4.451982in}{1.235001in}}{\pgfqpoint{4.462581in}{1.239391in}}{\pgfqpoint{4.470395in}{1.247205in}}%
\pgfpathcurveto{\pgfqpoint{4.478209in}{1.255018in}}{\pgfqpoint{4.482599in}{1.265617in}}{\pgfqpoint{4.482599in}{1.276668in}}%
\pgfpathcurveto{\pgfqpoint{4.482599in}{1.287718in}}{\pgfqpoint{4.478209in}{1.298317in}}{\pgfqpoint{4.470395in}{1.306130in}}%
\pgfpathcurveto{\pgfqpoint{4.462581in}{1.313944in}}{\pgfqpoint{4.451982in}{1.318334in}}{\pgfqpoint{4.440932in}{1.318334in}}%
\pgfpathcurveto{\pgfqpoint{4.429882in}{1.318334in}}{\pgfqpoint{4.419283in}{1.313944in}}{\pgfqpoint{4.411470in}{1.306130in}}%
\pgfpathcurveto{\pgfqpoint{4.403656in}{1.298317in}}{\pgfqpoint{4.399266in}{1.287718in}}{\pgfqpoint{4.399266in}{1.276668in}}%
\pgfpathcurveto{\pgfqpoint{4.399266in}{1.265617in}}{\pgfqpoint{4.403656in}{1.255018in}}{\pgfqpoint{4.411470in}{1.247205in}}%
\pgfpathcurveto{\pgfqpoint{4.419283in}{1.239391in}}{\pgfqpoint{4.429882in}{1.235001in}}{\pgfqpoint{4.440932in}{1.235001in}}%
\pgfpathclose%
\pgfusepath{stroke,fill}%
\end{pgfscope}%
\begin{pgfscope}%
\pgfpathrectangle{\pgfqpoint{0.772069in}{0.515123in}}{\pgfqpoint{3.875000in}{2.695000in}}%
\pgfusepath{clip}%
\pgfsetbuttcap%
\pgfsetroundjoin%
\definecolor{currentfill}{rgb}{0.196078,0.803922,0.196078}%
\pgfsetfillcolor{currentfill}%
\pgfsetlinewidth{1.003750pt}%
\definecolor{currentstroke}{rgb}{0.196078,0.803922,0.196078}%
\pgfsetstrokecolor{currentstroke}%
\pgfsetdash{}{0pt}%
\pgfpathmoveto{\pgfqpoint{0.978205in}{0.696108in}}%
\pgfpathcurveto{\pgfqpoint{0.989255in}{0.696108in}}{\pgfqpoint{0.999854in}{0.700498in}}{\pgfqpoint{1.007668in}{0.708312in}}%
\pgfpathcurveto{\pgfqpoint{1.015481in}{0.716126in}}{\pgfqpoint{1.019872in}{0.726725in}}{\pgfqpoint{1.019872in}{0.737775in}}%
\pgfpathcurveto{\pgfqpoint{1.019872in}{0.748825in}}{\pgfqpoint{1.015481in}{0.759424in}}{\pgfqpoint{1.007668in}{0.767238in}}%
\pgfpathcurveto{\pgfqpoint{0.999854in}{0.775051in}}{\pgfqpoint{0.989255in}{0.779442in}}{\pgfqpoint{0.978205in}{0.779442in}}%
\pgfpathcurveto{\pgfqpoint{0.967155in}{0.779442in}}{\pgfqpoint{0.956556in}{0.775051in}}{\pgfqpoint{0.948742in}{0.767238in}}%
\pgfpathcurveto{\pgfqpoint{0.940929in}{0.759424in}}{\pgfqpoint{0.936538in}{0.748825in}}{\pgfqpoint{0.936538in}{0.737775in}}%
\pgfpathcurveto{\pgfqpoint{0.936538in}{0.726725in}}{\pgfqpoint{0.940929in}{0.716126in}}{\pgfqpoint{0.948742in}{0.708312in}}%
\pgfpathcurveto{\pgfqpoint{0.956556in}{0.700498in}}{\pgfqpoint{0.967155in}{0.696108in}}{\pgfqpoint{0.978205in}{0.696108in}}%
\pgfpathclose%
\pgfusepath{stroke,fill}%
\end{pgfscope}%
\begin{pgfscope}%
\pgfpathrectangle{\pgfqpoint{0.772069in}{0.515123in}}{\pgfqpoint{3.875000in}{2.695000in}}%
\pgfusepath{clip}%
\pgfsetbuttcap%
\pgfsetroundjoin%
\definecolor{currentfill}{rgb}{0.196078,0.803922,0.196078}%
\pgfsetfillcolor{currentfill}%
\pgfsetlinewidth{1.003750pt}%
\definecolor{currentstroke}{rgb}{0.196078,0.803922,0.196078}%
\pgfsetstrokecolor{currentstroke}%
\pgfsetdash{}{0pt}%
\pgfpathmoveto{\pgfqpoint{1.342703in}{0.814728in}}%
\pgfpathcurveto{\pgfqpoint{1.353753in}{0.814728in}}{\pgfqpoint{1.364352in}{0.819118in}}{\pgfqpoint{1.372165in}{0.826932in}}%
\pgfpathcurveto{\pgfqpoint{1.379979in}{0.834745in}}{\pgfqpoint{1.384369in}{0.845344in}}{\pgfqpoint{1.384369in}{0.856394in}}%
\pgfpathcurveto{\pgfqpoint{1.384369in}{0.867445in}}{\pgfqpoint{1.379979in}{0.878044in}}{\pgfqpoint{1.372165in}{0.885857in}}%
\pgfpathcurveto{\pgfqpoint{1.364352in}{0.893671in}}{\pgfqpoint{1.353753in}{0.898061in}}{\pgfqpoint{1.342703in}{0.898061in}}%
\pgfpathcurveto{\pgfqpoint{1.331653in}{0.898061in}}{\pgfqpoint{1.321053in}{0.893671in}}{\pgfqpoint{1.313240in}{0.885857in}}%
\pgfpathcurveto{\pgfqpoint{1.305426in}{0.878044in}}{\pgfqpoint{1.301036in}{0.867445in}}{\pgfqpoint{1.301036in}{0.856394in}}%
\pgfpathcurveto{\pgfqpoint{1.301036in}{0.845344in}}{\pgfqpoint{1.305426in}{0.834745in}}{\pgfqpoint{1.313240in}{0.826932in}}%
\pgfpathcurveto{\pgfqpoint{1.321053in}{0.819118in}}{\pgfqpoint{1.331653in}{0.814728in}}{\pgfqpoint{1.342703in}{0.814728in}}%
\pgfpathclose%
\pgfusepath{stroke,fill}%
\end{pgfscope}%
\begin{pgfscope}%
\pgfpathrectangle{\pgfqpoint{0.772069in}{0.515123in}}{\pgfqpoint{3.875000in}{2.695000in}}%
\pgfusepath{clip}%
\pgfsetbuttcap%
\pgfsetroundjoin%
\definecolor{currentfill}{rgb}{0.196078,0.803922,0.196078}%
\pgfsetfillcolor{currentfill}%
\pgfsetlinewidth{1.003750pt}%
\definecolor{currentstroke}{rgb}{0.196078,0.803922,0.196078}%
\pgfsetstrokecolor{currentstroke}%
\pgfsetdash{}{0pt}%
\pgfpathmoveto{\pgfqpoint{1.737575in}{0.934077in}}%
\pgfpathcurveto{\pgfqpoint{1.748625in}{0.934077in}}{\pgfqpoint{1.759224in}{0.938467in}}{\pgfqpoint{1.767038in}{0.946280in}}%
\pgfpathcurveto{\pgfqpoint{1.774851in}{0.954094in}}{\pgfqpoint{1.779242in}{0.964693in}}{\pgfqpoint{1.779242in}{0.975743in}}%
\pgfpathcurveto{\pgfqpoint{1.779242in}{0.986793in}}{\pgfqpoint{1.774851in}{0.997392in}}{\pgfqpoint{1.767038in}{1.005206in}}%
\pgfpathcurveto{\pgfqpoint{1.759224in}{1.013020in}}{\pgfqpoint{1.748625in}{1.017410in}}{\pgfqpoint{1.737575in}{1.017410in}}%
\pgfpathcurveto{\pgfqpoint{1.726525in}{1.017410in}}{\pgfqpoint{1.715926in}{1.013020in}}{\pgfqpoint{1.708112in}{1.005206in}}%
\pgfpathcurveto{\pgfqpoint{1.700299in}{0.997392in}}{\pgfqpoint{1.695908in}{0.986793in}}{\pgfqpoint{1.695908in}{0.975743in}}%
\pgfpathcurveto{\pgfqpoint{1.695908in}{0.964693in}}{\pgfqpoint{1.700299in}{0.954094in}}{\pgfqpoint{1.708112in}{0.946280in}}%
\pgfpathcurveto{\pgfqpoint{1.715926in}{0.938467in}}{\pgfqpoint{1.726525in}{0.934077in}}{\pgfqpoint{1.737575in}{0.934077in}}%
\pgfpathclose%
\pgfusepath{stroke,fill}%
\end{pgfscope}%
\begin{pgfscope}%
\pgfpathrectangle{\pgfqpoint{0.772069in}{0.515123in}}{\pgfqpoint{3.875000in}{2.695000in}}%
\pgfusepath{clip}%
\pgfsetbuttcap%
\pgfsetroundjoin%
\definecolor{currentfill}{rgb}{0.196078,0.803922,0.196078}%
\pgfsetfillcolor{currentfill}%
\pgfsetlinewidth{1.003750pt}%
\definecolor{currentstroke}{rgb}{0.196078,0.803922,0.196078}%
\pgfsetstrokecolor{currentstroke}%
\pgfsetdash{}{0pt}%
\pgfpathmoveto{\pgfqpoint{2.132447in}{1.052696in}}%
\pgfpathcurveto{\pgfqpoint{2.143498in}{1.052696in}}{\pgfqpoint{2.154097in}{1.057086in}}{\pgfqpoint{2.161910in}{1.064900in}}%
\pgfpathcurveto{\pgfqpoint{2.169724in}{1.072714in}}{\pgfqpoint{2.174114in}{1.083313in}}{\pgfqpoint{2.174114in}{1.094363in}}%
\pgfpathcurveto{\pgfqpoint{2.174114in}{1.105413in}}{\pgfqpoint{2.169724in}{1.116012in}}{\pgfqpoint{2.161910in}{1.123826in}}%
\pgfpathcurveto{\pgfqpoint{2.154097in}{1.131639in}}{\pgfqpoint{2.143498in}{1.136030in}}{\pgfqpoint{2.132447in}{1.136030in}}%
\pgfpathcurveto{\pgfqpoint{2.121397in}{1.136030in}}{\pgfqpoint{2.110798in}{1.131639in}}{\pgfqpoint{2.102985in}{1.123826in}}%
\pgfpathcurveto{\pgfqpoint{2.095171in}{1.116012in}}{\pgfqpoint{2.090781in}{1.105413in}}{\pgfqpoint{2.090781in}{1.094363in}}%
\pgfpathcurveto{\pgfqpoint{2.090781in}{1.083313in}}{\pgfqpoint{2.095171in}{1.072714in}}{\pgfqpoint{2.102985in}{1.064900in}}%
\pgfpathcurveto{\pgfqpoint{2.110798in}{1.057086in}}{\pgfqpoint{2.121397in}{1.052696in}}{\pgfqpoint{2.132447in}{1.052696in}}%
\pgfpathclose%
\pgfusepath{stroke,fill}%
\end{pgfscope}%
\begin{pgfscope}%
\pgfpathrectangle{\pgfqpoint{0.772069in}{0.515123in}}{\pgfqpoint{3.875000in}{2.695000in}}%
\pgfusepath{clip}%
\pgfsetbuttcap%
\pgfsetroundjoin%
\definecolor{currentfill}{rgb}{0.196078,0.803922,0.196078}%
\pgfsetfillcolor{currentfill}%
\pgfsetlinewidth{1.003750pt}%
\definecolor{currentstroke}{rgb}{0.196078,0.803922,0.196078}%
\pgfsetstrokecolor{currentstroke}%
\pgfsetdash{}{0pt}%
\pgfpathmoveto{\pgfqpoint{2.496945in}{1.169371in}}%
\pgfpathcurveto{\pgfqpoint{2.507995in}{1.169371in}}{\pgfqpoint{2.518594in}{1.173761in}}{\pgfqpoint{2.526408in}{1.181575in}}%
\pgfpathcurveto{\pgfqpoint{2.534221in}{1.189389in}}{\pgfqpoint{2.538612in}{1.199988in}}{\pgfqpoint{2.538612in}{1.211038in}}%
\pgfpathcurveto{\pgfqpoint{2.538612in}{1.222088in}}{\pgfqpoint{2.534221in}{1.232687in}}{\pgfqpoint{2.526408in}{1.240501in}}%
\pgfpathcurveto{\pgfqpoint{2.518594in}{1.248314in}}{\pgfqpoint{2.507995in}{1.252705in}}{\pgfqpoint{2.496945in}{1.252705in}}%
\pgfpathcurveto{\pgfqpoint{2.485895in}{1.252705in}}{\pgfqpoint{2.475296in}{1.248314in}}{\pgfqpoint{2.467482in}{1.240501in}}%
\pgfpathcurveto{\pgfqpoint{2.459669in}{1.232687in}}{\pgfqpoint{2.455278in}{1.222088in}}{\pgfqpoint{2.455278in}{1.211038in}}%
\pgfpathcurveto{\pgfqpoint{2.455278in}{1.199988in}}{\pgfqpoint{2.459669in}{1.189389in}}{\pgfqpoint{2.467482in}{1.181575in}}%
\pgfpathcurveto{\pgfqpoint{2.475296in}{1.173761in}}{\pgfqpoint{2.485895in}{1.169371in}}{\pgfqpoint{2.496945in}{1.169371in}}%
\pgfpathclose%
\pgfusepath{stroke,fill}%
\end{pgfscope}%
\begin{pgfscope}%
\pgfpathrectangle{\pgfqpoint{0.772069in}{0.515123in}}{\pgfqpoint{3.875000in}{2.695000in}}%
\pgfusepath{clip}%
\pgfsetbuttcap%
\pgfsetroundjoin%
\definecolor{currentfill}{rgb}{0.196078,0.803922,0.196078}%
\pgfsetfillcolor{currentfill}%
\pgfsetlinewidth{1.003750pt}%
\definecolor{currentstroke}{rgb}{0.196078,0.803922,0.196078}%
\pgfsetstrokecolor{currentstroke}%
\pgfsetdash{}{0pt}%
\pgfpathmoveto{\pgfqpoint{2.922192in}{1.298200in}}%
\pgfpathcurveto{\pgfqpoint{2.933242in}{1.298200in}}{\pgfqpoint{2.943841in}{1.302590in}}{\pgfqpoint{2.951655in}{1.310404in}}%
\pgfpathcurveto{\pgfqpoint{2.959469in}{1.318217in}}{\pgfqpoint{2.963859in}{1.328816in}}{\pgfqpoint{2.963859in}{1.339866in}}%
\pgfpathcurveto{\pgfqpoint{2.963859in}{1.350917in}}{\pgfqpoint{2.959469in}{1.361516in}}{\pgfqpoint{2.951655in}{1.369329in}}%
\pgfpathcurveto{\pgfqpoint{2.943841in}{1.377143in}}{\pgfqpoint{2.933242in}{1.381533in}}{\pgfqpoint{2.922192in}{1.381533in}}%
\pgfpathcurveto{\pgfqpoint{2.911142in}{1.381533in}}{\pgfqpoint{2.900543in}{1.377143in}}{\pgfqpoint{2.892730in}{1.369329in}}%
\pgfpathcurveto{\pgfqpoint{2.884916in}{1.361516in}}{\pgfqpoint{2.880526in}{1.350917in}}{\pgfqpoint{2.880526in}{1.339866in}}%
\pgfpathcurveto{\pgfqpoint{2.880526in}{1.328816in}}{\pgfqpoint{2.884916in}{1.318217in}}{\pgfqpoint{2.892730in}{1.310404in}}%
\pgfpathcurveto{\pgfqpoint{2.900543in}{1.302590in}}{\pgfqpoint{2.911142in}{1.298200in}}{\pgfqpoint{2.922192in}{1.298200in}}%
\pgfpathclose%
\pgfusepath{stroke,fill}%
\end{pgfscope}%
\begin{pgfscope}%
\pgfpathrectangle{\pgfqpoint{0.772069in}{0.515123in}}{\pgfqpoint{3.875000in}{2.695000in}}%
\pgfusepath{clip}%
\pgfsetbuttcap%
\pgfsetroundjoin%
\definecolor{currentfill}{rgb}{0.196078,0.803922,0.196078}%
\pgfsetfillcolor{currentfill}%
\pgfsetlinewidth{1.003750pt}%
\definecolor{currentstroke}{rgb}{0.196078,0.803922,0.196078}%
\pgfsetstrokecolor{currentstroke}%
\pgfsetdash{}{0pt}%
\pgfpathmoveto{\pgfqpoint{3.286690in}{1.412444in}}%
\pgfpathcurveto{\pgfqpoint{3.297740in}{1.412444in}}{\pgfqpoint{3.308339in}{1.416834in}}{\pgfqpoint{3.316153in}{1.424648in}}%
\pgfpathcurveto{\pgfqpoint{3.323966in}{1.432462in}}{\pgfqpoint{3.328357in}{1.443061in}}{\pgfqpoint{3.328357in}{1.454111in}}%
\pgfpathcurveto{\pgfqpoint{3.328357in}{1.465161in}}{\pgfqpoint{3.323966in}{1.475760in}}{\pgfqpoint{3.316153in}{1.483574in}}%
\pgfpathcurveto{\pgfqpoint{3.308339in}{1.491387in}}{\pgfqpoint{3.297740in}{1.495777in}}{\pgfqpoint{3.286690in}{1.495777in}}%
\pgfpathcurveto{\pgfqpoint{3.275640in}{1.495777in}}{\pgfqpoint{3.265041in}{1.491387in}}{\pgfqpoint{3.257227in}{1.483574in}}%
\pgfpathcurveto{\pgfqpoint{3.249413in}{1.475760in}}{\pgfqpoint{3.245023in}{1.465161in}}{\pgfqpoint{3.245023in}{1.454111in}}%
\pgfpathcurveto{\pgfqpoint{3.245023in}{1.443061in}}{\pgfqpoint{3.249413in}{1.432462in}}{\pgfqpoint{3.257227in}{1.424648in}}%
\pgfpathcurveto{\pgfqpoint{3.265041in}{1.416834in}}{\pgfqpoint{3.275640in}{1.412444in}}{\pgfqpoint{3.286690in}{1.412444in}}%
\pgfpathclose%
\pgfusepath{stroke,fill}%
\end{pgfscope}%
\begin{pgfscope}%
\pgfpathrectangle{\pgfqpoint{0.772069in}{0.515123in}}{\pgfqpoint{3.875000in}{2.695000in}}%
\pgfusepath{clip}%
\pgfsetbuttcap%
\pgfsetroundjoin%
\definecolor{currentfill}{rgb}{0.196078,0.803922,0.196078}%
\pgfsetfillcolor{currentfill}%
\pgfsetlinewidth{1.003750pt}%
\definecolor{currentstroke}{rgb}{0.196078,0.803922,0.196078}%
\pgfsetstrokecolor{currentstroke}%
\pgfsetdash{}{0pt}%
\pgfpathmoveto{\pgfqpoint{3.681562in}{1.531550in}}%
\pgfpathcurveto{\pgfqpoint{3.692612in}{1.531550in}}{\pgfqpoint{3.703211in}{1.535940in}}{\pgfqpoint{3.711025in}{1.543754in}}%
\pgfpathcurveto{\pgfqpoint{3.718839in}{1.551567in}}{\pgfqpoint{3.723229in}{1.562166in}}{\pgfqpoint{3.723229in}{1.573216in}}%
\pgfpathcurveto{\pgfqpoint{3.723229in}{1.584267in}}{\pgfqpoint{3.718839in}{1.594866in}}{\pgfqpoint{3.711025in}{1.602679in}}%
\pgfpathcurveto{\pgfqpoint{3.703211in}{1.610493in}}{\pgfqpoint{3.692612in}{1.614883in}}{\pgfqpoint{3.681562in}{1.614883in}}%
\pgfpathcurveto{\pgfqpoint{3.670512in}{1.614883in}}{\pgfqpoint{3.659913in}{1.610493in}}{\pgfqpoint{3.652100in}{1.602679in}}%
\pgfpathcurveto{\pgfqpoint{3.644286in}{1.594866in}}{\pgfqpoint{3.639896in}{1.584267in}}{\pgfqpoint{3.639896in}{1.573216in}}%
\pgfpathcurveto{\pgfqpoint{3.639896in}{1.562166in}}{\pgfqpoint{3.644286in}{1.551567in}}{\pgfqpoint{3.652100in}{1.543754in}}%
\pgfpathcurveto{\pgfqpoint{3.659913in}{1.535940in}}{\pgfqpoint{3.670512in}{1.531550in}}{\pgfqpoint{3.681562in}{1.531550in}}%
\pgfpathclose%
\pgfusepath{stroke,fill}%
\end{pgfscope}%
\begin{pgfscope}%
\pgfpathrectangle{\pgfqpoint{0.772069in}{0.515123in}}{\pgfqpoint{3.875000in}{2.695000in}}%
\pgfusepath{clip}%
\pgfsetbuttcap%
\pgfsetroundjoin%
\definecolor{currentfill}{rgb}{0.196078,0.803922,0.196078}%
\pgfsetfillcolor{currentfill}%
\pgfsetlinewidth{1.003750pt}%
\definecolor{currentstroke}{rgb}{0.196078,0.803922,0.196078}%
\pgfsetstrokecolor{currentstroke}%
\pgfsetdash{}{0pt}%
\pgfpathmoveto{\pgfqpoint{4.015685in}{1.643363in}}%
\pgfpathcurveto{\pgfqpoint{4.026735in}{1.643363in}}{\pgfqpoint{4.037334in}{1.647754in}}{\pgfqpoint{4.045148in}{1.655567in}}%
\pgfpathcurveto{\pgfqpoint{4.052962in}{1.663381in}}{\pgfqpoint{4.057352in}{1.673980in}}{\pgfqpoint{4.057352in}{1.685030in}}%
\pgfpathcurveto{\pgfqpoint{4.057352in}{1.696080in}}{\pgfqpoint{4.052962in}{1.706679in}}{\pgfqpoint{4.045148in}{1.714493in}}%
\pgfpathcurveto{\pgfqpoint{4.037334in}{1.722306in}}{\pgfqpoint{4.026735in}{1.726697in}}{\pgfqpoint{4.015685in}{1.726697in}}%
\pgfpathcurveto{\pgfqpoint{4.004635in}{1.726697in}}{\pgfqpoint{3.994036in}{1.722306in}}{\pgfqpoint{3.986222in}{1.714493in}}%
\pgfpathcurveto{\pgfqpoint{3.978409in}{1.706679in}}{\pgfqpoint{3.974018in}{1.696080in}}{\pgfqpoint{3.974018in}{1.685030in}}%
\pgfpathcurveto{\pgfqpoint{3.974018in}{1.673980in}}{\pgfqpoint{3.978409in}{1.663381in}}{\pgfqpoint{3.986222in}{1.655567in}}%
\pgfpathcurveto{\pgfqpoint{3.994036in}{1.647754in}}{\pgfqpoint{4.004635in}{1.643363in}}{\pgfqpoint{4.015685in}{1.643363in}}%
\pgfpathclose%
\pgfusepath{stroke,fill}%
\end{pgfscope}%
\begin{pgfscope}%
\pgfpathrectangle{\pgfqpoint{0.772069in}{0.515123in}}{\pgfqpoint{3.875000in}{2.695000in}}%
\pgfusepath{clip}%
\pgfsetbuttcap%
\pgfsetroundjoin%
\definecolor{currentfill}{rgb}{0.196078,0.803922,0.196078}%
\pgfsetfillcolor{currentfill}%
\pgfsetlinewidth{1.003750pt}%
\definecolor{currentstroke}{rgb}{0.196078,0.803922,0.196078}%
\pgfsetstrokecolor{currentstroke}%
\pgfsetdash{}{0pt}%
\pgfpathmoveto{\pgfqpoint{4.440932in}{1.772192in}}%
\pgfpathcurveto{\pgfqpoint{4.451982in}{1.772192in}}{\pgfqpoint{4.462581in}{1.776582in}}{\pgfqpoint{4.470395in}{1.784396in}}%
\pgfpathcurveto{\pgfqpoint{4.478209in}{1.792210in}}{\pgfqpoint{4.482599in}{1.802809in}}{\pgfqpoint{4.482599in}{1.813859in}}%
\pgfpathcurveto{\pgfqpoint{4.482599in}{1.824909in}}{\pgfqpoint{4.478209in}{1.835508in}}{\pgfqpoint{4.470395in}{1.843321in}}%
\pgfpathcurveto{\pgfqpoint{4.462581in}{1.851135in}}{\pgfqpoint{4.451982in}{1.855525in}}{\pgfqpoint{4.440932in}{1.855525in}}%
\pgfpathcurveto{\pgfqpoint{4.429882in}{1.855525in}}{\pgfqpoint{4.419283in}{1.851135in}}{\pgfqpoint{4.411470in}{1.843321in}}%
\pgfpathcurveto{\pgfqpoint{4.403656in}{1.835508in}}{\pgfqpoint{4.399266in}{1.824909in}}{\pgfqpoint{4.399266in}{1.813859in}}%
\pgfpathcurveto{\pgfqpoint{4.399266in}{1.802809in}}{\pgfqpoint{4.403656in}{1.792210in}}{\pgfqpoint{4.411470in}{1.784396in}}%
\pgfpathcurveto{\pgfqpoint{4.419283in}{1.776582in}}{\pgfqpoint{4.429882in}{1.772192in}}{\pgfqpoint{4.440932in}{1.772192in}}%
\pgfpathclose%
\pgfusepath{stroke,fill}%
\end{pgfscope}%
\begin{pgfscope}%
\pgfsetrectcap%
\pgfsetmiterjoin%
\pgfsetlinewidth{0.803000pt}%
\definecolor{currentstroke}{rgb}{0.000000,0.000000,0.000000}%
\pgfsetstrokecolor{currentstroke}%
\pgfsetdash{}{0pt}%
\pgfpathmoveto{\pgfqpoint{0.772069in}{0.515123in}}%
\pgfpathlineto{\pgfqpoint{0.772069in}{3.210123in}}%
\pgfusepath{stroke}%
\end{pgfscope}%
\begin{pgfscope}%
\pgfsetrectcap%
\pgfsetmiterjoin%
\pgfsetlinewidth{0.803000pt}%
\definecolor{currentstroke}{rgb}{0.000000,0.000000,0.000000}%
\pgfsetstrokecolor{currentstroke}%
\pgfsetdash{}{0pt}%
\pgfpathmoveto{\pgfqpoint{4.647069in}{0.515123in}}%
\pgfpathlineto{\pgfqpoint{4.647069in}{3.210123in}}%
\pgfusepath{stroke}%
\end{pgfscope}%
\begin{pgfscope}%
\pgfsetrectcap%
\pgfsetmiterjoin%
\pgfsetlinewidth{0.803000pt}%
\definecolor{currentstroke}{rgb}{0.000000,0.000000,0.000000}%
\pgfsetstrokecolor{currentstroke}%
\pgfsetdash{}{0pt}%
\pgfpathmoveto{\pgfqpoint{0.772069in}{0.515123in}}%
\pgfpathlineto{\pgfqpoint{4.647069in}{0.515123in}}%
\pgfusepath{stroke}%
\end{pgfscope}%
\begin{pgfscope}%
\pgfsetrectcap%
\pgfsetmiterjoin%
\pgfsetlinewidth{0.803000pt}%
\definecolor{currentstroke}{rgb}{0.000000,0.000000,0.000000}%
\pgfsetstrokecolor{currentstroke}%
\pgfsetdash{}{0pt}%
\pgfpathmoveto{\pgfqpoint{0.772069in}{3.210123in}}%
\pgfpathlineto{\pgfqpoint{4.647069in}{3.210123in}}%
\pgfusepath{stroke}%
\end{pgfscope}%
\end{pgfpicture}%
\makeatother%
\endgroup%

    \caption{Voltajes (\textcolor{Blue}{$V$}, \textcolor{Red}{$V_1$}, \textcolor{Yellow}{$V_2$}, \textcolor{Green}{$V_3$}) frente a intensidad (I) con regresión lineal}
  \end{figure}

  En cuanto a los coeficientes de regresión lineal, calculados con la fórmula \ref{ec:r}, de nuevo obtenemos resultados satisfactorios: $r = 0,99997$, $r_1 = 0,99998$, $r_2 = 0,99997$ y $r_3 = 0,99997$, un ajuste de cuatro nueves en ambos casos.

  También podemos comprobar si el valor de la resistencia equivalente calculado en \ref{sec:reseqserie} ($R_S = 7,855 \cdot 10^5$) se corresponde con la pendiente de la gráfica ($b = 7,90 \cdot 10^5$), y podemos ver que se adecuan. Pasa lo mismo si tomamos cualquiera de las otras rectas con respecto a $V_1$, $V_2$ y $V_3$, cada una tiene una pendiente similar a la resistencia de ese tramo. Se puede observar a simple vista viendo que la recta roja corresponde a $R_1$, la resistencia más pequeña, y también la que menor pendiente tiene, mientras que la recta verde es de $R_3$, la más grande y la de mayor pendiente. La recta azul tiene más pendiente que todas ya que es la suma de las tres resistencias en serie.


  \newpage
  \section{Circuito en Paralelo}

  En este apartado construiremos un circuito utilizando tres resistencias ($R_1$, $R_2$ y $R_3$) colocadas en paralelo. Ahora mediremos la intensidad en cada resistencia y total, además del voltaje del circuito. Colocamos los componentes siguiendo el diagrama:

  \begin{figure}[H]
    \centering
    \begin{circuitikz}[european]
      \draw (0,0) to[voltage source] (0,3) -- (3.25,3)
      to[R=$R_2$] (3.25,0) -- (0,0);
      \draw (3.25, 2.75) -- (2, 2.75)
      to[R=$R_1$] (2, 0.25) -- (4, 0.25);
      \draw (3.25, 2.75) -- (4.5, 2.75)
      to[R=$R_3$] (4.5, 0.25) -- (4, 0.25);
    \end{circuitikz}
    \caption{Circuito con tres resistencias en paralelo}
    \label{circuito:paralelo}
  \end{figure}

  \subsection{Procedimiento de medición}

  Con el objetivo de medir las distintas magnitudes, conectaremos el polímetro en serie para las itensidades y en paralelo para los voltajes.

  \begin{figure}[H]
    \centering
    \raisebox{0.53in}{
    \begin{circuitikz}[european]
      \draw (0,0) to[voltage source] (0,3) -- (3.25,3)
      to[R=$R_2$] (3.25,0) -- (0,0);
      \draw (3.25, 2.75) -- (2, 2.75)
      to[R=$R_1$] (2, 0.25) -- (4, 0.25);
      \draw (3.25, 2.75) -- (4.5, 2.75)
      to[R=$R_3$] (4.5, 0.25) -- (4, 0.25);
      \draw (3.25, 3) -- (6.5, 3)
      to[voltmeter, l_=$V$] (6.5, 0) -- (3.25, 0);
    \end{circuitikz}} \qquad
    \begin{circuitikz}[european]
      \draw (0,0) to[voltage source] (0,5) -- (4.5,5) -- (4.5, 4.75)
      to[R=$R_2$] (4.5, 2.25) to[ammeter, l=$I_2$] (4.5,0.25) -- (4.5, 0) -- (3, 0)
      to[ammeter, l_=$I$] (0,0);
      \draw (4.5, 4.75) -- (3, 4.75)
      to[R=$R_1$] (3, 2.25) to[ammeter, l=$I_1$] (3, 0.25) -- (4.5, 0.25);
      \draw (4.5, 4.75) -- (6, 4.75)
      to[R=$R_3$] (6, 2.25) to[ammeter, l=$I_3$] (6, 0.25) -- (4.5, 0.25);
    \end{circuitikz}
    \caption{Medición de potencial del circuito ($V$) y de intensidades, total ($I$) y de cada resistencia ($I_1$, $I_2$ y $I_3$ respectivamente)}
  \end{figure}

  \subsection{Resistencia equivalente}
  \label{sec:reseqparalelo}

  Calculemos la resistencia equivalente al circuito aplicando la fórmula para resistencias en paralelo:
  \begin{equation}
      \label{ec:resparalelo}
      \frac{1}{R_P} = \sum^N_{k=1} \frac{1}{R_k} \qquad \frac{1}{R_P} = \frac{1}{R_1} + \frac{1}{R_2} + \frac{1}{R_3}
  \end{equation}
  Además de esto sabemos que la intensidad total del circuito se distribuirá entre las tres resistencias, mientras que el potencial permanecerá constante (\textbf{Leyes de Kirchhoff}), verificándose que:
  \begin{gather}
      I = \sum^N_{k=1} I_k \qquad I = I_1 + I_2 + I_3 \nonumber \\
      V = V_k \qquad V = V_1 = V_2 = V_3 \label{ec:kripar}
  \end{gather}
  Con esta información podemos calcular la resistencia equivalente al circuito aplicando la fórmula \ref{ec:resparalelo}:
  \begin{gather}
      \frac{1}{R_P} = \frac{1}{175500} + \frac{1}{216000} + \frac{1}{394000} \qquad R_P = \frac{1}{\frac{1}{175500} + \frac{1}{216000} + \frac{1}{394000}} = 7,773 \cdot 10^4 \Omega \nonumber \\
      s(R_P) = \sqrt{\left ( \frac{\partial R_P}{\partial R_1} \right )^2 s^2(R_1) + \left ( \frac{\partial R_P}{\partial R_2} \right )^2 s^2(R_2) + \left ( \frac{\partial R_P}{\partial R_3} \right )^2 s^2(R_3)} \nonumber \\
      \frac{\partial R_P}{\partial R_1} = \left ( \frac{R_2^2 R_3^2}{(R_1(R_2+R_3) + R_2R_3)^2} \right )^2
      s(R_P) = \pm1,4\cdot10^2 \Omega \nonumber \\
      R_P = 7,773 \cdot 10^4 \pm1,4\cdot10^2 \Omega \nonumber
  \end{gather}

  Si tomamos el valor de la resistencia total del circuito de manera experimental obenemos un valor de $R_E = 7,8 \cdot 10^4 \pm 10^3$. En este caso si que entra dentro de nuestro intervalo de confianza, por lo que consideramos que el cálculo teórico es correcto.

  \subsection{Medición experimental}

  Utilizando el procedimiento descrito con anterioridad, realizamos una serie de mediciones en el circuito \ref{circuito:paralelo}. Variaremos el voltaje para obtener diferentes medidas de los voltajes (\textit{V}) e intensidades (\textit{I}). El resultado se expone en la siguiente tabla (Por cuestiones de presentación se omitirá $\pm s(I)$ en las columnas correspondientes, ya que quedaría redundante y sólo dificultaría la lectura sin aportar información. $s(I) = s(I_1) = s(I_2) = s(I_3) = 1 \cdot 10^-7 A$):

  \begin{table}[H]
  \centering
  \csvreader[
    tabular=|c|c|c|c|c|c|,
    table head=\hline Medida & $V~(V) \pm s(V)$ & $I_1~(V)$ & $I_2~(V)$ & $I_3~(V)$ & $I~(A)$ \\ \hline,
    late after last line=\\\hline,
    separator=semicolon
    ]{CC7.csv}
    {v=\v, i1=\ia, i2=\ib, i3=\ic, i=\int, sv=\sv}
    {\thecsvrow & \v \hspace{4pt}$\pm$ \sv & \ia & \ib & \ic & \int }
  \caption{Potenciales e intensidades del circuito en paralelo}
  \end{table}

  \subsection{Representación gráfica de V frente a I}

  Cargamos estos datos en nuestro programa de \code{python} y representamos la gráfica correspondiente.

  \begin{figure}[H]
    %\centering
    \hspace{2.5em} %% Creator: Matplotlib, PGF backend
%%
%% To include the figure in your LaTeX document, write
%%   \input{<filename>.pgf}
%%
%% Make sure the required packages are loaded in your preamble
%%   \usepackage{pgf}
%%
%% Figures using additional raster images can only be included by \input if
%% they are in the same directory as the main LaTeX file. For loading figures
%% from other directories you can use the `import` package
%%   \usepackage{import}
%% and then include the figures with
%%   \import{<path to file>}{<filename>.pgf}
%%
%% Matplotlib used the following preamble
%%
\begingroup%
\makeatletter%
\begin{pgfpicture}%
\pgfpathrectangle{\pgfpointorigin}{\pgfqpoint{4.747069in}{3.310123in}}%
\pgfusepath{use as bounding box, clip}%
\begin{pgfscope}%
\pgfsetbuttcap%
\pgfsetmiterjoin%
\definecolor{currentfill}{rgb}{1.000000,1.000000,1.000000}%
\pgfsetfillcolor{currentfill}%
\pgfsetlinewidth{0.000000pt}%
\definecolor{currentstroke}{rgb}{1.000000,1.000000,1.000000}%
\pgfsetstrokecolor{currentstroke}%
\pgfsetdash{}{0pt}%
\pgfpathmoveto{\pgfqpoint{0.000000in}{0.000000in}}%
\pgfpathlineto{\pgfqpoint{4.747069in}{0.000000in}}%
\pgfpathlineto{\pgfqpoint{4.747069in}{3.310123in}}%
\pgfpathlineto{\pgfqpoint{0.000000in}{3.310123in}}%
\pgfpathclose%
\pgfusepath{fill}%
\end{pgfscope}%
\begin{pgfscope}%
\pgfsetbuttcap%
\pgfsetmiterjoin%
\definecolor{currentfill}{rgb}{1.000000,1.000000,1.000000}%
\pgfsetfillcolor{currentfill}%
\pgfsetlinewidth{0.000000pt}%
\definecolor{currentstroke}{rgb}{0.000000,0.000000,0.000000}%
\pgfsetstrokecolor{currentstroke}%
\pgfsetstrokeopacity{0.000000}%
\pgfsetdash{}{0pt}%
\pgfpathmoveto{\pgfqpoint{0.772069in}{0.515123in}}%
\pgfpathlineto{\pgfqpoint{4.647069in}{0.515123in}}%
\pgfpathlineto{\pgfqpoint{4.647069in}{3.210123in}}%
\pgfpathlineto{\pgfqpoint{0.772069in}{3.210123in}}%
\pgfpathclose%
\pgfusepath{fill}%
\end{pgfscope}%
\begin{pgfscope}%
\pgfsetbuttcap%
\pgfsetroundjoin%
\definecolor{currentfill}{rgb}{0.000000,0.000000,0.000000}%
\pgfsetfillcolor{currentfill}%
\pgfsetlinewidth{0.803000pt}%
\definecolor{currentstroke}{rgb}{0.000000,0.000000,0.000000}%
\pgfsetstrokecolor{currentstroke}%
\pgfsetdash{}{0pt}%
\pgfsys@defobject{currentmarker}{\pgfqpoint{0.000000in}{-0.048611in}}{\pgfqpoint{0.000000in}{0.000000in}}{%
\pgfpathmoveto{\pgfqpoint{0.000000in}{0.000000in}}%
\pgfpathlineto{\pgfqpoint{0.000000in}{-0.048611in}}%
\pgfusepath{stroke,fill}%
}%
\begin{pgfscope}%
\pgfsys@transformshift{1.117634in}{0.515123in}%
\pgfsys@useobject{currentmarker}{}%
\end{pgfscope}%
\end{pgfscope}%
\begin{pgfscope}%
\definecolor{textcolor}{rgb}{0.000000,0.000000,0.000000}%
\pgfsetstrokecolor{textcolor}%
\pgfsetfillcolor{textcolor}%
\pgftext[x=1.117634in,y=0.417901in,,top]{\color{textcolor}\rmfamily\fontsize{10.000000}{12.000000}\selectfont \(\displaystyle 20\)}%
\end{pgfscope}%
\begin{pgfscope}%
\pgfsetbuttcap%
\pgfsetroundjoin%
\definecolor{currentfill}{rgb}{0.000000,0.000000,0.000000}%
\pgfsetfillcolor{currentfill}%
\pgfsetlinewidth{0.803000pt}%
\definecolor{currentstroke}{rgb}{0.000000,0.000000,0.000000}%
\pgfsetstrokecolor{currentstroke}%
\pgfsetdash{}{0pt}%
\pgfsys@defobject{currentmarker}{\pgfqpoint{0.000000in}{-0.048611in}}{\pgfqpoint{0.000000in}{0.000000in}}{%
\pgfpathmoveto{\pgfqpoint{0.000000in}{0.000000in}}%
\pgfpathlineto{\pgfqpoint{0.000000in}{-0.048611in}}%
\pgfusepath{stroke,fill}%
}%
\begin{pgfscope}%
\pgfsys@transformshift{1.732875in}{0.515123in}%
\pgfsys@useobject{currentmarker}{}%
\end{pgfscope}%
\end{pgfscope}%
\begin{pgfscope}%
\definecolor{textcolor}{rgb}{0.000000,0.000000,0.000000}%
\pgfsetstrokecolor{textcolor}%
\pgfsetfillcolor{textcolor}%
\pgftext[x=1.732875in,y=0.417901in,,top]{\color{textcolor}\rmfamily\fontsize{10.000000}{12.000000}\selectfont \(\displaystyle 40\)}%
\end{pgfscope}%
\begin{pgfscope}%
\pgfsetbuttcap%
\pgfsetroundjoin%
\definecolor{currentfill}{rgb}{0.000000,0.000000,0.000000}%
\pgfsetfillcolor{currentfill}%
\pgfsetlinewidth{0.803000pt}%
\definecolor{currentstroke}{rgb}{0.000000,0.000000,0.000000}%
\pgfsetstrokecolor{currentstroke}%
\pgfsetdash{}{0pt}%
\pgfsys@defobject{currentmarker}{\pgfqpoint{0.000000in}{-0.048611in}}{\pgfqpoint{0.000000in}{0.000000in}}{%
\pgfpathmoveto{\pgfqpoint{0.000000in}{0.000000in}}%
\pgfpathlineto{\pgfqpoint{0.000000in}{-0.048611in}}%
\pgfusepath{stroke,fill}%
}%
\begin{pgfscope}%
\pgfsys@transformshift{2.348115in}{0.515123in}%
\pgfsys@useobject{currentmarker}{}%
\end{pgfscope}%
\end{pgfscope}%
\begin{pgfscope}%
\definecolor{textcolor}{rgb}{0.000000,0.000000,0.000000}%
\pgfsetstrokecolor{textcolor}%
\pgfsetfillcolor{textcolor}%
\pgftext[x=2.348115in,y=0.417901in,,top]{\color{textcolor}\rmfamily\fontsize{10.000000}{12.000000}\selectfont \(\displaystyle 60\)}%
\end{pgfscope}%
\begin{pgfscope}%
\pgfsetbuttcap%
\pgfsetroundjoin%
\definecolor{currentfill}{rgb}{0.000000,0.000000,0.000000}%
\pgfsetfillcolor{currentfill}%
\pgfsetlinewidth{0.803000pt}%
\definecolor{currentstroke}{rgb}{0.000000,0.000000,0.000000}%
\pgfsetstrokecolor{currentstroke}%
\pgfsetdash{}{0pt}%
\pgfsys@defobject{currentmarker}{\pgfqpoint{0.000000in}{-0.048611in}}{\pgfqpoint{0.000000in}{0.000000in}}{%
\pgfpathmoveto{\pgfqpoint{0.000000in}{0.000000in}}%
\pgfpathlineto{\pgfqpoint{0.000000in}{-0.048611in}}%
\pgfusepath{stroke,fill}%
}%
\begin{pgfscope}%
\pgfsys@transformshift{2.963355in}{0.515123in}%
\pgfsys@useobject{currentmarker}{}%
\end{pgfscope}%
\end{pgfscope}%
\begin{pgfscope}%
\definecolor{textcolor}{rgb}{0.000000,0.000000,0.000000}%
\pgfsetstrokecolor{textcolor}%
\pgfsetfillcolor{textcolor}%
\pgftext[x=2.963355in,y=0.417901in,,top]{\color{textcolor}\rmfamily\fontsize{10.000000}{12.000000}\selectfont \(\displaystyle 80\)}%
\end{pgfscope}%
\begin{pgfscope}%
\pgfsetbuttcap%
\pgfsetroundjoin%
\definecolor{currentfill}{rgb}{0.000000,0.000000,0.000000}%
\pgfsetfillcolor{currentfill}%
\pgfsetlinewidth{0.803000pt}%
\definecolor{currentstroke}{rgb}{0.000000,0.000000,0.000000}%
\pgfsetstrokecolor{currentstroke}%
\pgfsetdash{}{0pt}%
\pgfsys@defobject{currentmarker}{\pgfqpoint{0.000000in}{-0.048611in}}{\pgfqpoint{0.000000in}{0.000000in}}{%
\pgfpathmoveto{\pgfqpoint{0.000000in}{0.000000in}}%
\pgfpathlineto{\pgfqpoint{0.000000in}{-0.048611in}}%
\pgfusepath{stroke,fill}%
}%
\begin{pgfscope}%
\pgfsys@transformshift{3.578596in}{0.515123in}%
\pgfsys@useobject{currentmarker}{}%
\end{pgfscope}%
\end{pgfscope}%
\begin{pgfscope}%
\definecolor{textcolor}{rgb}{0.000000,0.000000,0.000000}%
\pgfsetstrokecolor{textcolor}%
\pgfsetfillcolor{textcolor}%
\pgftext[x=3.578596in,y=0.417901in,,top]{\color{textcolor}\rmfamily\fontsize{10.000000}{12.000000}\selectfont \(\displaystyle 100\)}%
\end{pgfscope}%
\begin{pgfscope}%
\pgfsetbuttcap%
\pgfsetroundjoin%
\definecolor{currentfill}{rgb}{0.000000,0.000000,0.000000}%
\pgfsetfillcolor{currentfill}%
\pgfsetlinewidth{0.803000pt}%
\definecolor{currentstroke}{rgb}{0.000000,0.000000,0.000000}%
\pgfsetstrokecolor{currentstroke}%
\pgfsetdash{}{0pt}%
\pgfsys@defobject{currentmarker}{\pgfqpoint{0.000000in}{-0.048611in}}{\pgfqpoint{0.000000in}{0.000000in}}{%
\pgfpathmoveto{\pgfqpoint{0.000000in}{0.000000in}}%
\pgfpathlineto{\pgfqpoint{0.000000in}{-0.048611in}}%
\pgfusepath{stroke,fill}%
}%
\begin{pgfscope}%
\pgfsys@transformshift{4.193836in}{0.515123in}%
\pgfsys@useobject{currentmarker}{}%
\end{pgfscope}%
\end{pgfscope}%
\begin{pgfscope}%
\definecolor{textcolor}{rgb}{0.000000,0.000000,0.000000}%
\pgfsetstrokecolor{textcolor}%
\pgfsetfillcolor{textcolor}%
\pgftext[x=4.193836in,y=0.417901in,,top]{\color{textcolor}\rmfamily\fontsize{10.000000}{12.000000}\selectfont \(\displaystyle 120\)}%
\end{pgfscope}%
\begin{pgfscope}%
\definecolor{textcolor}{rgb}{0.000000,0.000000,0.000000}%
\pgfsetstrokecolor{textcolor}%
\pgfsetfillcolor{textcolor}%
\pgftext[x=2.709569in,y=0.238889in,,top]{\color{textcolor}\rmfamily\fontsize{10.000000}{12.000000}\selectfont I(\(\displaystyle \mu\)A)}%
\end{pgfscope}%
\begin{pgfscope}%
\pgfsetbuttcap%
\pgfsetroundjoin%
\definecolor{currentfill}{rgb}{0.000000,0.000000,0.000000}%
\pgfsetfillcolor{currentfill}%
\pgfsetlinewidth{0.803000pt}%
\definecolor{currentstroke}{rgb}{0.000000,0.000000,0.000000}%
\pgfsetstrokecolor{currentstroke}%
\pgfsetdash{}{0pt}%
\pgfsys@defobject{currentmarker}{\pgfqpoint{-0.048611in}{0.000000in}}{\pgfqpoint{0.000000in}{0.000000in}}{%
\pgfpathmoveto{\pgfqpoint{0.000000in}{0.000000in}}%
\pgfpathlineto{\pgfqpoint{-0.048611in}{0.000000in}}%
\pgfusepath{stroke,fill}%
}%
\begin{pgfscope}%
\pgfsys@transformshift{0.772069in}{0.876670in}%
\pgfsys@useobject{currentmarker}{}%
\end{pgfscope}%
\end{pgfscope}%
\begin{pgfscope}%
\definecolor{textcolor}{rgb}{0.000000,0.000000,0.000000}%
\pgfsetstrokecolor{textcolor}%
\pgfsetfillcolor{textcolor}%
\pgftext[x=0.605402in,y=0.828445in,left,base]{\color{textcolor}\rmfamily\fontsize{10.000000}{12.000000}\selectfont \(\displaystyle 2\)}%
\end{pgfscope}%
\begin{pgfscope}%
\pgfsetbuttcap%
\pgfsetroundjoin%
\definecolor{currentfill}{rgb}{0.000000,0.000000,0.000000}%
\pgfsetfillcolor{currentfill}%
\pgfsetlinewidth{0.803000pt}%
\definecolor{currentstroke}{rgb}{0.000000,0.000000,0.000000}%
\pgfsetstrokecolor{currentstroke}%
\pgfsetdash{}{0pt}%
\pgfsys@defobject{currentmarker}{\pgfqpoint{-0.048611in}{0.000000in}}{\pgfqpoint{0.000000in}{0.000000in}}{%
\pgfpathmoveto{\pgfqpoint{0.000000in}{0.000000in}}%
\pgfpathlineto{\pgfqpoint{-0.048611in}{0.000000in}}%
\pgfusepath{stroke,fill}%
}%
\begin{pgfscope}%
\pgfsys@transformshift{0.772069in}{1.420494in}%
\pgfsys@useobject{currentmarker}{}%
\end{pgfscope}%
\end{pgfscope}%
\begin{pgfscope}%
\definecolor{textcolor}{rgb}{0.000000,0.000000,0.000000}%
\pgfsetstrokecolor{textcolor}%
\pgfsetfillcolor{textcolor}%
\pgftext[x=0.605402in,y=1.372269in,left,base]{\color{textcolor}\rmfamily\fontsize{10.000000}{12.000000}\selectfont \(\displaystyle 4\)}%
\end{pgfscope}%
\begin{pgfscope}%
\pgfsetbuttcap%
\pgfsetroundjoin%
\definecolor{currentfill}{rgb}{0.000000,0.000000,0.000000}%
\pgfsetfillcolor{currentfill}%
\pgfsetlinewidth{0.803000pt}%
\definecolor{currentstroke}{rgb}{0.000000,0.000000,0.000000}%
\pgfsetstrokecolor{currentstroke}%
\pgfsetdash{}{0pt}%
\pgfsys@defobject{currentmarker}{\pgfqpoint{-0.048611in}{0.000000in}}{\pgfqpoint{0.000000in}{0.000000in}}{%
\pgfpathmoveto{\pgfqpoint{0.000000in}{0.000000in}}%
\pgfpathlineto{\pgfqpoint{-0.048611in}{0.000000in}}%
\pgfusepath{stroke,fill}%
}%
\begin{pgfscope}%
\pgfsys@transformshift{0.772069in}{1.964318in}%
\pgfsys@useobject{currentmarker}{}%
\end{pgfscope}%
\end{pgfscope}%
\begin{pgfscope}%
\definecolor{textcolor}{rgb}{0.000000,0.000000,0.000000}%
\pgfsetstrokecolor{textcolor}%
\pgfsetfillcolor{textcolor}%
\pgftext[x=0.605402in,y=1.916093in,left,base]{\color{textcolor}\rmfamily\fontsize{10.000000}{12.000000}\selectfont \(\displaystyle 6\)}%
\end{pgfscope}%
\begin{pgfscope}%
\pgfsetbuttcap%
\pgfsetroundjoin%
\definecolor{currentfill}{rgb}{0.000000,0.000000,0.000000}%
\pgfsetfillcolor{currentfill}%
\pgfsetlinewidth{0.803000pt}%
\definecolor{currentstroke}{rgb}{0.000000,0.000000,0.000000}%
\pgfsetstrokecolor{currentstroke}%
\pgfsetdash{}{0pt}%
\pgfsys@defobject{currentmarker}{\pgfqpoint{-0.048611in}{0.000000in}}{\pgfqpoint{0.000000in}{0.000000in}}{%
\pgfpathmoveto{\pgfqpoint{0.000000in}{0.000000in}}%
\pgfpathlineto{\pgfqpoint{-0.048611in}{0.000000in}}%
\pgfusepath{stroke,fill}%
}%
\begin{pgfscope}%
\pgfsys@transformshift{0.772069in}{2.508143in}%
\pgfsys@useobject{currentmarker}{}%
\end{pgfscope}%
\end{pgfscope}%
\begin{pgfscope}%
\definecolor{textcolor}{rgb}{0.000000,0.000000,0.000000}%
\pgfsetstrokecolor{textcolor}%
\pgfsetfillcolor{textcolor}%
\pgftext[x=0.605402in,y=2.459917in,left,base]{\color{textcolor}\rmfamily\fontsize{10.000000}{12.000000}\selectfont \(\displaystyle 8\)}%
\end{pgfscope}%
\begin{pgfscope}%
\pgfsetbuttcap%
\pgfsetroundjoin%
\definecolor{currentfill}{rgb}{0.000000,0.000000,0.000000}%
\pgfsetfillcolor{currentfill}%
\pgfsetlinewidth{0.803000pt}%
\definecolor{currentstroke}{rgb}{0.000000,0.000000,0.000000}%
\pgfsetstrokecolor{currentstroke}%
\pgfsetdash{}{0pt}%
\pgfsys@defobject{currentmarker}{\pgfqpoint{-0.048611in}{0.000000in}}{\pgfqpoint{0.000000in}{0.000000in}}{%
\pgfpathmoveto{\pgfqpoint{0.000000in}{0.000000in}}%
\pgfpathlineto{\pgfqpoint{-0.048611in}{0.000000in}}%
\pgfusepath{stroke,fill}%
}%
\begin{pgfscope}%
\pgfsys@transformshift{0.772069in}{3.051967in}%
\pgfsys@useobject{currentmarker}{}%
\end{pgfscope}%
\end{pgfscope}%
\begin{pgfscope}%
\definecolor{textcolor}{rgb}{0.000000,0.000000,0.000000}%
\pgfsetstrokecolor{textcolor}%
\pgfsetfillcolor{textcolor}%
\pgftext[x=0.535957in,y=3.003742in,left,base]{\color{textcolor}\rmfamily\fontsize{10.000000}{12.000000}\selectfont \(\displaystyle 10\)}%
\end{pgfscope}%
\begin{pgfscope}%
\definecolor{textcolor}{rgb}{0.000000,0.000000,0.000000}%
\pgfsetstrokecolor{textcolor}%
\pgfsetfillcolor{textcolor}%
\pgftext[x=0.258179in,y=1.862623in,,bottom]{\color{textcolor}\rmfamily\fontsize{10.000000}{12.000000}\selectfont V(V)}%
\end{pgfscope}%
\begin{pgfscope}%
\pgfpathrectangle{\pgfqpoint{0.772069in}{0.515123in}}{\pgfqpoint{3.875000in}{2.695000in}}%
\pgfusepath{clip}%
\pgfsetbuttcap%
\pgfsetroundjoin%
\definecolor{currentfill}{rgb}{0.121569,0.466667,0.705882}%
\pgfsetfillcolor{currentfill}%
\pgfsetlinewidth{1.003750pt}%
\definecolor{currentstroke}{rgb}{0.121569,0.466667,0.705882}%
\pgfsetstrokecolor{currentstroke}%
\pgfsetdash{}{0pt}%
\pgfpathmoveto{\pgfqpoint{0.948443in}{0.598984in}}%
\pgfpathcurveto{\pgfqpoint{0.959493in}{0.598984in}}{\pgfqpoint{0.970092in}{0.603374in}}{\pgfqpoint{0.977906in}{0.611187in}}%
\pgfpathcurveto{\pgfqpoint{0.985720in}{0.619001in}}{\pgfqpoint{0.990110in}{0.629600in}}{\pgfqpoint{0.990110in}{0.640650in}}%
\pgfpathcurveto{\pgfqpoint{0.990110in}{0.651700in}}{\pgfqpoint{0.985720in}{0.662299in}}{\pgfqpoint{0.977906in}{0.670113in}}%
\pgfpathcurveto{\pgfqpoint{0.970092in}{0.677927in}}{\pgfqpoint{0.959493in}{0.682317in}}{\pgfqpoint{0.948443in}{0.682317in}}%
\pgfpathcurveto{\pgfqpoint{0.937393in}{0.682317in}}{\pgfqpoint{0.926794in}{0.677927in}}{\pgfqpoint{0.918980in}{0.670113in}}%
\pgfpathcurveto{\pgfqpoint{0.911167in}{0.662299in}}{\pgfqpoint{0.906777in}{0.651700in}}{\pgfqpoint{0.906777in}{0.640650in}}%
\pgfpathcurveto{\pgfqpoint{0.906777in}{0.629600in}}{\pgfqpoint{0.911167in}{0.619001in}}{\pgfqpoint{0.918980in}{0.611187in}}%
\pgfpathcurveto{\pgfqpoint{0.926794in}{0.603374in}}{\pgfqpoint{0.937393in}{0.598984in}}{\pgfqpoint{0.948443in}{0.598984in}}%
\pgfpathclose%
\pgfusepath{stroke,fill}%
\end{pgfscope}%
\begin{pgfscope}%
\pgfpathrectangle{\pgfqpoint{0.772069in}{0.515123in}}{\pgfqpoint{3.875000in}{2.695000in}}%
\pgfusepath{clip}%
\pgfsetbuttcap%
\pgfsetroundjoin%
\definecolor{currentfill}{rgb}{0.121569,0.466667,0.705882}%
\pgfsetfillcolor{currentfill}%
\pgfsetlinewidth{1.003750pt}%
\definecolor{currentstroke}{rgb}{0.121569,0.466667,0.705882}%
\pgfsetstrokecolor{currentstroke}%
\pgfsetdash{}{0pt}%
\pgfpathmoveto{\pgfqpoint{1.314511in}{0.854037in}}%
\pgfpathcurveto{\pgfqpoint{1.325561in}{0.854037in}}{\pgfqpoint{1.336160in}{0.858427in}}{\pgfqpoint{1.343974in}{0.866241in}}%
\pgfpathcurveto{\pgfqpoint{1.351788in}{0.874055in}}{\pgfqpoint{1.356178in}{0.884654in}}{\pgfqpoint{1.356178in}{0.895704in}}%
\pgfpathcurveto{\pgfqpoint{1.356178in}{0.906754in}}{\pgfqpoint{1.351788in}{0.917353in}}{\pgfqpoint{1.343974in}{0.925167in}}%
\pgfpathcurveto{\pgfqpoint{1.336160in}{0.932980in}}{\pgfqpoint{1.325561in}{0.937370in}}{\pgfqpoint{1.314511in}{0.937370in}}%
\pgfpathcurveto{\pgfqpoint{1.303461in}{0.937370in}}{\pgfqpoint{1.292862in}{0.932980in}}{\pgfqpoint{1.285048in}{0.925167in}}%
\pgfpathcurveto{\pgfqpoint{1.277235in}{0.917353in}}{\pgfqpoint{1.272845in}{0.906754in}}{\pgfqpoint{1.272845in}{0.895704in}}%
\pgfpathcurveto{\pgfqpoint{1.272845in}{0.884654in}}{\pgfqpoint{1.277235in}{0.874055in}}{\pgfqpoint{1.285048in}{0.866241in}}%
\pgfpathcurveto{\pgfqpoint{1.292862in}{0.858427in}}{\pgfqpoint{1.303461in}{0.854037in}}{\pgfqpoint{1.314511in}{0.854037in}}%
\pgfpathclose%
\pgfusepath{stroke,fill}%
\end{pgfscope}%
\begin{pgfscope}%
\pgfpathrectangle{\pgfqpoint{0.772069in}{0.515123in}}{\pgfqpoint{3.875000in}{2.695000in}}%
\pgfusepath{clip}%
\pgfsetbuttcap%
\pgfsetroundjoin%
\definecolor{currentfill}{rgb}{0.121569,0.466667,0.705882}%
\pgfsetfillcolor{currentfill}%
\pgfsetlinewidth{1.003750pt}%
\definecolor{currentstroke}{rgb}{0.121569,0.466667,0.705882}%
\pgfsetstrokecolor{currentstroke}%
\pgfsetdash{}{0pt}%
\pgfpathmoveto{\pgfqpoint{1.680579in}{1.109635in}}%
\pgfpathcurveto{\pgfqpoint{1.691629in}{1.109635in}}{\pgfqpoint{1.702228in}{1.114025in}}{\pgfqpoint{1.710042in}{1.121838in}}%
\pgfpathcurveto{\pgfqpoint{1.717856in}{1.129652in}}{\pgfqpoint{1.722246in}{1.140251in}}{\pgfqpoint{1.722246in}{1.151301in}}%
\pgfpathcurveto{\pgfqpoint{1.722246in}{1.162351in}}{\pgfqpoint{1.717856in}{1.172950in}}{\pgfqpoint{1.710042in}{1.180764in}}%
\pgfpathcurveto{\pgfqpoint{1.702228in}{1.188578in}}{\pgfqpoint{1.691629in}{1.192968in}}{\pgfqpoint{1.680579in}{1.192968in}}%
\pgfpathcurveto{\pgfqpoint{1.669529in}{1.192968in}}{\pgfqpoint{1.658930in}{1.188578in}}{\pgfqpoint{1.651116in}{1.180764in}}%
\pgfpathcurveto{\pgfqpoint{1.643303in}{1.172950in}}{\pgfqpoint{1.638913in}{1.162351in}}{\pgfqpoint{1.638913in}{1.151301in}}%
\pgfpathcurveto{\pgfqpoint{1.638913in}{1.140251in}}{\pgfqpoint{1.643303in}{1.129652in}}{\pgfqpoint{1.651116in}{1.121838in}}%
\pgfpathcurveto{\pgfqpoint{1.658930in}{1.114025in}}{\pgfqpoint{1.669529in}{1.109635in}}{\pgfqpoint{1.680579in}{1.109635in}}%
\pgfpathclose%
\pgfusepath{stroke,fill}%
\end{pgfscope}%
\begin{pgfscope}%
\pgfpathrectangle{\pgfqpoint{0.772069in}{0.515123in}}{\pgfqpoint{3.875000in}{2.695000in}}%
\pgfusepath{clip}%
\pgfsetbuttcap%
\pgfsetroundjoin%
\definecolor{currentfill}{rgb}{0.121569,0.466667,0.705882}%
\pgfsetfillcolor{currentfill}%
\pgfsetlinewidth{1.003750pt}%
\definecolor{currentstroke}{rgb}{0.121569,0.466667,0.705882}%
\pgfsetstrokecolor{currentstroke}%
\pgfsetdash{}{0pt}%
\pgfpathmoveto{\pgfqpoint{2.089714in}{1.389704in}}%
\pgfpathcurveto{\pgfqpoint{2.100764in}{1.389704in}}{\pgfqpoint{2.111363in}{1.394094in}}{\pgfqpoint{2.119177in}{1.401908in}}%
\pgfpathcurveto{\pgfqpoint{2.126990in}{1.409722in}}{\pgfqpoint{2.131381in}{1.420321in}}{\pgfqpoint{2.131381in}{1.431371in}}%
\pgfpathcurveto{\pgfqpoint{2.131381in}{1.442421in}}{\pgfqpoint{2.126990in}{1.453020in}}{\pgfqpoint{2.119177in}{1.460833in}}%
\pgfpathcurveto{\pgfqpoint{2.111363in}{1.468647in}}{\pgfqpoint{2.100764in}{1.473037in}}{\pgfqpoint{2.089714in}{1.473037in}}%
\pgfpathcurveto{\pgfqpoint{2.078664in}{1.473037in}}{\pgfqpoint{2.068065in}{1.468647in}}{\pgfqpoint{2.060251in}{1.460833in}}%
\pgfpathcurveto{\pgfqpoint{2.052438in}{1.453020in}}{\pgfqpoint{2.048047in}{1.442421in}}{\pgfqpoint{2.048047in}{1.431371in}}%
\pgfpathcurveto{\pgfqpoint{2.048047in}{1.420321in}}{\pgfqpoint{2.052438in}{1.409722in}}{\pgfqpoint{2.060251in}{1.401908in}}%
\pgfpathcurveto{\pgfqpoint{2.068065in}{1.394094in}}{\pgfqpoint{2.078664in}{1.389704in}}{\pgfqpoint{2.089714in}{1.389704in}}%
\pgfpathclose%
\pgfusepath{stroke,fill}%
\end{pgfscope}%
\begin{pgfscope}%
\pgfpathrectangle{\pgfqpoint{0.772069in}{0.515123in}}{\pgfqpoint{3.875000in}{2.695000in}}%
\pgfusepath{clip}%
\pgfsetbuttcap%
\pgfsetroundjoin%
\definecolor{currentfill}{rgb}{0.121569,0.466667,0.705882}%
\pgfsetfillcolor{currentfill}%
\pgfsetlinewidth{1.003750pt}%
\definecolor{currentstroke}{rgb}{0.121569,0.466667,0.705882}%
\pgfsetstrokecolor{currentstroke}%
\pgfsetdash{}{0pt}%
\pgfpathmoveto{\pgfqpoint{2.495773in}{1.667054in}}%
\pgfpathcurveto{\pgfqpoint{2.506823in}{1.667054in}}{\pgfqpoint{2.517422in}{1.671445in}}{\pgfqpoint{2.525235in}{1.679258in}}%
\pgfpathcurveto{\pgfqpoint{2.533049in}{1.687072in}}{\pgfqpoint{2.537439in}{1.697671in}}{\pgfqpoint{2.537439in}{1.708721in}}%
\pgfpathcurveto{\pgfqpoint{2.537439in}{1.719771in}}{\pgfqpoint{2.533049in}{1.730370in}}{\pgfqpoint{2.525235in}{1.738184in}}%
\pgfpathcurveto{\pgfqpoint{2.517422in}{1.745997in}}{\pgfqpoint{2.506823in}{1.750388in}}{\pgfqpoint{2.495773in}{1.750388in}}%
\pgfpathcurveto{\pgfqpoint{2.484723in}{1.750388in}}{\pgfqpoint{2.474124in}{1.745997in}}{\pgfqpoint{2.466310in}{1.738184in}}%
\pgfpathcurveto{\pgfqpoint{2.458496in}{1.730370in}}{\pgfqpoint{2.454106in}{1.719771in}}{\pgfqpoint{2.454106in}{1.708721in}}%
\pgfpathcurveto{\pgfqpoint{2.454106in}{1.697671in}}{\pgfqpoint{2.458496in}{1.687072in}}{\pgfqpoint{2.466310in}{1.679258in}}%
\pgfpathcurveto{\pgfqpoint{2.474124in}{1.671445in}}{\pgfqpoint{2.484723in}{1.667054in}}{\pgfqpoint{2.495773in}{1.667054in}}%
\pgfpathclose%
\pgfusepath{stroke,fill}%
\end{pgfscope}%
\begin{pgfscope}%
\pgfpathrectangle{\pgfqpoint{0.772069in}{0.515123in}}{\pgfqpoint{3.875000in}{2.695000in}}%
\pgfusepath{clip}%
\pgfsetbuttcap%
\pgfsetroundjoin%
\definecolor{currentfill}{rgb}{0.121569,0.466667,0.705882}%
\pgfsetfillcolor{currentfill}%
\pgfsetlinewidth{1.003750pt}%
\definecolor{currentstroke}{rgb}{0.121569,0.466667,0.705882}%
\pgfsetstrokecolor{currentstroke}%
\pgfsetdash{}{0pt}%
\pgfpathmoveto{\pgfqpoint{2.889526in}{1.941686in}}%
\pgfpathcurveto{\pgfqpoint{2.900577in}{1.941686in}}{\pgfqpoint{2.911176in}{1.946076in}}{\pgfqpoint{2.918989in}{1.953890in}}%
\pgfpathcurveto{\pgfqpoint{2.926803in}{1.961703in}}{\pgfqpoint{2.931193in}{1.972302in}}{\pgfqpoint{2.931193in}{1.983352in}}%
\pgfpathcurveto{\pgfqpoint{2.931193in}{1.994402in}}{\pgfqpoint{2.926803in}{2.005001in}}{\pgfqpoint{2.918989in}{2.012815in}}%
\pgfpathcurveto{\pgfqpoint{2.911176in}{2.020629in}}{\pgfqpoint{2.900577in}{2.025019in}}{\pgfqpoint{2.889526in}{2.025019in}}%
\pgfpathcurveto{\pgfqpoint{2.878476in}{2.025019in}}{\pgfqpoint{2.867877in}{2.020629in}}{\pgfqpoint{2.860064in}{2.012815in}}%
\pgfpathcurveto{\pgfqpoint{2.852250in}{2.005001in}}{\pgfqpoint{2.847860in}{1.994402in}}{\pgfqpoint{2.847860in}{1.983352in}}%
\pgfpathcurveto{\pgfqpoint{2.847860in}{1.972302in}}{\pgfqpoint{2.852250in}{1.961703in}}{\pgfqpoint{2.860064in}{1.953890in}}%
\pgfpathcurveto{\pgfqpoint{2.867877in}{1.946076in}}{\pgfqpoint{2.878476in}{1.941686in}}{\pgfqpoint{2.889526in}{1.941686in}}%
\pgfpathclose%
\pgfusepath{stroke,fill}%
\end{pgfscope}%
\begin{pgfscope}%
\pgfpathrectangle{\pgfqpoint{0.772069in}{0.515123in}}{\pgfqpoint{3.875000in}{2.695000in}}%
\pgfusepath{clip}%
\pgfsetbuttcap%
\pgfsetroundjoin%
\definecolor{currentfill}{rgb}{0.121569,0.466667,0.705882}%
\pgfsetfillcolor{currentfill}%
\pgfsetlinewidth{1.003750pt}%
\definecolor{currentstroke}{rgb}{0.121569,0.466667,0.705882}%
\pgfsetstrokecolor{currentstroke}%
\pgfsetdash{}{0pt}%
\pgfpathmoveto{\pgfqpoint{3.292509in}{2.219036in}}%
\pgfpathcurveto{\pgfqpoint{3.303559in}{2.219036in}}{\pgfqpoint{3.314158in}{2.223426in}}{\pgfqpoint{3.321972in}{2.231240in}}%
\pgfpathcurveto{\pgfqpoint{3.329785in}{2.239054in}}{\pgfqpoint{3.334176in}{2.249653in}}{\pgfqpoint{3.334176in}{2.260703in}}%
\pgfpathcurveto{\pgfqpoint{3.334176in}{2.271753in}}{\pgfqpoint{3.329785in}{2.282352in}}{\pgfqpoint{3.321972in}{2.290165in}}%
\pgfpathcurveto{\pgfqpoint{3.314158in}{2.297979in}}{\pgfqpoint{3.303559in}{2.302369in}}{\pgfqpoint{3.292509in}{2.302369in}}%
\pgfpathcurveto{\pgfqpoint{3.281459in}{2.302369in}}{\pgfqpoint{3.270860in}{2.297979in}}{\pgfqpoint{3.263046in}{2.290165in}}%
\pgfpathcurveto{\pgfqpoint{3.255233in}{2.282352in}}{\pgfqpoint{3.250842in}{2.271753in}}{\pgfqpoint{3.250842in}{2.260703in}}%
\pgfpathcurveto{\pgfqpoint{3.250842in}{2.249653in}}{\pgfqpoint{3.255233in}{2.239054in}}{\pgfqpoint{3.263046in}{2.231240in}}%
\pgfpathcurveto{\pgfqpoint{3.270860in}{2.223426in}}{\pgfqpoint{3.281459in}{2.219036in}}{\pgfqpoint{3.292509in}{2.219036in}}%
\pgfpathclose%
\pgfusepath{stroke,fill}%
\end{pgfscope}%
\begin{pgfscope}%
\pgfpathrectangle{\pgfqpoint{0.772069in}{0.515123in}}{\pgfqpoint{3.875000in}{2.695000in}}%
\pgfusepath{clip}%
\pgfsetbuttcap%
\pgfsetroundjoin%
\definecolor{currentfill}{rgb}{0.121569,0.466667,0.705882}%
\pgfsetfillcolor{currentfill}%
\pgfsetlinewidth{1.003750pt}%
\definecolor{currentstroke}{rgb}{0.121569,0.466667,0.705882}%
\pgfsetstrokecolor{currentstroke}%
\pgfsetdash{}{0pt}%
\pgfpathmoveto{\pgfqpoint{3.670882in}{2.488229in}}%
\pgfpathcurveto{\pgfqpoint{3.681932in}{2.488229in}}{\pgfqpoint{3.692531in}{2.492619in}}{\pgfqpoint{3.700344in}{2.500433in}}%
\pgfpathcurveto{\pgfqpoint{3.708158in}{2.508247in}}{\pgfqpoint{3.712548in}{2.518846in}}{\pgfqpoint{3.712548in}{2.529896in}}%
\pgfpathcurveto{\pgfqpoint{3.712548in}{2.540946in}}{\pgfqpoint{3.708158in}{2.551545in}}{\pgfqpoint{3.700344in}{2.559359in}}%
\pgfpathcurveto{\pgfqpoint{3.692531in}{2.567172in}}{\pgfqpoint{3.681932in}{2.571562in}}{\pgfqpoint{3.670882in}{2.571562in}}%
\pgfpathcurveto{\pgfqpoint{3.659832in}{2.571562in}}{\pgfqpoint{3.649233in}{2.567172in}}{\pgfqpoint{3.641419in}{2.559359in}}%
\pgfpathcurveto{\pgfqpoint{3.633605in}{2.551545in}}{\pgfqpoint{3.629215in}{2.540946in}}{\pgfqpoint{3.629215in}{2.529896in}}%
\pgfpathcurveto{\pgfqpoint{3.629215in}{2.518846in}}{\pgfqpoint{3.633605in}{2.508247in}}{\pgfqpoint{3.641419in}{2.500433in}}%
\pgfpathcurveto{\pgfqpoint{3.649233in}{2.492619in}}{\pgfqpoint{3.659832in}{2.488229in}}{\pgfqpoint{3.670882in}{2.488229in}}%
\pgfpathclose%
\pgfusepath{stroke,fill}%
\end{pgfscope}%
\begin{pgfscope}%
\pgfpathrectangle{\pgfqpoint{0.772069in}{0.515123in}}{\pgfqpoint{3.875000in}{2.695000in}}%
\pgfusepath{clip}%
\pgfsetbuttcap%
\pgfsetroundjoin%
\definecolor{currentfill}{rgb}{0.121569,0.466667,0.705882}%
\pgfsetfillcolor{currentfill}%
\pgfsetlinewidth{1.003750pt}%
\definecolor{currentstroke}{rgb}{0.121569,0.466667,0.705882}%
\pgfsetstrokecolor{currentstroke}%
\pgfsetdash{}{0pt}%
\pgfpathmoveto{\pgfqpoint{4.070788in}{2.757422in}}%
\pgfpathcurveto{\pgfqpoint{4.081838in}{2.757422in}}{\pgfqpoint{4.092437in}{2.761812in}}{\pgfqpoint{4.100251in}{2.769626in}}%
\pgfpathcurveto{\pgfqpoint{4.108064in}{2.777440in}}{\pgfqpoint{4.112455in}{2.788039in}}{\pgfqpoint{4.112455in}{2.799089in}}%
\pgfpathcurveto{\pgfqpoint{4.112455in}{2.810139in}}{\pgfqpoint{4.108064in}{2.820738in}}{\pgfqpoint{4.100251in}{2.828552in}}%
\pgfpathcurveto{\pgfqpoint{4.092437in}{2.836365in}}{\pgfqpoint{4.081838in}{2.840755in}}{\pgfqpoint{4.070788in}{2.840755in}}%
\pgfpathcurveto{\pgfqpoint{4.059738in}{2.840755in}}{\pgfqpoint{4.049139in}{2.836365in}}{\pgfqpoint{4.041325in}{2.828552in}}%
\pgfpathcurveto{\pgfqpoint{4.033512in}{2.820738in}}{\pgfqpoint{4.029121in}{2.810139in}}{\pgfqpoint{4.029121in}{2.799089in}}%
\pgfpathcurveto{\pgfqpoint{4.029121in}{2.788039in}}{\pgfqpoint{4.033512in}{2.777440in}}{\pgfqpoint{4.041325in}{2.769626in}}%
\pgfpathcurveto{\pgfqpoint{4.049139in}{2.761812in}}{\pgfqpoint{4.059738in}{2.757422in}}{\pgfqpoint{4.070788in}{2.757422in}}%
\pgfpathclose%
\pgfusepath{stroke,fill}%
\end{pgfscope}%
\begin{pgfscope}%
\pgfpathrectangle{\pgfqpoint{0.772069in}{0.515123in}}{\pgfqpoint{3.875000in}{2.695000in}}%
\pgfusepath{clip}%
\pgfsetbuttcap%
\pgfsetroundjoin%
\definecolor{currentfill}{rgb}{0.121569,0.466667,0.705882}%
\pgfsetfillcolor{currentfill}%
\pgfsetlinewidth{1.003750pt}%
\definecolor{currentstroke}{rgb}{0.121569,0.466667,0.705882}%
\pgfsetstrokecolor{currentstroke}%
\pgfsetdash{}{0pt}%
\pgfpathmoveto{\pgfqpoint{4.470694in}{3.042930in}}%
\pgfpathcurveto{\pgfqpoint{4.481744in}{3.042930in}}{\pgfqpoint{4.492343in}{3.047320in}}{\pgfqpoint{4.500157in}{3.055134in}}%
\pgfpathcurveto{\pgfqpoint{4.507971in}{3.062947in}}{\pgfqpoint{4.512361in}{3.073546in}}{\pgfqpoint{4.512361in}{3.084596in}}%
\pgfpathcurveto{\pgfqpoint{4.512361in}{3.095647in}}{\pgfqpoint{4.507971in}{3.106246in}}{\pgfqpoint{4.500157in}{3.114059in}}%
\pgfpathcurveto{\pgfqpoint{4.492343in}{3.121873in}}{\pgfqpoint{4.481744in}{3.126263in}}{\pgfqpoint{4.470694in}{3.126263in}}%
\pgfpathcurveto{\pgfqpoint{4.459644in}{3.126263in}}{\pgfqpoint{4.449045in}{3.121873in}}{\pgfqpoint{4.441231in}{3.114059in}}%
\pgfpathcurveto{\pgfqpoint{4.433418in}{3.106246in}}{\pgfqpoint{4.429027in}{3.095647in}}{\pgfqpoint{4.429027in}{3.084596in}}%
\pgfpathcurveto{\pgfqpoint{4.429027in}{3.073546in}}{\pgfqpoint{4.433418in}{3.062947in}}{\pgfqpoint{4.441231in}{3.055134in}}%
\pgfpathcurveto{\pgfqpoint{4.449045in}{3.047320in}}{\pgfqpoint{4.459644in}{3.042930in}}{\pgfqpoint{4.470694in}{3.042930in}}%
\pgfpathclose%
\pgfusepath{stroke,fill}%
\end{pgfscope}%
\begin{pgfscope}%
\pgfsetrectcap%
\pgfsetmiterjoin%
\pgfsetlinewidth{0.803000pt}%
\definecolor{currentstroke}{rgb}{0.000000,0.000000,0.000000}%
\pgfsetstrokecolor{currentstroke}%
\pgfsetdash{}{0pt}%
\pgfpathmoveto{\pgfqpoint{0.772069in}{0.515123in}}%
\pgfpathlineto{\pgfqpoint{0.772069in}{3.210123in}}%
\pgfusepath{stroke}%
\end{pgfscope}%
\begin{pgfscope}%
\pgfsetrectcap%
\pgfsetmiterjoin%
\pgfsetlinewidth{0.803000pt}%
\definecolor{currentstroke}{rgb}{0.000000,0.000000,0.000000}%
\pgfsetstrokecolor{currentstroke}%
\pgfsetdash{}{0pt}%
\pgfpathmoveto{\pgfqpoint{4.647069in}{0.515123in}}%
\pgfpathlineto{\pgfqpoint{4.647069in}{3.210123in}}%
\pgfusepath{stroke}%
\end{pgfscope}%
\begin{pgfscope}%
\pgfsetrectcap%
\pgfsetmiterjoin%
\pgfsetlinewidth{0.803000pt}%
\definecolor{currentstroke}{rgb}{0.000000,0.000000,0.000000}%
\pgfsetstrokecolor{currentstroke}%
\pgfsetdash{}{0pt}%
\pgfpathmoveto{\pgfqpoint{0.772069in}{0.515123in}}%
\pgfpathlineto{\pgfqpoint{4.647069in}{0.515123in}}%
\pgfusepath{stroke}%
\end{pgfscope}%
\begin{pgfscope}%
\pgfsetrectcap%
\pgfsetmiterjoin%
\pgfsetlinewidth{0.803000pt}%
\definecolor{currentstroke}{rgb}{0.000000,0.000000,0.000000}%
\pgfsetstrokecolor{currentstroke}%
\pgfsetdash{}{0pt}%
\pgfpathmoveto{\pgfqpoint{0.772069in}{3.210123in}}%
\pgfpathlineto{\pgfqpoint{4.647069in}{3.210123in}}%
\pgfusepath{stroke}%
\end{pgfscope}%
\end{pgfpicture}%
\makeatother%
\endgroup%

    \caption{Voltaje (V) frente a intensidad (I)}
  \end{figure}

  Así mismo, también podemos comparar la diferencia de las intensidades para un potencial fijo en cada resistencia en paralelo, añadiendo a la anterior las gráficas de \textit{V} frente a $I_1$, $I_2$ e $I_3$.

  \begin{figure}[H]
    %\centering
    \hspace{2.5em} %% Creator: Matplotlib, PGF backend
%%
%% To include the figure in your LaTeX document, write
%%   \input{<filename>.pgf}
%%
%% Make sure the required packages are loaded in your preamble
%%   \usepackage{pgf}
%%
%% Figures using additional raster images can only be included by \input if
%% they are in the same directory as the main LaTeX file. For loading figures
%% from other directories you can use the `import` package
%%   \usepackage{import}
%% and then include the figures with
%%   \import{<path to file>}{<filename>.pgf}
%%
%% Matplotlib used the following preamble
%%
\begingroup%
\makeatletter%
\begin{pgfpicture}%
\pgfpathrectangle{\pgfpointorigin}{\pgfqpoint{4.747069in}{3.310123in}}%
\pgfusepath{use as bounding box, clip}%
\begin{pgfscope}%
\pgfsetbuttcap%
\pgfsetmiterjoin%
\definecolor{currentfill}{rgb}{1.000000,1.000000,1.000000}%
\pgfsetfillcolor{currentfill}%
\pgfsetlinewidth{0.000000pt}%
\definecolor{currentstroke}{rgb}{1.000000,1.000000,1.000000}%
\pgfsetstrokecolor{currentstroke}%
\pgfsetdash{}{0pt}%
\pgfpathmoveto{\pgfqpoint{0.000000in}{0.000000in}}%
\pgfpathlineto{\pgfqpoint{4.747069in}{0.000000in}}%
\pgfpathlineto{\pgfqpoint{4.747069in}{3.310123in}}%
\pgfpathlineto{\pgfqpoint{0.000000in}{3.310123in}}%
\pgfpathclose%
\pgfusepath{fill}%
\end{pgfscope}%
\begin{pgfscope}%
\pgfsetbuttcap%
\pgfsetmiterjoin%
\definecolor{currentfill}{rgb}{1.000000,1.000000,1.000000}%
\pgfsetfillcolor{currentfill}%
\pgfsetlinewidth{0.000000pt}%
\definecolor{currentstroke}{rgb}{0.000000,0.000000,0.000000}%
\pgfsetstrokecolor{currentstroke}%
\pgfsetstrokeopacity{0.000000}%
\pgfsetdash{}{0pt}%
\pgfpathmoveto{\pgfqpoint{0.772069in}{0.515123in}}%
\pgfpathlineto{\pgfqpoint{4.647069in}{0.515123in}}%
\pgfpathlineto{\pgfqpoint{4.647069in}{3.210123in}}%
\pgfpathlineto{\pgfqpoint{0.772069in}{3.210123in}}%
\pgfpathclose%
\pgfusepath{fill}%
\end{pgfscope}%
\begin{pgfscope}%
\pgfsetbuttcap%
\pgfsetroundjoin%
\definecolor{currentfill}{rgb}{0.000000,0.000000,0.000000}%
\pgfsetfillcolor{currentfill}%
\pgfsetlinewidth{0.803000pt}%
\definecolor{currentstroke}{rgb}{0.000000,0.000000,0.000000}%
\pgfsetstrokecolor{currentstroke}%
\pgfsetdash{}{0pt}%
\pgfsys@defobject{currentmarker}{\pgfqpoint{0.000000in}{-0.048611in}}{\pgfqpoint{0.000000in}{0.000000in}}{%
\pgfpathmoveto{\pgfqpoint{0.000000in}{0.000000in}}%
\pgfpathlineto{\pgfqpoint{0.000000in}{-0.048611in}}%
\pgfusepath{stroke,fill}%
}%
\begin{pgfscope}%
\pgfsys@transformshift{0.900160in}{0.515123in}%
\pgfsys@useobject{currentmarker}{}%
\end{pgfscope}%
\end{pgfscope}%
\begin{pgfscope}%
\definecolor{textcolor}{rgb}{0.000000,0.000000,0.000000}%
\pgfsetstrokecolor{textcolor}%
\pgfsetfillcolor{textcolor}%
\pgftext[x=0.900160in,y=0.417901in,,top]{\color{textcolor}\rmfamily\fontsize{10.000000}{12.000000}\selectfont \(\displaystyle 0\)}%
\end{pgfscope}%
\begin{pgfscope}%
\pgfsetbuttcap%
\pgfsetroundjoin%
\definecolor{currentfill}{rgb}{0.000000,0.000000,0.000000}%
\pgfsetfillcolor{currentfill}%
\pgfsetlinewidth{0.803000pt}%
\definecolor{currentstroke}{rgb}{0.000000,0.000000,0.000000}%
\pgfsetstrokecolor{currentstroke}%
\pgfsetdash{}{0pt}%
\pgfsys@defobject{currentmarker}{\pgfqpoint{0.000000in}{-0.048611in}}{\pgfqpoint{0.000000in}{0.000000in}}{%
\pgfpathmoveto{\pgfqpoint{0.000000in}{0.000000in}}%
\pgfpathlineto{\pgfqpoint{0.000000in}{-0.048611in}}%
\pgfusepath{stroke,fill}%
}%
\begin{pgfscope}%
\pgfsys@transformshift{1.453735in}{0.515123in}%
\pgfsys@useobject{currentmarker}{}%
\end{pgfscope}%
\end{pgfscope}%
\begin{pgfscope}%
\definecolor{textcolor}{rgb}{0.000000,0.000000,0.000000}%
\pgfsetstrokecolor{textcolor}%
\pgfsetfillcolor{textcolor}%
\pgftext[x=1.453735in,y=0.417901in,,top]{\color{textcolor}\rmfamily\fontsize{10.000000}{12.000000}\selectfont \(\displaystyle 20\)}%
\end{pgfscope}%
\begin{pgfscope}%
\pgfsetbuttcap%
\pgfsetroundjoin%
\definecolor{currentfill}{rgb}{0.000000,0.000000,0.000000}%
\pgfsetfillcolor{currentfill}%
\pgfsetlinewidth{0.803000pt}%
\definecolor{currentstroke}{rgb}{0.000000,0.000000,0.000000}%
\pgfsetstrokecolor{currentstroke}%
\pgfsetdash{}{0pt}%
\pgfsys@defobject{currentmarker}{\pgfqpoint{0.000000in}{-0.048611in}}{\pgfqpoint{0.000000in}{0.000000in}}{%
\pgfpathmoveto{\pgfqpoint{0.000000in}{0.000000in}}%
\pgfpathlineto{\pgfqpoint{0.000000in}{-0.048611in}}%
\pgfusepath{stroke,fill}%
}%
\begin{pgfscope}%
\pgfsys@transformshift{2.007310in}{0.515123in}%
\pgfsys@useobject{currentmarker}{}%
\end{pgfscope}%
\end{pgfscope}%
\begin{pgfscope}%
\definecolor{textcolor}{rgb}{0.000000,0.000000,0.000000}%
\pgfsetstrokecolor{textcolor}%
\pgfsetfillcolor{textcolor}%
\pgftext[x=2.007310in,y=0.417901in,,top]{\color{textcolor}\rmfamily\fontsize{10.000000}{12.000000}\selectfont \(\displaystyle 40\)}%
\end{pgfscope}%
\begin{pgfscope}%
\pgfsetbuttcap%
\pgfsetroundjoin%
\definecolor{currentfill}{rgb}{0.000000,0.000000,0.000000}%
\pgfsetfillcolor{currentfill}%
\pgfsetlinewidth{0.803000pt}%
\definecolor{currentstroke}{rgb}{0.000000,0.000000,0.000000}%
\pgfsetstrokecolor{currentstroke}%
\pgfsetdash{}{0pt}%
\pgfsys@defobject{currentmarker}{\pgfqpoint{0.000000in}{-0.048611in}}{\pgfqpoint{0.000000in}{0.000000in}}{%
\pgfpathmoveto{\pgfqpoint{0.000000in}{0.000000in}}%
\pgfpathlineto{\pgfqpoint{0.000000in}{-0.048611in}}%
\pgfusepath{stroke,fill}%
}%
\begin{pgfscope}%
\pgfsys@transformshift{2.560884in}{0.515123in}%
\pgfsys@useobject{currentmarker}{}%
\end{pgfscope}%
\end{pgfscope}%
\begin{pgfscope}%
\definecolor{textcolor}{rgb}{0.000000,0.000000,0.000000}%
\pgfsetstrokecolor{textcolor}%
\pgfsetfillcolor{textcolor}%
\pgftext[x=2.560884in,y=0.417901in,,top]{\color{textcolor}\rmfamily\fontsize{10.000000}{12.000000}\selectfont \(\displaystyle 60\)}%
\end{pgfscope}%
\begin{pgfscope}%
\pgfsetbuttcap%
\pgfsetroundjoin%
\definecolor{currentfill}{rgb}{0.000000,0.000000,0.000000}%
\pgfsetfillcolor{currentfill}%
\pgfsetlinewidth{0.803000pt}%
\definecolor{currentstroke}{rgb}{0.000000,0.000000,0.000000}%
\pgfsetstrokecolor{currentstroke}%
\pgfsetdash{}{0pt}%
\pgfsys@defobject{currentmarker}{\pgfqpoint{0.000000in}{-0.048611in}}{\pgfqpoint{0.000000in}{0.000000in}}{%
\pgfpathmoveto{\pgfqpoint{0.000000in}{0.000000in}}%
\pgfpathlineto{\pgfqpoint{0.000000in}{-0.048611in}}%
\pgfusepath{stroke,fill}%
}%
\begin{pgfscope}%
\pgfsys@transformshift{3.114459in}{0.515123in}%
\pgfsys@useobject{currentmarker}{}%
\end{pgfscope}%
\end{pgfscope}%
\begin{pgfscope}%
\definecolor{textcolor}{rgb}{0.000000,0.000000,0.000000}%
\pgfsetstrokecolor{textcolor}%
\pgfsetfillcolor{textcolor}%
\pgftext[x=3.114459in,y=0.417901in,,top]{\color{textcolor}\rmfamily\fontsize{10.000000}{12.000000}\selectfont \(\displaystyle 80\)}%
\end{pgfscope}%
\begin{pgfscope}%
\pgfsetbuttcap%
\pgfsetroundjoin%
\definecolor{currentfill}{rgb}{0.000000,0.000000,0.000000}%
\pgfsetfillcolor{currentfill}%
\pgfsetlinewidth{0.803000pt}%
\definecolor{currentstroke}{rgb}{0.000000,0.000000,0.000000}%
\pgfsetstrokecolor{currentstroke}%
\pgfsetdash{}{0pt}%
\pgfsys@defobject{currentmarker}{\pgfqpoint{0.000000in}{-0.048611in}}{\pgfqpoint{0.000000in}{0.000000in}}{%
\pgfpathmoveto{\pgfqpoint{0.000000in}{0.000000in}}%
\pgfpathlineto{\pgfqpoint{0.000000in}{-0.048611in}}%
\pgfusepath{stroke,fill}%
}%
\begin{pgfscope}%
\pgfsys@transformshift{3.668034in}{0.515123in}%
\pgfsys@useobject{currentmarker}{}%
\end{pgfscope}%
\end{pgfscope}%
\begin{pgfscope}%
\definecolor{textcolor}{rgb}{0.000000,0.000000,0.000000}%
\pgfsetstrokecolor{textcolor}%
\pgfsetfillcolor{textcolor}%
\pgftext[x=3.668034in,y=0.417901in,,top]{\color{textcolor}\rmfamily\fontsize{10.000000}{12.000000}\selectfont \(\displaystyle 100\)}%
\end{pgfscope}%
\begin{pgfscope}%
\pgfsetbuttcap%
\pgfsetroundjoin%
\definecolor{currentfill}{rgb}{0.000000,0.000000,0.000000}%
\pgfsetfillcolor{currentfill}%
\pgfsetlinewidth{0.803000pt}%
\definecolor{currentstroke}{rgb}{0.000000,0.000000,0.000000}%
\pgfsetstrokecolor{currentstroke}%
\pgfsetdash{}{0pt}%
\pgfsys@defobject{currentmarker}{\pgfqpoint{0.000000in}{-0.048611in}}{\pgfqpoint{0.000000in}{0.000000in}}{%
\pgfpathmoveto{\pgfqpoint{0.000000in}{0.000000in}}%
\pgfpathlineto{\pgfqpoint{0.000000in}{-0.048611in}}%
\pgfusepath{stroke,fill}%
}%
\begin{pgfscope}%
\pgfsys@transformshift{4.221609in}{0.515123in}%
\pgfsys@useobject{currentmarker}{}%
\end{pgfscope}%
\end{pgfscope}%
\begin{pgfscope}%
\definecolor{textcolor}{rgb}{0.000000,0.000000,0.000000}%
\pgfsetstrokecolor{textcolor}%
\pgfsetfillcolor{textcolor}%
\pgftext[x=4.221609in,y=0.417901in,,top]{\color{textcolor}\rmfamily\fontsize{10.000000}{12.000000}\selectfont \(\displaystyle 120\)}%
\end{pgfscope}%
\begin{pgfscope}%
\definecolor{textcolor}{rgb}{0.000000,0.000000,0.000000}%
\pgfsetstrokecolor{textcolor}%
\pgfsetfillcolor{textcolor}%
\pgftext[x=2.709569in,y=0.238889in,,top]{\color{textcolor}\rmfamily\fontsize{10.000000}{12.000000}\selectfont I(\(\displaystyle \mu\)A)}%
\end{pgfscope}%
\begin{pgfscope}%
\pgfsetbuttcap%
\pgfsetroundjoin%
\definecolor{currentfill}{rgb}{0.000000,0.000000,0.000000}%
\pgfsetfillcolor{currentfill}%
\pgfsetlinewidth{0.803000pt}%
\definecolor{currentstroke}{rgb}{0.000000,0.000000,0.000000}%
\pgfsetstrokecolor{currentstroke}%
\pgfsetdash{}{0pt}%
\pgfsys@defobject{currentmarker}{\pgfqpoint{-0.048611in}{0.000000in}}{\pgfqpoint{0.000000in}{0.000000in}}{%
\pgfpathmoveto{\pgfqpoint{0.000000in}{0.000000in}}%
\pgfpathlineto{\pgfqpoint{-0.048611in}{0.000000in}}%
\pgfusepath{stroke,fill}%
}%
\begin{pgfscope}%
\pgfsys@transformshift{0.772069in}{0.898433in}%
\pgfsys@useobject{currentmarker}{}%
\end{pgfscope}%
\end{pgfscope}%
\begin{pgfscope}%
\definecolor{textcolor}{rgb}{0.000000,0.000000,0.000000}%
\pgfsetstrokecolor{textcolor}%
\pgfsetfillcolor{textcolor}%
\pgftext[x=0.605402in,y=0.850208in,left,base]{\color{textcolor}\rmfamily\fontsize{10.000000}{12.000000}\selectfont \(\displaystyle 2\)}%
\end{pgfscope}%
\begin{pgfscope}%
\pgfsetbuttcap%
\pgfsetroundjoin%
\definecolor{currentfill}{rgb}{0.000000,0.000000,0.000000}%
\pgfsetfillcolor{currentfill}%
\pgfsetlinewidth{0.803000pt}%
\definecolor{currentstroke}{rgb}{0.000000,0.000000,0.000000}%
\pgfsetstrokecolor{currentstroke}%
\pgfsetdash{}{0pt}%
\pgfsys@defobject{currentmarker}{\pgfqpoint{-0.048611in}{0.000000in}}{\pgfqpoint{0.000000in}{0.000000in}}{%
\pgfpathmoveto{\pgfqpoint{0.000000in}{0.000000in}}%
\pgfpathlineto{\pgfqpoint{-0.048611in}{0.000000in}}%
\pgfusepath{stroke,fill}%
}%
\begin{pgfscope}%
\pgfsys@transformshift{0.772069in}{1.430253in}%
\pgfsys@useobject{currentmarker}{}%
\end{pgfscope}%
\end{pgfscope}%
\begin{pgfscope}%
\definecolor{textcolor}{rgb}{0.000000,0.000000,0.000000}%
\pgfsetstrokecolor{textcolor}%
\pgfsetfillcolor{textcolor}%
\pgftext[x=0.605402in,y=1.382028in,left,base]{\color{textcolor}\rmfamily\fontsize{10.000000}{12.000000}\selectfont \(\displaystyle 4\)}%
\end{pgfscope}%
\begin{pgfscope}%
\pgfsetbuttcap%
\pgfsetroundjoin%
\definecolor{currentfill}{rgb}{0.000000,0.000000,0.000000}%
\pgfsetfillcolor{currentfill}%
\pgfsetlinewidth{0.803000pt}%
\definecolor{currentstroke}{rgb}{0.000000,0.000000,0.000000}%
\pgfsetstrokecolor{currentstroke}%
\pgfsetdash{}{0pt}%
\pgfsys@defobject{currentmarker}{\pgfqpoint{-0.048611in}{0.000000in}}{\pgfqpoint{0.000000in}{0.000000in}}{%
\pgfpathmoveto{\pgfqpoint{0.000000in}{0.000000in}}%
\pgfpathlineto{\pgfqpoint{-0.048611in}{0.000000in}}%
\pgfusepath{stroke,fill}%
}%
\begin{pgfscope}%
\pgfsys@transformshift{0.772069in}{1.962074in}%
\pgfsys@useobject{currentmarker}{}%
\end{pgfscope}%
\end{pgfscope}%
\begin{pgfscope}%
\definecolor{textcolor}{rgb}{0.000000,0.000000,0.000000}%
\pgfsetstrokecolor{textcolor}%
\pgfsetfillcolor{textcolor}%
\pgftext[x=0.605402in,y=1.913848in,left,base]{\color{textcolor}\rmfamily\fontsize{10.000000}{12.000000}\selectfont \(\displaystyle 6\)}%
\end{pgfscope}%
\begin{pgfscope}%
\pgfsetbuttcap%
\pgfsetroundjoin%
\definecolor{currentfill}{rgb}{0.000000,0.000000,0.000000}%
\pgfsetfillcolor{currentfill}%
\pgfsetlinewidth{0.803000pt}%
\definecolor{currentstroke}{rgb}{0.000000,0.000000,0.000000}%
\pgfsetstrokecolor{currentstroke}%
\pgfsetdash{}{0pt}%
\pgfsys@defobject{currentmarker}{\pgfqpoint{-0.048611in}{0.000000in}}{\pgfqpoint{0.000000in}{0.000000in}}{%
\pgfpathmoveto{\pgfqpoint{0.000000in}{0.000000in}}%
\pgfpathlineto{\pgfqpoint{-0.048611in}{0.000000in}}%
\pgfusepath{stroke,fill}%
}%
\begin{pgfscope}%
\pgfsys@transformshift{0.772069in}{2.493894in}%
\pgfsys@useobject{currentmarker}{}%
\end{pgfscope}%
\end{pgfscope}%
\begin{pgfscope}%
\definecolor{textcolor}{rgb}{0.000000,0.000000,0.000000}%
\pgfsetstrokecolor{textcolor}%
\pgfsetfillcolor{textcolor}%
\pgftext[x=0.605402in,y=2.445669in,left,base]{\color{textcolor}\rmfamily\fontsize{10.000000}{12.000000}\selectfont \(\displaystyle 8\)}%
\end{pgfscope}%
\begin{pgfscope}%
\pgfsetbuttcap%
\pgfsetroundjoin%
\definecolor{currentfill}{rgb}{0.000000,0.000000,0.000000}%
\pgfsetfillcolor{currentfill}%
\pgfsetlinewidth{0.803000pt}%
\definecolor{currentstroke}{rgb}{0.000000,0.000000,0.000000}%
\pgfsetstrokecolor{currentstroke}%
\pgfsetdash{}{0pt}%
\pgfsys@defobject{currentmarker}{\pgfqpoint{-0.048611in}{0.000000in}}{\pgfqpoint{0.000000in}{0.000000in}}{%
\pgfpathmoveto{\pgfqpoint{0.000000in}{0.000000in}}%
\pgfpathlineto{\pgfqpoint{-0.048611in}{0.000000in}}%
\pgfusepath{stroke,fill}%
}%
\begin{pgfscope}%
\pgfsys@transformshift{0.772069in}{3.025714in}%
\pgfsys@useobject{currentmarker}{}%
\end{pgfscope}%
\end{pgfscope}%
\begin{pgfscope}%
\definecolor{textcolor}{rgb}{0.000000,0.000000,0.000000}%
\pgfsetstrokecolor{textcolor}%
\pgfsetfillcolor{textcolor}%
\pgftext[x=0.535957in,y=2.977489in,left,base]{\color{textcolor}\rmfamily\fontsize{10.000000}{12.000000}\selectfont \(\displaystyle 10\)}%
\end{pgfscope}%
\begin{pgfscope}%
\definecolor{textcolor}{rgb}{0.000000,0.000000,0.000000}%
\pgfsetstrokecolor{textcolor}%
\pgfsetfillcolor{textcolor}%
\pgftext[x=0.258179in,y=1.862623in,,bottom]{\color{textcolor}\rmfamily\fontsize{10.000000}{12.000000}\selectfont V(V)}%
\end{pgfscope}%
\begin{pgfscope}%
\pgfpathrectangle{\pgfqpoint{0.772069in}{0.515123in}}{\pgfqpoint{3.875000in}{2.695000in}}%
\pgfusepath{clip}%
\pgfsetbuttcap%
\pgfsetroundjoin%
\definecolor{currentfill}{rgb}{0.121569,0.466667,0.705882}%
\pgfsetfillcolor{currentfill}%
\pgfsetlinewidth{1.003750pt}%
\definecolor{currentstroke}{rgb}{0.121569,0.466667,0.705882}%
\pgfsetstrokecolor{currentstroke}%
\pgfsetdash{}{0pt}%
\pgfpathmoveto{\pgfqpoint{1.301501in}{0.625956in}}%
\pgfpathcurveto{\pgfqpoint{1.312552in}{0.625956in}}{\pgfqpoint{1.323151in}{0.630347in}}{\pgfqpoint{1.330964in}{0.638160in}}%
\pgfpathcurveto{\pgfqpoint{1.338778in}{0.645974in}}{\pgfqpoint{1.343168in}{0.656573in}}{\pgfqpoint{1.343168in}{0.667623in}}%
\pgfpathcurveto{\pgfqpoint{1.343168in}{0.678673in}}{\pgfqpoint{1.338778in}{0.689272in}}{\pgfqpoint{1.330964in}{0.697086in}}%
\pgfpathcurveto{\pgfqpoint{1.323151in}{0.704899in}}{\pgfqpoint{1.312552in}{0.709290in}}{\pgfqpoint{1.301501in}{0.709290in}}%
\pgfpathcurveto{\pgfqpoint{1.290451in}{0.709290in}}{\pgfqpoint{1.279852in}{0.704899in}}{\pgfqpoint{1.272039in}{0.697086in}}%
\pgfpathcurveto{\pgfqpoint{1.264225in}{0.689272in}}{\pgfqpoint{1.259835in}{0.678673in}}{\pgfqpoint{1.259835in}{0.667623in}}%
\pgfpathcurveto{\pgfqpoint{1.259835in}{0.656573in}}{\pgfqpoint{1.264225in}{0.645974in}}{\pgfqpoint{1.272039in}{0.638160in}}%
\pgfpathcurveto{\pgfqpoint{1.279852in}{0.630347in}}{\pgfqpoint{1.290451in}{0.625956in}}{\pgfqpoint{1.301501in}{0.625956in}}%
\pgfpathclose%
\pgfusepath{stroke,fill}%
\end{pgfscope}%
\begin{pgfscope}%
\pgfpathrectangle{\pgfqpoint{0.772069in}{0.515123in}}{\pgfqpoint{3.875000in}{2.695000in}}%
\pgfusepath{clip}%
\pgfsetbuttcap%
\pgfsetroundjoin%
\definecolor{currentfill}{rgb}{0.121569,0.466667,0.705882}%
\pgfsetfillcolor{currentfill}%
\pgfsetlinewidth{1.003750pt}%
\definecolor{currentstroke}{rgb}{0.121569,0.466667,0.705882}%
\pgfsetstrokecolor{currentstroke}%
\pgfsetdash{}{0pt}%
\pgfpathmoveto{\pgfqpoint{1.630879in}{0.875380in}}%
\pgfpathcurveto{\pgfqpoint{1.641929in}{0.875380in}}{\pgfqpoint{1.652528in}{0.879770in}}{\pgfqpoint{1.660341in}{0.887584in}}%
\pgfpathcurveto{\pgfqpoint{1.668155in}{0.895398in}}{\pgfqpoint{1.672545in}{0.905997in}}{\pgfqpoint{1.672545in}{0.917047in}}%
\pgfpathcurveto{\pgfqpoint{1.672545in}{0.928097in}}{\pgfqpoint{1.668155in}{0.938696in}}{\pgfqpoint{1.660341in}{0.946509in}}%
\pgfpathcurveto{\pgfqpoint{1.652528in}{0.954323in}}{\pgfqpoint{1.641929in}{0.958713in}}{\pgfqpoint{1.630879in}{0.958713in}}%
\pgfpathcurveto{\pgfqpoint{1.619828in}{0.958713in}}{\pgfqpoint{1.609229in}{0.954323in}}{\pgfqpoint{1.601416in}{0.946509in}}%
\pgfpathcurveto{\pgfqpoint{1.593602in}{0.938696in}}{\pgfqpoint{1.589212in}{0.928097in}}{\pgfqpoint{1.589212in}{0.917047in}}%
\pgfpathcurveto{\pgfqpoint{1.589212in}{0.905997in}}{\pgfqpoint{1.593602in}{0.895398in}}{\pgfqpoint{1.601416in}{0.887584in}}%
\pgfpathcurveto{\pgfqpoint{1.609229in}{0.879770in}}{\pgfqpoint{1.619828in}{0.875380in}}{\pgfqpoint{1.630879in}{0.875380in}}%
\pgfpathclose%
\pgfusepath{stroke,fill}%
\end{pgfscope}%
\begin{pgfscope}%
\pgfpathrectangle{\pgfqpoint{0.772069in}{0.515123in}}{\pgfqpoint{3.875000in}{2.695000in}}%
\pgfusepath{clip}%
\pgfsetbuttcap%
\pgfsetroundjoin%
\definecolor{currentfill}{rgb}{0.121569,0.466667,0.705882}%
\pgfsetfillcolor{currentfill}%
\pgfsetlinewidth{1.003750pt}%
\definecolor{currentstroke}{rgb}{0.121569,0.466667,0.705882}%
\pgfsetstrokecolor{currentstroke}%
\pgfsetdash{}{0pt}%
\pgfpathmoveto{\pgfqpoint{1.960256in}{1.125336in}}%
\pgfpathcurveto{\pgfqpoint{1.971306in}{1.125336in}}{\pgfqpoint{1.981905in}{1.129726in}}{\pgfqpoint{1.989718in}{1.137539in}}%
\pgfpathcurveto{\pgfqpoint{1.997532in}{1.145353in}}{\pgfqpoint{2.001922in}{1.155952in}}{\pgfqpoint{2.001922in}{1.167002in}}%
\pgfpathcurveto{\pgfqpoint{2.001922in}{1.178052in}}{\pgfqpoint{1.997532in}{1.188651in}}{\pgfqpoint{1.989718in}{1.196465in}}%
\pgfpathcurveto{\pgfqpoint{1.981905in}{1.204279in}}{\pgfqpoint{1.971306in}{1.208669in}}{\pgfqpoint{1.960256in}{1.208669in}}%
\pgfpathcurveto{\pgfqpoint{1.949206in}{1.208669in}}{\pgfqpoint{1.938606in}{1.204279in}}{\pgfqpoint{1.930793in}{1.196465in}}%
\pgfpathcurveto{\pgfqpoint{1.922979in}{1.188651in}}{\pgfqpoint{1.918589in}{1.178052in}}{\pgfqpoint{1.918589in}{1.167002in}}%
\pgfpathcurveto{\pgfqpoint{1.918589in}{1.155952in}}{\pgfqpoint{1.922979in}{1.145353in}}{\pgfqpoint{1.930793in}{1.137539in}}%
\pgfpathcurveto{\pgfqpoint{1.938606in}{1.129726in}}{\pgfqpoint{1.949206in}{1.125336in}}{\pgfqpoint{1.960256in}{1.125336in}}%
\pgfpathclose%
\pgfusepath{stroke,fill}%
\end{pgfscope}%
\begin{pgfscope}%
\pgfpathrectangle{\pgfqpoint{0.772069in}{0.515123in}}{\pgfqpoint{3.875000in}{2.695000in}}%
\pgfusepath{clip}%
\pgfsetbuttcap%
\pgfsetroundjoin%
\definecolor{currentfill}{rgb}{0.121569,0.466667,0.705882}%
\pgfsetfillcolor{currentfill}%
\pgfsetlinewidth{1.003750pt}%
\definecolor{currentstroke}{rgb}{0.121569,0.466667,0.705882}%
\pgfsetstrokecolor{currentstroke}%
\pgfsetdash{}{0pt}%
\pgfpathmoveto{\pgfqpoint{2.328383in}{1.399223in}}%
\pgfpathcurveto{\pgfqpoint{2.339433in}{1.399223in}}{\pgfqpoint{2.350032in}{1.403613in}}{\pgfqpoint{2.357846in}{1.411427in}}%
\pgfpathcurveto{\pgfqpoint{2.365659in}{1.419241in}}{\pgfqpoint{2.370050in}{1.429840in}}{\pgfqpoint{2.370050in}{1.440890in}}%
\pgfpathcurveto{\pgfqpoint{2.370050in}{1.451940in}}{\pgfqpoint{2.365659in}{1.462539in}}{\pgfqpoint{2.357846in}{1.470353in}}%
\pgfpathcurveto{\pgfqpoint{2.350032in}{1.478166in}}{\pgfqpoint{2.339433in}{1.482556in}}{\pgfqpoint{2.328383in}{1.482556in}}%
\pgfpathcurveto{\pgfqpoint{2.317333in}{1.482556in}}{\pgfqpoint{2.306734in}{1.478166in}}{\pgfqpoint{2.298920in}{1.470353in}}%
\pgfpathcurveto{\pgfqpoint{2.291107in}{1.462539in}}{\pgfqpoint{2.286716in}{1.451940in}}{\pgfqpoint{2.286716in}{1.440890in}}%
\pgfpathcurveto{\pgfqpoint{2.286716in}{1.429840in}}{\pgfqpoint{2.291107in}{1.419241in}}{\pgfqpoint{2.298920in}{1.411427in}}%
\pgfpathcurveto{\pgfqpoint{2.306734in}{1.403613in}}{\pgfqpoint{2.317333in}{1.399223in}}{\pgfqpoint{2.328383in}{1.399223in}}%
\pgfpathclose%
\pgfusepath{stroke,fill}%
\end{pgfscope}%
\begin{pgfscope}%
\pgfpathrectangle{\pgfqpoint{0.772069in}{0.515123in}}{\pgfqpoint{3.875000in}{2.695000in}}%
\pgfusepath{clip}%
\pgfsetbuttcap%
\pgfsetroundjoin%
\definecolor{currentfill}{rgb}{0.121569,0.466667,0.705882}%
\pgfsetfillcolor{currentfill}%
\pgfsetlinewidth{1.003750pt}%
\definecolor{currentstroke}{rgb}{0.121569,0.466667,0.705882}%
\pgfsetstrokecolor{currentstroke}%
\pgfsetdash{}{0pt}%
\pgfpathmoveto{\pgfqpoint{2.693742in}{1.670451in}}%
\pgfpathcurveto{\pgfqpoint{2.704793in}{1.670451in}}{\pgfqpoint{2.715392in}{1.674842in}}{\pgfqpoint{2.723205in}{1.682655in}}%
\pgfpathcurveto{\pgfqpoint{2.731019in}{1.690469in}}{\pgfqpoint{2.735409in}{1.701068in}}{\pgfqpoint{2.735409in}{1.712118in}}%
\pgfpathcurveto{\pgfqpoint{2.735409in}{1.723168in}}{\pgfqpoint{2.731019in}{1.733767in}}{\pgfqpoint{2.723205in}{1.741581in}}%
\pgfpathcurveto{\pgfqpoint{2.715392in}{1.749395in}}{\pgfqpoint{2.704793in}{1.753785in}}{\pgfqpoint{2.693742in}{1.753785in}}%
\pgfpathcurveto{\pgfqpoint{2.682692in}{1.753785in}}{\pgfqpoint{2.672093in}{1.749395in}}{\pgfqpoint{2.664280in}{1.741581in}}%
\pgfpathcurveto{\pgfqpoint{2.656466in}{1.733767in}}{\pgfqpoint{2.652076in}{1.723168in}}{\pgfqpoint{2.652076in}{1.712118in}}%
\pgfpathcurveto{\pgfqpoint{2.652076in}{1.701068in}}{\pgfqpoint{2.656466in}{1.690469in}}{\pgfqpoint{2.664280in}{1.682655in}}%
\pgfpathcurveto{\pgfqpoint{2.672093in}{1.674842in}}{\pgfqpoint{2.682692in}{1.670451in}}{\pgfqpoint{2.693742in}{1.670451in}}%
\pgfpathclose%
\pgfusepath{stroke,fill}%
\end{pgfscope}%
\begin{pgfscope}%
\pgfpathrectangle{\pgfqpoint{0.772069in}{0.515123in}}{\pgfqpoint{3.875000in}{2.695000in}}%
\pgfusepath{clip}%
\pgfsetbuttcap%
\pgfsetroundjoin%
\definecolor{currentfill}{rgb}{0.121569,0.466667,0.705882}%
\pgfsetfillcolor{currentfill}%
\pgfsetlinewidth{1.003750pt}%
\definecolor{currentstroke}{rgb}{0.121569,0.466667,0.705882}%
\pgfsetstrokecolor{currentstroke}%
\pgfsetdash{}{0pt}%
\pgfpathmoveto{\pgfqpoint{3.048030in}{1.939021in}}%
\pgfpathcurveto{\pgfqpoint{3.059081in}{1.939021in}}{\pgfqpoint{3.069680in}{1.943411in}}{\pgfqpoint{3.077493in}{1.951225in}}%
\pgfpathcurveto{\pgfqpoint{3.085307in}{1.959038in}}{\pgfqpoint{3.089697in}{1.969637in}}{\pgfqpoint{3.089697in}{1.980687in}}%
\pgfpathcurveto{\pgfqpoint{3.089697in}{1.991738in}}{\pgfqpoint{3.085307in}{2.002337in}}{\pgfqpoint{3.077493in}{2.010150in}}%
\pgfpathcurveto{\pgfqpoint{3.069680in}{2.017964in}}{\pgfqpoint{3.059081in}{2.022354in}}{\pgfqpoint{3.048030in}{2.022354in}}%
\pgfpathcurveto{\pgfqpoint{3.036980in}{2.022354in}}{\pgfqpoint{3.026381in}{2.017964in}}{\pgfqpoint{3.018568in}{2.010150in}}%
\pgfpathcurveto{\pgfqpoint{3.010754in}{2.002337in}}{\pgfqpoint{3.006364in}{1.991738in}}{\pgfqpoint{3.006364in}{1.980687in}}%
\pgfpathcurveto{\pgfqpoint{3.006364in}{1.969637in}}{\pgfqpoint{3.010754in}{1.959038in}}{\pgfqpoint{3.018568in}{1.951225in}}%
\pgfpathcurveto{\pgfqpoint{3.026381in}{1.943411in}}{\pgfqpoint{3.036980in}{1.939021in}}{\pgfqpoint{3.048030in}{1.939021in}}%
\pgfpathclose%
\pgfusepath{stroke,fill}%
\end{pgfscope}%
\begin{pgfscope}%
\pgfpathrectangle{\pgfqpoint{0.772069in}{0.515123in}}{\pgfqpoint{3.875000in}{2.695000in}}%
\pgfusepath{clip}%
\pgfsetbuttcap%
\pgfsetroundjoin%
\definecolor{currentfill}{rgb}{0.121569,0.466667,0.705882}%
\pgfsetfillcolor{currentfill}%
\pgfsetlinewidth{1.003750pt}%
\definecolor{currentstroke}{rgb}{0.121569,0.466667,0.705882}%
\pgfsetstrokecolor{currentstroke}%
\pgfsetdash{}{0pt}%
\pgfpathmoveto{\pgfqpoint{3.410622in}{2.210249in}}%
\pgfpathcurveto{\pgfqpoint{3.421672in}{2.210249in}}{\pgfqpoint{3.432271in}{2.214639in}}{\pgfqpoint{3.440085in}{2.222453in}}%
\pgfpathcurveto{\pgfqpoint{3.447898in}{2.230267in}}{\pgfqpoint{3.452289in}{2.240866in}}{\pgfqpoint{3.452289in}{2.251916in}}%
\pgfpathcurveto{\pgfqpoint{3.452289in}{2.262966in}}{\pgfqpoint{3.447898in}{2.273565in}}{\pgfqpoint{3.440085in}{2.281379in}}%
\pgfpathcurveto{\pgfqpoint{3.432271in}{2.289192in}}{\pgfqpoint{3.421672in}{2.293583in}}{\pgfqpoint{3.410622in}{2.293583in}}%
\pgfpathcurveto{\pgfqpoint{3.399572in}{2.293583in}}{\pgfqpoint{3.388973in}{2.289192in}}{\pgfqpoint{3.381159in}{2.281379in}}%
\pgfpathcurveto{\pgfqpoint{3.373346in}{2.273565in}}{\pgfqpoint{3.368955in}{2.262966in}}{\pgfqpoint{3.368955in}{2.251916in}}%
\pgfpathcurveto{\pgfqpoint{3.368955in}{2.240866in}}{\pgfqpoint{3.373346in}{2.230267in}}{\pgfqpoint{3.381159in}{2.222453in}}%
\pgfpathcurveto{\pgfqpoint{3.388973in}{2.214639in}}{\pgfqpoint{3.399572in}{2.210249in}}{\pgfqpoint{3.410622in}{2.210249in}}%
\pgfpathclose%
\pgfusepath{stroke,fill}%
\end{pgfscope}%
\begin{pgfscope}%
\pgfpathrectangle{\pgfqpoint{0.772069in}{0.515123in}}{\pgfqpoint{3.875000in}{2.695000in}}%
\pgfusepath{clip}%
\pgfsetbuttcap%
\pgfsetroundjoin%
\definecolor{currentfill}{rgb}{0.121569,0.466667,0.705882}%
\pgfsetfillcolor{currentfill}%
\pgfsetlinewidth{1.003750pt}%
\definecolor{currentstroke}{rgb}{0.121569,0.466667,0.705882}%
\pgfsetstrokecolor{currentstroke}%
\pgfsetdash{}{0pt}%
\pgfpathmoveto{\pgfqpoint{3.751071in}{2.473500in}}%
\pgfpathcurveto{\pgfqpoint{3.762121in}{2.473500in}}{\pgfqpoint{3.772720in}{2.477891in}}{\pgfqpoint{3.780533in}{2.485704in}}%
\pgfpathcurveto{\pgfqpoint{3.788347in}{2.493518in}}{\pgfqpoint{3.792737in}{2.504117in}}{\pgfqpoint{3.792737in}{2.515167in}}%
\pgfpathcurveto{\pgfqpoint{3.792737in}{2.526217in}}{\pgfqpoint{3.788347in}{2.536816in}}{\pgfqpoint{3.780533in}{2.544630in}}%
\pgfpathcurveto{\pgfqpoint{3.772720in}{2.552443in}}{\pgfqpoint{3.762121in}{2.556834in}}{\pgfqpoint{3.751071in}{2.556834in}}%
\pgfpathcurveto{\pgfqpoint{3.740020in}{2.556834in}}{\pgfqpoint{3.729421in}{2.552443in}}{\pgfqpoint{3.721608in}{2.544630in}}%
\pgfpathcurveto{\pgfqpoint{3.713794in}{2.536816in}}{\pgfqpoint{3.709404in}{2.526217in}}{\pgfqpoint{3.709404in}{2.515167in}}%
\pgfpathcurveto{\pgfqpoint{3.709404in}{2.504117in}}{\pgfqpoint{3.713794in}{2.493518in}}{\pgfqpoint{3.721608in}{2.485704in}}%
\pgfpathcurveto{\pgfqpoint{3.729421in}{2.477891in}}{\pgfqpoint{3.740020in}{2.473500in}}{\pgfqpoint{3.751071in}{2.473500in}}%
\pgfpathclose%
\pgfusepath{stroke,fill}%
\end{pgfscope}%
\begin{pgfscope}%
\pgfpathrectangle{\pgfqpoint{0.772069in}{0.515123in}}{\pgfqpoint{3.875000in}{2.695000in}}%
\pgfusepath{clip}%
\pgfsetbuttcap%
\pgfsetroundjoin%
\definecolor{currentfill}{rgb}{0.121569,0.466667,0.705882}%
\pgfsetfillcolor{currentfill}%
\pgfsetlinewidth{1.003750pt}%
\definecolor{currentstroke}{rgb}{0.121569,0.466667,0.705882}%
\pgfsetstrokecolor{currentstroke}%
\pgfsetdash{}{0pt}%
\pgfpathmoveto{\pgfqpoint{4.110894in}{2.736751in}}%
\pgfpathcurveto{\pgfqpoint{4.121944in}{2.736751in}}{\pgfqpoint{4.132543in}{2.741142in}}{\pgfqpoint{4.140357in}{2.748955in}}%
\pgfpathcurveto{\pgfqpoint{4.148171in}{2.756769in}}{\pgfqpoint{4.152561in}{2.767368in}}{\pgfqpoint{4.152561in}{2.778418in}}%
\pgfpathcurveto{\pgfqpoint{4.152561in}{2.789468in}}{\pgfqpoint{4.148171in}{2.800067in}}{\pgfqpoint{4.140357in}{2.807881in}}%
\pgfpathcurveto{\pgfqpoint{4.132543in}{2.815694in}}{\pgfqpoint{4.121944in}{2.820085in}}{\pgfqpoint{4.110894in}{2.820085in}}%
\pgfpathcurveto{\pgfqpoint{4.099844in}{2.820085in}}{\pgfqpoint{4.089245in}{2.815694in}}{\pgfqpoint{4.081432in}{2.807881in}}%
\pgfpathcurveto{\pgfqpoint{4.073618in}{2.800067in}}{\pgfqpoint{4.069228in}{2.789468in}}{\pgfqpoint{4.069228in}{2.778418in}}%
\pgfpathcurveto{\pgfqpoint{4.069228in}{2.767368in}}{\pgfqpoint{4.073618in}{2.756769in}}{\pgfqpoint{4.081432in}{2.748955in}}%
\pgfpathcurveto{\pgfqpoint{4.089245in}{2.741142in}}{\pgfqpoint{4.099844in}{2.736751in}}{\pgfqpoint{4.110894in}{2.736751in}}%
\pgfpathclose%
\pgfusepath{stroke,fill}%
\end{pgfscope}%
\begin{pgfscope}%
\pgfpathrectangle{\pgfqpoint{0.772069in}{0.515123in}}{\pgfqpoint{3.875000in}{2.695000in}}%
\pgfusepath{clip}%
\pgfsetbuttcap%
\pgfsetroundjoin%
\definecolor{currentfill}{rgb}{0.121569,0.466667,0.705882}%
\pgfsetfillcolor{currentfill}%
\pgfsetlinewidth{1.003750pt}%
\definecolor{currentstroke}{rgb}{0.121569,0.466667,0.705882}%
\pgfsetstrokecolor{currentstroke}%
\pgfsetdash{}{0pt}%
\pgfpathmoveto{\pgfqpoint{4.470718in}{3.015957in}}%
\pgfpathcurveto{\pgfqpoint{4.481768in}{3.015957in}}{\pgfqpoint{4.492367in}{3.020347in}}{\pgfqpoint{4.500181in}{3.028161in}}%
\pgfpathcurveto{\pgfqpoint{4.507994in}{3.035975in}}{\pgfqpoint{4.512385in}{3.046574in}}{\pgfqpoint{4.512385in}{3.057624in}}%
\pgfpathcurveto{\pgfqpoint{4.512385in}{3.068674in}}{\pgfqpoint{4.507994in}{3.079273in}}{\pgfqpoint{4.500181in}{3.087087in}}%
\pgfpathcurveto{\pgfqpoint{4.492367in}{3.094900in}}{\pgfqpoint{4.481768in}{3.099290in}}{\pgfqpoint{4.470718in}{3.099290in}}%
\pgfpathcurveto{\pgfqpoint{4.459668in}{3.099290in}}{\pgfqpoint{4.449069in}{3.094900in}}{\pgfqpoint{4.441255in}{3.087087in}}%
\pgfpathcurveto{\pgfqpoint{4.433442in}{3.079273in}}{\pgfqpoint{4.429051in}{3.068674in}}{\pgfqpoint{4.429051in}{3.057624in}}%
\pgfpathcurveto{\pgfqpoint{4.429051in}{3.046574in}}{\pgfqpoint{4.433442in}{3.035975in}}{\pgfqpoint{4.441255in}{3.028161in}}%
\pgfpathcurveto{\pgfqpoint{4.449069in}{3.020347in}}{\pgfqpoint{4.459668in}{3.015957in}}{\pgfqpoint{4.470718in}{3.015957in}}%
\pgfpathclose%
\pgfusepath{stroke,fill}%
\end{pgfscope}%
\begin{pgfscope}%
\pgfpathrectangle{\pgfqpoint{0.772069in}{0.515123in}}{\pgfqpoint{3.875000in}{2.695000in}}%
\pgfusepath{clip}%
\pgfsetbuttcap%
\pgfsetroundjoin%
\definecolor{currentfill}{rgb}{1.000000,0.388235,0.278431}%
\pgfsetfillcolor{currentfill}%
\pgfsetlinewidth{1.003750pt}%
\definecolor{currentstroke}{rgb}{1.000000,0.388235,0.278431}%
\pgfsetstrokecolor{currentstroke}%
\pgfsetdash{}{0pt}%
\pgfpathmoveto{\pgfqpoint{1.077304in}{0.625956in}}%
\pgfpathcurveto{\pgfqpoint{1.088354in}{0.625956in}}{\pgfqpoint{1.098953in}{0.630347in}}{\pgfqpoint{1.106766in}{0.638160in}}%
\pgfpathcurveto{\pgfqpoint{1.114580in}{0.645974in}}{\pgfqpoint{1.118970in}{0.656573in}}{\pgfqpoint{1.118970in}{0.667623in}}%
\pgfpathcurveto{\pgfqpoint{1.118970in}{0.678673in}}{\pgfqpoint{1.114580in}{0.689272in}}{\pgfqpoint{1.106766in}{0.697086in}}%
\pgfpathcurveto{\pgfqpoint{1.098953in}{0.704899in}}{\pgfqpoint{1.088354in}{0.709290in}}{\pgfqpoint{1.077304in}{0.709290in}}%
\pgfpathcurveto{\pgfqpoint{1.066253in}{0.709290in}}{\pgfqpoint{1.055654in}{0.704899in}}{\pgfqpoint{1.047841in}{0.697086in}}%
\pgfpathcurveto{\pgfqpoint{1.040027in}{0.689272in}}{\pgfqpoint{1.035637in}{0.678673in}}{\pgfqpoint{1.035637in}{0.667623in}}%
\pgfpathcurveto{\pgfqpoint{1.035637in}{0.656573in}}{\pgfqpoint{1.040027in}{0.645974in}}{\pgfqpoint{1.047841in}{0.638160in}}%
\pgfpathcurveto{\pgfqpoint{1.055654in}{0.630347in}}{\pgfqpoint{1.066253in}{0.625956in}}{\pgfqpoint{1.077304in}{0.625956in}}%
\pgfpathclose%
\pgfusepath{stroke,fill}%
\end{pgfscope}%
\begin{pgfscope}%
\pgfpathrectangle{\pgfqpoint{0.772069in}{0.515123in}}{\pgfqpoint{3.875000in}{2.695000in}}%
\pgfusepath{clip}%
\pgfsetbuttcap%
\pgfsetroundjoin%
\definecolor{currentfill}{rgb}{1.000000,0.388235,0.278431}%
\pgfsetfillcolor{currentfill}%
\pgfsetlinewidth{1.003750pt}%
\definecolor{currentstroke}{rgb}{1.000000,0.388235,0.278431}%
\pgfsetstrokecolor{currentstroke}%
\pgfsetdash{}{0pt}%
\pgfpathmoveto{\pgfqpoint{1.224001in}{0.875380in}}%
\pgfpathcurveto{\pgfqpoint{1.235051in}{0.875380in}}{\pgfqpoint{1.245650in}{0.879770in}}{\pgfqpoint{1.253464in}{0.887584in}}%
\pgfpathcurveto{\pgfqpoint{1.261277in}{0.895398in}}{\pgfqpoint{1.265668in}{0.905997in}}{\pgfqpoint{1.265668in}{0.917047in}}%
\pgfpathcurveto{\pgfqpoint{1.265668in}{0.928097in}}{\pgfqpoint{1.261277in}{0.938696in}}{\pgfqpoint{1.253464in}{0.946509in}}%
\pgfpathcurveto{\pgfqpoint{1.245650in}{0.954323in}}{\pgfqpoint{1.235051in}{0.958713in}}{\pgfqpoint{1.224001in}{0.958713in}}%
\pgfpathcurveto{\pgfqpoint{1.212951in}{0.958713in}}{\pgfqpoint{1.202352in}{0.954323in}}{\pgfqpoint{1.194538in}{0.946509in}}%
\pgfpathcurveto{\pgfqpoint{1.186725in}{0.938696in}}{\pgfqpoint{1.182334in}{0.928097in}}{\pgfqpoint{1.182334in}{0.917047in}}%
\pgfpathcurveto{\pgfqpoint{1.182334in}{0.905997in}}{\pgfqpoint{1.186725in}{0.895398in}}{\pgfqpoint{1.194538in}{0.887584in}}%
\pgfpathcurveto{\pgfqpoint{1.202352in}{0.879770in}}{\pgfqpoint{1.212951in}{0.875380in}}{\pgfqpoint{1.224001in}{0.875380in}}%
\pgfpathclose%
\pgfusepath{stroke,fill}%
\end{pgfscope}%
\begin{pgfscope}%
\pgfpathrectangle{\pgfqpoint{0.772069in}{0.515123in}}{\pgfqpoint{3.875000in}{2.695000in}}%
\pgfusepath{clip}%
\pgfsetbuttcap%
\pgfsetroundjoin%
\definecolor{currentfill}{rgb}{1.000000,0.388235,0.278431}%
\pgfsetfillcolor{currentfill}%
\pgfsetlinewidth{1.003750pt}%
\definecolor{currentstroke}{rgb}{1.000000,0.388235,0.278431}%
\pgfsetstrokecolor{currentstroke}%
\pgfsetdash{}{0pt}%
\pgfpathmoveto{\pgfqpoint{1.370698in}{1.125336in}}%
\pgfpathcurveto{\pgfqpoint{1.381748in}{1.125336in}}{\pgfqpoint{1.392348in}{1.129726in}}{\pgfqpoint{1.400161in}{1.137539in}}%
\pgfpathcurveto{\pgfqpoint{1.407975in}{1.145353in}}{\pgfqpoint{1.412365in}{1.155952in}}{\pgfqpoint{1.412365in}{1.167002in}}%
\pgfpathcurveto{\pgfqpoint{1.412365in}{1.178052in}}{\pgfqpoint{1.407975in}{1.188651in}}{\pgfqpoint{1.400161in}{1.196465in}}%
\pgfpathcurveto{\pgfqpoint{1.392348in}{1.204279in}}{\pgfqpoint{1.381748in}{1.208669in}}{\pgfqpoint{1.370698in}{1.208669in}}%
\pgfpathcurveto{\pgfqpoint{1.359648in}{1.208669in}}{\pgfqpoint{1.349049in}{1.204279in}}{\pgfqpoint{1.341236in}{1.196465in}}%
\pgfpathcurveto{\pgfqpoint{1.333422in}{1.188651in}}{\pgfqpoint{1.329032in}{1.178052in}}{\pgfqpoint{1.329032in}{1.167002in}}%
\pgfpathcurveto{\pgfqpoint{1.329032in}{1.155952in}}{\pgfqpoint{1.333422in}{1.145353in}}{\pgfqpoint{1.341236in}{1.137539in}}%
\pgfpathcurveto{\pgfqpoint{1.349049in}{1.129726in}}{\pgfqpoint{1.359648in}{1.125336in}}{\pgfqpoint{1.370698in}{1.125336in}}%
\pgfpathclose%
\pgfusepath{stroke,fill}%
\end{pgfscope}%
\begin{pgfscope}%
\pgfpathrectangle{\pgfqpoint{0.772069in}{0.515123in}}{\pgfqpoint{3.875000in}{2.695000in}}%
\pgfusepath{clip}%
\pgfsetbuttcap%
\pgfsetroundjoin%
\definecolor{currentfill}{rgb}{1.000000,0.388235,0.278431}%
\pgfsetfillcolor{currentfill}%
\pgfsetlinewidth{1.003750pt}%
\definecolor{currentstroke}{rgb}{1.000000,0.388235,0.278431}%
\pgfsetstrokecolor{currentstroke}%
\pgfsetdash{}{0pt}%
\pgfpathmoveto{\pgfqpoint{1.534003in}{1.399223in}}%
\pgfpathcurveto{\pgfqpoint{1.545053in}{1.399223in}}{\pgfqpoint{1.555652in}{1.403613in}}{\pgfqpoint{1.563466in}{1.411427in}}%
\pgfpathcurveto{\pgfqpoint{1.571279in}{1.419241in}}{\pgfqpoint{1.575670in}{1.429840in}}{\pgfqpoint{1.575670in}{1.440890in}}%
\pgfpathcurveto{\pgfqpoint{1.575670in}{1.451940in}}{\pgfqpoint{1.571279in}{1.462539in}}{\pgfqpoint{1.563466in}{1.470353in}}%
\pgfpathcurveto{\pgfqpoint{1.555652in}{1.478166in}}{\pgfqpoint{1.545053in}{1.482556in}}{\pgfqpoint{1.534003in}{1.482556in}}%
\pgfpathcurveto{\pgfqpoint{1.522953in}{1.482556in}}{\pgfqpoint{1.512354in}{1.478166in}}{\pgfqpoint{1.504540in}{1.470353in}}%
\pgfpathcurveto{\pgfqpoint{1.496727in}{1.462539in}}{\pgfqpoint{1.492336in}{1.451940in}}{\pgfqpoint{1.492336in}{1.440890in}}%
\pgfpathcurveto{\pgfqpoint{1.492336in}{1.429840in}}{\pgfqpoint{1.496727in}{1.419241in}}{\pgfqpoint{1.504540in}{1.411427in}}%
\pgfpathcurveto{\pgfqpoint{1.512354in}{1.403613in}}{\pgfqpoint{1.522953in}{1.399223in}}{\pgfqpoint{1.534003in}{1.399223in}}%
\pgfpathclose%
\pgfusepath{stroke,fill}%
\end{pgfscope}%
\begin{pgfscope}%
\pgfpathrectangle{\pgfqpoint{0.772069in}{0.515123in}}{\pgfqpoint{3.875000in}{2.695000in}}%
\pgfusepath{clip}%
\pgfsetbuttcap%
\pgfsetroundjoin%
\definecolor{currentfill}{rgb}{1.000000,0.388235,0.278431}%
\pgfsetfillcolor{currentfill}%
\pgfsetlinewidth{1.003750pt}%
\definecolor{currentstroke}{rgb}{1.000000,0.388235,0.278431}%
\pgfsetstrokecolor{currentstroke}%
\pgfsetdash{}{0pt}%
\pgfpathmoveto{\pgfqpoint{1.697308in}{1.670451in}}%
\pgfpathcurveto{\pgfqpoint{1.708358in}{1.670451in}}{\pgfqpoint{1.718957in}{1.674842in}}{\pgfqpoint{1.726770in}{1.682655in}}%
\pgfpathcurveto{\pgfqpoint{1.734584in}{1.690469in}}{\pgfqpoint{1.738974in}{1.701068in}}{\pgfqpoint{1.738974in}{1.712118in}}%
\pgfpathcurveto{\pgfqpoint{1.738974in}{1.723168in}}{\pgfqpoint{1.734584in}{1.733767in}}{\pgfqpoint{1.726770in}{1.741581in}}%
\pgfpathcurveto{\pgfqpoint{1.718957in}{1.749395in}}{\pgfqpoint{1.708358in}{1.753785in}}{\pgfqpoint{1.697308in}{1.753785in}}%
\pgfpathcurveto{\pgfqpoint{1.686257in}{1.753785in}}{\pgfqpoint{1.675658in}{1.749395in}}{\pgfqpoint{1.667845in}{1.741581in}}%
\pgfpathcurveto{\pgfqpoint{1.660031in}{1.733767in}}{\pgfqpoint{1.655641in}{1.723168in}}{\pgfqpoint{1.655641in}{1.712118in}}%
\pgfpathcurveto{\pgfqpoint{1.655641in}{1.701068in}}{\pgfqpoint{1.660031in}{1.690469in}}{\pgfqpoint{1.667845in}{1.682655in}}%
\pgfpathcurveto{\pgfqpoint{1.675658in}{1.674842in}}{\pgfqpoint{1.686257in}{1.670451in}}{\pgfqpoint{1.697308in}{1.670451in}}%
\pgfpathclose%
\pgfusepath{stroke,fill}%
\end{pgfscope}%
\begin{pgfscope}%
\pgfpathrectangle{\pgfqpoint{0.772069in}{0.515123in}}{\pgfqpoint{3.875000in}{2.695000in}}%
\pgfusepath{clip}%
\pgfsetbuttcap%
\pgfsetroundjoin%
\definecolor{currentfill}{rgb}{1.000000,0.388235,0.278431}%
\pgfsetfillcolor{currentfill}%
\pgfsetlinewidth{1.003750pt}%
\definecolor{currentstroke}{rgb}{1.000000,0.388235,0.278431}%
\pgfsetstrokecolor{currentstroke}%
\pgfsetdash{}{0pt}%
\pgfpathmoveto{\pgfqpoint{1.855076in}{1.939021in}}%
\pgfpathcurveto{\pgfqpoint{1.866127in}{1.939021in}}{\pgfqpoint{1.876726in}{1.943411in}}{\pgfqpoint{1.884539in}{1.951225in}}%
\pgfpathcurveto{\pgfqpoint{1.892353in}{1.959038in}}{\pgfqpoint{1.896743in}{1.969637in}}{\pgfqpoint{1.896743in}{1.980687in}}%
\pgfpathcurveto{\pgfqpoint{1.896743in}{1.991738in}}{\pgfqpoint{1.892353in}{2.002337in}}{\pgfqpoint{1.884539in}{2.010150in}}%
\pgfpathcurveto{\pgfqpoint{1.876726in}{2.017964in}}{\pgfqpoint{1.866127in}{2.022354in}}{\pgfqpoint{1.855076in}{2.022354in}}%
\pgfpathcurveto{\pgfqpoint{1.844026in}{2.022354in}}{\pgfqpoint{1.833427in}{2.017964in}}{\pgfqpoint{1.825614in}{2.010150in}}%
\pgfpathcurveto{\pgfqpoint{1.817800in}{2.002337in}}{\pgfqpoint{1.813410in}{1.991738in}}{\pgfqpoint{1.813410in}{1.980687in}}%
\pgfpathcurveto{\pgfqpoint{1.813410in}{1.969637in}}{\pgfqpoint{1.817800in}{1.959038in}}{\pgfqpoint{1.825614in}{1.951225in}}%
\pgfpathcurveto{\pgfqpoint{1.833427in}{1.943411in}}{\pgfqpoint{1.844026in}{1.939021in}}{\pgfqpoint{1.855076in}{1.939021in}}%
\pgfpathclose%
\pgfusepath{stroke,fill}%
\end{pgfscope}%
\begin{pgfscope}%
\pgfpathrectangle{\pgfqpoint{0.772069in}{0.515123in}}{\pgfqpoint{3.875000in}{2.695000in}}%
\pgfusepath{clip}%
\pgfsetbuttcap%
\pgfsetroundjoin%
\definecolor{currentfill}{rgb}{1.000000,0.388235,0.278431}%
\pgfsetfillcolor{currentfill}%
\pgfsetlinewidth{1.003750pt}%
\definecolor{currentstroke}{rgb}{1.000000,0.388235,0.278431}%
\pgfsetstrokecolor{currentstroke}%
\pgfsetdash{}{0pt}%
\pgfpathmoveto{\pgfqpoint{2.015613in}{2.210249in}}%
\pgfpathcurveto{\pgfqpoint{2.026663in}{2.210249in}}{\pgfqpoint{2.037262in}{2.214639in}}{\pgfqpoint{2.045076in}{2.222453in}}%
\pgfpathcurveto{\pgfqpoint{2.052890in}{2.230267in}}{\pgfqpoint{2.057280in}{2.240866in}}{\pgfqpoint{2.057280in}{2.251916in}}%
\pgfpathcurveto{\pgfqpoint{2.057280in}{2.262966in}}{\pgfqpoint{2.052890in}{2.273565in}}{\pgfqpoint{2.045076in}{2.281379in}}%
\pgfpathcurveto{\pgfqpoint{2.037262in}{2.289192in}}{\pgfqpoint{2.026663in}{2.293583in}}{\pgfqpoint{2.015613in}{2.293583in}}%
\pgfpathcurveto{\pgfqpoint{2.004563in}{2.293583in}}{\pgfqpoint{1.993964in}{2.289192in}}{\pgfqpoint{1.986150in}{2.281379in}}%
\pgfpathcurveto{\pgfqpoint{1.978337in}{2.273565in}}{\pgfqpoint{1.973946in}{2.262966in}}{\pgfqpoint{1.973946in}{2.251916in}}%
\pgfpathcurveto{\pgfqpoint{1.973946in}{2.240866in}}{\pgfqpoint{1.978337in}{2.230267in}}{\pgfqpoint{1.986150in}{2.222453in}}%
\pgfpathcurveto{\pgfqpoint{1.993964in}{2.214639in}}{\pgfqpoint{2.004563in}{2.210249in}}{\pgfqpoint{2.015613in}{2.210249in}}%
\pgfpathclose%
\pgfusepath{stroke,fill}%
\end{pgfscope}%
\begin{pgfscope}%
\pgfpathrectangle{\pgfqpoint{0.772069in}{0.515123in}}{\pgfqpoint{3.875000in}{2.695000in}}%
\pgfusepath{clip}%
\pgfsetbuttcap%
\pgfsetroundjoin%
\definecolor{currentfill}{rgb}{1.000000,0.388235,0.278431}%
\pgfsetfillcolor{currentfill}%
\pgfsetlinewidth{1.003750pt}%
\definecolor{currentstroke}{rgb}{1.000000,0.388235,0.278431}%
\pgfsetstrokecolor{currentstroke}%
\pgfsetdash{}{0pt}%
\pgfpathmoveto{\pgfqpoint{2.170614in}{2.473500in}}%
\pgfpathcurveto{\pgfqpoint{2.181664in}{2.473500in}}{\pgfqpoint{2.192263in}{2.477891in}}{\pgfqpoint{2.200077in}{2.485704in}}%
\pgfpathcurveto{\pgfqpoint{2.207891in}{2.493518in}}{\pgfqpoint{2.212281in}{2.504117in}}{\pgfqpoint{2.212281in}{2.515167in}}%
\pgfpathcurveto{\pgfqpoint{2.212281in}{2.526217in}}{\pgfqpoint{2.207891in}{2.536816in}}{\pgfqpoint{2.200077in}{2.544630in}}%
\pgfpathcurveto{\pgfqpoint{2.192263in}{2.552443in}}{\pgfqpoint{2.181664in}{2.556834in}}{\pgfqpoint{2.170614in}{2.556834in}}%
\pgfpathcurveto{\pgfqpoint{2.159564in}{2.556834in}}{\pgfqpoint{2.148965in}{2.552443in}}{\pgfqpoint{2.141151in}{2.544630in}}%
\pgfpathcurveto{\pgfqpoint{2.133338in}{2.536816in}}{\pgfqpoint{2.128947in}{2.526217in}}{\pgfqpoint{2.128947in}{2.515167in}}%
\pgfpathcurveto{\pgfqpoint{2.128947in}{2.504117in}}{\pgfqpoint{2.133338in}{2.493518in}}{\pgfqpoint{2.141151in}{2.485704in}}%
\pgfpathcurveto{\pgfqpoint{2.148965in}{2.477891in}}{\pgfqpoint{2.159564in}{2.473500in}}{\pgfqpoint{2.170614in}{2.473500in}}%
\pgfpathclose%
\pgfusepath{stroke,fill}%
\end{pgfscope}%
\begin{pgfscope}%
\pgfpathrectangle{\pgfqpoint{0.772069in}{0.515123in}}{\pgfqpoint{3.875000in}{2.695000in}}%
\pgfusepath{clip}%
\pgfsetbuttcap%
\pgfsetroundjoin%
\definecolor{currentfill}{rgb}{1.000000,0.388235,0.278431}%
\pgfsetfillcolor{currentfill}%
\pgfsetlinewidth{1.003750pt}%
\definecolor{currentstroke}{rgb}{1.000000,0.388235,0.278431}%
\pgfsetstrokecolor{currentstroke}%
\pgfsetdash{}{0pt}%
\pgfpathmoveto{\pgfqpoint{2.325615in}{2.736751in}}%
\pgfpathcurveto{\pgfqpoint{2.336665in}{2.736751in}}{\pgfqpoint{2.347264in}{2.741142in}}{\pgfqpoint{2.355078in}{2.748955in}}%
\pgfpathcurveto{\pgfqpoint{2.362892in}{2.756769in}}{\pgfqpoint{2.367282in}{2.767368in}}{\pgfqpoint{2.367282in}{2.778418in}}%
\pgfpathcurveto{\pgfqpoint{2.367282in}{2.789468in}}{\pgfqpoint{2.362892in}{2.800067in}}{\pgfqpoint{2.355078in}{2.807881in}}%
\pgfpathcurveto{\pgfqpoint{2.347264in}{2.815694in}}{\pgfqpoint{2.336665in}{2.820085in}}{\pgfqpoint{2.325615in}{2.820085in}}%
\pgfpathcurveto{\pgfqpoint{2.314565in}{2.820085in}}{\pgfqpoint{2.303966in}{2.815694in}}{\pgfqpoint{2.296152in}{2.807881in}}%
\pgfpathcurveto{\pgfqpoint{2.288339in}{2.800067in}}{\pgfqpoint{2.283948in}{2.789468in}}{\pgfqpoint{2.283948in}{2.778418in}}%
\pgfpathcurveto{\pgfqpoint{2.283948in}{2.767368in}}{\pgfqpoint{2.288339in}{2.756769in}}{\pgfqpoint{2.296152in}{2.748955in}}%
\pgfpathcurveto{\pgfqpoint{2.303966in}{2.741142in}}{\pgfqpoint{2.314565in}{2.736751in}}{\pgfqpoint{2.325615in}{2.736751in}}%
\pgfpathclose%
\pgfusepath{stroke,fill}%
\end{pgfscope}%
\begin{pgfscope}%
\pgfpathrectangle{\pgfqpoint{0.772069in}{0.515123in}}{\pgfqpoint{3.875000in}{2.695000in}}%
\pgfusepath{clip}%
\pgfsetbuttcap%
\pgfsetroundjoin%
\definecolor{currentfill}{rgb}{1.000000,0.388235,0.278431}%
\pgfsetfillcolor{currentfill}%
\pgfsetlinewidth{1.003750pt}%
\definecolor{currentstroke}{rgb}{1.000000,0.388235,0.278431}%
\pgfsetstrokecolor{currentstroke}%
\pgfsetdash{}{0pt}%
\pgfpathmoveto{\pgfqpoint{2.491688in}{3.015957in}}%
\pgfpathcurveto{\pgfqpoint{2.502738in}{3.015957in}}{\pgfqpoint{2.513337in}{3.020347in}}{\pgfqpoint{2.521150in}{3.028161in}}%
\pgfpathcurveto{\pgfqpoint{2.528964in}{3.035975in}}{\pgfqpoint{2.533354in}{3.046574in}}{\pgfqpoint{2.533354in}{3.057624in}}%
\pgfpathcurveto{\pgfqpoint{2.533354in}{3.068674in}}{\pgfqpoint{2.528964in}{3.079273in}}{\pgfqpoint{2.521150in}{3.087087in}}%
\pgfpathcurveto{\pgfqpoint{2.513337in}{3.094900in}}{\pgfqpoint{2.502738in}{3.099290in}}{\pgfqpoint{2.491688in}{3.099290in}}%
\pgfpathcurveto{\pgfqpoint{2.480637in}{3.099290in}}{\pgfqpoint{2.470038in}{3.094900in}}{\pgfqpoint{2.462225in}{3.087087in}}%
\pgfpathcurveto{\pgfqpoint{2.454411in}{3.079273in}}{\pgfqpoint{2.450021in}{3.068674in}}{\pgfqpoint{2.450021in}{3.057624in}}%
\pgfpathcurveto{\pgfqpoint{2.450021in}{3.046574in}}{\pgfqpoint{2.454411in}{3.035975in}}{\pgfqpoint{2.462225in}{3.028161in}}%
\pgfpathcurveto{\pgfqpoint{2.470038in}{3.020347in}}{\pgfqpoint{2.480637in}{3.015957in}}{\pgfqpoint{2.491688in}{3.015957in}}%
\pgfpathclose%
\pgfusepath{stroke,fill}%
\end{pgfscope}%
\begin{pgfscope}%
\pgfpathrectangle{\pgfqpoint{0.772069in}{0.515123in}}{\pgfqpoint{3.875000in}{2.695000in}}%
\pgfusepath{clip}%
\pgfsetbuttcap%
\pgfsetroundjoin%
\definecolor{currentfill}{rgb}{1.000000,0.843137,0.000000}%
\pgfsetfillcolor{currentfill}%
\pgfsetlinewidth{1.003750pt}%
\definecolor{currentstroke}{rgb}{1.000000,0.843137,0.000000}%
\pgfsetstrokecolor{currentstroke}%
\pgfsetdash{}{0pt}%
\pgfpathmoveto{\pgfqpoint{1.041321in}{0.625956in}}%
\pgfpathcurveto{\pgfqpoint{1.052371in}{0.625956in}}{\pgfqpoint{1.062970in}{0.630347in}}{\pgfqpoint{1.070784in}{0.638160in}}%
\pgfpathcurveto{\pgfqpoint{1.078598in}{0.645974in}}{\pgfqpoint{1.082988in}{0.656573in}}{\pgfqpoint{1.082988in}{0.667623in}}%
\pgfpathcurveto{\pgfqpoint{1.082988in}{0.678673in}}{\pgfqpoint{1.078598in}{0.689272in}}{\pgfqpoint{1.070784in}{0.697086in}}%
\pgfpathcurveto{\pgfqpoint{1.062970in}{0.704899in}}{\pgfqpoint{1.052371in}{0.709290in}}{\pgfqpoint{1.041321in}{0.709290in}}%
\pgfpathcurveto{\pgfqpoint{1.030271in}{0.709290in}}{\pgfqpoint{1.019672in}{0.704899in}}{\pgfqpoint{1.011858in}{0.697086in}}%
\pgfpathcurveto{\pgfqpoint{1.004045in}{0.689272in}}{\pgfqpoint{0.999655in}{0.678673in}}{\pgfqpoint{0.999655in}{0.667623in}}%
\pgfpathcurveto{\pgfqpoint{0.999655in}{0.656573in}}{\pgfqpoint{1.004045in}{0.645974in}}{\pgfqpoint{1.011858in}{0.638160in}}%
\pgfpathcurveto{\pgfqpoint{1.019672in}{0.630347in}}{\pgfqpoint{1.030271in}{0.625956in}}{\pgfqpoint{1.041321in}{0.625956in}}%
\pgfpathclose%
\pgfusepath{stroke,fill}%
\end{pgfscope}%
\begin{pgfscope}%
\pgfpathrectangle{\pgfqpoint{0.772069in}{0.515123in}}{\pgfqpoint{3.875000in}{2.695000in}}%
\pgfusepath{clip}%
\pgfsetbuttcap%
\pgfsetroundjoin%
\definecolor{currentfill}{rgb}{1.000000,0.843137,0.000000}%
\pgfsetfillcolor{currentfill}%
\pgfsetlinewidth{1.003750pt}%
\definecolor{currentstroke}{rgb}{1.000000,0.843137,0.000000}%
\pgfsetstrokecolor{currentstroke}%
\pgfsetdash{}{0pt}%
\pgfpathmoveto{\pgfqpoint{1.160340in}{0.875380in}}%
\pgfpathcurveto{\pgfqpoint{1.171390in}{0.875380in}}{\pgfqpoint{1.181989in}{0.879770in}}{\pgfqpoint{1.189803in}{0.887584in}}%
\pgfpathcurveto{\pgfqpoint{1.197616in}{0.895398in}}{\pgfqpoint{1.202007in}{0.905997in}}{\pgfqpoint{1.202007in}{0.917047in}}%
\pgfpathcurveto{\pgfqpoint{1.202007in}{0.928097in}}{\pgfqpoint{1.197616in}{0.938696in}}{\pgfqpoint{1.189803in}{0.946509in}}%
\pgfpathcurveto{\pgfqpoint{1.181989in}{0.954323in}}{\pgfqpoint{1.171390in}{0.958713in}}{\pgfqpoint{1.160340in}{0.958713in}}%
\pgfpathcurveto{\pgfqpoint{1.149290in}{0.958713in}}{\pgfqpoint{1.138691in}{0.954323in}}{\pgfqpoint{1.130877in}{0.946509in}}%
\pgfpathcurveto{\pgfqpoint{1.123063in}{0.938696in}}{\pgfqpoint{1.118673in}{0.928097in}}{\pgfqpoint{1.118673in}{0.917047in}}%
\pgfpathcurveto{\pgfqpoint{1.118673in}{0.905997in}}{\pgfqpoint{1.123063in}{0.895398in}}{\pgfqpoint{1.130877in}{0.887584in}}%
\pgfpathcurveto{\pgfqpoint{1.138691in}{0.879770in}}{\pgfqpoint{1.149290in}{0.875380in}}{\pgfqpoint{1.160340in}{0.875380in}}%
\pgfpathclose%
\pgfusepath{stroke,fill}%
\end{pgfscope}%
\begin{pgfscope}%
\pgfpathrectangle{\pgfqpoint{0.772069in}{0.515123in}}{\pgfqpoint{3.875000in}{2.695000in}}%
\pgfusepath{clip}%
\pgfsetbuttcap%
\pgfsetroundjoin%
\definecolor{currentfill}{rgb}{1.000000,0.843137,0.000000}%
\pgfsetfillcolor{currentfill}%
\pgfsetlinewidth{1.003750pt}%
\definecolor{currentstroke}{rgb}{1.000000,0.843137,0.000000}%
\pgfsetstrokecolor{currentstroke}%
\pgfsetdash{}{0pt}%
\pgfpathmoveto{\pgfqpoint{1.279358in}{1.125336in}}%
\pgfpathcurveto{\pgfqpoint{1.290409in}{1.125336in}}{\pgfqpoint{1.301008in}{1.129726in}}{\pgfqpoint{1.308821in}{1.137539in}}%
\pgfpathcurveto{\pgfqpoint{1.316635in}{1.145353in}}{\pgfqpoint{1.321025in}{1.155952in}}{\pgfqpoint{1.321025in}{1.167002in}}%
\pgfpathcurveto{\pgfqpoint{1.321025in}{1.178052in}}{\pgfqpoint{1.316635in}{1.188651in}}{\pgfqpoint{1.308821in}{1.196465in}}%
\pgfpathcurveto{\pgfqpoint{1.301008in}{1.204279in}}{\pgfqpoint{1.290409in}{1.208669in}}{\pgfqpoint{1.279358in}{1.208669in}}%
\pgfpathcurveto{\pgfqpoint{1.268308in}{1.208669in}}{\pgfqpoint{1.257709in}{1.204279in}}{\pgfqpoint{1.249896in}{1.196465in}}%
\pgfpathcurveto{\pgfqpoint{1.242082in}{1.188651in}}{\pgfqpoint{1.237692in}{1.178052in}}{\pgfqpoint{1.237692in}{1.167002in}}%
\pgfpathcurveto{\pgfqpoint{1.237692in}{1.155952in}}{\pgfqpoint{1.242082in}{1.145353in}}{\pgfqpoint{1.249896in}{1.137539in}}%
\pgfpathcurveto{\pgfqpoint{1.257709in}{1.129726in}}{\pgfqpoint{1.268308in}{1.125336in}}{\pgfqpoint{1.279358in}{1.125336in}}%
\pgfpathclose%
\pgfusepath{stroke,fill}%
\end{pgfscope}%
\begin{pgfscope}%
\pgfpathrectangle{\pgfqpoint{0.772069in}{0.515123in}}{\pgfqpoint{3.875000in}{2.695000in}}%
\pgfusepath{clip}%
\pgfsetbuttcap%
\pgfsetroundjoin%
\definecolor{currentfill}{rgb}{1.000000,0.843137,0.000000}%
\pgfsetfillcolor{currentfill}%
\pgfsetlinewidth{1.003750pt}%
\definecolor{currentstroke}{rgb}{1.000000,0.843137,0.000000}%
\pgfsetstrokecolor{currentstroke}%
\pgfsetdash{}{0pt}%
\pgfpathmoveto{\pgfqpoint{1.412216in}{1.399223in}}%
\pgfpathcurveto{\pgfqpoint{1.423267in}{1.399223in}}{\pgfqpoint{1.433866in}{1.403613in}}{\pgfqpoint{1.441679in}{1.411427in}}%
\pgfpathcurveto{\pgfqpoint{1.449493in}{1.419241in}}{\pgfqpoint{1.453883in}{1.429840in}}{\pgfqpoint{1.453883in}{1.440890in}}%
\pgfpathcurveto{\pgfqpoint{1.453883in}{1.451940in}}{\pgfqpoint{1.449493in}{1.462539in}}{\pgfqpoint{1.441679in}{1.470353in}}%
\pgfpathcurveto{\pgfqpoint{1.433866in}{1.478166in}}{\pgfqpoint{1.423267in}{1.482556in}}{\pgfqpoint{1.412216in}{1.482556in}}%
\pgfpathcurveto{\pgfqpoint{1.401166in}{1.482556in}}{\pgfqpoint{1.390567in}{1.478166in}}{\pgfqpoint{1.382754in}{1.470353in}}%
\pgfpathcurveto{\pgfqpoint{1.374940in}{1.462539in}}{\pgfqpoint{1.370550in}{1.451940in}}{\pgfqpoint{1.370550in}{1.440890in}}%
\pgfpathcurveto{\pgfqpoint{1.370550in}{1.429840in}}{\pgfqpoint{1.374940in}{1.419241in}}{\pgfqpoint{1.382754in}{1.411427in}}%
\pgfpathcurveto{\pgfqpoint{1.390567in}{1.403613in}}{\pgfqpoint{1.401166in}{1.399223in}}{\pgfqpoint{1.412216in}{1.399223in}}%
\pgfpathclose%
\pgfusepath{stroke,fill}%
\end{pgfscope}%
\begin{pgfscope}%
\pgfpathrectangle{\pgfqpoint{0.772069in}{0.515123in}}{\pgfqpoint{3.875000in}{2.695000in}}%
\pgfusepath{clip}%
\pgfsetbuttcap%
\pgfsetroundjoin%
\definecolor{currentfill}{rgb}{1.000000,0.843137,0.000000}%
\pgfsetfillcolor{currentfill}%
\pgfsetlinewidth{1.003750pt}%
\definecolor{currentstroke}{rgb}{1.000000,0.843137,0.000000}%
\pgfsetstrokecolor{currentstroke}%
\pgfsetdash{}{0pt}%
\pgfpathmoveto{\pgfqpoint{1.542307in}{1.670451in}}%
\pgfpathcurveto{\pgfqpoint{1.553357in}{1.670451in}}{\pgfqpoint{1.563956in}{1.674842in}}{\pgfqpoint{1.571769in}{1.682655in}}%
\pgfpathcurveto{\pgfqpoint{1.579583in}{1.690469in}}{\pgfqpoint{1.583973in}{1.701068in}}{\pgfqpoint{1.583973in}{1.712118in}}%
\pgfpathcurveto{\pgfqpoint{1.583973in}{1.723168in}}{\pgfqpoint{1.579583in}{1.733767in}}{\pgfqpoint{1.571769in}{1.741581in}}%
\pgfpathcurveto{\pgfqpoint{1.563956in}{1.749395in}}{\pgfqpoint{1.553357in}{1.753785in}}{\pgfqpoint{1.542307in}{1.753785in}}%
\pgfpathcurveto{\pgfqpoint{1.531256in}{1.753785in}}{\pgfqpoint{1.520657in}{1.749395in}}{\pgfqpoint{1.512844in}{1.741581in}}%
\pgfpathcurveto{\pgfqpoint{1.505030in}{1.733767in}}{\pgfqpoint{1.500640in}{1.723168in}}{\pgfqpoint{1.500640in}{1.712118in}}%
\pgfpathcurveto{\pgfqpoint{1.500640in}{1.701068in}}{\pgfqpoint{1.505030in}{1.690469in}}{\pgfqpoint{1.512844in}{1.682655in}}%
\pgfpathcurveto{\pgfqpoint{1.520657in}{1.674842in}}{\pgfqpoint{1.531256in}{1.670451in}}{\pgfqpoint{1.542307in}{1.670451in}}%
\pgfpathclose%
\pgfusepath{stroke,fill}%
\end{pgfscope}%
\begin{pgfscope}%
\pgfpathrectangle{\pgfqpoint{0.772069in}{0.515123in}}{\pgfqpoint{3.875000in}{2.695000in}}%
\pgfusepath{clip}%
\pgfsetbuttcap%
\pgfsetroundjoin%
\definecolor{currentfill}{rgb}{1.000000,0.843137,0.000000}%
\pgfsetfillcolor{currentfill}%
\pgfsetlinewidth{1.003750pt}%
\definecolor{currentstroke}{rgb}{1.000000,0.843137,0.000000}%
\pgfsetstrokecolor{currentstroke}%
\pgfsetdash{}{0pt}%
\pgfpathmoveto{\pgfqpoint{1.669629in}{1.939021in}}%
\pgfpathcurveto{\pgfqpoint{1.680679in}{1.939021in}}{\pgfqpoint{1.691278in}{1.943411in}}{\pgfqpoint{1.699092in}{1.951225in}}%
\pgfpathcurveto{\pgfqpoint{1.706905in}{1.959038in}}{\pgfqpoint{1.711295in}{1.969637in}}{\pgfqpoint{1.711295in}{1.980687in}}%
\pgfpathcurveto{\pgfqpoint{1.711295in}{1.991738in}}{\pgfqpoint{1.706905in}{2.002337in}}{\pgfqpoint{1.699092in}{2.010150in}}%
\pgfpathcurveto{\pgfqpoint{1.691278in}{2.017964in}}{\pgfqpoint{1.680679in}{2.022354in}}{\pgfqpoint{1.669629in}{2.022354in}}%
\pgfpathcurveto{\pgfqpoint{1.658579in}{2.022354in}}{\pgfqpoint{1.647980in}{2.017964in}}{\pgfqpoint{1.640166in}{2.010150in}}%
\pgfpathcurveto{\pgfqpoint{1.632352in}{2.002337in}}{\pgfqpoint{1.627962in}{1.991738in}}{\pgfqpoint{1.627962in}{1.980687in}}%
\pgfpathcurveto{\pgfqpoint{1.627962in}{1.969637in}}{\pgfqpoint{1.632352in}{1.959038in}}{\pgfqpoint{1.640166in}{1.951225in}}%
\pgfpathcurveto{\pgfqpoint{1.647980in}{1.943411in}}{\pgfqpoint{1.658579in}{1.939021in}}{\pgfqpoint{1.669629in}{1.939021in}}%
\pgfpathclose%
\pgfusepath{stroke,fill}%
\end{pgfscope}%
\begin{pgfscope}%
\pgfpathrectangle{\pgfqpoint{0.772069in}{0.515123in}}{\pgfqpoint{3.875000in}{2.695000in}}%
\pgfusepath{clip}%
\pgfsetbuttcap%
\pgfsetroundjoin%
\definecolor{currentfill}{rgb}{1.000000,0.843137,0.000000}%
\pgfsetfillcolor{currentfill}%
\pgfsetlinewidth{1.003750pt}%
\definecolor{currentstroke}{rgb}{1.000000,0.843137,0.000000}%
\pgfsetstrokecolor{currentstroke}%
\pgfsetdash{}{0pt}%
\pgfpathmoveto{\pgfqpoint{1.799719in}{2.210249in}}%
\pgfpathcurveto{\pgfqpoint{1.810769in}{2.210249in}}{\pgfqpoint{1.821368in}{2.214639in}}{\pgfqpoint{1.829182in}{2.222453in}}%
\pgfpathcurveto{\pgfqpoint{1.836995in}{2.230267in}}{\pgfqpoint{1.841386in}{2.240866in}}{\pgfqpoint{1.841386in}{2.251916in}}%
\pgfpathcurveto{\pgfqpoint{1.841386in}{2.262966in}}{\pgfqpoint{1.836995in}{2.273565in}}{\pgfqpoint{1.829182in}{2.281379in}}%
\pgfpathcurveto{\pgfqpoint{1.821368in}{2.289192in}}{\pgfqpoint{1.810769in}{2.293583in}}{\pgfqpoint{1.799719in}{2.293583in}}%
\pgfpathcurveto{\pgfqpoint{1.788669in}{2.293583in}}{\pgfqpoint{1.778070in}{2.289192in}}{\pgfqpoint{1.770256in}{2.281379in}}%
\pgfpathcurveto{\pgfqpoint{1.762443in}{2.273565in}}{\pgfqpoint{1.758052in}{2.262966in}}{\pgfqpoint{1.758052in}{2.251916in}}%
\pgfpathcurveto{\pgfqpoint{1.758052in}{2.240866in}}{\pgfqpoint{1.762443in}{2.230267in}}{\pgfqpoint{1.770256in}{2.222453in}}%
\pgfpathcurveto{\pgfqpoint{1.778070in}{2.214639in}}{\pgfqpoint{1.788669in}{2.210249in}}{\pgfqpoint{1.799719in}{2.210249in}}%
\pgfpathclose%
\pgfusepath{stroke,fill}%
\end{pgfscope}%
\begin{pgfscope}%
\pgfpathrectangle{\pgfqpoint{0.772069in}{0.515123in}}{\pgfqpoint{3.875000in}{2.695000in}}%
\pgfusepath{clip}%
\pgfsetbuttcap%
\pgfsetroundjoin%
\definecolor{currentfill}{rgb}{1.000000,0.843137,0.000000}%
\pgfsetfillcolor{currentfill}%
\pgfsetlinewidth{1.003750pt}%
\definecolor{currentstroke}{rgb}{1.000000,0.843137,0.000000}%
\pgfsetstrokecolor{currentstroke}%
\pgfsetdash{}{0pt}%
\pgfpathmoveto{\pgfqpoint{1.924273in}{2.473500in}}%
\pgfpathcurveto{\pgfqpoint{1.935323in}{2.473500in}}{\pgfqpoint{1.945922in}{2.477891in}}{\pgfqpoint{1.953736in}{2.485704in}}%
\pgfpathcurveto{\pgfqpoint{1.961550in}{2.493518in}}{\pgfqpoint{1.965940in}{2.504117in}}{\pgfqpoint{1.965940in}{2.515167in}}%
\pgfpathcurveto{\pgfqpoint{1.965940in}{2.526217in}}{\pgfqpoint{1.961550in}{2.536816in}}{\pgfqpoint{1.953736in}{2.544630in}}%
\pgfpathcurveto{\pgfqpoint{1.945922in}{2.552443in}}{\pgfqpoint{1.935323in}{2.556834in}}{\pgfqpoint{1.924273in}{2.556834in}}%
\pgfpathcurveto{\pgfqpoint{1.913223in}{2.556834in}}{\pgfqpoint{1.902624in}{2.552443in}}{\pgfqpoint{1.894811in}{2.544630in}}%
\pgfpathcurveto{\pgfqpoint{1.886997in}{2.536816in}}{\pgfqpoint{1.882607in}{2.526217in}}{\pgfqpoint{1.882607in}{2.515167in}}%
\pgfpathcurveto{\pgfqpoint{1.882607in}{2.504117in}}{\pgfqpoint{1.886997in}{2.493518in}}{\pgfqpoint{1.894811in}{2.485704in}}%
\pgfpathcurveto{\pgfqpoint{1.902624in}{2.477891in}}{\pgfqpoint{1.913223in}{2.473500in}}{\pgfqpoint{1.924273in}{2.473500in}}%
\pgfpathclose%
\pgfusepath{stroke,fill}%
\end{pgfscope}%
\begin{pgfscope}%
\pgfpathrectangle{\pgfqpoint{0.772069in}{0.515123in}}{\pgfqpoint{3.875000in}{2.695000in}}%
\pgfusepath{clip}%
\pgfsetbuttcap%
\pgfsetroundjoin%
\definecolor{currentfill}{rgb}{1.000000,0.843137,0.000000}%
\pgfsetfillcolor{currentfill}%
\pgfsetlinewidth{1.003750pt}%
\definecolor{currentstroke}{rgb}{1.000000,0.843137,0.000000}%
\pgfsetstrokecolor{currentstroke}%
\pgfsetdash{}{0pt}%
\pgfpathmoveto{\pgfqpoint{2.051596in}{2.736751in}}%
\pgfpathcurveto{\pgfqpoint{2.062646in}{2.736751in}}{\pgfqpoint{2.073245in}{2.741142in}}{\pgfqpoint{2.081058in}{2.748955in}}%
\pgfpathcurveto{\pgfqpoint{2.088872in}{2.756769in}}{\pgfqpoint{2.093262in}{2.767368in}}{\pgfqpoint{2.093262in}{2.778418in}}%
\pgfpathcurveto{\pgfqpoint{2.093262in}{2.789468in}}{\pgfqpoint{2.088872in}{2.800067in}}{\pgfqpoint{2.081058in}{2.807881in}}%
\pgfpathcurveto{\pgfqpoint{2.073245in}{2.815694in}}{\pgfqpoint{2.062646in}{2.820085in}}{\pgfqpoint{2.051596in}{2.820085in}}%
\pgfpathcurveto{\pgfqpoint{2.040545in}{2.820085in}}{\pgfqpoint{2.029946in}{2.815694in}}{\pgfqpoint{2.022133in}{2.807881in}}%
\pgfpathcurveto{\pgfqpoint{2.014319in}{2.800067in}}{\pgfqpoint{2.009929in}{2.789468in}}{\pgfqpoint{2.009929in}{2.778418in}}%
\pgfpathcurveto{\pgfqpoint{2.009929in}{2.767368in}}{\pgfqpoint{2.014319in}{2.756769in}}{\pgfqpoint{2.022133in}{2.748955in}}%
\pgfpathcurveto{\pgfqpoint{2.029946in}{2.741142in}}{\pgfqpoint{2.040545in}{2.736751in}}{\pgfqpoint{2.051596in}{2.736751in}}%
\pgfpathclose%
\pgfusepath{stroke,fill}%
\end{pgfscope}%
\begin{pgfscope}%
\pgfpathrectangle{\pgfqpoint{0.772069in}{0.515123in}}{\pgfqpoint{3.875000in}{2.695000in}}%
\pgfusepath{clip}%
\pgfsetbuttcap%
\pgfsetroundjoin%
\definecolor{currentfill}{rgb}{1.000000,0.843137,0.000000}%
\pgfsetfillcolor{currentfill}%
\pgfsetlinewidth{1.003750pt}%
\definecolor{currentstroke}{rgb}{1.000000,0.843137,0.000000}%
\pgfsetstrokecolor{currentstroke}%
\pgfsetdash{}{0pt}%
\pgfpathmoveto{\pgfqpoint{2.184454in}{3.015957in}}%
\pgfpathcurveto{\pgfqpoint{2.195504in}{3.015957in}}{\pgfqpoint{2.206103in}{3.020347in}}{\pgfqpoint{2.213916in}{3.028161in}}%
\pgfpathcurveto{\pgfqpoint{2.221730in}{3.035975in}}{\pgfqpoint{2.226120in}{3.046574in}}{\pgfqpoint{2.226120in}{3.057624in}}%
\pgfpathcurveto{\pgfqpoint{2.226120in}{3.068674in}}{\pgfqpoint{2.221730in}{3.079273in}}{\pgfqpoint{2.213916in}{3.087087in}}%
\pgfpathcurveto{\pgfqpoint{2.206103in}{3.094900in}}{\pgfqpoint{2.195504in}{3.099290in}}{\pgfqpoint{2.184454in}{3.099290in}}%
\pgfpathcurveto{\pgfqpoint{2.173403in}{3.099290in}}{\pgfqpoint{2.162804in}{3.094900in}}{\pgfqpoint{2.154991in}{3.087087in}}%
\pgfpathcurveto{\pgfqpoint{2.147177in}{3.079273in}}{\pgfqpoint{2.142787in}{3.068674in}}{\pgfqpoint{2.142787in}{3.057624in}}%
\pgfpathcurveto{\pgfqpoint{2.142787in}{3.046574in}}{\pgfqpoint{2.147177in}{3.035975in}}{\pgfqpoint{2.154991in}{3.028161in}}%
\pgfpathcurveto{\pgfqpoint{2.162804in}{3.020347in}}{\pgfqpoint{2.173403in}{3.015957in}}{\pgfqpoint{2.184454in}{3.015957in}}%
\pgfpathclose%
\pgfusepath{stroke,fill}%
\end{pgfscope}%
\begin{pgfscope}%
\pgfpathrectangle{\pgfqpoint{0.772069in}{0.515123in}}{\pgfqpoint{3.875000in}{2.695000in}}%
\pgfusepath{clip}%
\pgfsetbuttcap%
\pgfsetroundjoin%
\definecolor{currentfill}{rgb}{0.196078,0.803922,0.196078}%
\pgfsetfillcolor{currentfill}%
\pgfsetlinewidth{1.003750pt}%
\definecolor{currentstroke}{rgb}{0.196078,0.803922,0.196078}%
\pgfsetstrokecolor{currentstroke}%
\pgfsetdash{}{0pt}%
\pgfpathmoveto{\pgfqpoint{0.977660in}{0.625956in}}%
\pgfpathcurveto{\pgfqpoint{0.988710in}{0.625956in}}{\pgfqpoint{0.999309in}{0.630347in}}{\pgfqpoint{1.007123in}{0.638160in}}%
\pgfpathcurveto{\pgfqpoint{1.014937in}{0.645974in}}{\pgfqpoint{1.019327in}{0.656573in}}{\pgfqpoint{1.019327in}{0.667623in}}%
\pgfpathcurveto{\pgfqpoint{1.019327in}{0.678673in}}{\pgfqpoint{1.014937in}{0.689272in}}{\pgfqpoint{1.007123in}{0.697086in}}%
\pgfpathcurveto{\pgfqpoint{0.999309in}{0.704899in}}{\pgfqpoint{0.988710in}{0.709290in}}{\pgfqpoint{0.977660in}{0.709290in}}%
\pgfpathcurveto{\pgfqpoint{0.966610in}{0.709290in}}{\pgfqpoint{0.956011in}{0.704899in}}{\pgfqpoint{0.948197in}{0.697086in}}%
\pgfpathcurveto{\pgfqpoint{0.940384in}{0.689272in}}{\pgfqpoint{0.935993in}{0.678673in}}{\pgfqpoint{0.935993in}{0.667623in}}%
\pgfpathcurveto{\pgfqpoint{0.935993in}{0.656573in}}{\pgfqpoint{0.940384in}{0.645974in}}{\pgfqpoint{0.948197in}{0.638160in}}%
\pgfpathcurveto{\pgfqpoint{0.956011in}{0.630347in}}{\pgfqpoint{0.966610in}{0.625956in}}{\pgfqpoint{0.977660in}{0.625956in}}%
\pgfpathclose%
\pgfusepath{stroke,fill}%
\end{pgfscope}%
\begin{pgfscope}%
\pgfpathrectangle{\pgfqpoint{0.772069in}{0.515123in}}{\pgfqpoint{3.875000in}{2.695000in}}%
\pgfusepath{clip}%
\pgfsetbuttcap%
\pgfsetroundjoin%
\definecolor{currentfill}{rgb}{0.196078,0.803922,0.196078}%
\pgfsetfillcolor{currentfill}%
\pgfsetlinewidth{1.003750pt}%
\definecolor{currentstroke}{rgb}{0.196078,0.803922,0.196078}%
\pgfsetstrokecolor{currentstroke}%
\pgfsetdash{}{0pt}%
\pgfpathmoveto{\pgfqpoint{1.044089in}{0.875380in}}%
\pgfpathcurveto{\pgfqpoint{1.055139in}{0.875380in}}{\pgfqpoint{1.065738in}{0.879770in}}{\pgfqpoint{1.073552in}{0.887584in}}%
\pgfpathcurveto{\pgfqpoint{1.081366in}{0.895398in}}{\pgfqpoint{1.085756in}{0.905997in}}{\pgfqpoint{1.085756in}{0.917047in}}%
\pgfpathcurveto{\pgfqpoint{1.085756in}{0.928097in}}{\pgfqpoint{1.081366in}{0.938696in}}{\pgfqpoint{1.073552in}{0.946509in}}%
\pgfpathcurveto{\pgfqpoint{1.065738in}{0.954323in}}{\pgfqpoint{1.055139in}{0.958713in}}{\pgfqpoint{1.044089in}{0.958713in}}%
\pgfpathcurveto{\pgfqpoint{1.033039in}{0.958713in}}{\pgfqpoint{1.022440in}{0.954323in}}{\pgfqpoint{1.014626in}{0.946509in}}%
\pgfpathcurveto{\pgfqpoint{1.006813in}{0.938696in}}{\pgfqpoint{1.002422in}{0.928097in}}{\pgfqpoint{1.002422in}{0.917047in}}%
\pgfpathcurveto{\pgfqpoint{1.002422in}{0.905997in}}{\pgfqpoint{1.006813in}{0.895398in}}{\pgfqpoint{1.014626in}{0.887584in}}%
\pgfpathcurveto{\pgfqpoint{1.022440in}{0.879770in}}{\pgfqpoint{1.033039in}{0.875380in}}{\pgfqpoint{1.044089in}{0.875380in}}%
\pgfpathclose%
\pgfusepath{stroke,fill}%
\end{pgfscope}%
\begin{pgfscope}%
\pgfpathrectangle{\pgfqpoint{0.772069in}{0.515123in}}{\pgfqpoint{3.875000in}{2.695000in}}%
\pgfusepath{clip}%
\pgfsetbuttcap%
\pgfsetroundjoin%
\definecolor{currentfill}{rgb}{0.196078,0.803922,0.196078}%
\pgfsetfillcolor{currentfill}%
\pgfsetlinewidth{1.003750pt}%
\definecolor{currentstroke}{rgb}{0.196078,0.803922,0.196078}%
\pgfsetstrokecolor{currentstroke}%
\pgfsetdash{}{0pt}%
\pgfpathmoveto{\pgfqpoint{1.104982in}{1.125336in}}%
\pgfpathcurveto{\pgfqpoint{1.116032in}{1.125336in}}{\pgfqpoint{1.126632in}{1.129726in}}{\pgfqpoint{1.134445in}{1.137539in}}%
\pgfpathcurveto{\pgfqpoint{1.142259in}{1.145353in}}{\pgfqpoint{1.146649in}{1.155952in}}{\pgfqpoint{1.146649in}{1.167002in}}%
\pgfpathcurveto{\pgfqpoint{1.146649in}{1.178052in}}{\pgfqpoint{1.142259in}{1.188651in}}{\pgfqpoint{1.134445in}{1.196465in}}%
\pgfpathcurveto{\pgfqpoint{1.126632in}{1.204279in}}{\pgfqpoint{1.116032in}{1.208669in}}{\pgfqpoint{1.104982in}{1.208669in}}%
\pgfpathcurveto{\pgfqpoint{1.093932in}{1.208669in}}{\pgfqpoint{1.083333in}{1.204279in}}{\pgfqpoint{1.075520in}{1.196465in}}%
\pgfpathcurveto{\pgfqpoint{1.067706in}{1.188651in}}{\pgfqpoint{1.063316in}{1.178052in}}{\pgfqpoint{1.063316in}{1.167002in}}%
\pgfpathcurveto{\pgfqpoint{1.063316in}{1.155952in}}{\pgfqpoint{1.067706in}{1.145353in}}{\pgfqpoint{1.075520in}{1.137539in}}%
\pgfpathcurveto{\pgfqpoint{1.083333in}{1.129726in}}{\pgfqpoint{1.093932in}{1.125336in}}{\pgfqpoint{1.104982in}{1.125336in}}%
\pgfpathclose%
\pgfusepath{stroke,fill}%
\end{pgfscope}%
\begin{pgfscope}%
\pgfpathrectangle{\pgfqpoint{0.772069in}{0.515123in}}{\pgfqpoint{3.875000in}{2.695000in}}%
\pgfusepath{clip}%
\pgfsetbuttcap%
\pgfsetroundjoin%
\definecolor{currentfill}{rgb}{0.196078,0.803922,0.196078}%
\pgfsetfillcolor{currentfill}%
\pgfsetlinewidth{1.003750pt}%
\definecolor{currentstroke}{rgb}{0.196078,0.803922,0.196078}%
\pgfsetstrokecolor{currentstroke}%
\pgfsetdash{}{0pt}%
\pgfpathmoveto{\pgfqpoint{1.182483in}{1.399223in}}%
\pgfpathcurveto{\pgfqpoint{1.193533in}{1.399223in}}{\pgfqpoint{1.204132in}{1.403613in}}{\pgfqpoint{1.211946in}{1.411427in}}%
\pgfpathcurveto{\pgfqpoint{1.219759in}{1.419241in}}{\pgfqpoint{1.224150in}{1.429840in}}{\pgfqpoint{1.224150in}{1.440890in}}%
\pgfpathcurveto{\pgfqpoint{1.224150in}{1.451940in}}{\pgfqpoint{1.219759in}{1.462539in}}{\pgfqpoint{1.211946in}{1.470353in}}%
\pgfpathcurveto{\pgfqpoint{1.204132in}{1.478166in}}{\pgfqpoint{1.193533in}{1.482556in}}{\pgfqpoint{1.182483in}{1.482556in}}%
\pgfpathcurveto{\pgfqpoint{1.171433in}{1.482556in}}{\pgfqpoint{1.160834in}{1.478166in}}{\pgfqpoint{1.153020in}{1.470353in}}%
\pgfpathcurveto{\pgfqpoint{1.145206in}{1.462539in}}{\pgfqpoint{1.140816in}{1.451940in}}{\pgfqpoint{1.140816in}{1.440890in}}%
\pgfpathcurveto{\pgfqpoint{1.140816in}{1.429840in}}{\pgfqpoint{1.145206in}{1.419241in}}{\pgfqpoint{1.153020in}{1.411427in}}%
\pgfpathcurveto{\pgfqpoint{1.160834in}{1.403613in}}{\pgfqpoint{1.171433in}{1.399223in}}{\pgfqpoint{1.182483in}{1.399223in}}%
\pgfpathclose%
\pgfusepath{stroke,fill}%
\end{pgfscope}%
\begin{pgfscope}%
\pgfpathrectangle{\pgfqpoint{0.772069in}{0.515123in}}{\pgfqpoint{3.875000in}{2.695000in}}%
\pgfusepath{clip}%
\pgfsetbuttcap%
\pgfsetroundjoin%
\definecolor{currentfill}{rgb}{0.196078,0.803922,0.196078}%
\pgfsetfillcolor{currentfill}%
\pgfsetlinewidth{1.003750pt}%
\definecolor{currentstroke}{rgb}{0.196078,0.803922,0.196078}%
\pgfsetstrokecolor{currentstroke}%
\pgfsetdash{}{0pt}%
\pgfpathmoveto{\pgfqpoint{1.251680in}{1.670451in}}%
\pgfpathcurveto{\pgfqpoint{1.262730in}{1.670451in}}{\pgfqpoint{1.273329in}{1.674842in}}{\pgfqpoint{1.281143in}{1.682655in}}%
\pgfpathcurveto{\pgfqpoint{1.288956in}{1.690469in}}{\pgfqpoint{1.293346in}{1.701068in}}{\pgfqpoint{1.293346in}{1.712118in}}%
\pgfpathcurveto{\pgfqpoint{1.293346in}{1.723168in}}{\pgfqpoint{1.288956in}{1.733767in}}{\pgfqpoint{1.281143in}{1.741581in}}%
\pgfpathcurveto{\pgfqpoint{1.273329in}{1.749395in}}{\pgfqpoint{1.262730in}{1.753785in}}{\pgfqpoint{1.251680in}{1.753785in}}%
\pgfpathcurveto{\pgfqpoint{1.240630in}{1.753785in}}{\pgfqpoint{1.230031in}{1.749395in}}{\pgfqpoint{1.222217in}{1.741581in}}%
\pgfpathcurveto{\pgfqpoint{1.214403in}{1.733767in}}{\pgfqpoint{1.210013in}{1.723168in}}{\pgfqpoint{1.210013in}{1.712118in}}%
\pgfpathcurveto{\pgfqpoint{1.210013in}{1.701068in}}{\pgfqpoint{1.214403in}{1.690469in}}{\pgfqpoint{1.222217in}{1.682655in}}%
\pgfpathcurveto{\pgfqpoint{1.230031in}{1.674842in}}{\pgfqpoint{1.240630in}{1.670451in}}{\pgfqpoint{1.251680in}{1.670451in}}%
\pgfpathclose%
\pgfusepath{stroke,fill}%
\end{pgfscope}%
\begin{pgfscope}%
\pgfpathrectangle{\pgfqpoint{0.772069in}{0.515123in}}{\pgfqpoint{3.875000in}{2.695000in}}%
\pgfusepath{clip}%
\pgfsetbuttcap%
\pgfsetroundjoin%
\definecolor{currentfill}{rgb}{0.196078,0.803922,0.196078}%
\pgfsetfillcolor{currentfill}%
\pgfsetlinewidth{1.003750pt}%
\definecolor{currentstroke}{rgb}{0.196078,0.803922,0.196078}%
\pgfsetstrokecolor{currentstroke}%
\pgfsetdash{}{0pt}%
\pgfpathmoveto{\pgfqpoint{1.323644in}{1.939021in}}%
\pgfpathcurveto{\pgfqpoint{1.334695in}{1.939021in}}{\pgfqpoint{1.345294in}{1.943411in}}{\pgfqpoint{1.353107in}{1.951225in}}%
\pgfpathcurveto{\pgfqpoint{1.360921in}{1.959038in}}{\pgfqpoint{1.365311in}{1.969637in}}{\pgfqpoint{1.365311in}{1.980687in}}%
\pgfpathcurveto{\pgfqpoint{1.365311in}{1.991738in}}{\pgfqpoint{1.360921in}{2.002337in}}{\pgfqpoint{1.353107in}{2.010150in}}%
\pgfpathcurveto{\pgfqpoint{1.345294in}{2.017964in}}{\pgfqpoint{1.334695in}{2.022354in}}{\pgfqpoint{1.323644in}{2.022354in}}%
\pgfpathcurveto{\pgfqpoint{1.312594in}{2.022354in}}{\pgfqpoint{1.301995in}{2.017964in}}{\pgfqpoint{1.294182in}{2.010150in}}%
\pgfpathcurveto{\pgfqpoint{1.286368in}{2.002337in}}{\pgfqpoint{1.281978in}{1.991738in}}{\pgfqpoint{1.281978in}{1.980687in}}%
\pgfpathcurveto{\pgfqpoint{1.281978in}{1.969637in}}{\pgfqpoint{1.286368in}{1.959038in}}{\pgfqpoint{1.294182in}{1.951225in}}%
\pgfpathcurveto{\pgfqpoint{1.301995in}{1.943411in}}{\pgfqpoint{1.312594in}{1.939021in}}{\pgfqpoint{1.323644in}{1.939021in}}%
\pgfpathclose%
\pgfusepath{stroke,fill}%
\end{pgfscope}%
\begin{pgfscope}%
\pgfpathrectangle{\pgfqpoint{0.772069in}{0.515123in}}{\pgfqpoint{3.875000in}{2.695000in}}%
\pgfusepath{clip}%
\pgfsetbuttcap%
\pgfsetroundjoin%
\definecolor{currentfill}{rgb}{0.196078,0.803922,0.196078}%
\pgfsetfillcolor{currentfill}%
\pgfsetlinewidth{1.003750pt}%
\definecolor{currentstroke}{rgb}{0.196078,0.803922,0.196078}%
\pgfsetstrokecolor{currentstroke}%
\pgfsetdash{}{0pt}%
\pgfpathmoveto{\pgfqpoint{1.395609in}{2.210249in}}%
\pgfpathcurveto{\pgfqpoint{1.406659in}{2.210249in}}{\pgfqpoint{1.417258in}{2.214639in}}{\pgfqpoint{1.425072in}{2.222453in}}%
\pgfpathcurveto{\pgfqpoint{1.432886in}{2.230267in}}{\pgfqpoint{1.437276in}{2.240866in}}{\pgfqpoint{1.437276in}{2.251916in}}%
\pgfpathcurveto{\pgfqpoint{1.437276in}{2.262966in}}{\pgfqpoint{1.432886in}{2.273565in}}{\pgfqpoint{1.425072in}{2.281379in}}%
\pgfpathcurveto{\pgfqpoint{1.417258in}{2.289192in}}{\pgfqpoint{1.406659in}{2.293583in}}{\pgfqpoint{1.395609in}{2.293583in}}%
\pgfpathcurveto{\pgfqpoint{1.384559in}{2.293583in}}{\pgfqpoint{1.373960in}{2.289192in}}{\pgfqpoint{1.366146in}{2.281379in}}%
\pgfpathcurveto{\pgfqpoint{1.358333in}{2.273565in}}{\pgfqpoint{1.353943in}{2.262966in}}{\pgfqpoint{1.353943in}{2.251916in}}%
\pgfpathcurveto{\pgfqpoint{1.353943in}{2.240866in}}{\pgfqpoint{1.358333in}{2.230267in}}{\pgfqpoint{1.366146in}{2.222453in}}%
\pgfpathcurveto{\pgfqpoint{1.373960in}{2.214639in}}{\pgfqpoint{1.384559in}{2.210249in}}{\pgfqpoint{1.395609in}{2.210249in}}%
\pgfpathclose%
\pgfusepath{stroke,fill}%
\end{pgfscope}%
\begin{pgfscope}%
\pgfpathrectangle{\pgfqpoint{0.772069in}{0.515123in}}{\pgfqpoint{3.875000in}{2.695000in}}%
\pgfusepath{clip}%
\pgfsetbuttcap%
\pgfsetroundjoin%
\definecolor{currentfill}{rgb}{0.196078,0.803922,0.196078}%
\pgfsetfillcolor{currentfill}%
\pgfsetlinewidth{1.003750pt}%
\definecolor{currentstroke}{rgb}{0.196078,0.803922,0.196078}%
\pgfsetstrokecolor{currentstroke}%
\pgfsetdash{}{0pt}%
\pgfpathmoveto{\pgfqpoint{1.464806in}{2.473500in}}%
\pgfpathcurveto{\pgfqpoint{1.475856in}{2.473500in}}{\pgfqpoint{1.486455in}{2.477891in}}{\pgfqpoint{1.494269in}{2.485704in}}%
\pgfpathcurveto{\pgfqpoint{1.502082in}{2.493518in}}{\pgfqpoint{1.506473in}{2.504117in}}{\pgfqpoint{1.506473in}{2.515167in}}%
\pgfpathcurveto{\pgfqpoint{1.506473in}{2.526217in}}{\pgfqpoint{1.502082in}{2.536816in}}{\pgfqpoint{1.494269in}{2.544630in}}%
\pgfpathcurveto{\pgfqpoint{1.486455in}{2.552443in}}{\pgfqpoint{1.475856in}{2.556834in}}{\pgfqpoint{1.464806in}{2.556834in}}%
\pgfpathcurveto{\pgfqpoint{1.453756in}{2.556834in}}{\pgfqpoint{1.443157in}{2.552443in}}{\pgfqpoint{1.435343in}{2.544630in}}%
\pgfpathcurveto{\pgfqpoint{1.427530in}{2.536816in}}{\pgfqpoint{1.423139in}{2.526217in}}{\pgfqpoint{1.423139in}{2.515167in}}%
\pgfpathcurveto{\pgfqpoint{1.423139in}{2.504117in}}{\pgfqpoint{1.427530in}{2.493518in}}{\pgfqpoint{1.435343in}{2.485704in}}%
\pgfpathcurveto{\pgfqpoint{1.443157in}{2.477891in}}{\pgfqpoint{1.453756in}{2.473500in}}{\pgfqpoint{1.464806in}{2.473500in}}%
\pgfpathclose%
\pgfusepath{stroke,fill}%
\end{pgfscope}%
\begin{pgfscope}%
\pgfpathrectangle{\pgfqpoint{0.772069in}{0.515123in}}{\pgfqpoint{3.875000in}{2.695000in}}%
\pgfusepath{clip}%
\pgfsetbuttcap%
\pgfsetroundjoin%
\definecolor{currentfill}{rgb}{0.196078,0.803922,0.196078}%
\pgfsetfillcolor{currentfill}%
\pgfsetlinewidth{1.003750pt}%
\definecolor{currentstroke}{rgb}{0.196078,0.803922,0.196078}%
\pgfsetstrokecolor{currentstroke}%
\pgfsetdash{}{0pt}%
\pgfpathmoveto{\pgfqpoint{1.534003in}{2.736751in}}%
\pgfpathcurveto{\pgfqpoint{1.545053in}{2.736751in}}{\pgfqpoint{1.555652in}{2.741142in}}{\pgfqpoint{1.563466in}{2.748955in}}%
\pgfpathcurveto{\pgfqpoint{1.571279in}{2.756769in}}{\pgfqpoint{1.575670in}{2.767368in}}{\pgfqpoint{1.575670in}{2.778418in}}%
\pgfpathcurveto{\pgfqpoint{1.575670in}{2.789468in}}{\pgfqpoint{1.571279in}{2.800067in}}{\pgfqpoint{1.563466in}{2.807881in}}%
\pgfpathcurveto{\pgfqpoint{1.555652in}{2.815694in}}{\pgfqpoint{1.545053in}{2.820085in}}{\pgfqpoint{1.534003in}{2.820085in}}%
\pgfpathcurveto{\pgfqpoint{1.522953in}{2.820085in}}{\pgfqpoint{1.512354in}{2.815694in}}{\pgfqpoint{1.504540in}{2.807881in}}%
\pgfpathcurveto{\pgfqpoint{1.496727in}{2.800067in}}{\pgfqpoint{1.492336in}{2.789468in}}{\pgfqpoint{1.492336in}{2.778418in}}%
\pgfpathcurveto{\pgfqpoint{1.492336in}{2.767368in}}{\pgfqpoint{1.496727in}{2.756769in}}{\pgfqpoint{1.504540in}{2.748955in}}%
\pgfpathcurveto{\pgfqpoint{1.512354in}{2.741142in}}{\pgfqpoint{1.522953in}{2.736751in}}{\pgfqpoint{1.534003in}{2.736751in}}%
\pgfpathclose%
\pgfusepath{stroke,fill}%
\end{pgfscope}%
\begin{pgfscope}%
\pgfpathrectangle{\pgfqpoint{0.772069in}{0.515123in}}{\pgfqpoint{3.875000in}{2.695000in}}%
\pgfusepath{clip}%
\pgfsetbuttcap%
\pgfsetroundjoin%
\definecolor{currentfill}{rgb}{0.196078,0.803922,0.196078}%
\pgfsetfillcolor{currentfill}%
\pgfsetlinewidth{1.003750pt}%
\definecolor{currentstroke}{rgb}{0.196078,0.803922,0.196078}%
\pgfsetstrokecolor{currentstroke}%
\pgfsetdash{}{0pt}%
\pgfpathmoveto{\pgfqpoint{1.608736in}{3.015957in}}%
\pgfpathcurveto{\pgfqpoint{1.619786in}{3.015957in}}{\pgfqpoint{1.630385in}{3.020347in}}{\pgfqpoint{1.638198in}{3.028161in}}%
\pgfpathcurveto{\pgfqpoint{1.646012in}{3.035975in}}{\pgfqpoint{1.650402in}{3.046574in}}{\pgfqpoint{1.650402in}{3.057624in}}%
\pgfpathcurveto{\pgfqpoint{1.650402in}{3.068674in}}{\pgfqpoint{1.646012in}{3.079273in}}{\pgfqpoint{1.638198in}{3.087087in}}%
\pgfpathcurveto{\pgfqpoint{1.630385in}{3.094900in}}{\pgfqpoint{1.619786in}{3.099290in}}{\pgfqpoint{1.608736in}{3.099290in}}%
\pgfpathcurveto{\pgfqpoint{1.597685in}{3.099290in}}{\pgfqpoint{1.587086in}{3.094900in}}{\pgfqpoint{1.579273in}{3.087087in}}%
\pgfpathcurveto{\pgfqpoint{1.571459in}{3.079273in}}{\pgfqpoint{1.567069in}{3.068674in}}{\pgfqpoint{1.567069in}{3.057624in}}%
\pgfpathcurveto{\pgfqpoint{1.567069in}{3.046574in}}{\pgfqpoint{1.571459in}{3.035975in}}{\pgfqpoint{1.579273in}{3.028161in}}%
\pgfpathcurveto{\pgfqpoint{1.587086in}{3.020347in}}{\pgfqpoint{1.597685in}{3.015957in}}{\pgfqpoint{1.608736in}{3.015957in}}%
\pgfpathclose%
\pgfusepath{stroke,fill}%
\end{pgfscope}%
\begin{pgfscope}%
\pgfsetrectcap%
\pgfsetmiterjoin%
\pgfsetlinewidth{0.803000pt}%
\definecolor{currentstroke}{rgb}{0.000000,0.000000,0.000000}%
\pgfsetstrokecolor{currentstroke}%
\pgfsetdash{}{0pt}%
\pgfpathmoveto{\pgfqpoint{0.772069in}{0.515123in}}%
\pgfpathlineto{\pgfqpoint{0.772069in}{3.210123in}}%
\pgfusepath{stroke}%
\end{pgfscope}%
\begin{pgfscope}%
\pgfsetrectcap%
\pgfsetmiterjoin%
\pgfsetlinewidth{0.803000pt}%
\definecolor{currentstroke}{rgb}{0.000000,0.000000,0.000000}%
\pgfsetstrokecolor{currentstroke}%
\pgfsetdash{}{0pt}%
\pgfpathmoveto{\pgfqpoint{4.647069in}{0.515123in}}%
\pgfpathlineto{\pgfqpoint{4.647069in}{3.210123in}}%
\pgfusepath{stroke}%
\end{pgfscope}%
\begin{pgfscope}%
\pgfsetrectcap%
\pgfsetmiterjoin%
\pgfsetlinewidth{0.803000pt}%
\definecolor{currentstroke}{rgb}{0.000000,0.000000,0.000000}%
\pgfsetstrokecolor{currentstroke}%
\pgfsetdash{}{0pt}%
\pgfpathmoveto{\pgfqpoint{0.772069in}{0.515123in}}%
\pgfpathlineto{\pgfqpoint{4.647069in}{0.515123in}}%
\pgfusepath{stroke}%
\end{pgfscope}%
\begin{pgfscope}%
\pgfsetrectcap%
\pgfsetmiterjoin%
\pgfsetlinewidth{0.803000pt}%
\definecolor{currentstroke}{rgb}{0.000000,0.000000,0.000000}%
\pgfsetstrokecolor{currentstroke}%
\pgfsetdash{}{0pt}%
\pgfpathmoveto{\pgfqpoint{0.772069in}{3.210123in}}%
\pgfpathlineto{\pgfqpoint{4.647069in}{3.210123in}}%
\pgfusepath{stroke}%
\end{pgfscope}%
\end{pgfpicture}%
\makeatother%
\endgroup%

    \caption{Voltajes (V) frente a intensidades (\textcolor{Blue}{$I$}, \textcolor{Red}{$I_1$}, \textcolor{Yellow}{$I_2$}, \textcolor{Green}{$I_3$})}
  \end{figure}

  \subsection{Ajuste por mínimos cuadrados}
  \label{v:paralelo}

  De nuevo parece que nos encontramos ante una relación lineal de las magnitudes, por lo que aplicaremos los métodos que ya describimos (\ref{sec:reglin}) para ajustar las rectas y representarlas gráficamente. Para la intensidad total (I), los parámetros son $a = -2,84 \cdot 10^-3$ y $b~=~78300$. El resto se pueden calcular análogamente con las fórmulas \ref{ec:a} y \ref{ec:b}.

  \begin{figure}[H]
    %\centering
    \hspace{2.5em} %% Creator: Matplotlib, PGF backend
%%
%% To include the figure in your LaTeX document, write
%%   \input{<filename>.pgf}
%%
%% Make sure the required packages are loaded in your preamble
%%   \usepackage{pgf}
%%
%% Figures using additional raster images can only be included by \input if
%% they are in the same directory as the main LaTeX file. For loading figures
%% from other directories you can use the `import` package
%%   \usepackage{import}
%% and then include the figures with
%%   \import{<path to file>}{<filename>.pgf}
%%
%% Matplotlib used the following preamble
%%
\begingroup%
\makeatletter%
\begin{pgfpicture}%
\pgfpathrectangle{\pgfpointorigin}{\pgfqpoint{4.747069in}{3.310123in}}%
\pgfusepath{use as bounding box, clip}%
\begin{pgfscope}%
\pgfsetbuttcap%
\pgfsetmiterjoin%
\definecolor{currentfill}{rgb}{1.000000,1.000000,1.000000}%
\pgfsetfillcolor{currentfill}%
\pgfsetlinewidth{0.000000pt}%
\definecolor{currentstroke}{rgb}{1.000000,1.000000,1.000000}%
\pgfsetstrokecolor{currentstroke}%
\pgfsetdash{}{0pt}%
\pgfpathmoveto{\pgfqpoint{0.000000in}{0.000000in}}%
\pgfpathlineto{\pgfqpoint{4.747069in}{0.000000in}}%
\pgfpathlineto{\pgfqpoint{4.747069in}{3.310123in}}%
\pgfpathlineto{\pgfqpoint{0.000000in}{3.310123in}}%
\pgfpathclose%
\pgfusepath{fill}%
\end{pgfscope}%
\begin{pgfscope}%
\pgfsetbuttcap%
\pgfsetmiterjoin%
\definecolor{currentfill}{rgb}{1.000000,1.000000,1.000000}%
\pgfsetfillcolor{currentfill}%
\pgfsetlinewidth{0.000000pt}%
\definecolor{currentstroke}{rgb}{0.000000,0.000000,0.000000}%
\pgfsetstrokecolor{currentstroke}%
\pgfsetstrokeopacity{0.000000}%
\pgfsetdash{}{0pt}%
\pgfpathmoveto{\pgfqpoint{0.772069in}{0.515123in}}%
\pgfpathlineto{\pgfqpoint{4.647069in}{0.515123in}}%
\pgfpathlineto{\pgfqpoint{4.647069in}{3.210123in}}%
\pgfpathlineto{\pgfqpoint{0.772069in}{3.210123in}}%
\pgfpathclose%
\pgfusepath{fill}%
\end{pgfscope}%
\begin{pgfscope}%
\pgfpathrectangle{\pgfqpoint{0.772069in}{0.515123in}}{\pgfqpoint{3.875000in}{2.695000in}}%
\pgfusepath{clip}%
\pgfsetrectcap%
\pgfsetroundjoin%
\pgfsetlinewidth{1.505625pt}%
\definecolor{currentstroke}{rgb}{0.529412,0.807843,0.921569}%
\pgfsetstrokecolor{currentstroke}%
\pgfsetdash{}{0pt}%
\pgfpathmoveto{\pgfqpoint{1.299303in}{0.668669in}}%
\pgfpathlineto{\pgfqpoint{1.648730in}{0.933691in}}%
\pgfpathlineto{\pgfqpoint{1.998158in}{1.198713in}}%
\pgfpathlineto{\pgfqpoint{2.347585in}{1.463736in}}%
\pgfpathlineto{\pgfqpoint{2.697013in}{1.728758in}}%
\pgfpathlineto{\pgfqpoint{3.046440in}{1.993780in}}%
\pgfpathlineto{\pgfqpoint{3.395868in}{2.258802in}}%
\pgfpathlineto{\pgfqpoint{3.745295in}{2.523824in}}%
\pgfpathlineto{\pgfqpoint{4.094723in}{2.788847in}}%
\pgfpathlineto{\pgfqpoint{4.444150in}{3.053869in}}%
\pgfusepath{stroke}%
\end{pgfscope}%
\begin{pgfscope}%
\pgfpathrectangle{\pgfqpoint{0.772069in}{0.515123in}}{\pgfqpoint{3.875000in}{2.695000in}}%
\pgfusepath{clip}%
\pgfsetrectcap%
\pgfsetroundjoin%
\pgfsetlinewidth{1.505625pt}%
\definecolor{currentstroke}{rgb}{1.000000,0.627451,0.478431}%
\pgfsetstrokecolor{currentstroke}%
\pgfsetdash{}{0pt}%
\pgfpathmoveto{\pgfqpoint{1.076829in}{0.665774in}}%
\pgfpathlineto{\pgfqpoint{1.232774in}{0.931176in}}%
\pgfpathlineto{\pgfqpoint{1.388720in}{1.196578in}}%
\pgfpathlineto{\pgfqpoint{1.544665in}{1.461979in}}%
\pgfpathlineto{\pgfqpoint{1.700611in}{1.727381in}}%
\pgfpathlineto{\pgfqpoint{1.856556in}{1.992783in}}%
\pgfpathlineto{\pgfqpoint{2.012501in}{2.258184in}}%
\pgfpathlineto{\pgfqpoint{2.168447in}{2.523586in}}%
\pgfpathlineto{\pgfqpoint{2.324392in}{2.788988in}}%
\pgfpathlineto{\pgfqpoint{2.480337in}{3.054389in}}%
\pgfusepath{stroke}%
\end{pgfscope}%
\begin{pgfscope}%
\pgfpathrectangle{\pgfqpoint{0.772069in}{0.515123in}}{\pgfqpoint{3.875000in}{2.695000in}}%
\pgfusepath{clip}%
\pgfsetrectcap%
\pgfsetroundjoin%
\pgfsetlinewidth{1.505625pt}%
\definecolor{currentstroke}{rgb}{1.000000,0.894118,0.709804}%
\pgfsetstrokecolor{currentstroke}%
\pgfsetdash{}{0pt}%
\pgfpathmoveto{\pgfqpoint{1.041123in}{0.661677in}}%
\pgfpathlineto{\pgfqpoint{1.167161in}{0.927171in}}%
\pgfpathlineto{\pgfqpoint{1.293199in}{1.192665in}}%
\pgfpathlineto{\pgfqpoint{1.419237in}{1.458158in}}%
\pgfpathlineto{\pgfqpoint{1.545275in}{1.723652in}}%
\pgfpathlineto{\pgfqpoint{1.671314in}{1.989145in}}%
\pgfpathlineto{\pgfqpoint{1.797352in}{2.254639in}}%
\pgfpathlineto{\pgfqpoint{1.923390in}{2.520132in}}%
\pgfpathlineto{\pgfqpoint{2.049428in}{2.785626in}}%
\pgfpathlineto{\pgfqpoint{2.175466in}{3.051120in}}%
\pgfusepath{stroke}%
\end{pgfscope}%
\begin{pgfscope}%
\pgfpathrectangle{\pgfqpoint{0.772069in}{0.515123in}}{\pgfqpoint{3.875000in}{2.695000in}}%
\pgfusepath{clip}%
\pgfsetrectcap%
\pgfsetroundjoin%
\pgfsetlinewidth{1.505625pt}%
\definecolor{currentstroke}{rgb}{0.564706,0.933333,0.564706}%
\pgfsetstrokecolor{currentstroke}%
\pgfsetdash{}{0pt}%
\pgfpathmoveto{\pgfqpoint{0.977952in}{0.659938in}}%
\pgfpathlineto{\pgfqpoint{1.047532in}{0.925327in}}%
\pgfpathlineto{\pgfqpoint{1.117112in}{1.190716in}}%
\pgfpathlineto{\pgfqpoint{1.186693in}{1.456105in}}%
\pgfpathlineto{\pgfqpoint{1.256273in}{1.721494in}}%
\pgfpathlineto{\pgfqpoint{1.325853in}{1.986883in}}%
\pgfpathlineto{\pgfqpoint{1.395434in}{2.252272in}}%
\pgfpathlineto{\pgfqpoint{1.465014in}{2.517661in}}%
\pgfpathlineto{\pgfqpoint{1.534594in}{2.783050in}}%
\pgfpathlineto{\pgfqpoint{1.604175in}{3.048439in}}%
\pgfusepath{stroke}%
\end{pgfscope}%
\begin{pgfscope}%
\pgfsetbuttcap%
\pgfsetroundjoin%
\definecolor{currentfill}{rgb}{0.000000,0.000000,0.000000}%
\pgfsetfillcolor{currentfill}%
\pgfsetlinewidth{0.803000pt}%
\definecolor{currentstroke}{rgb}{0.000000,0.000000,0.000000}%
\pgfsetstrokecolor{currentstroke}%
\pgfsetdash{}{0pt}%
\pgfsys@defobject{currentmarker}{\pgfqpoint{0.000000in}{-0.048611in}}{\pgfqpoint{0.000000in}{0.000000in}}{%
\pgfpathmoveto{\pgfqpoint{0.000000in}{0.000000in}}%
\pgfpathlineto{\pgfqpoint{0.000000in}{-0.048611in}}%
\pgfusepath{stroke,fill}%
}%
\begin{pgfscope}%
\pgfsys@transformshift{0.901047in}{0.515123in}%
\pgfsys@useobject{currentmarker}{}%
\end{pgfscope}%
\end{pgfscope}%
\begin{pgfscope}%
\definecolor{textcolor}{rgb}{0.000000,0.000000,0.000000}%
\pgfsetstrokecolor{textcolor}%
\pgfsetfillcolor{textcolor}%
\pgftext[x=0.901047in,y=0.417901in,,top]{\color{textcolor}\rmfamily\fontsize{10.000000}{12.000000}\selectfont \(\displaystyle 0\)}%
\end{pgfscope}%
\begin{pgfscope}%
\pgfsetbuttcap%
\pgfsetroundjoin%
\definecolor{currentfill}{rgb}{0.000000,0.000000,0.000000}%
\pgfsetfillcolor{currentfill}%
\pgfsetlinewidth{0.803000pt}%
\definecolor{currentstroke}{rgb}{0.000000,0.000000,0.000000}%
\pgfsetstrokecolor{currentstroke}%
\pgfsetdash{}{0pt}%
\pgfsys@defobject{currentmarker}{\pgfqpoint{0.000000in}{-0.048611in}}{\pgfqpoint{0.000000in}{0.000000in}}{%
\pgfpathmoveto{\pgfqpoint{0.000000in}{0.000000in}}%
\pgfpathlineto{\pgfqpoint{0.000000in}{-0.048611in}}%
\pgfusepath{stroke,fill}%
}%
\begin{pgfscope}%
\pgfsys@transformshift{1.450365in}{0.515123in}%
\pgfsys@useobject{currentmarker}{}%
\end{pgfscope}%
\end{pgfscope}%
\begin{pgfscope}%
\definecolor{textcolor}{rgb}{0.000000,0.000000,0.000000}%
\pgfsetstrokecolor{textcolor}%
\pgfsetfillcolor{textcolor}%
\pgftext[x=1.450365in,y=0.417901in,,top]{\color{textcolor}\rmfamily\fontsize{10.000000}{12.000000}\selectfont \(\displaystyle 20\)}%
\end{pgfscope}%
\begin{pgfscope}%
\pgfsetbuttcap%
\pgfsetroundjoin%
\definecolor{currentfill}{rgb}{0.000000,0.000000,0.000000}%
\pgfsetfillcolor{currentfill}%
\pgfsetlinewidth{0.803000pt}%
\definecolor{currentstroke}{rgb}{0.000000,0.000000,0.000000}%
\pgfsetstrokecolor{currentstroke}%
\pgfsetdash{}{0pt}%
\pgfsys@defobject{currentmarker}{\pgfqpoint{0.000000in}{-0.048611in}}{\pgfqpoint{0.000000in}{0.000000in}}{%
\pgfpathmoveto{\pgfqpoint{0.000000in}{0.000000in}}%
\pgfpathlineto{\pgfqpoint{0.000000in}{-0.048611in}}%
\pgfusepath{stroke,fill}%
}%
\begin{pgfscope}%
\pgfsys@transformshift{1.999684in}{0.515123in}%
\pgfsys@useobject{currentmarker}{}%
\end{pgfscope}%
\end{pgfscope}%
\begin{pgfscope}%
\definecolor{textcolor}{rgb}{0.000000,0.000000,0.000000}%
\pgfsetstrokecolor{textcolor}%
\pgfsetfillcolor{textcolor}%
\pgftext[x=1.999684in,y=0.417901in,,top]{\color{textcolor}\rmfamily\fontsize{10.000000}{12.000000}\selectfont \(\displaystyle 40\)}%
\end{pgfscope}%
\begin{pgfscope}%
\pgfsetbuttcap%
\pgfsetroundjoin%
\definecolor{currentfill}{rgb}{0.000000,0.000000,0.000000}%
\pgfsetfillcolor{currentfill}%
\pgfsetlinewidth{0.803000pt}%
\definecolor{currentstroke}{rgb}{0.000000,0.000000,0.000000}%
\pgfsetstrokecolor{currentstroke}%
\pgfsetdash{}{0pt}%
\pgfsys@defobject{currentmarker}{\pgfqpoint{0.000000in}{-0.048611in}}{\pgfqpoint{0.000000in}{0.000000in}}{%
\pgfpathmoveto{\pgfqpoint{0.000000in}{0.000000in}}%
\pgfpathlineto{\pgfqpoint{0.000000in}{-0.048611in}}%
\pgfusepath{stroke,fill}%
}%
\begin{pgfscope}%
\pgfsys@transformshift{2.549002in}{0.515123in}%
\pgfsys@useobject{currentmarker}{}%
\end{pgfscope}%
\end{pgfscope}%
\begin{pgfscope}%
\definecolor{textcolor}{rgb}{0.000000,0.000000,0.000000}%
\pgfsetstrokecolor{textcolor}%
\pgfsetfillcolor{textcolor}%
\pgftext[x=2.549002in,y=0.417901in,,top]{\color{textcolor}\rmfamily\fontsize{10.000000}{12.000000}\selectfont \(\displaystyle 60\)}%
\end{pgfscope}%
\begin{pgfscope}%
\pgfsetbuttcap%
\pgfsetroundjoin%
\definecolor{currentfill}{rgb}{0.000000,0.000000,0.000000}%
\pgfsetfillcolor{currentfill}%
\pgfsetlinewidth{0.803000pt}%
\definecolor{currentstroke}{rgb}{0.000000,0.000000,0.000000}%
\pgfsetstrokecolor{currentstroke}%
\pgfsetdash{}{0pt}%
\pgfsys@defobject{currentmarker}{\pgfqpoint{0.000000in}{-0.048611in}}{\pgfqpoint{0.000000in}{0.000000in}}{%
\pgfpathmoveto{\pgfqpoint{0.000000in}{0.000000in}}%
\pgfpathlineto{\pgfqpoint{0.000000in}{-0.048611in}}%
\pgfusepath{stroke,fill}%
}%
\begin{pgfscope}%
\pgfsys@transformshift{3.098320in}{0.515123in}%
\pgfsys@useobject{currentmarker}{}%
\end{pgfscope}%
\end{pgfscope}%
\begin{pgfscope}%
\definecolor{textcolor}{rgb}{0.000000,0.000000,0.000000}%
\pgfsetstrokecolor{textcolor}%
\pgfsetfillcolor{textcolor}%
\pgftext[x=3.098320in,y=0.417901in,,top]{\color{textcolor}\rmfamily\fontsize{10.000000}{12.000000}\selectfont \(\displaystyle 80\)}%
\end{pgfscope}%
\begin{pgfscope}%
\pgfsetbuttcap%
\pgfsetroundjoin%
\definecolor{currentfill}{rgb}{0.000000,0.000000,0.000000}%
\pgfsetfillcolor{currentfill}%
\pgfsetlinewidth{0.803000pt}%
\definecolor{currentstroke}{rgb}{0.000000,0.000000,0.000000}%
\pgfsetstrokecolor{currentstroke}%
\pgfsetdash{}{0pt}%
\pgfsys@defobject{currentmarker}{\pgfqpoint{0.000000in}{-0.048611in}}{\pgfqpoint{0.000000in}{0.000000in}}{%
\pgfpathmoveto{\pgfqpoint{0.000000in}{0.000000in}}%
\pgfpathlineto{\pgfqpoint{0.000000in}{-0.048611in}}%
\pgfusepath{stroke,fill}%
}%
\begin{pgfscope}%
\pgfsys@transformshift{3.647639in}{0.515123in}%
\pgfsys@useobject{currentmarker}{}%
\end{pgfscope}%
\end{pgfscope}%
\begin{pgfscope}%
\definecolor{textcolor}{rgb}{0.000000,0.000000,0.000000}%
\pgfsetstrokecolor{textcolor}%
\pgfsetfillcolor{textcolor}%
\pgftext[x=3.647639in,y=0.417901in,,top]{\color{textcolor}\rmfamily\fontsize{10.000000}{12.000000}\selectfont \(\displaystyle 100\)}%
\end{pgfscope}%
\begin{pgfscope}%
\pgfsetbuttcap%
\pgfsetroundjoin%
\definecolor{currentfill}{rgb}{0.000000,0.000000,0.000000}%
\pgfsetfillcolor{currentfill}%
\pgfsetlinewidth{0.803000pt}%
\definecolor{currentstroke}{rgb}{0.000000,0.000000,0.000000}%
\pgfsetstrokecolor{currentstroke}%
\pgfsetdash{}{0pt}%
\pgfsys@defobject{currentmarker}{\pgfqpoint{0.000000in}{-0.048611in}}{\pgfqpoint{0.000000in}{0.000000in}}{%
\pgfpathmoveto{\pgfqpoint{0.000000in}{0.000000in}}%
\pgfpathlineto{\pgfqpoint{0.000000in}{-0.048611in}}%
\pgfusepath{stroke,fill}%
}%
\begin{pgfscope}%
\pgfsys@transformshift{4.196957in}{0.515123in}%
\pgfsys@useobject{currentmarker}{}%
\end{pgfscope}%
\end{pgfscope}%
\begin{pgfscope}%
\definecolor{textcolor}{rgb}{0.000000,0.000000,0.000000}%
\pgfsetstrokecolor{textcolor}%
\pgfsetfillcolor{textcolor}%
\pgftext[x=4.196957in,y=0.417901in,,top]{\color{textcolor}\rmfamily\fontsize{10.000000}{12.000000}\selectfont \(\displaystyle 120\)}%
\end{pgfscope}%
\begin{pgfscope}%
\definecolor{textcolor}{rgb}{0.000000,0.000000,0.000000}%
\pgfsetstrokecolor{textcolor}%
\pgfsetfillcolor{textcolor}%
\pgftext[x=2.709569in,y=0.238889in,,top]{\color{textcolor}\rmfamily\fontsize{10.000000}{12.000000}\selectfont I(\(\displaystyle \mu\)A)}%
\end{pgfscope}%
\begin{pgfscope}%
\pgfsetbuttcap%
\pgfsetroundjoin%
\definecolor{currentfill}{rgb}{0.000000,0.000000,0.000000}%
\pgfsetfillcolor{currentfill}%
\pgfsetlinewidth{0.803000pt}%
\definecolor{currentstroke}{rgb}{0.000000,0.000000,0.000000}%
\pgfsetstrokecolor{currentstroke}%
\pgfsetdash{}{0pt}%
\pgfsys@defobject{currentmarker}{\pgfqpoint{-0.048611in}{0.000000in}}{\pgfqpoint{0.000000in}{0.000000in}}{%
\pgfpathmoveto{\pgfqpoint{0.000000in}{0.000000in}}%
\pgfpathlineto{\pgfqpoint{-0.048611in}{0.000000in}}%
\pgfusepath{stroke,fill}%
}%
\begin{pgfscope}%
\pgfsys@transformshift{0.772069in}{0.898433in}%
\pgfsys@useobject{currentmarker}{}%
\end{pgfscope}%
\end{pgfscope}%
\begin{pgfscope}%
\definecolor{textcolor}{rgb}{0.000000,0.000000,0.000000}%
\pgfsetstrokecolor{textcolor}%
\pgfsetfillcolor{textcolor}%
\pgftext[x=0.605402in,y=0.850208in,left,base]{\color{textcolor}\rmfamily\fontsize{10.000000}{12.000000}\selectfont \(\displaystyle 2\)}%
\end{pgfscope}%
\begin{pgfscope}%
\pgfsetbuttcap%
\pgfsetroundjoin%
\definecolor{currentfill}{rgb}{0.000000,0.000000,0.000000}%
\pgfsetfillcolor{currentfill}%
\pgfsetlinewidth{0.803000pt}%
\definecolor{currentstroke}{rgb}{0.000000,0.000000,0.000000}%
\pgfsetstrokecolor{currentstroke}%
\pgfsetdash{}{0pt}%
\pgfsys@defobject{currentmarker}{\pgfqpoint{-0.048611in}{0.000000in}}{\pgfqpoint{0.000000in}{0.000000in}}{%
\pgfpathmoveto{\pgfqpoint{0.000000in}{0.000000in}}%
\pgfpathlineto{\pgfqpoint{-0.048611in}{0.000000in}}%
\pgfusepath{stroke,fill}%
}%
\begin{pgfscope}%
\pgfsys@transformshift{0.772069in}{1.430254in}%
\pgfsys@useobject{currentmarker}{}%
\end{pgfscope}%
\end{pgfscope}%
\begin{pgfscope}%
\definecolor{textcolor}{rgb}{0.000000,0.000000,0.000000}%
\pgfsetstrokecolor{textcolor}%
\pgfsetfillcolor{textcolor}%
\pgftext[x=0.605402in,y=1.382028in,left,base]{\color{textcolor}\rmfamily\fontsize{10.000000}{12.000000}\selectfont \(\displaystyle 4\)}%
\end{pgfscope}%
\begin{pgfscope}%
\pgfsetbuttcap%
\pgfsetroundjoin%
\definecolor{currentfill}{rgb}{0.000000,0.000000,0.000000}%
\pgfsetfillcolor{currentfill}%
\pgfsetlinewidth{0.803000pt}%
\definecolor{currentstroke}{rgb}{0.000000,0.000000,0.000000}%
\pgfsetstrokecolor{currentstroke}%
\pgfsetdash{}{0pt}%
\pgfsys@defobject{currentmarker}{\pgfqpoint{-0.048611in}{0.000000in}}{\pgfqpoint{0.000000in}{0.000000in}}{%
\pgfpathmoveto{\pgfqpoint{0.000000in}{0.000000in}}%
\pgfpathlineto{\pgfqpoint{-0.048611in}{0.000000in}}%
\pgfusepath{stroke,fill}%
}%
\begin{pgfscope}%
\pgfsys@transformshift{0.772069in}{1.962074in}%
\pgfsys@useobject{currentmarker}{}%
\end{pgfscope}%
\end{pgfscope}%
\begin{pgfscope}%
\definecolor{textcolor}{rgb}{0.000000,0.000000,0.000000}%
\pgfsetstrokecolor{textcolor}%
\pgfsetfillcolor{textcolor}%
\pgftext[x=0.605402in,y=1.913848in,left,base]{\color{textcolor}\rmfamily\fontsize{10.000000}{12.000000}\selectfont \(\displaystyle 6\)}%
\end{pgfscope}%
\begin{pgfscope}%
\pgfsetbuttcap%
\pgfsetroundjoin%
\definecolor{currentfill}{rgb}{0.000000,0.000000,0.000000}%
\pgfsetfillcolor{currentfill}%
\pgfsetlinewidth{0.803000pt}%
\definecolor{currentstroke}{rgb}{0.000000,0.000000,0.000000}%
\pgfsetstrokecolor{currentstroke}%
\pgfsetdash{}{0pt}%
\pgfsys@defobject{currentmarker}{\pgfqpoint{-0.048611in}{0.000000in}}{\pgfqpoint{0.000000in}{0.000000in}}{%
\pgfpathmoveto{\pgfqpoint{0.000000in}{0.000000in}}%
\pgfpathlineto{\pgfqpoint{-0.048611in}{0.000000in}}%
\pgfusepath{stroke,fill}%
}%
\begin{pgfscope}%
\pgfsys@transformshift{0.772069in}{2.493894in}%
\pgfsys@useobject{currentmarker}{}%
\end{pgfscope}%
\end{pgfscope}%
\begin{pgfscope}%
\definecolor{textcolor}{rgb}{0.000000,0.000000,0.000000}%
\pgfsetstrokecolor{textcolor}%
\pgfsetfillcolor{textcolor}%
\pgftext[x=0.605402in,y=2.445669in,left,base]{\color{textcolor}\rmfamily\fontsize{10.000000}{12.000000}\selectfont \(\displaystyle 8\)}%
\end{pgfscope}%
\begin{pgfscope}%
\pgfsetbuttcap%
\pgfsetroundjoin%
\definecolor{currentfill}{rgb}{0.000000,0.000000,0.000000}%
\pgfsetfillcolor{currentfill}%
\pgfsetlinewidth{0.803000pt}%
\definecolor{currentstroke}{rgb}{0.000000,0.000000,0.000000}%
\pgfsetstrokecolor{currentstroke}%
\pgfsetdash{}{0pt}%
\pgfsys@defobject{currentmarker}{\pgfqpoint{-0.048611in}{0.000000in}}{\pgfqpoint{0.000000in}{0.000000in}}{%
\pgfpathmoveto{\pgfqpoint{0.000000in}{0.000000in}}%
\pgfpathlineto{\pgfqpoint{-0.048611in}{0.000000in}}%
\pgfusepath{stroke,fill}%
}%
\begin{pgfscope}%
\pgfsys@transformshift{0.772069in}{3.025714in}%
\pgfsys@useobject{currentmarker}{}%
\end{pgfscope}%
\end{pgfscope}%
\begin{pgfscope}%
\definecolor{textcolor}{rgb}{0.000000,0.000000,0.000000}%
\pgfsetstrokecolor{textcolor}%
\pgfsetfillcolor{textcolor}%
\pgftext[x=0.535957in,y=2.977489in,left,base]{\color{textcolor}\rmfamily\fontsize{10.000000}{12.000000}\selectfont \(\displaystyle 10\)}%
\end{pgfscope}%
\begin{pgfscope}%
\definecolor{textcolor}{rgb}{0.000000,0.000000,0.000000}%
\pgfsetstrokecolor{textcolor}%
\pgfsetfillcolor{textcolor}%
\pgftext[x=0.258179in,y=1.862623in,,bottom]{\color{textcolor}\rmfamily\fontsize{10.000000}{12.000000}\selectfont V(V)}%
\end{pgfscope}%
\begin{pgfscope}%
\pgfpathrectangle{\pgfqpoint{0.772069in}{0.515123in}}{\pgfqpoint{3.875000in}{2.695000in}}%
\pgfusepath{clip}%
\pgfsetbuttcap%
\pgfsetroundjoin%
\definecolor{currentfill}{rgb}{0.121569,0.466667,0.705882}%
\pgfsetfillcolor{currentfill}%
\pgfsetlinewidth{1.003750pt}%
\definecolor{currentstroke}{rgb}{0.121569,0.466667,0.705882}%
\pgfsetstrokecolor{currentstroke}%
\pgfsetdash{}{0pt}%
\pgfpathmoveto{\pgfqpoint{1.299303in}{0.625957in}}%
\pgfpathcurveto{\pgfqpoint{1.310353in}{0.625957in}}{\pgfqpoint{1.320952in}{0.630347in}}{\pgfqpoint{1.328766in}{0.638161in}}%
\pgfpathcurveto{\pgfqpoint{1.336579in}{0.645974in}}{\pgfqpoint{1.340970in}{0.656573in}}{\pgfqpoint{1.340970in}{0.667623in}}%
\pgfpathcurveto{\pgfqpoint{1.340970in}{0.678673in}}{\pgfqpoint{1.336579in}{0.689272in}}{\pgfqpoint{1.328766in}{0.697086in}}%
\pgfpathcurveto{\pgfqpoint{1.320952in}{0.704900in}}{\pgfqpoint{1.310353in}{0.709290in}}{\pgfqpoint{1.299303in}{0.709290in}}%
\pgfpathcurveto{\pgfqpoint{1.288253in}{0.709290in}}{\pgfqpoint{1.277654in}{0.704900in}}{\pgfqpoint{1.269840in}{0.697086in}}%
\pgfpathcurveto{\pgfqpoint{1.262027in}{0.689272in}}{\pgfqpoint{1.257636in}{0.678673in}}{\pgfqpoint{1.257636in}{0.667623in}}%
\pgfpathcurveto{\pgfqpoint{1.257636in}{0.656573in}}{\pgfqpoint{1.262027in}{0.645974in}}{\pgfqpoint{1.269840in}{0.638161in}}%
\pgfpathcurveto{\pgfqpoint{1.277654in}{0.630347in}}{\pgfqpoint{1.288253in}{0.625957in}}{\pgfqpoint{1.299303in}{0.625957in}}%
\pgfpathclose%
\pgfusepath{stroke,fill}%
\end{pgfscope}%
\begin{pgfscope}%
\pgfpathrectangle{\pgfqpoint{0.772069in}{0.515123in}}{\pgfqpoint{3.875000in}{2.695000in}}%
\pgfusepath{clip}%
\pgfsetbuttcap%
\pgfsetroundjoin%
\definecolor{currentfill}{rgb}{0.121569,0.466667,0.705882}%
\pgfsetfillcolor{currentfill}%
\pgfsetlinewidth{1.003750pt}%
\definecolor{currentstroke}{rgb}{0.121569,0.466667,0.705882}%
\pgfsetstrokecolor{currentstroke}%
\pgfsetdash{}{0pt}%
\pgfpathmoveto{\pgfqpoint{1.626147in}{0.875380in}}%
\pgfpathcurveto{\pgfqpoint{1.637197in}{0.875380in}}{\pgfqpoint{1.647797in}{0.879771in}}{\pgfqpoint{1.655610in}{0.887584in}}%
\pgfpathcurveto{\pgfqpoint{1.663424in}{0.895398in}}{\pgfqpoint{1.667814in}{0.905997in}}{\pgfqpoint{1.667814in}{0.917047in}}%
\pgfpathcurveto{\pgfqpoint{1.667814in}{0.928097in}}{\pgfqpoint{1.663424in}{0.938696in}}{\pgfqpoint{1.655610in}{0.946510in}}%
\pgfpathcurveto{\pgfqpoint{1.647797in}{0.954323in}}{\pgfqpoint{1.637197in}{0.958714in}}{\pgfqpoint{1.626147in}{0.958714in}}%
\pgfpathcurveto{\pgfqpoint{1.615097in}{0.958714in}}{\pgfqpoint{1.604498in}{0.954323in}}{\pgfqpoint{1.596685in}{0.946510in}}%
\pgfpathcurveto{\pgfqpoint{1.588871in}{0.938696in}}{\pgfqpoint{1.584481in}{0.928097in}}{\pgfqpoint{1.584481in}{0.917047in}}%
\pgfpathcurveto{\pgfqpoint{1.584481in}{0.905997in}}{\pgfqpoint{1.588871in}{0.895398in}}{\pgfqpoint{1.596685in}{0.887584in}}%
\pgfpathcurveto{\pgfqpoint{1.604498in}{0.879771in}}{\pgfqpoint{1.615097in}{0.875380in}}{\pgfqpoint{1.626147in}{0.875380in}}%
\pgfpathclose%
\pgfusepath{stroke,fill}%
\end{pgfscope}%
\begin{pgfscope}%
\pgfpathrectangle{\pgfqpoint{0.772069in}{0.515123in}}{\pgfqpoint{3.875000in}{2.695000in}}%
\pgfusepath{clip}%
\pgfsetbuttcap%
\pgfsetroundjoin%
\definecolor{currentfill}{rgb}{0.121569,0.466667,0.705882}%
\pgfsetfillcolor{currentfill}%
\pgfsetlinewidth{1.003750pt}%
\definecolor{currentstroke}{rgb}{0.121569,0.466667,0.705882}%
\pgfsetstrokecolor{currentstroke}%
\pgfsetdash{}{0pt}%
\pgfpathmoveto{\pgfqpoint{1.952992in}{1.125336in}}%
\pgfpathcurveto{\pgfqpoint{1.964042in}{1.125336in}}{\pgfqpoint{1.974641in}{1.129726in}}{\pgfqpoint{1.982455in}{1.137540in}}%
\pgfpathcurveto{\pgfqpoint{1.990268in}{1.145353in}}{\pgfqpoint{1.994658in}{1.155952in}}{\pgfqpoint{1.994658in}{1.167002in}}%
\pgfpathcurveto{\pgfqpoint{1.994658in}{1.178053in}}{\pgfqpoint{1.990268in}{1.188652in}}{\pgfqpoint{1.982455in}{1.196465in}}%
\pgfpathcurveto{\pgfqpoint{1.974641in}{1.204279in}}{\pgfqpoint{1.964042in}{1.208669in}}{\pgfqpoint{1.952992in}{1.208669in}}%
\pgfpathcurveto{\pgfqpoint{1.941942in}{1.208669in}}{\pgfqpoint{1.931343in}{1.204279in}}{\pgfqpoint{1.923529in}{1.196465in}}%
\pgfpathcurveto{\pgfqpoint{1.915715in}{1.188652in}}{\pgfqpoint{1.911325in}{1.178053in}}{\pgfqpoint{1.911325in}{1.167002in}}%
\pgfpathcurveto{\pgfqpoint{1.911325in}{1.155952in}}{\pgfqpoint{1.915715in}{1.145353in}}{\pgfqpoint{1.923529in}{1.137540in}}%
\pgfpathcurveto{\pgfqpoint{1.931343in}{1.129726in}}{\pgfqpoint{1.941942in}{1.125336in}}{\pgfqpoint{1.952992in}{1.125336in}}%
\pgfpathclose%
\pgfusepath{stroke,fill}%
\end{pgfscope}%
\begin{pgfscope}%
\pgfpathrectangle{\pgfqpoint{0.772069in}{0.515123in}}{\pgfqpoint{3.875000in}{2.695000in}}%
\pgfusepath{clip}%
\pgfsetbuttcap%
\pgfsetroundjoin%
\definecolor{currentfill}{rgb}{0.121569,0.466667,0.705882}%
\pgfsetfillcolor{currentfill}%
\pgfsetlinewidth{1.003750pt}%
\definecolor{currentstroke}{rgb}{0.121569,0.466667,0.705882}%
\pgfsetstrokecolor{currentstroke}%
\pgfsetdash{}{0pt}%
\pgfpathmoveto{\pgfqpoint{2.318288in}{1.399223in}}%
\pgfpathcurveto{\pgfqpoint{2.329339in}{1.399223in}}{\pgfqpoint{2.339938in}{1.403614in}}{\pgfqpoint{2.347751in}{1.411427in}}%
\pgfpathcurveto{\pgfqpoint{2.355565in}{1.419241in}}{\pgfqpoint{2.359955in}{1.429840in}}{\pgfqpoint{2.359955in}{1.440890in}}%
\pgfpathcurveto{\pgfqpoint{2.359955in}{1.451940in}}{\pgfqpoint{2.355565in}{1.462539in}}{\pgfqpoint{2.347751in}{1.470353in}}%
\pgfpathcurveto{\pgfqpoint{2.339938in}{1.478166in}}{\pgfqpoint{2.329339in}{1.482557in}}{\pgfqpoint{2.318288in}{1.482557in}}%
\pgfpathcurveto{\pgfqpoint{2.307238in}{1.482557in}}{\pgfqpoint{2.296639in}{1.478166in}}{\pgfqpoint{2.288826in}{1.470353in}}%
\pgfpathcurveto{\pgfqpoint{2.281012in}{1.462539in}}{\pgfqpoint{2.276622in}{1.451940in}}{\pgfqpoint{2.276622in}{1.440890in}}%
\pgfpathcurveto{\pgfqpoint{2.276622in}{1.429840in}}{\pgfqpoint{2.281012in}{1.419241in}}{\pgfqpoint{2.288826in}{1.411427in}}%
\pgfpathcurveto{\pgfqpoint{2.296639in}{1.403614in}}{\pgfqpoint{2.307238in}{1.399223in}}{\pgfqpoint{2.318288in}{1.399223in}}%
\pgfpathclose%
\pgfusepath{stroke,fill}%
\end{pgfscope}%
\begin{pgfscope}%
\pgfpathrectangle{\pgfqpoint{0.772069in}{0.515123in}}{\pgfqpoint{3.875000in}{2.695000in}}%
\pgfusepath{clip}%
\pgfsetbuttcap%
\pgfsetroundjoin%
\definecolor{currentfill}{rgb}{0.121569,0.466667,0.705882}%
\pgfsetfillcolor{currentfill}%
\pgfsetlinewidth{1.003750pt}%
\definecolor{currentstroke}{rgb}{0.121569,0.466667,0.705882}%
\pgfsetstrokecolor{currentstroke}%
\pgfsetdash{}{0pt}%
\pgfpathmoveto{\pgfqpoint{2.680839in}{1.670452in}}%
\pgfpathcurveto{\pgfqpoint{2.691889in}{1.670452in}}{\pgfqpoint{2.702488in}{1.674842in}}{\pgfqpoint{2.710301in}{1.682655in}}%
\pgfpathcurveto{\pgfqpoint{2.718115in}{1.690469in}}{\pgfqpoint{2.722505in}{1.701068in}}{\pgfqpoint{2.722505in}{1.712118in}}%
\pgfpathcurveto{\pgfqpoint{2.722505in}{1.723168in}}{\pgfqpoint{2.718115in}{1.733767in}}{\pgfqpoint{2.710301in}{1.741581in}}%
\pgfpathcurveto{\pgfqpoint{2.702488in}{1.749395in}}{\pgfqpoint{2.691889in}{1.753785in}}{\pgfqpoint{2.680839in}{1.753785in}}%
\pgfpathcurveto{\pgfqpoint{2.669788in}{1.753785in}}{\pgfqpoint{2.659189in}{1.749395in}}{\pgfqpoint{2.651376in}{1.741581in}}%
\pgfpathcurveto{\pgfqpoint{2.643562in}{1.733767in}}{\pgfqpoint{2.639172in}{1.723168in}}{\pgfqpoint{2.639172in}{1.712118in}}%
\pgfpathcurveto{\pgfqpoint{2.639172in}{1.701068in}}{\pgfqpoint{2.643562in}{1.690469in}}{\pgfqpoint{2.651376in}{1.682655in}}%
\pgfpathcurveto{\pgfqpoint{2.659189in}{1.674842in}}{\pgfqpoint{2.669788in}{1.670452in}}{\pgfqpoint{2.680839in}{1.670452in}}%
\pgfpathclose%
\pgfusepath{stroke,fill}%
\end{pgfscope}%
\begin{pgfscope}%
\pgfpathrectangle{\pgfqpoint{0.772069in}{0.515123in}}{\pgfqpoint{3.875000in}{2.695000in}}%
\pgfusepath{clip}%
\pgfsetbuttcap%
\pgfsetroundjoin%
\definecolor{currentfill}{rgb}{0.121569,0.466667,0.705882}%
\pgfsetfillcolor{currentfill}%
\pgfsetlinewidth{1.003750pt}%
\definecolor{currentstroke}{rgb}{0.121569,0.466667,0.705882}%
\pgfsetstrokecolor{currentstroke}%
\pgfsetdash{}{0pt}%
\pgfpathmoveto{\pgfqpoint{3.032402in}{1.939021in}}%
\pgfpathcurveto{\pgfqpoint{3.043452in}{1.939021in}}{\pgfqpoint{3.054051in}{1.943411in}}{\pgfqpoint{3.061865in}{1.951225in}}%
\pgfpathcurveto{\pgfqpoint{3.069679in}{1.959038in}}{\pgfqpoint{3.074069in}{1.969637in}}{\pgfqpoint{3.074069in}{1.980687in}}%
\pgfpathcurveto{\pgfqpoint{3.074069in}{1.991738in}}{\pgfqpoint{3.069679in}{2.002337in}}{\pgfqpoint{3.061865in}{2.010150in}}%
\pgfpathcurveto{\pgfqpoint{3.054051in}{2.017964in}}{\pgfqpoint{3.043452in}{2.022354in}}{\pgfqpoint{3.032402in}{2.022354in}}%
\pgfpathcurveto{\pgfqpoint{3.021352in}{2.022354in}}{\pgfqpoint{3.010753in}{2.017964in}}{\pgfqpoint{3.002939in}{2.010150in}}%
\pgfpathcurveto{\pgfqpoint{2.995126in}{2.002337in}}{\pgfqpoint{2.990736in}{1.991738in}}{\pgfqpoint{2.990736in}{1.980687in}}%
\pgfpathcurveto{\pgfqpoint{2.990736in}{1.969637in}}{\pgfqpoint{2.995126in}{1.959038in}}{\pgfqpoint{3.002939in}{1.951225in}}%
\pgfpathcurveto{\pgfqpoint{3.010753in}{1.943411in}}{\pgfqpoint{3.021352in}{1.939021in}}{\pgfqpoint{3.032402in}{1.939021in}}%
\pgfpathclose%
\pgfusepath{stroke,fill}%
\end{pgfscope}%
\begin{pgfscope}%
\pgfpathrectangle{\pgfqpoint{0.772069in}{0.515123in}}{\pgfqpoint{3.875000in}{2.695000in}}%
\pgfusepath{clip}%
\pgfsetbuttcap%
\pgfsetroundjoin%
\definecolor{currentfill}{rgb}{0.121569,0.466667,0.705882}%
\pgfsetfillcolor{currentfill}%
\pgfsetlinewidth{1.003750pt}%
\definecolor{currentstroke}{rgb}{0.121569,0.466667,0.705882}%
\pgfsetstrokecolor{currentstroke}%
\pgfsetdash{}{0pt}%
\pgfpathmoveto{\pgfqpoint{3.392206in}{2.210249in}}%
\pgfpathcurveto{\pgfqpoint{3.403256in}{2.210249in}}{\pgfqpoint{3.413855in}{2.214639in}}{\pgfqpoint{3.421669in}{2.222453in}}%
\pgfpathcurveto{\pgfqpoint{3.429482in}{2.230267in}}{\pgfqpoint{3.433872in}{2.240866in}}{\pgfqpoint{3.433872in}{2.251916in}}%
\pgfpathcurveto{\pgfqpoint{3.433872in}{2.262966in}}{\pgfqpoint{3.429482in}{2.273565in}}{\pgfqpoint{3.421669in}{2.281379in}}%
\pgfpathcurveto{\pgfqpoint{3.413855in}{2.289192in}}{\pgfqpoint{3.403256in}{2.293582in}}{\pgfqpoint{3.392206in}{2.293582in}}%
\pgfpathcurveto{\pgfqpoint{3.381156in}{2.293582in}}{\pgfqpoint{3.370557in}{2.289192in}}{\pgfqpoint{3.362743in}{2.281379in}}%
\pgfpathcurveto{\pgfqpoint{3.354929in}{2.273565in}}{\pgfqpoint{3.350539in}{2.262966in}}{\pgfqpoint{3.350539in}{2.251916in}}%
\pgfpathcurveto{\pgfqpoint{3.350539in}{2.240866in}}{\pgfqpoint{3.354929in}{2.230267in}}{\pgfqpoint{3.362743in}{2.222453in}}%
\pgfpathcurveto{\pgfqpoint{3.370557in}{2.214639in}}{\pgfqpoint{3.381156in}{2.210249in}}{\pgfqpoint{3.392206in}{2.210249in}}%
\pgfpathclose%
\pgfusepath{stroke,fill}%
\end{pgfscope}%
\begin{pgfscope}%
\pgfpathrectangle{\pgfqpoint{0.772069in}{0.515123in}}{\pgfqpoint{3.875000in}{2.695000in}}%
\pgfusepath{clip}%
\pgfsetbuttcap%
\pgfsetroundjoin%
\definecolor{currentfill}{rgb}{0.121569,0.466667,0.705882}%
\pgfsetfillcolor{currentfill}%
\pgfsetlinewidth{1.003750pt}%
\definecolor{currentstroke}{rgb}{0.121569,0.466667,0.705882}%
\pgfsetstrokecolor{currentstroke}%
\pgfsetdash{}{0pt}%
\pgfpathmoveto{\pgfqpoint{3.730037in}{2.473500in}}%
\pgfpathcurveto{\pgfqpoint{3.741087in}{2.473500in}}{\pgfqpoint{3.751686in}{2.477890in}}{\pgfqpoint{3.759499in}{2.485704in}}%
\pgfpathcurveto{\pgfqpoint{3.767313in}{2.493518in}}{\pgfqpoint{3.771703in}{2.504117in}}{\pgfqpoint{3.771703in}{2.515167in}}%
\pgfpathcurveto{\pgfqpoint{3.771703in}{2.526217in}}{\pgfqpoint{3.767313in}{2.536816in}}{\pgfqpoint{3.759499in}{2.544630in}}%
\pgfpathcurveto{\pgfqpoint{3.751686in}{2.552443in}}{\pgfqpoint{3.741087in}{2.556833in}}{\pgfqpoint{3.730037in}{2.556833in}}%
\pgfpathcurveto{\pgfqpoint{3.718986in}{2.556833in}}{\pgfqpoint{3.708387in}{2.552443in}}{\pgfqpoint{3.700574in}{2.544630in}}%
\pgfpathcurveto{\pgfqpoint{3.692760in}{2.536816in}}{\pgfqpoint{3.688370in}{2.526217in}}{\pgfqpoint{3.688370in}{2.515167in}}%
\pgfpathcurveto{\pgfqpoint{3.688370in}{2.504117in}}{\pgfqpoint{3.692760in}{2.493518in}}{\pgfqpoint{3.700574in}{2.485704in}}%
\pgfpathcurveto{\pgfqpoint{3.708387in}{2.477890in}}{\pgfqpoint{3.718986in}{2.473500in}}{\pgfqpoint{3.730037in}{2.473500in}}%
\pgfpathclose%
\pgfusepath{stroke,fill}%
\end{pgfscope}%
\begin{pgfscope}%
\pgfpathrectangle{\pgfqpoint{0.772069in}{0.515123in}}{\pgfqpoint{3.875000in}{2.695000in}}%
\pgfusepath{clip}%
\pgfsetbuttcap%
\pgfsetroundjoin%
\definecolor{currentfill}{rgb}{0.121569,0.466667,0.705882}%
\pgfsetfillcolor{currentfill}%
\pgfsetlinewidth{1.003750pt}%
\definecolor{currentstroke}{rgb}{0.121569,0.466667,0.705882}%
\pgfsetstrokecolor{currentstroke}%
\pgfsetdash{}{0pt}%
\pgfpathmoveto{\pgfqpoint{4.087093in}{2.736751in}}%
\pgfpathcurveto{\pgfqpoint{4.098144in}{2.736751in}}{\pgfqpoint{4.108743in}{2.741141in}}{\pgfqpoint{4.116556in}{2.748955in}}%
\pgfpathcurveto{\pgfqpoint{4.124370in}{2.756769in}}{\pgfqpoint{4.128760in}{2.767368in}}{\pgfqpoint{4.128760in}{2.778418in}}%
\pgfpathcurveto{\pgfqpoint{4.128760in}{2.789468in}}{\pgfqpoint{4.124370in}{2.800067in}}{\pgfqpoint{4.116556in}{2.807881in}}%
\pgfpathcurveto{\pgfqpoint{4.108743in}{2.815694in}}{\pgfqpoint{4.098144in}{2.820084in}}{\pgfqpoint{4.087093in}{2.820084in}}%
\pgfpathcurveto{\pgfqpoint{4.076043in}{2.820084in}}{\pgfqpoint{4.065444in}{2.815694in}}{\pgfqpoint{4.057631in}{2.807881in}}%
\pgfpathcurveto{\pgfqpoint{4.049817in}{2.800067in}}{\pgfqpoint{4.045427in}{2.789468in}}{\pgfqpoint{4.045427in}{2.778418in}}%
\pgfpathcurveto{\pgfqpoint{4.045427in}{2.767368in}}{\pgfqpoint{4.049817in}{2.756769in}}{\pgfqpoint{4.057631in}{2.748955in}}%
\pgfpathcurveto{\pgfqpoint{4.065444in}{2.741141in}}{\pgfqpoint{4.076043in}{2.736751in}}{\pgfqpoint{4.087093in}{2.736751in}}%
\pgfpathclose%
\pgfusepath{stroke,fill}%
\end{pgfscope}%
\begin{pgfscope}%
\pgfpathrectangle{\pgfqpoint{0.772069in}{0.515123in}}{\pgfqpoint{3.875000in}{2.695000in}}%
\pgfusepath{clip}%
\pgfsetbuttcap%
\pgfsetroundjoin%
\definecolor{currentfill}{rgb}{0.121569,0.466667,0.705882}%
\pgfsetfillcolor{currentfill}%
\pgfsetlinewidth{1.003750pt}%
\definecolor{currentstroke}{rgb}{0.121569,0.466667,0.705882}%
\pgfsetstrokecolor{currentstroke}%
\pgfsetdash{}{0pt}%
\pgfpathmoveto{\pgfqpoint{4.444150in}{3.015957in}}%
\pgfpathcurveto{\pgfqpoint{4.455201in}{3.015957in}}{\pgfqpoint{4.465800in}{3.020347in}}{\pgfqpoint{4.473613in}{3.028161in}}%
\pgfpathcurveto{\pgfqpoint{4.481427in}{3.035974in}}{\pgfqpoint{4.485817in}{3.046573in}}{\pgfqpoint{4.485817in}{3.057623in}}%
\pgfpathcurveto{\pgfqpoint{4.485817in}{3.068673in}}{\pgfqpoint{4.481427in}{3.079273in}}{\pgfqpoint{4.473613in}{3.087086in}}%
\pgfpathcurveto{\pgfqpoint{4.465800in}{3.094900in}}{\pgfqpoint{4.455201in}{3.099290in}}{\pgfqpoint{4.444150in}{3.099290in}}%
\pgfpathcurveto{\pgfqpoint{4.433100in}{3.099290in}}{\pgfqpoint{4.422501in}{3.094900in}}{\pgfqpoint{4.414688in}{3.087086in}}%
\pgfpathcurveto{\pgfqpoint{4.406874in}{3.079273in}}{\pgfqpoint{4.402484in}{3.068673in}}{\pgfqpoint{4.402484in}{3.057623in}}%
\pgfpathcurveto{\pgfqpoint{4.402484in}{3.046573in}}{\pgfqpoint{4.406874in}{3.035974in}}{\pgfqpoint{4.414688in}{3.028161in}}%
\pgfpathcurveto{\pgfqpoint{4.422501in}{3.020347in}}{\pgfqpoint{4.433100in}{3.015957in}}{\pgfqpoint{4.444150in}{3.015957in}}%
\pgfpathclose%
\pgfusepath{stroke,fill}%
\end{pgfscope}%
\begin{pgfscope}%
\pgfpathrectangle{\pgfqpoint{0.772069in}{0.515123in}}{\pgfqpoint{3.875000in}{2.695000in}}%
\pgfusepath{clip}%
\pgfsetbuttcap%
\pgfsetroundjoin%
\definecolor{currentfill}{rgb}{1.000000,0.388235,0.278431}%
\pgfsetfillcolor{currentfill}%
\pgfsetlinewidth{1.003750pt}%
\definecolor{currentstroke}{rgb}{1.000000,0.388235,0.278431}%
\pgfsetstrokecolor{currentstroke}%
\pgfsetdash{}{0pt}%
\pgfpathmoveto{\pgfqpoint{1.076829in}{0.625957in}}%
\pgfpathcurveto{\pgfqpoint{1.087879in}{0.625957in}}{\pgfqpoint{1.098478in}{0.630347in}}{\pgfqpoint{1.106292in}{0.638161in}}%
\pgfpathcurveto{\pgfqpoint{1.114105in}{0.645974in}}{\pgfqpoint{1.118496in}{0.656573in}}{\pgfqpoint{1.118496in}{0.667623in}}%
\pgfpathcurveto{\pgfqpoint{1.118496in}{0.678673in}}{\pgfqpoint{1.114105in}{0.689272in}}{\pgfqpoint{1.106292in}{0.697086in}}%
\pgfpathcurveto{\pgfqpoint{1.098478in}{0.704900in}}{\pgfqpoint{1.087879in}{0.709290in}}{\pgfqpoint{1.076829in}{0.709290in}}%
\pgfpathcurveto{\pgfqpoint{1.065779in}{0.709290in}}{\pgfqpoint{1.055180in}{0.704900in}}{\pgfqpoint{1.047366in}{0.697086in}}%
\pgfpathcurveto{\pgfqpoint{1.039553in}{0.689272in}}{\pgfqpoint{1.035162in}{0.678673in}}{\pgfqpoint{1.035162in}{0.667623in}}%
\pgfpathcurveto{\pgfqpoint{1.035162in}{0.656573in}}{\pgfqpoint{1.039553in}{0.645974in}}{\pgfqpoint{1.047366in}{0.638161in}}%
\pgfpathcurveto{\pgfqpoint{1.055180in}{0.630347in}}{\pgfqpoint{1.065779in}{0.625957in}}{\pgfqpoint{1.076829in}{0.625957in}}%
\pgfpathclose%
\pgfusepath{stroke,fill}%
\end{pgfscope}%
\begin{pgfscope}%
\pgfpathrectangle{\pgfqpoint{0.772069in}{0.515123in}}{\pgfqpoint{3.875000in}{2.695000in}}%
\pgfusepath{clip}%
\pgfsetbuttcap%
\pgfsetroundjoin%
\definecolor{currentfill}{rgb}{1.000000,0.388235,0.278431}%
\pgfsetfillcolor{currentfill}%
\pgfsetlinewidth{1.003750pt}%
\definecolor{currentstroke}{rgb}{1.000000,0.388235,0.278431}%
\pgfsetstrokecolor{currentstroke}%
\pgfsetdash{}{0pt}%
\pgfpathmoveto{\pgfqpoint{1.222398in}{0.875380in}}%
\pgfpathcurveto{\pgfqpoint{1.233449in}{0.875380in}}{\pgfqpoint{1.244048in}{0.879771in}}{\pgfqpoint{1.251861in}{0.887584in}}%
\pgfpathcurveto{\pgfqpoint{1.259675in}{0.895398in}}{\pgfqpoint{1.264065in}{0.905997in}}{\pgfqpoint{1.264065in}{0.917047in}}%
\pgfpathcurveto{\pgfqpoint{1.264065in}{0.928097in}}{\pgfqpoint{1.259675in}{0.938696in}}{\pgfqpoint{1.251861in}{0.946510in}}%
\pgfpathcurveto{\pgfqpoint{1.244048in}{0.954323in}}{\pgfqpoint{1.233449in}{0.958714in}}{\pgfqpoint{1.222398in}{0.958714in}}%
\pgfpathcurveto{\pgfqpoint{1.211348in}{0.958714in}}{\pgfqpoint{1.200749in}{0.954323in}}{\pgfqpoint{1.192936in}{0.946510in}}%
\pgfpathcurveto{\pgfqpoint{1.185122in}{0.938696in}}{\pgfqpoint{1.180732in}{0.928097in}}{\pgfqpoint{1.180732in}{0.917047in}}%
\pgfpathcurveto{\pgfqpoint{1.180732in}{0.905997in}}{\pgfqpoint{1.185122in}{0.895398in}}{\pgfqpoint{1.192936in}{0.887584in}}%
\pgfpathcurveto{\pgfqpoint{1.200749in}{0.879771in}}{\pgfqpoint{1.211348in}{0.875380in}}{\pgfqpoint{1.222398in}{0.875380in}}%
\pgfpathclose%
\pgfusepath{stroke,fill}%
\end{pgfscope}%
\begin{pgfscope}%
\pgfpathrectangle{\pgfqpoint{0.772069in}{0.515123in}}{\pgfqpoint{3.875000in}{2.695000in}}%
\pgfusepath{clip}%
\pgfsetbuttcap%
\pgfsetroundjoin%
\definecolor{currentfill}{rgb}{1.000000,0.388235,0.278431}%
\pgfsetfillcolor{currentfill}%
\pgfsetlinewidth{1.003750pt}%
\definecolor{currentstroke}{rgb}{1.000000,0.388235,0.278431}%
\pgfsetstrokecolor{currentstroke}%
\pgfsetdash{}{0pt}%
\pgfpathmoveto{\pgfqpoint{1.367968in}{1.125336in}}%
\pgfpathcurveto{\pgfqpoint{1.379018in}{1.125336in}}{\pgfqpoint{1.389617in}{1.129726in}}{\pgfqpoint{1.397431in}{1.137540in}}%
\pgfpathcurveto{\pgfqpoint{1.405244in}{1.145353in}}{\pgfqpoint{1.409634in}{1.155952in}}{\pgfqpoint{1.409634in}{1.167002in}}%
\pgfpathcurveto{\pgfqpoint{1.409634in}{1.178053in}}{\pgfqpoint{1.405244in}{1.188652in}}{\pgfqpoint{1.397431in}{1.196465in}}%
\pgfpathcurveto{\pgfqpoint{1.389617in}{1.204279in}}{\pgfqpoint{1.379018in}{1.208669in}}{\pgfqpoint{1.367968in}{1.208669in}}%
\pgfpathcurveto{\pgfqpoint{1.356918in}{1.208669in}}{\pgfqpoint{1.346319in}{1.204279in}}{\pgfqpoint{1.338505in}{1.196465in}}%
\pgfpathcurveto{\pgfqpoint{1.330691in}{1.188652in}}{\pgfqpoint{1.326301in}{1.178053in}}{\pgfqpoint{1.326301in}{1.167002in}}%
\pgfpathcurveto{\pgfqpoint{1.326301in}{1.155952in}}{\pgfqpoint{1.330691in}{1.145353in}}{\pgfqpoint{1.338505in}{1.137540in}}%
\pgfpathcurveto{\pgfqpoint{1.346319in}{1.129726in}}{\pgfqpoint{1.356918in}{1.125336in}}{\pgfqpoint{1.367968in}{1.125336in}}%
\pgfpathclose%
\pgfusepath{stroke,fill}%
\end{pgfscope}%
\begin{pgfscope}%
\pgfpathrectangle{\pgfqpoint{0.772069in}{0.515123in}}{\pgfqpoint{3.875000in}{2.695000in}}%
\pgfusepath{clip}%
\pgfsetbuttcap%
\pgfsetroundjoin%
\definecolor{currentfill}{rgb}{1.000000,0.388235,0.278431}%
\pgfsetfillcolor{currentfill}%
\pgfsetlinewidth{1.003750pt}%
\definecolor{currentstroke}{rgb}{1.000000,0.388235,0.278431}%
\pgfsetstrokecolor{currentstroke}%
\pgfsetdash{}{0pt}%
\pgfpathmoveto{\pgfqpoint{1.530017in}{1.399223in}}%
\pgfpathcurveto{\pgfqpoint{1.541067in}{1.399223in}}{\pgfqpoint{1.551666in}{1.403614in}}{\pgfqpoint{1.559479in}{1.411427in}}%
\pgfpathcurveto{\pgfqpoint{1.567293in}{1.419241in}}{\pgfqpoint{1.571683in}{1.429840in}}{\pgfqpoint{1.571683in}{1.440890in}}%
\pgfpathcurveto{\pgfqpoint{1.571683in}{1.451940in}}{\pgfqpoint{1.567293in}{1.462539in}}{\pgfqpoint{1.559479in}{1.470353in}}%
\pgfpathcurveto{\pgfqpoint{1.551666in}{1.478166in}}{\pgfqpoint{1.541067in}{1.482557in}}{\pgfqpoint{1.530017in}{1.482557in}}%
\pgfpathcurveto{\pgfqpoint{1.518967in}{1.482557in}}{\pgfqpoint{1.508367in}{1.478166in}}{\pgfqpoint{1.500554in}{1.470353in}}%
\pgfpathcurveto{\pgfqpoint{1.492740in}{1.462539in}}{\pgfqpoint{1.488350in}{1.451940in}}{\pgfqpoint{1.488350in}{1.440890in}}%
\pgfpathcurveto{\pgfqpoint{1.488350in}{1.429840in}}{\pgfqpoint{1.492740in}{1.419241in}}{\pgfqpoint{1.500554in}{1.411427in}}%
\pgfpathcurveto{\pgfqpoint{1.508367in}{1.403614in}}{\pgfqpoint{1.518967in}{1.399223in}}{\pgfqpoint{1.530017in}{1.399223in}}%
\pgfpathclose%
\pgfusepath{stroke,fill}%
\end{pgfscope}%
\begin{pgfscope}%
\pgfpathrectangle{\pgfqpoint{0.772069in}{0.515123in}}{\pgfqpoint{3.875000in}{2.695000in}}%
\pgfusepath{clip}%
\pgfsetbuttcap%
\pgfsetroundjoin%
\definecolor{currentfill}{rgb}{1.000000,0.388235,0.278431}%
\pgfsetfillcolor{currentfill}%
\pgfsetlinewidth{1.003750pt}%
\definecolor{currentstroke}{rgb}{1.000000,0.388235,0.278431}%
\pgfsetstrokecolor{currentstroke}%
\pgfsetdash{}{0pt}%
\pgfpathmoveto{\pgfqpoint{1.692066in}{1.670452in}}%
\pgfpathcurveto{\pgfqpoint{1.703116in}{1.670452in}}{\pgfqpoint{1.713715in}{1.674842in}}{\pgfqpoint{1.721528in}{1.682655in}}%
\pgfpathcurveto{\pgfqpoint{1.729342in}{1.690469in}}{\pgfqpoint{1.733732in}{1.701068in}}{\pgfqpoint{1.733732in}{1.712118in}}%
\pgfpathcurveto{\pgfqpoint{1.733732in}{1.723168in}}{\pgfqpoint{1.729342in}{1.733767in}}{\pgfqpoint{1.721528in}{1.741581in}}%
\pgfpathcurveto{\pgfqpoint{1.713715in}{1.749395in}}{\pgfqpoint{1.703116in}{1.753785in}}{\pgfqpoint{1.692066in}{1.753785in}}%
\pgfpathcurveto{\pgfqpoint{1.681015in}{1.753785in}}{\pgfqpoint{1.670416in}{1.749395in}}{\pgfqpoint{1.662603in}{1.741581in}}%
\pgfpathcurveto{\pgfqpoint{1.654789in}{1.733767in}}{\pgfqpoint{1.650399in}{1.723168in}}{\pgfqpoint{1.650399in}{1.712118in}}%
\pgfpathcurveto{\pgfqpoint{1.650399in}{1.701068in}}{\pgfqpoint{1.654789in}{1.690469in}}{\pgfqpoint{1.662603in}{1.682655in}}%
\pgfpathcurveto{\pgfqpoint{1.670416in}{1.674842in}}{\pgfqpoint{1.681015in}{1.670452in}}{\pgfqpoint{1.692066in}{1.670452in}}%
\pgfpathclose%
\pgfusepath{stroke,fill}%
\end{pgfscope}%
\begin{pgfscope}%
\pgfpathrectangle{\pgfqpoint{0.772069in}{0.515123in}}{\pgfqpoint{3.875000in}{2.695000in}}%
\pgfusepath{clip}%
\pgfsetbuttcap%
\pgfsetroundjoin%
\definecolor{currentfill}{rgb}{1.000000,0.388235,0.278431}%
\pgfsetfillcolor{currentfill}%
\pgfsetlinewidth{1.003750pt}%
\definecolor{currentstroke}{rgb}{1.000000,0.388235,0.278431}%
\pgfsetstrokecolor{currentstroke}%
\pgfsetdash{}{0pt}%
\pgfpathmoveto{\pgfqpoint{1.848621in}{1.939021in}}%
\pgfpathcurveto{\pgfqpoint{1.859671in}{1.939021in}}{\pgfqpoint{1.870270in}{1.943411in}}{\pgfqpoint{1.878084in}{1.951225in}}%
\pgfpathcurveto{\pgfqpoint{1.885898in}{1.959038in}}{\pgfqpoint{1.890288in}{1.969637in}}{\pgfqpoint{1.890288in}{1.980687in}}%
\pgfpathcurveto{\pgfqpoint{1.890288in}{1.991738in}}{\pgfqpoint{1.885898in}{2.002337in}}{\pgfqpoint{1.878084in}{2.010150in}}%
\pgfpathcurveto{\pgfqpoint{1.870270in}{2.017964in}}{\pgfqpoint{1.859671in}{2.022354in}}{\pgfqpoint{1.848621in}{2.022354in}}%
\pgfpathcurveto{\pgfqpoint{1.837571in}{2.022354in}}{\pgfqpoint{1.826972in}{2.017964in}}{\pgfqpoint{1.819158in}{2.010150in}}%
\pgfpathcurveto{\pgfqpoint{1.811345in}{2.002337in}}{\pgfqpoint{1.806955in}{1.991738in}}{\pgfqpoint{1.806955in}{1.980687in}}%
\pgfpathcurveto{\pgfqpoint{1.806955in}{1.969637in}}{\pgfqpoint{1.811345in}{1.959038in}}{\pgfqpoint{1.819158in}{1.951225in}}%
\pgfpathcurveto{\pgfqpoint{1.826972in}{1.943411in}}{\pgfqpoint{1.837571in}{1.939021in}}{\pgfqpoint{1.848621in}{1.939021in}}%
\pgfpathclose%
\pgfusepath{stroke,fill}%
\end{pgfscope}%
\begin{pgfscope}%
\pgfpathrectangle{\pgfqpoint{0.772069in}{0.515123in}}{\pgfqpoint{3.875000in}{2.695000in}}%
\pgfusepath{clip}%
\pgfsetbuttcap%
\pgfsetroundjoin%
\definecolor{currentfill}{rgb}{1.000000,0.388235,0.278431}%
\pgfsetfillcolor{currentfill}%
\pgfsetlinewidth{1.003750pt}%
\definecolor{currentstroke}{rgb}{1.000000,0.388235,0.278431}%
\pgfsetstrokecolor{currentstroke}%
\pgfsetdash{}{0pt}%
\pgfpathmoveto{\pgfqpoint{2.007924in}{2.210249in}}%
\pgfpathcurveto{\pgfqpoint{2.018974in}{2.210249in}}{\pgfqpoint{2.029573in}{2.214639in}}{\pgfqpoint{2.037386in}{2.222453in}}%
\pgfpathcurveto{\pgfqpoint{2.045200in}{2.230267in}}{\pgfqpoint{2.049590in}{2.240866in}}{\pgfqpoint{2.049590in}{2.251916in}}%
\pgfpathcurveto{\pgfqpoint{2.049590in}{2.262966in}}{\pgfqpoint{2.045200in}{2.273565in}}{\pgfqpoint{2.037386in}{2.281379in}}%
\pgfpathcurveto{\pgfqpoint{2.029573in}{2.289192in}}{\pgfqpoint{2.018974in}{2.293582in}}{\pgfqpoint{2.007924in}{2.293582in}}%
\pgfpathcurveto{\pgfqpoint{1.996873in}{2.293582in}}{\pgfqpoint{1.986274in}{2.289192in}}{\pgfqpoint{1.978461in}{2.281379in}}%
\pgfpathcurveto{\pgfqpoint{1.970647in}{2.273565in}}{\pgfqpoint{1.966257in}{2.262966in}}{\pgfqpoint{1.966257in}{2.251916in}}%
\pgfpathcurveto{\pgfqpoint{1.966257in}{2.240866in}}{\pgfqpoint{1.970647in}{2.230267in}}{\pgfqpoint{1.978461in}{2.222453in}}%
\pgfpathcurveto{\pgfqpoint{1.986274in}{2.214639in}}{\pgfqpoint{1.996873in}{2.210249in}}{\pgfqpoint{2.007924in}{2.210249in}}%
\pgfpathclose%
\pgfusepath{stroke,fill}%
\end{pgfscope}%
\begin{pgfscope}%
\pgfpathrectangle{\pgfqpoint{0.772069in}{0.515123in}}{\pgfqpoint{3.875000in}{2.695000in}}%
\pgfusepath{clip}%
\pgfsetbuttcap%
\pgfsetroundjoin%
\definecolor{currentfill}{rgb}{1.000000,0.388235,0.278431}%
\pgfsetfillcolor{currentfill}%
\pgfsetlinewidth{1.003750pt}%
\definecolor{currentstroke}{rgb}{1.000000,0.388235,0.278431}%
\pgfsetstrokecolor{currentstroke}%
\pgfsetdash{}{0pt}%
\pgfpathmoveto{\pgfqpoint{2.161733in}{2.473500in}}%
\pgfpathcurveto{\pgfqpoint{2.172783in}{2.473500in}}{\pgfqpoint{2.183382in}{2.477890in}}{\pgfqpoint{2.191196in}{2.485704in}}%
\pgfpathcurveto{\pgfqpoint{2.199009in}{2.493518in}}{\pgfqpoint{2.203399in}{2.504117in}}{\pgfqpoint{2.203399in}{2.515167in}}%
\pgfpathcurveto{\pgfqpoint{2.203399in}{2.526217in}}{\pgfqpoint{2.199009in}{2.536816in}}{\pgfqpoint{2.191196in}{2.544630in}}%
\pgfpathcurveto{\pgfqpoint{2.183382in}{2.552443in}}{\pgfqpoint{2.172783in}{2.556833in}}{\pgfqpoint{2.161733in}{2.556833in}}%
\pgfpathcurveto{\pgfqpoint{2.150683in}{2.556833in}}{\pgfqpoint{2.140084in}{2.552443in}}{\pgfqpoint{2.132270in}{2.544630in}}%
\pgfpathcurveto{\pgfqpoint{2.124456in}{2.536816in}}{\pgfqpoint{2.120066in}{2.526217in}}{\pgfqpoint{2.120066in}{2.515167in}}%
\pgfpathcurveto{\pgfqpoint{2.120066in}{2.504117in}}{\pgfqpoint{2.124456in}{2.493518in}}{\pgfqpoint{2.132270in}{2.485704in}}%
\pgfpathcurveto{\pgfqpoint{2.140084in}{2.477890in}}{\pgfqpoint{2.150683in}{2.473500in}}{\pgfqpoint{2.161733in}{2.473500in}}%
\pgfpathclose%
\pgfusepath{stroke,fill}%
\end{pgfscope}%
\begin{pgfscope}%
\pgfpathrectangle{\pgfqpoint{0.772069in}{0.515123in}}{\pgfqpoint{3.875000in}{2.695000in}}%
\pgfusepath{clip}%
\pgfsetbuttcap%
\pgfsetroundjoin%
\definecolor{currentfill}{rgb}{1.000000,0.388235,0.278431}%
\pgfsetfillcolor{currentfill}%
\pgfsetlinewidth{1.003750pt}%
\definecolor{currentstroke}{rgb}{1.000000,0.388235,0.278431}%
\pgfsetstrokecolor{currentstroke}%
\pgfsetdash{}{0pt}%
\pgfpathmoveto{\pgfqpoint{2.315542in}{2.736751in}}%
\pgfpathcurveto{\pgfqpoint{2.326592in}{2.736751in}}{\pgfqpoint{2.337191in}{2.741141in}}{\pgfqpoint{2.345005in}{2.748955in}}%
\pgfpathcurveto{\pgfqpoint{2.352818in}{2.756769in}}{\pgfqpoint{2.357209in}{2.767368in}}{\pgfqpoint{2.357209in}{2.778418in}}%
\pgfpathcurveto{\pgfqpoint{2.357209in}{2.789468in}}{\pgfqpoint{2.352818in}{2.800067in}}{\pgfqpoint{2.345005in}{2.807881in}}%
\pgfpathcurveto{\pgfqpoint{2.337191in}{2.815694in}}{\pgfqpoint{2.326592in}{2.820084in}}{\pgfqpoint{2.315542in}{2.820084in}}%
\pgfpathcurveto{\pgfqpoint{2.304492in}{2.820084in}}{\pgfqpoint{2.293893in}{2.815694in}}{\pgfqpoint{2.286079in}{2.807881in}}%
\pgfpathcurveto{\pgfqpoint{2.278265in}{2.800067in}}{\pgfqpoint{2.273875in}{2.789468in}}{\pgfqpoint{2.273875in}{2.778418in}}%
\pgfpathcurveto{\pgfqpoint{2.273875in}{2.767368in}}{\pgfqpoint{2.278265in}{2.756769in}}{\pgfqpoint{2.286079in}{2.748955in}}%
\pgfpathcurveto{\pgfqpoint{2.293893in}{2.741141in}}{\pgfqpoint{2.304492in}{2.736751in}}{\pgfqpoint{2.315542in}{2.736751in}}%
\pgfpathclose%
\pgfusepath{stroke,fill}%
\end{pgfscope}%
\begin{pgfscope}%
\pgfpathrectangle{\pgfqpoint{0.772069in}{0.515123in}}{\pgfqpoint{3.875000in}{2.695000in}}%
\pgfusepath{clip}%
\pgfsetbuttcap%
\pgfsetroundjoin%
\definecolor{currentfill}{rgb}{1.000000,0.388235,0.278431}%
\pgfsetfillcolor{currentfill}%
\pgfsetlinewidth{1.003750pt}%
\definecolor{currentstroke}{rgb}{1.000000,0.388235,0.278431}%
\pgfsetstrokecolor{currentstroke}%
\pgfsetdash{}{0pt}%
\pgfpathmoveto{\pgfqpoint{2.480337in}{3.015957in}}%
\pgfpathcurveto{\pgfqpoint{2.491387in}{3.015957in}}{\pgfqpoint{2.501987in}{3.020347in}}{\pgfqpoint{2.509800in}{3.028161in}}%
\pgfpathcurveto{\pgfqpoint{2.517614in}{3.035974in}}{\pgfqpoint{2.522004in}{3.046573in}}{\pgfqpoint{2.522004in}{3.057623in}}%
\pgfpathcurveto{\pgfqpoint{2.522004in}{3.068673in}}{\pgfqpoint{2.517614in}{3.079273in}}{\pgfqpoint{2.509800in}{3.087086in}}%
\pgfpathcurveto{\pgfqpoint{2.501987in}{3.094900in}}{\pgfqpoint{2.491387in}{3.099290in}}{\pgfqpoint{2.480337in}{3.099290in}}%
\pgfpathcurveto{\pgfqpoint{2.469287in}{3.099290in}}{\pgfqpoint{2.458688in}{3.094900in}}{\pgfqpoint{2.450875in}{3.087086in}}%
\pgfpathcurveto{\pgfqpoint{2.443061in}{3.079273in}}{\pgfqpoint{2.438671in}{3.068673in}}{\pgfqpoint{2.438671in}{3.057623in}}%
\pgfpathcurveto{\pgfqpoint{2.438671in}{3.046573in}}{\pgfqpoint{2.443061in}{3.035974in}}{\pgfqpoint{2.450875in}{3.028161in}}%
\pgfpathcurveto{\pgfqpoint{2.458688in}{3.020347in}}{\pgfqpoint{2.469287in}{3.015957in}}{\pgfqpoint{2.480337in}{3.015957in}}%
\pgfpathclose%
\pgfusepath{stroke,fill}%
\end{pgfscope}%
\begin{pgfscope}%
\pgfpathrectangle{\pgfqpoint{0.772069in}{0.515123in}}{\pgfqpoint{3.875000in}{2.695000in}}%
\pgfusepath{clip}%
\pgfsetbuttcap%
\pgfsetroundjoin%
\definecolor{currentfill}{rgb}{1.000000,0.843137,0.000000}%
\pgfsetfillcolor{currentfill}%
\pgfsetlinewidth{1.003750pt}%
\definecolor{currentstroke}{rgb}{1.000000,0.843137,0.000000}%
\pgfsetstrokecolor{currentstroke}%
\pgfsetdash{}{0pt}%
\pgfpathmoveto{\pgfqpoint{1.041123in}{0.625957in}}%
\pgfpathcurveto{\pgfqpoint{1.052173in}{0.625957in}}{\pgfqpoint{1.062772in}{0.630347in}}{\pgfqpoint{1.070586in}{0.638161in}}%
\pgfpathcurveto{\pgfqpoint{1.078400in}{0.645974in}}{\pgfqpoint{1.082790in}{0.656573in}}{\pgfqpoint{1.082790in}{0.667623in}}%
\pgfpathcurveto{\pgfqpoint{1.082790in}{0.678673in}}{\pgfqpoint{1.078400in}{0.689272in}}{\pgfqpoint{1.070586in}{0.697086in}}%
\pgfpathcurveto{\pgfqpoint{1.062772in}{0.704900in}}{\pgfqpoint{1.052173in}{0.709290in}}{\pgfqpoint{1.041123in}{0.709290in}}%
\pgfpathcurveto{\pgfqpoint{1.030073in}{0.709290in}}{\pgfqpoint{1.019474in}{0.704900in}}{\pgfqpoint{1.011661in}{0.697086in}}%
\pgfpathcurveto{\pgfqpoint{1.003847in}{0.689272in}}{\pgfqpoint{0.999457in}{0.678673in}}{\pgfqpoint{0.999457in}{0.667623in}}%
\pgfpathcurveto{\pgfqpoint{0.999457in}{0.656573in}}{\pgfqpoint{1.003847in}{0.645974in}}{\pgfqpoint{1.011661in}{0.638161in}}%
\pgfpathcurveto{\pgfqpoint{1.019474in}{0.630347in}}{\pgfqpoint{1.030073in}{0.625957in}}{\pgfqpoint{1.041123in}{0.625957in}}%
\pgfpathclose%
\pgfusepath{stroke,fill}%
\end{pgfscope}%
\begin{pgfscope}%
\pgfpathrectangle{\pgfqpoint{0.772069in}{0.515123in}}{\pgfqpoint{3.875000in}{2.695000in}}%
\pgfusepath{clip}%
\pgfsetbuttcap%
\pgfsetroundjoin%
\definecolor{currentfill}{rgb}{1.000000,0.843137,0.000000}%
\pgfsetfillcolor{currentfill}%
\pgfsetlinewidth{1.003750pt}%
\definecolor{currentstroke}{rgb}{1.000000,0.843137,0.000000}%
\pgfsetstrokecolor{currentstroke}%
\pgfsetdash{}{0pt}%
\pgfpathmoveto{\pgfqpoint{1.159227in}{0.875380in}}%
\pgfpathcurveto{\pgfqpoint{1.170277in}{0.875380in}}{\pgfqpoint{1.180876in}{0.879771in}}{\pgfqpoint{1.188690in}{0.887584in}}%
\pgfpathcurveto{\pgfqpoint{1.196503in}{0.895398in}}{\pgfqpoint{1.200893in}{0.905997in}}{\pgfqpoint{1.200893in}{0.917047in}}%
\pgfpathcurveto{\pgfqpoint{1.200893in}{0.928097in}}{\pgfqpoint{1.196503in}{0.938696in}}{\pgfqpoint{1.188690in}{0.946510in}}%
\pgfpathcurveto{\pgfqpoint{1.180876in}{0.954323in}}{\pgfqpoint{1.170277in}{0.958714in}}{\pgfqpoint{1.159227in}{0.958714in}}%
\pgfpathcurveto{\pgfqpoint{1.148177in}{0.958714in}}{\pgfqpoint{1.137578in}{0.954323in}}{\pgfqpoint{1.129764in}{0.946510in}}%
\pgfpathcurveto{\pgfqpoint{1.121950in}{0.938696in}}{\pgfqpoint{1.117560in}{0.928097in}}{\pgfqpoint{1.117560in}{0.917047in}}%
\pgfpathcurveto{\pgfqpoint{1.117560in}{0.905997in}}{\pgfqpoint{1.121950in}{0.895398in}}{\pgfqpoint{1.129764in}{0.887584in}}%
\pgfpathcurveto{\pgfqpoint{1.137578in}{0.879771in}}{\pgfqpoint{1.148177in}{0.875380in}}{\pgfqpoint{1.159227in}{0.875380in}}%
\pgfpathclose%
\pgfusepath{stroke,fill}%
\end{pgfscope}%
\begin{pgfscope}%
\pgfpathrectangle{\pgfqpoint{0.772069in}{0.515123in}}{\pgfqpoint{3.875000in}{2.695000in}}%
\pgfusepath{clip}%
\pgfsetbuttcap%
\pgfsetroundjoin%
\definecolor{currentfill}{rgb}{1.000000,0.843137,0.000000}%
\pgfsetfillcolor{currentfill}%
\pgfsetlinewidth{1.003750pt}%
\definecolor{currentstroke}{rgb}{1.000000,0.843137,0.000000}%
\pgfsetstrokecolor{currentstroke}%
\pgfsetdash{}{0pt}%
\pgfpathmoveto{\pgfqpoint{1.277330in}{1.125336in}}%
\pgfpathcurveto{\pgfqpoint{1.288380in}{1.125336in}}{\pgfqpoint{1.298979in}{1.129726in}}{\pgfqpoint{1.306793in}{1.137540in}}%
\pgfpathcurveto{\pgfqpoint{1.314607in}{1.145353in}}{\pgfqpoint{1.318997in}{1.155952in}}{\pgfqpoint{1.318997in}{1.167002in}}%
\pgfpathcurveto{\pgfqpoint{1.318997in}{1.178053in}}{\pgfqpoint{1.314607in}{1.188652in}}{\pgfqpoint{1.306793in}{1.196465in}}%
\pgfpathcurveto{\pgfqpoint{1.298979in}{1.204279in}}{\pgfqpoint{1.288380in}{1.208669in}}{\pgfqpoint{1.277330in}{1.208669in}}%
\pgfpathcurveto{\pgfqpoint{1.266280in}{1.208669in}}{\pgfqpoint{1.255681in}{1.204279in}}{\pgfqpoint{1.247867in}{1.196465in}}%
\pgfpathcurveto{\pgfqpoint{1.240054in}{1.188652in}}{\pgfqpoint{1.235664in}{1.178053in}}{\pgfqpoint{1.235664in}{1.167002in}}%
\pgfpathcurveto{\pgfqpoint{1.235664in}{1.155952in}}{\pgfqpoint{1.240054in}{1.145353in}}{\pgfqpoint{1.247867in}{1.137540in}}%
\pgfpathcurveto{\pgfqpoint{1.255681in}{1.129726in}}{\pgfqpoint{1.266280in}{1.125336in}}{\pgfqpoint{1.277330in}{1.125336in}}%
\pgfpathclose%
\pgfusepath{stroke,fill}%
\end{pgfscope}%
\begin{pgfscope}%
\pgfpathrectangle{\pgfqpoint{0.772069in}{0.515123in}}{\pgfqpoint{3.875000in}{2.695000in}}%
\pgfusepath{clip}%
\pgfsetbuttcap%
\pgfsetroundjoin%
\definecolor{currentfill}{rgb}{1.000000,0.843137,0.000000}%
\pgfsetfillcolor{currentfill}%
\pgfsetlinewidth{1.003750pt}%
\definecolor{currentstroke}{rgb}{1.000000,0.843137,0.000000}%
\pgfsetstrokecolor{currentstroke}%
\pgfsetdash{}{0pt}%
\pgfpathmoveto{\pgfqpoint{1.409167in}{1.399223in}}%
\pgfpathcurveto{\pgfqpoint{1.420217in}{1.399223in}}{\pgfqpoint{1.430816in}{1.403614in}}{\pgfqpoint{1.438629in}{1.411427in}}%
\pgfpathcurveto{\pgfqpoint{1.446443in}{1.419241in}}{\pgfqpoint{1.450833in}{1.429840in}}{\pgfqpoint{1.450833in}{1.440890in}}%
\pgfpathcurveto{\pgfqpoint{1.450833in}{1.451940in}}{\pgfqpoint{1.446443in}{1.462539in}}{\pgfqpoint{1.438629in}{1.470353in}}%
\pgfpathcurveto{\pgfqpoint{1.430816in}{1.478166in}}{\pgfqpoint{1.420217in}{1.482557in}}{\pgfqpoint{1.409167in}{1.482557in}}%
\pgfpathcurveto{\pgfqpoint{1.398116in}{1.482557in}}{\pgfqpoint{1.387517in}{1.478166in}}{\pgfqpoint{1.379704in}{1.470353in}}%
\pgfpathcurveto{\pgfqpoint{1.371890in}{1.462539in}}{\pgfqpoint{1.367500in}{1.451940in}}{\pgfqpoint{1.367500in}{1.440890in}}%
\pgfpathcurveto{\pgfqpoint{1.367500in}{1.429840in}}{\pgfqpoint{1.371890in}{1.419241in}}{\pgfqpoint{1.379704in}{1.411427in}}%
\pgfpathcurveto{\pgfqpoint{1.387517in}{1.403614in}}{\pgfqpoint{1.398116in}{1.399223in}}{\pgfqpoint{1.409167in}{1.399223in}}%
\pgfpathclose%
\pgfusepath{stroke,fill}%
\end{pgfscope}%
\begin{pgfscope}%
\pgfpathrectangle{\pgfqpoint{0.772069in}{0.515123in}}{\pgfqpoint{3.875000in}{2.695000in}}%
\pgfusepath{clip}%
\pgfsetbuttcap%
\pgfsetroundjoin%
\definecolor{currentfill}{rgb}{1.000000,0.843137,0.000000}%
\pgfsetfillcolor{currentfill}%
\pgfsetlinewidth{1.003750pt}%
\definecolor{currentstroke}{rgb}{1.000000,0.843137,0.000000}%
\pgfsetstrokecolor{currentstroke}%
\pgfsetdash{}{0pt}%
\pgfpathmoveto{\pgfqpoint{1.538256in}{1.670452in}}%
\pgfpathcurveto{\pgfqpoint{1.549307in}{1.670452in}}{\pgfqpoint{1.559906in}{1.674842in}}{\pgfqpoint{1.567719in}{1.682655in}}%
\pgfpathcurveto{\pgfqpoint{1.575533in}{1.690469in}}{\pgfqpoint{1.579923in}{1.701068in}}{\pgfqpoint{1.579923in}{1.712118in}}%
\pgfpathcurveto{\pgfqpoint{1.579923in}{1.723168in}}{\pgfqpoint{1.575533in}{1.733767in}}{\pgfqpoint{1.567719in}{1.741581in}}%
\pgfpathcurveto{\pgfqpoint{1.559906in}{1.749395in}}{\pgfqpoint{1.549307in}{1.753785in}}{\pgfqpoint{1.538256in}{1.753785in}}%
\pgfpathcurveto{\pgfqpoint{1.527206in}{1.753785in}}{\pgfqpoint{1.516607in}{1.749395in}}{\pgfqpoint{1.508794in}{1.741581in}}%
\pgfpathcurveto{\pgfqpoint{1.500980in}{1.733767in}}{\pgfqpoint{1.496590in}{1.723168in}}{\pgfqpoint{1.496590in}{1.712118in}}%
\pgfpathcurveto{\pgfqpoint{1.496590in}{1.701068in}}{\pgfqpoint{1.500980in}{1.690469in}}{\pgfqpoint{1.508794in}{1.682655in}}%
\pgfpathcurveto{\pgfqpoint{1.516607in}{1.674842in}}{\pgfqpoint{1.527206in}{1.670452in}}{\pgfqpoint{1.538256in}{1.670452in}}%
\pgfpathclose%
\pgfusepath{stroke,fill}%
\end{pgfscope}%
\begin{pgfscope}%
\pgfpathrectangle{\pgfqpoint{0.772069in}{0.515123in}}{\pgfqpoint{3.875000in}{2.695000in}}%
\pgfusepath{clip}%
\pgfsetbuttcap%
\pgfsetroundjoin%
\definecolor{currentfill}{rgb}{1.000000,0.843137,0.000000}%
\pgfsetfillcolor{currentfill}%
\pgfsetlinewidth{1.003750pt}%
\definecolor{currentstroke}{rgb}{1.000000,0.843137,0.000000}%
\pgfsetstrokecolor{currentstroke}%
\pgfsetdash{}{0pt}%
\pgfpathmoveto{\pgfqpoint{1.664600in}{1.939021in}}%
\pgfpathcurveto{\pgfqpoint{1.675650in}{1.939021in}}{\pgfqpoint{1.686249in}{1.943411in}}{\pgfqpoint{1.694062in}{1.951225in}}%
\pgfpathcurveto{\pgfqpoint{1.701876in}{1.959038in}}{\pgfqpoint{1.706266in}{1.969637in}}{\pgfqpoint{1.706266in}{1.980687in}}%
\pgfpathcurveto{\pgfqpoint{1.706266in}{1.991738in}}{\pgfqpoint{1.701876in}{2.002337in}}{\pgfqpoint{1.694062in}{2.010150in}}%
\pgfpathcurveto{\pgfqpoint{1.686249in}{2.017964in}}{\pgfqpoint{1.675650in}{2.022354in}}{\pgfqpoint{1.664600in}{2.022354in}}%
\pgfpathcurveto{\pgfqpoint{1.653550in}{2.022354in}}{\pgfqpoint{1.642950in}{2.017964in}}{\pgfqpoint{1.635137in}{2.010150in}}%
\pgfpathcurveto{\pgfqpoint{1.627323in}{2.002337in}}{\pgfqpoint{1.622933in}{1.991738in}}{\pgfqpoint{1.622933in}{1.980687in}}%
\pgfpathcurveto{\pgfqpoint{1.622933in}{1.969637in}}{\pgfqpoint{1.627323in}{1.959038in}}{\pgfqpoint{1.635137in}{1.951225in}}%
\pgfpathcurveto{\pgfqpoint{1.642950in}{1.943411in}}{\pgfqpoint{1.653550in}{1.939021in}}{\pgfqpoint{1.664600in}{1.939021in}}%
\pgfpathclose%
\pgfusepath{stroke,fill}%
\end{pgfscope}%
\begin{pgfscope}%
\pgfpathrectangle{\pgfqpoint{0.772069in}{0.515123in}}{\pgfqpoint{3.875000in}{2.695000in}}%
\pgfusepath{clip}%
\pgfsetbuttcap%
\pgfsetroundjoin%
\definecolor{currentfill}{rgb}{1.000000,0.843137,0.000000}%
\pgfsetfillcolor{currentfill}%
\pgfsetlinewidth{1.003750pt}%
\definecolor{currentstroke}{rgb}{1.000000,0.843137,0.000000}%
\pgfsetstrokecolor{currentstroke}%
\pgfsetdash{}{0pt}%
\pgfpathmoveto{\pgfqpoint{1.793689in}{2.210249in}}%
\pgfpathcurveto{\pgfqpoint{1.804740in}{2.210249in}}{\pgfqpoint{1.815339in}{2.214639in}}{\pgfqpoint{1.823152in}{2.222453in}}%
\pgfpathcurveto{\pgfqpoint{1.830966in}{2.230267in}}{\pgfqpoint{1.835356in}{2.240866in}}{\pgfqpoint{1.835356in}{2.251916in}}%
\pgfpathcurveto{\pgfqpoint{1.835356in}{2.262966in}}{\pgfqpoint{1.830966in}{2.273565in}}{\pgfqpoint{1.823152in}{2.281379in}}%
\pgfpathcurveto{\pgfqpoint{1.815339in}{2.289192in}}{\pgfqpoint{1.804740in}{2.293582in}}{\pgfqpoint{1.793689in}{2.293582in}}%
\pgfpathcurveto{\pgfqpoint{1.782639in}{2.293582in}}{\pgfqpoint{1.772040in}{2.289192in}}{\pgfqpoint{1.764227in}{2.281379in}}%
\pgfpathcurveto{\pgfqpoint{1.756413in}{2.273565in}}{\pgfqpoint{1.752023in}{2.262966in}}{\pgfqpoint{1.752023in}{2.251916in}}%
\pgfpathcurveto{\pgfqpoint{1.752023in}{2.240866in}}{\pgfqpoint{1.756413in}{2.230267in}}{\pgfqpoint{1.764227in}{2.222453in}}%
\pgfpathcurveto{\pgfqpoint{1.772040in}{2.214639in}}{\pgfqpoint{1.782639in}{2.210249in}}{\pgfqpoint{1.793689in}{2.210249in}}%
\pgfpathclose%
\pgfusepath{stroke,fill}%
\end{pgfscope}%
\begin{pgfscope}%
\pgfpathrectangle{\pgfqpoint{0.772069in}{0.515123in}}{\pgfqpoint{3.875000in}{2.695000in}}%
\pgfusepath{clip}%
\pgfsetbuttcap%
\pgfsetroundjoin%
\definecolor{currentfill}{rgb}{1.000000,0.843137,0.000000}%
\pgfsetfillcolor{currentfill}%
\pgfsetlinewidth{1.003750pt}%
\definecolor{currentstroke}{rgb}{1.000000,0.843137,0.000000}%
\pgfsetstrokecolor{currentstroke}%
\pgfsetdash{}{0pt}%
\pgfpathmoveto{\pgfqpoint{1.917286in}{2.473500in}}%
\pgfpathcurveto{\pgfqpoint{1.928336in}{2.473500in}}{\pgfqpoint{1.938935in}{2.477890in}}{\pgfqpoint{1.946749in}{2.485704in}}%
\pgfpathcurveto{\pgfqpoint{1.954562in}{2.493518in}}{\pgfqpoint{1.958953in}{2.504117in}}{\pgfqpoint{1.958953in}{2.515167in}}%
\pgfpathcurveto{\pgfqpoint{1.958953in}{2.526217in}}{\pgfqpoint{1.954562in}{2.536816in}}{\pgfqpoint{1.946749in}{2.544630in}}%
\pgfpathcurveto{\pgfqpoint{1.938935in}{2.552443in}}{\pgfqpoint{1.928336in}{2.556833in}}{\pgfqpoint{1.917286in}{2.556833in}}%
\pgfpathcurveto{\pgfqpoint{1.906236in}{2.556833in}}{\pgfqpoint{1.895637in}{2.552443in}}{\pgfqpoint{1.887823in}{2.544630in}}%
\pgfpathcurveto{\pgfqpoint{1.880010in}{2.536816in}}{\pgfqpoint{1.875619in}{2.526217in}}{\pgfqpoint{1.875619in}{2.515167in}}%
\pgfpathcurveto{\pgfqpoint{1.875619in}{2.504117in}}{\pgfqpoint{1.880010in}{2.493518in}}{\pgfqpoint{1.887823in}{2.485704in}}%
\pgfpathcurveto{\pgfqpoint{1.895637in}{2.477890in}}{\pgfqpoint{1.906236in}{2.473500in}}{\pgfqpoint{1.917286in}{2.473500in}}%
\pgfpathclose%
\pgfusepath{stroke,fill}%
\end{pgfscope}%
\begin{pgfscope}%
\pgfpathrectangle{\pgfqpoint{0.772069in}{0.515123in}}{\pgfqpoint{3.875000in}{2.695000in}}%
\pgfusepath{clip}%
\pgfsetbuttcap%
\pgfsetroundjoin%
\definecolor{currentfill}{rgb}{1.000000,0.843137,0.000000}%
\pgfsetfillcolor{currentfill}%
\pgfsetlinewidth{1.003750pt}%
\definecolor{currentstroke}{rgb}{1.000000,0.843137,0.000000}%
\pgfsetstrokecolor{currentstroke}%
\pgfsetdash{}{0pt}%
\pgfpathmoveto{\pgfqpoint{2.043629in}{2.736751in}}%
\pgfpathcurveto{\pgfqpoint{2.054679in}{2.736751in}}{\pgfqpoint{2.065278in}{2.741141in}}{\pgfqpoint{2.073092in}{2.748955in}}%
\pgfpathcurveto{\pgfqpoint{2.080906in}{2.756769in}}{\pgfqpoint{2.085296in}{2.767368in}}{\pgfqpoint{2.085296in}{2.778418in}}%
\pgfpathcurveto{\pgfqpoint{2.085296in}{2.789468in}}{\pgfqpoint{2.080906in}{2.800067in}}{\pgfqpoint{2.073092in}{2.807881in}}%
\pgfpathcurveto{\pgfqpoint{2.065278in}{2.815694in}}{\pgfqpoint{2.054679in}{2.820084in}}{\pgfqpoint{2.043629in}{2.820084in}}%
\pgfpathcurveto{\pgfqpoint{2.032579in}{2.820084in}}{\pgfqpoint{2.021980in}{2.815694in}}{\pgfqpoint{2.014166in}{2.807881in}}%
\pgfpathcurveto{\pgfqpoint{2.006353in}{2.800067in}}{\pgfqpoint{2.001963in}{2.789468in}}{\pgfqpoint{2.001963in}{2.778418in}}%
\pgfpathcurveto{\pgfqpoint{2.001963in}{2.767368in}}{\pgfqpoint{2.006353in}{2.756769in}}{\pgfqpoint{2.014166in}{2.748955in}}%
\pgfpathcurveto{\pgfqpoint{2.021980in}{2.741141in}}{\pgfqpoint{2.032579in}{2.736751in}}{\pgfqpoint{2.043629in}{2.736751in}}%
\pgfpathclose%
\pgfusepath{stroke,fill}%
\end{pgfscope}%
\begin{pgfscope}%
\pgfpathrectangle{\pgfqpoint{0.772069in}{0.515123in}}{\pgfqpoint{3.875000in}{2.695000in}}%
\pgfusepath{clip}%
\pgfsetbuttcap%
\pgfsetroundjoin%
\definecolor{currentfill}{rgb}{1.000000,0.843137,0.000000}%
\pgfsetfillcolor{currentfill}%
\pgfsetlinewidth{1.003750pt}%
\definecolor{currentstroke}{rgb}{1.000000,0.843137,0.000000}%
\pgfsetstrokecolor{currentstroke}%
\pgfsetdash{}{0pt}%
\pgfpathmoveto{\pgfqpoint{2.175466in}{3.015957in}}%
\pgfpathcurveto{\pgfqpoint{2.186516in}{3.015957in}}{\pgfqpoint{2.197115in}{3.020347in}}{\pgfqpoint{2.204928in}{3.028161in}}%
\pgfpathcurveto{\pgfqpoint{2.212742in}{3.035974in}}{\pgfqpoint{2.217132in}{3.046573in}}{\pgfqpoint{2.217132in}{3.057623in}}%
\pgfpathcurveto{\pgfqpoint{2.217132in}{3.068673in}}{\pgfqpoint{2.212742in}{3.079273in}}{\pgfqpoint{2.204928in}{3.087086in}}%
\pgfpathcurveto{\pgfqpoint{2.197115in}{3.094900in}}{\pgfqpoint{2.186516in}{3.099290in}}{\pgfqpoint{2.175466in}{3.099290in}}%
\pgfpathcurveto{\pgfqpoint{2.164416in}{3.099290in}}{\pgfqpoint{2.153817in}{3.094900in}}{\pgfqpoint{2.146003in}{3.087086in}}%
\pgfpathcurveto{\pgfqpoint{2.138189in}{3.079273in}}{\pgfqpoint{2.133799in}{3.068673in}}{\pgfqpoint{2.133799in}{3.057623in}}%
\pgfpathcurveto{\pgfqpoint{2.133799in}{3.046573in}}{\pgfqpoint{2.138189in}{3.035974in}}{\pgfqpoint{2.146003in}{3.028161in}}%
\pgfpathcurveto{\pgfqpoint{2.153817in}{3.020347in}}{\pgfqpoint{2.164416in}{3.015957in}}{\pgfqpoint{2.175466in}{3.015957in}}%
\pgfpathclose%
\pgfusepath{stroke,fill}%
\end{pgfscope}%
\begin{pgfscope}%
\pgfpathrectangle{\pgfqpoint{0.772069in}{0.515123in}}{\pgfqpoint{3.875000in}{2.695000in}}%
\pgfusepath{clip}%
\pgfsetbuttcap%
\pgfsetroundjoin%
\definecolor{currentfill}{rgb}{0.196078,0.803922,0.196078}%
\pgfsetfillcolor{currentfill}%
\pgfsetlinewidth{1.003750pt}%
\definecolor{currentstroke}{rgb}{0.196078,0.803922,0.196078}%
\pgfsetstrokecolor{currentstroke}%
\pgfsetdash{}{0pt}%
\pgfpathmoveto{\pgfqpoint{0.977952in}{0.625957in}}%
\pgfpathcurveto{\pgfqpoint{0.989002in}{0.625957in}}{\pgfqpoint{0.999601in}{0.630347in}}{\pgfqpoint{1.007415in}{0.638161in}}%
\pgfpathcurveto{\pgfqpoint{1.015228in}{0.645974in}}{\pgfqpoint{1.019618in}{0.656573in}}{\pgfqpoint{1.019618in}{0.667623in}}%
\pgfpathcurveto{\pgfqpoint{1.019618in}{0.678673in}}{\pgfqpoint{1.015228in}{0.689272in}}{\pgfqpoint{1.007415in}{0.697086in}}%
\pgfpathcurveto{\pgfqpoint{0.999601in}{0.704900in}}{\pgfqpoint{0.989002in}{0.709290in}}{\pgfqpoint{0.977952in}{0.709290in}}%
\pgfpathcurveto{\pgfqpoint{0.966902in}{0.709290in}}{\pgfqpoint{0.956303in}{0.704900in}}{\pgfqpoint{0.948489in}{0.697086in}}%
\pgfpathcurveto{\pgfqpoint{0.940675in}{0.689272in}}{\pgfqpoint{0.936285in}{0.678673in}}{\pgfqpoint{0.936285in}{0.667623in}}%
\pgfpathcurveto{\pgfqpoint{0.936285in}{0.656573in}}{\pgfqpoint{0.940675in}{0.645974in}}{\pgfqpoint{0.948489in}{0.638161in}}%
\pgfpathcurveto{\pgfqpoint{0.956303in}{0.630347in}}{\pgfqpoint{0.966902in}{0.625957in}}{\pgfqpoint{0.977952in}{0.625957in}}%
\pgfpathclose%
\pgfusepath{stroke,fill}%
\end{pgfscope}%
\begin{pgfscope}%
\pgfpathrectangle{\pgfqpoint{0.772069in}{0.515123in}}{\pgfqpoint{3.875000in}{2.695000in}}%
\pgfusepath{clip}%
\pgfsetbuttcap%
\pgfsetroundjoin%
\definecolor{currentfill}{rgb}{0.196078,0.803922,0.196078}%
\pgfsetfillcolor{currentfill}%
\pgfsetlinewidth{1.003750pt}%
\definecolor{currentstroke}{rgb}{0.196078,0.803922,0.196078}%
\pgfsetstrokecolor{currentstroke}%
\pgfsetdash{}{0pt}%
\pgfpathmoveto{\pgfqpoint{1.043870in}{0.875380in}}%
\pgfpathcurveto{\pgfqpoint{1.054920in}{0.875380in}}{\pgfqpoint{1.065519in}{0.879771in}}{\pgfqpoint{1.073333in}{0.887584in}}%
\pgfpathcurveto{\pgfqpoint{1.081146in}{0.895398in}}{\pgfqpoint{1.085537in}{0.905997in}}{\pgfqpoint{1.085537in}{0.917047in}}%
\pgfpathcurveto{\pgfqpoint{1.085537in}{0.928097in}}{\pgfqpoint{1.081146in}{0.938696in}}{\pgfqpoint{1.073333in}{0.946510in}}%
\pgfpathcurveto{\pgfqpoint{1.065519in}{0.954323in}}{\pgfqpoint{1.054920in}{0.958714in}}{\pgfqpoint{1.043870in}{0.958714in}}%
\pgfpathcurveto{\pgfqpoint{1.032820in}{0.958714in}}{\pgfqpoint{1.022221in}{0.954323in}}{\pgfqpoint{1.014407in}{0.946510in}}%
\pgfpathcurveto{\pgfqpoint{1.006594in}{0.938696in}}{\pgfqpoint{1.002203in}{0.928097in}}{\pgfqpoint{1.002203in}{0.917047in}}%
\pgfpathcurveto{\pgfqpoint{1.002203in}{0.905997in}}{\pgfqpoint{1.006594in}{0.895398in}}{\pgfqpoint{1.014407in}{0.887584in}}%
\pgfpathcurveto{\pgfqpoint{1.022221in}{0.879771in}}{\pgfqpoint{1.032820in}{0.875380in}}{\pgfqpoint{1.043870in}{0.875380in}}%
\pgfpathclose%
\pgfusepath{stroke,fill}%
\end{pgfscope}%
\begin{pgfscope}%
\pgfpathrectangle{\pgfqpoint{0.772069in}{0.515123in}}{\pgfqpoint{3.875000in}{2.695000in}}%
\pgfusepath{clip}%
\pgfsetbuttcap%
\pgfsetroundjoin%
\definecolor{currentfill}{rgb}{0.196078,0.803922,0.196078}%
\pgfsetfillcolor{currentfill}%
\pgfsetlinewidth{1.003750pt}%
\definecolor{currentstroke}{rgb}{0.196078,0.803922,0.196078}%
\pgfsetstrokecolor{currentstroke}%
\pgfsetdash{}{0pt}%
\pgfpathmoveto{\pgfqpoint{1.104295in}{1.125336in}}%
\pgfpathcurveto{\pgfqpoint{1.115345in}{1.125336in}}{\pgfqpoint{1.125944in}{1.129726in}}{\pgfqpoint{1.133758in}{1.137540in}}%
\pgfpathcurveto{\pgfqpoint{1.141571in}{1.145353in}}{\pgfqpoint{1.145962in}{1.155952in}}{\pgfqpoint{1.145962in}{1.167002in}}%
\pgfpathcurveto{\pgfqpoint{1.145962in}{1.178053in}}{\pgfqpoint{1.141571in}{1.188652in}}{\pgfqpoint{1.133758in}{1.196465in}}%
\pgfpathcurveto{\pgfqpoint{1.125944in}{1.204279in}}{\pgfqpoint{1.115345in}{1.208669in}}{\pgfqpoint{1.104295in}{1.208669in}}%
\pgfpathcurveto{\pgfqpoint{1.093245in}{1.208669in}}{\pgfqpoint{1.082646in}{1.204279in}}{\pgfqpoint{1.074832in}{1.196465in}}%
\pgfpathcurveto{\pgfqpoint{1.067019in}{1.188652in}}{\pgfqpoint{1.062628in}{1.178053in}}{\pgfqpoint{1.062628in}{1.167002in}}%
\pgfpathcurveto{\pgfqpoint{1.062628in}{1.155952in}}{\pgfqpoint{1.067019in}{1.145353in}}{\pgfqpoint{1.074832in}{1.137540in}}%
\pgfpathcurveto{\pgfqpoint{1.082646in}{1.129726in}}{\pgfqpoint{1.093245in}{1.125336in}}{\pgfqpoint{1.104295in}{1.125336in}}%
\pgfpathclose%
\pgfusepath{stroke,fill}%
\end{pgfscope}%
\begin{pgfscope}%
\pgfpathrectangle{\pgfqpoint{0.772069in}{0.515123in}}{\pgfqpoint{3.875000in}{2.695000in}}%
\pgfusepath{clip}%
\pgfsetbuttcap%
\pgfsetroundjoin%
\definecolor{currentfill}{rgb}{0.196078,0.803922,0.196078}%
\pgfsetfillcolor{currentfill}%
\pgfsetlinewidth{1.003750pt}%
\definecolor{currentstroke}{rgb}{0.196078,0.803922,0.196078}%
\pgfsetstrokecolor{currentstroke}%
\pgfsetdash{}{0pt}%
\pgfpathmoveto{\pgfqpoint{1.181200in}{1.399223in}}%
\pgfpathcurveto{\pgfqpoint{1.192250in}{1.399223in}}{\pgfqpoint{1.202849in}{1.403614in}}{\pgfqpoint{1.210662in}{1.411427in}}%
\pgfpathcurveto{\pgfqpoint{1.218476in}{1.419241in}}{\pgfqpoint{1.222866in}{1.429840in}}{\pgfqpoint{1.222866in}{1.440890in}}%
\pgfpathcurveto{\pgfqpoint{1.222866in}{1.451940in}}{\pgfqpoint{1.218476in}{1.462539in}}{\pgfqpoint{1.210662in}{1.470353in}}%
\pgfpathcurveto{\pgfqpoint{1.202849in}{1.478166in}}{\pgfqpoint{1.192250in}{1.482557in}}{\pgfqpoint{1.181200in}{1.482557in}}%
\pgfpathcurveto{\pgfqpoint{1.170149in}{1.482557in}}{\pgfqpoint{1.159550in}{1.478166in}}{\pgfqpoint{1.151737in}{1.470353in}}%
\pgfpathcurveto{\pgfqpoint{1.143923in}{1.462539in}}{\pgfqpoint{1.139533in}{1.451940in}}{\pgfqpoint{1.139533in}{1.440890in}}%
\pgfpathcurveto{\pgfqpoint{1.139533in}{1.429840in}}{\pgfqpoint{1.143923in}{1.419241in}}{\pgfqpoint{1.151737in}{1.411427in}}%
\pgfpathcurveto{\pgfqpoint{1.159550in}{1.403614in}}{\pgfqpoint{1.170149in}{1.399223in}}{\pgfqpoint{1.181200in}{1.399223in}}%
\pgfpathclose%
\pgfusepath{stroke,fill}%
\end{pgfscope}%
\begin{pgfscope}%
\pgfpathrectangle{\pgfqpoint{0.772069in}{0.515123in}}{\pgfqpoint{3.875000in}{2.695000in}}%
\pgfusepath{clip}%
\pgfsetbuttcap%
\pgfsetroundjoin%
\definecolor{currentfill}{rgb}{0.196078,0.803922,0.196078}%
\pgfsetfillcolor{currentfill}%
\pgfsetlinewidth{1.003750pt}%
\definecolor{currentstroke}{rgb}{0.196078,0.803922,0.196078}%
\pgfsetstrokecolor{currentstroke}%
\pgfsetdash{}{0pt}%
\pgfpathmoveto{\pgfqpoint{1.249864in}{1.670452in}}%
\pgfpathcurveto{\pgfqpoint{1.260914in}{1.670452in}}{\pgfqpoint{1.271513in}{1.674842in}}{\pgfqpoint{1.279327in}{1.682655in}}%
\pgfpathcurveto{\pgfqpoint{1.287141in}{1.690469in}}{\pgfqpoint{1.291531in}{1.701068in}}{\pgfqpoint{1.291531in}{1.712118in}}%
\pgfpathcurveto{\pgfqpoint{1.291531in}{1.723168in}}{\pgfqpoint{1.287141in}{1.733767in}}{\pgfqpoint{1.279327in}{1.741581in}}%
\pgfpathcurveto{\pgfqpoint{1.271513in}{1.749395in}}{\pgfqpoint{1.260914in}{1.753785in}}{\pgfqpoint{1.249864in}{1.753785in}}%
\pgfpathcurveto{\pgfqpoint{1.238814in}{1.753785in}}{\pgfqpoint{1.228215in}{1.749395in}}{\pgfqpoint{1.220402in}{1.741581in}}%
\pgfpathcurveto{\pgfqpoint{1.212588in}{1.733767in}}{\pgfqpoint{1.208198in}{1.723168in}}{\pgfqpoint{1.208198in}{1.712118in}}%
\pgfpathcurveto{\pgfqpoint{1.208198in}{1.701068in}}{\pgfqpoint{1.212588in}{1.690469in}}{\pgfqpoint{1.220402in}{1.682655in}}%
\pgfpathcurveto{\pgfqpoint{1.228215in}{1.674842in}}{\pgfqpoint{1.238814in}{1.670452in}}{\pgfqpoint{1.249864in}{1.670452in}}%
\pgfpathclose%
\pgfusepath{stroke,fill}%
\end{pgfscope}%
\begin{pgfscope}%
\pgfpathrectangle{\pgfqpoint{0.772069in}{0.515123in}}{\pgfqpoint{3.875000in}{2.695000in}}%
\pgfusepath{clip}%
\pgfsetbuttcap%
\pgfsetroundjoin%
\definecolor{currentfill}{rgb}{0.196078,0.803922,0.196078}%
\pgfsetfillcolor{currentfill}%
\pgfsetlinewidth{1.003750pt}%
\definecolor{currentstroke}{rgb}{0.196078,0.803922,0.196078}%
\pgfsetstrokecolor{currentstroke}%
\pgfsetdash{}{0pt}%
\pgfpathmoveto{\pgfqpoint{1.321276in}{1.939021in}}%
\pgfpathcurveto{\pgfqpoint{1.332326in}{1.939021in}}{\pgfqpoint{1.342925in}{1.943411in}}{\pgfqpoint{1.350738in}{1.951225in}}%
\pgfpathcurveto{\pgfqpoint{1.358552in}{1.959038in}}{\pgfqpoint{1.362942in}{1.969637in}}{\pgfqpoint{1.362942in}{1.980687in}}%
\pgfpathcurveto{\pgfqpoint{1.362942in}{1.991738in}}{\pgfqpoint{1.358552in}{2.002337in}}{\pgfqpoint{1.350738in}{2.010150in}}%
\pgfpathcurveto{\pgfqpoint{1.342925in}{2.017964in}}{\pgfqpoint{1.332326in}{2.022354in}}{\pgfqpoint{1.321276in}{2.022354in}}%
\pgfpathcurveto{\pgfqpoint{1.310226in}{2.022354in}}{\pgfqpoint{1.299627in}{2.017964in}}{\pgfqpoint{1.291813in}{2.010150in}}%
\pgfpathcurveto{\pgfqpoint{1.283999in}{2.002337in}}{\pgfqpoint{1.279609in}{1.991738in}}{\pgfqpoint{1.279609in}{1.980687in}}%
\pgfpathcurveto{\pgfqpoint{1.279609in}{1.969637in}}{\pgfqpoint{1.283999in}{1.959038in}}{\pgfqpoint{1.291813in}{1.951225in}}%
\pgfpathcurveto{\pgfqpoint{1.299627in}{1.943411in}}{\pgfqpoint{1.310226in}{1.939021in}}{\pgfqpoint{1.321276in}{1.939021in}}%
\pgfpathclose%
\pgfusepath{stroke,fill}%
\end{pgfscope}%
\begin{pgfscope}%
\pgfpathrectangle{\pgfqpoint{0.772069in}{0.515123in}}{\pgfqpoint{3.875000in}{2.695000in}}%
\pgfusepath{clip}%
\pgfsetbuttcap%
\pgfsetroundjoin%
\definecolor{currentfill}{rgb}{0.196078,0.803922,0.196078}%
\pgfsetfillcolor{currentfill}%
\pgfsetlinewidth{1.003750pt}%
\definecolor{currentstroke}{rgb}{0.196078,0.803922,0.196078}%
\pgfsetstrokecolor{currentstroke}%
\pgfsetdash{}{0pt}%
\pgfpathmoveto{\pgfqpoint{1.392687in}{2.210249in}}%
\pgfpathcurveto{\pgfqpoint{1.403737in}{2.210249in}}{\pgfqpoint{1.414336in}{2.214639in}}{\pgfqpoint{1.422150in}{2.222453in}}%
\pgfpathcurveto{\pgfqpoint{1.429963in}{2.230267in}}{\pgfqpoint{1.434354in}{2.240866in}}{\pgfqpoint{1.434354in}{2.251916in}}%
\pgfpathcurveto{\pgfqpoint{1.434354in}{2.262966in}}{\pgfqpoint{1.429963in}{2.273565in}}{\pgfqpoint{1.422150in}{2.281379in}}%
\pgfpathcurveto{\pgfqpoint{1.414336in}{2.289192in}}{\pgfqpoint{1.403737in}{2.293582in}}{\pgfqpoint{1.392687in}{2.293582in}}%
\pgfpathcurveto{\pgfqpoint{1.381637in}{2.293582in}}{\pgfqpoint{1.371038in}{2.289192in}}{\pgfqpoint{1.363224in}{2.281379in}}%
\pgfpathcurveto{\pgfqpoint{1.355411in}{2.273565in}}{\pgfqpoint{1.351020in}{2.262966in}}{\pgfqpoint{1.351020in}{2.251916in}}%
\pgfpathcurveto{\pgfqpoint{1.351020in}{2.240866in}}{\pgfqpoint{1.355411in}{2.230267in}}{\pgfqpoint{1.363224in}{2.222453in}}%
\pgfpathcurveto{\pgfqpoint{1.371038in}{2.214639in}}{\pgfqpoint{1.381637in}{2.210249in}}{\pgfqpoint{1.392687in}{2.210249in}}%
\pgfpathclose%
\pgfusepath{stroke,fill}%
\end{pgfscope}%
\begin{pgfscope}%
\pgfpathrectangle{\pgfqpoint{0.772069in}{0.515123in}}{\pgfqpoint{3.875000in}{2.695000in}}%
\pgfusepath{clip}%
\pgfsetbuttcap%
\pgfsetroundjoin%
\definecolor{currentfill}{rgb}{0.196078,0.803922,0.196078}%
\pgfsetfillcolor{currentfill}%
\pgfsetlinewidth{1.003750pt}%
\definecolor{currentstroke}{rgb}{0.196078,0.803922,0.196078}%
\pgfsetstrokecolor{currentstroke}%
\pgfsetdash{}{0pt}%
\pgfpathmoveto{\pgfqpoint{1.461352in}{2.473500in}}%
\pgfpathcurveto{\pgfqpoint{1.472402in}{2.473500in}}{\pgfqpoint{1.483001in}{2.477890in}}{\pgfqpoint{1.490815in}{2.485704in}}%
\pgfpathcurveto{\pgfqpoint{1.498628in}{2.493518in}}{\pgfqpoint{1.503019in}{2.504117in}}{\pgfqpoint{1.503019in}{2.515167in}}%
\pgfpathcurveto{\pgfqpoint{1.503019in}{2.526217in}}{\pgfqpoint{1.498628in}{2.536816in}}{\pgfqpoint{1.490815in}{2.544630in}}%
\pgfpathcurveto{\pgfqpoint{1.483001in}{2.552443in}}{\pgfqpoint{1.472402in}{2.556833in}}{\pgfqpoint{1.461352in}{2.556833in}}%
\pgfpathcurveto{\pgfqpoint{1.450302in}{2.556833in}}{\pgfqpoint{1.439703in}{2.552443in}}{\pgfqpoint{1.431889in}{2.544630in}}%
\pgfpathcurveto{\pgfqpoint{1.424075in}{2.536816in}}{\pgfqpoint{1.419685in}{2.526217in}}{\pgfqpoint{1.419685in}{2.515167in}}%
\pgfpathcurveto{\pgfqpoint{1.419685in}{2.504117in}}{\pgfqpoint{1.424075in}{2.493518in}}{\pgfqpoint{1.431889in}{2.485704in}}%
\pgfpathcurveto{\pgfqpoint{1.439703in}{2.477890in}}{\pgfqpoint{1.450302in}{2.473500in}}{\pgfqpoint{1.461352in}{2.473500in}}%
\pgfpathclose%
\pgfusepath{stroke,fill}%
\end{pgfscope}%
\begin{pgfscope}%
\pgfpathrectangle{\pgfqpoint{0.772069in}{0.515123in}}{\pgfqpoint{3.875000in}{2.695000in}}%
\pgfusepath{clip}%
\pgfsetbuttcap%
\pgfsetroundjoin%
\definecolor{currentfill}{rgb}{0.196078,0.803922,0.196078}%
\pgfsetfillcolor{currentfill}%
\pgfsetlinewidth{1.003750pt}%
\definecolor{currentstroke}{rgb}{0.196078,0.803922,0.196078}%
\pgfsetstrokecolor{currentstroke}%
\pgfsetdash{}{0pt}%
\pgfpathmoveto{\pgfqpoint{1.530017in}{2.736751in}}%
\pgfpathcurveto{\pgfqpoint{1.541067in}{2.736751in}}{\pgfqpoint{1.551666in}{2.741141in}}{\pgfqpoint{1.559479in}{2.748955in}}%
\pgfpathcurveto{\pgfqpoint{1.567293in}{2.756769in}}{\pgfqpoint{1.571683in}{2.767368in}}{\pgfqpoint{1.571683in}{2.778418in}}%
\pgfpathcurveto{\pgfqpoint{1.571683in}{2.789468in}}{\pgfqpoint{1.567293in}{2.800067in}}{\pgfqpoint{1.559479in}{2.807881in}}%
\pgfpathcurveto{\pgfqpoint{1.551666in}{2.815694in}}{\pgfqpoint{1.541067in}{2.820084in}}{\pgfqpoint{1.530017in}{2.820084in}}%
\pgfpathcurveto{\pgfqpoint{1.518967in}{2.820084in}}{\pgfqpoint{1.508367in}{2.815694in}}{\pgfqpoint{1.500554in}{2.807881in}}%
\pgfpathcurveto{\pgfqpoint{1.492740in}{2.800067in}}{\pgfqpoint{1.488350in}{2.789468in}}{\pgfqpoint{1.488350in}{2.778418in}}%
\pgfpathcurveto{\pgfqpoint{1.488350in}{2.767368in}}{\pgfqpoint{1.492740in}{2.756769in}}{\pgfqpoint{1.500554in}{2.748955in}}%
\pgfpathcurveto{\pgfqpoint{1.508367in}{2.741141in}}{\pgfqpoint{1.518967in}{2.736751in}}{\pgfqpoint{1.530017in}{2.736751in}}%
\pgfpathclose%
\pgfusepath{stroke,fill}%
\end{pgfscope}%
\begin{pgfscope}%
\pgfpathrectangle{\pgfqpoint{0.772069in}{0.515123in}}{\pgfqpoint{3.875000in}{2.695000in}}%
\pgfusepath{clip}%
\pgfsetbuttcap%
\pgfsetroundjoin%
\definecolor{currentfill}{rgb}{0.196078,0.803922,0.196078}%
\pgfsetfillcolor{currentfill}%
\pgfsetlinewidth{1.003750pt}%
\definecolor{currentstroke}{rgb}{0.196078,0.803922,0.196078}%
\pgfsetstrokecolor{currentstroke}%
\pgfsetdash{}{0pt}%
\pgfpathmoveto{\pgfqpoint{1.604175in}{3.015957in}}%
\pgfpathcurveto{\pgfqpoint{1.615225in}{3.015957in}}{\pgfqpoint{1.625824in}{3.020347in}}{\pgfqpoint{1.633637in}{3.028161in}}%
\pgfpathcurveto{\pgfqpoint{1.641451in}{3.035974in}}{\pgfqpoint{1.645841in}{3.046573in}}{\pgfqpoint{1.645841in}{3.057623in}}%
\pgfpathcurveto{\pgfqpoint{1.645841in}{3.068673in}}{\pgfqpoint{1.641451in}{3.079273in}}{\pgfqpoint{1.633637in}{3.087086in}}%
\pgfpathcurveto{\pgfqpoint{1.625824in}{3.094900in}}{\pgfqpoint{1.615225in}{3.099290in}}{\pgfqpoint{1.604175in}{3.099290in}}%
\pgfpathcurveto{\pgfqpoint{1.593124in}{3.099290in}}{\pgfqpoint{1.582525in}{3.094900in}}{\pgfqpoint{1.574712in}{3.087086in}}%
\pgfpathcurveto{\pgfqpoint{1.566898in}{3.079273in}}{\pgfqpoint{1.562508in}{3.068673in}}{\pgfqpoint{1.562508in}{3.057623in}}%
\pgfpathcurveto{\pgfqpoint{1.562508in}{3.046573in}}{\pgfqpoint{1.566898in}{3.035974in}}{\pgfqpoint{1.574712in}{3.028161in}}%
\pgfpathcurveto{\pgfqpoint{1.582525in}{3.020347in}}{\pgfqpoint{1.593124in}{3.015957in}}{\pgfqpoint{1.604175in}{3.015957in}}%
\pgfpathclose%
\pgfusepath{stroke,fill}%
\end{pgfscope}%
\begin{pgfscope}%
\pgfsetrectcap%
\pgfsetmiterjoin%
\pgfsetlinewidth{0.803000pt}%
\definecolor{currentstroke}{rgb}{0.000000,0.000000,0.000000}%
\pgfsetstrokecolor{currentstroke}%
\pgfsetdash{}{0pt}%
\pgfpathmoveto{\pgfqpoint{0.772069in}{0.515123in}}%
\pgfpathlineto{\pgfqpoint{0.772069in}{3.210123in}}%
\pgfusepath{stroke}%
\end{pgfscope}%
\begin{pgfscope}%
\pgfsetrectcap%
\pgfsetmiterjoin%
\pgfsetlinewidth{0.803000pt}%
\definecolor{currentstroke}{rgb}{0.000000,0.000000,0.000000}%
\pgfsetstrokecolor{currentstroke}%
\pgfsetdash{}{0pt}%
\pgfpathmoveto{\pgfqpoint{4.647069in}{0.515123in}}%
\pgfpathlineto{\pgfqpoint{4.647069in}{3.210123in}}%
\pgfusepath{stroke}%
\end{pgfscope}%
\begin{pgfscope}%
\pgfsetrectcap%
\pgfsetmiterjoin%
\pgfsetlinewidth{0.803000pt}%
\definecolor{currentstroke}{rgb}{0.000000,0.000000,0.000000}%
\pgfsetstrokecolor{currentstroke}%
\pgfsetdash{}{0pt}%
\pgfpathmoveto{\pgfqpoint{0.772069in}{0.515123in}}%
\pgfpathlineto{\pgfqpoint{4.647069in}{0.515123in}}%
\pgfusepath{stroke}%
\end{pgfscope}%
\begin{pgfscope}%
\pgfsetrectcap%
\pgfsetmiterjoin%
\pgfsetlinewidth{0.803000pt}%
\definecolor{currentstroke}{rgb}{0.000000,0.000000,0.000000}%
\pgfsetstrokecolor{currentstroke}%
\pgfsetdash{}{0pt}%
\pgfpathmoveto{\pgfqpoint{0.772069in}{3.210123in}}%
\pgfpathlineto{\pgfqpoint{4.647069in}{3.210123in}}%
\pgfusepath{stroke}%
\end{pgfscope}%
\end{pgfpicture}%
\makeatother%
\endgroup%

    \caption{Voltajes (V) frente a intensidades (\textcolor{Blue}{$I$}, \textcolor{Red}{$I_1$}, \textcolor{Yellow}{$I_2$}, \textcolor{Green}{$I_3$}) con regresión lineal}
  \end{figure}

  Calculamos los coeficientes de regresión lineal con la fórmula \ref{ec:r} y los resultados son adecuados: $r = 0,999992$, $r_1 = 0,999997$, $r_2 = 0,999998$ y $r_3 = 0,99997$. Excepto $r_3$, los resultados tienen un ajuste excelente de cinco nueves. En el caso de $r_3$ son cuatro nueves, lo cual también es satisfactorio.

  Comprobamos también si el valor de la resistencia equivalente calculado en \ref{sec:reseqparalelo} ($R_P = 7,7726 \cdot 10^4$) se corresponde con la pendiente de la gráfica ($b = 7,83 \cdot 10^4$), y vemos que son muy similares.


  \newpage
  \section{Circuito Mixto}

  La última práctica de corriente continua comprende la construcción de un circuito mixto utilizando las cuatro resistencias ($R_1$, $R_2$, $R_3$ y $R_4$). Colocaremos $R_2$ y $R_3$ en paralelo entre sí, en un grupo llamado $R_{23}$, y luego conectaremos en serie $R_1$, $R_{23}$ y $R_4$. Podemos ver la disposición en el siguiente diagrama:

  \begin{figure}[H]
    \centering
    \begin{circuitikz}[european]
      \draw (0,0) to[voltage source] (0,4)
      to[R=$R_1$] (4,4) -- (4,3.5);
      \draw (4,0.5) -- (4,0)
      to[R=$R_4$] (0,0);
      \draw (4,3.5) -- (3.25,3.5)
      to[R=$R_2$] (3.25,0.5) -- (4,0.5);
      \draw (4,3.5) -- (4.75,3.5)
      to[R=$R_3$] (4.75,0.5) -- (4,0.5);
    \end{circuitikz}
    \caption{Circuito mixto}
    \label{circuito:mixto}
  \end{figure}

  \subsection{Procedimiento de medición}

  Para medir los diferentes valores que se piden, tenemos que utilizar ambas técnicas que aprendimos en los circuitos anteriores. Primero, trataremos a las dos resistencias $R_2$ y $R_3$ como un circuito aislado en paralelo, y así determinaremos su voltaje total ($V_{23}$) y cada una de sus intensidades ($I_1$ y $I_2$).

  \begin{figure}[H]
    \centering
    \begin{circuitikz}[european]
      \draw (0,0) to[voltage source] (0,4)
      to[R=$R_1$] (4,4) -- (4,3.5);
      \draw (4,0.5) -- (4,0)
      to[R=$R_4$] (0,0);
      \draw (4,3.5) -- (3.25,3.5)
      to[R=$R_2$] (3.25,0.5) -- (4,0.5);
      \draw (4,3.5) -- (4.75,3.5)
      to[R=$R_3$] (4.75,0.5) -- (4,0.5);
      \draw (4,4) -- (6.5,4)
      to[voltmeter, label=$V_{23}$] (6.5,0) -- (4,0);
    \end{circuitikz} \qquad
    \begin{circuitikz}[european]
      \draw (0,0) to[voltage source] (0,4)
      to[R=$R_1$] (4,4) -- (4,3.5);
      \draw (4,0.5) -- (4,0)
      to[R=$R_4$] (0,0);
      \draw (4,3.5) -- (3.25,3.5)
      to[R=$R_2$] (3.25,2.25)
      to[ammeter, label=$I_1$] (3.25,0.5) -- (4,0.5);
      \draw (4,3.5) -- (4.75,3.5)
      to[R=$R_3$] (4.75,2.25)
      to[ammeter, label=$I_2$] (4.75,0.5) -- (4,0.5);
    \end{circuitikz}
    \caption{Medición de potencial ($V_{23}$) y intensidades ($I_1$ y $I_2$) de las resistencias $R_2$ y $R_3$ en paralelo dentro del circuito mixto}
  \end{figure}

  Ahora que tenemos esos datos, trataremos a la unión de las resistencias $R_2$ y $R_3$ como un componente más de un circuito en serie, en conjunto con las otras resistencias, $R_1$ y $R_4$. Mediremos los voltajes de estas últimas resitencias en paralelo ($V_1$ y $V_4$). Luego tomaremos el voltaje de todo el circuito también en paralelo ($V$) y la intensidad del mismo, en serie ($I$).

  \begin{figure}[H]
    \centering
    \begin{circuitikz}[european]
      \draw (0,0) to[ammeter,label=$I$] (0,2) to[voltage source] (0,4)
      to[R=$R_1$] (4,4) -- (4,3.5);
      \draw (4,0.5) -- (4,0)
      to[R, l_=$R_4$] (0,0);
      \draw (4,3.5) -- (3.25,3.5)
      to[R=$R_2$] (3.25,0.5) -- (4,0.5);
      \draw (4,3.5) -- (4.75,3.5)
      to[R=$R_3$] (4.75,0.5) -- (4,0.5);
      \draw (0,4) -- (0,5.5)
      to[voltmeter,label=$V_1$] (4,5.5) -- (4,4);
      \draw (0,0) -- (0,-1.5)
      to[voltmeter, label=$V_4$] (4,-1.5) -- (4,0);
      \draw (0,0) -- (-1.5,0)
      to[voltmeter, label=$V$] (-1.5,4) -- (0,4);
    \end{circuitikz}
    \caption{Medición de potenciales ($V_1$ y $V_4$) de las resistencias $R_1$ y $R_4$ en serie y del pontencial ($V$) e intensidad ($I$) total del circuito mixto}
  \end{figure}

  \subsection{Reistencia equivalente}
  \label{sec:reseqmixto}

  Para calcular la resistencia equivalente del circuito, primero utilizaremos la ecuación \ref{ec:resparalelo} para calcular la de las dos resistencias en paralelo $R_2$ y $R_3$, a la que llamaremos $R_{23}$. Posteriormente, aplicamos la ecuación \ref{ec:reseq} a $R_1$, $R_{23}$ y $R_4$, que están en serie. Así obtenemos la reistencia total.
  \begin{gather}
    R_{23} = \frac{1}{\frac{1}{216000} + \frac{1}{394000}} = 1,3951\cdot10^5\Omega \nonumber \\
    s(R_{23}) = \sqrt{\left ( \frac{\partial R_{23}}{\partial R_2} \right )^2 s^2(R_2) + \left ( \frac{\partial R_{23}}{\partial R_3} \right )^2 s^2(R_3)} \nonumber \\
    s(R_{23}) = \sqrt{\left ( \frac{R_3^2}{(R_2+R_3)^2} \right )^2 s^2(R_2) + \left ( \frac{R_2^2}{(R_2+R_3)^2} \right )^2 s^2(R_3)} = \pm 4,4\cdot 10^2\Omega \nonumber \\
    R = 175500 + 139510 + 1006000 = 1,3211\cdot10^6\Omega \nonumber \quad s(R) = 100 + 440 + 1000 = 1,54\cdot 10^3\Omega \nonumber \\
    R = 1,3210\cdot10^6 \pm 1,54\cdot 10^3\Omega
  \end{gather}

  Si calculamos experimentalmente el valor de la resistencia total del circuito obtenemos $R_E = 1,324 \cdot 10^6 \pm 10^3$, lo cual entra en nuestro intervalo de confianza, así que consideraremos ambas medidas correctas.

  \subsection{Medición experimental}

  Construímos el circuito \ref{circuito:mixto} y tomamos las medidas descritas con anterioridad. Variaremos el voltaje (V) desde la fuente de alimentación para obtener distintos valores a medir. Exponemos el resultado en la siguiente tabla dónde, como en la sección anterior, omitiremos las incertidubres de las intensidades ($s(I)$) por ser todas iguales $s(I) = s(I_1) = s(I_2) = 1 \cdot 10^-7 A$, con el objetivo de hacer la tabla más compacta.

  \begin{table}[H]
  \centering
  \resizebox{\columnwidth}{!}{
  \csvreader[
    tabular=|c|c|c|c|c|c|c|c|,
    table head=\hline Medida & $V~(V) \pm s(V)$ & $V_1~(V) \pm s(V_1)$ & $V_{s3}~(V) \pm s(V_{23})$ & $V_4~(V) \pm s(V_4)$ & $I_1~(V)$ & $I_2~(V)$ & $I~(A)$ \\ \hline,
    late after last line=\\\hline,
    separator=semicolon
    ]{CC8.csv}
    {v=\v, sv=\sv, v1=\va, sv1=\sva, v23=\vb, sv23=\svb, v4=\vc, sv4=\svc, i1=\ia, i2=\ib, i=\int}
    {\thecsvrow & \v \hspace{4pt}$\pm$ \sv & \va \hspace{4pt}$\pm$ \sva & \vb \hspace{4pt}$\pm$ \svb & \vc \hspace{4pt}$\pm$ \svc & \ia & \ib & \int }}
  \caption{Potenciales e intensidades del circuito mixto}
  \end{table}

  \subsection{Representación gráfica de V frente a I}

  Cargamos los datos de la tabla anterior en el programa de \code{python} para obtener la representación de la gráfica del voltaje total (V) frente a la intensidad total (I).

  \begin{figure}[H]
    %\centering
    \hspace{2.5em} %% Creator: Matplotlib, PGF backend
%%
%% To include the figure in your LaTeX document, write
%%   \input{<filename>.pgf}
%%
%% Make sure the required packages are loaded in your preamble
%%   \usepackage{pgf}
%%
%% Figures using additional raster images can only be included by \input if
%% they are in the same directory as the main LaTeX file. For loading figures
%% from other directories you can use the `import` package
%%   \usepackage{import}
%% and then include the figures with
%%   \import{<path to file>}{<filename>.pgf}
%%
%% Matplotlib used the following preamble
%%
\begingroup%
\makeatletter%
\begin{pgfpicture}%
\pgfpathrectangle{\pgfpointorigin}{\pgfqpoint{4.747069in}{3.310123in}}%
\pgfusepath{use as bounding box, clip}%
\begin{pgfscope}%
\pgfsetbuttcap%
\pgfsetmiterjoin%
\definecolor{currentfill}{rgb}{1.000000,1.000000,1.000000}%
\pgfsetfillcolor{currentfill}%
\pgfsetlinewidth{0.000000pt}%
\definecolor{currentstroke}{rgb}{1.000000,1.000000,1.000000}%
\pgfsetstrokecolor{currentstroke}%
\pgfsetdash{}{0pt}%
\pgfpathmoveto{\pgfqpoint{0.000000in}{0.000000in}}%
\pgfpathlineto{\pgfqpoint{4.747069in}{0.000000in}}%
\pgfpathlineto{\pgfqpoint{4.747069in}{3.310123in}}%
\pgfpathlineto{\pgfqpoint{0.000000in}{3.310123in}}%
\pgfpathclose%
\pgfusepath{fill}%
\end{pgfscope}%
\begin{pgfscope}%
\pgfsetbuttcap%
\pgfsetmiterjoin%
\definecolor{currentfill}{rgb}{1.000000,1.000000,1.000000}%
\pgfsetfillcolor{currentfill}%
\pgfsetlinewidth{0.000000pt}%
\definecolor{currentstroke}{rgb}{0.000000,0.000000,0.000000}%
\pgfsetstrokecolor{currentstroke}%
\pgfsetstrokeopacity{0.000000}%
\pgfsetdash{}{0pt}%
\pgfpathmoveto{\pgfqpoint{0.772069in}{0.515123in}}%
\pgfpathlineto{\pgfqpoint{4.647069in}{0.515123in}}%
\pgfpathlineto{\pgfqpoint{4.647069in}{3.210123in}}%
\pgfpathlineto{\pgfqpoint{0.772069in}{3.210123in}}%
\pgfpathclose%
\pgfusepath{fill}%
\end{pgfscope}%
\begin{pgfscope}%
\pgfsetbuttcap%
\pgfsetroundjoin%
\definecolor{currentfill}{rgb}{0.000000,0.000000,0.000000}%
\pgfsetfillcolor{currentfill}%
\pgfsetlinewidth{0.803000pt}%
\definecolor{currentstroke}{rgb}{0.000000,0.000000,0.000000}%
\pgfsetstrokecolor{currentstroke}%
\pgfsetdash{}{0pt}%
\pgfsys@defobject{currentmarker}{\pgfqpoint{0.000000in}{-0.048611in}}{\pgfqpoint{0.000000in}{0.000000in}}{%
\pgfpathmoveto{\pgfqpoint{0.000000in}{0.000000in}}%
\pgfpathlineto{\pgfqpoint{0.000000in}{-0.048611in}}%
\pgfusepath{stroke,fill}%
}%
\begin{pgfscope}%
\pgfsys@transformshift{1.104968in}{0.515123in}%
\pgfsys@useobject{currentmarker}{}%
\end{pgfscope}%
\end{pgfscope}%
\begin{pgfscope}%
\definecolor{textcolor}{rgb}{0.000000,0.000000,0.000000}%
\pgfsetstrokecolor{textcolor}%
\pgfsetfillcolor{textcolor}%
\pgftext[x=1.104968in,y=0.417901in,,top]{\color{textcolor}\rmfamily\fontsize{10.000000}{12.000000}\selectfont \(\displaystyle 1\)}%
\end{pgfscope}%
\begin{pgfscope}%
\pgfsetbuttcap%
\pgfsetroundjoin%
\definecolor{currentfill}{rgb}{0.000000,0.000000,0.000000}%
\pgfsetfillcolor{currentfill}%
\pgfsetlinewidth{0.803000pt}%
\definecolor{currentstroke}{rgb}{0.000000,0.000000,0.000000}%
\pgfsetstrokecolor{currentstroke}%
\pgfsetdash{}{0pt}%
\pgfsys@defobject{currentmarker}{\pgfqpoint{0.000000in}{-0.048611in}}{\pgfqpoint{0.000000in}{0.000000in}}{%
\pgfpathmoveto{\pgfqpoint{0.000000in}{0.000000in}}%
\pgfpathlineto{\pgfqpoint{0.000000in}{-0.048611in}}%
\pgfusepath{stroke,fill}%
}%
\begin{pgfscope}%
\pgfsys@transformshift{1.614365in}{0.515123in}%
\pgfsys@useobject{currentmarker}{}%
\end{pgfscope}%
\end{pgfscope}%
\begin{pgfscope}%
\definecolor{textcolor}{rgb}{0.000000,0.000000,0.000000}%
\pgfsetstrokecolor{textcolor}%
\pgfsetfillcolor{textcolor}%
\pgftext[x=1.614365in,y=0.417901in,,top]{\color{textcolor}\rmfamily\fontsize{10.000000}{12.000000}\selectfont \(\displaystyle 2\)}%
\end{pgfscope}%
\begin{pgfscope}%
\pgfsetbuttcap%
\pgfsetroundjoin%
\definecolor{currentfill}{rgb}{0.000000,0.000000,0.000000}%
\pgfsetfillcolor{currentfill}%
\pgfsetlinewidth{0.803000pt}%
\definecolor{currentstroke}{rgb}{0.000000,0.000000,0.000000}%
\pgfsetstrokecolor{currentstroke}%
\pgfsetdash{}{0pt}%
\pgfsys@defobject{currentmarker}{\pgfqpoint{0.000000in}{-0.048611in}}{\pgfqpoint{0.000000in}{0.000000in}}{%
\pgfpathmoveto{\pgfqpoint{0.000000in}{0.000000in}}%
\pgfpathlineto{\pgfqpoint{0.000000in}{-0.048611in}}%
\pgfusepath{stroke,fill}%
}%
\begin{pgfscope}%
\pgfsys@transformshift{2.123762in}{0.515123in}%
\pgfsys@useobject{currentmarker}{}%
\end{pgfscope}%
\end{pgfscope}%
\begin{pgfscope}%
\definecolor{textcolor}{rgb}{0.000000,0.000000,0.000000}%
\pgfsetstrokecolor{textcolor}%
\pgfsetfillcolor{textcolor}%
\pgftext[x=2.123762in,y=0.417901in,,top]{\color{textcolor}\rmfamily\fontsize{10.000000}{12.000000}\selectfont \(\displaystyle 3\)}%
\end{pgfscope}%
\begin{pgfscope}%
\pgfsetbuttcap%
\pgfsetroundjoin%
\definecolor{currentfill}{rgb}{0.000000,0.000000,0.000000}%
\pgfsetfillcolor{currentfill}%
\pgfsetlinewidth{0.803000pt}%
\definecolor{currentstroke}{rgb}{0.000000,0.000000,0.000000}%
\pgfsetstrokecolor{currentstroke}%
\pgfsetdash{}{0pt}%
\pgfsys@defobject{currentmarker}{\pgfqpoint{0.000000in}{-0.048611in}}{\pgfqpoint{0.000000in}{0.000000in}}{%
\pgfpathmoveto{\pgfqpoint{0.000000in}{0.000000in}}%
\pgfpathlineto{\pgfqpoint{0.000000in}{-0.048611in}}%
\pgfusepath{stroke,fill}%
}%
\begin{pgfscope}%
\pgfsys@transformshift{2.633159in}{0.515123in}%
\pgfsys@useobject{currentmarker}{}%
\end{pgfscope}%
\end{pgfscope}%
\begin{pgfscope}%
\definecolor{textcolor}{rgb}{0.000000,0.000000,0.000000}%
\pgfsetstrokecolor{textcolor}%
\pgfsetfillcolor{textcolor}%
\pgftext[x=2.633159in,y=0.417901in,,top]{\color{textcolor}\rmfamily\fontsize{10.000000}{12.000000}\selectfont \(\displaystyle 4\)}%
\end{pgfscope}%
\begin{pgfscope}%
\pgfsetbuttcap%
\pgfsetroundjoin%
\definecolor{currentfill}{rgb}{0.000000,0.000000,0.000000}%
\pgfsetfillcolor{currentfill}%
\pgfsetlinewidth{0.803000pt}%
\definecolor{currentstroke}{rgb}{0.000000,0.000000,0.000000}%
\pgfsetstrokecolor{currentstroke}%
\pgfsetdash{}{0pt}%
\pgfsys@defobject{currentmarker}{\pgfqpoint{0.000000in}{-0.048611in}}{\pgfqpoint{0.000000in}{0.000000in}}{%
\pgfpathmoveto{\pgfqpoint{0.000000in}{0.000000in}}%
\pgfpathlineto{\pgfqpoint{0.000000in}{-0.048611in}}%
\pgfusepath{stroke,fill}%
}%
\begin{pgfscope}%
\pgfsys@transformshift{3.142556in}{0.515123in}%
\pgfsys@useobject{currentmarker}{}%
\end{pgfscope}%
\end{pgfscope}%
\begin{pgfscope}%
\definecolor{textcolor}{rgb}{0.000000,0.000000,0.000000}%
\pgfsetstrokecolor{textcolor}%
\pgfsetfillcolor{textcolor}%
\pgftext[x=3.142556in,y=0.417901in,,top]{\color{textcolor}\rmfamily\fontsize{10.000000}{12.000000}\selectfont \(\displaystyle 5\)}%
\end{pgfscope}%
\begin{pgfscope}%
\pgfsetbuttcap%
\pgfsetroundjoin%
\definecolor{currentfill}{rgb}{0.000000,0.000000,0.000000}%
\pgfsetfillcolor{currentfill}%
\pgfsetlinewidth{0.803000pt}%
\definecolor{currentstroke}{rgb}{0.000000,0.000000,0.000000}%
\pgfsetstrokecolor{currentstroke}%
\pgfsetdash{}{0pt}%
\pgfsys@defobject{currentmarker}{\pgfqpoint{0.000000in}{-0.048611in}}{\pgfqpoint{0.000000in}{0.000000in}}{%
\pgfpathmoveto{\pgfqpoint{0.000000in}{0.000000in}}%
\pgfpathlineto{\pgfqpoint{0.000000in}{-0.048611in}}%
\pgfusepath{stroke,fill}%
}%
\begin{pgfscope}%
\pgfsys@transformshift{3.651953in}{0.515123in}%
\pgfsys@useobject{currentmarker}{}%
\end{pgfscope}%
\end{pgfscope}%
\begin{pgfscope}%
\definecolor{textcolor}{rgb}{0.000000,0.000000,0.000000}%
\pgfsetstrokecolor{textcolor}%
\pgfsetfillcolor{textcolor}%
\pgftext[x=3.651953in,y=0.417901in,,top]{\color{textcolor}\rmfamily\fontsize{10.000000}{12.000000}\selectfont \(\displaystyle 6\)}%
\end{pgfscope}%
\begin{pgfscope}%
\pgfsetbuttcap%
\pgfsetroundjoin%
\definecolor{currentfill}{rgb}{0.000000,0.000000,0.000000}%
\pgfsetfillcolor{currentfill}%
\pgfsetlinewidth{0.803000pt}%
\definecolor{currentstroke}{rgb}{0.000000,0.000000,0.000000}%
\pgfsetstrokecolor{currentstroke}%
\pgfsetdash{}{0pt}%
\pgfsys@defobject{currentmarker}{\pgfqpoint{0.000000in}{-0.048611in}}{\pgfqpoint{0.000000in}{0.000000in}}{%
\pgfpathmoveto{\pgfqpoint{0.000000in}{0.000000in}}%
\pgfpathlineto{\pgfqpoint{0.000000in}{-0.048611in}}%
\pgfusepath{stroke,fill}%
}%
\begin{pgfscope}%
\pgfsys@transformshift{4.161350in}{0.515123in}%
\pgfsys@useobject{currentmarker}{}%
\end{pgfscope}%
\end{pgfscope}%
\begin{pgfscope}%
\definecolor{textcolor}{rgb}{0.000000,0.000000,0.000000}%
\pgfsetstrokecolor{textcolor}%
\pgfsetfillcolor{textcolor}%
\pgftext[x=4.161350in,y=0.417901in,,top]{\color{textcolor}\rmfamily\fontsize{10.000000}{12.000000}\selectfont \(\displaystyle 7\)}%
\end{pgfscope}%
\begin{pgfscope}%
\definecolor{textcolor}{rgb}{0.000000,0.000000,0.000000}%
\pgfsetstrokecolor{textcolor}%
\pgfsetfillcolor{textcolor}%
\pgftext[x=2.709569in,y=0.238889in,,top]{\color{textcolor}\rmfamily\fontsize{10.000000}{12.000000}\selectfont I(\(\displaystyle \mu\)A)}%
\end{pgfscope}%
\begin{pgfscope}%
\pgfsetbuttcap%
\pgfsetroundjoin%
\definecolor{currentfill}{rgb}{0.000000,0.000000,0.000000}%
\pgfsetfillcolor{currentfill}%
\pgfsetlinewidth{0.803000pt}%
\definecolor{currentstroke}{rgb}{0.000000,0.000000,0.000000}%
\pgfsetstrokecolor{currentstroke}%
\pgfsetdash{}{0pt}%
\pgfsys@defobject{currentmarker}{\pgfqpoint{-0.048611in}{0.000000in}}{\pgfqpoint{0.000000in}{0.000000in}}{%
\pgfpathmoveto{\pgfqpoint{0.000000in}{0.000000in}}%
\pgfpathlineto{\pgfqpoint{-0.048611in}{0.000000in}}%
\pgfusepath{stroke,fill}%
}%
\begin{pgfscope}%
\pgfsys@transformshift{0.772069in}{0.900939in}%
\pgfsys@useobject{currentmarker}{}%
\end{pgfscope}%
\end{pgfscope}%
\begin{pgfscope}%
\definecolor{textcolor}{rgb}{0.000000,0.000000,0.000000}%
\pgfsetstrokecolor{textcolor}%
\pgfsetfillcolor{textcolor}%
\pgftext[x=0.605402in,y=0.852713in,left,base]{\color{textcolor}\rmfamily\fontsize{10.000000}{12.000000}\selectfont \(\displaystyle 2\)}%
\end{pgfscope}%
\begin{pgfscope}%
\pgfsetbuttcap%
\pgfsetroundjoin%
\definecolor{currentfill}{rgb}{0.000000,0.000000,0.000000}%
\pgfsetfillcolor{currentfill}%
\pgfsetlinewidth{0.803000pt}%
\definecolor{currentstroke}{rgb}{0.000000,0.000000,0.000000}%
\pgfsetstrokecolor{currentstroke}%
\pgfsetdash{}{0pt}%
\pgfsys@defobject{currentmarker}{\pgfqpoint{-0.048611in}{0.000000in}}{\pgfqpoint{0.000000in}{0.000000in}}{%
\pgfpathmoveto{\pgfqpoint{0.000000in}{0.000000in}}%
\pgfpathlineto{\pgfqpoint{-0.048611in}{0.000000in}}%
\pgfusepath{stroke,fill}%
}%
\begin{pgfscope}%
\pgfsys@transformshift{0.772069in}{1.438794in}%
\pgfsys@useobject{currentmarker}{}%
\end{pgfscope}%
\end{pgfscope}%
\begin{pgfscope}%
\definecolor{textcolor}{rgb}{0.000000,0.000000,0.000000}%
\pgfsetstrokecolor{textcolor}%
\pgfsetfillcolor{textcolor}%
\pgftext[x=0.605402in,y=1.390568in,left,base]{\color{textcolor}\rmfamily\fontsize{10.000000}{12.000000}\selectfont \(\displaystyle 4\)}%
\end{pgfscope}%
\begin{pgfscope}%
\pgfsetbuttcap%
\pgfsetroundjoin%
\definecolor{currentfill}{rgb}{0.000000,0.000000,0.000000}%
\pgfsetfillcolor{currentfill}%
\pgfsetlinewidth{0.803000pt}%
\definecolor{currentstroke}{rgb}{0.000000,0.000000,0.000000}%
\pgfsetstrokecolor{currentstroke}%
\pgfsetdash{}{0pt}%
\pgfsys@defobject{currentmarker}{\pgfqpoint{-0.048611in}{0.000000in}}{\pgfqpoint{0.000000in}{0.000000in}}{%
\pgfpathmoveto{\pgfqpoint{0.000000in}{0.000000in}}%
\pgfpathlineto{\pgfqpoint{-0.048611in}{0.000000in}}%
\pgfusepath{stroke,fill}%
}%
\begin{pgfscope}%
\pgfsys@transformshift{0.772069in}{1.976649in}%
\pgfsys@useobject{currentmarker}{}%
\end{pgfscope}%
\end{pgfscope}%
\begin{pgfscope}%
\definecolor{textcolor}{rgb}{0.000000,0.000000,0.000000}%
\pgfsetstrokecolor{textcolor}%
\pgfsetfillcolor{textcolor}%
\pgftext[x=0.605402in,y=1.928423in,left,base]{\color{textcolor}\rmfamily\fontsize{10.000000}{12.000000}\selectfont \(\displaystyle 6\)}%
\end{pgfscope}%
\begin{pgfscope}%
\pgfsetbuttcap%
\pgfsetroundjoin%
\definecolor{currentfill}{rgb}{0.000000,0.000000,0.000000}%
\pgfsetfillcolor{currentfill}%
\pgfsetlinewidth{0.803000pt}%
\definecolor{currentstroke}{rgb}{0.000000,0.000000,0.000000}%
\pgfsetstrokecolor{currentstroke}%
\pgfsetdash{}{0pt}%
\pgfsys@defobject{currentmarker}{\pgfqpoint{-0.048611in}{0.000000in}}{\pgfqpoint{0.000000in}{0.000000in}}{%
\pgfpathmoveto{\pgfqpoint{0.000000in}{0.000000in}}%
\pgfpathlineto{\pgfqpoint{-0.048611in}{0.000000in}}%
\pgfusepath{stroke,fill}%
}%
\begin{pgfscope}%
\pgfsys@transformshift{0.772069in}{2.514503in}%
\pgfsys@useobject{currentmarker}{}%
\end{pgfscope}%
\end{pgfscope}%
\begin{pgfscope}%
\definecolor{textcolor}{rgb}{0.000000,0.000000,0.000000}%
\pgfsetstrokecolor{textcolor}%
\pgfsetfillcolor{textcolor}%
\pgftext[x=0.605402in,y=2.466278in,left,base]{\color{textcolor}\rmfamily\fontsize{10.000000}{12.000000}\selectfont \(\displaystyle 8\)}%
\end{pgfscope}%
\begin{pgfscope}%
\pgfsetbuttcap%
\pgfsetroundjoin%
\definecolor{currentfill}{rgb}{0.000000,0.000000,0.000000}%
\pgfsetfillcolor{currentfill}%
\pgfsetlinewidth{0.803000pt}%
\definecolor{currentstroke}{rgb}{0.000000,0.000000,0.000000}%
\pgfsetstrokecolor{currentstroke}%
\pgfsetdash{}{0pt}%
\pgfsys@defobject{currentmarker}{\pgfqpoint{-0.048611in}{0.000000in}}{\pgfqpoint{0.000000in}{0.000000in}}{%
\pgfpathmoveto{\pgfqpoint{0.000000in}{0.000000in}}%
\pgfpathlineto{\pgfqpoint{-0.048611in}{0.000000in}}%
\pgfusepath{stroke,fill}%
}%
\begin{pgfscope}%
\pgfsys@transformshift{0.772069in}{3.052358in}%
\pgfsys@useobject{currentmarker}{}%
\end{pgfscope}%
\end{pgfscope}%
\begin{pgfscope}%
\definecolor{textcolor}{rgb}{0.000000,0.000000,0.000000}%
\pgfsetstrokecolor{textcolor}%
\pgfsetfillcolor{textcolor}%
\pgftext[x=0.535957in,y=3.004133in,left,base]{\color{textcolor}\rmfamily\fontsize{10.000000}{12.000000}\selectfont \(\displaystyle 10\)}%
\end{pgfscope}%
\begin{pgfscope}%
\definecolor{textcolor}{rgb}{0.000000,0.000000,0.000000}%
\pgfsetstrokecolor{textcolor}%
\pgfsetfillcolor{textcolor}%
\pgftext[x=0.258179in,y=1.862623in,,bottom]{\color{textcolor}\rmfamily\fontsize{10.000000}{12.000000}\selectfont V(V)}%
\end{pgfscope}%
\begin{pgfscope}%
\pgfpathrectangle{\pgfqpoint{0.772069in}{0.515123in}}{\pgfqpoint{3.875000in}{2.695000in}}%
\pgfusepath{clip}%
\pgfsetbuttcap%
\pgfsetroundjoin%
\definecolor{currentfill}{rgb}{0.121569,0.466667,0.705882}%
\pgfsetfillcolor{currentfill}%
\pgfsetlinewidth{1.003750pt}%
\definecolor{currentstroke}{rgb}{0.121569,0.466667,0.705882}%
\pgfsetstrokecolor{currentstroke}%
\pgfsetdash{}{0pt}%
\pgfpathmoveto{\pgfqpoint{0.952149in}{0.598950in}}%
\pgfpathcurveto{\pgfqpoint{0.963199in}{0.598950in}}{\pgfqpoint{0.973798in}{0.603341in}}{\pgfqpoint{0.981612in}{0.611154in}}%
\pgfpathcurveto{\pgfqpoint{0.989425in}{0.618968in}}{\pgfqpoint{0.993815in}{0.629567in}}{\pgfqpoint{0.993815in}{0.640617in}}%
\pgfpathcurveto{\pgfqpoint{0.993815in}{0.651667in}}{\pgfqpoint{0.989425in}{0.662266in}}{\pgfqpoint{0.981612in}{0.670080in}}%
\pgfpathcurveto{\pgfqpoint{0.973798in}{0.677893in}}{\pgfqpoint{0.963199in}{0.682284in}}{\pgfqpoint{0.952149in}{0.682284in}}%
\pgfpathcurveto{\pgfqpoint{0.941099in}{0.682284in}}{\pgfqpoint{0.930500in}{0.677893in}}{\pgfqpoint{0.922686in}{0.670080in}}%
\pgfpathcurveto{\pgfqpoint{0.914872in}{0.662266in}}{\pgfqpoint{0.910482in}{0.651667in}}{\pgfqpoint{0.910482in}{0.640617in}}%
\pgfpathcurveto{\pgfqpoint{0.910482in}{0.629567in}}{\pgfqpoint{0.914872in}{0.618968in}}{\pgfqpoint{0.922686in}{0.611154in}}%
\pgfpathcurveto{\pgfqpoint{0.930500in}{0.603341in}}{\pgfqpoint{0.941099in}{0.598950in}}{\pgfqpoint{0.952149in}{0.598950in}}%
\pgfpathclose%
\pgfusepath{stroke,fill}%
\end{pgfscope}%
\begin{pgfscope}%
\pgfpathrectangle{\pgfqpoint{0.772069in}{0.515123in}}{\pgfqpoint{3.875000in}{2.695000in}}%
\pgfusepath{clip}%
\pgfsetbuttcap%
\pgfsetroundjoin%
\definecolor{currentfill}{rgb}{0.121569,0.466667,0.705882}%
\pgfsetfillcolor{currentfill}%
\pgfsetlinewidth{1.003750pt}%
\definecolor{currentstroke}{rgb}{0.121569,0.466667,0.705882}%
\pgfsetstrokecolor{currentstroke}%
\pgfsetdash{}{0pt}%
\pgfpathmoveto{\pgfqpoint{1.359666in}{0.859272in}}%
\pgfpathcurveto{\pgfqpoint{1.370717in}{0.859272in}}{\pgfqpoint{1.381316in}{0.863662in}}{\pgfqpoint{1.389129in}{0.871476in}}%
\pgfpathcurveto{\pgfqpoint{1.396943in}{0.879290in}}{\pgfqpoint{1.401333in}{0.889889in}}{\pgfqpoint{1.401333in}{0.900939in}}%
\pgfpathcurveto{\pgfqpoint{1.401333in}{0.911989in}}{\pgfqpoint{1.396943in}{0.922588in}}{\pgfqpoint{1.389129in}{0.930402in}}%
\pgfpathcurveto{\pgfqpoint{1.381316in}{0.938215in}}{\pgfqpoint{1.370717in}{0.942605in}}{\pgfqpoint{1.359666in}{0.942605in}}%
\pgfpathcurveto{\pgfqpoint{1.348616in}{0.942605in}}{\pgfqpoint{1.338017in}{0.938215in}}{\pgfqpoint{1.330204in}{0.930402in}}%
\pgfpathcurveto{\pgfqpoint{1.322390in}{0.922588in}}{\pgfqpoint{1.318000in}{0.911989in}}{\pgfqpoint{1.318000in}{0.900939in}}%
\pgfpathcurveto{\pgfqpoint{1.318000in}{0.889889in}}{\pgfqpoint{1.322390in}{0.879290in}}{\pgfqpoint{1.330204in}{0.871476in}}%
\pgfpathcurveto{\pgfqpoint{1.338017in}{0.863662in}}{\pgfqpoint{1.348616in}{0.859272in}}{\pgfqpoint{1.359666in}{0.859272in}}%
\pgfpathclose%
\pgfusepath{stroke,fill}%
\end{pgfscope}%
\begin{pgfscope}%
\pgfpathrectangle{\pgfqpoint{0.772069in}{0.515123in}}{\pgfqpoint{3.875000in}{2.695000in}}%
\pgfusepath{clip}%
\pgfsetbuttcap%
\pgfsetroundjoin%
\definecolor{currentfill}{rgb}{0.121569,0.466667,0.705882}%
\pgfsetfillcolor{currentfill}%
\pgfsetlinewidth{1.003750pt}%
\definecolor{currentstroke}{rgb}{0.121569,0.466667,0.705882}%
\pgfsetstrokecolor{currentstroke}%
\pgfsetdash{}{0pt}%
\pgfpathmoveto{\pgfqpoint{1.716244in}{1.138957in}}%
\pgfpathcurveto{\pgfqpoint{1.727295in}{1.138957in}}{\pgfqpoint{1.737894in}{1.143347in}}{\pgfqpoint{1.745707in}{1.151161in}}%
\pgfpathcurveto{\pgfqpoint{1.753521in}{1.158974in}}{\pgfqpoint{1.757911in}{1.169573in}}{\pgfqpoint{1.757911in}{1.180623in}}%
\pgfpathcurveto{\pgfqpoint{1.757911in}{1.191673in}}{\pgfqpoint{1.753521in}{1.202272in}}{\pgfqpoint{1.745707in}{1.210086in}}%
\pgfpathcurveto{\pgfqpoint{1.737894in}{1.217900in}}{\pgfqpoint{1.727295in}{1.222290in}}{\pgfqpoint{1.716244in}{1.222290in}}%
\pgfpathcurveto{\pgfqpoint{1.705194in}{1.222290in}}{\pgfqpoint{1.694595in}{1.217900in}}{\pgfqpoint{1.686782in}{1.210086in}}%
\pgfpathcurveto{\pgfqpoint{1.678968in}{1.202272in}}{\pgfqpoint{1.674578in}{1.191673in}}{\pgfqpoint{1.674578in}{1.180623in}}%
\pgfpathcurveto{\pgfqpoint{1.674578in}{1.169573in}}{\pgfqpoint{1.678968in}{1.158974in}}{\pgfqpoint{1.686782in}{1.151161in}}%
\pgfpathcurveto{\pgfqpoint{1.694595in}{1.143347in}}{\pgfqpoint{1.705194in}{1.138957in}}{\pgfqpoint{1.716244in}{1.138957in}}%
\pgfpathclose%
\pgfusepath{stroke,fill}%
\end{pgfscope}%
\begin{pgfscope}%
\pgfpathrectangle{\pgfqpoint{0.772069in}{0.515123in}}{\pgfqpoint{3.875000in}{2.695000in}}%
\pgfusepath{clip}%
\pgfsetbuttcap%
\pgfsetroundjoin%
\definecolor{currentfill}{rgb}{0.121569,0.466667,0.705882}%
\pgfsetfillcolor{currentfill}%
\pgfsetlinewidth{1.003750pt}%
\definecolor{currentstroke}{rgb}{0.121569,0.466667,0.705882}%
\pgfsetstrokecolor{currentstroke}%
\pgfsetdash{}{0pt}%
\pgfpathmoveto{\pgfqpoint{2.123762in}{1.413263in}}%
\pgfpathcurveto{\pgfqpoint{2.134812in}{1.413263in}}{\pgfqpoint{2.145411in}{1.417653in}}{\pgfqpoint{2.153225in}{1.425467in}}%
\pgfpathcurveto{\pgfqpoint{2.161038in}{1.433280in}}{\pgfqpoint{2.165429in}{1.443879in}}{\pgfqpoint{2.165429in}{1.454929in}}%
\pgfpathcurveto{\pgfqpoint{2.165429in}{1.465979in}}{\pgfqpoint{2.161038in}{1.476578in}}{\pgfqpoint{2.153225in}{1.484392in}}%
\pgfpathcurveto{\pgfqpoint{2.145411in}{1.492206in}}{\pgfqpoint{2.134812in}{1.496596in}}{\pgfqpoint{2.123762in}{1.496596in}}%
\pgfpathcurveto{\pgfqpoint{2.112712in}{1.496596in}}{\pgfqpoint{2.102113in}{1.492206in}}{\pgfqpoint{2.094299in}{1.484392in}}%
\pgfpathcurveto{\pgfqpoint{2.086486in}{1.476578in}}{\pgfqpoint{2.082095in}{1.465979in}}{\pgfqpoint{2.082095in}{1.454929in}}%
\pgfpathcurveto{\pgfqpoint{2.082095in}{1.443879in}}{\pgfqpoint{2.086486in}{1.433280in}}{\pgfqpoint{2.094299in}{1.425467in}}%
\pgfpathcurveto{\pgfqpoint{2.102113in}{1.417653in}}{\pgfqpoint{2.112712in}{1.413263in}}{\pgfqpoint{2.123762in}{1.413263in}}%
\pgfpathclose%
\pgfusepath{stroke,fill}%
\end{pgfscope}%
\begin{pgfscope}%
\pgfpathrectangle{\pgfqpoint{0.772069in}{0.515123in}}{\pgfqpoint{3.875000in}{2.695000in}}%
\pgfusepath{clip}%
\pgfsetbuttcap%
\pgfsetroundjoin%
\definecolor{currentfill}{rgb}{0.121569,0.466667,0.705882}%
\pgfsetfillcolor{currentfill}%
\pgfsetlinewidth{1.003750pt}%
\definecolor{currentstroke}{rgb}{0.121569,0.466667,0.705882}%
\pgfsetstrokecolor{currentstroke}%
\pgfsetdash{}{0pt}%
\pgfpathmoveto{\pgfqpoint{2.531280in}{1.687569in}}%
\pgfpathcurveto{\pgfqpoint{2.542330in}{1.687569in}}{\pgfqpoint{2.552929in}{1.691959in}}{\pgfqpoint{2.560742in}{1.699773in}}%
\pgfpathcurveto{\pgfqpoint{2.568556in}{1.707586in}}{\pgfqpoint{2.572946in}{1.718185in}}{\pgfqpoint{2.572946in}{1.729235in}}%
\pgfpathcurveto{\pgfqpoint{2.572946in}{1.740285in}}{\pgfqpoint{2.568556in}{1.750884in}}{\pgfqpoint{2.560742in}{1.758698in}}%
\pgfpathcurveto{\pgfqpoint{2.552929in}{1.766512in}}{\pgfqpoint{2.542330in}{1.770902in}}{\pgfqpoint{2.531280in}{1.770902in}}%
\pgfpathcurveto{\pgfqpoint{2.520230in}{1.770902in}}{\pgfqpoint{2.509631in}{1.766512in}}{\pgfqpoint{2.501817in}{1.758698in}}%
\pgfpathcurveto{\pgfqpoint{2.494003in}{1.750884in}}{\pgfqpoint{2.489613in}{1.740285in}}{\pgfqpoint{2.489613in}{1.729235in}}%
\pgfpathcurveto{\pgfqpoint{2.489613in}{1.718185in}}{\pgfqpoint{2.494003in}{1.707586in}}{\pgfqpoint{2.501817in}{1.699773in}}%
\pgfpathcurveto{\pgfqpoint{2.509631in}{1.691959in}}{\pgfqpoint{2.520230in}{1.687569in}}{\pgfqpoint{2.531280in}{1.687569in}}%
\pgfpathclose%
\pgfusepath{stroke,fill}%
\end{pgfscope}%
\begin{pgfscope}%
\pgfpathrectangle{\pgfqpoint{0.772069in}{0.515123in}}{\pgfqpoint{3.875000in}{2.695000in}}%
\pgfusepath{clip}%
\pgfsetbuttcap%
\pgfsetroundjoin%
\definecolor{currentfill}{rgb}{0.121569,0.466667,0.705882}%
\pgfsetfillcolor{currentfill}%
\pgfsetlinewidth{1.003750pt}%
\definecolor{currentstroke}{rgb}{0.121569,0.466667,0.705882}%
\pgfsetstrokecolor{currentstroke}%
\pgfsetdash{}{0pt}%
\pgfpathmoveto{\pgfqpoint{2.887858in}{1.948428in}}%
\pgfpathcurveto{\pgfqpoint{2.898908in}{1.948428in}}{\pgfqpoint{2.909507in}{1.952819in}}{\pgfqpoint{2.917320in}{1.960632in}}%
\pgfpathcurveto{\pgfqpoint{2.925134in}{1.968446in}}{\pgfqpoint{2.929524in}{1.979045in}}{\pgfqpoint{2.929524in}{1.990095in}}%
\pgfpathcurveto{\pgfqpoint{2.929524in}{2.001145in}}{\pgfqpoint{2.925134in}{2.011744in}}{\pgfqpoint{2.917320in}{2.019558in}}%
\pgfpathcurveto{\pgfqpoint{2.909507in}{2.027371in}}{\pgfqpoint{2.898908in}{2.031762in}}{\pgfqpoint{2.887858in}{2.031762in}}%
\pgfpathcurveto{\pgfqpoint{2.876808in}{2.031762in}}{\pgfqpoint{2.866208in}{2.027371in}}{\pgfqpoint{2.858395in}{2.019558in}}%
\pgfpathcurveto{\pgfqpoint{2.850581in}{2.011744in}}{\pgfqpoint{2.846191in}{2.001145in}}{\pgfqpoint{2.846191in}{1.990095in}}%
\pgfpathcurveto{\pgfqpoint{2.846191in}{1.979045in}}{\pgfqpoint{2.850581in}{1.968446in}}{\pgfqpoint{2.858395in}{1.960632in}}%
\pgfpathcurveto{\pgfqpoint{2.866208in}{1.952819in}}{\pgfqpoint{2.876808in}{1.948428in}}{\pgfqpoint{2.887858in}{1.948428in}}%
\pgfpathclose%
\pgfusepath{stroke,fill}%
\end{pgfscope}%
\begin{pgfscope}%
\pgfpathrectangle{\pgfqpoint{0.772069in}{0.515123in}}{\pgfqpoint{3.875000in}{2.695000in}}%
\pgfusepath{clip}%
\pgfsetbuttcap%
\pgfsetroundjoin%
\definecolor{currentfill}{rgb}{0.121569,0.466667,0.705882}%
\pgfsetfillcolor{currentfill}%
\pgfsetlinewidth{1.003750pt}%
\definecolor{currentstroke}{rgb}{0.121569,0.466667,0.705882}%
\pgfsetstrokecolor{currentstroke}%
\pgfsetdash{}{0pt}%
\pgfpathmoveto{\pgfqpoint{3.295375in}{2.230802in}}%
\pgfpathcurveto{\pgfqpoint{3.306425in}{2.230802in}}{\pgfqpoint{3.317024in}{2.235192in}}{\pgfqpoint{3.324838in}{2.243006in}}%
\pgfpathcurveto{\pgfqpoint{3.332652in}{2.250820in}}{\pgfqpoint{3.337042in}{2.261419in}}{\pgfqpoint{3.337042in}{2.272469in}}%
\pgfpathcurveto{\pgfqpoint{3.337042in}{2.283519in}}{\pgfqpoint{3.332652in}{2.294118in}}{\pgfqpoint{3.324838in}{2.301932in}}%
\pgfpathcurveto{\pgfqpoint{3.317024in}{2.309745in}}{\pgfqpoint{3.306425in}{2.314135in}}{\pgfqpoint{3.295375in}{2.314135in}}%
\pgfpathcurveto{\pgfqpoint{3.284325in}{2.314135in}}{\pgfqpoint{3.273726in}{2.309745in}}{\pgfqpoint{3.265913in}{2.301932in}}%
\pgfpathcurveto{\pgfqpoint{3.258099in}{2.294118in}}{\pgfqpoint{3.253709in}{2.283519in}}{\pgfqpoint{3.253709in}{2.272469in}}%
\pgfpathcurveto{\pgfqpoint{3.253709in}{2.261419in}}{\pgfqpoint{3.258099in}{2.250820in}}{\pgfqpoint{3.265913in}{2.243006in}}%
\pgfpathcurveto{\pgfqpoint{3.273726in}{2.235192in}}{\pgfqpoint{3.284325in}{2.230802in}}{\pgfqpoint{3.295375in}{2.230802in}}%
\pgfpathclose%
\pgfusepath{stroke,fill}%
\end{pgfscope}%
\begin{pgfscope}%
\pgfpathrectangle{\pgfqpoint{0.772069in}{0.515123in}}{\pgfqpoint{3.875000in}{2.695000in}}%
\pgfusepath{clip}%
\pgfsetbuttcap%
\pgfsetroundjoin%
\definecolor{currentfill}{rgb}{0.121569,0.466667,0.705882}%
\pgfsetfillcolor{currentfill}%
\pgfsetlinewidth{1.003750pt}%
\definecolor{currentstroke}{rgb}{0.121569,0.466667,0.705882}%
\pgfsetstrokecolor{currentstroke}%
\pgfsetdash{}{0pt}%
\pgfpathmoveto{\pgfqpoint{3.702893in}{2.502419in}}%
\pgfpathcurveto{\pgfqpoint{3.713943in}{2.502419in}}{\pgfqpoint{3.724542in}{2.506809in}}{\pgfqpoint{3.732356in}{2.514623in}}%
\pgfpathcurveto{\pgfqpoint{3.740169in}{2.522436in}}{\pgfqpoint{3.744560in}{2.533035in}}{\pgfqpoint{3.744560in}{2.544086in}}%
\pgfpathcurveto{\pgfqpoint{3.744560in}{2.555136in}}{\pgfqpoint{3.740169in}{2.565735in}}{\pgfqpoint{3.732356in}{2.573548in}}%
\pgfpathcurveto{\pgfqpoint{3.724542in}{2.581362in}}{\pgfqpoint{3.713943in}{2.585752in}}{\pgfqpoint{3.702893in}{2.585752in}}%
\pgfpathcurveto{\pgfqpoint{3.691843in}{2.585752in}}{\pgfqpoint{3.681244in}{2.581362in}}{\pgfqpoint{3.673430in}{2.573548in}}%
\pgfpathcurveto{\pgfqpoint{3.665617in}{2.565735in}}{\pgfqpoint{3.661226in}{2.555136in}}{\pgfqpoint{3.661226in}{2.544086in}}%
\pgfpathcurveto{\pgfqpoint{3.661226in}{2.533035in}}{\pgfqpoint{3.665617in}{2.522436in}}{\pgfqpoint{3.673430in}{2.514623in}}%
\pgfpathcurveto{\pgfqpoint{3.681244in}{2.506809in}}{\pgfqpoint{3.691843in}{2.502419in}}{\pgfqpoint{3.702893in}{2.502419in}}%
\pgfpathclose%
\pgfusepath{stroke,fill}%
\end{pgfscope}%
\begin{pgfscope}%
\pgfpathrectangle{\pgfqpoint{0.772069in}{0.515123in}}{\pgfqpoint{3.875000in}{2.695000in}}%
\pgfusepath{clip}%
\pgfsetbuttcap%
\pgfsetroundjoin%
\definecolor{currentfill}{rgb}{0.121569,0.466667,0.705882}%
\pgfsetfillcolor{currentfill}%
\pgfsetlinewidth{1.003750pt}%
\definecolor{currentstroke}{rgb}{0.121569,0.466667,0.705882}%
\pgfsetstrokecolor{currentstroke}%
\pgfsetdash{}{0pt}%
\pgfpathmoveto{\pgfqpoint{4.059471in}{2.768657in}}%
\pgfpathcurveto{\pgfqpoint{4.070521in}{2.768657in}}{\pgfqpoint{4.081120in}{2.773047in}}{\pgfqpoint{4.088934in}{2.780861in}}%
\pgfpathcurveto{\pgfqpoint{4.096747in}{2.788675in}}{\pgfqpoint{4.101138in}{2.799274in}}{\pgfqpoint{4.101138in}{2.810324in}}%
\pgfpathcurveto{\pgfqpoint{4.101138in}{2.821374in}}{\pgfqpoint{4.096747in}{2.831973in}}{\pgfqpoint{4.088934in}{2.839786in}}%
\pgfpathcurveto{\pgfqpoint{4.081120in}{2.847600in}}{\pgfqpoint{4.070521in}{2.851990in}}{\pgfqpoint{4.059471in}{2.851990in}}%
\pgfpathcurveto{\pgfqpoint{4.048421in}{2.851990in}}{\pgfqpoint{4.037822in}{2.847600in}}{\pgfqpoint{4.030008in}{2.839786in}}%
\pgfpathcurveto{\pgfqpoint{4.022195in}{2.831973in}}{\pgfqpoint{4.017804in}{2.821374in}}{\pgfqpoint{4.017804in}{2.810324in}}%
\pgfpathcurveto{\pgfqpoint{4.017804in}{2.799274in}}{\pgfqpoint{4.022195in}{2.788675in}}{\pgfqpoint{4.030008in}{2.780861in}}%
\pgfpathcurveto{\pgfqpoint{4.037822in}{2.773047in}}{\pgfqpoint{4.048421in}{2.768657in}}{\pgfqpoint{4.059471in}{2.768657in}}%
\pgfpathclose%
\pgfusepath{stroke,fill}%
\end{pgfscope}%
\begin{pgfscope}%
\pgfpathrectangle{\pgfqpoint{0.772069in}{0.515123in}}{\pgfqpoint{3.875000in}{2.695000in}}%
\pgfusepath{clip}%
\pgfsetbuttcap%
\pgfsetroundjoin%
\definecolor{currentfill}{rgb}{0.121569,0.466667,0.705882}%
\pgfsetfillcolor{currentfill}%
\pgfsetlinewidth{1.003750pt}%
\definecolor{currentstroke}{rgb}{0.121569,0.466667,0.705882}%
\pgfsetstrokecolor{currentstroke}%
\pgfsetdash{}{0pt}%
\pgfpathmoveto{\pgfqpoint{4.466989in}{3.042963in}}%
\pgfpathcurveto{\pgfqpoint{4.478039in}{3.042963in}}{\pgfqpoint{4.488638in}{3.047353in}}{\pgfqpoint{4.496451in}{3.055167in}}%
\pgfpathcurveto{\pgfqpoint{4.504265in}{3.062981in}}{\pgfqpoint{4.508655in}{3.073580in}}{\pgfqpoint{4.508655in}{3.084630in}}%
\pgfpathcurveto{\pgfqpoint{4.508655in}{3.095680in}}{\pgfqpoint{4.504265in}{3.106279in}}{\pgfqpoint{4.496451in}{3.114092in}}%
\pgfpathcurveto{\pgfqpoint{4.488638in}{3.121906in}}{\pgfqpoint{4.478039in}{3.126296in}}{\pgfqpoint{4.466989in}{3.126296in}}%
\pgfpathcurveto{\pgfqpoint{4.455938in}{3.126296in}}{\pgfqpoint{4.445339in}{3.121906in}}{\pgfqpoint{4.437526in}{3.114092in}}%
\pgfpathcurveto{\pgfqpoint{4.429712in}{3.106279in}}{\pgfqpoint{4.425322in}{3.095680in}}{\pgfqpoint{4.425322in}{3.084630in}}%
\pgfpathcurveto{\pgfqpoint{4.425322in}{3.073580in}}{\pgfqpoint{4.429712in}{3.062981in}}{\pgfqpoint{4.437526in}{3.055167in}}%
\pgfpathcurveto{\pgfqpoint{4.445339in}{3.047353in}}{\pgfqpoint{4.455938in}{3.042963in}}{\pgfqpoint{4.466989in}{3.042963in}}%
\pgfpathclose%
\pgfusepath{stroke,fill}%
\end{pgfscope}%
\begin{pgfscope}%
\pgfsetrectcap%
\pgfsetmiterjoin%
\pgfsetlinewidth{0.803000pt}%
\definecolor{currentstroke}{rgb}{0.000000,0.000000,0.000000}%
\pgfsetstrokecolor{currentstroke}%
\pgfsetdash{}{0pt}%
\pgfpathmoveto{\pgfqpoint{0.772069in}{0.515123in}}%
\pgfpathlineto{\pgfqpoint{0.772069in}{3.210123in}}%
\pgfusepath{stroke}%
\end{pgfscope}%
\begin{pgfscope}%
\pgfsetrectcap%
\pgfsetmiterjoin%
\pgfsetlinewidth{0.803000pt}%
\definecolor{currentstroke}{rgb}{0.000000,0.000000,0.000000}%
\pgfsetstrokecolor{currentstroke}%
\pgfsetdash{}{0pt}%
\pgfpathmoveto{\pgfqpoint{4.647069in}{0.515123in}}%
\pgfpathlineto{\pgfqpoint{4.647069in}{3.210123in}}%
\pgfusepath{stroke}%
\end{pgfscope}%
\begin{pgfscope}%
\pgfsetrectcap%
\pgfsetmiterjoin%
\pgfsetlinewidth{0.803000pt}%
\definecolor{currentstroke}{rgb}{0.000000,0.000000,0.000000}%
\pgfsetstrokecolor{currentstroke}%
\pgfsetdash{}{0pt}%
\pgfpathmoveto{\pgfqpoint{0.772069in}{0.515123in}}%
\pgfpathlineto{\pgfqpoint{4.647069in}{0.515123in}}%
\pgfusepath{stroke}%
\end{pgfscope}%
\begin{pgfscope}%
\pgfsetrectcap%
\pgfsetmiterjoin%
\pgfsetlinewidth{0.803000pt}%
\definecolor{currentstroke}{rgb}{0.000000,0.000000,0.000000}%
\pgfsetstrokecolor{currentstroke}%
\pgfsetdash{}{0pt}%
\pgfpathmoveto{\pgfqpoint{0.772069in}{3.210123in}}%
\pgfpathlineto{\pgfqpoint{4.647069in}{3.210123in}}%
\pgfusepath{stroke}%
\end{pgfscope}%
\end{pgfpicture}%
\makeatother%
\endgroup%

    \caption{Voltaje (V) frente a intensidad (I)}
    \label{graf:cmixinttotal}
  \end{figure}

  También podemos probar a hacer gráficas más interesantes. Aquí no hay una relación tan clara como en los anteriores ejemplos, ya que existen varias magnitudes tanto en el eje x (I) como en el eje y (V). Podríamos probar a representar los diferentes valores de V ($V$, $V_1$, $V_{23}$ y $V_4$) frente a la intensidad total ($I$), como en la primera gráfica. También podemos hacer lo contrario, y representar el potencial total ($V$) frente a distintas intensidades ($I$, $I_1$ e $I_2$), como en la segunda gráfica. Como curiosidad más que como experimento, podemos mezclar ambas gráficas, y representar todas las combinaciones entre \textcolor{Blue}{$V$},~\textcolor{Red}{$V_1$},~\textcolor{Yellow}{$V_{23}$},~\textcolor{Green}{$V_4$} (\textit{color}) y \textcolor{Black}{$I$},~\textcolor{DarkGrey}{$I_1$},~\textcolor{Grey}{$I_2$} (\textit{luminosidad}).

  \begin{figure}[H]
    \centering
    %% Creator: Matplotlib, PGF backend
%%
%% To include the figure in your LaTeX document, write
%%   \input{<filename>.pgf}
%%
%% Make sure the required packages are loaded in your preamble
%%   \usepackage{pgf}
%%
%% Figures using additional raster images can only be included by \input if
%% they are in the same directory as the main LaTeX file. For loading figures
%% from other directories you can use the `import` package
%%   \usepackage{import}
%% and then include the figures with
%%   \import{<path to file>}{<filename>.pgf}
%%
%% Matplotlib used the following preamble
%%
\begingroup%
\makeatletter%
\begin{pgfpicture}%
\pgfpathrectangle{\pgfpointorigin}{\pgfqpoint{2.824854in}{1.962623in}}%
\pgfusepath{use as bounding box, clip}%
\begin{pgfscope}%
\pgfsetbuttcap%
\pgfsetmiterjoin%
\definecolor{currentfill}{rgb}{1.000000,1.000000,1.000000}%
\pgfsetfillcolor{currentfill}%
\pgfsetlinewidth{0.000000pt}%
\definecolor{currentstroke}{rgb}{1.000000,1.000000,1.000000}%
\pgfsetstrokecolor{currentstroke}%
\pgfsetdash{}{0pt}%
\pgfpathmoveto{\pgfqpoint{0.000000in}{0.000000in}}%
\pgfpathlineto{\pgfqpoint{2.824854in}{0.000000in}}%
\pgfpathlineto{\pgfqpoint{2.824854in}{1.962623in}}%
\pgfpathlineto{\pgfqpoint{0.000000in}{1.962623in}}%
\pgfpathclose%
\pgfusepath{fill}%
\end{pgfscope}%
\begin{pgfscope}%
\pgfsetbuttcap%
\pgfsetmiterjoin%
\definecolor{currentfill}{rgb}{1.000000,1.000000,1.000000}%
\pgfsetfillcolor{currentfill}%
\pgfsetlinewidth{0.000000pt}%
\definecolor{currentstroke}{rgb}{0.000000,0.000000,0.000000}%
\pgfsetstrokecolor{currentstroke}%
\pgfsetstrokeopacity{0.000000}%
\pgfsetdash{}{0pt}%
\pgfpathmoveto{\pgfqpoint{0.772069in}{0.515123in}}%
\pgfpathlineto{\pgfqpoint{2.709569in}{0.515123in}}%
\pgfpathlineto{\pgfqpoint{2.709569in}{1.862623in}}%
\pgfpathlineto{\pgfqpoint{0.772069in}{1.862623in}}%
\pgfpathclose%
\pgfusepath{fill}%
\end{pgfscope}%
\begin{pgfscope}%
\pgfsetbuttcap%
\pgfsetroundjoin%
\definecolor{currentfill}{rgb}{0.000000,0.000000,0.000000}%
\pgfsetfillcolor{currentfill}%
\pgfsetlinewidth{0.803000pt}%
\definecolor{currentstroke}{rgb}{0.000000,0.000000,0.000000}%
\pgfsetstrokecolor{currentstroke}%
\pgfsetdash{}{0pt}%
\pgfsys@defobject{currentmarker}{\pgfqpoint{0.000000in}{-0.048611in}}{\pgfqpoint{0.000000in}{0.000000in}}{%
\pgfpathmoveto{\pgfqpoint{0.000000in}{0.000000in}}%
\pgfpathlineto{\pgfqpoint{0.000000in}{-0.048611in}}%
\pgfusepath{stroke,fill}%
}%
\begin{pgfscope}%
\pgfsys@transformshift{1.210683in}{0.515123in}%
\pgfsys@useobject{currentmarker}{}%
\end{pgfscope}%
\end{pgfscope}%
\begin{pgfscope}%
\definecolor{textcolor}{rgb}{0.000000,0.000000,0.000000}%
\pgfsetstrokecolor{textcolor}%
\pgfsetfillcolor{textcolor}%
\pgftext[x=1.210683in,y=0.417901in,,top]{\color{textcolor}\rmfamily\fontsize{10.000000}{12.000000}\selectfont \(\displaystyle 2\)}%
\end{pgfscope}%
\begin{pgfscope}%
\pgfsetbuttcap%
\pgfsetroundjoin%
\definecolor{currentfill}{rgb}{0.000000,0.000000,0.000000}%
\pgfsetfillcolor{currentfill}%
\pgfsetlinewidth{0.803000pt}%
\definecolor{currentstroke}{rgb}{0.000000,0.000000,0.000000}%
\pgfsetstrokecolor{currentstroke}%
\pgfsetdash{}{0pt}%
\pgfsys@defobject{currentmarker}{\pgfqpoint{0.000000in}{-0.048611in}}{\pgfqpoint{0.000000in}{0.000000in}}{%
\pgfpathmoveto{\pgfqpoint{0.000000in}{0.000000in}}%
\pgfpathlineto{\pgfqpoint{0.000000in}{-0.048611in}}%
\pgfusepath{stroke,fill}%
}%
\begin{pgfscope}%
\pgfsys@transformshift{1.703832in}{0.515123in}%
\pgfsys@useobject{currentmarker}{}%
\end{pgfscope}%
\end{pgfscope}%
\begin{pgfscope}%
\definecolor{textcolor}{rgb}{0.000000,0.000000,0.000000}%
\pgfsetstrokecolor{textcolor}%
\pgfsetfillcolor{textcolor}%
\pgftext[x=1.703832in,y=0.417901in,,top]{\color{textcolor}\rmfamily\fontsize{10.000000}{12.000000}\selectfont \(\displaystyle 4\)}%
\end{pgfscope}%
\begin{pgfscope}%
\pgfsetbuttcap%
\pgfsetroundjoin%
\definecolor{currentfill}{rgb}{0.000000,0.000000,0.000000}%
\pgfsetfillcolor{currentfill}%
\pgfsetlinewidth{0.803000pt}%
\definecolor{currentstroke}{rgb}{0.000000,0.000000,0.000000}%
\pgfsetstrokecolor{currentstroke}%
\pgfsetdash{}{0pt}%
\pgfsys@defobject{currentmarker}{\pgfqpoint{0.000000in}{-0.048611in}}{\pgfqpoint{0.000000in}{0.000000in}}{%
\pgfpathmoveto{\pgfqpoint{0.000000in}{0.000000in}}%
\pgfpathlineto{\pgfqpoint{0.000000in}{-0.048611in}}%
\pgfusepath{stroke,fill}%
}%
\begin{pgfscope}%
\pgfsys@transformshift{2.196982in}{0.515123in}%
\pgfsys@useobject{currentmarker}{}%
\end{pgfscope}%
\end{pgfscope}%
\begin{pgfscope}%
\definecolor{textcolor}{rgb}{0.000000,0.000000,0.000000}%
\pgfsetstrokecolor{textcolor}%
\pgfsetfillcolor{textcolor}%
\pgftext[x=2.196982in,y=0.417901in,,top]{\color{textcolor}\rmfamily\fontsize{10.000000}{12.000000}\selectfont \(\displaystyle 6\)}%
\end{pgfscope}%
\begin{pgfscope}%
\pgfsetbuttcap%
\pgfsetroundjoin%
\definecolor{currentfill}{rgb}{0.000000,0.000000,0.000000}%
\pgfsetfillcolor{currentfill}%
\pgfsetlinewidth{0.803000pt}%
\definecolor{currentstroke}{rgb}{0.000000,0.000000,0.000000}%
\pgfsetstrokecolor{currentstroke}%
\pgfsetdash{}{0pt}%
\pgfsys@defobject{currentmarker}{\pgfqpoint{0.000000in}{-0.048611in}}{\pgfqpoint{0.000000in}{0.000000in}}{%
\pgfpathmoveto{\pgfqpoint{0.000000in}{0.000000in}}%
\pgfpathlineto{\pgfqpoint{0.000000in}{-0.048611in}}%
\pgfusepath{stroke,fill}%
}%
\begin{pgfscope}%
\pgfsys@transformshift{2.690131in}{0.515123in}%
\pgfsys@useobject{currentmarker}{}%
\end{pgfscope}%
\end{pgfscope}%
\begin{pgfscope}%
\definecolor{textcolor}{rgb}{0.000000,0.000000,0.000000}%
\pgfsetstrokecolor{textcolor}%
\pgfsetfillcolor{textcolor}%
\pgftext[x=2.690131in,y=0.417901in,,top]{\color{textcolor}\rmfamily\fontsize{10.000000}{12.000000}\selectfont \(\displaystyle 8\)}%
\end{pgfscope}%
\begin{pgfscope}%
\definecolor{textcolor}{rgb}{0.000000,0.000000,0.000000}%
\pgfsetstrokecolor{textcolor}%
\pgfsetfillcolor{textcolor}%
\pgftext[x=1.740819in,y=0.238889in,,top]{\color{textcolor}\rmfamily\fontsize{10.000000}{12.000000}\selectfont I(\(\displaystyle \mu\)A)}%
\end{pgfscope}%
\begin{pgfscope}%
\pgfsetbuttcap%
\pgfsetroundjoin%
\definecolor{currentfill}{rgb}{0.000000,0.000000,0.000000}%
\pgfsetfillcolor{currentfill}%
\pgfsetlinewidth{0.803000pt}%
\definecolor{currentstroke}{rgb}{0.000000,0.000000,0.000000}%
\pgfsetstrokecolor{currentstroke}%
\pgfsetdash{}{0pt}%
\pgfsys@defobject{currentmarker}{\pgfqpoint{-0.048611in}{0.000000in}}{\pgfqpoint{0.000000in}{0.000000in}}{%
\pgfpathmoveto{\pgfqpoint{0.000000in}{0.000000in}}%
\pgfpathlineto{\pgfqpoint{-0.048611in}{0.000000in}}%
\pgfusepath{stroke,fill}%
}%
\begin{pgfscope}%
\pgfsys@transformshift{0.772069in}{0.593207in}%
\pgfsys@useobject{currentmarker}{}%
\end{pgfscope}%
\end{pgfscope}%
\begin{pgfscope}%
\definecolor{textcolor}{rgb}{0.000000,0.000000,0.000000}%
\pgfsetstrokecolor{textcolor}%
\pgfsetfillcolor{textcolor}%
\pgftext[x=0.605402in,y=0.544981in,left,base]{\color{textcolor}\rmfamily\fontsize{10.000000}{12.000000}\selectfont \(\displaystyle 0\)}%
\end{pgfscope}%
\begin{pgfscope}%
\pgfsetbuttcap%
\pgfsetroundjoin%
\definecolor{currentfill}{rgb}{0.000000,0.000000,0.000000}%
\pgfsetfillcolor{currentfill}%
\pgfsetlinewidth{0.803000pt}%
\definecolor{currentstroke}{rgb}{0.000000,0.000000,0.000000}%
\pgfsetstrokecolor{currentstroke}%
\pgfsetdash{}{0pt}%
\pgfsys@defobject{currentmarker}{\pgfqpoint{-0.048611in}{0.000000in}}{\pgfqpoint{0.000000in}{0.000000in}}{%
\pgfpathmoveto{\pgfqpoint{0.000000in}{0.000000in}}%
\pgfpathlineto{\pgfqpoint{-0.048611in}{0.000000in}}%
\pgfusepath{stroke,fill}%
}%
\begin{pgfscope}%
\pgfsys@transformshift{0.772069in}{1.188817in}%
\pgfsys@useobject{currentmarker}{}%
\end{pgfscope}%
\end{pgfscope}%
\begin{pgfscope}%
\definecolor{textcolor}{rgb}{0.000000,0.000000,0.000000}%
\pgfsetstrokecolor{textcolor}%
\pgfsetfillcolor{textcolor}%
\pgftext[x=0.605402in,y=1.140591in,left,base]{\color{textcolor}\rmfamily\fontsize{10.000000}{12.000000}\selectfont \(\displaystyle 5\)}%
\end{pgfscope}%
\begin{pgfscope}%
\pgfsetbuttcap%
\pgfsetroundjoin%
\definecolor{currentfill}{rgb}{0.000000,0.000000,0.000000}%
\pgfsetfillcolor{currentfill}%
\pgfsetlinewidth{0.803000pt}%
\definecolor{currentstroke}{rgb}{0.000000,0.000000,0.000000}%
\pgfsetstrokecolor{currentstroke}%
\pgfsetdash{}{0pt}%
\pgfsys@defobject{currentmarker}{\pgfqpoint{-0.048611in}{0.000000in}}{\pgfqpoint{0.000000in}{0.000000in}}{%
\pgfpathmoveto{\pgfqpoint{0.000000in}{0.000000in}}%
\pgfpathlineto{\pgfqpoint{-0.048611in}{0.000000in}}%
\pgfusepath{stroke,fill}%
}%
\begin{pgfscope}%
\pgfsys@transformshift{0.772069in}{1.784427in}%
\pgfsys@useobject{currentmarker}{}%
\end{pgfscope}%
\end{pgfscope}%
\begin{pgfscope}%
\definecolor{textcolor}{rgb}{0.000000,0.000000,0.000000}%
\pgfsetstrokecolor{textcolor}%
\pgfsetfillcolor{textcolor}%
\pgftext[x=0.535957in,y=1.736201in,left,base]{\color{textcolor}\rmfamily\fontsize{10.000000}{12.000000}\selectfont \(\displaystyle 10\)}%
\end{pgfscope}%
\begin{pgfscope}%
\definecolor{textcolor}{rgb}{0.000000,0.000000,0.000000}%
\pgfsetstrokecolor{textcolor}%
\pgfsetfillcolor{textcolor}%
\pgftext[x=0.258179in,y=1.188873in,,bottom]{\color{textcolor}\rmfamily\fontsize{10.000000}{12.000000}\selectfont V(V)}%
\end{pgfscope}%
\begin{pgfscope}%
\pgfpathrectangle{\pgfqpoint{0.772069in}{0.515123in}}{\pgfqpoint{1.937500in}{1.347500in}}%
\pgfusepath{clip}%
\pgfsetbuttcap%
\pgfsetroundjoin%
\definecolor{currentfill}{rgb}{0.121569,0.466667,0.705882}%
\pgfsetfillcolor{currentfill}%
\pgfsetlinewidth{1.003750pt}%
\definecolor{currentstroke}{rgb}{0.121569,0.466667,0.705882}%
\pgfsetstrokecolor{currentstroke}%
\pgfsetdash{}{0pt}%
\pgfpathmoveto{\pgfqpoint{0.890136in}{0.674474in}}%
\pgfpathcurveto{\pgfqpoint{0.901186in}{0.674474in}}{\pgfqpoint{0.911785in}{0.678864in}}{\pgfqpoint{0.919599in}{0.686678in}}%
\pgfpathcurveto{\pgfqpoint{0.927412in}{0.694491in}}{\pgfqpoint{0.931803in}{0.705090in}}{\pgfqpoint{0.931803in}{0.716141in}}%
\pgfpathcurveto{\pgfqpoint{0.931803in}{0.727191in}}{\pgfqpoint{0.927412in}{0.737790in}}{\pgfqpoint{0.919599in}{0.745603in}}%
\pgfpathcurveto{\pgfqpoint{0.911785in}{0.753417in}}{\pgfqpoint{0.901186in}{0.757807in}}{\pgfqpoint{0.890136in}{0.757807in}}%
\pgfpathcurveto{\pgfqpoint{0.879086in}{0.757807in}}{\pgfqpoint{0.868487in}{0.753417in}}{\pgfqpoint{0.860673in}{0.745603in}}%
\pgfpathcurveto{\pgfqpoint{0.852859in}{0.737790in}}{\pgfqpoint{0.848469in}{0.727191in}}{\pgfqpoint{0.848469in}{0.716141in}}%
\pgfpathcurveto{\pgfqpoint{0.848469in}{0.705090in}}{\pgfqpoint{0.852859in}{0.694491in}}{\pgfqpoint{0.860673in}{0.686678in}}%
\pgfpathcurveto{\pgfqpoint{0.868487in}{0.678864in}}{\pgfqpoint{0.879086in}{0.674474in}}{\pgfqpoint{0.890136in}{0.674474in}}%
\pgfpathclose%
\pgfusepath{stroke,fill}%
\end{pgfscope}%
\begin{pgfscope}%
\pgfpathrectangle{\pgfqpoint{0.772069in}{0.515123in}}{\pgfqpoint{1.937500in}{1.347500in}}%
\pgfusepath{clip}%
\pgfsetbuttcap%
\pgfsetroundjoin%
\definecolor{currentfill}{rgb}{0.121569,0.466667,0.705882}%
\pgfsetfillcolor{currentfill}%
\pgfsetlinewidth{1.003750pt}%
\definecolor{currentstroke}{rgb}{0.121569,0.466667,0.705882}%
\pgfsetstrokecolor{currentstroke}%
\pgfsetdash{}{0pt}%
\pgfpathmoveto{\pgfqpoint{1.087396in}{0.789784in}}%
\pgfpathcurveto{\pgfqpoint{1.098446in}{0.789784in}}{\pgfqpoint{1.109045in}{0.794174in}}{\pgfqpoint{1.116858in}{0.801988in}}%
\pgfpathcurveto{\pgfqpoint{1.124672in}{0.809801in}}{\pgfqpoint{1.129062in}{0.820401in}}{\pgfqpoint{1.129062in}{0.831451in}}%
\pgfpathcurveto{\pgfqpoint{1.129062in}{0.842501in}}{\pgfqpoint{1.124672in}{0.853100in}}{\pgfqpoint{1.116858in}{0.860913in}}%
\pgfpathcurveto{\pgfqpoint{1.109045in}{0.868727in}}{\pgfqpoint{1.098446in}{0.873117in}}{\pgfqpoint{1.087396in}{0.873117in}}%
\pgfpathcurveto{\pgfqpoint{1.076346in}{0.873117in}}{\pgfqpoint{1.065746in}{0.868727in}}{\pgfqpoint{1.057933in}{0.860913in}}%
\pgfpathcurveto{\pgfqpoint{1.050119in}{0.853100in}}{\pgfqpoint{1.045729in}{0.842501in}}{\pgfqpoint{1.045729in}{0.831451in}}%
\pgfpathcurveto{\pgfqpoint{1.045729in}{0.820401in}}{\pgfqpoint{1.050119in}{0.809801in}}{\pgfqpoint{1.057933in}{0.801988in}}%
\pgfpathcurveto{\pgfqpoint{1.065746in}{0.794174in}}{\pgfqpoint{1.076346in}{0.789784in}}{\pgfqpoint{1.087396in}{0.789784in}}%
\pgfpathclose%
\pgfusepath{stroke,fill}%
\end{pgfscope}%
\begin{pgfscope}%
\pgfpathrectangle{\pgfqpoint{0.772069in}{0.515123in}}{\pgfqpoint{1.937500in}{1.347500in}}%
\pgfusepath{clip}%
\pgfsetbuttcap%
\pgfsetroundjoin%
\definecolor{currentfill}{rgb}{0.121569,0.466667,0.705882}%
\pgfsetfillcolor{currentfill}%
\pgfsetlinewidth{1.003750pt}%
\definecolor{currentstroke}{rgb}{0.121569,0.466667,0.705882}%
\pgfsetstrokecolor{currentstroke}%
\pgfsetdash{}{0pt}%
\pgfpathmoveto{\pgfqpoint{1.259998in}{0.913671in}}%
\pgfpathcurveto{\pgfqpoint{1.271048in}{0.913671in}}{\pgfqpoint{1.281647in}{0.918061in}}{\pgfqpoint{1.289461in}{0.925875in}}%
\pgfpathcurveto{\pgfqpoint{1.297274in}{0.933688in}}{\pgfqpoint{1.301665in}{0.944287in}}{\pgfqpoint{1.301665in}{0.955338in}}%
\pgfpathcurveto{\pgfqpoint{1.301665in}{0.966388in}}{\pgfqpoint{1.297274in}{0.976987in}}{\pgfqpoint{1.289461in}{0.984800in}}%
\pgfpathcurveto{\pgfqpoint{1.281647in}{0.992614in}}{\pgfqpoint{1.271048in}{0.997004in}}{\pgfqpoint{1.259998in}{0.997004in}}%
\pgfpathcurveto{\pgfqpoint{1.248948in}{0.997004in}}{\pgfqpoint{1.238349in}{0.992614in}}{\pgfqpoint{1.230535in}{0.984800in}}%
\pgfpathcurveto{\pgfqpoint{1.222722in}{0.976987in}}{\pgfqpoint{1.218331in}{0.966388in}}{\pgfqpoint{1.218331in}{0.955338in}}%
\pgfpathcurveto{\pgfqpoint{1.218331in}{0.944287in}}{\pgfqpoint{1.222722in}{0.933688in}}{\pgfqpoint{1.230535in}{0.925875in}}%
\pgfpathcurveto{\pgfqpoint{1.238349in}{0.918061in}}{\pgfqpoint{1.248948in}{0.913671in}}{\pgfqpoint{1.259998in}{0.913671in}}%
\pgfpathclose%
\pgfusepath{stroke,fill}%
\end{pgfscope}%
\begin{pgfscope}%
\pgfpathrectangle{\pgfqpoint{0.772069in}{0.515123in}}{\pgfqpoint{1.937500in}{1.347500in}}%
\pgfusepath{clip}%
\pgfsetbuttcap%
\pgfsetroundjoin%
\definecolor{currentfill}{rgb}{0.121569,0.466667,0.705882}%
\pgfsetfillcolor{currentfill}%
\pgfsetlinewidth{1.003750pt}%
\definecolor{currentstroke}{rgb}{0.121569,0.466667,0.705882}%
\pgfsetstrokecolor{currentstroke}%
\pgfsetdash{}{0pt}%
\pgfpathmoveto{\pgfqpoint{1.457258in}{1.035175in}}%
\pgfpathcurveto{\pgfqpoint{1.468308in}{1.035175in}}{\pgfqpoint{1.478907in}{1.039566in}}{\pgfqpoint{1.486721in}{1.047379in}}%
\pgfpathcurveto{\pgfqpoint{1.494534in}{1.055193in}}{\pgfqpoint{1.498924in}{1.065792in}}{\pgfqpoint{1.498924in}{1.076842in}}%
\pgfpathcurveto{\pgfqpoint{1.498924in}{1.087892in}}{\pgfqpoint{1.494534in}{1.098491in}}{\pgfqpoint{1.486721in}{1.106305in}}%
\pgfpathcurveto{\pgfqpoint{1.478907in}{1.114118in}}{\pgfqpoint{1.468308in}{1.118509in}}{\pgfqpoint{1.457258in}{1.118509in}}%
\pgfpathcurveto{\pgfqpoint{1.446208in}{1.118509in}}{\pgfqpoint{1.435609in}{1.114118in}}{\pgfqpoint{1.427795in}{1.106305in}}%
\pgfpathcurveto{\pgfqpoint{1.419981in}{1.098491in}}{\pgfqpoint{1.415591in}{1.087892in}}{\pgfqpoint{1.415591in}{1.076842in}}%
\pgfpathcurveto{\pgfqpoint{1.415591in}{1.065792in}}{\pgfqpoint{1.419981in}{1.055193in}}{\pgfqpoint{1.427795in}{1.047379in}}%
\pgfpathcurveto{\pgfqpoint{1.435609in}{1.039566in}}{\pgfqpoint{1.446208in}{1.035175in}}{\pgfqpoint{1.457258in}{1.035175in}}%
\pgfpathclose%
\pgfusepath{stroke,fill}%
\end{pgfscope}%
\begin{pgfscope}%
\pgfpathrectangle{\pgfqpoint{0.772069in}{0.515123in}}{\pgfqpoint{1.937500in}{1.347500in}}%
\pgfusepath{clip}%
\pgfsetbuttcap%
\pgfsetroundjoin%
\definecolor{currentfill}{rgb}{0.121569,0.466667,0.705882}%
\pgfsetfillcolor{currentfill}%
\pgfsetlinewidth{1.003750pt}%
\definecolor{currentstroke}{rgb}{0.121569,0.466667,0.705882}%
\pgfsetstrokecolor{currentstroke}%
\pgfsetdash{}{0pt}%
\pgfpathmoveto{\pgfqpoint{1.654518in}{1.156680in}}%
\pgfpathcurveto{\pgfqpoint{1.665568in}{1.156680in}}{\pgfqpoint{1.676167in}{1.161070in}}{\pgfqpoint{1.683980in}{1.168884in}}%
\pgfpathcurveto{\pgfqpoint{1.691794in}{1.176697in}}{\pgfqpoint{1.696184in}{1.187296in}}{\pgfqpoint{1.696184in}{1.198346in}}%
\pgfpathcurveto{\pgfqpoint{1.696184in}{1.209397in}}{\pgfqpoint{1.691794in}{1.219996in}}{\pgfqpoint{1.683980in}{1.227809in}}%
\pgfpathcurveto{\pgfqpoint{1.676167in}{1.235623in}}{\pgfqpoint{1.665568in}{1.240013in}}{\pgfqpoint{1.654518in}{1.240013in}}%
\pgfpathcurveto{\pgfqpoint{1.643467in}{1.240013in}}{\pgfqpoint{1.632868in}{1.235623in}}{\pgfqpoint{1.625055in}{1.227809in}}%
\pgfpathcurveto{\pgfqpoint{1.617241in}{1.219996in}}{\pgfqpoint{1.612851in}{1.209397in}}{\pgfqpoint{1.612851in}{1.198346in}}%
\pgfpathcurveto{\pgfqpoint{1.612851in}{1.187296in}}{\pgfqpoint{1.617241in}{1.176697in}}{\pgfqpoint{1.625055in}{1.168884in}}%
\pgfpathcurveto{\pgfqpoint{1.632868in}{1.161070in}}{\pgfqpoint{1.643467in}{1.156680in}}{\pgfqpoint{1.654518in}{1.156680in}}%
\pgfpathclose%
\pgfusepath{stroke,fill}%
\end{pgfscope}%
\begin{pgfscope}%
\pgfpathrectangle{\pgfqpoint{0.772069in}{0.515123in}}{\pgfqpoint{1.937500in}{1.347500in}}%
\pgfusepath{clip}%
\pgfsetbuttcap%
\pgfsetroundjoin%
\definecolor{currentfill}{rgb}{0.121569,0.466667,0.705882}%
\pgfsetfillcolor{currentfill}%
\pgfsetlinewidth{1.003750pt}%
\definecolor{currentstroke}{rgb}{0.121569,0.466667,0.705882}%
\pgfsetstrokecolor{currentstroke}%
\pgfsetdash{}{0pt}%
\pgfpathmoveto{\pgfqpoint{1.827120in}{1.272228in}}%
\pgfpathcurveto{\pgfqpoint{1.838170in}{1.272228in}}{\pgfqpoint{1.848769in}{1.276618in}}{\pgfqpoint{1.856583in}{1.284432in}}%
\pgfpathcurveto{\pgfqpoint{1.864396in}{1.292246in}}{\pgfqpoint{1.868787in}{1.302845in}}{\pgfqpoint{1.868787in}{1.313895in}}%
\pgfpathcurveto{\pgfqpoint{1.868787in}{1.324945in}}{\pgfqpoint{1.864396in}{1.335544in}}{\pgfqpoint{1.856583in}{1.343358in}}%
\pgfpathcurveto{\pgfqpoint{1.848769in}{1.351171in}}{\pgfqpoint{1.838170in}{1.355561in}}{\pgfqpoint{1.827120in}{1.355561in}}%
\pgfpathcurveto{\pgfqpoint{1.816070in}{1.355561in}}{\pgfqpoint{1.805471in}{1.351171in}}{\pgfqpoint{1.797657in}{1.343358in}}%
\pgfpathcurveto{\pgfqpoint{1.789843in}{1.335544in}}{\pgfqpoint{1.785453in}{1.324945in}}{\pgfqpoint{1.785453in}{1.313895in}}%
\pgfpathcurveto{\pgfqpoint{1.785453in}{1.302845in}}{\pgfqpoint{1.789843in}{1.292246in}}{\pgfqpoint{1.797657in}{1.284432in}}%
\pgfpathcurveto{\pgfqpoint{1.805471in}{1.276618in}}{\pgfqpoint{1.816070in}{1.272228in}}{\pgfqpoint{1.827120in}{1.272228in}}%
\pgfpathclose%
\pgfusepath{stroke,fill}%
\end{pgfscope}%
\begin{pgfscope}%
\pgfpathrectangle{\pgfqpoint{0.772069in}{0.515123in}}{\pgfqpoint{1.937500in}{1.347500in}}%
\pgfusepath{clip}%
\pgfsetbuttcap%
\pgfsetroundjoin%
\definecolor{currentfill}{rgb}{0.121569,0.466667,0.705882}%
\pgfsetfillcolor{currentfill}%
\pgfsetlinewidth{1.003750pt}%
\definecolor{currentstroke}{rgb}{0.121569,0.466667,0.705882}%
\pgfsetstrokecolor{currentstroke}%
\pgfsetdash{}{0pt}%
\pgfpathmoveto{\pgfqpoint{2.024380in}{1.397306in}}%
\pgfpathcurveto{\pgfqpoint{2.035430in}{1.397306in}}{\pgfqpoint{2.046029in}{1.401696in}}{\pgfqpoint{2.053842in}{1.409510in}}%
\pgfpathcurveto{\pgfqpoint{2.061656in}{1.417324in}}{\pgfqpoint{2.066046in}{1.427923in}}{\pgfqpoint{2.066046in}{1.438973in}}%
\pgfpathcurveto{\pgfqpoint{2.066046in}{1.450023in}}{\pgfqpoint{2.061656in}{1.460622in}}{\pgfqpoint{2.053842in}{1.468436in}}%
\pgfpathcurveto{\pgfqpoint{2.046029in}{1.476249in}}{\pgfqpoint{2.035430in}{1.480639in}}{\pgfqpoint{2.024380in}{1.480639in}}%
\pgfpathcurveto{\pgfqpoint{2.013329in}{1.480639in}}{\pgfqpoint{2.002730in}{1.476249in}}{\pgfqpoint{1.994917in}{1.468436in}}%
\pgfpathcurveto{\pgfqpoint{1.987103in}{1.460622in}}{\pgfqpoint{1.982713in}{1.450023in}}{\pgfqpoint{1.982713in}{1.438973in}}%
\pgfpathcurveto{\pgfqpoint{1.982713in}{1.427923in}}{\pgfqpoint{1.987103in}{1.417324in}}{\pgfqpoint{1.994917in}{1.409510in}}%
\pgfpathcurveto{\pgfqpoint{2.002730in}{1.401696in}}{\pgfqpoint{2.013329in}{1.397306in}}{\pgfqpoint{2.024380in}{1.397306in}}%
\pgfpathclose%
\pgfusepath{stroke,fill}%
\end{pgfscope}%
\begin{pgfscope}%
\pgfpathrectangle{\pgfqpoint{0.772069in}{0.515123in}}{\pgfqpoint{1.937500in}{1.347500in}}%
\pgfusepath{clip}%
\pgfsetbuttcap%
\pgfsetroundjoin%
\definecolor{currentfill}{rgb}{0.121569,0.466667,0.705882}%
\pgfsetfillcolor{currentfill}%
\pgfsetlinewidth{1.003750pt}%
\definecolor{currentstroke}{rgb}{0.121569,0.466667,0.705882}%
\pgfsetstrokecolor{currentstroke}%
\pgfsetdash{}{0pt}%
\pgfpathmoveto{\pgfqpoint{2.221639in}{1.517619in}}%
\pgfpathcurveto{\pgfqpoint{2.232690in}{1.517619in}}{\pgfqpoint{2.243289in}{1.522010in}}{\pgfqpoint{2.251102in}{1.529823in}}%
\pgfpathcurveto{\pgfqpoint{2.258916in}{1.537637in}}{\pgfqpoint{2.263306in}{1.548236in}}{\pgfqpoint{2.263306in}{1.559286in}}%
\pgfpathcurveto{\pgfqpoint{2.263306in}{1.570336in}}{\pgfqpoint{2.258916in}{1.580935in}}{\pgfqpoint{2.251102in}{1.588749in}}%
\pgfpathcurveto{\pgfqpoint{2.243289in}{1.596562in}}{\pgfqpoint{2.232690in}{1.600953in}}{\pgfqpoint{2.221639in}{1.600953in}}%
\pgfpathcurveto{\pgfqpoint{2.210589in}{1.600953in}}{\pgfqpoint{2.199990in}{1.596562in}}{\pgfqpoint{2.192177in}{1.588749in}}%
\pgfpathcurveto{\pgfqpoint{2.184363in}{1.580935in}}{\pgfqpoint{2.179973in}{1.570336in}}{\pgfqpoint{2.179973in}{1.559286in}}%
\pgfpathcurveto{\pgfqpoint{2.179973in}{1.548236in}}{\pgfqpoint{2.184363in}{1.537637in}}{\pgfqpoint{2.192177in}{1.529823in}}%
\pgfpathcurveto{\pgfqpoint{2.199990in}{1.522010in}}{\pgfqpoint{2.210589in}{1.517619in}}{\pgfqpoint{2.221639in}{1.517619in}}%
\pgfpathclose%
\pgfusepath{stroke,fill}%
\end{pgfscope}%
\begin{pgfscope}%
\pgfpathrectangle{\pgfqpoint{0.772069in}{0.515123in}}{\pgfqpoint{1.937500in}{1.347500in}}%
\pgfusepath{clip}%
\pgfsetbuttcap%
\pgfsetroundjoin%
\definecolor{currentfill}{rgb}{0.121569,0.466667,0.705882}%
\pgfsetfillcolor{currentfill}%
\pgfsetlinewidth{1.003750pt}%
\definecolor{currentstroke}{rgb}{0.121569,0.466667,0.705882}%
\pgfsetstrokecolor{currentstroke}%
\pgfsetdash{}{0pt}%
\pgfpathmoveto{\pgfqpoint{2.394242in}{1.635550in}}%
\pgfpathcurveto{\pgfqpoint{2.405292in}{1.635550in}}{\pgfqpoint{2.415891in}{1.639940in}}{\pgfqpoint{2.423704in}{1.647754in}}%
\pgfpathcurveto{\pgfqpoint{2.431518in}{1.655568in}}{\pgfqpoint{2.435908in}{1.666167in}}{\pgfqpoint{2.435908in}{1.677217in}}%
\pgfpathcurveto{\pgfqpoint{2.435908in}{1.688267in}}{\pgfqpoint{2.431518in}{1.698866in}}{\pgfqpoint{2.423704in}{1.706680in}}%
\pgfpathcurveto{\pgfqpoint{2.415891in}{1.714493in}}{\pgfqpoint{2.405292in}{1.718883in}}{\pgfqpoint{2.394242in}{1.718883in}}%
\pgfpathcurveto{\pgfqpoint{2.383192in}{1.718883in}}{\pgfqpoint{2.372593in}{1.714493in}}{\pgfqpoint{2.364779in}{1.706680in}}%
\pgfpathcurveto{\pgfqpoint{2.356965in}{1.698866in}}{\pgfqpoint{2.352575in}{1.688267in}}{\pgfqpoint{2.352575in}{1.677217in}}%
\pgfpathcurveto{\pgfqpoint{2.352575in}{1.666167in}}{\pgfqpoint{2.356965in}{1.655568in}}{\pgfqpoint{2.364779in}{1.647754in}}%
\pgfpathcurveto{\pgfqpoint{2.372593in}{1.639940in}}{\pgfqpoint{2.383192in}{1.635550in}}{\pgfqpoint{2.394242in}{1.635550in}}%
\pgfpathclose%
\pgfusepath{stroke,fill}%
\end{pgfscope}%
\begin{pgfscope}%
\pgfpathrectangle{\pgfqpoint{0.772069in}{0.515123in}}{\pgfqpoint{1.937500in}{1.347500in}}%
\pgfusepath{clip}%
\pgfsetbuttcap%
\pgfsetroundjoin%
\definecolor{currentfill}{rgb}{0.121569,0.466667,0.705882}%
\pgfsetfillcolor{currentfill}%
\pgfsetlinewidth{1.003750pt}%
\definecolor{currentstroke}{rgb}{0.121569,0.466667,0.705882}%
\pgfsetstrokecolor{currentstroke}%
\pgfsetdash{}{0pt}%
\pgfpathmoveto{\pgfqpoint{2.591501in}{1.757055in}}%
\pgfpathcurveto{\pgfqpoint{2.602552in}{1.757055in}}{\pgfqpoint{2.613151in}{1.761445in}}{\pgfqpoint{2.620964in}{1.769258in}}%
\pgfpathcurveto{\pgfqpoint{2.628778in}{1.777072in}}{\pgfqpoint{2.633168in}{1.787671in}}{\pgfqpoint{2.633168in}{1.798721in}}%
\pgfpathcurveto{\pgfqpoint{2.633168in}{1.809771in}}{\pgfqpoint{2.628778in}{1.820370in}}{\pgfqpoint{2.620964in}{1.828184in}}%
\pgfpathcurveto{\pgfqpoint{2.613151in}{1.835998in}}{\pgfqpoint{2.602552in}{1.840388in}}{\pgfqpoint{2.591501in}{1.840388in}}%
\pgfpathcurveto{\pgfqpoint{2.580451in}{1.840388in}}{\pgfqpoint{2.569852in}{1.835998in}}{\pgfqpoint{2.562039in}{1.828184in}}%
\pgfpathcurveto{\pgfqpoint{2.554225in}{1.820370in}}{\pgfqpoint{2.549835in}{1.809771in}}{\pgfqpoint{2.549835in}{1.798721in}}%
\pgfpathcurveto{\pgfqpoint{2.549835in}{1.787671in}}{\pgfqpoint{2.554225in}{1.777072in}}{\pgfqpoint{2.562039in}{1.769258in}}%
\pgfpathcurveto{\pgfqpoint{2.569852in}{1.761445in}}{\pgfqpoint{2.580451in}{1.757055in}}{\pgfqpoint{2.591501in}{1.757055in}}%
\pgfpathclose%
\pgfusepath{stroke,fill}%
\end{pgfscope}%
\begin{pgfscope}%
\pgfpathrectangle{\pgfqpoint{0.772069in}{0.515123in}}{\pgfqpoint{1.937500in}{1.347500in}}%
\pgfusepath{clip}%
\pgfsetbuttcap%
\pgfsetroundjoin%
\definecolor{currentfill}{rgb}{1.000000,0.388235,0.278431}%
\pgfsetfillcolor{currentfill}%
\pgfsetlinewidth{1.003750pt}%
\definecolor{currentstroke}{rgb}{1.000000,0.388235,0.278431}%
\pgfsetstrokecolor{currentstroke}%
\pgfsetdash{}{0pt}%
\pgfpathmoveto{\pgfqpoint{0.890136in}{0.567824in}}%
\pgfpathcurveto{\pgfqpoint{0.901186in}{0.567824in}}{\pgfqpoint{0.911785in}{0.572214in}}{\pgfqpoint{0.919599in}{0.580028in}}%
\pgfpathcurveto{\pgfqpoint{0.927412in}{0.587841in}}{\pgfqpoint{0.931803in}{0.598440in}}{\pgfqpoint{0.931803in}{0.609491in}}%
\pgfpathcurveto{\pgfqpoint{0.931803in}{0.620541in}}{\pgfqpoint{0.927412in}{0.631140in}}{\pgfqpoint{0.919599in}{0.638953in}}%
\pgfpathcurveto{\pgfqpoint{0.911785in}{0.646767in}}{\pgfqpoint{0.901186in}{0.651157in}}{\pgfqpoint{0.890136in}{0.651157in}}%
\pgfpathcurveto{\pgfqpoint{0.879086in}{0.651157in}}{\pgfqpoint{0.868487in}{0.646767in}}{\pgfqpoint{0.860673in}{0.638953in}}%
\pgfpathcurveto{\pgfqpoint{0.852859in}{0.631140in}}{\pgfqpoint{0.848469in}{0.620541in}}{\pgfqpoint{0.848469in}{0.609491in}}%
\pgfpathcurveto{\pgfqpoint{0.848469in}{0.598440in}}{\pgfqpoint{0.852859in}{0.587841in}}{\pgfqpoint{0.860673in}{0.580028in}}%
\pgfpathcurveto{\pgfqpoint{0.868487in}{0.572214in}}{\pgfqpoint{0.879086in}{0.567824in}}{\pgfqpoint{0.890136in}{0.567824in}}%
\pgfpathclose%
\pgfusepath{stroke,fill}%
\end{pgfscope}%
\begin{pgfscope}%
\pgfpathrectangle{\pgfqpoint{0.772069in}{0.515123in}}{\pgfqpoint{1.937500in}{1.347500in}}%
\pgfusepath{clip}%
\pgfsetbuttcap%
\pgfsetroundjoin%
\definecolor{currentfill}{rgb}{1.000000,0.388235,0.278431}%
\pgfsetfillcolor{currentfill}%
\pgfsetlinewidth{1.003750pt}%
\definecolor{currentstroke}{rgb}{1.000000,0.388235,0.278431}%
\pgfsetstrokecolor{currentstroke}%
\pgfsetdash{}{0pt}%
\pgfpathmoveto{\pgfqpoint{1.087396in}{0.582631in}}%
\pgfpathcurveto{\pgfqpoint{1.098446in}{0.582631in}}{\pgfqpoint{1.109045in}{0.587021in}}{\pgfqpoint{1.116858in}{0.594835in}}%
\pgfpathcurveto{\pgfqpoint{1.124672in}{0.602648in}}{\pgfqpoint{1.129062in}{0.613247in}}{\pgfqpoint{1.129062in}{0.624297in}}%
\pgfpathcurveto{\pgfqpoint{1.129062in}{0.635348in}}{\pgfqpoint{1.124672in}{0.645947in}}{\pgfqpoint{1.116858in}{0.653760in}}%
\pgfpathcurveto{\pgfqpoint{1.109045in}{0.661574in}}{\pgfqpoint{1.098446in}{0.665964in}}{\pgfqpoint{1.087396in}{0.665964in}}%
\pgfpathcurveto{\pgfqpoint{1.076346in}{0.665964in}}{\pgfqpoint{1.065746in}{0.661574in}}{\pgfqpoint{1.057933in}{0.653760in}}%
\pgfpathcurveto{\pgfqpoint{1.050119in}{0.645947in}}{\pgfqpoint{1.045729in}{0.635348in}}{\pgfqpoint{1.045729in}{0.624297in}}%
\pgfpathcurveto{\pgfqpoint{1.045729in}{0.613247in}}{\pgfqpoint{1.050119in}{0.602648in}}{\pgfqpoint{1.057933in}{0.594835in}}%
\pgfpathcurveto{\pgfqpoint{1.065746in}{0.587021in}}{\pgfqpoint{1.076346in}{0.582631in}}{\pgfqpoint{1.087396in}{0.582631in}}%
\pgfpathclose%
\pgfusepath{stroke,fill}%
\end{pgfscope}%
\begin{pgfscope}%
\pgfpathrectangle{\pgfqpoint{0.772069in}{0.515123in}}{\pgfqpoint{1.937500in}{1.347500in}}%
\pgfusepath{clip}%
\pgfsetbuttcap%
\pgfsetroundjoin%
\definecolor{currentfill}{rgb}{1.000000,0.388235,0.278431}%
\pgfsetfillcolor{currentfill}%
\pgfsetlinewidth{1.003750pt}%
\definecolor{currentstroke}{rgb}{1.000000,0.388235,0.278431}%
\pgfsetstrokecolor{currentstroke}%
\pgfsetdash{}{0pt}%
\pgfpathmoveto{\pgfqpoint{1.259998in}{0.586919in}}%
\pgfpathcurveto{\pgfqpoint{1.271048in}{0.586919in}}{\pgfqpoint{1.281647in}{0.591309in}}{\pgfqpoint{1.289461in}{0.599123in}}%
\pgfpathcurveto{\pgfqpoint{1.297274in}{0.606937in}}{\pgfqpoint{1.301665in}{0.617536in}}{\pgfqpoint{1.301665in}{0.628586in}}%
\pgfpathcurveto{\pgfqpoint{1.301665in}{0.639636in}}{\pgfqpoint{1.297274in}{0.650235in}}{\pgfqpoint{1.289461in}{0.658049in}}%
\pgfpathcurveto{\pgfqpoint{1.281647in}{0.665862in}}{\pgfqpoint{1.271048in}{0.670253in}}{\pgfqpoint{1.259998in}{0.670253in}}%
\pgfpathcurveto{\pgfqpoint{1.248948in}{0.670253in}}{\pgfqpoint{1.238349in}{0.665862in}}{\pgfqpoint{1.230535in}{0.658049in}}%
\pgfpathcurveto{\pgfqpoint{1.222722in}{0.650235in}}{\pgfqpoint{1.218331in}{0.639636in}}{\pgfqpoint{1.218331in}{0.628586in}}%
\pgfpathcurveto{\pgfqpoint{1.218331in}{0.617536in}}{\pgfqpoint{1.222722in}{0.606937in}}{\pgfqpoint{1.230535in}{0.599123in}}%
\pgfpathcurveto{\pgfqpoint{1.238349in}{0.591309in}}{\pgfqpoint{1.248948in}{0.586919in}}{\pgfqpoint{1.259998in}{0.586919in}}%
\pgfpathclose%
\pgfusepath{stroke,fill}%
\end{pgfscope}%
\begin{pgfscope}%
\pgfpathrectangle{\pgfqpoint{0.772069in}{0.515123in}}{\pgfqpoint{1.937500in}{1.347500in}}%
\pgfusepath{clip}%
\pgfsetbuttcap%
\pgfsetroundjoin%
\definecolor{currentfill}{rgb}{1.000000,0.388235,0.278431}%
\pgfsetfillcolor{currentfill}%
\pgfsetlinewidth{1.003750pt}%
\definecolor{currentstroke}{rgb}{1.000000,0.388235,0.278431}%
\pgfsetstrokecolor{currentstroke}%
\pgfsetdash{}{0pt}%
\pgfpathmoveto{\pgfqpoint{1.457258in}{0.614794in}}%
\pgfpathcurveto{\pgfqpoint{1.468308in}{0.614794in}}{\pgfqpoint{1.478907in}{0.619184in}}{\pgfqpoint{1.486721in}{0.626998in}}%
\pgfpathcurveto{\pgfqpoint{1.494534in}{0.634811in}}{\pgfqpoint{1.498924in}{0.645410in}}{\pgfqpoint{1.498924in}{0.656460in}}%
\pgfpathcurveto{\pgfqpoint{1.498924in}{0.667511in}}{\pgfqpoint{1.494534in}{0.678110in}}{\pgfqpoint{1.486721in}{0.685923in}}%
\pgfpathcurveto{\pgfqpoint{1.478907in}{0.693737in}}{\pgfqpoint{1.468308in}{0.698127in}}{\pgfqpoint{1.457258in}{0.698127in}}%
\pgfpathcurveto{\pgfqpoint{1.446208in}{0.698127in}}{\pgfqpoint{1.435609in}{0.693737in}}{\pgfqpoint{1.427795in}{0.685923in}}%
\pgfpathcurveto{\pgfqpoint{1.419981in}{0.678110in}}{\pgfqpoint{1.415591in}{0.667511in}}{\pgfqpoint{1.415591in}{0.656460in}}%
\pgfpathcurveto{\pgfqpoint{1.415591in}{0.645410in}}{\pgfqpoint{1.419981in}{0.634811in}}{\pgfqpoint{1.427795in}{0.626998in}}%
\pgfpathcurveto{\pgfqpoint{1.435609in}{0.619184in}}{\pgfqpoint{1.446208in}{0.614794in}}{\pgfqpoint{1.457258in}{0.614794in}}%
\pgfpathclose%
\pgfusepath{stroke,fill}%
\end{pgfscope}%
\begin{pgfscope}%
\pgfpathrectangle{\pgfqpoint{0.772069in}{0.515123in}}{\pgfqpoint{1.937500in}{1.347500in}}%
\pgfusepath{clip}%
\pgfsetbuttcap%
\pgfsetroundjoin%
\definecolor{currentfill}{rgb}{1.000000,0.388235,0.278431}%
\pgfsetfillcolor{currentfill}%
\pgfsetlinewidth{1.003750pt}%
\definecolor{currentstroke}{rgb}{1.000000,0.388235,0.278431}%
\pgfsetstrokecolor{currentstroke}%
\pgfsetdash{}{0pt}%
\pgfpathmoveto{\pgfqpoint{1.654518in}{0.630637in}}%
\pgfpathcurveto{\pgfqpoint{1.665568in}{0.630637in}}{\pgfqpoint{1.676167in}{0.635027in}}{\pgfqpoint{1.683980in}{0.642841in}}%
\pgfpathcurveto{\pgfqpoint{1.691794in}{0.650654in}}{\pgfqpoint{1.696184in}{0.661254in}}{\pgfqpoint{1.696184in}{0.672304in}}%
\pgfpathcurveto{\pgfqpoint{1.696184in}{0.683354in}}{\pgfqpoint{1.691794in}{0.693953in}}{\pgfqpoint{1.683980in}{0.701766in}}%
\pgfpathcurveto{\pgfqpoint{1.676167in}{0.709580in}}{\pgfqpoint{1.665568in}{0.713970in}}{\pgfqpoint{1.654518in}{0.713970in}}%
\pgfpathcurveto{\pgfqpoint{1.643467in}{0.713970in}}{\pgfqpoint{1.632868in}{0.709580in}}{\pgfqpoint{1.625055in}{0.701766in}}%
\pgfpathcurveto{\pgfqpoint{1.617241in}{0.693953in}}{\pgfqpoint{1.612851in}{0.683354in}}{\pgfqpoint{1.612851in}{0.672304in}}%
\pgfpathcurveto{\pgfqpoint{1.612851in}{0.661254in}}{\pgfqpoint{1.617241in}{0.650654in}}{\pgfqpoint{1.625055in}{0.642841in}}%
\pgfpathcurveto{\pgfqpoint{1.632868in}{0.635027in}}{\pgfqpoint{1.643467in}{0.630637in}}{\pgfqpoint{1.654518in}{0.630637in}}%
\pgfpathclose%
\pgfusepath{stroke,fill}%
\end{pgfscope}%
\begin{pgfscope}%
\pgfpathrectangle{\pgfqpoint{0.772069in}{0.515123in}}{\pgfqpoint{1.937500in}{1.347500in}}%
\pgfusepath{clip}%
\pgfsetbuttcap%
\pgfsetroundjoin%
\definecolor{currentfill}{rgb}{1.000000,0.388235,0.278431}%
\pgfsetfillcolor{currentfill}%
\pgfsetlinewidth{1.003750pt}%
\definecolor{currentstroke}{rgb}{1.000000,0.388235,0.278431}%
\pgfsetstrokecolor{currentstroke}%
\pgfsetdash{}{0pt}%
\pgfpathmoveto{\pgfqpoint{1.827120in}{0.645765in}}%
\pgfpathcurveto{\pgfqpoint{1.838170in}{0.645765in}}{\pgfqpoint{1.848769in}{0.650156in}}{\pgfqpoint{1.856583in}{0.657969in}}%
\pgfpathcurveto{\pgfqpoint{1.864396in}{0.665783in}}{\pgfqpoint{1.868787in}{0.676382in}}{\pgfqpoint{1.868787in}{0.687432in}}%
\pgfpathcurveto{\pgfqpoint{1.868787in}{0.698482in}}{\pgfqpoint{1.864396in}{0.709081in}}{\pgfqpoint{1.856583in}{0.716895in}}%
\pgfpathcurveto{\pgfqpoint{1.848769in}{0.724709in}}{\pgfqpoint{1.838170in}{0.729099in}}{\pgfqpoint{1.827120in}{0.729099in}}%
\pgfpathcurveto{\pgfqpoint{1.816070in}{0.729099in}}{\pgfqpoint{1.805471in}{0.724709in}}{\pgfqpoint{1.797657in}{0.716895in}}%
\pgfpathcurveto{\pgfqpoint{1.789843in}{0.709081in}}{\pgfqpoint{1.785453in}{0.698482in}}{\pgfqpoint{1.785453in}{0.687432in}}%
\pgfpathcurveto{\pgfqpoint{1.785453in}{0.676382in}}{\pgfqpoint{1.789843in}{0.665783in}}{\pgfqpoint{1.797657in}{0.657969in}}%
\pgfpathcurveto{\pgfqpoint{1.805471in}{0.650156in}}{\pgfqpoint{1.816070in}{0.645765in}}{\pgfqpoint{1.827120in}{0.645765in}}%
\pgfpathclose%
\pgfusepath{stroke,fill}%
\end{pgfscope}%
\begin{pgfscope}%
\pgfpathrectangle{\pgfqpoint{0.772069in}{0.515123in}}{\pgfqpoint{1.937500in}{1.347500in}}%
\pgfusepath{clip}%
\pgfsetbuttcap%
\pgfsetroundjoin%
\definecolor{currentfill}{rgb}{1.000000,0.388235,0.278431}%
\pgfsetfillcolor{currentfill}%
\pgfsetlinewidth{1.003750pt}%
\definecolor{currentstroke}{rgb}{1.000000,0.388235,0.278431}%
\pgfsetstrokecolor{currentstroke}%
\pgfsetdash{}{0pt}%
\pgfpathmoveto{\pgfqpoint{2.024380in}{0.662204in}}%
\pgfpathcurveto{\pgfqpoint{2.035430in}{0.662204in}}{\pgfqpoint{2.046029in}{0.666595in}}{\pgfqpoint{2.053842in}{0.674408in}}%
\pgfpathcurveto{\pgfqpoint{2.061656in}{0.682222in}}{\pgfqpoint{2.066046in}{0.692821in}}{\pgfqpoint{2.066046in}{0.703871in}}%
\pgfpathcurveto{\pgfqpoint{2.066046in}{0.714921in}}{\pgfqpoint{2.061656in}{0.725520in}}{\pgfqpoint{2.053842in}{0.733334in}}%
\pgfpathcurveto{\pgfqpoint{2.046029in}{0.741147in}}{\pgfqpoint{2.035430in}{0.745538in}}{\pgfqpoint{2.024380in}{0.745538in}}%
\pgfpathcurveto{\pgfqpoint{2.013329in}{0.745538in}}{\pgfqpoint{2.002730in}{0.741147in}}{\pgfqpoint{1.994917in}{0.733334in}}%
\pgfpathcurveto{\pgfqpoint{1.987103in}{0.725520in}}{\pgfqpoint{1.982713in}{0.714921in}}{\pgfqpoint{1.982713in}{0.703871in}}%
\pgfpathcurveto{\pgfqpoint{1.982713in}{0.692821in}}{\pgfqpoint{1.987103in}{0.682222in}}{\pgfqpoint{1.994917in}{0.674408in}}%
\pgfpathcurveto{\pgfqpoint{2.002730in}{0.666595in}}{\pgfqpoint{2.013329in}{0.662204in}}{\pgfqpoint{2.024380in}{0.662204in}}%
\pgfpathclose%
\pgfusepath{stroke,fill}%
\end{pgfscope}%
\begin{pgfscope}%
\pgfpathrectangle{\pgfqpoint{0.772069in}{0.515123in}}{\pgfqpoint{1.937500in}{1.347500in}}%
\pgfusepath{clip}%
\pgfsetbuttcap%
\pgfsetroundjoin%
\definecolor{currentfill}{rgb}{1.000000,0.388235,0.278431}%
\pgfsetfillcolor{currentfill}%
\pgfsetlinewidth{1.003750pt}%
\definecolor{currentstroke}{rgb}{1.000000,0.388235,0.278431}%
\pgfsetstrokecolor{currentstroke}%
\pgfsetdash{}{0pt}%
\pgfpathmoveto{\pgfqpoint{2.221639in}{0.677690in}}%
\pgfpathcurveto{\pgfqpoint{2.232690in}{0.677690in}}{\pgfqpoint{2.243289in}{0.682080in}}{\pgfqpoint{2.251102in}{0.689894in}}%
\pgfpathcurveto{\pgfqpoint{2.258916in}{0.697708in}}{\pgfqpoint{2.263306in}{0.708307in}}{\pgfqpoint{2.263306in}{0.719357in}}%
\pgfpathcurveto{\pgfqpoint{2.263306in}{0.730407in}}{\pgfqpoint{2.258916in}{0.741006in}}{\pgfqpoint{2.251102in}{0.748820in}}%
\pgfpathcurveto{\pgfqpoint{2.243289in}{0.756633in}}{\pgfqpoint{2.232690in}{0.761023in}}{\pgfqpoint{2.221639in}{0.761023in}}%
\pgfpathcurveto{\pgfqpoint{2.210589in}{0.761023in}}{\pgfqpoint{2.199990in}{0.756633in}}{\pgfqpoint{2.192177in}{0.748820in}}%
\pgfpathcurveto{\pgfqpoint{2.184363in}{0.741006in}}{\pgfqpoint{2.179973in}{0.730407in}}{\pgfqpoint{2.179973in}{0.719357in}}%
\pgfpathcurveto{\pgfqpoint{2.179973in}{0.708307in}}{\pgfqpoint{2.184363in}{0.697708in}}{\pgfqpoint{2.192177in}{0.689894in}}%
\pgfpathcurveto{\pgfqpoint{2.199990in}{0.682080in}}{\pgfqpoint{2.210589in}{0.677690in}}{\pgfqpoint{2.221639in}{0.677690in}}%
\pgfpathclose%
\pgfusepath{stroke,fill}%
\end{pgfscope}%
\begin{pgfscope}%
\pgfpathrectangle{\pgfqpoint{0.772069in}{0.515123in}}{\pgfqpoint{1.937500in}{1.347500in}}%
\pgfusepath{clip}%
\pgfsetbuttcap%
\pgfsetroundjoin%
\definecolor{currentfill}{rgb}{1.000000,0.388235,0.278431}%
\pgfsetfillcolor{currentfill}%
\pgfsetlinewidth{1.003750pt}%
\definecolor{currentstroke}{rgb}{1.000000,0.388235,0.278431}%
\pgfsetstrokecolor{currentstroke}%
\pgfsetdash{}{0pt}%
\pgfpathmoveto{\pgfqpoint{2.394242in}{0.693176in}}%
\pgfpathcurveto{\pgfqpoint{2.405292in}{0.693176in}}{\pgfqpoint{2.415891in}{0.697566in}}{\pgfqpoint{2.423704in}{0.705380in}}%
\pgfpathcurveto{\pgfqpoint{2.431518in}{0.713194in}}{\pgfqpoint{2.435908in}{0.723793in}}{\pgfqpoint{2.435908in}{0.734843in}}%
\pgfpathcurveto{\pgfqpoint{2.435908in}{0.745893in}}{\pgfqpoint{2.431518in}{0.756492in}}{\pgfqpoint{2.423704in}{0.764305in}}%
\pgfpathcurveto{\pgfqpoint{2.415891in}{0.772119in}}{\pgfqpoint{2.405292in}{0.776509in}}{\pgfqpoint{2.394242in}{0.776509in}}%
\pgfpathcurveto{\pgfqpoint{2.383192in}{0.776509in}}{\pgfqpoint{2.372593in}{0.772119in}}{\pgfqpoint{2.364779in}{0.764305in}}%
\pgfpathcurveto{\pgfqpoint{2.356965in}{0.756492in}}{\pgfqpoint{2.352575in}{0.745893in}}{\pgfqpoint{2.352575in}{0.734843in}}%
\pgfpathcurveto{\pgfqpoint{2.352575in}{0.723793in}}{\pgfqpoint{2.356965in}{0.713194in}}{\pgfqpoint{2.364779in}{0.705380in}}%
\pgfpathcurveto{\pgfqpoint{2.372593in}{0.697566in}}{\pgfqpoint{2.383192in}{0.693176in}}{\pgfqpoint{2.394242in}{0.693176in}}%
\pgfpathclose%
\pgfusepath{stroke,fill}%
\end{pgfscope}%
\begin{pgfscope}%
\pgfpathrectangle{\pgfqpoint{0.772069in}{0.515123in}}{\pgfqpoint{1.937500in}{1.347500in}}%
\pgfusepath{clip}%
\pgfsetbuttcap%
\pgfsetroundjoin%
\definecolor{currentfill}{rgb}{1.000000,0.388235,0.278431}%
\pgfsetfillcolor{currentfill}%
\pgfsetlinewidth{1.003750pt}%
\definecolor{currentstroke}{rgb}{1.000000,0.388235,0.278431}%
\pgfsetstrokecolor{currentstroke}%
\pgfsetdash{}{0pt}%
\pgfpathmoveto{\pgfqpoint{2.591501in}{0.709138in}}%
\pgfpathcurveto{\pgfqpoint{2.602552in}{0.709138in}}{\pgfqpoint{2.613151in}{0.713529in}}{\pgfqpoint{2.620964in}{0.721342in}}%
\pgfpathcurveto{\pgfqpoint{2.628778in}{0.729156in}}{\pgfqpoint{2.633168in}{0.739755in}}{\pgfqpoint{2.633168in}{0.750805in}}%
\pgfpathcurveto{\pgfqpoint{2.633168in}{0.761855in}}{\pgfqpoint{2.628778in}{0.772454in}}{\pgfqpoint{2.620964in}{0.780268in}}%
\pgfpathcurveto{\pgfqpoint{2.613151in}{0.788081in}}{\pgfqpoint{2.602552in}{0.792472in}}{\pgfqpoint{2.591501in}{0.792472in}}%
\pgfpathcurveto{\pgfqpoint{2.580451in}{0.792472in}}{\pgfqpoint{2.569852in}{0.788081in}}{\pgfqpoint{2.562039in}{0.780268in}}%
\pgfpathcurveto{\pgfqpoint{2.554225in}{0.772454in}}{\pgfqpoint{2.549835in}{0.761855in}}{\pgfqpoint{2.549835in}{0.750805in}}%
\pgfpathcurveto{\pgfqpoint{2.549835in}{0.739755in}}{\pgfqpoint{2.554225in}{0.729156in}}{\pgfqpoint{2.562039in}{0.721342in}}%
\pgfpathcurveto{\pgfqpoint{2.569852in}{0.713529in}}{\pgfqpoint{2.580451in}{0.709138in}}{\pgfqpoint{2.591501in}{0.709138in}}%
\pgfpathclose%
\pgfusepath{stroke,fill}%
\end{pgfscope}%
\begin{pgfscope}%
\pgfpathrectangle{\pgfqpoint{0.772069in}{0.515123in}}{\pgfqpoint{1.937500in}{1.347500in}}%
\pgfusepath{clip}%
\pgfsetbuttcap%
\pgfsetroundjoin%
\definecolor{currentfill}{rgb}{1.000000,0.843137,0.000000}%
\pgfsetfillcolor{currentfill}%
\pgfsetlinewidth{1.003750pt}%
\definecolor{currentstroke}{rgb}{1.000000,0.843137,0.000000}%
\pgfsetstrokecolor{currentstroke}%
\pgfsetdash{}{0pt}%
\pgfpathmoveto{\pgfqpoint{0.890136in}{0.564548in}}%
\pgfpathcurveto{\pgfqpoint{0.901186in}{0.564548in}}{\pgfqpoint{0.911785in}{0.568938in}}{\pgfqpoint{0.919599in}{0.576752in}}%
\pgfpathcurveto{\pgfqpoint{0.927412in}{0.584566in}}{\pgfqpoint{0.931803in}{0.595165in}}{\pgfqpoint{0.931803in}{0.606215in}}%
\pgfpathcurveto{\pgfqpoint{0.931803in}{0.617265in}}{\pgfqpoint{0.927412in}{0.627864in}}{\pgfqpoint{0.919599in}{0.635678in}}%
\pgfpathcurveto{\pgfqpoint{0.911785in}{0.643491in}}{\pgfqpoint{0.901186in}{0.647881in}}{\pgfqpoint{0.890136in}{0.647881in}}%
\pgfpathcurveto{\pgfqpoint{0.879086in}{0.647881in}}{\pgfqpoint{0.868487in}{0.643491in}}{\pgfqpoint{0.860673in}{0.635678in}}%
\pgfpathcurveto{\pgfqpoint{0.852859in}{0.627864in}}{\pgfqpoint{0.848469in}{0.617265in}}{\pgfqpoint{0.848469in}{0.606215in}}%
\pgfpathcurveto{\pgfqpoint{0.848469in}{0.595165in}}{\pgfqpoint{0.852859in}{0.584566in}}{\pgfqpoint{0.860673in}{0.576752in}}%
\pgfpathcurveto{\pgfqpoint{0.868487in}{0.568938in}}{\pgfqpoint{0.879086in}{0.564548in}}{\pgfqpoint{0.890136in}{0.564548in}}%
\pgfpathclose%
\pgfusepath{stroke,fill}%
\end{pgfscope}%
\begin{pgfscope}%
\pgfpathrectangle{\pgfqpoint{0.772069in}{0.515123in}}{\pgfqpoint{1.937500in}{1.347500in}}%
\pgfusepath{clip}%
\pgfsetbuttcap%
\pgfsetroundjoin%
\definecolor{currentfill}{rgb}{1.000000,0.843137,0.000000}%
\pgfsetfillcolor{currentfill}%
\pgfsetlinewidth{1.003750pt}%
\definecolor{currentstroke}{rgb}{1.000000,0.843137,0.000000}%
\pgfsetstrokecolor{currentstroke}%
\pgfsetdash{}{0pt}%
\pgfpathmoveto{\pgfqpoint{1.087396in}{0.576436in}}%
\pgfpathcurveto{\pgfqpoint{1.098446in}{0.576436in}}{\pgfqpoint{1.109045in}{0.580827in}}{\pgfqpoint{1.116858in}{0.588640in}}%
\pgfpathcurveto{\pgfqpoint{1.124672in}{0.596454in}}{\pgfqpoint{1.129062in}{0.607053in}}{\pgfqpoint{1.129062in}{0.618103in}}%
\pgfpathcurveto{\pgfqpoint{1.129062in}{0.629153in}}{\pgfqpoint{1.124672in}{0.639752in}}{\pgfqpoint{1.116858in}{0.647566in}}%
\pgfpathcurveto{\pgfqpoint{1.109045in}{0.655380in}}{\pgfqpoint{1.098446in}{0.659770in}}{\pgfqpoint{1.087396in}{0.659770in}}%
\pgfpathcurveto{\pgfqpoint{1.076346in}{0.659770in}}{\pgfqpoint{1.065746in}{0.655380in}}{\pgfqpoint{1.057933in}{0.647566in}}%
\pgfpathcurveto{\pgfqpoint{1.050119in}{0.639752in}}{\pgfqpoint{1.045729in}{0.629153in}}{\pgfqpoint{1.045729in}{0.618103in}}%
\pgfpathcurveto{\pgfqpoint{1.045729in}{0.607053in}}{\pgfqpoint{1.050119in}{0.596454in}}{\pgfqpoint{1.057933in}{0.588640in}}%
\pgfpathcurveto{\pgfqpoint{1.065746in}{0.580827in}}{\pgfqpoint{1.076346in}{0.576436in}}{\pgfqpoint{1.087396in}{0.576436in}}%
\pgfpathclose%
\pgfusepath{stroke,fill}%
\end{pgfscope}%
\begin{pgfscope}%
\pgfpathrectangle{\pgfqpoint{0.772069in}{0.515123in}}{\pgfqpoint{1.937500in}{1.347500in}}%
\pgfusepath{clip}%
\pgfsetbuttcap%
\pgfsetroundjoin%
\definecolor{currentfill}{rgb}{1.000000,0.843137,0.000000}%
\pgfsetfillcolor{currentfill}%
\pgfsetlinewidth{1.003750pt}%
\definecolor{currentstroke}{rgb}{1.000000,0.843137,0.000000}%
\pgfsetstrokecolor{currentstroke}%
\pgfsetdash{}{0pt}%
\pgfpathmoveto{\pgfqpoint{1.259998in}{0.589302in}}%
\pgfpathcurveto{\pgfqpoint{1.271048in}{0.589302in}}{\pgfqpoint{1.281647in}{0.593692in}}{\pgfqpoint{1.289461in}{0.601506in}}%
\pgfpathcurveto{\pgfqpoint{1.297274in}{0.609319in}}{\pgfqpoint{1.301665in}{0.619918in}}{\pgfqpoint{1.301665in}{0.630968in}}%
\pgfpathcurveto{\pgfqpoint{1.301665in}{0.642018in}}{\pgfqpoint{1.297274in}{0.652617in}}{\pgfqpoint{1.289461in}{0.660431in}}%
\pgfpathcurveto{\pgfqpoint{1.281647in}{0.668245in}}{\pgfqpoint{1.271048in}{0.672635in}}{\pgfqpoint{1.259998in}{0.672635in}}%
\pgfpathcurveto{\pgfqpoint{1.248948in}{0.672635in}}{\pgfqpoint{1.238349in}{0.668245in}}{\pgfqpoint{1.230535in}{0.660431in}}%
\pgfpathcurveto{\pgfqpoint{1.222722in}{0.652617in}}{\pgfqpoint{1.218331in}{0.642018in}}{\pgfqpoint{1.218331in}{0.630968in}}%
\pgfpathcurveto{\pgfqpoint{1.218331in}{0.619918in}}{\pgfqpoint{1.222722in}{0.609319in}}{\pgfqpoint{1.230535in}{0.601506in}}%
\pgfpathcurveto{\pgfqpoint{1.238349in}{0.593692in}}{\pgfqpoint{1.248948in}{0.589302in}}{\pgfqpoint{1.259998in}{0.589302in}}%
\pgfpathclose%
\pgfusepath{stroke,fill}%
\end{pgfscope}%
\begin{pgfscope}%
\pgfpathrectangle{\pgfqpoint{0.772069in}{0.515123in}}{\pgfqpoint{1.937500in}{1.347500in}}%
\pgfusepath{clip}%
\pgfsetbuttcap%
\pgfsetroundjoin%
\definecolor{currentfill}{rgb}{1.000000,0.843137,0.000000}%
\pgfsetfillcolor{currentfill}%
\pgfsetlinewidth{1.003750pt}%
\definecolor{currentstroke}{rgb}{1.000000,0.843137,0.000000}%
\pgfsetstrokecolor{currentstroke}%
\pgfsetdash{}{0pt}%
\pgfpathmoveto{\pgfqpoint{1.457258in}{0.601929in}}%
\pgfpathcurveto{\pgfqpoint{1.468308in}{0.601929in}}{\pgfqpoint{1.478907in}{0.606319in}}{\pgfqpoint{1.486721in}{0.614132in}}%
\pgfpathcurveto{\pgfqpoint{1.494534in}{0.621946in}}{\pgfqpoint{1.498924in}{0.632545in}}{\pgfqpoint{1.498924in}{0.643595in}}%
\pgfpathcurveto{\pgfqpoint{1.498924in}{0.654645in}}{\pgfqpoint{1.494534in}{0.665244in}}{\pgfqpoint{1.486721in}{0.673058in}}%
\pgfpathcurveto{\pgfqpoint{1.478907in}{0.680872in}}{\pgfqpoint{1.468308in}{0.685262in}}{\pgfqpoint{1.457258in}{0.685262in}}%
\pgfpathcurveto{\pgfqpoint{1.446208in}{0.685262in}}{\pgfqpoint{1.435609in}{0.680872in}}{\pgfqpoint{1.427795in}{0.673058in}}%
\pgfpathcurveto{\pgfqpoint{1.419981in}{0.665244in}}{\pgfqpoint{1.415591in}{0.654645in}}{\pgfqpoint{1.415591in}{0.643595in}}%
\pgfpathcurveto{\pgfqpoint{1.415591in}{0.632545in}}{\pgfqpoint{1.419981in}{0.621946in}}{\pgfqpoint{1.427795in}{0.614132in}}%
\pgfpathcurveto{\pgfqpoint{1.435609in}{0.606319in}}{\pgfqpoint{1.446208in}{0.601929in}}{\pgfqpoint{1.457258in}{0.601929in}}%
\pgfpathclose%
\pgfusepath{stroke,fill}%
\end{pgfscope}%
\begin{pgfscope}%
\pgfpathrectangle{\pgfqpoint{0.772069in}{0.515123in}}{\pgfqpoint{1.937500in}{1.347500in}}%
\pgfusepath{clip}%
\pgfsetbuttcap%
\pgfsetroundjoin%
\definecolor{currentfill}{rgb}{1.000000,0.843137,0.000000}%
\pgfsetfillcolor{currentfill}%
\pgfsetlinewidth{1.003750pt}%
\definecolor{currentstroke}{rgb}{1.000000,0.843137,0.000000}%
\pgfsetstrokecolor{currentstroke}%
\pgfsetdash{}{0pt}%
\pgfpathmoveto{\pgfqpoint{1.654518in}{0.614556in}}%
\pgfpathcurveto{\pgfqpoint{1.665568in}{0.614556in}}{\pgfqpoint{1.676167in}{0.618946in}}{\pgfqpoint{1.683980in}{0.626759in}}%
\pgfpathcurveto{\pgfqpoint{1.691794in}{0.634573in}}{\pgfqpoint{1.696184in}{0.645172in}}{\pgfqpoint{1.696184in}{0.656222in}}%
\pgfpathcurveto{\pgfqpoint{1.696184in}{0.667272in}}{\pgfqpoint{1.691794in}{0.677871in}}{\pgfqpoint{1.683980in}{0.685685in}}%
\pgfpathcurveto{\pgfqpoint{1.676167in}{0.693499in}}{\pgfqpoint{1.665568in}{0.697889in}}{\pgfqpoint{1.654518in}{0.697889in}}%
\pgfpathcurveto{\pgfqpoint{1.643467in}{0.697889in}}{\pgfqpoint{1.632868in}{0.693499in}}{\pgfqpoint{1.625055in}{0.685685in}}%
\pgfpathcurveto{\pgfqpoint{1.617241in}{0.677871in}}{\pgfqpoint{1.612851in}{0.667272in}}{\pgfqpoint{1.612851in}{0.656222in}}%
\pgfpathcurveto{\pgfqpoint{1.612851in}{0.645172in}}{\pgfqpoint{1.617241in}{0.634573in}}{\pgfqpoint{1.625055in}{0.626759in}}%
\pgfpathcurveto{\pgfqpoint{1.632868in}{0.618946in}}{\pgfqpoint{1.643467in}{0.614556in}}{\pgfqpoint{1.654518in}{0.614556in}}%
\pgfpathclose%
\pgfusepath{stroke,fill}%
\end{pgfscope}%
\begin{pgfscope}%
\pgfpathrectangle{\pgfqpoint{0.772069in}{0.515123in}}{\pgfqpoint{1.937500in}{1.347500in}}%
\pgfusepath{clip}%
\pgfsetbuttcap%
\pgfsetroundjoin%
\definecolor{currentfill}{rgb}{1.000000,0.843137,0.000000}%
\pgfsetfillcolor{currentfill}%
\pgfsetlinewidth{1.003750pt}%
\definecolor{currentstroke}{rgb}{1.000000,0.843137,0.000000}%
\pgfsetstrokecolor{currentstroke}%
\pgfsetdash{}{0pt}%
\pgfpathmoveto{\pgfqpoint{1.827120in}{0.626587in}}%
\pgfpathcurveto{\pgfqpoint{1.838170in}{0.626587in}}{\pgfqpoint{1.848769in}{0.630977in}}{\pgfqpoint{1.856583in}{0.638791in}}%
\pgfpathcurveto{\pgfqpoint{1.864396in}{0.646604in}}{\pgfqpoint{1.868787in}{0.657203in}}{\pgfqpoint{1.868787in}{0.668253in}}%
\pgfpathcurveto{\pgfqpoint{1.868787in}{0.679304in}}{\pgfqpoint{1.864396in}{0.689903in}}{\pgfqpoint{1.856583in}{0.697716in}}%
\pgfpathcurveto{\pgfqpoint{1.848769in}{0.705530in}}{\pgfqpoint{1.838170in}{0.709920in}}{\pgfqpoint{1.827120in}{0.709920in}}%
\pgfpathcurveto{\pgfqpoint{1.816070in}{0.709920in}}{\pgfqpoint{1.805471in}{0.705530in}}{\pgfqpoint{1.797657in}{0.697716in}}%
\pgfpathcurveto{\pgfqpoint{1.789843in}{0.689903in}}{\pgfqpoint{1.785453in}{0.679304in}}{\pgfqpoint{1.785453in}{0.668253in}}%
\pgfpathcurveto{\pgfqpoint{1.785453in}{0.657203in}}{\pgfqpoint{1.789843in}{0.646604in}}{\pgfqpoint{1.797657in}{0.638791in}}%
\pgfpathcurveto{\pgfqpoint{1.805471in}{0.630977in}}{\pgfqpoint{1.816070in}{0.626587in}}{\pgfqpoint{1.827120in}{0.626587in}}%
\pgfpathclose%
\pgfusepath{stroke,fill}%
\end{pgfscope}%
\begin{pgfscope}%
\pgfpathrectangle{\pgfqpoint{0.772069in}{0.515123in}}{\pgfqpoint{1.937500in}{1.347500in}}%
\pgfusepath{clip}%
\pgfsetbuttcap%
\pgfsetroundjoin%
\definecolor{currentfill}{rgb}{1.000000,0.843137,0.000000}%
\pgfsetfillcolor{currentfill}%
\pgfsetlinewidth{1.003750pt}%
\definecolor{currentstroke}{rgb}{1.000000,0.843137,0.000000}%
\pgfsetstrokecolor{currentstroke}%
\pgfsetdash{}{0pt}%
\pgfpathmoveto{\pgfqpoint{2.024380in}{0.646004in}}%
\pgfpathcurveto{\pgfqpoint{2.035430in}{0.646004in}}{\pgfqpoint{2.046029in}{0.650394in}}{\pgfqpoint{2.053842in}{0.658208in}}%
\pgfpathcurveto{\pgfqpoint{2.061656in}{0.666021in}}{\pgfqpoint{2.066046in}{0.676620in}}{\pgfqpoint{2.066046in}{0.687670in}}%
\pgfpathcurveto{\pgfqpoint{2.066046in}{0.698721in}}{\pgfqpoint{2.061656in}{0.709320in}}{\pgfqpoint{2.053842in}{0.717133in}}%
\pgfpathcurveto{\pgfqpoint{2.046029in}{0.724947in}}{\pgfqpoint{2.035430in}{0.729337in}}{\pgfqpoint{2.024380in}{0.729337in}}%
\pgfpathcurveto{\pgfqpoint{2.013329in}{0.729337in}}{\pgfqpoint{2.002730in}{0.724947in}}{\pgfqpoint{1.994917in}{0.717133in}}%
\pgfpathcurveto{\pgfqpoint{1.987103in}{0.709320in}}{\pgfqpoint{1.982713in}{0.698721in}}{\pgfqpoint{1.982713in}{0.687670in}}%
\pgfpathcurveto{\pgfqpoint{1.982713in}{0.676620in}}{\pgfqpoint{1.987103in}{0.666021in}}{\pgfqpoint{1.994917in}{0.658208in}}%
\pgfpathcurveto{\pgfqpoint{2.002730in}{0.650394in}}{\pgfqpoint{2.013329in}{0.646004in}}{\pgfqpoint{2.024380in}{0.646004in}}%
\pgfpathclose%
\pgfusepath{stroke,fill}%
\end{pgfscope}%
\begin{pgfscope}%
\pgfpathrectangle{\pgfqpoint{0.772069in}{0.515123in}}{\pgfqpoint{1.937500in}{1.347500in}}%
\pgfusepath{clip}%
\pgfsetbuttcap%
\pgfsetroundjoin%
\definecolor{currentfill}{rgb}{1.000000,0.843137,0.000000}%
\pgfsetfillcolor{currentfill}%
\pgfsetlinewidth{1.003750pt}%
\definecolor{currentstroke}{rgb}{1.000000,0.843137,0.000000}%
\pgfsetstrokecolor{currentstroke}%
\pgfsetdash{}{0pt}%
\pgfpathmoveto{\pgfqpoint{2.221639in}{0.652198in}}%
\pgfpathcurveto{\pgfqpoint{2.232690in}{0.652198in}}{\pgfqpoint{2.243289in}{0.656588in}}{\pgfqpoint{2.251102in}{0.664402in}}%
\pgfpathcurveto{\pgfqpoint{2.258916in}{0.672216in}}{\pgfqpoint{2.263306in}{0.682815in}}{\pgfqpoint{2.263306in}{0.693865in}}%
\pgfpathcurveto{\pgfqpoint{2.263306in}{0.704915in}}{\pgfqpoint{2.258916in}{0.715514in}}{\pgfqpoint{2.251102in}{0.723328in}}%
\pgfpathcurveto{\pgfqpoint{2.243289in}{0.731141in}}{\pgfqpoint{2.232690in}{0.735531in}}{\pgfqpoint{2.221639in}{0.735531in}}%
\pgfpathcurveto{\pgfqpoint{2.210589in}{0.735531in}}{\pgfqpoint{2.199990in}{0.731141in}}{\pgfqpoint{2.192177in}{0.723328in}}%
\pgfpathcurveto{\pgfqpoint{2.184363in}{0.715514in}}{\pgfqpoint{2.179973in}{0.704915in}}{\pgfqpoint{2.179973in}{0.693865in}}%
\pgfpathcurveto{\pgfqpoint{2.179973in}{0.682815in}}{\pgfqpoint{2.184363in}{0.672216in}}{\pgfqpoint{2.192177in}{0.664402in}}%
\pgfpathcurveto{\pgfqpoint{2.199990in}{0.656588in}}{\pgfqpoint{2.210589in}{0.652198in}}{\pgfqpoint{2.221639in}{0.652198in}}%
\pgfpathclose%
\pgfusepath{stroke,fill}%
\end{pgfscope}%
\begin{pgfscope}%
\pgfpathrectangle{\pgfqpoint{0.772069in}{0.515123in}}{\pgfqpoint{1.937500in}{1.347500in}}%
\pgfusepath{clip}%
\pgfsetbuttcap%
\pgfsetroundjoin%
\definecolor{currentfill}{rgb}{1.000000,0.843137,0.000000}%
\pgfsetfillcolor{currentfill}%
\pgfsetlinewidth{1.003750pt}%
\definecolor{currentstroke}{rgb}{1.000000,0.843137,0.000000}%
\pgfsetstrokecolor{currentstroke}%
\pgfsetdash{}{0pt}%
\pgfpathmoveto{\pgfqpoint{2.394242in}{0.664468in}}%
\pgfpathcurveto{\pgfqpoint{2.405292in}{0.664468in}}{\pgfqpoint{2.415891in}{0.668858in}}{\pgfqpoint{2.423704in}{0.676672in}}%
\pgfpathcurveto{\pgfqpoint{2.431518in}{0.684485in}}{\pgfqpoint{2.435908in}{0.695084in}}{\pgfqpoint{2.435908in}{0.706134in}}%
\pgfpathcurveto{\pgfqpoint{2.435908in}{0.717184in}}{\pgfqpoint{2.431518in}{0.727783in}}{\pgfqpoint{2.423704in}{0.735597in}}%
\pgfpathcurveto{\pgfqpoint{2.415891in}{0.743411in}}{\pgfqpoint{2.405292in}{0.747801in}}{\pgfqpoint{2.394242in}{0.747801in}}%
\pgfpathcurveto{\pgfqpoint{2.383192in}{0.747801in}}{\pgfqpoint{2.372593in}{0.743411in}}{\pgfqpoint{2.364779in}{0.735597in}}%
\pgfpathcurveto{\pgfqpoint{2.356965in}{0.727783in}}{\pgfqpoint{2.352575in}{0.717184in}}{\pgfqpoint{2.352575in}{0.706134in}}%
\pgfpathcurveto{\pgfqpoint{2.352575in}{0.695084in}}{\pgfqpoint{2.356965in}{0.684485in}}{\pgfqpoint{2.364779in}{0.676672in}}%
\pgfpathcurveto{\pgfqpoint{2.372593in}{0.668858in}}{\pgfqpoint{2.383192in}{0.664468in}}{\pgfqpoint{2.394242in}{0.664468in}}%
\pgfpathclose%
\pgfusepath{stroke,fill}%
\end{pgfscope}%
\begin{pgfscope}%
\pgfpathrectangle{\pgfqpoint{0.772069in}{0.515123in}}{\pgfqpoint{1.937500in}{1.347500in}}%
\pgfusepath{clip}%
\pgfsetbuttcap%
\pgfsetroundjoin%
\definecolor{currentfill}{rgb}{1.000000,0.843137,0.000000}%
\pgfsetfillcolor{currentfill}%
\pgfsetlinewidth{1.003750pt}%
\definecolor{currentstroke}{rgb}{1.000000,0.843137,0.000000}%
\pgfsetstrokecolor{currentstroke}%
\pgfsetdash{}{0pt}%
\pgfpathmoveto{\pgfqpoint{2.591501in}{0.677095in}}%
\pgfpathcurveto{\pgfqpoint{2.602552in}{0.677095in}}{\pgfqpoint{2.613151in}{0.681485in}}{\pgfqpoint{2.620964in}{0.689298in}}%
\pgfpathcurveto{\pgfqpoint{2.628778in}{0.697112in}}{\pgfqpoint{2.633168in}{0.707711in}}{\pgfqpoint{2.633168in}{0.718761in}}%
\pgfpathcurveto{\pgfqpoint{2.633168in}{0.729811in}}{\pgfqpoint{2.628778in}{0.740410in}}{\pgfqpoint{2.620964in}{0.748224in}}%
\pgfpathcurveto{\pgfqpoint{2.613151in}{0.756038in}}{\pgfqpoint{2.602552in}{0.760428in}}{\pgfqpoint{2.591501in}{0.760428in}}%
\pgfpathcurveto{\pgfqpoint{2.580451in}{0.760428in}}{\pgfqpoint{2.569852in}{0.756038in}}{\pgfqpoint{2.562039in}{0.748224in}}%
\pgfpathcurveto{\pgfqpoint{2.554225in}{0.740410in}}{\pgfqpoint{2.549835in}{0.729811in}}{\pgfqpoint{2.549835in}{0.718761in}}%
\pgfpathcurveto{\pgfqpoint{2.549835in}{0.707711in}}{\pgfqpoint{2.554225in}{0.697112in}}{\pgfqpoint{2.562039in}{0.689298in}}%
\pgfpathcurveto{\pgfqpoint{2.569852in}{0.681485in}}{\pgfqpoint{2.580451in}{0.677095in}}{\pgfqpoint{2.591501in}{0.677095in}}%
\pgfpathclose%
\pgfusepath{stroke,fill}%
\end{pgfscope}%
\begin{pgfscope}%
\pgfpathrectangle{\pgfqpoint{0.772069in}{0.515123in}}{\pgfqpoint{1.937500in}{1.347500in}}%
\pgfusepath{clip}%
\pgfsetbuttcap%
\pgfsetroundjoin%
\definecolor{currentfill}{rgb}{0.196078,0.803922,0.196078}%
\pgfsetfillcolor{currentfill}%
\pgfsetlinewidth{1.003750pt}%
\definecolor{currentstroke}{rgb}{0.196078,0.803922,0.196078}%
\pgfsetstrokecolor{currentstroke}%
\pgfsetdash{}{0pt}%
\pgfpathmoveto{\pgfqpoint{0.890136in}{0.643145in}}%
\pgfpathcurveto{\pgfqpoint{0.901186in}{0.643145in}}{\pgfqpoint{0.911785in}{0.647535in}}{\pgfqpoint{0.919599in}{0.655349in}}%
\pgfpathcurveto{\pgfqpoint{0.927412in}{0.663162in}}{\pgfqpoint{0.931803in}{0.673761in}}{\pgfqpoint{0.931803in}{0.684811in}}%
\pgfpathcurveto{\pgfqpoint{0.931803in}{0.695862in}}{\pgfqpoint{0.927412in}{0.706461in}}{\pgfqpoint{0.919599in}{0.714274in}}%
\pgfpathcurveto{\pgfqpoint{0.911785in}{0.722088in}}{\pgfqpoint{0.901186in}{0.726478in}}{\pgfqpoint{0.890136in}{0.726478in}}%
\pgfpathcurveto{\pgfqpoint{0.879086in}{0.726478in}}{\pgfqpoint{0.868487in}{0.722088in}}{\pgfqpoint{0.860673in}{0.714274in}}%
\pgfpathcurveto{\pgfqpoint{0.852859in}{0.706461in}}{\pgfqpoint{0.848469in}{0.695862in}}{\pgfqpoint{0.848469in}{0.684811in}}%
\pgfpathcurveto{\pgfqpoint{0.848469in}{0.673761in}}{\pgfqpoint{0.852859in}{0.663162in}}{\pgfqpoint{0.860673in}{0.655349in}}%
\pgfpathcurveto{\pgfqpoint{0.868487in}{0.647535in}}{\pgfqpoint{0.879086in}{0.643145in}}{\pgfqpoint{0.890136in}{0.643145in}}%
\pgfpathclose%
\pgfusepath{stroke,fill}%
\end{pgfscope}%
\begin{pgfscope}%
\pgfpathrectangle{\pgfqpoint{0.772069in}{0.515123in}}{\pgfqpoint{1.937500in}{1.347500in}}%
\pgfusepath{clip}%
\pgfsetbuttcap%
\pgfsetroundjoin%
\definecolor{currentfill}{rgb}{0.196078,0.803922,0.196078}%
\pgfsetfillcolor{currentfill}%
\pgfsetlinewidth{1.003750pt}%
\definecolor{currentstroke}{rgb}{0.196078,0.803922,0.196078}%
\pgfsetstrokecolor{currentstroke}%
\pgfsetdash{}{0pt}%
\pgfpathmoveto{\pgfqpoint{1.087396in}{0.729270in}}%
\pgfpathcurveto{\pgfqpoint{1.098446in}{0.729270in}}{\pgfqpoint{1.109045in}{0.733660in}}{\pgfqpoint{1.116858in}{0.741474in}}%
\pgfpathcurveto{\pgfqpoint{1.124672in}{0.749287in}}{\pgfqpoint{1.129062in}{0.759887in}}{\pgfqpoint{1.129062in}{0.770937in}}%
\pgfpathcurveto{\pgfqpoint{1.129062in}{0.781987in}}{\pgfqpoint{1.124672in}{0.792586in}}{\pgfqpoint{1.116858in}{0.800399in}}%
\pgfpathcurveto{\pgfqpoint{1.109045in}{0.808213in}}{\pgfqpoint{1.098446in}{0.812603in}}{\pgfqpoint{1.087396in}{0.812603in}}%
\pgfpathcurveto{\pgfqpoint{1.076346in}{0.812603in}}{\pgfqpoint{1.065746in}{0.808213in}}{\pgfqpoint{1.057933in}{0.800399in}}%
\pgfpathcurveto{\pgfqpoint{1.050119in}{0.792586in}}{\pgfqpoint{1.045729in}{0.781987in}}{\pgfqpoint{1.045729in}{0.770937in}}%
\pgfpathcurveto{\pgfqpoint{1.045729in}{0.759887in}}{\pgfqpoint{1.050119in}{0.749287in}}{\pgfqpoint{1.057933in}{0.741474in}}%
\pgfpathcurveto{\pgfqpoint{1.065746in}{0.733660in}}{\pgfqpoint{1.076346in}{0.729270in}}{\pgfqpoint{1.087396in}{0.729270in}}%
\pgfpathclose%
\pgfusepath{stroke,fill}%
\end{pgfscope}%
\begin{pgfscope}%
\pgfpathrectangle{\pgfqpoint{0.772069in}{0.515123in}}{\pgfqpoint{1.937500in}{1.347500in}}%
\pgfusepath{clip}%
\pgfsetbuttcap%
\pgfsetroundjoin%
\definecolor{currentfill}{rgb}{0.196078,0.803922,0.196078}%
\pgfsetfillcolor{currentfill}%
\pgfsetlinewidth{1.003750pt}%
\definecolor{currentstroke}{rgb}{0.196078,0.803922,0.196078}%
\pgfsetstrokecolor{currentstroke}%
\pgfsetdash{}{0pt}%
\pgfpathmoveto{\pgfqpoint{1.259998in}{0.819564in}}%
\pgfpathcurveto{\pgfqpoint{1.271048in}{0.819564in}}{\pgfqpoint{1.281647in}{0.823955in}}{\pgfqpoint{1.289461in}{0.831768in}}%
\pgfpathcurveto{\pgfqpoint{1.297274in}{0.839582in}}{\pgfqpoint{1.301665in}{0.850181in}}{\pgfqpoint{1.301665in}{0.861231in}}%
\pgfpathcurveto{\pgfqpoint{1.301665in}{0.872281in}}{\pgfqpoint{1.297274in}{0.882880in}}{\pgfqpoint{1.289461in}{0.890694in}}%
\pgfpathcurveto{\pgfqpoint{1.281647in}{0.898508in}}{\pgfqpoint{1.271048in}{0.902898in}}{\pgfqpoint{1.259998in}{0.902898in}}%
\pgfpathcurveto{\pgfqpoint{1.248948in}{0.902898in}}{\pgfqpoint{1.238349in}{0.898508in}}{\pgfqpoint{1.230535in}{0.890694in}}%
\pgfpathcurveto{\pgfqpoint{1.222722in}{0.882880in}}{\pgfqpoint{1.218331in}{0.872281in}}{\pgfqpoint{1.218331in}{0.861231in}}%
\pgfpathcurveto{\pgfqpoint{1.218331in}{0.850181in}}{\pgfqpoint{1.222722in}{0.839582in}}{\pgfqpoint{1.230535in}{0.831768in}}%
\pgfpathcurveto{\pgfqpoint{1.238349in}{0.823955in}}{\pgfqpoint{1.248948in}{0.819564in}}{\pgfqpoint{1.259998in}{0.819564in}}%
\pgfpathclose%
\pgfusepath{stroke,fill}%
\end{pgfscope}%
\begin{pgfscope}%
\pgfpathrectangle{\pgfqpoint{0.772069in}{0.515123in}}{\pgfqpoint{1.937500in}{1.347500in}}%
\pgfusepath{clip}%
\pgfsetbuttcap%
\pgfsetroundjoin%
\definecolor{currentfill}{rgb}{0.196078,0.803922,0.196078}%
\pgfsetfillcolor{currentfill}%
\pgfsetlinewidth{1.003750pt}%
\definecolor{currentstroke}{rgb}{0.196078,0.803922,0.196078}%
\pgfsetstrokecolor{currentstroke}%
\pgfsetdash{}{0pt}%
\pgfpathmoveto{\pgfqpoint{1.457258in}{0.911288in}}%
\pgfpathcurveto{\pgfqpoint{1.468308in}{0.911288in}}{\pgfqpoint{1.478907in}{0.915679in}}{\pgfqpoint{1.486721in}{0.923492in}}%
\pgfpathcurveto{\pgfqpoint{1.494534in}{0.931306in}}{\pgfqpoint{1.498924in}{0.941905in}}{\pgfqpoint{1.498924in}{0.952955in}}%
\pgfpathcurveto{\pgfqpoint{1.498924in}{0.964005in}}{\pgfqpoint{1.494534in}{0.974604in}}{\pgfqpoint{1.486721in}{0.982418in}}%
\pgfpathcurveto{\pgfqpoint{1.478907in}{0.990231in}}{\pgfqpoint{1.468308in}{0.994622in}}{\pgfqpoint{1.457258in}{0.994622in}}%
\pgfpathcurveto{\pgfqpoint{1.446208in}{0.994622in}}{\pgfqpoint{1.435609in}{0.990231in}}{\pgfqpoint{1.427795in}{0.982418in}}%
\pgfpathcurveto{\pgfqpoint{1.419981in}{0.974604in}}{\pgfqpoint{1.415591in}{0.964005in}}{\pgfqpoint{1.415591in}{0.952955in}}%
\pgfpathcurveto{\pgfqpoint{1.415591in}{0.941905in}}{\pgfqpoint{1.419981in}{0.931306in}}{\pgfqpoint{1.427795in}{0.923492in}}%
\pgfpathcurveto{\pgfqpoint{1.435609in}{0.915679in}}{\pgfqpoint{1.446208in}{0.911288in}}{\pgfqpoint{1.457258in}{0.911288in}}%
\pgfpathclose%
\pgfusepath{stroke,fill}%
\end{pgfscope}%
\begin{pgfscope}%
\pgfpathrectangle{\pgfqpoint{0.772069in}{0.515123in}}{\pgfqpoint{1.937500in}{1.347500in}}%
\pgfusepath{clip}%
\pgfsetbuttcap%
\pgfsetroundjoin%
\definecolor{currentfill}{rgb}{0.196078,0.803922,0.196078}%
\pgfsetfillcolor{currentfill}%
\pgfsetlinewidth{1.003750pt}%
\definecolor{currentstroke}{rgb}{0.196078,0.803922,0.196078}%
\pgfsetstrokecolor{currentstroke}%
\pgfsetdash{}{0pt}%
\pgfpathmoveto{\pgfqpoint{1.654518in}{1.001821in}}%
\pgfpathcurveto{\pgfqpoint{1.665568in}{1.001821in}}{\pgfqpoint{1.676167in}{1.006211in}}{\pgfqpoint{1.683980in}{1.014025in}}%
\pgfpathcurveto{\pgfqpoint{1.691794in}{1.021839in}}{\pgfqpoint{1.696184in}{1.032438in}}{\pgfqpoint{1.696184in}{1.043488in}}%
\pgfpathcurveto{\pgfqpoint{1.696184in}{1.054538in}}{\pgfqpoint{1.691794in}{1.065137in}}{\pgfqpoint{1.683980in}{1.072951in}}%
\pgfpathcurveto{\pgfqpoint{1.676167in}{1.080764in}}{\pgfqpoint{1.665568in}{1.085154in}}{\pgfqpoint{1.654518in}{1.085154in}}%
\pgfpathcurveto{\pgfqpoint{1.643467in}{1.085154in}}{\pgfqpoint{1.632868in}{1.080764in}}{\pgfqpoint{1.625055in}{1.072951in}}%
\pgfpathcurveto{\pgfqpoint{1.617241in}{1.065137in}}{\pgfqpoint{1.612851in}{1.054538in}}{\pgfqpoint{1.612851in}{1.043488in}}%
\pgfpathcurveto{\pgfqpoint{1.612851in}{1.032438in}}{\pgfqpoint{1.617241in}{1.021839in}}{\pgfqpoint{1.625055in}{1.014025in}}%
\pgfpathcurveto{\pgfqpoint{1.632868in}{1.006211in}}{\pgfqpoint{1.643467in}{1.001821in}}{\pgfqpoint{1.654518in}{1.001821in}}%
\pgfpathclose%
\pgfusepath{stroke,fill}%
\end{pgfscope}%
\begin{pgfscope}%
\pgfpathrectangle{\pgfqpoint{0.772069in}{0.515123in}}{\pgfqpoint{1.937500in}{1.347500in}}%
\pgfusepath{clip}%
\pgfsetbuttcap%
\pgfsetroundjoin%
\definecolor{currentfill}{rgb}{0.196078,0.803922,0.196078}%
\pgfsetfillcolor{currentfill}%
\pgfsetlinewidth{1.003750pt}%
\definecolor{currentstroke}{rgb}{0.196078,0.803922,0.196078}%
\pgfsetstrokecolor{currentstroke}%
\pgfsetdash{}{0pt}%
\pgfpathmoveto{\pgfqpoint{1.827120in}{1.087589in}}%
\pgfpathcurveto{\pgfqpoint{1.838170in}{1.087589in}}{\pgfqpoint{1.848769in}{1.091979in}}{\pgfqpoint{1.856583in}{1.099793in}}%
\pgfpathcurveto{\pgfqpoint{1.864396in}{1.107606in}}{\pgfqpoint{1.868787in}{1.118206in}}{\pgfqpoint{1.868787in}{1.129256in}}%
\pgfpathcurveto{\pgfqpoint{1.868787in}{1.140306in}}{\pgfqpoint{1.864396in}{1.150905in}}{\pgfqpoint{1.856583in}{1.158718in}}%
\pgfpathcurveto{\pgfqpoint{1.848769in}{1.166532in}}{\pgfqpoint{1.838170in}{1.170922in}}{\pgfqpoint{1.827120in}{1.170922in}}%
\pgfpathcurveto{\pgfqpoint{1.816070in}{1.170922in}}{\pgfqpoint{1.805471in}{1.166532in}}{\pgfqpoint{1.797657in}{1.158718in}}%
\pgfpathcurveto{\pgfqpoint{1.789843in}{1.150905in}}{\pgfqpoint{1.785453in}{1.140306in}}{\pgfqpoint{1.785453in}{1.129256in}}%
\pgfpathcurveto{\pgfqpoint{1.785453in}{1.118206in}}{\pgfqpoint{1.789843in}{1.107606in}}{\pgfqpoint{1.797657in}{1.099793in}}%
\pgfpathcurveto{\pgfqpoint{1.805471in}{1.091979in}}{\pgfqpoint{1.816070in}{1.087589in}}{\pgfqpoint{1.827120in}{1.087589in}}%
\pgfpathclose%
\pgfusepath{stroke,fill}%
\end{pgfscope}%
\begin{pgfscope}%
\pgfpathrectangle{\pgfqpoint{0.772069in}{0.515123in}}{\pgfqpoint{1.937500in}{1.347500in}}%
\pgfusepath{clip}%
\pgfsetbuttcap%
\pgfsetroundjoin%
\definecolor{currentfill}{rgb}{0.196078,0.803922,0.196078}%
\pgfsetfillcolor{currentfill}%
\pgfsetlinewidth{1.003750pt}%
\definecolor{currentstroke}{rgb}{0.196078,0.803922,0.196078}%
\pgfsetstrokecolor{currentstroke}%
\pgfsetdash{}{0pt}%
\pgfpathmoveto{\pgfqpoint{2.024380in}{1.180504in}}%
\pgfpathcurveto{\pgfqpoint{2.035430in}{1.180504in}}{\pgfqpoint{2.046029in}{1.184894in}}{\pgfqpoint{2.053842in}{1.192708in}}%
\pgfpathcurveto{\pgfqpoint{2.061656in}{1.200522in}}{\pgfqpoint{2.066046in}{1.211121in}}{\pgfqpoint{2.066046in}{1.222171in}}%
\pgfpathcurveto{\pgfqpoint{2.066046in}{1.233221in}}{\pgfqpoint{2.061656in}{1.243820in}}{\pgfqpoint{2.053842in}{1.251634in}}%
\pgfpathcurveto{\pgfqpoint{2.046029in}{1.259447in}}{\pgfqpoint{2.035430in}{1.263837in}}{\pgfqpoint{2.024380in}{1.263837in}}%
\pgfpathcurveto{\pgfqpoint{2.013329in}{1.263837in}}{\pgfqpoint{2.002730in}{1.259447in}}{\pgfqpoint{1.994917in}{1.251634in}}%
\pgfpathcurveto{\pgfqpoint{1.987103in}{1.243820in}}{\pgfqpoint{1.982713in}{1.233221in}}{\pgfqpoint{1.982713in}{1.222171in}}%
\pgfpathcurveto{\pgfqpoint{1.982713in}{1.211121in}}{\pgfqpoint{1.987103in}{1.200522in}}{\pgfqpoint{1.994917in}{1.192708in}}%
\pgfpathcurveto{\pgfqpoint{2.002730in}{1.184894in}}{\pgfqpoint{2.013329in}{1.180504in}}{\pgfqpoint{2.024380in}{1.180504in}}%
\pgfpathclose%
\pgfusepath{stroke,fill}%
\end{pgfscope}%
\begin{pgfscope}%
\pgfpathrectangle{\pgfqpoint{0.772069in}{0.515123in}}{\pgfqpoint{1.937500in}{1.347500in}}%
\pgfusepath{clip}%
\pgfsetbuttcap%
\pgfsetroundjoin%
\definecolor{currentfill}{rgb}{0.196078,0.803922,0.196078}%
\pgfsetfillcolor{currentfill}%
\pgfsetlinewidth{1.003750pt}%
\definecolor{currentstroke}{rgb}{0.196078,0.803922,0.196078}%
\pgfsetstrokecolor{currentstroke}%
\pgfsetdash{}{0pt}%
\pgfpathmoveto{\pgfqpoint{2.221639in}{1.269846in}}%
\pgfpathcurveto{\pgfqpoint{2.232690in}{1.269846in}}{\pgfqpoint{2.243289in}{1.274236in}}{\pgfqpoint{2.251102in}{1.282050in}}%
\pgfpathcurveto{\pgfqpoint{2.258916in}{1.289863in}}{\pgfqpoint{2.263306in}{1.300462in}}{\pgfqpoint{2.263306in}{1.311512in}}%
\pgfpathcurveto{\pgfqpoint{2.263306in}{1.322562in}}{\pgfqpoint{2.258916in}{1.333161in}}{\pgfqpoint{2.251102in}{1.340975in}}%
\pgfpathcurveto{\pgfqpoint{2.243289in}{1.348789in}}{\pgfqpoint{2.232690in}{1.353179in}}{\pgfqpoint{2.221639in}{1.353179in}}%
\pgfpathcurveto{\pgfqpoint{2.210589in}{1.353179in}}{\pgfqpoint{2.199990in}{1.348789in}}{\pgfqpoint{2.192177in}{1.340975in}}%
\pgfpathcurveto{\pgfqpoint{2.184363in}{1.333161in}}{\pgfqpoint{2.179973in}{1.322562in}}{\pgfqpoint{2.179973in}{1.311512in}}%
\pgfpathcurveto{\pgfqpoint{2.179973in}{1.300462in}}{\pgfqpoint{2.184363in}{1.289863in}}{\pgfqpoint{2.192177in}{1.282050in}}%
\pgfpathcurveto{\pgfqpoint{2.199990in}{1.274236in}}{\pgfqpoint{2.210589in}{1.269846in}}{\pgfqpoint{2.221639in}{1.269846in}}%
\pgfpathclose%
\pgfusepath{stroke,fill}%
\end{pgfscope}%
\begin{pgfscope}%
\pgfpathrectangle{\pgfqpoint{0.772069in}{0.515123in}}{\pgfqpoint{1.937500in}{1.347500in}}%
\pgfusepath{clip}%
\pgfsetbuttcap%
\pgfsetroundjoin%
\definecolor{currentfill}{rgb}{0.196078,0.803922,0.196078}%
\pgfsetfillcolor{currentfill}%
\pgfsetlinewidth{1.003750pt}%
\definecolor{currentstroke}{rgb}{0.196078,0.803922,0.196078}%
\pgfsetstrokecolor{currentstroke}%
\pgfsetdash{}{0pt}%
\pgfpathmoveto{\pgfqpoint{2.394242in}{1.357996in}}%
\pgfpathcurveto{\pgfqpoint{2.405292in}{1.357996in}}{\pgfqpoint{2.415891in}{1.362386in}}{\pgfqpoint{2.423704in}{1.370200in}}%
\pgfpathcurveto{\pgfqpoint{2.431518in}{1.378013in}}{\pgfqpoint{2.435908in}{1.388612in}}{\pgfqpoint{2.435908in}{1.399663in}}%
\pgfpathcurveto{\pgfqpoint{2.435908in}{1.410713in}}{\pgfqpoint{2.431518in}{1.421312in}}{\pgfqpoint{2.423704in}{1.429125in}}%
\pgfpathcurveto{\pgfqpoint{2.415891in}{1.436939in}}{\pgfqpoint{2.405292in}{1.441329in}}{\pgfqpoint{2.394242in}{1.441329in}}%
\pgfpathcurveto{\pgfqpoint{2.383192in}{1.441329in}}{\pgfqpoint{2.372593in}{1.436939in}}{\pgfqpoint{2.364779in}{1.429125in}}%
\pgfpathcurveto{\pgfqpoint{2.356965in}{1.421312in}}{\pgfqpoint{2.352575in}{1.410713in}}{\pgfqpoint{2.352575in}{1.399663in}}%
\pgfpathcurveto{\pgfqpoint{2.352575in}{1.388612in}}{\pgfqpoint{2.356965in}{1.378013in}}{\pgfqpoint{2.364779in}{1.370200in}}%
\pgfpathcurveto{\pgfqpoint{2.372593in}{1.362386in}}{\pgfqpoint{2.383192in}{1.357996in}}{\pgfqpoint{2.394242in}{1.357996in}}%
\pgfpathclose%
\pgfusepath{stroke,fill}%
\end{pgfscope}%
\begin{pgfscope}%
\pgfpathrectangle{\pgfqpoint{0.772069in}{0.515123in}}{\pgfqpoint{1.937500in}{1.347500in}}%
\pgfusepath{clip}%
\pgfsetbuttcap%
\pgfsetroundjoin%
\definecolor{currentfill}{rgb}{0.196078,0.803922,0.196078}%
\pgfsetfillcolor{currentfill}%
\pgfsetlinewidth{1.003750pt}%
\definecolor{currentstroke}{rgb}{0.196078,0.803922,0.196078}%
\pgfsetstrokecolor{currentstroke}%
\pgfsetdash{}{0pt}%
\pgfpathmoveto{\pgfqpoint{2.591501in}{1.448529in}}%
\pgfpathcurveto{\pgfqpoint{2.602552in}{1.448529in}}{\pgfqpoint{2.613151in}{1.452919in}}{\pgfqpoint{2.620964in}{1.460733in}}%
\pgfpathcurveto{\pgfqpoint{2.628778in}{1.468546in}}{\pgfqpoint{2.633168in}{1.479145in}}{\pgfqpoint{2.633168in}{1.490195in}}%
\pgfpathcurveto{\pgfqpoint{2.633168in}{1.501245in}}{\pgfqpoint{2.628778in}{1.511844in}}{\pgfqpoint{2.620964in}{1.519658in}}%
\pgfpathcurveto{\pgfqpoint{2.613151in}{1.527472in}}{\pgfqpoint{2.602552in}{1.531862in}}{\pgfqpoint{2.591501in}{1.531862in}}%
\pgfpathcurveto{\pgfqpoint{2.580451in}{1.531862in}}{\pgfqpoint{2.569852in}{1.527472in}}{\pgfqpoint{2.562039in}{1.519658in}}%
\pgfpathcurveto{\pgfqpoint{2.554225in}{1.511844in}}{\pgfqpoint{2.549835in}{1.501245in}}{\pgfqpoint{2.549835in}{1.490195in}}%
\pgfpathcurveto{\pgfqpoint{2.549835in}{1.479145in}}{\pgfqpoint{2.554225in}{1.468546in}}{\pgfqpoint{2.562039in}{1.460733in}}%
\pgfpathcurveto{\pgfqpoint{2.569852in}{1.452919in}}{\pgfqpoint{2.580451in}{1.448529in}}{\pgfqpoint{2.591501in}{1.448529in}}%
\pgfpathclose%
\pgfusepath{stroke,fill}%
\end{pgfscope}%
\begin{pgfscope}%
\pgfsetrectcap%
\pgfsetmiterjoin%
\pgfsetlinewidth{0.803000pt}%
\definecolor{currentstroke}{rgb}{0.000000,0.000000,0.000000}%
\pgfsetstrokecolor{currentstroke}%
\pgfsetdash{}{0pt}%
\pgfpathmoveto{\pgfqpoint{0.772069in}{0.515123in}}%
\pgfpathlineto{\pgfqpoint{0.772069in}{1.862623in}}%
\pgfusepath{stroke}%
\end{pgfscope}%
\begin{pgfscope}%
\pgfsetrectcap%
\pgfsetmiterjoin%
\pgfsetlinewidth{0.803000pt}%
\definecolor{currentstroke}{rgb}{0.000000,0.000000,0.000000}%
\pgfsetstrokecolor{currentstroke}%
\pgfsetdash{}{0pt}%
\pgfpathmoveto{\pgfqpoint{2.709569in}{0.515123in}}%
\pgfpathlineto{\pgfqpoint{2.709569in}{1.862623in}}%
\pgfusepath{stroke}%
\end{pgfscope}%
\begin{pgfscope}%
\pgfsetrectcap%
\pgfsetmiterjoin%
\pgfsetlinewidth{0.803000pt}%
\definecolor{currentstroke}{rgb}{0.000000,0.000000,0.000000}%
\pgfsetstrokecolor{currentstroke}%
\pgfsetdash{}{0pt}%
\pgfpathmoveto{\pgfqpoint{0.772069in}{0.515123in}}%
\pgfpathlineto{\pgfqpoint{2.709569in}{0.515123in}}%
\pgfusepath{stroke}%
\end{pgfscope}%
\begin{pgfscope}%
\pgfsetrectcap%
\pgfsetmiterjoin%
\pgfsetlinewidth{0.803000pt}%
\definecolor{currentstroke}{rgb}{0.000000,0.000000,0.000000}%
\pgfsetstrokecolor{currentstroke}%
\pgfsetdash{}{0pt}%
\pgfpathmoveto{\pgfqpoint{0.772069in}{1.862623in}}%
\pgfpathlineto{\pgfqpoint{2.709569in}{1.862623in}}%
\pgfusepath{stroke}%
\end{pgfscope}%
\end{pgfpicture}%
\makeatother%
\endgroup%

    %% Creator: Matplotlib, PGF backend
%%
%% To include the figure in your LaTeX document, write
%%   \input{<filename>.pgf}
%%
%% Make sure the required packages are loaded in your preamble
%%   \usepackage{pgf}
%%
%% Figures using additional raster images can only be included by \input if
%% they are in the same directory as the main LaTeX file. For loading figures
%% from other directories you can use the `import` package
%%   \usepackage{import}
%% and then include the figures with
%%   \import{<path to file>}{<filename>.pgf}
%%
%% Matplotlib used the following preamble
%%
\begingroup%
\makeatletter%
\begin{pgfpicture}%
\pgfpathrectangle{\pgfpointorigin}{\pgfqpoint{2.809569in}{1.962623in}}%
\pgfusepath{use as bounding box, clip}%
\begin{pgfscope}%
\pgfsetbuttcap%
\pgfsetmiterjoin%
\definecolor{currentfill}{rgb}{1.000000,1.000000,1.000000}%
\pgfsetfillcolor{currentfill}%
\pgfsetlinewidth{0.000000pt}%
\definecolor{currentstroke}{rgb}{1.000000,1.000000,1.000000}%
\pgfsetstrokecolor{currentstroke}%
\pgfsetdash{}{0pt}%
\pgfpathmoveto{\pgfqpoint{0.000000in}{0.000000in}}%
\pgfpathlineto{\pgfqpoint{2.809569in}{0.000000in}}%
\pgfpathlineto{\pgfqpoint{2.809569in}{1.962623in}}%
\pgfpathlineto{\pgfqpoint{0.000000in}{1.962623in}}%
\pgfpathclose%
\pgfusepath{fill}%
\end{pgfscope}%
\begin{pgfscope}%
\pgfsetbuttcap%
\pgfsetmiterjoin%
\definecolor{currentfill}{rgb}{1.000000,1.000000,1.000000}%
\pgfsetfillcolor{currentfill}%
\pgfsetlinewidth{0.000000pt}%
\definecolor{currentstroke}{rgb}{0.000000,0.000000,0.000000}%
\pgfsetstrokecolor{currentstroke}%
\pgfsetstrokeopacity{0.000000}%
\pgfsetdash{}{0pt}%
\pgfpathmoveto{\pgfqpoint{0.772069in}{0.515123in}}%
\pgfpathlineto{\pgfqpoint{2.709569in}{0.515123in}}%
\pgfpathlineto{\pgfqpoint{2.709569in}{1.862623in}}%
\pgfpathlineto{\pgfqpoint{0.772069in}{1.862623in}}%
\pgfpathclose%
\pgfusepath{fill}%
\end{pgfscope}%
\begin{pgfscope}%
\pgfsetbuttcap%
\pgfsetroundjoin%
\definecolor{currentfill}{rgb}{0.000000,0.000000,0.000000}%
\pgfsetfillcolor{currentfill}%
\pgfsetlinewidth{0.803000pt}%
\definecolor{currentstroke}{rgb}{0.000000,0.000000,0.000000}%
\pgfsetstrokecolor{currentstroke}%
\pgfsetdash{}{0pt}%
\pgfsys@defobject{currentmarker}{\pgfqpoint{0.000000in}{-0.048611in}}{\pgfqpoint{0.000000in}{0.000000in}}{%
\pgfpathmoveto{\pgfqpoint{0.000000in}{0.000000in}}%
\pgfpathlineto{\pgfqpoint{0.000000in}{-0.048611in}}%
\pgfusepath{stroke,fill}%
}%
\begin{pgfscope}%
\pgfsys@transformshift{0.842596in}{0.515123in}%
\pgfsys@useobject{currentmarker}{}%
\end{pgfscope}%
\end{pgfscope}%
\begin{pgfscope}%
\definecolor{textcolor}{rgb}{0.000000,0.000000,0.000000}%
\pgfsetstrokecolor{textcolor}%
\pgfsetfillcolor{textcolor}%
\pgftext[x=0.842596in,y=0.417901in,,top]{\color{textcolor}\rmfamily\fontsize{10.000000}{12.000000}\selectfont \(\displaystyle 0.0\)}%
\end{pgfscope}%
\begin{pgfscope}%
\pgfsetbuttcap%
\pgfsetroundjoin%
\definecolor{currentfill}{rgb}{0.000000,0.000000,0.000000}%
\pgfsetfillcolor{currentfill}%
\pgfsetlinewidth{0.803000pt}%
\definecolor{currentstroke}{rgb}{0.000000,0.000000,0.000000}%
\pgfsetstrokecolor{currentstroke}%
\pgfsetdash{}{0pt}%
\pgfsys@defobject{currentmarker}{\pgfqpoint{0.000000in}{-0.048611in}}{\pgfqpoint{0.000000in}{0.000000in}}{%
\pgfpathmoveto{\pgfqpoint{0.000000in}{0.000000in}}%
\pgfpathlineto{\pgfqpoint{0.000000in}{-0.048611in}}%
\pgfusepath{stroke,fill}%
}%
\begin{pgfscope}%
\pgfsys@transformshift{1.426572in}{0.515123in}%
\pgfsys@useobject{currentmarker}{}%
\end{pgfscope}%
\end{pgfscope}%
\begin{pgfscope}%
\definecolor{textcolor}{rgb}{0.000000,0.000000,0.000000}%
\pgfsetstrokecolor{textcolor}%
\pgfsetfillcolor{textcolor}%
\pgftext[x=1.426572in,y=0.417901in,,top]{\color{textcolor}\rmfamily\fontsize{10.000000}{12.000000}\selectfont \(\displaystyle 2.5\)}%
\end{pgfscope}%
\begin{pgfscope}%
\pgfsetbuttcap%
\pgfsetroundjoin%
\definecolor{currentfill}{rgb}{0.000000,0.000000,0.000000}%
\pgfsetfillcolor{currentfill}%
\pgfsetlinewidth{0.803000pt}%
\definecolor{currentstroke}{rgb}{0.000000,0.000000,0.000000}%
\pgfsetstrokecolor{currentstroke}%
\pgfsetdash{}{0pt}%
\pgfsys@defobject{currentmarker}{\pgfqpoint{0.000000in}{-0.048611in}}{\pgfqpoint{0.000000in}{0.000000in}}{%
\pgfpathmoveto{\pgfqpoint{0.000000in}{0.000000in}}%
\pgfpathlineto{\pgfqpoint{0.000000in}{-0.048611in}}%
\pgfusepath{stroke,fill}%
}%
\begin{pgfscope}%
\pgfsys@transformshift{2.010548in}{0.515123in}%
\pgfsys@useobject{currentmarker}{}%
\end{pgfscope}%
\end{pgfscope}%
\begin{pgfscope}%
\definecolor{textcolor}{rgb}{0.000000,0.000000,0.000000}%
\pgfsetstrokecolor{textcolor}%
\pgfsetfillcolor{textcolor}%
\pgftext[x=2.010548in,y=0.417901in,,top]{\color{textcolor}\rmfamily\fontsize{10.000000}{12.000000}\selectfont \(\displaystyle 5.0\)}%
\end{pgfscope}%
\begin{pgfscope}%
\pgfsetbuttcap%
\pgfsetroundjoin%
\definecolor{currentfill}{rgb}{0.000000,0.000000,0.000000}%
\pgfsetfillcolor{currentfill}%
\pgfsetlinewidth{0.803000pt}%
\definecolor{currentstroke}{rgb}{0.000000,0.000000,0.000000}%
\pgfsetstrokecolor{currentstroke}%
\pgfsetdash{}{0pt}%
\pgfsys@defobject{currentmarker}{\pgfqpoint{0.000000in}{-0.048611in}}{\pgfqpoint{0.000000in}{0.000000in}}{%
\pgfpathmoveto{\pgfqpoint{0.000000in}{0.000000in}}%
\pgfpathlineto{\pgfqpoint{0.000000in}{-0.048611in}}%
\pgfusepath{stroke,fill}%
}%
\begin{pgfscope}%
\pgfsys@transformshift{2.594525in}{0.515123in}%
\pgfsys@useobject{currentmarker}{}%
\end{pgfscope}%
\end{pgfscope}%
\begin{pgfscope}%
\definecolor{textcolor}{rgb}{0.000000,0.000000,0.000000}%
\pgfsetstrokecolor{textcolor}%
\pgfsetfillcolor{textcolor}%
\pgftext[x=2.594525in,y=0.417901in,,top]{\color{textcolor}\rmfamily\fontsize{10.000000}{12.000000}\selectfont \(\displaystyle 7.5\)}%
\end{pgfscope}%
\begin{pgfscope}%
\definecolor{textcolor}{rgb}{0.000000,0.000000,0.000000}%
\pgfsetstrokecolor{textcolor}%
\pgfsetfillcolor{textcolor}%
\pgftext[x=1.740819in,y=0.238889in,,top]{\color{textcolor}\rmfamily\fontsize{10.000000}{12.000000}\selectfont I(\(\displaystyle \mu\)A)}%
\end{pgfscope}%
\begin{pgfscope}%
\pgfsetbuttcap%
\pgfsetroundjoin%
\definecolor{currentfill}{rgb}{0.000000,0.000000,0.000000}%
\pgfsetfillcolor{currentfill}%
\pgfsetlinewidth{0.803000pt}%
\definecolor{currentstroke}{rgb}{0.000000,0.000000,0.000000}%
\pgfsetstrokecolor{currentstroke}%
\pgfsetdash{}{0pt}%
\pgfsys@defobject{currentmarker}{\pgfqpoint{-0.048611in}{0.000000in}}{\pgfqpoint{0.000000in}{0.000000in}}{%
\pgfpathmoveto{\pgfqpoint{0.000000in}{0.000000in}}%
\pgfpathlineto{\pgfqpoint{-0.048611in}{0.000000in}}%
\pgfusepath{stroke,fill}%
}%
\begin{pgfscope}%
\pgfsys@transformshift{0.772069in}{1.115027in}%
\pgfsys@useobject{currentmarker}{}%
\end{pgfscope}%
\end{pgfscope}%
\begin{pgfscope}%
\definecolor{textcolor}{rgb}{0.000000,0.000000,0.000000}%
\pgfsetstrokecolor{textcolor}%
\pgfsetfillcolor{textcolor}%
\pgftext[x=0.605402in,y=1.066802in,left,base]{\color{textcolor}\rmfamily\fontsize{10.000000}{12.000000}\selectfont \(\displaystyle 5\)}%
\end{pgfscope}%
\begin{pgfscope}%
\pgfsetbuttcap%
\pgfsetroundjoin%
\definecolor{currentfill}{rgb}{0.000000,0.000000,0.000000}%
\pgfsetfillcolor{currentfill}%
\pgfsetlinewidth{0.803000pt}%
\definecolor{currentstroke}{rgb}{0.000000,0.000000,0.000000}%
\pgfsetstrokecolor{currentstroke}%
\pgfsetdash{}{0pt}%
\pgfsys@defobject{currentmarker}{\pgfqpoint{-0.048611in}{0.000000in}}{\pgfqpoint{0.000000in}{0.000000in}}{%
\pgfpathmoveto{\pgfqpoint{0.000000in}{0.000000in}}%
\pgfpathlineto{\pgfqpoint{-0.048611in}{0.000000in}}%
\pgfusepath{stroke,fill}%
}%
\begin{pgfscope}%
\pgfsys@transformshift{0.772069in}{1.756051in}%
\pgfsys@useobject{currentmarker}{}%
\end{pgfscope}%
\end{pgfscope}%
\begin{pgfscope}%
\definecolor{textcolor}{rgb}{0.000000,0.000000,0.000000}%
\pgfsetstrokecolor{textcolor}%
\pgfsetfillcolor{textcolor}%
\pgftext[x=0.535957in,y=1.707825in,left,base]{\color{textcolor}\rmfamily\fontsize{10.000000}{12.000000}\selectfont \(\displaystyle 10\)}%
\end{pgfscope}%
\begin{pgfscope}%
\definecolor{textcolor}{rgb}{0.000000,0.000000,0.000000}%
\pgfsetstrokecolor{textcolor}%
\pgfsetfillcolor{textcolor}%
\pgftext[x=0.258179in,y=1.188873in,,bottom]{\color{textcolor}\rmfamily\fontsize{10.000000}{12.000000}\selectfont V(V)}%
\end{pgfscope}%
\begin{pgfscope}%
\pgfpathrectangle{\pgfqpoint{0.772069in}{0.515123in}}{\pgfqpoint{1.937500in}{1.347500in}}%
\pgfusepath{clip}%
\pgfsetbuttcap%
\pgfsetroundjoin%
\definecolor{currentfill}{rgb}{0.121569,0.466667,0.705882}%
\pgfsetfillcolor{currentfill}%
\pgfsetlinewidth{1.003750pt}%
\definecolor{currentstroke}{rgb}{0.121569,0.466667,0.705882}%
\pgfsetstrokecolor{currentstroke}%
\pgfsetdash{}{0pt}%
\pgfpathmoveto{\pgfqpoint{1.006109in}{0.564645in}}%
\pgfpathcurveto{\pgfqpoint{1.017159in}{0.564645in}}{\pgfqpoint{1.027758in}{0.569035in}}{\pgfqpoint{1.035572in}{0.576849in}}%
\pgfpathcurveto{\pgfqpoint{1.043386in}{0.584662in}}{\pgfqpoint{1.047776in}{0.595261in}}{\pgfqpoint{1.047776in}{0.606311in}}%
\pgfpathcurveto{\pgfqpoint{1.047776in}{0.617362in}}{\pgfqpoint{1.043386in}{0.627961in}}{\pgfqpoint{1.035572in}{0.635774in}}%
\pgfpathcurveto{\pgfqpoint{1.027758in}{0.643588in}}{\pgfqpoint{1.017159in}{0.647978in}}{\pgfqpoint{1.006109in}{0.647978in}}%
\pgfpathcurveto{\pgfqpoint{0.995059in}{0.647978in}}{\pgfqpoint{0.984460in}{0.643588in}}{\pgfqpoint{0.976646in}{0.635774in}}%
\pgfpathcurveto{\pgfqpoint{0.968833in}{0.627961in}}{\pgfqpoint{0.964442in}{0.617362in}}{\pgfqpoint{0.964442in}{0.606311in}}%
\pgfpathcurveto{\pgfqpoint{0.964442in}{0.595261in}}{\pgfqpoint{0.968833in}{0.584662in}}{\pgfqpoint{0.976646in}{0.576849in}}%
\pgfpathcurveto{\pgfqpoint{0.984460in}{0.569035in}}{\pgfqpoint{0.995059in}{0.564645in}}{\pgfqpoint{1.006109in}{0.564645in}}%
\pgfpathclose%
\pgfusepath{stroke,fill}%
\end{pgfscope}%
\begin{pgfscope}%
\pgfpathrectangle{\pgfqpoint{0.772069in}{0.515123in}}{\pgfqpoint{1.937500in}{1.347500in}}%
\pgfusepath{clip}%
\pgfsetbuttcap%
\pgfsetroundjoin%
\definecolor{currentfill}{rgb}{0.121569,0.466667,0.705882}%
\pgfsetfillcolor{currentfill}%
\pgfsetlinewidth{1.003750pt}%
\definecolor{currentstroke}{rgb}{0.121569,0.466667,0.705882}%
\pgfsetstrokecolor{currentstroke}%
\pgfsetdash{}{0pt}%
\pgfpathmoveto{\pgfqpoint{1.192982in}{0.688747in}}%
\pgfpathcurveto{\pgfqpoint{1.204032in}{0.688747in}}{\pgfqpoint{1.214631in}{0.693137in}}{\pgfqpoint{1.222444in}{0.700951in}}%
\pgfpathcurveto{\pgfqpoint{1.230258in}{0.708764in}}{\pgfqpoint{1.234648in}{0.719363in}}{\pgfqpoint{1.234648in}{0.730413in}}%
\pgfpathcurveto{\pgfqpoint{1.234648in}{0.741464in}}{\pgfqpoint{1.230258in}{0.752063in}}{\pgfqpoint{1.222444in}{0.759876in}}%
\pgfpathcurveto{\pgfqpoint{1.214631in}{0.767690in}}{\pgfqpoint{1.204032in}{0.772080in}}{\pgfqpoint{1.192982in}{0.772080in}}%
\pgfpathcurveto{\pgfqpoint{1.181931in}{0.772080in}}{\pgfqpoint{1.171332in}{0.767690in}}{\pgfqpoint{1.163519in}{0.759876in}}%
\pgfpathcurveto{\pgfqpoint{1.155705in}{0.752063in}}{\pgfqpoint{1.151315in}{0.741464in}}{\pgfqpoint{1.151315in}{0.730413in}}%
\pgfpathcurveto{\pgfqpoint{1.151315in}{0.719363in}}{\pgfqpoint{1.155705in}{0.708764in}}{\pgfqpoint{1.163519in}{0.700951in}}%
\pgfpathcurveto{\pgfqpoint{1.171332in}{0.693137in}}{\pgfqpoint{1.181931in}{0.688747in}}{\pgfqpoint{1.192982in}{0.688747in}}%
\pgfpathclose%
\pgfusepath{stroke,fill}%
\end{pgfscope}%
\begin{pgfscope}%
\pgfpathrectangle{\pgfqpoint{0.772069in}{0.515123in}}{\pgfqpoint{1.937500in}{1.347500in}}%
\pgfusepath{clip}%
\pgfsetbuttcap%
\pgfsetroundjoin%
\definecolor{currentfill}{rgb}{0.121569,0.466667,0.705882}%
\pgfsetfillcolor{currentfill}%
\pgfsetlinewidth{1.003750pt}%
\definecolor{currentstroke}{rgb}{0.121569,0.466667,0.705882}%
\pgfsetstrokecolor{currentstroke}%
\pgfsetdash{}{0pt}%
\pgfpathmoveto{\pgfqpoint{1.356495in}{0.822080in}}%
\pgfpathcurveto{\pgfqpoint{1.367545in}{0.822080in}}{\pgfqpoint{1.378144in}{0.826470in}}{\pgfqpoint{1.385958in}{0.834284in}}%
\pgfpathcurveto{\pgfqpoint{1.393771in}{0.842097in}}{\pgfqpoint{1.398162in}{0.852696in}}{\pgfqpoint{1.398162in}{0.863746in}}%
\pgfpathcurveto{\pgfqpoint{1.398162in}{0.874796in}}{\pgfqpoint{1.393771in}{0.885395in}}{\pgfqpoint{1.385958in}{0.893209in}}%
\pgfpathcurveto{\pgfqpoint{1.378144in}{0.901023in}}{\pgfqpoint{1.367545in}{0.905413in}}{\pgfqpoint{1.356495in}{0.905413in}}%
\pgfpathcurveto{\pgfqpoint{1.345445in}{0.905413in}}{\pgfqpoint{1.334846in}{0.901023in}}{\pgfqpoint{1.327032in}{0.893209in}}%
\pgfpathcurveto{\pgfqpoint{1.319218in}{0.885395in}}{\pgfqpoint{1.314828in}{0.874796in}}{\pgfqpoint{1.314828in}{0.863746in}}%
\pgfpathcurveto{\pgfqpoint{1.314828in}{0.852696in}}{\pgfqpoint{1.319218in}{0.842097in}}{\pgfqpoint{1.327032in}{0.834284in}}%
\pgfpathcurveto{\pgfqpoint{1.334846in}{0.826470in}}{\pgfqpoint{1.345445in}{0.822080in}}{\pgfqpoint{1.356495in}{0.822080in}}%
\pgfpathclose%
\pgfusepath{stroke,fill}%
\end{pgfscope}%
\begin{pgfscope}%
\pgfpathrectangle{\pgfqpoint{0.772069in}{0.515123in}}{\pgfqpoint{1.937500in}{1.347500in}}%
\pgfusepath{clip}%
\pgfsetbuttcap%
\pgfsetroundjoin%
\definecolor{currentfill}{rgb}{0.121569,0.466667,0.705882}%
\pgfsetfillcolor{currentfill}%
\pgfsetlinewidth{1.003750pt}%
\definecolor{currentstroke}{rgb}{0.121569,0.466667,0.705882}%
\pgfsetstrokecolor{currentstroke}%
\pgfsetdash{}{0pt}%
\pgfpathmoveto{\pgfqpoint{1.543367in}{0.952848in}}%
\pgfpathcurveto{\pgfqpoint{1.554417in}{0.952848in}}{\pgfqpoint{1.565016in}{0.957239in}}{\pgfqpoint{1.572830in}{0.965052in}}%
\pgfpathcurveto{\pgfqpoint{1.580644in}{0.972866in}}{\pgfqpoint{1.585034in}{0.983465in}}{\pgfqpoint{1.585034in}{0.994515in}}%
\pgfpathcurveto{\pgfqpoint{1.585034in}{1.005565in}}{\pgfqpoint{1.580644in}{1.016164in}}{\pgfqpoint{1.572830in}{1.023978in}}%
\pgfpathcurveto{\pgfqpoint{1.565016in}{1.031791in}}{\pgfqpoint{1.554417in}{1.036182in}}{\pgfqpoint{1.543367in}{1.036182in}}%
\pgfpathcurveto{\pgfqpoint{1.532317in}{1.036182in}}{\pgfqpoint{1.521718in}{1.031791in}}{\pgfqpoint{1.513905in}{1.023978in}}%
\pgfpathcurveto{\pgfqpoint{1.506091in}{1.016164in}}{\pgfqpoint{1.501701in}{1.005565in}}{\pgfqpoint{1.501701in}{0.994515in}}%
\pgfpathcurveto{\pgfqpoint{1.501701in}{0.983465in}}{\pgfqpoint{1.506091in}{0.972866in}}{\pgfqpoint{1.513905in}{0.965052in}}%
\pgfpathcurveto{\pgfqpoint{1.521718in}{0.957239in}}{\pgfqpoint{1.532317in}{0.952848in}}{\pgfqpoint{1.543367in}{0.952848in}}%
\pgfpathclose%
\pgfusepath{stroke,fill}%
\end{pgfscope}%
\begin{pgfscope}%
\pgfpathrectangle{\pgfqpoint{0.772069in}{0.515123in}}{\pgfqpoint{1.937500in}{1.347500in}}%
\pgfusepath{clip}%
\pgfsetbuttcap%
\pgfsetroundjoin%
\definecolor{currentfill}{rgb}{0.121569,0.466667,0.705882}%
\pgfsetfillcolor{currentfill}%
\pgfsetlinewidth{1.003750pt}%
\definecolor{currentstroke}{rgb}{0.121569,0.466667,0.705882}%
\pgfsetstrokecolor{currentstroke}%
\pgfsetdash{}{0pt}%
\pgfpathmoveto{\pgfqpoint{1.730240in}{1.083617in}}%
\pgfpathcurveto{\pgfqpoint{1.741290in}{1.083617in}}{\pgfqpoint{1.751889in}{1.088007in}}{\pgfqpoint{1.759702in}{1.095821in}}%
\pgfpathcurveto{\pgfqpoint{1.767516in}{1.103635in}}{\pgfqpoint{1.771906in}{1.114234in}}{\pgfqpoint{1.771906in}{1.125284in}}%
\pgfpathcurveto{\pgfqpoint{1.771906in}{1.136334in}}{\pgfqpoint{1.767516in}{1.146933in}}{\pgfqpoint{1.759702in}{1.154747in}}%
\pgfpathcurveto{\pgfqpoint{1.751889in}{1.162560in}}{\pgfqpoint{1.741290in}{1.166950in}}{\pgfqpoint{1.730240in}{1.166950in}}%
\pgfpathcurveto{\pgfqpoint{1.719190in}{1.166950in}}{\pgfqpoint{1.708591in}{1.162560in}}{\pgfqpoint{1.700777in}{1.154747in}}%
\pgfpathcurveto{\pgfqpoint{1.692963in}{1.146933in}}{\pgfqpoint{1.688573in}{1.136334in}}{\pgfqpoint{1.688573in}{1.125284in}}%
\pgfpathcurveto{\pgfqpoint{1.688573in}{1.114234in}}{\pgfqpoint{1.692963in}{1.103635in}}{\pgfqpoint{1.700777in}{1.095821in}}%
\pgfpathcurveto{\pgfqpoint{1.708591in}{1.088007in}}{\pgfqpoint{1.719190in}{1.083617in}}{\pgfqpoint{1.730240in}{1.083617in}}%
\pgfpathclose%
\pgfusepath{stroke,fill}%
\end{pgfscope}%
\begin{pgfscope}%
\pgfpathrectangle{\pgfqpoint{0.772069in}{0.515123in}}{\pgfqpoint{1.937500in}{1.347500in}}%
\pgfusepath{clip}%
\pgfsetbuttcap%
\pgfsetroundjoin%
\definecolor{currentfill}{rgb}{0.121569,0.466667,0.705882}%
\pgfsetfillcolor{currentfill}%
\pgfsetlinewidth{1.003750pt}%
\definecolor{currentstroke}{rgb}{0.121569,0.466667,0.705882}%
\pgfsetstrokecolor{currentstroke}%
\pgfsetdash{}{0pt}%
\pgfpathmoveto{\pgfqpoint{1.893753in}{1.207976in}}%
\pgfpathcurveto{\pgfqpoint{1.904803in}{1.207976in}}{\pgfqpoint{1.915402in}{1.212366in}}{\pgfqpoint{1.923216in}{1.220180in}}%
\pgfpathcurveto{\pgfqpoint{1.931029in}{1.227993in}}{\pgfqpoint{1.935420in}{1.238592in}}{\pgfqpoint{1.935420in}{1.249642in}}%
\pgfpathcurveto{\pgfqpoint{1.935420in}{1.260692in}}{\pgfqpoint{1.931029in}{1.271291in}}{\pgfqpoint{1.923216in}{1.279105in}}%
\pgfpathcurveto{\pgfqpoint{1.915402in}{1.286919in}}{\pgfqpoint{1.904803in}{1.291309in}}{\pgfqpoint{1.893753in}{1.291309in}}%
\pgfpathcurveto{\pgfqpoint{1.882703in}{1.291309in}}{\pgfqpoint{1.872104in}{1.286919in}}{\pgfqpoint{1.864290in}{1.279105in}}%
\pgfpathcurveto{\pgfqpoint{1.856477in}{1.271291in}}{\pgfqpoint{1.852086in}{1.260692in}}{\pgfqpoint{1.852086in}{1.249642in}}%
\pgfpathcurveto{\pgfqpoint{1.852086in}{1.238592in}}{\pgfqpoint{1.856477in}{1.227993in}}{\pgfqpoint{1.864290in}{1.220180in}}%
\pgfpathcurveto{\pgfqpoint{1.872104in}{1.212366in}}{\pgfqpoint{1.882703in}{1.207976in}}{\pgfqpoint{1.893753in}{1.207976in}}%
\pgfpathclose%
\pgfusepath{stroke,fill}%
\end{pgfscope}%
\begin{pgfscope}%
\pgfpathrectangle{\pgfqpoint{0.772069in}{0.515123in}}{\pgfqpoint{1.937500in}{1.347500in}}%
\pgfusepath{clip}%
\pgfsetbuttcap%
\pgfsetroundjoin%
\definecolor{currentfill}{rgb}{0.121569,0.466667,0.705882}%
\pgfsetfillcolor{currentfill}%
\pgfsetlinewidth{1.003750pt}%
\definecolor{currentstroke}{rgb}{0.121569,0.466667,0.705882}%
\pgfsetstrokecolor{currentstroke}%
\pgfsetdash{}{0pt}%
\pgfpathmoveto{\pgfqpoint{2.080625in}{1.342591in}}%
\pgfpathcurveto{\pgfqpoint{2.091676in}{1.342591in}}{\pgfqpoint{2.102275in}{1.346981in}}{\pgfqpoint{2.110088in}{1.354794in}}%
\pgfpathcurveto{\pgfqpoint{2.117902in}{1.362608in}}{\pgfqpoint{2.122292in}{1.373207in}}{\pgfqpoint{2.122292in}{1.384257in}}%
\pgfpathcurveto{\pgfqpoint{2.122292in}{1.395307in}}{\pgfqpoint{2.117902in}{1.405906in}}{\pgfqpoint{2.110088in}{1.413720in}}%
\pgfpathcurveto{\pgfqpoint{2.102275in}{1.421534in}}{\pgfqpoint{2.091676in}{1.425924in}}{\pgfqpoint{2.080625in}{1.425924in}}%
\pgfpathcurveto{\pgfqpoint{2.069575in}{1.425924in}}{\pgfqpoint{2.058976in}{1.421534in}}{\pgfqpoint{2.051163in}{1.413720in}}%
\pgfpathcurveto{\pgfqpoint{2.043349in}{1.405906in}}{\pgfqpoint{2.038959in}{1.395307in}}{\pgfqpoint{2.038959in}{1.384257in}}%
\pgfpathcurveto{\pgfqpoint{2.038959in}{1.373207in}}{\pgfqpoint{2.043349in}{1.362608in}}{\pgfqpoint{2.051163in}{1.354794in}}%
\pgfpathcurveto{\pgfqpoint{2.058976in}{1.346981in}}{\pgfqpoint{2.069575in}{1.342591in}}{\pgfqpoint{2.080625in}{1.342591in}}%
\pgfpathclose%
\pgfusepath{stroke,fill}%
\end{pgfscope}%
\begin{pgfscope}%
\pgfpathrectangle{\pgfqpoint{0.772069in}{0.515123in}}{\pgfqpoint{1.937500in}{1.347500in}}%
\pgfusepath{clip}%
\pgfsetbuttcap%
\pgfsetroundjoin%
\definecolor{currentfill}{rgb}{0.121569,0.466667,0.705882}%
\pgfsetfillcolor{currentfill}%
\pgfsetlinewidth{1.003750pt}%
\definecolor{currentstroke}{rgb}{0.121569,0.466667,0.705882}%
\pgfsetstrokecolor{currentstroke}%
\pgfsetdash{}{0pt}%
\pgfpathmoveto{\pgfqpoint{2.267498in}{1.472077in}}%
\pgfpathcurveto{\pgfqpoint{2.278548in}{1.472077in}}{\pgfqpoint{2.289147in}{1.476468in}}{\pgfqpoint{2.296961in}{1.484281in}}%
\pgfpathcurveto{\pgfqpoint{2.304774in}{1.492095in}}{\pgfqpoint{2.309165in}{1.502694in}}{\pgfqpoint{2.309165in}{1.513744in}}%
\pgfpathcurveto{\pgfqpoint{2.309165in}{1.524794in}}{\pgfqpoint{2.304774in}{1.535393in}}{\pgfqpoint{2.296961in}{1.543207in}}%
\pgfpathcurveto{\pgfqpoint{2.289147in}{1.551020in}}{\pgfqpoint{2.278548in}{1.555411in}}{\pgfqpoint{2.267498in}{1.555411in}}%
\pgfpathcurveto{\pgfqpoint{2.256448in}{1.555411in}}{\pgfqpoint{2.245849in}{1.551020in}}{\pgfqpoint{2.238035in}{1.543207in}}%
\pgfpathcurveto{\pgfqpoint{2.230221in}{1.535393in}}{\pgfqpoint{2.225831in}{1.524794in}}{\pgfqpoint{2.225831in}{1.513744in}}%
\pgfpathcurveto{\pgfqpoint{2.225831in}{1.502694in}}{\pgfqpoint{2.230221in}{1.492095in}}{\pgfqpoint{2.238035in}{1.484281in}}%
\pgfpathcurveto{\pgfqpoint{2.245849in}{1.476468in}}{\pgfqpoint{2.256448in}{1.472077in}}{\pgfqpoint{2.267498in}{1.472077in}}%
\pgfpathclose%
\pgfusepath{stroke,fill}%
\end{pgfscope}%
\begin{pgfscope}%
\pgfpathrectangle{\pgfqpoint{0.772069in}{0.515123in}}{\pgfqpoint{1.937500in}{1.347500in}}%
\pgfusepath{clip}%
\pgfsetbuttcap%
\pgfsetroundjoin%
\definecolor{currentfill}{rgb}{0.121569,0.466667,0.705882}%
\pgfsetfillcolor{currentfill}%
\pgfsetlinewidth{1.003750pt}%
\definecolor{currentstroke}{rgb}{0.121569,0.466667,0.705882}%
\pgfsetstrokecolor{currentstroke}%
\pgfsetdash{}{0pt}%
\pgfpathmoveto{\pgfqpoint{2.431011in}{1.599000in}}%
\pgfpathcurveto{\pgfqpoint{2.442061in}{1.599000in}}{\pgfqpoint{2.452660in}{1.603390in}}{\pgfqpoint{2.460474in}{1.611204in}}%
\pgfpathcurveto{\pgfqpoint{2.468288in}{1.619017in}}{\pgfqpoint{2.472678in}{1.629616in}}{\pgfqpoint{2.472678in}{1.640667in}}%
\pgfpathcurveto{\pgfqpoint{2.472678in}{1.651717in}}{\pgfqpoint{2.468288in}{1.662316in}}{\pgfqpoint{2.460474in}{1.670129in}}%
\pgfpathcurveto{\pgfqpoint{2.452660in}{1.677943in}}{\pgfqpoint{2.442061in}{1.682333in}}{\pgfqpoint{2.431011in}{1.682333in}}%
\pgfpathcurveto{\pgfqpoint{2.419961in}{1.682333in}}{\pgfqpoint{2.409362in}{1.677943in}}{\pgfqpoint{2.401548in}{1.670129in}}%
\pgfpathcurveto{\pgfqpoint{2.393735in}{1.662316in}}{\pgfqpoint{2.389345in}{1.651717in}}{\pgfqpoint{2.389345in}{1.640667in}}%
\pgfpathcurveto{\pgfqpoint{2.389345in}{1.629616in}}{\pgfqpoint{2.393735in}{1.619017in}}{\pgfqpoint{2.401548in}{1.611204in}}%
\pgfpathcurveto{\pgfqpoint{2.409362in}{1.603390in}}{\pgfqpoint{2.419961in}{1.599000in}}{\pgfqpoint{2.431011in}{1.599000in}}%
\pgfpathclose%
\pgfusepath{stroke,fill}%
\end{pgfscope}%
\begin{pgfscope}%
\pgfpathrectangle{\pgfqpoint{0.772069in}{0.515123in}}{\pgfqpoint{1.937500in}{1.347500in}}%
\pgfusepath{clip}%
\pgfsetbuttcap%
\pgfsetroundjoin%
\definecolor{currentfill}{rgb}{0.121569,0.466667,0.705882}%
\pgfsetfillcolor{currentfill}%
\pgfsetlinewidth{1.003750pt}%
\definecolor{currentstroke}{rgb}{0.121569,0.466667,0.705882}%
\pgfsetstrokecolor{currentstroke}%
\pgfsetdash{}{0pt}%
\pgfpathmoveto{\pgfqpoint{2.617884in}{1.729769in}}%
\pgfpathcurveto{\pgfqpoint{2.628934in}{1.729769in}}{\pgfqpoint{2.639533in}{1.734159in}}{\pgfqpoint{2.647346in}{1.741972in}}%
\pgfpathcurveto{\pgfqpoint{2.655160in}{1.749786in}}{\pgfqpoint{2.659550in}{1.760385in}}{\pgfqpoint{2.659550in}{1.771435in}}%
\pgfpathcurveto{\pgfqpoint{2.659550in}{1.782485in}}{\pgfqpoint{2.655160in}{1.793084in}}{\pgfqpoint{2.647346in}{1.800898in}}%
\pgfpathcurveto{\pgfqpoint{2.639533in}{1.808712in}}{\pgfqpoint{2.628934in}{1.813102in}}{\pgfqpoint{2.617884in}{1.813102in}}%
\pgfpathcurveto{\pgfqpoint{2.606833in}{1.813102in}}{\pgfqpoint{2.596234in}{1.808712in}}{\pgfqpoint{2.588421in}{1.800898in}}%
\pgfpathcurveto{\pgfqpoint{2.580607in}{1.793084in}}{\pgfqpoint{2.576217in}{1.782485in}}{\pgfqpoint{2.576217in}{1.771435in}}%
\pgfpathcurveto{\pgfqpoint{2.576217in}{1.760385in}}{\pgfqpoint{2.580607in}{1.749786in}}{\pgfqpoint{2.588421in}{1.741972in}}%
\pgfpathcurveto{\pgfqpoint{2.596234in}{1.734159in}}{\pgfqpoint{2.606833in}{1.729769in}}{\pgfqpoint{2.617884in}{1.729769in}}%
\pgfpathclose%
\pgfusepath{stroke,fill}%
\end{pgfscope}%
\begin{pgfscope}%
\pgfpathrectangle{\pgfqpoint{0.772069in}{0.515123in}}{\pgfqpoint{1.937500in}{1.347500in}}%
\pgfusepath{clip}%
\pgfsetbuttcap%
\pgfsetroundjoin%
\definecolor{currentfill}{rgb}{0.117647,0.564706,1.000000}%
\pgfsetfillcolor{currentfill}%
\pgfsetlinewidth{1.003750pt}%
\definecolor{currentstroke}{rgb}{0.117647,0.564706,1.000000}%
\pgfsetstrokecolor{currentstroke}%
\pgfsetdash{}{0pt}%
\pgfpathmoveto{\pgfqpoint{0.936032in}{0.564645in}}%
\pgfpathcurveto{\pgfqpoint{0.947082in}{0.564645in}}{\pgfqpoint{0.957681in}{0.569035in}}{\pgfqpoint{0.965495in}{0.576849in}}%
\pgfpathcurveto{\pgfqpoint{0.973308in}{0.584662in}}{\pgfqpoint{0.977699in}{0.595261in}}{\pgfqpoint{0.977699in}{0.606311in}}%
\pgfpathcurveto{\pgfqpoint{0.977699in}{0.617362in}}{\pgfqpoint{0.973308in}{0.627961in}}{\pgfqpoint{0.965495in}{0.635774in}}%
\pgfpathcurveto{\pgfqpoint{0.957681in}{0.643588in}}{\pgfqpoint{0.947082in}{0.647978in}}{\pgfqpoint{0.936032in}{0.647978in}}%
\pgfpathcurveto{\pgfqpoint{0.924982in}{0.647978in}}{\pgfqpoint{0.914383in}{0.643588in}}{\pgfqpoint{0.906569in}{0.635774in}}%
\pgfpathcurveto{\pgfqpoint{0.898756in}{0.627961in}}{\pgfqpoint{0.894365in}{0.617362in}}{\pgfqpoint{0.894365in}{0.606311in}}%
\pgfpathcurveto{\pgfqpoint{0.894365in}{0.595261in}}{\pgfqpoint{0.898756in}{0.584662in}}{\pgfqpoint{0.906569in}{0.576849in}}%
\pgfpathcurveto{\pgfqpoint{0.914383in}{0.569035in}}{\pgfqpoint{0.924982in}{0.564645in}}{\pgfqpoint{0.936032in}{0.564645in}}%
\pgfpathclose%
\pgfusepath{stroke,fill}%
\end{pgfscope}%
\begin{pgfscope}%
\pgfpathrectangle{\pgfqpoint{0.772069in}{0.515123in}}{\pgfqpoint{1.937500in}{1.347500in}}%
\pgfusepath{clip}%
\pgfsetbuttcap%
\pgfsetroundjoin%
\definecolor{currentfill}{rgb}{0.117647,0.564706,1.000000}%
\pgfsetfillcolor{currentfill}%
\pgfsetlinewidth{1.003750pt}%
\definecolor{currentstroke}{rgb}{0.117647,0.564706,1.000000}%
\pgfsetstrokecolor{currentstroke}%
\pgfsetdash{}{0pt}%
\pgfpathmoveto{\pgfqpoint{1.052827in}{0.688747in}}%
\pgfpathcurveto{\pgfqpoint{1.063877in}{0.688747in}}{\pgfqpoint{1.074476in}{0.693137in}}{\pgfqpoint{1.082290in}{0.700951in}}%
\pgfpathcurveto{\pgfqpoint{1.090104in}{0.708764in}}{\pgfqpoint{1.094494in}{0.719363in}}{\pgfqpoint{1.094494in}{0.730413in}}%
\pgfpathcurveto{\pgfqpoint{1.094494in}{0.741464in}}{\pgfqpoint{1.090104in}{0.752063in}}{\pgfqpoint{1.082290in}{0.759876in}}%
\pgfpathcurveto{\pgfqpoint{1.074476in}{0.767690in}}{\pgfqpoint{1.063877in}{0.772080in}}{\pgfqpoint{1.052827in}{0.772080in}}%
\pgfpathcurveto{\pgfqpoint{1.041777in}{0.772080in}}{\pgfqpoint{1.031178in}{0.767690in}}{\pgfqpoint{1.023364in}{0.759876in}}%
\pgfpathcurveto{\pgfqpoint{1.015551in}{0.752063in}}{\pgfqpoint{1.011161in}{0.741464in}}{\pgfqpoint{1.011161in}{0.730413in}}%
\pgfpathcurveto{\pgfqpoint{1.011161in}{0.719363in}}{\pgfqpoint{1.015551in}{0.708764in}}{\pgfqpoint{1.023364in}{0.700951in}}%
\pgfpathcurveto{\pgfqpoint{1.031178in}{0.693137in}}{\pgfqpoint{1.041777in}{0.688747in}}{\pgfqpoint{1.052827in}{0.688747in}}%
\pgfpathclose%
\pgfusepath{stroke,fill}%
\end{pgfscope}%
\begin{pgfscope}%
\pgfpathrectangle{\pgfqpoint{0.772069in}{0.515123in}}{\pgfqpoint{1.937500in}{1.347500in}}%
\pgfusepath{clip}%
\pgfsetbuttcap%
\pgfsetroundjoin%
\definecolor{currentfill}{rgb}{0.117647,0.564706,1.000000}%
\pgfsetfillcolor{currentfill}%
\pgfsetlinewidth{1.003750pt}%
\definecolor{currentstroke}{rgb}{0.117647,0.564706,1.000000}%
\pgfsetstrokecolor{currentstroke}%
\pgfsetdash{}{0pt}%
\pgfpathmoveto{\pgfqpoint{1.169622in}{0.822080in}}%
\pgfpathcurveto{\pgfqpoint{1.180673in}{0.822080in}}{\pgfqpoint{1.191272in}{0.826470in}}{\pgfqpoint{1.199085in}{0.834284in}}%
\pgfpathcurveto{\pgfqpoint{1.206899in}{0.842097in}}{\pgfqpoint{1.211289in}{0.852696in}}{\pgfqpoint{1.211289in}{0.863746in}}%
\pgfpathcurveto{\pgfqpoint{1.211289in}{0.874796in}}{\pgfqpoint{1.206899in}{0.885395in}}{\pgfqpoint{1.199085in}{0.893209in}}%
\pgfpathcurveto{\pgfqpoint{1.191272in}{0.901023in}}{\pgfqpoint{1.180673in}{0.905413in}}{\pgfqpoint{1.169622in}{0.905413in}}%
\pgfpathcurveto{\pgfqpoint{1.158572in}{0.905413in}}{\pgfqpoint{1.147973in}{0.901023in}}{\pgfqpoint{1.140160in}{0.893209in}}%
\pgfpathcurveto{\pgfqpoint{1.132346in}{0.885395in}}{\pgfqpoint{1.127956in}{0.874796in}}{\pgfqpoint{1.127956in}{0.863746in}}%
\pgfpathcurveto{\pgfqpoint{1.127956in}{0.852696in}}{\pgfqpoint{1.132346in}{0.842097in}}{\pgfqpoint{1.140160in}{0.834284in}}%
\pgfpathcurveto{\pgfqpoint{1.147973in}{0.826470in}}{\pgfqpoint{1.158572in}{0.822080in}}{\pgfqpoint{1.169622in}{0.822080in}}%
\pgfpathclose%
\pgfusepath{stroke,fill}%
\end{pgfscope}%
\begin{pgfscope}%
\pgfpathrectangle{\pgfqpoint{0.772069in}{0.515123in}}{\pgfqpoint{1.937500in}{1.347500in}}%
\pgfusepath{clip}%
\pgfsetbuttcap%
\pgfsetroundjoin%
\definecolor{currentfill}{rgb}{0.117647,0.564706,1.000000}%
\pgfsetfillcolor{currentfill}%
\pgfsetlinewidth{1.003750pt}%
\definecolor{currentstroke}{rgb}{0.117647,0.564706,1.000000}%
\pgfsetstrokecolor{currentstroke}%
\pgfsetdash{}{0pt}%
\pgfpathmoveto{\pgfqpoint{1.286418in}{0.952848in}}%
\pgfpathcurveto{\pgfqpoint{1.297468in}{0.952848in}}{\pgfqpoint{1.308067in}{0.957239in}}{\pgfqpoint{1.315881in}{0.965052in}}%
\pgfpathcurveto{\pgfqpoint{1.323694in}{0.972866in}}{\pgfqpoint{1.328084in}{0.983465in}}{\pgfqpoint{1.328084in}{0.994515in}}%
\pgfpathcurveto{\pgfqpoint{1.328084in}{1.005565in}}{\pgfqpoint{1.323694in}{1.016164in}}{\pgfqpoint{1.315881in}{1.023978in}}%
\pgfpathcurveto{\pgfqpoint{1.308067in}{1.031791in}}{\pgfqpoint{1.297468in}{1.036182in}}{\pgfqpoint{1.286418in}{1.036182in}}%
\pgfpathcurveto{\pgfqpoint{1.275368in}{1.036182in}}{\pgfqpoint{1.264769in}{1.031791in}}{\pgfqpoint{1.256955in}{1.023978in}}%
\pgfpathcurveto{\pgfqpoint{1.249141in}{1.016164in}}{\pgfqpoint{1.244751in}{1.005565in}}{\pgfqpoint{1.244751in}{0.994515in}}%
\pgfpathcurveto{\pgfqpoint{1.244751in}{0.983465in}}{\pgfqpoint{1.249141in}{0.972866in}}{\pgfqpoint{1.256955in}{0.965052in}}%
\pgfpathcurveto{\pgfqpoint{1.264769in}{0.957239in}}{\pgfqpoint{1.275368in}{0.952848in}}{\pgfqpoint{1.286418in}{0.952848in}}%
\pgfpathclose%
\pgfusepath{stroke,fill}%
\end{pgfscope}%
\begin{pgfscope}%
\pgfpathrectangle{\pgfqpoint{0.772069in}{0.515123in}}{\pgfqpoint{1.937500in}{1.347500in}}%
\pgfusepath{clip}%
\pgfsetbuttcap%
\pgfsetroundjoin%
\definecolor{currentfill}{rgb}{0.117647,0.564706,1.000000}%
\pgfsetfillcolor{currentfill}%
\pgfsetlinewidth{1.003750pt}%
\definecolor{currentstroke}{rgb}{0.117647,0.564706,1.000000}%
\pgfsetstrokecolor{currentstroke}%
\pgfsetdash{}{0pt}%
\pgfpathmoveto{\pgfqpoint{1.403213in}{1.083617in}}%
\pgfpathcurveto{\pgfqpoint{1.414263in}{1.083617in}}{\pgfqpoint{1.424862in}{1.088007in}}{\pgfqpoint{1.432676in}{1.095821in}}%
\pgfpathcurveto{\pgfqpoint{1.440489in}{1.103635in}}{\pgfqpoint{1.444880in}{1.114234in}}{\pgfqpoint{1.444880in}{1.125284in}}%
\pgfpathcurveto{\pgfqpoint{1.444880in}{1.136334in}}{\pgfqpoint{1.440489in}{1.146933in}}{\pgfqpoint{1.432676in}{1.154747in}}%
\pgfpathcurveto{\pgfqpoint{1.424862in}{1.162560in}}{\pgfqpoint{1.414263in}{1.166950in}}{\pgfqpoint{1.403213in}{1.166950in}}%
\pgfpathcurveto{\pgfqpoint{1.392163in}{1.166950in}}{\pgfqpoint{1.381564in}{1.162560in}}{\pgfqpoint{1.373750in}{1.154747in}}%
\pgfpathcurveto{\pgfqpoint{1.365937in}{1.146933in}}{\pgfqpoint{1.361546in}{1.136334in}}{\pgfqpoint{1.361546in}{1.125284in}}%
\pgfpathcurveto{\pgfqpoint{1.361546in}{1.114234in}}{\pgfqpoint{1.365937in}{1.103635in}}{\pgfqpoint{1.373750in}{1.095821in}}%
\pgfpathcurveto{\pgfqpoint{1.381564in}{1.088007in}}{\pgfqpoint{1.392163in}{1.083617in}}{\pgfqpoint{1.403213in}{1.083617in}}%
\pgfpathclose%
\pgfusepath{stroke,fill}%
\end{pgfscope}%
\begin{pgfscope}%
\pgfpathrectangle{\pgfqpoint{0.772069in}{0.515123in}}{\pgfqpoint{1.937500in}{1.347500in}}%
\pgfusepath{clip}%
\pgfsetbuttcap%
\pgfsetroundjoin%
\definecolor{currentfill}{rgb}{0.117647,0.564706,1.000000}%
\pgfsetfillcolor{currentfill}%
\pgfsetlinewidth{1.003750pt}%
\definecolor{currentstroke}{rgb}{0.117647,0.564706,1.000000}%
\pgfsetstrokecolor{currentstroke}%
\pgfsetdash{}{0pt}%
\pgfpathmoveto{\pgfqpoint{1.520008in}{1.207976in}}%
\pgfpathcurveto{\pgfqpoint{1.531058in}{1.207976in}}{\pgfqpoint{1.541657in}{1.212366in}}{\pgfqpoint{1.549471in}{1.220180in}}%
\pgfpathcurveto{\pgfqpoint{1.557285in}{1.227993in}}{\pgfqpoint{1.561675in}{1.238592in}}{\pgfqpoint{1.561675in}{1.249642in}}%
\pgfpathcurveto{\pgfqpoint{1.561675in}{1.260692in}}{\pgfqpoint{1.557285in}{1.271291in}}{\pgfqpoint{1.549471in}{1.279105in}}%
\pgfpathcurveto{\pgfqpoint{1.541657in}{1.286919in}}{\pgfqpoint{1.531058in}{1.291309in}}{\pgfqpoint{1.520008in}{1.291309in}}%
\pgfpathcurveto{\pgfqpoint{1.508958in}{1.291309in}}{\pgfqpoint{1.498359in}{1.286919in}}{\pgfqpoint{1.490545in}{1.279105in}}%
\pgfpathcurveto{\pgfqpoint{1.482732in}{1.271291in}}{\pgfqpoint{1.478342in}{1.260692in}}{\pgfqpoint{1.478342in}{1.249642in}}%
\pgfpathcurveto{\pgfqpoint{1.478342in}{1.238592in}}{\pgfqpoint{1.482732in}{1.227993in}}{\pgfqpoint{1.490545in}{1.220180in}}%
\pgfpathcurveto{\pgfqpoint{1.498359in}{1.212366in}}{\pgfqpoint{1.508958in}{1.207976in}}{\pgfqpoint{1.520008in}{1.207976in}}%
\pgfpathclose%
\pgfusepath{stroke,fill}%
\end{pgfscope}%
\begin{pgfscope}%
\pgfpathrectangle{\pgfqpoint{0.772069in}{0.515123in}}{\pgfqpoint{1.937500in}{1.347500in}}%
\pgfusepath{clip}%
\pgfsetbuttcap%
\pgfsetroundjoin%
\definecolor{currentfill}{rgb}{0.117647,0.564706,1.000000}%
\pgfsetfillcolor{currentfill}%
\pgfsetlinewidth{1.003750pt}%
\definecolor{currentstroke}{rgb}{0.117647,0.564706,1.000000}%
\pgfsetstrokecolor{currentstroke}%
\pgfsetdash{}{0pt}%
\pgfpathmoveto{\pgfqpoint{1.636803in}{1.342591in}}%
\pgfpathcurveto{\pgfqpoint{1.647854in}{1.342591in}}{\pgfqpoint{1.658453in}{1.346981in}}{\pgfqpoint{1.666266in}{1.354794in}}%
\pgfpathcurveto{\pgfqpoint{1.674080in}{1.362608in}}{\pgfqpoint{1.678470in}{1.373207in}}{\pgfqpoint{1.678470in}{1.384257in}}%
\pgfpathcurveto{\pgfqpoint{1.678470in}{1.395307in}}{\pgfqpoint{1.674080in}{1.405906in}}{\pgfqpoint{1.666266in}{1.413720in}}%
\pgfpathcurveto{\pgfqpoint{1.658453in}{1.421534in}}{\pgfqpoint{1.647854in}{1.425924in}}{\pgfqpoint{1.636803in}{1.425924in}}%
\pgfpathcurveto{\pgfqpoint{1.625753in}{1.425924in}}{\pgfqpoint{1.615154in}{1.421534in}}{\pgfqpoint{1.607341in}{1.413720in}}%
\pgfpathcurveto{\pgfqpoint{1.599527in}{1.405906in}}{\pgfqpoint{1.595137in}{1.395307in}}{\pgfqpoint{1.595137in}{1.384257in}}%
\pgfpathcurveto{\pgfqpoint{1.595137in}{1.373207in}}{\pgfqpoint{1.599527in}{1.362608in}}{\pgfqpoint{1.607341in}{1.354794in}}%
\pgfpathcurveto{\pgfqpoint{1.615154in}{1.346981in}}{\pgfqpoint{1.625753in}{1.342591in}}{\pgfqpoint{1.636803in}{1.342591in}}%
\pgfpathclose%
\pgfusepath{stroke,fill}%
\end{pgfscope}%
\begin{pgfscope}%
\pgfpathrectangle{\pgfqpoint{0.772069in}{0.515123in}}{\pgfqpoint{1.937500in}{1.347500in}}%
\pgfusepath{clip}%
\pgfsetbuttcap%
\pgfsetroundjoin%
\definecolor{currentfill}{rgb}{0.117647,0.564706,1.000000}%
\pgfsetfillcolor{currentfill}%
\pgfsetlinewidth{1.003750pt}%
\definecolor{currentstroke}{rgb}{0.117647,0.564706,1.000000}%
\pgfsetstrokecolor{currentstroke}%
\pgfsetdash{}{0pt}%
\pgfpathmoveto{\pgfqpoint{1.753599in}{1.472077in}}%
\pgfpathcurveto{\pgfqpoint{1.764649in}{1.472077in}}{\pgfqpoint{1.775248in}{1.476468in}}{\pgfqpoint{1.783062in}{1.484281in}}%
\pgfpathcurveto{\pgfqpoint{1.790875in}{1.492095in}}{\pgfqpoint{1.795265in}{1.502694in}}{\pgfqpoint{1.795265in}{1.513744in}}%
\pgfpathcurveto{\pgfqpoint{1.795265in}{1.524794in}}{\pgfqpoint{1.790875in}{1.535393in}}{\pgfqpoint{1.783062in}{1.543207in}}%
\pgfpathcurveto{\pgfqpoint{1.775248in}{1.551020in}}{\pgfqpoint{1.764649in}{1.555411in}}{\pgfqpoint{1.753599in}{1.555411in}}%
\pgfpathcurveto{\pgfqpoint{1.742549in}{1.555411in}}{\pgfqpoint{1.731950in}{1.551020in}}{\pgfqpoint{1.724136in}{1.543207in}}%
\pgfpathcurveto{\pgfqpoint{1.716322in}{1.535393in}}{\pgfqpoint{1.711932in}{1.524794in}}{\pgfqpoint{1.711932in}{1.513744in}}%
\pgfpathcurveto{\pgfqpoint{1.711932in}{1.502694in}}{\pgfqpoint{1.716322in}{1.492095in}}{\pgfqpoint{1.724136in}{1.484281in}}%
\pgfpathcurveto{\pgfqpoint{1.731950in}{1.476468in}}{\pgfqpoint{1.742549in}{1.472077in}}{\pgfqpoint{1.753599in}{1.472077in}}%
\pgfpathclose%
\pgfusepath{stroke,fill}%
\end{pgfscope}%
\begin{pgfscope}%
\pgfpathrectangle{\pgfqpoint{0.772069in}{0.515123in}}{\pgfqpoint{1.937500in}{1.347500in}}%
\pgfusepath{clip}%
\pgfsetbuttcap%
\pgfsetroundjoin%
\definecolor{currentfill}{rgb}{0.117647,0.564706,1.000000}%
\pgfsetfillcolor{currentfill}%
\pgfsetlinewidth{1.003750pt}%
\definecolor{currentstroke}{rgb}{0.117647,0.564706,1.000000}%
\pgfsetstrokecolor{currentstroke}%
\pgfsetdash{}{0pt}%
\pgfpathmoveto{\pgfqpoint{1.870394in}{1.599000in}}%
\pgfpathcurveto{\pgfqpoint{1.881444in}{1.599000in}}{\pgfqpoint{1.892043in}{1.603390in}}{\pgfqpoint{1.899857in}{1.611204in}}%
\pgfpathcurveto{\pgfqpoint{1.907670in}{1.619017in}}{\pgfqpoint{1.912061in}{1.629616in}}{\pgfqpoint{1.912061in}{1.640667in}}%
\pgfpathcurveto{\pgfqpoint{1.912061in}{1.651717in}}{\pgfqpoint{1.907670in}{1.662316in}}{\pgfqpoint{1.899857in}{1.670129in}}%
\pgfpathcurveto{\pgfqpoint{1.892043in}{1.677943in}}{\pgfqpoint{1.881444in}{1.682333in}}{\pgfqpoint{1.870394in}{1.682333in}}%
\pgfpathcurveto{\pgfqpoint{1.859344in}{1.682333in}}{\pgfqpoint{1.848745in}{1.677943in}}{\pgfqpoint{1.840931in}{1.670129in}}%
\pgfpathcurveto{\pgfqpoint{1.833118in}{1.662316in}}{\pgfqpoint{1.828727in}{1.651717in}}{\pgfqpoint{1.828727in}{1.640667in}}%
\pgfpathcurveto{\pgfqpoint{1.828727in}{1.629616in}}{\pgfqpoint{1.833118in}{1.619017in}}{\pgfqpoint{1.840931in}{1.611204in}}%
\pgfpathcurveto{\pgfqpoint{1.848745in}{1.603390in}}{\pgfqpoint{1.859344in}{1.599000in}}{\pgfqpoint{1.870394in}{1.599000in}}%
\pgfpathclose%
\pgfusepath{stroke,fill}%
\end{pgfscope}%
\begin{pgfscope}%
\pgfpathrectangle{\pgfqpoint{0.772069in}{0.515123in}}{\pgfqpoint{1.937500in}{1.347500in}}%
\pgfusepath{clip}%
\pgfsetbuttcap%
\pgfsetroundjoin%
\definecolor{currentfill}{rgb}{0.117647,0.564706,1.000000}%
\pgfsetfillcolor{currentfill}%
\pgfsetlinewidth{1.003750pt}%
\definecolor{currentstroke}{rgb}{0.117647,0.564706,1.000000}%
\pgfsetstrokecolor{currentstroke}%
\pgfsetdash{}{0pt}%
\pgfpathmoveto{\pgfqpoint{1.987189in}{1.729769in}}%
\pgfpathcurveto{\pgfqpoint{1.998239in}{1.729769in}}{\pgfqpoint{2.008838in}{1.734159in}}{\pgfqpoint{2.016652in}{1.741972in}}%
\pgfpathcurveto{\pgfqpoint{2.024466in}{1.749786in}}{\pgfqpoint{2.028856in}{1.760385in}}{\pgfqpoint{2.028856in}{1.771435in}}%
\pgfpathcurveto{\pgfqpoint{2.028856in}{1.782485in}}{\pgfqpoint{2.024466in}{1.793084in}}{\pgfqpoint{2.016652in}{1.800898in}}%
\pgfpathcurveto{\pgfqpoint{2.008838in}{1.808712in}}{\pgfqpoint{1.998239in}{1.813102in}}{\pgfqpoint{1.987189in}{1.813102in}}%
\pgfpathcurveto{\pgfqpoint{1.976139in}{1.813102in}}{\pgfqpoint{1.965540in}{1.808712in}}{\pgfqpoint{1.957726in}{1.800898in}}%
\pgfpathcurveto{\pgfqpoint{1.949913in}{1.793084in}}{\pgfqpoint{1.945523in}{1.782485in}}{\pgfqpoint{1.945523in}{1.771435in}}%
\pgfpathcurveto{\pgfqpoint{1.945523in}{1.760385in}}{\pgfqpoint{1.949913in}{1.749786in}}{\pgfqpoint{1.957726in}{1.741972in}}%
\pgfpathcurveto{\pgfqpoint{1.965540in}{1.734159in}}{\pgfqpoint{1.976139in}{1.729769in}}{\pgfqpoint{1.987189in}{1.729769in}}%
\pgfpathclose%
\pgfusepath{stroke,fill}%
\end{pgfscope}%
\begin{pgfscope}%
\pgfpathrectangle{\pgfqpoint{0.772069in}{0.515123in}}{\pgfqpoint{1.937500in}{1.347500in}}%
\pgfusepath{clip}%
\pgfsetbuttcap%
\pgfsetroundjoin%
\definecolor{currentfill}{rgb}{0.529412,0.807843,0.921569}%
\pgfsetfillcolor{currentfill}%
\pgfsetlinewidth{1.003750pt}%
\definecolor{currentstroke}{rgb}{0.529412,0.807843,0.921569}%
\pgfsetstrokecolor{currentstroke}%
\pgfsetdash{}{0pt}%
\pgfpathmoveto{\pgfqpoint{0.889314in}{0.564645in}}%
\pgfpathcurveto{\pgfqpoint{0.900364in}{0.564645in}}{\pgfqpoint{0.910963in}{0.569035in}}{\pgfqpoint{0.918777in}{0.576849in}}%
\pgfpathcurveto{\pgfqpoint{0.926590in}{0.584662in}}{\pgfqpoint{0.930981in}{0.595261in}}{\pgfqpoint{0.930981in}{0.606311in}}%
\pgfpathcurveto{\pgfqpoint{0.930981in}{0.617362in}}{\pgfqpoint{0.926590in}{0.627961in}}{\pgfqpoint{0.918777in}{0.635774in}}%
\pgfpathcurveto{\pgfqpoint{0.910963in}{0.643588in}}{\pgfqpoint{0.900364in}{0.647978in}}{\pgfqpoint{0.889314in}{0.647978in}}%
\pgfpathcurveto{\pgfqpoint{0.878264in}{0.647978in}}{\pgfqpoint{0.867665in}{0.643588in}}{\pgfqpoint{0.859851in}{0.635774in}}%
\pgfpathcurveto{\pgfqpoint{0.852037in}{0.627961in}}{\pgfqpoint{0.847647in}{0.617362in}}{\pgfqpoint{0.847647in}{0.606311in}}%
\pgfpathcurveto{\pgfqpoint{0.847647in}{0.595261in}}{\pgfqpoint{0.852037in}{0.584662in}}{\pgfqpoint{0.859851in}{0.576849in}}%
\pgfpathcurveto{\pgfqpoint{0.867665in}{0.569035in}}{\pgfqpoint{0.878264in}{0.564645in}}{\pgfqpoint{0.889314in}{0.564645in}}%
\pgfpathclose%
\pgfusepath{stroke,fill}%
\end{pgfscope}%
\begin{pgfscope}%
\pgfpathrectangle{\pgfqpoint{0.772069in}{0.515123in}}{\pgfqpoint{1.937500in}{1.347500in}}%
\pgfusepath{clip}%
\pgfsetbuttcap%
\pgfsetroundjoin%
\definecolor{currentfill}{rgb}{0.529412,0.807843,0.921569}%
\pgfsetfillcolor{currentfill}%
\pgfsetlinewidth{1.003750pt}%
\definecolor{currentstroke}{rgb}{0.529412,0.807843,0.921569}%
\pgfsetstrokecolor{currentstroke}%
\pgfsetdash{}{0pt}%
\pgfpathmoveto{\pgfqpoint{0.959391in}{0.688747in}}%
\pgfpathcurveto{\pgfqpoint{0.970441in}{0.688747in}}{\pgfqpoint{0.981040in}{0.693137in}}{\pgfqpoint{0.988854in}{0.700951in}}%
\pgfpathcurveto{\pgfqpoint{0.996667in}{0.708764in}}{\pgfqpoint{1.001058in}{0.719363in}}{\pgfqpoint{1.001058in}{0.730413in}}%
\pgfpathcurveto{\pgfqpoint{1.001058in}{0.741464in}}{\pgfqpoint{0.996667in}{0.752063in}}{\pgfqpoint{0.988854in}{0.759876in}}%
\pgfpathcurveto{\pgfqpoint{0.981040in}{0.767690in}}{\pgfqpoint{0.970441in}{0.772080in}}{\pgfqpoint{0.959391in}{0.772080in}}%
\pgfpathcurveto{\pgfqpoint{0.948341in}{0.772080in}}{\pgfqpoint{0.937742in}{0.767690in}}{\pgfqpoint{0.929928in}{0.759876in}}%
\pgfpathcurveto{\pgfqpoint{0.922115in}{0.752063in}}{\pgfqpoint{0.917724in}{0.741464in}}{\pgfqpoint{0.917724in}{0.730413in}}%
\pgfpathcurveto{\pgfqpoint{0.917724in}{0.719363in}}{\pgfqpoint{0.922115in}{0.708764in}}{\pgfqpoint{0.929928in}{0.700951in}}%
\pgfpathcurveto{\pgfqpoint{0.937742in}{0.693137in}}{\pgfqpoint{0.948341in}{0.688747in}}{\pgfqpoint{0.959391in}{0.688747in}}%
\pgfpathclose%
\pgfusepath{stroke,fill}%
\end{pgfscope}%
\begin{pgfscope}%
\pgfpathrectangle{\pgfqpoint{0.772069in}{0.515123in}}{\pgfqpoint{1.937500in}{1.347500in}}%
\pgfusepath{clip}%
\pgfsetbuttcap%
\pgfsetroundjoin%
\definecolor{currentfill}{rgb}{0.529412,0.807843,0.921569}%
\pgfsetfillcolor{currentfill}%
\pgfsetlinewidth{1.003750pt}%
\definecolor{currentstroke}{rgb}{0.529412,0.807843,0.921569}%
\pgfsetstrokecolor{currentstroke}%
\pgfsetdash{}{0pt}%
\pgfpathmoveto{\pgfqpoint{1.006109in}{0.822080in}}%
\pgfpathcurveto{\pgfqpoint{1.017159in}{0.822080in}}{\pgfqpoint{1.027758in}{0.826470in}}{\pgfqpoint{1.035572in}{0.834284in}}%
\pgfpathcurveto{\pgfqpoint{1.043386in}{0.842097in}}{\pgfqpoint{1.047776in}{0.852696in}}{\pgfqpoint{1.047776in}{0.863746in}}%
\pgfpathcurveto{\pgfqpoint{1.047776in}{0.874796in}}{\pgfqpoint{1.043386in}{0.885395in}}{\pgfqpoint{1.035572in}{0.893209in}}%
\pgfpathcurveto{\pgfqpoint{1.027758in}{0.901023in}}{\pgfqpoint{1.017159in}{0.905413in}}{\pgfqpoint{1.006109in}{0.905413in}}%
\pgfpathcurveto{\pgfqpoint{0.995059in}{0.905413in}}{\pgfqpoint{0.984460in}{0.901023in}}{\pgfqpoint{0.976646in}{0.893209in}}%
\pgfpathcurveto{\pgfqpoint{0.968833in}{0.885395in}}{\pgfqpoint{0.964442in}{0.874796in}}{\pgfqpoint{0.964442in}{0.863746in}}%
\pgfpathcurveto{\pgfqpoint{0.964442in}{0.852696in}}{\pgfqpoint{0.968833in}{0.842097in}}{\pgfqpoint{0.976646in}{0.834284in}}%
\pgfpathcurveto{\pgfqpoint{0.984460in}{0.826470in}}{\pgfqpoint{0.995059in}{0.822080in}}{\pgfqpoint{1.006109in}{0.822080in}}%
\pgfpathclose%
\pgfusepath{stroke,fill}%
\end{pgfscope}%
\begin{pgfscope}%
\pgfpathrectangle{\pgfqpoint{0.772069in}{0.515123in}}{\pgfqpoint{1.937500in}{1.347500in}}%
\pgfusepath{clip}%
\pgfsetbuttcap%
\pgfsetroundjoin%
\definecolor{currentfill}{rgb}{0.529412,0.807843,0.921569}%
\pgfsetfillcolor{currentfill}%
\pgfsetlinewidth{1.003750pt}%
\definecolor{currentstroke}{rgb}{0.529412,0.807843,0.921569}%
\pgfsetstrokecolor{currentstroke}%
\pgfsetdash{}{0pt}%
\pgfpathmoveto{\pgfqpoint{1.076186in}{0.952848in}}%
\pgfpathcurveto{\pgfqpoint{1.087236in}{0.952848in}}{\pgfqpoint{1.097835in}{0.957239in}}{\pgfqpoint{1.105649in}{0.965052in}}%
\pgfpathcurveto{\pgfqpoint{1.113463in}{0.972866in}}{\pgfqpoint{1.117853in}{0.983465in}}{\pgfqpoint{1.117853in}{0.994515in}}%
\pgfpathcurveto{\pgfqpoint{1.117853in}{1.005565in}}{\pgfqpoint{1.113463in}{1.016164in}}{\pgfqpoint{1.105649in}{1.023978in}}%
\pgfpathcurveto{\pgfqpoint{1.097835in}{1.031791in}}{\pgfqpoint{1.087236in}{1.036182in}}{\pgfqpoint{1.076186in}{1.036182in}}%
\pgfpathcurveto{\pgfqpoint{1.065136in}{1.036182in}}{\pgfqpoint{1.054537in}{1.031791in}}{\pgfqpoint{1.046723in}{1.023978in}}%
\pgfpathcurveto{\pgfqpoint{1.038910in}{1.016164in}}{\pgfqpoint{1.034520in}{1.005565in}}{\pgfqpoint{1.034520in}{0.994515in}}%
\pgfpathcurveto{\pgfqpoint{1.034520in}{0.983465in}}{\pgfqpoint{1.038910in}{0.972866in}}{\pgfqpoint{1.046723in}{0.965052in}}%
\pgfpathcurveto{\pgfqpoint{1.054537in}{0.957239in}}{\pgfqpoint{1.065136in}{0.952848in}}{\pgfqpoint{1.076186in}{0.952848in}}%
\pgfpathclose%
\pgfusepath{stroke,fill}%
\end{pgfscope}%
\begin{pgfscope}%
\pgfpathrectangle{\pgfqpoint{0.772069in}{0.515123in}}{\pgfqpoint{1.937500in}{1.347500in}}%
\pgfusepath{clip}%
\pgfsetbuttcap%
\pgfsetroundjoin%
\definecolor{currentfill}{rgb}{0.529412,0.807843,0.921569}%
\pgfsetfillcolor{currentfill}%
\pgfsetlinewidth{1.003750pt}%
\definecolor{currentstroke}{rgb}{0.529412,0.807843,0.921569}%
\pgfsetstrokecolor{currentstroke}%
\pgfsetdash{}{0pt}%
\pgfpathmoveto{\pgfqpoint{1.146263in}{1.083617in}}%
\pgfpathcurveto{\pgfqpoint{1.157314in}{1.083617in}}{\pgfqpoint{1.167913in}{1.088007in}}{\pgfqpoint{1.175726in}{1.095821in}}%
\pgfpathcurveto{\pgfqpoint{1.183540in}{1.103635in}}{\pgfqpoint{1.187930in}{1.114234in}}{\pgfqpoint{1.187930in}{1.125284in}}%
\pgfpathcurveto{\pgfqpoint{1.187930in}{1.136334in}}{\pgfqpoint{1.183540in}{1.146933in}}{\pgfqpoint{1.175726in}{1.154747in}}%
\pgfpathcurveto{\pgfqpoint{1.167913in}{1.162560in}}{\pgfqpoint{1.157314in}{1.166950in}}{\pgfqpoint{1.146263in}{1.166950in}}%
\pgfpathcurveto{\pgfqpoint{1.135213in}{1.166950in}}{\pgfqpoint{1.124614in}{1.162560in}}{\pgfqpoint{1.116801in}{1.154747in}}%
\pgfpathcurveto{\pgfqpoint{1.108987in}{1.146933in}}{\pgfqpoint{1.104597in}{1.136334in}}{\pgfqpoint{1.104597in}{1.125284in}}%
\pgfpathcurveto{\pgfqpoint{1.104597in}{1.114234in}}{\pgfqpoint{1.108987in}{1.103635in}}{\pgfqpoint{1.116801in}{1.095821in}}%
\pgfpathcurveto{\pgfqpoint{1.124614in}{1.088007in}}{\pgfqpoint{1.135213in}{1.083617in}}{\pgfqpoint{1.146263in}{1.083617in}}%
\pgfpathclose%
\pgfusepath{stroke,fill}%
\end{pgfscope}%
\begin{pgfscope}%
\pgfpathrectangle{\pgfqpoint{0.772069in}{0.515123in}}{\pgfqpoint{1.937500in}{1.347500in}}%
\pgfusepath{clip}%
\pgfsetbuttcap%
\pgfsetroundjoin%
\definecolor{currentfill}{rgb}{0.529412,0.807843,0.921569}%
\pgfsetfillcolor{currentfill}%
\pgfsetlinewidth{1.003750pt}%
\definecolor{currentstroke}{rgb}{0.529412,0.807843,0.921569}%
\pgfsetstrokecolor{currentstroke}%
\pgfsetdash{}{0pt}%
\pgfpathmoveto{\pgfqpoint{1.192982in}{1.207976in}}%
\pgfpathcurveto{\pgfqpoint{1.204032in}{1.207976in}}{\pgfqpoint{1.214631in}{1.212366in}}{\pgfqpoint{1.222444in}{1.220180in}}%
\pgfpathcurveto{\pgfqpoint{1.230258in}{1.227993in}}{\pgfqpoint{1.234648in}{1.238592in}}{\pgfqpoint{1.234648in}{1.249642in}}%
\pgfpathcurveto{\pgfqpoint{1.234648in}{1.260692in}}{\pgfqpoint{1.230258in}{1.271291in}}{\pgfqpoint{1.222444in}{1.279105in}}%
\pgfpathcurveto{\pgfqpoint{1.214631in}{1.286919in}}{\pgfqpoint{1.204032in}{1.291309in}}{\pgfqpoint{1.192982in}{1.291309in}}%
\pgfpathcurveto{\pgfqpoint{1.181931in}{1.291309in}}{\pgfqpoint{1.171332in}{1.286919in}}{\pgfqpoint{1.163519in}{1.279105in}}%
\pgfpathcurveto{\pgfqpoint{1.155705in}{1.271291in}}{\pgfqpoint{1.151315in}{1.260692in}}{\pgfqpoint{1.151315in}{1.249642in}}%
\pgfpathcurveto{\pgfqpoint{1.151315in}{1.238592in}}{\pgfqpoint{1.155705in}{1.227993in}}{\pgfqpoint{1.163519in}{1.220180in}}%
\pgfpathcurveto{\pgfqpoint{1.171332in}{1.212366in}}{\pgfqpoint{1.181931in}{1.207976in}}{\pgfqpoint{1.192982in}{1.207976in}}%
\pgfpathclose%
\pgfusepath{stroke,fill}%
\end{pgfscope}%
\begin{pgfscope}%
\pgfpathrectangle{\pgfqpoint{0.772069in}{0.515123in}}{\pgfqpoint{1.937500in}{1.347500in}}%
\pgfusepath{clip}%
\pgfsetbuttcap%
\pgfsetroundjoin%
\definecolor{currentfill}{rgb}{0.529412,0.807843,0.921569}%
\pgfsetfillcolor{currentfill}%
\pgfsetlinewidth{1.003750pt}%
\definecolor{currentstroke}{rgb}{0.529412,0.807843,0.921569}%
\pgfsetstrokecolor{currentstroke}%
\pgfsetdash{}{0pt}%
\pgfpathmoveto{\pgfqpoint{1.263059in}{1.342591in}}%
\pgfpathcurveto{\pgfqpoint{1.274109in}{1.342591in}}{\pgfqpoint{1.284708in}{1.346981in}}{\pgfqpoint{1.292521in}{1.354794in}}%
\pgfpathcurveto{\pgfqpoint{1.300335in}{1.362608in}}{\pgfqpoint{1.304725in}{1.373207in}}{\pgfqpoint{1.304725in}{1.384257in}}%
\pgfpathcurveto{\pgfqpoint{1.304725in}{1.395307in}}{\pgfqpoint{1.300335in}{1.405906in}}{\pgfqpoint{1.292521in}{1.413720in}}%
\pgfpathcurveto{\pgfqpoint{1.284708in}{1.421534in}}{\pgfqpoint{1.274109in}{1.425924in}}{\pgfqpoint{1.263059in}{1.425924in}}%
\pgfpathcurveto{\pgfqpoint{1.252009in}{1.425924in}}{\pgfqpoint{1.241410in}{1.421534in}}{\pgfqpoint{1.233596in}{1.413720in}}%
\pgfpathcurveto{\pgfqpoint{1.225782in}{1.405906in}}{\pgfqpoint{1.221392in}{1.395307in}}{\pgfqpoint{1.221392in}{1.384257in}}%
\pgfpathcurveto{\pgfqpoint{1.221392in}{1.373207in}}{\pgfqpoint{1.225782in}{1.362608in}}{\pgfqpoint{1.233596in}{1.354794in}}%
\pgfpathcurveto{\pgfqpoint{1.241410in}{1.346981in}}{\pgfqpoint{1.252009in}{1.342591in}}{\pgfqpoint{1.263059in}{1.342591in}}%
\pgfpathclose%
\pgfusepath{stroke,fill}%
\end{pgfscope}%
\begin{pgfscope}%
\pgfpathrectangle{\pgfqpoint{0.772069in}{0.515123in}}{\pgfqpoint{1.937500in}{1.347500in}}%
\pgfusepath{clip}%
\pgfsetbuttcap%
\pgfsetroundjoin%
\definecolor{currentfill}{rgb}{0.529412,0.807843,0.921569}%
\pgfsetfillcolor{currentfill}%
\pgfsetlinewidth{1.003750pt}%
\definecolor{currentstroke}{rgb}{0.529412,0.807843,0.921569}%
\pgfsetstrokecolor{currentstroke}%
\pgfsetdash{}{0pt}%
\pgfpathmoveto{\pgfqpoint{1.333136in}{1.472077in}}%
\pgfpathcurveto{\pgfqpoint{1.344186in}{1.472077in}}{\pgfqpoint{1.354785in}{1.476468in}}{\pgfqpoint{1.362599in}{1.484281in}}%
\pgfpathcurveto{\pgfqpoint{1.370412in}{1.492095in}}{\pgfqpoint{1.374802in}{1.502694in}}{\pgfqpoint{1.374802in}{1.513744in}}%
\pgfpathcurveto{\pgfqpoint{1.374802in}{1.524794in}}{\pgfqpoint{1.370412in}{1.535393in}}{\pgfqpoint{1.362599in}{1.543207in}}%
\pgfpathcurveto{\pgfqpoint{1.354785in}{1.551020in}}{\pgfqpoint{1.344186in}{1.555411in}}{\pgfqpoint{1.333136in}{1.555411in}}%
\pgfpathcurveto{\pgfqpoint{1.322086in}{1.555411in}}{\pgfqpoint{1.311487in}{1.551020in}}{\pgfqpoint{1.303673in}{1.543207in}}%
\pgfpathcurveto{\pgfqpoint{1.295859in}{1.535393in}}{\pgfqpoint{1.291469in}{1.524794in}}{\pgfqpoint{1.291469in}{1.513744in}}%
\pgfpathcurveto{\pgfqpoint{1.291469in}{1.502694in}}{\pgfqpoint{1.295859in}{1.492095in}}{\pgfqpoint{1.303673in}{1.484281in}}%
\pgfpathcurveto{\pgfqpoint{1.311487in}{1.476468in}}{\pgfqpoint{1.322086in}{1.472077in}}{\pgfqpoint{1.333136in}{1.472077in}}%
\pgfpathclose%
\pgfusepath{stroke,fill}%
\end{pgfscope}%
\begin{pgfscope}%
\pgfpathrectangle{\pgfqpoint{0.772069in}{0.515123in}}{\pgfqpoint{1.937500in}{1.347500in}}%
\pgfusepath{clip}%
\pgfsetbuttcap%
\pgfsetroundjoin%
\definecolor{currentfill}{rgb}{0.529412,0.807843,0.921569}%
\pgfsetfillcolor{currentfill}%
\pgfsetlinewidth{1.003750pt}%
\definecolor{currentstroke}{rgb}{0.529412,0.807843,0.921569}%
\pgfsetstrokecolor{currentstroke}%
\pgfsetdash{}{0pt}%
\pgfpathmoveto{\pgfqpoint{1.403213in}{1.599000in}}%
\pgfpathcurveto{\pgfqpoint{1.414263in}{1.599000in}}{\pgfqpoint{1.424862in}{1.603390in}}{\pgfqpoint{1.432676in}{1.611204in}}%
\pgfpathcurveto{\pgfqpoint{1.440489in}{1.619017in}}{\pgfqpoint{1.444880in}{1.629616in}}{\pgfqpoint{1.444880in}{1.640667in}}%
\pgfpathcurveto{\pgfqpoint{1.444880in}{1.651717in}}{\pgfqpoint{1.440489in}{1.662316in}}{\pgfqpoint{1.432676in}{1.670129in}}%
\pgfpathcurveto{\pgfqpoint{1.424862in}{1.677943in}}{\pgfqpoint{1.414263in}{1.682333in}}{\pgfqpoint{1.403213in}{1.682333in}}%
\pgfpathcurveto{\pgfqpoint{1.392163in}{1.682333in}}{\pgfqpoint{1.381564in}{1.677943in}}{\pgfqpoint{1.373750in}{1.670129in}}%
\pgfpathcurveto{\pgfqpoint{1.365937in}{1.662316in}}{\pgfqpoint{1.361546in}{1.651717in}}{\pgfqpoint{1.361546in}{1.640667in}}%
\pgfpathcurveto{\pgfqpoint{1.361546in}{1.629616in}}{\pgfqpoint{1.365937in}{1.619017in}}{\pgfqpoint{1.373750in}{1.611204in}}%
\pgfpathcurveto{\pgfqpoint{1.381564in}{1.603390in}}{\pgfqpoint{1.392163in}{1.599000in}}{\pgfqpoint{1.403213in}{1.599000in}}%
\pgfpathclose%
\pgfusepath{stroke,fill}%
\end{pgfscope}%
\begin{pgfscope}%
\pgfpathrectangle{\pgfqpoint{0.772069in}{0.515123in}}{\pgfqpoint{1.937500in}{1.347500in}}%
\pgfusepath{clip}%
\pgfsetbuttcap%
\pgfsetroundjoin%
\definecolor{currentfill}{rgb}{0.529412,0.807843,0.921569}%
\pgfsetfillcolor{currentfill}%
\pgfsetlinewidth{1.003750pt}%
\definecolor{currentstroke}{rgb}{0.529412,0.807843,0.921569}%
\pgfsetstrokecolor{currentstroke}%
\pgfsetdash{}{0pt}%
\pgfpathmoveto{\pgfqpoint{1.449931in}{1.729769in}}%
\pgfpathcurveto{\pgfqpoint{1.460981in}{1.729769in}}{\pgfqpoint{1.471580in}{1.734159in}}{\pgfqpoint{1.479394in}{1.741972in}}%
\pgfpathcurveto{\pgfqpoint{1.487207in}{1.749786in}}{\pgfqpoint{1.491598in}{1.760385in}}{\pgfqpoint{1.491598in}{1.771435in}}%
\pgfpathcurveto{\pgfqpoint{1.491598in}{1.782485in}}{\pgfqpoint{1.487207in}{1.793084in}}{\pgfqpoint{1.479394in}{1.800898in}}%
\pgfpathcurveto{\pgfqpoint{1.471580in}{1.808712in}}{\pgfqpoint{1.460981in}{1.813102in}}{\pgfqpoint{1.449931in}{1.813102in}}%
\pgfpathcurveto{\pgfqpoint{1.438881in}{1.813102in}}{\pgfqpoint{1.428282in}{1.808712in}}{\pgfqpoint{1.420468in}{1.800898in}}%
\pgfpathcurveto{\pgfqpoint{1.412655in}{1.793084in}}{\pgfqpoint{1.408264in}{1.782485in}}{\pgfqpoint{1.408264in}{1.771435in}}%
\pgfpathcurveto{\pgfqpoint{1.408264in}{1.760385in}}{\pgfqpoint{1.412655in}{1.749786in}}{\pgfqpoint{1.420468in}{1.741972in}}%
\pgfpathcurveto{\pgfqpoint{1.428282in}{1.734159in}}{\pgfqpoint{1.438881in}{1.729769in}}{\pgfqpoint{1.449931in}{1.729769in}}%
\pgfpathclose%
\pgfusepath{stroke,fill}%
\end{pgfscope}%
\begin{pgfscope}%
\pgfsetrectcap%
\pgfsetmiterjoin%
\pgfsetlinewidth{0.803000pt}%
\definecolor{currentstroke}{rgb}{0.000000,0.000000,0.000000}%
\pgfsetstrokecolor{currentstroke}%
\pgfsetdash{}{0pt}%
\pgfpathmoveto{\pgfqpoint{0.772069in}{0.515123in}}%
\pgfpathlineto{\pgfqpoint{0.772069in}{1.862623in}}%
\pgfusepath{stroke}%
\end{pgfscope}%
\begin{pgfscope}%
\pgfsetrectcap%
\pgfsetmiterjoin%
\pgfsetlinewidth{0.803000pt}%
\definecolor{currentstroke}{rgb}{0.000000,0.000000,0.000000}%
\pgfsetstrokecolor{currentstroke}%
\pgfsetdash{}{0pt}%
\pgfpathmoveto{\pgfqpoint{2.709569in}{0.515123in}}%
\pgfpathlineto{\pgfqpoint{2.709569in}{1.862623in}}%
\pgfusepath{stroke}%
\end{pgfscope}%
\begin{pgfscope}%
\pgfsetrectcap%
\pgfsetmiterjoin%
\pgfsetlinewidth{0.803000pt}%
\definecolor{currentstroke}{rgb}{0.000000,0.000000,0.000000}%
\pgfsetstrokecolor{currentstroke}%
\pgfsetdash{}{0pt}%
\pgfpathmoveto{\pgfqpoint{0.772069in}{0.515123in}}%
\pgfpathlineto{\pgfqpoint{2.709569in}{0.515123in}}%
\pgfusepath{stroke}%
\end{pgfscope}%
\begin{pgfscope}%
\pgfsetrectcap%
\pgfsetmiterjoin%
\pgfsetlinewidth{0.803000pt}%
\definecolor{currentstroke}{rgb}{0.000000,0.000000,0.000000}%
\pgfsetstrokecolor{currentstroke}%
\pgfsetdash{}{0pt}%
\pgfpathmoveto{\pgfqpoint{0.772069in}{1.862623in}}%
\pgfpathlineto{\pgfqpoint{2.709569in}{1.862623in}}%
\pgfusepath{stroke}%
\end{pgfscope}%
\end{pgfpicture}%
\makeatother%
\endgroup%

    %% Creator: Matplotlib, PGF backend
%%
%% To include the figure in your LaTeX document, write
%%   \input{<filename>.pgf}
%%
%% Make sure the required packages are loaded in your preamble
%%   \usepackage{pgf}
%%
%% Figures using additional raster images can only be included by \input if
%% they are in the same directory as the main LaTeX file. For loading figures
%% from other directories you can use the `import` package
%%   \usepackage{import}
%% and then include the figures with
%%   \import{<path to file>}{<filename>.pgf}
%%
%% Matplotlib used the following preamble
%%
\begingroup%
\makeatletter%
\begin{pgfpicture}%
\pgfpathrectangle{\pgfpointorigin}{\pgfqpoint{2.809569in}{1.962623in}}%
\pgfusepath{use as bounding box, clip}%
\begin{pgfscope}%
\pgfsetbuttcap%
\pgfsetmiterjoin%
\definecolor{currentfill}{rgb}{1.000000,1.000000,1.000000}%
\pgfsetfillcolor{currentfill}%
\pgfsetlinewidth{0.000000pt}%
\definecolor{currentstroke}{rgb}{1.000000,1.000000,1.000000}%
\pgfsetstrokecolor{currentstroke}%
\pgfsetdash{}{0pt}%
\pgfpathmoveto{\pgfqpoint{0.000000in}{0.000000in}}%
\pgfpathlineto{\pgfqpoint{2.809569in}{0.000000in}}%
\pgfpathlineto{\pgfqpoint{2.809569in}{1.962623in}}%
\pgfpathlineto{\pgfqpoint{0.000000in}{1.962623in}}%
\pgfpathclose%
\pgfusepath{fill}%
\end{pgfscope}%
\begin{pgfscope}%
\pgfsetbuttcap%
\pgfsetmiterjoin%
\definecolor{currentfill}{rgb}{1.000000,1.000000,1.000000}%
\pgfsetfillcolor{currentfill}%
\pgfsetlinewidth{0.000000pt}%
\definecolor{currentstroke}{rgb}{0.000000,0.000000,0.000000}%
\pgfsetstrokecolor{currentstroke}%
\pgfsetstrokeopacity{0.000000}%
\pgfsetdash{}{0pt}%
\pgfpathmoveto{\pgfqpoint{0.772069in}{0.515123in}}%
\pgfpathlineto{\pgfqpoint{2.709569in}{0.515123in}}%
\pgfpathlineto{\pgfqpoint{2.709569in}{1.862623in}}%
\pgfpathlineto{\pgfqpoint{0.772069in}{1.862623in}}%
\pgfpathclose%
\pgfusepath{fill}%
\end{pgfscope}%
\begin{pgfscope}%
\pgfsetbuttcap%
\pgfsetroundjoin%
\definecolor{currentfill}{rgb}{0.000000,0.000000,0.000000}%
\pgfsetfillcolor{currentfill}%
\pgfsetlinewidth{0.803000pt}%
\definecolor{currentstroke}{rgb}{0.000000,0.000000,0.000000}%
\pgfsetstrokecolor{currentstroke}%
\pgfsetdash{}{0pt}%
\pgfsys@defobject{currentmarker}{\pgfqpoint{0.000000in}{-0.048611in}}{\pgfqpoint{0.000000in}{0.000000in}}{%
\pgfpathmoveto{\pgfqpoint{0.000000in}{0.000000in}}%
\pgfpathlineto{\pgfqpoint{0.000000in}{-0.048611in}}%
\pgfusepath{stroke,fill}%
}%
\begin{pgfscope}%
\pgfsys@transformshift{0.844100in}{0.515123in}%
\pgfsys@useobject{currentmarker}{}%
\end{pgfscope}%
\end{pgfscope}%
\begin{pgfscope}%
\definecolor{textcolor}{rgb}{0.000000,0.000000,0.000000}%
\pgfsetstrokecolor{textcolor}%
\pgfsetfillcolor{textcolor}%
\pgftext[x=0.844100in,y=0.417901in,,top]{\color{textcolor}\rmfamily\fontsize{10.000000}{12.000000}\selectfont \(\displaystyle 0.0\)}%
\end{pgfscope}%
\begin{pgfscope}%
\pgfsetbuttcap%
\pgfsetroundjoin%
\definecolor{currentfill}{rgb}{0.000000,0.000000,0.000000}%
\pgfsetfillcolor{currentfill}%
\pgfsetlinewidth{0.803000pt}%
\definecolor{currentstroke}{rgb}{0.000000,0.000000,0.000000}%
\pgfsetstrokecolor{currentstroke}%
\pgfsetdash{}{0pt}%
\pgfsys@defobject{currentmarker}{\pgfqpoint{0.000000in}{-0.048611in}}{\pgfqpoint{0.000000in}{0.000000in}}{%
\pgfpathmoveto{\pgfqpoint{0.000000in}{0.000000in}}%
\pgfpathlineto{\pgfqpoint{0.000000in}{-0.048611in}}%
\pgfusepath{stroke,fill}%
}%
\begin{pgfscope}%
\pgfsys@transformshift{1.419559in}{0.515123in}%
\pgfsys@useobject{currentmarker}{}%
\end{pgfscope}%
\end{pgfscope}%
\begin{pgfscope}%
\definecolor{textcolor}{rgb}{0.000000,0.000000,0.000000}%
\pgfsetstrokecolor{textcolor}%
\pgfsetfillcolor{textcolor}%
\pgftext[x=1.419559in,y=0.417901in,,top]{\color{textcolor}\rmfamily\fontsize{10.000000}{12.000000}\selectfont \(\displaystyle 2.5\)}%
\end{pgfscope}%
\begin{pgfscope}%
\pgfsetbuttcap%
\pgfsetroundjoin%
\definecolor{currentfill}{rgb}{0.000000,0.000000,0.000000}%
\pgfsetfillcolor{currentfill}%
\pgfsetlinewidth{0.803000pt}%
\definecolor{currentstroke}{rgb}{0.000000,0.000000,0.000000}%
\pgfsetstrokecolor{currentstroke}%
\pgfsetdash{}{0pt}%
\pgfsys@defobject{currentmarker}{\pgfqpoint{0.000000in}{-0.048611in}}{\pgfqpoint{0.000000in}{0.000000in}}{%
\pgfpathmoveto{\pgfqpoint{0.000000in}{0.000000in}}%
\pgfpathlineto{\pgfqpoint{0.000000in}{-0.048611in}}%
\pgfusepath{stroke,fill}%
}%
\begin{pgfscope}%
\pgfsys@transformshift{1.995018in}{0.515123in}%
\pgfsys@useobject{currentmarker}{}%
\end{pgfscope}%
\end{pgfscope}%
\begin{pgfscope}%
\definecolor{textcolor}{rgb}{0.000000,0.000000,0.000000}%
\pgfsetstrokecolor{textcolor}%
\pgfsetfillcolor{textcolor}%
\pgftext[x=1.995018in,y=0.417901in,,top]{\color{textcolor}\rmfamily\fontsize{10.000000}{12.000000}\selectfont \(\displaystyle 5.0\)}%
\end{pgfscope}%
\begin{pgfscope}%
\pgfsetbuttcap%
\pgfsetroundjoin%
\definecolor{currentfill}{rgb}{0.000000,0.000000,0.000000}%
\pgfsetfillcolor{currentfill}%
\pgfsetlinewidth{0.803000pt}%
\definecolor{currentstroke}{rgb}{0.000000,0.000000,0.000000}%
\pgfsetstrokecolor{currentstroke}%
\pgfsetdash{}{0pt}%
\pgfsys@defobject{currentmarker}{\pgfqpoint{0.000000in}{-0.048611in}}{\pgfqpoint{0.000000in}{0.000000in}}{%
\pgfpathmoveto{\pgfqpoint{0.000000in}{0.000000in}}%
\pgfpathlineto{\pgfqpoint{0.000000in}{-0.048611in}}%
\pgfusepath{stroke,fill}%
}%
\begin{pgfscope}%
\pgfsys@transformshift{2.570477in}{0.515123in}%
\pgfsys@useobject{currentmarker}{}%
\end{pgfscope}%
\end{pgfscope}%
\begin{pgfscope}%
\definecolor{textcolor}{rgb}{0.000000,0.000000,0.000000}%
\pgfsetstrokecolor{textcolor}%
\pgfsetfillcolor{textcolor}%
\pgftext[x=2.570477in,y=0.417901in,,top]{\color{textcolor}\rmfamily\fontsize{10.000000}{12.000000}\selectfont \(\displaystyle 7.5\)}%
\end{pgfscope}%
\begin{pgfscope}%
\definecolor{textcolor}{rgb}{0.000000,0.000000,0.000000}%
\pgfsetstrokecolor{textcolor}%
\pgfsetfillcolor{textcolor}%
\pgftext[x=1.740819in,y=0.238889in,,top]{\color{textcolor}\rmfamily\fontsize{10.000000}{12.000000}\selectfont I(\(\displaystyle \mu\)A)}%
\end{pgfscope}%
\begin{pgfscope}%
\pgfsetbuttcap%
\pgfsetroundjoin%
\definecolor{currentfill}{rgb}{0.000000,0.000000,0.000000}%
\pgfsetfillcolor{currentfill}%
\pgfsetlinewidth{0.803000pt}%
\definecolor{currentstroke}{rgb}{0.000000,0.000000,0.000000}%
\pgfsetstrokecolor{currentstroke}%
\pgfsetdash{}{0pt}%
\pgfsys@defobject{currentmarker}{\pgfqpoint{-0.048611in}{0.000000in}}{\pgfqpoint{0.000000in}{0.000000in}}{%
\pgfpathmoveto{\pgfqpoint{0.000000in}{0.000000in}}%
\pgfpathlineto{\pgfqpoint{-0.048611in}{0.000000in}}%
\pgfusepath{stroke,fill}%
}%
\begin{pgfscope}%
\pgfsys@transformshift{0.772069in}{0.593664in}%
\pgfsys@useobject{currentmarker}{}%
\end{pgfscope}%
\end{pgfscope}%
\begin{pgfscope}%
\definecolor{textcolor}{rgb}{0.000000,0.000000,0.000000}%
\pgfsetstrokecolor{textcolor}%
\pgfsetfillcolor{textcolor}%
\pgftext[x=0.605402in,y=0.545439in,left,base]{\color{textcolor}\rmfamily\fontsize{10.000000}{12.000000}\selectfont \(\displaystyle 0\)}%
\end{pgfscope}%
\begin{pgfscope}%
\pgfsetbuttcap%
\pgfsetroundjoin%
\definecolor{currentfill}{rgb}{0.000000,0.000000,0.000000}%
\pgfsetfillcolor{currentfill}%
\pgfsetlinewidth{0.803000pt}%
\definecolor{currentstroke}{rgb}{0.000000,0.000000,0.000000}%
\pgfsetstrokecolor{currentstroke}%
\pgfsetdash{}{0pt}%
\pgfsys@defobject{currentmarker}{\pgfqpoint{-0.048611in}{0.000000in}}{\pgfqpoint{0.000000in}{0.000000in}}{%
\pgfpathmoveto{\pgfqpoint{0.000000in}{0.000000in}}%
\pgfpathlineto{\pgfqpoint{-0.048611in}{0.000000in}}%
\pgfusepath{stroke,fill}%
}%
\begin{pgfscope}%
\pgfsys@transformshift{0.772069in}{1.175545in}%
\pgfsys@useobject{currentmarker}{}%
\end{pgfscope}%
\end{pgfscope}%
\begin{pgfscope}%
\definecolor{textcolor}{rgb}{0.000000,0.000000,0.000000}%
\pgfsetstrokecolor{textcolor}%
\pgfsetfillcolor{textcolor}%
\pgftext[x=0.605402in,y=1.127320in,left,base]{\color{textcolor}\rmfamily\fontsize{10.000000}{12.000000}\selectfont \(\displaystyle 5\)}%
\end{pgfscope}%
\begin{pgfscope}%
\pgfsetbuttcap%
\pgfsetroundjoin%
\definecolor{currentfill}{rgb}{0.000000,0.000000,0.000000}%
\pgfsetfillcolor{currentfill}%
\pgfsetlinewidth{0.803000pt}%
\definecolor{currentstroke}{rgb}{0.000000,0.000000,0.000000}%
\pgfsetstrokecolor{currentstroke}%
\pgfsetdash{}{0pt}%
\pgfsys@defobject{currentmarker}{\pgfqpoint{-0.048611in}{0.000000in}}{\pgfqpoint{0.000000in}{0.000000in}}{%
\pgfpathmoveto{\pgfqpoint{0.000000in}{0.000000in}}%
\pgfpathlineto{\pgfqpoint{-0.048611in}{0.000000in}}%
\pgfusepath{stroke,fill}%
}%
\begin{pgfscope}%
\pgfsys@transformshift{0.772069in}{1.757426in}%
\pgfsys@useobject{currentmarker}{}%
\end{pgfscope}%
\end{pgfscope}%
\begin{pgfscope}%
\definecolor{textcolor}{rgb}{0.000000,0.000000,0.000000}%
\pgfsetstrokecolor{textcolor}%
\pgfsetfillcolor{textcolor}%
\pgftext[x=0.535957in,y=1.709200in,left,base]{\color{textcolor}\rmfamily\fontsize{10.000000}{12.000000}\selectfont \(\displaystyle 10\)}%
\end{pgfscope}%
\begin{pgfscope}%
\definecolor{textcolor}{rgb}{0.000000,0.000000,0.000000}%
\pgfsetstrokecolor{textcolor}%
\pgfsetfillcolor{textcolor}%
\pgftext[x=0.258179in,y=1.188873in,,bottom]{\color{textcolor}\rmfamily\fontsize{10.000000}{12.000000}\selectfont V(V)}%
\end{pgfscope}%
\begin{pgfscope}%
\pgfpathrectangle{\pgfqpoint{0.772069in}{0.515123in}}{\pgfqpoint{1.937500in}{1.347500in}}%
\pgfusepath{clip}%
\pgfsetbuttcap%
\pgfsetroundjoin%
\definecolor{currentfill}{rgb}{0.121569,0.466667,0.705882}%
\pgfsetfillcolor{currentfill}%
\pgfsetlinewidth{1.003750pt}%
\definecolor{currentstroke}{rgb}{0.121569,0.466667,0.705882}%
\pgfsetstrokecolor{currentstroke}%
\pgfsetdash{}{0pt}%
\pgfpathmoveto{\pgfqpoint{1.005229in}{0.672098in}}%
\pgfpathcurveto{\pgfqpoint{1.016279in}{0.672098in}}{\pgfqpoint{1.026878in}{0.676488in}}{\pgfqpoint{1.034691in}{0.684302in}}%
\pgfpathcurveto{\pgfqpoint{1.042505in}{0.692115in}}{\pgfqpoint{1.046895in}{0.702714in}}{\pgfqpoint{1.046895in}{0.713764in}}%
\pgfpathcurveto{\pgfqpoint{1.046895in}{0.724814in}}{\pgfqpoint{1.042505in}{0.735414in}}{\pgfqpoint{1.034691in}{0.743227in}}%
\pgfpathcurveto{\pgfqpoint{1.026878in}{0.751041in}}{\pgfqpoint{1.016279in}{0.755431in}}{\pgfqpoint{1.005229in}{0.755431in}}%
\pgfpathcurveto{\pgfqpoint{0.994179in}{0.755431in}}{\pgfqpoint{0.983580in}{0.751041in}}{\pgfqpoint{0.975766in}{0.743227in}}%
\pgfpathcurveto{\pgfqpoint{0.967952in}{0.735414in}}{\pgfqpoint{0.963562in}{0.724814in}}{\pgfqpoint{0.963562in}{0.713764in}}%
\pgfpathcurveto{\pgfqpoint{0.963562in}{0.702714in}}{\pgfqpoint{0.967952in}{0.692115in}}{\pgfqpoint{0.975766in}{0.684302in}}%
\pgfpathcurveto{\pgfqpoint{0.983580in}{0.676488in}}{\pgfqpoint{0.994179in}{0.672098in}}{\pgfqpoint{1.005229in}{0.672098in}}%
\pgfpathclose%
\pgfusepath{stroke,fill}%
\end{pgfscope}%
\begin{pgfscope}%
\pgfpathrectangle{\pgfqpoint{0.772069in}{0.515123in}}{\pgfqpoint{1.937500in}{1.347500in}}%
\pgfusepath{clip}%
\pgfsetbuttcap%
\pgfsetroundjoin%
\definecolor{currentfill}{rgb}{0.121569,0.466667,0.705882}%
\pgfsetfillcolor{currentfill}%
\pgfsetlinewidth{1.003750pt}%
\definecolor{currentstroke}{rgb}{0.121569,0.466667,0.705882}%
\pgfsetstrokecolor{currentstroke}%
\pgfsetdash{}{0pt}%
\pgfpathmoveto{\pgfqpoint{1.189376in}{0.784750in}}%
\pgfpathcurveto{\pgfqpoint{1.200426in}{0.784750in}}{\pgfqpoint{1.211025in}{0.789140in}}{\pgfqpoint{1.218838in}{0.796954in}}%
\pgfpathcurveto{\pgfqpoint{1.226652in}{0.804767in}}{\pgfqpoint{1.231042in}{0.815366in}}{\pgfqpoint{1.231042in}{0.826416in}}%
\pgfpathcurveto{\pgfqpoint{1.231042in}{0.837467in}}{\pgfqpoint{1.226652in}{0.848066in}}{\pgfqpoint{1.218838in}{0.855879in}}%
\pgfpathcurveto{\pgfqpoint{1.211025in}{0.863693in}}{\pgfqpoint{1.200426in}{0.868083in}}{\pgfqpoint{1.189376in}{0.868083in}}%
\pgfpathcurveto{\pgfqpoint{1.178325in}{0.868083in}}{\pgfqpoint{1.167726in}{0.863693in}}{\pgfqpoint{1.159913in}{0.855879in}}%
\pgfpathcurveto{\pgfqpoint{1.152099in}{0.848066in}}{\pgfqpoint{1.147709in}{0.837467in}}{\pgfqpoint{1.147709in}{0.826416in}}%
\pgfpathcurveto{\pgfqpoint{1.147709in}{0.815366in}}{\pgfqpoint{1.152099in}{0.804767in}}{\pgfqpoint{1.159913in}{0.796954in}}%
\pgfpathcurveto{\pgfqpoint{1.167726in}{0.789140in}}{\pgfqpoint{1.178325in}{0.784750in}}{\pgfqpoint{1.189376in}{0.784750in}}%
\pgfpathclose%
\pgfusepath{stroke,fill}%
\end{pgfscope}%
\begin{pgfscope}%
\pgfpathrectangle{\pgfqpoint{0.772069in}{0.515123in}}{\pgfqpoint{1.937500in}{1.347500in}}%
\pgfusepath{clip}%
\pgfsetbuttcap%
\pgfsetroundjoin%
\definecolor{currentfill}{rgb}{0.121569,0.466667,0.705882}%
\pgfsetfillcolor{currentfill}%
\pgfsetlinewidth{1.003750pt}%
\definecolor{currentstroke}{rgb}{0.121569,0.466667,0.705882}%
\pgfsetstrokecolor{currentstroke}%
\pgfsetdash{}{0pt}%
\pgfpathmoveto{\pgfqpoint{1.350504in}{0.905781in}}%
\pgfpathcurveto{\pgfqpoint{1.361554in}{0.905781in}}{\pgfqpoint{1.372153in}{0.910171in}}{\pgfqpoint{1.379967in}{0.917985in}}%
\pgfpathcurveto{\pgfqpoint{1.387781in}{0.925798in}}{\pgfqpoint{1.392171in}{0.936398in}}{\pgfqpoint{1.392171in}{0.947448in}}%
\pgfpathcurveto{\pgfqpoint{1.392171in}{0.958498in}}{\pgfqpoint{1.387781in}{0.969097in}}{\pgfqpoint{1.379967in}{0.976910in}}%
\pgfpathcurveto{\pgfqpoint{1.372153in}{0.984724in}}{\pgfqpoint{1.361554in}{0.989114in}}{\pgfqpoint{1.350504in}{0.989114in}}%
\pgfpathcurveto{\pgfqpoint{1.339454in}{0.989114in}}{\pgfqpoint{1.328855in}{0.984724in}}{\pgfqpoint{1.321041in}{0.976910in}}%
\pgfpathcurveto{\pgfqpoint{1.313228in}{0.969097in}}{\pgfqpoint{1.308837in}{0.958498in}}{\pgfqpoint{1.308837in}{0.947448in}}%
\pgfpathcurveto{\pgfqpoint{1.308837in}{0.936398in}}{\pgfqpoint{1.313228in}{0.925798in}}{\pgfqpoint{1.321041in}{0.917985in}}%
\pgfpathcurveto{\pgfqpoint{1.328855in}{0.910171in}}{\pgfqpoint{1.339454in}{0.905781in}}{\pgfqpoint{1.350504in}{0.905781in}}%
\pgfpathclose%
\pgfusepath{stroke,fill}%
\end{pgfscope}%
\begin{pgfscope}%
\pgfpathrectangle{\pgfqpoint{0.772069in}{0.515123in}}{\pgfqpoint{1.937500in}{1.347500in}}%
\pgfusepath{clip}%
\pgfsetbuttcap%
\pgfsetroundjoin%
\definecolor{currentfill}{rgb}{0.121569,0.466667,0.705882}%
\pgfsetfillcolor{currentfill}%
\pgfsetlinewidth{1.003750pt}%
\definecolor{currentstroke}{rgb}{0.121569,0.466667,0.705882}%
\pgfsetstrokecolor{currentstroke}%
\pgfsetdash{}{0pt}%
\pgfpathmoveto{\pgfqpoint{1.534651in}{1.024485in}}%
\pgfpathcurveto{\pgfqpoint{1.545701in}{1.024485in}}{\pgfqpoint{1.556300in}{1.028875in}}{\pgfqpoint{1.564114in}{1.036689in}}%
\pgfpathcurveto{\pgfqpoint{1.571927in}{1.044502in}}{\pgfqpoint{1.576318in}{1.055101in}}{\pgfqpoint{1.576318in}{1.066151in}}%
\pgfpathcurveto{\pgfqpoint{1.576318in}{1.077201in}}{\pgfqpoint{1.571927in}{1.087800in}}{\pgfqpoint{1.564114in}{1.095614in}}%
\pgfpathcurveto{\pgfqpoint{1.556300in}{1.103428in}}{\pgfqpoint{1.545701in}{1.107818in}}{\pgfqpoint{1.534651in}{1.107818in}}%
\pgfpathcurveto{\pgfqpoint{1.523601in}{1.107818in}}{\pgfqpoint{1.513002in}{1.103428in}}{\pgfqpoint{1.505188in}{1.095614in}}%
\pgfpathcurveto{\pgfqpoint{1.497375in}{1.087800in}}{\pgfqpoint{1.492984in}{1.077201in}}{\pgfqpoint{1.492984in}{1.066151in}}%
\pgfpathcurveto{\pgfqpoint{1.492984in}{1.055101in}}{\pgfqpoint{1.497375in}{1.044502in}}{\pgfqpoint{1.505188in}{1.036689in}}%
\pgfpathcurveto{\pgfqpoint{1.513002in}{1.028875in}}{\pgfqpoint{1.523601in}{1.024485in}}{\pgfqpoint{1.534651in}{1.024485in}}%
\pgfpathclose%
\pgfusepath{stroke,fill}%
\end{pgfscope}%
\begin{pgfscope}%
\pgfpathrectangle{\pgfqpoint{0.772069in}{0.515123in}}{\pgfqpoint{1.937500in}{1.347500in}}%
\pgfusepath{clip}%
\pgfsetbuttcap%
\pgfsetroundjoin%
\definecolor{currentfill}{rgb}{0.121569,0.466667,0.705882}%
\pgfsetfillcolor{currentfill}%
\pgfsetlinewidth{1.003750pt}%
\definecolor{currentstroke}{rgb}{0.121569,0.466667,0.705882}%
\pgfsetstrokecolor{currentstroke}%
\pgfsetdash{}{0pt}%
\pgfpathmoveto{\pgfqpoint{1.718798in}{1.143188in}}%
\pgfpathcurveto{\pgfqpoint{1.729848in}{1.143188in}}{\pgfqpoint{1.740447in}{1.147579in}}{\pgfqpoint{1.748261in}{1.155392in}}%
\pgfpathcurveto{\pgfqpoint{1.756074in}{1.163206in}}{\pgfqpoint{1.760465in}{1.173805in}}{\pgfqpoint{1.760465in}{1.184855in}}%
\pgfpathcurveto{\pgfqpoint{1.760465in}{1.195905in}}{\pgfqpoint{1.756074in}{1.206504in}}{\pgfqpoint{1.748261in}{1.214318in}}%
\pgfpathcurveto{\pgfqpoint{1.740447in}{1.222131in}}{\pgfqpoint{1.729848in}{1.226522in}}{\pgfqpoint{1.718798in}{1.226522in}}%
\pgfpathcurveto{\pgfqpoint{1.707748in}{1.226522in}}{\pgfqpoint{1.697149in}{1.222131in}}{\pgfqpoint{1.689335in}{1.214318in}}%
\pgfpathcurveto{\pgfqpoint{1.681522in}{1.206504in}}{\pgfqpoint{1.677131in}{1.195905in}}{\pgfqpoint{1.677131in}{1.184855in}}%
\pgfpathcurveto{\pgfqpoint{1.677131in}{1.173805in}}{\pgfqpoint{1.681522in}{1.163206in}}{\pgfqpoint{1.689335in}{1.155392in}}%
\pgfpathcurveto{\pgfqpoint{1.697149in}{1.147579in}}{\pgfqpoint{1.707748in}{1.143188in}}{\pgfqpoint{1.718798in}{1.143188in}}%
\pgfpathclose%
\pgfusepath{stroke,fill}%
\end{pgfscope}%
\begin{pgfscope}%
\pgfpathrectangle{\pgfqpoint{0.772069in}{0.515123in}}{\pgfqpoint{1.937500in}{1.347500in}}%
\pgfusepath{clip}%
\pgfsetbuttcap%
\pgfsetroundjoin%
\definecolor{currentfill}{rgb}{0.121569,0.466667,0.705882}%
\pgfsetfillcolor{currentfill}%
\pgfsetlinewidth{1.003750pt}%
\definecolor{currentstroke}{rgb}{0.121569,0.466667,0.705882}%
\pgfsetstrokecolor{currentstroke}%
\pgfsetdash{}{0pt}%
\pgfpathmoveto{\pgfqpoint{1.879926in}{1.256073in}}%
\pgfpathcurveto{\pgfqpoint{1.890977in}{1.256073in}}{\pgfqpoint{1.901576in}{1.260463in}}{\pgfqpoint{1.909389in}{1.268277in}}%
\pgfpathcurveto{\pgfqpoint{1.917203in}{1.276091in}}{\pgfqpoint{1.921593in}{1.286690in}}{\pgfqpoint{1.921593in}{1.297740in}}%
\pgfpathcurveto{\pgfqpoint{1.921593in}{1.308790in}}{\pgfqpoint{1.917203in}{1.319389in}}{\pgfqpoint{1.909389in}{1.327203in}}%
\pgfpathcurveto{\pgfqpoint{1.901576in}{1.335016in}}{\pgfqpoint{1.890977in}{1.339406in}}{\pgfqpoint{1.879926in}{1.339406in}}%
\pgfpathcurveto{\pgfqpoint{1.868876in}{1.339406in}}{\pgfqpoint{1.858277in}{1.335016in}}{\pgfqpoint{1.850464in}{1.327203in}}%
\pgfpathcurveto{\pgfqpoint{1.842650in}{1.319389in}}{\pgfqpoint{1.838260in}{1.308790in}}{\pgfqpoint{1.838260in}{1.297740in}}%
\pgfpathcurveto{\pgfqpoint{1.838260in}{1.286690in}}{\pgfqpoint{1.842650in}{1.276091in}}{\pgfqpoint{1.850464in}{1.268277in}}%
\pgfpathcurveto{\pgfqpoint{1.858277in}{1.260463in}}{\pgfqpoint{1.868876in}{1.256073in}}{\pgfqpoint{1.879926in}{1.256073in}}%
\pgfpathclose%
\pgfusepath{stroke,fill}%
\end{pgfscope}%
\begin{pgfscope}%
\pgfpathrectangle{\pgfqpoint{0.772069in}{0.515123in}}{\pgfqpoint{1.937500in}{1.347500in}}%
\pgfusepath{clip}%
\pgfsetbuttcap%
\pgfsetroundjoin%
\definecolor{currentfill}{rgb}{0.121569,0.466667,0.705882}%
\pgfsetfillcolor{currentfill}%
\pgfsetlinewidth{1.003750pt}%
\definecolor{currentstroke}{rgb}{0.121569,0.466667,0.705882}%
\pgfsetstrokecolor{currentstroke}%
\pgfsetdash{}{0pt}%
\pgfpathmoveto{\pgfqpoint{2.064073in}{1.378268in}}%
\pgfpathcurveto{\pgfqpoint{2.075123in}{1.378268in}}{\pgfqpoint{2.085723in}{1.382658in}}{\pgfqpoint{2.093536in}{1.390472in}}%
\pgfpathcurveto{\pgfqpoint{2.101350in}{1.398286in}}{\pgfqpoint{2.105740in}{1.408885in}}{\pgfqpoint{2.105740in}{1.419935in}}%
\pgfpathcurveto{\pgfqpoint{2.105740in}{1.430985in}}{\pgfqpoint{2.101350in}{1.441584in}}{\pgfqpoint{2.093536in}{1.449398in}}%
\pgfpathcurveto{\pgfqpoint{2.085723in}{1.457211in}}{\pgfqpoint{2.075123in}{1.461601in}}{\pgfqpoint{2.064073in}{1.461601in}}%
\pgfpathcurveto{\pgfqpoint{2.053023in}{1.461601in}}{\pgfqpoint{2.042424in}{1.457211in}}{\pgfqpoint{2.034611in}{1.449398in}}%
\pgfpathcurveto{\pgfqpoint{2.026797in}{1.441584in}}{\pgfqpoint{2.022407in}{1.430985in}}{\pgfqpoint{2.022407in}{1.419935in}}%
\pgfpathcurveto{\pgfqpoint{2.022407in}{1.408885in}}{\pgfqpoint{2.026797in}{1.398286in}}{\pgfqpoint{2.034611in}{1.390472in}}%
\pgfpathcurveto{\pgfqpoint{2.042424in}{1.382658in}}{\pgfqpoint{2.053023in}{1.378268in}}{\pgfqpoint{2.064073in}{1.378268in}}%
\pgfpathclose%
\pgfusepath{stroke,fill}%
\end{pgfscope}%
\begin{pgfscope}%
\pgfpathrectangle{\pgfqpoint{0.772069in}{0.515123in}}{\pgfqpoint{1.937500in}{1.347500in}}%
\pgfusepath{clip}%
\pgfsetbuttcap%
\pgfsetroundjoin%
\definecolor{currentfill}{rgb}{0.121569,0.466667,0.705882}%
\pgfsetfillcolor{currentfill}%
\pgfsetlinewidth{1.003750pt}%
\definecolor{currentstroke}{rgb}{0.121569,0.466667,0.705882}%
\pgfsetstrokecolor{currentstroke}%
\pgfsetdash{}{0pt}%
\pgfpathmoveto{\pgfqpoint{2.248220in}{1.495808in}}%
\pgfpathcurveto{\pgfqpoint{2.259270in}{1.495808in}}{\pgfqpoint{2.269869in}{1.500198in}}{\pgfqpoint{2.277683in}{1.508012in}}%
\pgfpathcurveto{\pgfqpoint{2.285497in}{1.515826in}}{\pgfqpoint{2.289887in}{1.526425in}}{\pgfqpoint{2.289887in}{1.537475in}}%
\pgfpathcurveto{\pgfqpoint{2.289887in}{1.548525in}}{\pgfqpoint{2.285497in}{1.559124in}}{\pgfqpoint{2.277683in}{1.566937in}}%
\pgfpathcurveto{\pgfqpoint{2.269869in}{1.574751in}}{\pgfqpoint{2.259270in}{1.579141in}}{\pgfqpoint{2.248220in}{1.579141in}}%
\pgfpathcurveto{\pgfqpoint{2.237170in}{1.579141in}}{\pgfqpoint{2.226571in}{1.574751in}}{\pgfqpoint{2.218757in}{1.566937in}}%
\pgfpathcurveto{\pgfqpoint{2.210944in}{1.559124in}}{\pgfqpoint{2.206554in}{1.548525in}}{\pgfqpoint{2.206554in}{1.537475in}}%
\pgfpathcurveto{\pgfqpoint{2.206554in}{1.526425in}}{\pgfqpoint{2.210944in}{1.515826in}}{\pgfqpoint{2.218757in}{1.508012in}}%
\pgfpathcurveto{\pgfqpoint{2.226571in}{1.500198in}}{\pgfqpoint{2.237170in}{1.495808in}}{\pgfqpoint{2.248220in}{1.495808in}}%
\pgfpathclose%
\pgfusepath{stroke,fill}%
\end{pgfscope}%
\begin{pgfscope}%
\pgfpathrectangle{\pgfqpoint{0.772069in}{0.515123in}}{\pgfqpoint{1.937500in}{1.347500in}}%
\pgfusepath{clip}%
\pgfsetbuttcap%
\pgfsetroundjoin%
\definecolor{currentfill}{rgb}{0.121569,0.466667,0.705882}%
\pgfsetfillcolor{currentfill}%
\pgfsetlinewidth{1.003750pt}%
\definecolor{currentstroke}{rgb}{0.121569,0.466667,0.705882}%
\pgfsetstrokecolor{currentstroke}%
\pgfsetdash{}{0pt}%
\pgfpathmoveto{\pgfqpoint{2.409349in}{1.611020in}}%
\pgfpathcurveto{\pgfqpoint{2.420399in}{1.611020in}}{\pgfqpoint{2.430998in}{1.615411in}}{\pgfqpoint{2.438812in}{1.623224in}}%
\pgfpathcurveto{\pgfqpoint{2.446625in}{1.631038in}}{\pgfqpoint{2.451015in}{1.641637in}}{\pgfqpoint{2.451015in}{1.652687in}}%
\pgfpathcurveto{\pgfqpoint{2.451015in}{1.663737in}}{\pgfqpoint{2.446625in}{1.674336in}}{\pgfqpoint{2.438812in}{1.682150in}}%
\pgfpathcurveto{\pgfqpoint{2.430998in}{1.689963in}}{\pgfqpoint{2.420399in}{1.694354in}}{\pgfqpoint{2.409349in}{1.694354in}}%
\pgfpathcurveto{\pgfqpoint{2.398299in}{1.694354in}}{\pgfqpoint{2.387700in}{1.689963in}}{\pgfqpoint{2.379886in}{1.682150in}}%
\pgfpathcurveto{\pgfqpoint{2.372072in}{1.674336in}}{\pgfqpoint{2.367682in}{1.663737in}}{\pgfqpoint{2.367682in}{1.652687in}}%
\pgfpathcurveto{\pgfqpoint{2.367682in}{1.641637in}}{\pgfqpoint{2.372072in}{1.631038in}}{\pgfqpoint{2.379886in}{1.623224in}}%
\pgfpathcurveto{\pgfqpoint{2.387700in}{1.615411in}}{\pgfqpoint{2.398299in}{1.611020in}}{\pgfqpoint{2.409349in}{1.611020in}}%
\pgfpathclose%
\pgfusepath{stroke,fill}%
\end{pgfscope}%
\begin{pgfscope}%
\pgfpathrectangle{\pgfqpoint{0.772069in}{0.515123in}}{\pgfqpoint{1.937500in}{1.347500in}}%
\pgfusepath{clip}%
\pgfsetbuttcap%
\pgfsetroundjoin%
\definecolor{currentfill}{rgb}{0.121569,0.466667,0.705882}%
\pgfsetfillcolor{currentfill}%
\pgfsetlinewidth{1.003750pt}%
\definecolor{currentstroke}{rgb}{0.121569,0.466667,0.705882}%
\pgfsetstrokecolor{currentstroke}%
\pgfsetdash{}{0pt}%
\pgfpathmoveto{\pgfqpoint{2.593496in}{1.729724in}}%
\pgfpathcurveto{\pgfqpoint{2.604546in}{1.729724in}}{\pgfqpoint{2.615145in}{1.734114in}}{\pgfqpoint{2.622958in}{1.741928in}}%
\pgfpathcurveto{\pgfqpoint{2.630772in}{1.749742in}}{\pgfqpoint{2.635162in}{1.760341in}}{\pgfqpoint{2.635162in}{1.771391in}}%
\pgfpathcurveto{\pgfqpoint{2.635162in}{1.782441in}}{\pgfqpoint{2.630772in}{1.793040in}}{\pgfqpoint{2.622958in}{1.800854in}}%
\pgfpathcurveto{\pgfqpoint{2.615145in}{1.808667in}}{\pgfqpoint{2.604546in}{1.813057in}}{\pgfqpoint{2.593496in}{1.813057in}}%
\pgfpathcurveto{\pgfqpoint{2.582446in}{1.813057in}}{\pgfqpoint{2.571847in}{1.808667in}}{\pgfqpoint{2.564033in}{1.800854in}}%
\pgfpathcurveto{\pgfqpoint{2.556219in}{1.793040in}}{\pgfqpoint{2.551829in}{1.782441in}}{\pgfqpoint{2.551829in}{1.771391in}}%
\pgfpathcurveto{\pgfqpoint{2.551829in}{1.760341in}}{\pgfqpoint{2.556219in}{1.749742in}}{\pgfqpoint{2.564033in}{1.741928in}}%
\pgfpathcurveto{\pgfqpoint{2.571847in}{1.734114in}}{\pgfqpoint{2.582446in}{1.729724in}}{\pgfqpoint{2.593496in}{1.729724in}}%
\pgfpathclose%
\pgfusepath{stroke,fill}%
\end{pgfscope}%
\begin{pgfscope}%
\pgfpathrectangle{\pgfqpoint{0.772069in}{0.515123in}}{\pgfqpoint{1.937500in}{1.347500in}}%
\pgfusepath{clip}%
\pgfsetbuttcap%
\pgfsetroundjoin%
\definecolor{currentfill}{rgb}{1.000000,0.388235,0.278431}%
\pgfsetfillcolor{currentfill}%
\pgfsetlinewidth{1.003750pt}%
\definecolor{currentstroke}{rgb}{1.000000,0.388235,0.278431}%
\pgfsetstrokecolor{currentstroke}%
\pgfsetdash{}{0pt}%
\pgfpathmoveto{\pgfqpoint{1.005229in}{0.567906in}}%
\pgfpathcurveto{\pgfqpoint{1.016279in}{0.567906in}}{\pgfqpoint{1.026878in}{0.572296in}}{\pgfqpoint{1.034691in}{0.580110in}}%
\pgfpathcurveto{\pgfqpoint{1.042505in}{0.587924in}}{\pgfqpoint{1.046895in}{0.598523in}}{\pgfqpoint{1.046895in}{0.609573in}}%
\pgfpathcurveto{\pgfqpoint{1.046895in}{0.620623in}}{\pgfqpoint{1.042505in}{0.631222in}}{\pgfqpoint{1.034691in}{0.639036in}}%
\pgfpathcurveto{\pgfqpoint{1.026878in}{0.646849in}}{\pgfqpoint{1.016279in}{0.651239in}}{\pgfqpoint{1.005229in}{0.651239in}}%
\pgfpathcurveto{\pgfqpoint{0.994179in}{0.651239in}}{\pgfqpoint{0.983580in}{0.646849in}}{\pgfqpoint{0.975766in}{0.639036in}}%
\pgfpathcurveto{\pgfqpoint{0.967952in}{0.631222in}}{\pgfqpoint{0.963562in}{0.620623in}}{\pgfqpoint{0.963562in}{0.609573in}}%
\pgfpathcurveto{\pgfqpoint{0.963562in}{0.598523in}}{\pgfqpoint{0.967952in}{0.587924in}}{\pgfqpoint{0.975766in}{0.580110in}}%
\pgfpathcurveto{\pgfqpoint{0.983580in}{0.572296in}}{\pgfqpoint{0.994179in}{0.567906in}}{\pgfqpoint{1.005229in}{0.567906in}}%
\pgfpathclose%
\pgfusepath{stroke,fill}%
\end{pgfscope}%
\begin{pgfscope}%
\pgfpathrectangle{\pgfqpoint{0.772069in}{0.515123in}}{\pgfqpoint{1.937500in}{1.347500in}}%
\pgfusepath{clip}%
\pgfsetbuttcap%
\pgfsetroundjoin%
\definecolor{currentfill}{rgb}{1.000000,0.388235,0.278431}%
\pgfsetfillcolor{currentfill}%
\pgfsetlinewidth{1.003750pt}%
\definecolor{currentstroke}{rgb}{1.000000,0.388235,0.278431}%
\pgfsetstrokecolor{currentstroke}%
\pgfsetdash{}{0pt}%
\pgfpathmoveto{\pgfqpoint{1.189376in}{0.582372in}}%
\pgfpathcurveto{\pgfqpoint{1.200426in}{0.582372in}}{\pgfqpoint{1.211025in}{0.586762in}}{\pgfqpoint{1.218838in}{0.594576in}}%
\pgfpathcurveto{\pgfqpoint{1.226652in}{0.602389in}}{\pgfqpoint{1.231042in}{0.612988in}}{\pgfqpoint{1.231042in}{0.624038in}}%
\pgfpathcurveto{\pgfqpoint{1.231042in}{0.635088in}}{\pgfqpoint{1.226652in}{0.645688in}}{\pgfqpoint{1.218838in}{0.653501in}}%
\pgfpathcurveto{\pgfqpoint{1.211025in}{0.661315in}}{\pgfqpoint{1.200426in}{0.665705in}}{\pgfqpoint{1.189376in}{0.665705in}}%
\pgfpathcurveto{\pgfqpoint{1.178325in}{0.665705in}}{\pgfqpoint{1.167726in}{0.661315in}}{\pgfqpoint{1.159913in}{0.653501in}}%
\pgfpathcurveto{\pgfqpoint{1.152099in}{0.645688in}}{\pgfqpoint{1.147709in}{0.635088in}}{\pgfqpoint{1.147709in}{0.624038in}}%
\pgfpathcurveto{\pgfqpoint{1.147709in}{0.612988in}}{\pgfqpoint{1.152099in}{0.602389in}}{\pgfqpoint{1.159913in}{0.594576in}}%
\pgfpathcurveto{\pgfqpoint{1.167726in}{0.586762in}}{\pgfqpoint{1.178325in}{0.582372in}}{\pgfqpoint{1.189376in}{0.582372in}}%
\pgfpathclose%
\pgfusepath{stroke,fill}%
\end{pgfscope}%
\begin{pgfscope}%
\pgfpathrectangle{\pgfqpoint{0.772069in}{0.515123in}}{\pgfqpoint{1.937500in}{1.347500in}}%
\pgfusepath{clip}%
\pgfsetbuttcap%
\pgfsetroundjoin%
\definecolor{currentfill}{rgb}{1.000000,0.388235,0.278431}%
\pgfsetfillcolor{currentfill}%
\pgfsetlinewidth{1.003750pt}%
\definecolor{currentstroke}{rgb}{1.000000,0.388235,0.278431}%
\pgfsetstrokecolor{currentstroke}%
\pgfsetdash{}{0pt}%
\pgfpathmoveto{\pgfqpoint{1.350504in}{0.586561in}}%
\pgfpathcurveto{\pgfqpoint{1.361554in}{0.586561in}}{\pgfqpoint{1.372153in}{0.590951in}}{\pgfqpoint{1.379967in}{0.598765in}}%
\pgfpathcurveto{\pgfqpoint{1.387781in}{0.606579in}}{\pgfqpoint{1.392171in}{0.617178in}}{\pgfqpoint{1.392171in}{0.628228in}}%
\pgfpathcurveto{\pgfqpoint{1.392171in}{0.639278in}}{\pgfqpoint{1.387781in}{0.649877in}}{\pgfqpoint{1.379967in}{0.657691in}}%
\pgfpathcurveto{\pgfqpoint{1.372153in}{0.665504in}}{\pgfqpoint{1.361554in}{0.669895in}}{\pgfqpoint{1.350504in}{0.669895in}}%
\pgfpathcurveto{\pgfqpoint{1.339454in}{0.669895in}}{\pgfqpoint{1.328855in}{0.665504in}}{\pgfqpoint{1.321041in}{0.657691in}}%
\pgfpathcurveto{\pgfqpoint{1.313228in}{0.649877in}}{\pgfqpoint{1.308837in}{0.639278in}}{\pgfqpoint{1.308837in}{0.628228in}}%
\pgfpathcurveto{\pgfqpoint{1.308837in}{0.617178in}}{\pgfqpoint{1.313228in}{0.606579in}}{\pgfqpoint{1.321041in}{0.598765in}}%
\pgfpathcurveto{\pgfqpoint{1.328855in}{0.590951in}}{\pgfqpoint{1.339454in}{0.586561in}}{\pgfqpoint{1.350504in}{0.586561in}}%
\pgfpathclose%
\pgfusepath{stroke,fill}%
\end{pgfscope}%
\begin{pgfscope}%
\pgfpathrectangle{\pgfqpoint{0.772069in}{0.515123in}}{\pgfqpoint{1.937500in}{1.347500in}}%
\pgfusepath{clip}%
\pgfsetbuttcap%
\pgfsetroundjoin%
\definecolor{currentfill}{rgb}{1.000000,0.388235,0.278431}%
\pgfsetfillcolor{currentfill}%
\pgfsetlinewidth{1.003750pt}%
\definecolor{currentstroke}{rgb}{1.000000,0.388235,0.278431}%
\pgfsetstrokecolor{currentstroke}%
\pgfsetdash{}{0pt}%
\pgfpathmoveto{\pgfqpoint{1.534651in}{0.613793in}}%
\pgfpathcurveto{\pgfqpoint{1.545701in}{0.613793in}}{\pgfqpoint{1.556300in}{0.618183in}}{\pgfqpoint{1.564114in}{0.625997in}}%
\pgfpathcurveto{\pgfqpoint{1.571927in}{0.633811in}}{\pgfqpoint{1.576318in}{0.644410in}}{\pgfqpoint{1.576318in}{0.655460in}}%
\pgfpathcurveto{\pgfqpoint{1.576318in}{0.666510in}}{\pgfqpoint{1.571927in}{0.677109in}}{\pgfqpoint{1.564114in}{0.684923in}}%
\pgfpathcurveto{\pgfqpoint{1.556300in}{0.692736in}}{\pgfqpoint{1.545701in}{0.697127in}}{\pgfqpoint{1.534651in}{0.697127in}}%
\pgfpathcurveto{\pgfqpoint{1.523601in}{0.697127in}}{\pgfqpoint{1.513002in}{0.692736in}}{\pgfqpoint{1.505188in}{0.684923in}}%
\pgfpathcurveto{\pgfqpoint{1.497375in}{0.677109in}}{\pgfqpoint{1.492984in}{0.666510in}}{\pgfqpoint{1.492984in}{0.655460in}}%
\pgfpathcurveto{\pgfqpoint{1.492984in}{0.644410in}}{\pgfqpoint{1.497375in}{0.633811in}}{\pgfqpoint{1.505188in}{0.625997in}}%
\pgfpathcurveto{\pgfqpoint{1.513002in}{0.618183in}}{\pgfqpoint{1.523601in}{0.613793in}}{\pgfqpoint{1.534651in}{0.613793in}}%
\pgfpathclose%
\pgfusepath{stroke,fill}%
\end{pgfscope}%
\begin{pgfscope}%
\pgfpathrectangle{\pgfqpoint{0.772069in}{0.515123in}}{\pgfqpoint{1.937500in}{1.347500in}}%
\pgfusepath{clip}%
\pgfsetbuttcap%
\pgfsetroundjoin%
\definecolor{currentfill}{rgb}{1.000000,0.388235,0.278431}%
\pgfsetfillcolor{currentfill}%
\pgfsetlinewidth{1.003750pt}%
\definecolor{currentstroke}{rgb}{1.000000,0.388235,0.278431}%
\pgfsetstrokecolor{currentstroke}%
\pgfsetdash{}{0pt}%
\pgfpathmoveto{\pgfqpoint{1.718798in}{0.629271in}}%
\pgfpathcurveto{\pgfqpoint{1.729848in}{0.629271in}}{\pgfqpoint{1.740447in}{0.633662in}}{\pgfqpoint{1.748261in}{0.641475in}}%
\pgfpathcurveto{\pgfqpoint{1.756074in}{0.649289in}}{\pgfqpoint{1.760465in}{0.659888in}}{\pgfqpoint{1.760465in}{0.670938in}}%
\pgfpathcurveto{\pgfqpoint{1.760465in}{0.681988in}}{\pgfqpoint{1.756074in}{0.692587in}}{\pgfqpoint{1.748261in}{0.700401in}}%
\pgfpathcurveto{\pgfqpoint{1.740447in}{0.708214in}}{\pgfqpoint{1.729848in}{0.712605in}}{\pgfqpoint{1.718798in}{0.712605in}}%
\pgfpathcurveto{\pgfqpoint{1.707748in}{0.712605in}}{\pgfqpoint{1.697149in}{0.708214in}}{\pgfqpoint{1.689335in}{0.700401in}}%
\pgfpathcurveto{\pgfqpoint{1.681522in}{0.692587in}}{\pgfqpoint{1.677131in}{0.681988in}}{\pgfqpoint{1.677131in}{0.670938in}}%
\pgfpathcurveto{\pgfqpoint{1.677131in}{0.659888in}}{\pgfqpoint{1.681522in}{0.649289in}}{\pgfqpoint{1.689335in}{0.641475in}}%
\pgfpathcurveto{\pgfqpoint{1.697149in}{0.633662in}}{\pgfqpoint{1.707748in}{0.629271in}}{\pgfqpoint{1.718798in}{0.629271in}}%
\pgfpathclose%
\pgfusepath{stroke,fill}%
\end{pgfscope}%
\begin{pgfscope}%
\pgfpathrectangle{\pgfqpoint{0.772069in}{0.515123in}}{\pgfqpoint{1.937500in}{1.347500in}}%
\pgfusepath{clip}%
\pgfsetbuttcap%
\pgfsetroundjoin%
\definecolor{currentfill}{rgb}{1.000000,0.388235,0.278431}%
\pgfsetfillcolor{currentfill}%
\pgfsetlinewidth{1.003750pt}%
\definecolor{currentstroke}{rgb}{1.000000,0.388235,0.278431}%
\pgfsetstrokecolor{currentstroke}%
\pgfsetdash{}{0pt}%
\pgfpathmoveto{\pgfqpoint{1.879926in}{0.644051in}}%
\pgfpathcurveto{\pgfqpoint{1.890977in}{0.644051in}}{\pgfqpoint{1.901576in}{0.648441in}}{\pgfqpoint{1.909389in}{0.656255in}}%
\pgfpathcurveto{\pgfqpoint{1.917203in}{0.664069in}}{\pgfqpoint{1.921593in}{0.674668in}}{\pgfqpoint{1.921593in}{0.685718in}}%
\pgfpathcurveto{\pgfqpoint{1.921593in}{0.696768in}}{\pgfqpoint{1.917203in}{0.707367in}}{\pgfqpoint{1.909389in}{0.715180in}}%
\pgfpathcurveto{\pgfqpoint{1.901576in}{0.722994in}}{\pgfqpoint{1.890977in}{0.727384in}}{\pgfqpoint{1.879926in}{0.727384in}}%
\pgfpathcurveto{\pgfqpoint{1.868876in}{0.727384in}}{\pgfqpoint{1.858277in}{0.722994in}}{\pgfqpoint{1.850464in}{0.715180in}}%
\pgfpathcurveto{\pgfqpoint{1.842650in}{0.707367in}}{\pgfqpoint{1.838260in}{0.696768in}}{\pgfqpoint{1.838260in}{0.685718in}}%
\pgfpathcurveto{\pgfqpoint{1.838260in}{0.674668in}}{\pgfqpoint{1.842650in}{0.664069in}}{\pgfqpoint{1.850464in}{0.656255in}}%
\pgfpathcurveto{\pgfqpoint{1.858277in}{0.648441in}}{\pgfqpoint{1.868876in}{0.644051in}}{\pgfqpoint{1.879926in}{0.644051in}}%
\pgfpathclose%
\pgfusepath{stroke,fill}%
\end{pgfscope}%
\begin{pgfscope}%
\pgfpathrectangle{\pgfqpoint{0.772069in}{0.515123in}}{\pgfqpoint{1.937500in}{1.347500in}}%
\pgfusepath{clip}%
\pgfsetbuttcap%
\pgfsetroundjoin%
\definecolor{currentfill}{rgb}{1.000000,0.388235,0.278431}%
\pgfsetfillcolor{currentfill}%
\pgfsetlinewidth{1.003750pt}%
\definecolor{currentstroke}{rgb}{1.000000,0.388235,0.278431}%
\pgfsetstrokecolor{currentstroke}%
\pgfsetdash{}{0pt}%
\pgfpathmoveto{\pgfqpoint{2.064073in}{0.660111in}}%
\pgfpathcurveto{\pgfqpoint{2.075123in}{0.660111in}}{\pgfqpoint{2.085723in}{0.664501in}}{\pgfqpoint{2.093536in}{0.672315in}}%
\pgfpathcurveto{\pgfqpoint{2.101350in}{0.680128in}}{\pgfqpoint{2.105740in}{0.690727in}}{\pgfqpoint{2.105740in}{0.701778in}}%
\pgfpathcurveto{\pgfqpoint{2.105740in}{0.712828in}}{\pgfqpoint{2.101350in}{0.723427in}}{\pgfqpoint{2.093536in}{0.731240in}}%
\pgfpathcurveto{\pgfqpoint{2.085723in}{0.739054in}}{\pgfqpoint{2.075123in}{0.743444in}}{\pgfqpoint{2.064073in}{0.743444in}}%
\pgfpathcurveto{\pgfqpoint{2.053023in}{0.743444in}}{\pgfqpoint{2.042424in}{0.739054in}}{\pgfqpoint{2.034611in}{0.731240in}}%
\pgfpathcurveto{\pgfqpoint{2.026797in}{0.723427in}}{\pgfqpoint{2.022407in}{0.712828in}}{\pgfqpoint{2.022407in}{0.701778in}}%
\pgfpathcurveto{\pgfqpoint{2.022407in}{0.690727in}}{\pgfqpoint{2.026797in}{0.680128in}}{\pgfqpoint{2.034611in}{0.672315in}}%
\pgfpathcurveto{\pgfqpoint{2.042424in}{0.664501in}}{\pgfqpoint{2.053023in}{0.660111in}}{\pgfqpoint{2.064073in}{0.660111in}}%
\pgfpathclose%
\pgfusepath{stroke,fill}%
\end{pgfscope}%
\begin{pgfscope}%
\pgfpathrectangle{\pgfqpoint{0.772069in}{0.515123in}}{\pgfqpoint{1.937500in}{1.347500in}}%
\pgfusepath{clip}%
\pgfsetbuttcap%
\pgfsetroundjoin%
\definecolor{currentfill}{rgb}{1.000000,0.388235,0.278431}%
\pgfsetfillcolor{currentfill}%
\pgfsetlinewidth{1.003750pt}%
\definecolor{currentstroke}{rgb}{1.000000,0.388235,0.278431}%
\pgfsetstrokecolor{currentstroke}%
\pgfsetdash{}{0pt}%
\pgfpathmoveto{\pgfqpoint{2.248220in}{0.675240in}}%
\pgfpathcurveto{\pgfqpoint{2.259270in}{0.675240in}}{\pgfqpoint{2.269869in}{0.679630in}}{\pgfqpoint{2.277683in}{0.687444in}}%
\pgfpathcurveto{\pgfqpoint{2.285497in}{0.695257in}}{\pgfqpoint{2.289887in}{0.705856in}}{\pgfqpoint{2.289887in}{0.716907in}}%
\pgfpathcurveto{\pgfqpoint{2.289887in}{0.727957in}}{\pgfqpoint{2.285497in}{0.738556in}}{\pgfqpoint{2.277683in}{0.746369in}}%
\pgfpathcurveto{\pgfqpoint{2.269869in}{0.754183in}}{\pgfqpoint{2.259270in}{0.758573in}}{\pgfqpoint{2.248220in}{0.758573in}}%
\pgfpathcurveto{\pgfqpoint{2.237170in}{0.758573in}}{\pgfqpoint{2.226571in}{0.754183in}}{\pgfqpoint{2.218757in}{0.746369in}}%
\pgfpathcurveto{\pgfqpoint{2.210944in}{0.738556in}}{\pgfqpoint{2.206554in}{0.727957in}}{\pgfqpoint{2.206554in}{0.716907in}}%
\pgfpathcurveto{\pgfqpoint{2.206554in}{0.705856in}}{\pgfqpoint{2.210944in}{0.695257in}}{\pgfqpoint{2.218757in}{0.687444in}}%
\pgfpathcurveto{\pgfqpoint{2.226571in}{0.679630in}}{\pgfqpoint{2.237170in}{0.675240in}}{\pgfqpoint{2.248220in}{0.675240in}}%
\pgfpathclose%
\pgfusepath{stroke,fill}%
\end{pgfscope}%
\begin{pgfscope}%
\pgfpathrectangle{\pgfqpoint{0.772069in}{0.515123in}}{\pgfqpoint{1.937500in}{1.347500in}}%
\pgfusepath{clip}%
\pgfsetbuttcap%
\pgfsetroundjoin%
\definecolor{currentfill}{rgb}{1.000000,0.388235,0.278431}%
\pgfsetfillcolor{currentfill}%
\pgfsetlinewidth{1.003750pt}%
\definecolor{currentstroke}{rgb}{1.000000,0.388235,0.278431}%
\pgfsetstrokecolor{currentstroke}%
\pgfsetdash{}{0pt}%
\pgfpathmoveto{\pgfqpoint{2.409349in}{0.690369in}}%
\pgfpathcurveto{\pgfqpoint{2.420399in}{0.690369in}}{\pgfqpoint{2.430998in}{0.694759in}}{\pgfqpoint{2.438812in}{0.702573in}}%
\pgfpathcurveto{\pgfqpoint{2.446625in}{0.710386in}}{\pgfqpoint{2.451015in}{0.720985in}}{\pgfqpoint{2.451015in}{0.732035in}}%
\pgfpathcurveto{\pgfqpoint{2.451015in}{0.743086in}}{\pgfqpoint{2.446625in}{0.753685in}}{\pgfqpoint{2.438812in}{0.761498in}}%
\pgfpathcurveto{\pgfqpoint{2.430998in}{0.769312in}}{\pgfqpoint{2.420399in}{0.773702in}}{\pgfqpoint{2.409349in}{0.773702in}}%
\pgfpathcurveto{\pgfqpoint{2.398299in}{0.773702in}}{\pgfqpoint{2.387700in}{0.769312in}}{\pgfqpoint{2.379886in}{0.761498in}}%
\pgfpathcurveto{\pgfqpoint{2.372072in}{0.753685in}}{\pgfqpoint{2.367682in}{0.743086in}}{\pgfqpoint{2.367682in}{0.732035in}}%
\pgfpathcurveto{\pgfqpoint{2.367682in}{0.720985in}}{\pgfqpoint{2.372072in}{0.710386in}}{\pgfqpoint{2.379886in}{0.702573in}}%
\pgfpathcurveto{\pgfqpoint{2.387700in}{0.694759in}}{\pgfqpoint{2.398299in}{0.690369in}}{\pgfqpoint{2.409349in}{0.690369in}}%
\pgfpathclose%
\pgfusepath{stroke,fill}%
\end{pgfscope}%
\begin{pgfscope}%
\pgfpathrectangle{\pgfqpoint{0.772069in}{0.515123in}}{\pgfqpoint{1.937500in}{1.347500in}}%
\pgfusepath{clip}%
\pgfsetbuttcap%
\pgfsetroundjoin%
\definecolor{currentfill}{rgb}{1.000000,0.388235,0.278431}%
\pgfsetfillcolor{currentfill}%
\pgfsetlinewidth{1.003750pt}%
\definecolor{currentstroke}{rgb}{1.000000,0.388235,0.278431}%
\pgfsetstrokecolor{currentstroke}%
\pgfsetdash{}{0pt}%
\pgfpathmoveto{\pgfqpoint{2.593496in}{0.705963in}}%
\pgfpathcurveto{\pgfqpoint{2.604546in}{0.705963in}}{\pgfqpoint{2.615145in}{0.710353in}}{\pgfqpoint{2.622958in}{0.718167in}}%
\pgfpathcurveto{\pgfqpoint{2.630772in}{0.725981in}}{\pgfqpoint{2.635162in}{0.736580in}}{\pgfqpoint{2.635162in}{0.747630in}}%
\pgfpathcurveto{\pgfqpoint{2.635162in}{0.758680in}}{\pgfqpoint{2.630772in}{0.769279in}}{\pgfqpoint{2.622958in}{0.777093in}}%
\pgfpathcurveto{\pgfqpoint{2.615145in}{0.784906in}}{\pgfqpoint{2.604546in}{0.789296in}}{\pgfqpoint{2.593496in}{0.789296in}}%
\pgfpathcurveto{\pgfqpoint{2.582446in}{0.789296in}}{\pgfqpoint{2.571847in}{0.784906in}}{\pgfqpoint{2.564033in}{0.777093in}}%
\pgfpathcurveto{\pgfqpoint{2.556219in}{0.769279in}}{\pgfqpoint{2.551829in}{0.758680in}}{\pgfqpoint{2.551829in}{0.747630in}}%
\pgfpathcurveto{\pgfqpoint{2.551829in}{0.736580in}}{\pgfqpoint{2.556219in}{0.725981in}}{\pgfqpoint{2.564033in}{0.718167in}}%
\pgfpathcurveto{\pgfqpoint{2.571847in}{0.710353in}}{\pgfqpoint{2.582446in}{0.705963in}}{\pgfqpoint{2.593496in}{0.705963in}}%
\pgfpathclose%
\pgfusepath{stroke,fill}%
\end{pgfscope}%
\begin{pgfscope}%
\pgfpathrectangle{\pgfqpoint{0.772069in}{0.515123in}}{\pgfqpoint{1.937500in}{1.347500in}}%
\pgfusepath{clip}%
\pgfsetbuttcap%
\pgfsetroundjoin%
\definecolor{currentfill}{rgb}{1.000000,0.843137,0.000000}%
\pgfsetfillcolor{currentfill}%
\pgfsetlinewidth{1.003750pt}%
\definecolor{currentstroke}{rgb}{1.000000,0.843137,0.000000}%
\pgfsetstrokecolor{currentstroke}%
\pgfsetdash{}{0pt}%
\pgfpathmoveto{\pgfqpoint{1.005229in}{0.564706in}}%
\pgfpathcurveto{\pgfqpoint{1.016279in}{0.564706in}}{\pgfqpoint{1.026878in}{0.569096in}}{\pgfqpoint{1.034691in}{0.576910in}}%
\pgfpathcurveto{\pgfqpoint{1.042505in}{0.584723in}}{\pgfqpoint{1.046895in}{0.595322in}}{\pgfqpoint{1.046895in}{0.606372in}}%
\pgfpathcurveto{\pgfqpoint{1.046895in}{0.617423in}}{\pgfqpoint{1.042505in}{0.628022in}}{\pgfqpoint{1.034691in}{0.635835in}}%
\pgfpathcurveto{\pgfqpoint{1.026878in}{0.643649in}}{\pgfqpoint{1.016279in}{0.648039in}}{\pgfqpoint{1.005229in}{0.648039in}}%
\pgfpathcurveto{\pgfqpoint{0.994179in}{0.648039in}}{\pgfqpoint{0.983580in}{0.643649in}}{\pgfqpoint{0.975766in}{0.635835in}}%
\pgfpathcurveto{\pgfqpoint{0.967952in}{0.628022in}}{\pgfqpoint{0.963562in}{0.617423in}}{\pgfqpoint{0.963562in}{0.606372in}}%
\pgfpathcurveto{\pgfqpoint{0.963562in}{0.595322in}}{\pgfqpoint{0.967952in}{0.584723in}}{\pgfqpoint{0.975766in}{0.576910in}}%
\pgfpathcurveto{\pgfqpoint{0.983580in}{0.569096in}}{\pgfqpoint{0.994179in}{0.564706in}}{\pgfqpoint{1.005229in}{0.564706in}}%
\pgfpathclose%
\pgfusepath{stroke,fill}%
\end{pgfscope}%
\begin{pgfscope}%
\pgfpathrectangle{\pgfqpoint{0.772069in}{0.515123in}}{\pgfqpoint{1.937500in}{1.347500in}}%
\pgfusepath{clip}%
\pgfsetbuttcap%
\pgfsetroundjoin%
\definecolor{currentfill}{rgb}{1.000000,0.843137,0.000000}%
\pgfsetfillcolor{currentfill}%
\pgfsetlinewidth{1.003750pt}%
\definecolor{currentstroke}{rgb}{1.000000,0.843137,0.000000}%
\pgfsetstrokecolor{currentstroke}%
\pgfsetdash{}{0pt}%
\pgfpathmoveto{\pgfqpoint{1.189376in}{0.576320in}}%
\pgfpathcurveto{\pgfqpoint{1.200426in}{0.576320in}}{\pgfqpoint{1.211025in}{0.580710in}}{\pgfqpoint{1.218838in}{0.588524in}}%
\pgfpathcurveto{\pgfqpoint{1.226652in}{0.596338in}}{\pgfqpoint{1.231042in}{0.606937in}}{\pgfqpoint{1.231042in}{0.617987in}}%
\pgfpathcurveto{\pgfqpoint{1.231042in}{0.629037in}}{\pgfqpoint{1.226652in}{0.639636in}}{\pgfqpoint{1.218838in}{0.647450in}}%
\pgfpathcurveto{\pgfqpoint{1.211025in}{0.655263in}}{\pgfqpoint{1.200426in}{0.659653in}}{\pgfqpoint{1.189376in}{0.659653in}}%
\pgfpathcurveto{\pgfqpoint{1.178325in}{0.659653in}}{\pgfqpoint{1.167726in}{0.655263in}}{\pgfqpoint{1.159913in}{0.647450in}}%
\pgfpathcurveto{\pgfqpoint{1.152099in}{0.639636in}}{\pgfqpoint{1.147709in}{0.629037in}}{\pgfqpoint{1.147709in}{0.617987in}}%
\pgfpathcurveto{\pgfqpoint{1.147709in}{0.606937in}}{\pgfqpoint{1.152099in}{0.596338in}}{\pgfqpoint{1.159913in}{0.588524in}}%
\pgfpathcurveto{\pgfqpoint{1.167726in}{0.580710in}}{\pgfqpoint{1.178325in}{0.576320in}}{\pgfqpoint{1.189376in}{0.576320in}}%
\pgfpathclose%
\pgfusepath{stroke,fill}%
\end{pgfscope}%
\begin{pgfscope}%
\pgfpathrectangle{\pgfqpoint{0.772069in}{0.515123in}}{\pgfqpoint{1.937500in}{1.347500in}}%
\pgfusepath{clip}%
\pgfsetbuttcap%
\pgfsetroundjoin%
\definecolor{currentfill}{rgb}{1.000000,0.843137,0.000000}%
\pgfsetfillcolor{currentfill}%
\pgfsetlinewidth{1.003750pt}%
\definecolor{currentstroke}{rgb}{1.000000,0.843137,0.000000}%
\pgfsetstrokecolor{currentstroke}%
\pgfsetdash{}{0pt}%
\pgfpathmoveto{\pgfqpoint{1.350504in}{0.588889in}}%
\pgfpathcurveto{\pgfqpoint{1.361554in}{0.588889in}}{\pgfqpoint{1.372153in}{0.593279in}}{\pgfqpoint{1.379967in}{0.601093in}}%
\pgfpathcurveto{\pgfqpoint{1.387781in}{0.608906in}}{\pgfqpoint{1.392171in}{0.619505in}}{\pgfqpoint{1.392171in}{0.630555in}}%
\pgfpathcurveto{\pgfqpoint{1.392171in}{0.641606in}}{\pgfqpoint{1.387781in}{0.652205in}}{\pgfqpoint{1.379967in}{0.660018in}}%
\pgfpathcurveto{\pgfqpoint{1.372153in}{0.667832in}}{\pgfqpoint{1.361554in}{0.672222in}}{\pgfqpoint{1.350504in}{0.672222in}}%
\pgfpathcurveto{\pgfqpoint{1.339454in}{0.672222in}}{\pgfqpoint{1.328855in}{0.667832in}}{\pgfqpoint{1.321041in}{0.660018in}}%
\pgfpathcurveto{\pgfqpoint{1.313228in}{0.652205in}}{\pgfqpoint{1.308837in}{0.641606in}}{\pgfqpoint{1.308837in}{0.630555in}}%
\pgfpathcurveto{\pgfqpoint{1.308837in}{0.619505in}}{\pgfqpoint{1.313228in}{0.608906in}}{\pgfqpoint{1.321041in}{0.601093in}}%
\pgfpathcurveto{\pgfqpoint{1.328855in}{0.593279in}}{\pgfqpoint{1.339454in}{0.588889in}}{\pgfqpoint{1.350504in}{0.588889in}}%
\pgfpathclose%
\pgfusepath{stroke,fill}%
\end{pgfscope}%
\begin{pgfscope}%
\pgfpathrectangle{\pgfqpoint{0.772069in}{0.515123in}}{\pgfqpoint{1.937500in}{1.347500in}}%
\pgfusepath{clip}%
\pgfsetbuttcap%
\pgfsetroundjoin%
\definecolor{currentfill}{rgb}{1.000000,0.843137,0.000000}%
\pgfsetfillcolor{currentfill}%
\pgfsetlinewidth{1.003750pt}%
\definecolor{currentstroke}{rgb}{1.000000,0.843137,0.000000}%
\pgfsetstrokecolor{currentstroke}%
\pgfsetdash{}{0pt}%
\pgfpathmoveto{\pgfqpoint{1.534651in}{0.601225in}}%
\pgfpathcurveto{\pgfqpoint{1.545701in}{0.601225in}}{\pgfqpoint{1.556300in}{0.605615in}}{\pgfqpoint{1.564114in}{0.613428in}}%
\pgfpathcurveto{\pgfqpoint{1.571927in}{0.621242in}}{\pgfqpoint{1.576318in}{0.631841in}}{\pgfqpoint{1.576318in}{0.642891in}}%
\pgfpathcurveto{\pgfqpoint{1.576318in}{0.653941in}}{\pgfqpoint{1.571927in}{0.664540in}}{\pgfqpoint{1.564114in}{0.672354in}}%
\pgfpathcurveto{\pgfqpoint{1.556300in}{0.680168in}}{\pgfqpoint{1.545701in}{0.684558in}}{\pgfqpoint{1.534651in}{0.684558in}}%
\pgfpathcurveto{\pgfqpoint{1.523601in}{0.684558in}}{\pgfqpoint{1.513002in}{0.680168in}}{\pgfqpoint{1.505188in}{0.672354in}}%
\pgfpathcurveto{\pgfqpoint{1.497375in}{0.664540in}}{\pgfqpoint{1.492984in}{0.653941in}}{\pgfqpoint{1.492984in}{0.642891in}}%
\pgfpathcurveto{\pgfqpoint{1.492984in}{0.631841in}}{\pgfqpoint{1.497375in}{0.621242in}}{\pgfqpoint{1.505188in}{0.613428in}}%
\pgfpathcurveto{\pgfqpoint{1.513002in}{0.605615in}}{\pgfqpoint{1.523601in}{0.601225in}}{\pgfqpoint{1.534651in}{0.601225in}}%
\pgfpathclose%
\pgfusepath{stroke,fill}%
\end{pgfscope}%
\begin{pgfscope}%
\pgfpathrectangle{\pgfqpoint{0.772069in}{0.515123in}}{\pgfqpoint{1.937500in}{1.347500in}}%
\pgfusepath{clip}%
\pgfsetbuttcap%
\pgfsetroundjoin%
\definecolor{currentfill}{rgb}{1.000000,0.843137,0.000000}%
\pgfsetfillcolor{currentfill}%
\pgfsetlinewidth{1.003750pt}%
\definecolor{currentstroke}{rgb}{1.000000,0.843137,0.000000}%
\pgfsetstrokecolor{currentstroke}%
\pgfsetdash{}{0pt}%
\pgfpathmoveto{\pgfqpoint{1.718798in}{0.613560in}}%
\pgfpathcurveto{\pgfqpoint{1.729848in}{0.613560in}}{\pgfqpoint{1.740447in}{0.617951in}}{\pgfqpoint{1.748261in}{0.625764in}}%
\pgfpathcurveto{\pgfqpoint{1.756074in}{0.633578in}}{\pgfqpoint{1.760465in}{0.644177in}}{\pgfqpoint{1.760465in}{0.655227in}}%
\pgfpathcurveto{\pgfqpoint{1.760465in}{0.666277in}}{\pgfqpoint{1.756074in}{0.676876in}}{\pgfqpoint{1.748261in}{0.684690in}}%
\pgfpathcurveto{\pgfqpoint{1.740447in}{0.692504in}}{\pgfqpoint{1.729848in}{0.696894in}}{\pgfqpoint{1.718798in}{0.696894in}}%
\pgfpathcurveto{\pgfqpoint{1.707748in}{0.696894in}}{\pgfqpoint{1.697149in}{0.692504in}}{\pgfqpoint{1.689335in}{0.684690in}}%
\pgfpathcurveto{\pgfqpoint{1.681522in}{0.676876in}}{\pgfqpoint{1.677131in}{0.666277in}}{\pgfqpoint{1.677131in}{0.655227in}}%
\pgfpathcurveto{\pgfqpoint{1.677131in}{0.644177in}}{\pgfqpoint{1.681522in}{0.633578in}}{\pgfqpoint{1.689335in}{0.625764in}}%
\pgfpathcurveto{\pgfqpoint{1.697149in}{0.617951in}}{\pgfqpoint{1.707748in}{0.613560in}}{\pgfqpoint{1.718798in}{0.613560in}}%
\pgfpathclose%
\pgfusepath{stroke,fill}%
\end{pgfscope}%
\begin{pgfscope}%
\pgfpathrectangle{\pgfqpoint{0.772069in}{0.515123in}}{\pgfqpoint{1.937500in}{1.347500in}}%
\pgfusepath{clip}%
\pgfsetbuttcap%
\pgfsetroundjoin%
\definecolor{currentfill}{rgb}{1.000000,0.843137,0.000000}%
\pgfsetfillcolor{currentfill}%
\pgfsetlinewidth{1.003750pt}%
\definecolor{currentstroke}{rgb}{1.000000,0.843137,0.000000}%
\pgfsetstrokecolor{currentstroke}%
\pgfsetdash{}{0pt}%
\pgfpathmoveto{\pgfqpoint{1.879926in}{0.625314in}}%
\pgfpathcurveto{\pgfqpoint{1.890977in}{0.625314in}}{\pgfqpoint{1.901576in}{0.629705in}}{\pgfqpoint{1.909389in}{0.637518in}}%
\pgfpathcurveto{\pgfqpoint{1.917203in}{0.645332in}}{\pgfqpoint{1.921593in}{0.655931in}}{\pgfqpoint{1.921593in}{0.666981in}}%
\pgfpathcurveto{\pgfqpoint{1.921593in}{0.678031in}}{\pgfqpoint{1.917203in}{0.688630in}}{\pgfqpoint{1.909389in}{0.696444in}}%
\pgfpathcurveto{\pgfqpoint{1.901576in}{0.704258in}}{\pgfqpoint{1.890977in}{0.708648in}}{\pgfqpoint{1.879926in}{0.708648in}}%
\pgfpathcurveto{\pgfqpoint{1.868876in}{0.708648in}}{\pgfqpoint{1.858277in}{0.704258in}}{\pgfqpoint{1.850464in}{0.696444in}}%
\pgfpathcurveto{\pgfqpoint{1.842650in}{0.688630in}}{\pgfqpoint{1.838260in}{0.678031in}}{\pgfqpoint{1.838260in}{0.666981in}}%
\pgfpathcurveto{\pgfqpoint{1.838260in}{0.655931in}}{\pgfqpoint{1.842650in}{0.645332in}}{\pgfqpoint{1.850464in}{0.637518in}}%
\pgfpathcurveto{\pgfqpoint{1.858277in}{0.629705in}}{\pgfqpoint{1.868876in}{0.625314in}}{\pgfqpoint{1.879926in}{0.625314in}}%
\pgfpathclose%
\pgfusepath{stroke,fill}%
\end{pgfscope}%
\begin{pgfscope}%
\pgfpathrectangle{\pgfqpoint{0.772069in}{0.515123in}}{\pgfqpoint{1.937500in}{1.347500in}}%
\pgfusepath{clip}%
\pgfsetbuttcap%
\pgfsetroundjoin%
\definecolor{currentfill}{rgb}{1.000000,0.843137,0.000000}%
\pgfsetfillcolor{currentfill}%
\pgfsetlinewidth{1.003750pt}%
\definecolor{currentstroke}{rgb}{1.000000,0.843137,0.000000}%
\pgfsetstrokecolor{currentstroke}%
\pgfsetdash{}{0pt}%
\pgfpathmoveto{\pgfqpoint{2.064073in}{0.644284in}}%
\pgfpathcurveto{\pgfqpoint{2.075123in}{0.644284in}}{\pgfqpoint{2.085723in}{0.648674in}}{\pgfqpoint{2.093536in}{0.656488in}}%
\pgfpathcurveto{\pgfqpoint{2.101350in}{0.664301in}}{\pgfqpoint{2.105740in}{0.674900in}}{\pgfqpoint{2.105740in}{0.685950in}}%
\pgfpathcurveto{\pgfqpoint{2.105740in}{0.697001in}}{\pgfqpoint{2.101350in}{0.707600in}}{\pgfqpoint{2.093536in}{0.715413in}}%
\pgfpathcurveto{\pgfqpoint{2.085723in}{0.723227in}}{\pgfqpoint{2.075123in}{0.727617in}}{\pgfqpoint{2.064073in}{0.727617in}}%
\pgfpathcurveto{\pgfqpoint{2.053023in}{0.727617in}}{\pgfqpoint{2.042424in}{0.723227in}}{\pgfqpoint{2.034611in}{0.715413in}}%
\pgfpathcurveto{\pgfqpoint{2.026797in}{0.707600in}}{\pgfqpoint{2.022407in}{0.697001in}}{\pgfqpoint{2.022407in}{0.685950in}}%
\pgfpathcurveto{\pgfqpoint{2.022407in}{0.674900in}}{\pgfqpoint{2.026797in}{0.664301in}}{\pgfqpoint{2.034611in}{0.656488in}}%
\pgfpathcurveto{\pgfqpoint{2.042424in}{0.648674in}}{\pgfqpoint{2.053023in}{0.644284in}}{\pgfqpoint{2.064073in}{0.644284in}}%
\pgfpathclose%
\pgfusepath{stroke,fill}%
\end{pgfscope}%
\begin{pgfscope}%
\pgfpathrectangle{\pgfqpoint{0.772069in}{0.515123in}}{\pgfqpoint{1.937500in}{1.347500in}}%
\pgfusepath{clip}%
\pgfsetbuttcap%
\pgfsetroundjoin%
\definecolor{currentfill}{rgb}{1.000000,0.843137,0.000000}%
\pgfsetfillcolor{currentfill}%
\pgfsetlinewidth{1.003750pt}%
\definecolor{currentstroke}{rgb}{1.000000,0.843137,0.000000}%
\pgfsetstrokecolor{currentstroke}%
\pgfsetdash{}{0pt}%
\pgfpathmoveto{\pgfqpoint{2.248220in}{0.650335in}}%
\pgfpathcurveto{\pgfqpoint{2.259270in}{0.650335in}}{\pgfqpoint{2.269869in}{0.654726in}}{\pgfqpoint{2.277683in}{0.662539in}}%
\pgfpathcurveto{\pgfqpoint{2.285497in}{0.670353in}}{\pgfqpoint{2.289887in}{0.680952in}}{\pgfqpoint{2.289887in}{0.692002in}}%
\pgfpathcurveto{\pgfqpoint{2.289887in}{0.703052in}}{\pgfqpoint{2.285497in}{0.713651in}}{\pgfqpoint{2.277683in}{0.721465in}}%
\pgfpathcurveto{\pgfqpoint{2.269869in}{0.729278in}}{\pgfqpoint{2.259270in}{0.733669in}}{\pgfqpoint{2.248220in}{0.733669in}}%
\pgfpathcurveto{\pgfqpoint{2.237170in}{0.733669in}}{\pgfqpoint{2.226571in}{0.729278in}}{\pgfqpoint{2.218757in}{0.721465in}}%
\pgfpathcurveto{\pgfqpoint{2.210944in}{0.713651in}}{\pgfqpoint{2.206554in}{0.703052in}}{\pgfqpoint{2.206554in}{0.692002in}}%
\pgfpathcurveto{\pgfqpoint{2.206554in}{0.680952in}}{\pgfqpoint{2.210944in}{0.670353in}}{\pgfqpoint{2.218757in}{0.662539in}}%
\pgfpathcurveto{\pgfqpoint{2.226571in}{0.654726in}}{\pgfqpoint{2.237170in}{0.650335in}}{\pgfqpoint{2.248220in}{0.650335in}}%
\pgfpathclose%
\pgfusepath{stroke,fill}%
\end{pgfscope}%
\begin{pgfscope}%
\pgfpathrectangle{\pgfqpoint{0.772069in}{0.515123in}}{\pgfqpoint{1.937500in}{1.347500in}}%
\pgfusepath{clip}%
\pgfsetbuttcap%
\pgfsetroundjoin%
\definecolor{currentfill}{rgb}{1.000000,0.843137,0.000000}%
\pgfsetfillcolor{currentfill}%
\pgfsetlinewidth{1.003750pt}%
\definecolor{currentstroke}{rgb}{1.000000,0.843137,0.000000}%
\pgfsetstrokecolor{currentstroke}%
\pgfsetdash{}{0pt}%
\pgfpathmoveto{\pgfqpoint{2.409349in}{0.662322in}}%
\pgfpathcurveto{\pgfqpoint{2.420399in}{0.662322in}}{\pgfqpoint{2.430998in}{0.666712in}}{\pgfqpoint{2.438812in}{0.674526in}}%
\pgfpathcurveto{\pgfqpoint{2.446625in}{0.682340in}}{\pgfqpoint{2.451015in}{0.692939in}}{\pgfqpoint{2.451015in}{0.703989in}}%
\pgfpathcurveto{\pgfqpoint{2.451015in}{0.715039in}}{\pgfqpoint{2.446625in}{0.725638in}}{\pgfqpoint{2.438812in}{0.733452in}}%
\pgfpathcurveto{\pgfqpoint{2.430998in}{0.741265in}}{\pgfqpoint{2.420399in}{0.745655in}}{\pgfqpoint{2.409349in}{0.745655in}}%
\pgfpathcurveto{\pgfqpoint{2.398299in}{0.745655in}}{\pgfqpoint{2.387700in}{0.741265in}}{\pgfqpoint{2.379886in}{0.733452in}}%
\pgfpathcurveto{\pgfqpoint{2.372072in}{0.725638in}}{\pgfqpoint{2.367682in}{0.715039in}}{\pgfqpoint{2.367682in}{0.703989in}}%
\pgfpathcurveto{\pgfqpoint{2.367682in}{0.692939in}}{\pgfqpoint{2.372072in}{0.682340in}}{\pgfqpoint{2.379886in}{0.674526in}}%
\pgfpathcurveto{\pgfqpoint{2.387700in}{0.666712in}}{\pgfqpoint{2.398299in}{0.662322in}}{\pgfqpoint{2.409349in}{0.662322in}}%
\pgfpathclose%
\pgfusepath{stroke,fill}%
\end{pgfscope}%
\begin{pgfscope}%
\pgfpathrectangle{\pgfqpoint{0.772069in}{0.515123in}}{\pgfqpoint{1.937500in}{1.347500in}}%
\pgfusepath{clip}%
\pgfsetbuttcap%
\pgfsetroundjoin%
\definecolor{currentfill}{rgb}{1.000000,0.843137,0.000000}%
\pgfsetfillcolor{currentfill}%
\pgfsetlinewidth{1.003750pt}%
\definecolor{currentstroke}{rgb}{1.000000,0.843137,0.000000}%
\pgfsetstrokecolor{currentstroke}%
\pgfsetdash{}{0pt}%
\pgfpathmoveto{\pgfqpoint{2.593496in}{0.674658in}}%
\pgfpathcurveto{\pgfqpoint{2.604546in}{0.674658in}}{\pgfqpoint{2.615145in}{0.679048in}}{\pgfqpoint{2.622958in}{0.686862in}}%
\pgfpathcurveto{\pgfqpoint{2.630772in}{0.694675in}}{\pgfqpoint{2.635162in}{0.705274in}}{\pgfqpoint{2.635162in}{0.716325in}}%
\pgfpathcurveto{\pgfqpoint{2.635162in}{0.727375in}}{\pgfqpoint{2.630772in}{0.737974in}}{\pgfqpoint{2.622958in}{0.745787in}}%
\pgfpathcurveto{\pgfqpoint{2.615145in}{0.753601in}}{\pgfqpoint{2.604546in}{0.757991in}}{\pgfqpoint{2.593496in}{0.757991in}}%
\pgfpathcurveto{\pgfqpoint{2.582446in}{0.757991in}}{\pgfqpoint{2.571847in}{0.753601in}}{\pgfqpoint{2.564033in}{0.745787in}}%
\pgfpathcurveto{\pgfqpoint{2.556219in}{0.737974in}}{\pgfqpoint{2.551829in}{0.727375in}}{\pgfqpoint{2.551829in}{0.716325in}}%
\pgfpathcurveto{\pgfqpoint{2.551829in}{0.705274in}}{\pgfqpoint{2.556219in}{0.694675in}}{\pgfqpoint{2.564033in}{0.686862in}}%
\pgfpathcurveto{\pgfqpoint{2.571847in}{0.679048in}}{\pgfqpoint{2.582446in}{0.674658in}}{\pgfqpoint{2.593496in}{0.674658in}}%
\pgfpathclose%
\pgfusepath{stroke,fill}%
\end{pgfscope}%
\begin{pgfscope}%
\pgfpathrectangle{\pgfqpoint{0.772069in}{0.515123in}}{\pgfqpoint{1.937500in}{1.347500in}}%
\pgfusepath{clip}%
\pgfsetbuttcap%
\pgfsetroundjoin%
\definecolor{currentfill}{rgb}{0.196078,0.803922,0.196078}%
\pgfsetfillcolor{currentfill}%
\pgfsetlinewidth{1.003750pt}%
\definecolor{currentstroke}{rgb}{0.196078,0.803922,0.196078}%
\pgfsetstrokecolor{currentstroke}%
\pgfsetdash{}{0pt}%
\pgfpathmoveto{\pgfqpoint{1.005229in}{0.641491in}}%
\pgfpathcurveto{\pgfqpoint{1.016279in}{0.641491in}}{\pgfqpoint{1.026878in}{0.645881in}}{\pgfqpoint{1.034691in}{0.653695in}}%
\pgfpathcurveto{\pgfqpoint{1.042505in}{0.661508in}}{\pgfqpoint{1.046895in}{0.672107in}}{\pgfqpoint{1.046895in}{0.683157in}}%
\pgfpathcurveto{\pgfqpoint{1.046895in}{0.694208in}}{\pgfqpoint{1.042505in}{0.704807in}}{\pgfqpoint{1.034691in}{0.712620in}}%
\pgfpathcurveto{\pgfqpoint{1.026878in}{0.720434in}}{\pgfqpoint{1.016279in}{0.724824in}}{\pgfqpoint{1.005229in}{0.724824in}}%
\pgfpathcurveto{\pgfqpoint{0.994179in}{0.724824in}}{\pgfqpoint{0.983580in}{0.720434in}}{\pgfqpoint{0.975766in}{0.712620in}}%
\pgfpathcurveto{\pgfqpoint{0.967952in}{0.704807in}}{\pgfqpoint{0.963562in}{0.694208in}}{\pgfqpoint{0.963562in}{0.683157in}}%
\pgfpathcurveto{\pgfqpoint{0.963562in}{0.672107in}}{\pgfqpoint{0.967952in}{0.661508in}}{\pgfqpoint{0.975766in}{0.653695in}}%
\pgfpathcurveto{\pgfqpoint{0.983580in}{0.645881in}}{\pgfqpoint{0.994179in}{0.641491in}}{\pgfqpoint{1.005229in}{0.641491in}}%
\pgfpathclose%
\pgfusepath{stroke,fill}%
\end{pgfscope}%
\begin{pgfscope}%
\pgfpathrectangle{\pgfqpoint{0.772069in}{0.515123in}}{\pgfqpoint{1.937500in}{1.347500in}}%
\pgfusepath{clip}%
\pgfsetbuttcap%
\pgfsetroundjoin%
\definecolor{currentfill}{rgb}{0.196078,0.803922,0.196078}%
\pgfsetfillcolor{currentfill}%
\pgfsetlinewidth{1.003750pt}%
\definecolor{currentstroke}{rgb}{0.196078,0.803922,0.196078}%
\pgfsetstrokecolor{currentstroke}%
\pgfsetdash{}{0pt}%
\pgfpathmoveto{\pgfqpoint{1.189376in}{0.725631in}}%
\pgfpathcurveto{\pgfqpoint{1.200426in}{0.725631in}}{\pgfqpoint{1.211025in}{0.730021in}}{\pgfqpoint{1.218838in}{0.737835in}}%
\pgfpathcurveto{\pgfqpoint{1.226652in}{0.745648in}}{\pgfqpoint{1.231042in}{0.756247in}}{\pgfqpoint{1.231042in}{0.767297in}}%
\pgfpathcurveto{\pgfqpoint{1.231042in}{0.778348in}}{\pgfqpoint{1.226652in}{0.788947in}}{\pgfqpoint{1.218838in}{0.796760in}}%
\pgfpathcurveto{\pgfqpoint{1.211025in}{0.804574in}}{\pgfqpoint{1.200426in}{0.808964in}}{\pgfqpoint{1.189376in}{0.808964in}}%
\pgfpathcurveto{\pgfqpoint{1.178325in}{0.808964in}}{\pgfqpoint{1.167726in}{0.804574in}}{\pgfqpoint{1.159913in}{0.796760in}}%
\pgfpathcurveto{\pgfqpoint{1.152099in}{0.788947in}}{\pgfqpoint{1.147709in}{0.778348in}}{\pgfqpoint{1.147709in}{0.767297in}}%
\pgfpathcurveto{\pgfqpoint{1.147709in}{0.756247in}}{\pgfqpoint{1.152099in}{0.745648in}}{\pgfqpoint{1.159913in}{0.737835in}}%
\pgfpathcurveto{\pgfqpoint{1.167726in}{0.730021in}}{\pgfqpoint{1.178325in}{0.725631in}}{\pgfqpoint{1.189376in}{0.725631in}}%
\pgfpathclose%
\pgfusepath{stroke,fill}%
\end{pgfscope}%
\begin{pgfscope}%
\pgfpathrectangle{\pgfqpoint{0.772069in}{0.515123in}}{\pgfqpoint{1.937500in}{1.347500in}}%
\pgfusepath{clip}%
\pgfsetbuttcap%
\pgfsetroundjoin%
\definecolor{currentfill}{rgb}{0.196078,0.803922,0.196078}%
\pgfsetfillcolor{currentfill}%
\pgfsetlinewidth{1.003750pt}%
\definecolor{currentstroke}{rgb}{0.196078,0.803922,0.196078}%
\pgfsetstrokecolor{currentstroke}%
\pgfsetdash{}{0pt}%
\pgfpathmoveto{\pgfqpoint{1.350504in}{0.813844in}}%
\pgfpathcurveto{\pgfqpoint{1.361554in}{0.813844in}}{\pgfqpoint{1.372153in}{0.818234in}}{\pgfqpoint{1.379967in}{0.826048in}}%
\pgfpathcurveto{\pgfqpoint{1.387781in}{0.833861in}}{\pgfqpoint{1.392171in}{0.844460in}}{\pgfqpoint{1.392171in}{0.855510in}}%
\pgfpathcurveto{\pgfqpoint{1.392171in}{0.866561in}}{\pgfqpoint{1.387781in}{0.877160in}}{\pgfqpoint{1.379967in}{0.884973in}}%
\pgfpathcurveto{\pgfqpoint{1.372153in}{0.892787in}}{\pgfqpoint{1.361554in}{0.897177in}}{\pgfqpoint{1.350504in}{0.897177in}}%
\pgfpathcurveto{\pgfqpoint{1.339454in}{0.897177in}}{\pgfqpoint{1.328855in}{0.892787in}}{\pgfqpoint{1.321041in}{0.884973in}}%
\pgfpathcurveto{\pgfqpoint{1.313228in}{0.877160in}}{\pgfqpoint{1.308837in}{0.866561in}}{\pgfqpoint{1.308837in}{0.855510in}}%
\pgfpathcurveto{\pgfqpoint{1.308837in}{0.844460in}}{\pgfqpoint{1.313228in}{0.833861in}}{\pgfqpoint{1.321041in}{0.826048in}}%
\pgfpathcurveto{\pgfqpoint{1.328855in}{0.818234in}}{\pgfqpoint{1.339454in}{0.813844in}}{\pgfqpoint{1.350504in}{0.813844in}}%
\pgfpathclose%
\pgfusepath{stroke,fill}%
\end{pgfscope}%
\begin{pgfscope}%
\pgfpathrectangle{\pgfqpoint{0.772069in}{0.515123in}}{\pgfqpoint{1.937500in}{1.347500in}}%
\pgfusepath{clip}%
\pgfsetbuttcap%
\pgfsetroundjoin%
\definecolor{currentfill}{rgb}{0.196078,0.803922,0.196078}%
\pgfsetfillcolor{currentfill}%
\pgfsetlinewidth{1.003750pt}%
\definecolor{currentstroke}{rgb}{0.196078,0.803922,0.196078}%
\pgfsetstrokecolor{currentstroke}%
\pgfsetdash{}{0pt}%
\pgfpathmoveto{\pgfqpoint{1.534651in}{0.903453in}}%
\pgfpathcurveto{\pgfqpoint{1.545701in}{0.903453in}}{\pgfqpoint{1.556300in}{0.907844in}}{\pgfqpoint{1.564114in}{0.915657in}}%
\pgfpathcurveto{\pgfqpoint{1.571927in}{0.923471in}}{\pgfqpoint{1.576318in}{0.934070in}}{\pgfqpoint{1.576318in}{0.945120in}}%
\pgfpathcurveto{\pgfqpoint{1.576318in}{0.956170in}}{\pgfqpoint{1.571927in}{0.966769in}}{\pgfqpoint{1.564114in}{0.974583in}}%
\pgfpathcurveto{\pgfqpoint{1.556300in}{0.982397in}}{\pgfqpoint{1.545701in}{0.986787in}}{\pgfqpoint{1.534651in}{0.986787in}}%
\pgfpathcurveto{\pgfqpoint{1.523601in}{0.986787in}}{\pgfqpoint{1.513002in}{0.982397in}}{\pgfqpoint{1.505188in}{0.974583in}}%
\pgfpathcurveto{\pgfqpoint{1.497375in}{0.966769in}}{\pgfqpoint{1.492984in}{0.956170in}}{\pgfqpoint{1.492984in}{0.945120in}}%
\pgfpathcurveto{\pgfqpoint{1.492984in}{0.934070in}}{\pgfqpoint{1.497375in}{0.923471in}}{\pgfqpoint{1.505188in}{0.915657in}}%
\pgfpathcurveto{\pgfqpoint{1.513002in}{0.907844in}}{\pgfqpoint{1.523601in}{0.903453in}}{\pgfqpoint{1.534651in}{0.903453in}}%
\pgfpathclose%
\pgfusepath{stroke,fill}%
\end{pgfscope}%
\begin{pgfscope}%
\pgfpathrectangle{\pgfqpoint{0.772069in}{0.515123in}}{\pgfqpoint{1.937500in}{1.347500in}}%
\pgfusepath{clip}%
\pgfsetbuttcap%
\pgfsetroundjoin%
\definecolor{currentfill}{rgb}{0.196078,0.803922,0.196078}%
\pgfsetfillcolor{currentfill}%
\pgfsetlinewidth{1.003750pt}%
\definecolor{currentstroke}{rgb}{0.196078,0.803922,0.196078}%
\pgfsetstrokecolor{currentstroke}%
\pgfsetdash{}{0pt}%
\pgfpathmoveto{\pgfqpoint{1.718798in}{0.991899in}}%
\pgfpathcurveto{\pgfqpoint{1.729848in}{0.991899in}}{\pgfqpoint{1.740447in}{0.996290in}}{\pgfqpoint{1.748261in}{1.004103in}}%
\pgfpathcurveto{\pgfqpoint{1.756074in}{1.011917in}}{\pgfqpoint{1.760465in}{1.022516in}}{\pgfqpoint{1.760465in}{1.033566in}}%
\pgfpathcurveto{\pgfqpoint{1.760465in}{1.044616in}}{\pgfqpoint{1.756074in}{1.055215in}}{\pgfqpoint{1.748261in}{1.063029in}}%
\pgfpathcurveto{\pgfqpoint{1.740447in}{1.070842in}}{\pgfqpoint{1.729848in}{1.075233in}}{\pgfqpoint{1.718798in}{1.075233in}}%
\pgfpathcurveto{\pgfqpoint{1.707748in}{1.075233in}}{\pgfqpoint{1.697149in}{1.070842in}}{\pgfqpoint{1.689335in}{1.063029in}}%
\pgfpathcurveto{\pgfqpoint{1.681522in}{1.055215in}}{\pgfqpoint{1.677131in}{1.044616in}}{\pgfqpoint{1.677131in}{1.033566in}}%
\pgfpathcurveto{\pgfqpoint{1.677131in}{1.022516in}}{\pgfqpoint{1.681522in}{1.011917in}}{\pgfqpoint{1.689335in}{1.004103in}}%
\pgfpathcurveto{\pgfqpoint{1.697149in}{0.996290in}}{\pgfqpoint{1.707748in}{0.991899in}}{\pgfqpoint{1.718798in}{0.991899in}}%
\pgfpathclose%
\pgfusepath{stroke,fill}%
\end{pgfscope}%
\begin{pgfscope}%
\pgfpathrectangle{\pgfqpoint{0.772069in}{0.515123in}}{\pgfqpoint{1.937500in}{1.347500in}}%
\pgfusepath{clip}%
\pgfsetbuttcap%
\pgfsetroundjoin%
\definecolor{currentfill}{rgb}{0.196078,0.803922,0.196078}%
\pgfsetfillcolor{currentfill}%
\pgfsetlinewidth{1.003750pt}%
\definecolor{currentstroke}{rgb}{0.196078,0.803922,0.196078}%
\pgfsetstrokecolor{currentstroke}%
\pgfsetdash{}{0pt}%
\pgfpathmoveto{\pgfqpoint{1.879926in}{1.075690in}}%
\pgfpathcurveto{\pgfqpoint{1.890977in}{1.075690in}}{\pgfqpoint{1.901576in}{1.080080in}}{\pgfqpoint{1.909389in}{1.087894in}}%
\pgfpathcurveto{\pgfqpoint{1.917203in}{1.095708in}}{\pgfqpoint{1.921593in}{1.106307in}}{\pgfqpoint{1.921593in}{1.117357in}}%
\pgfpathcurveto{\pgfqpoint{1.921593in}{1.128407in}}{\pgfqpoint{1.917203in}{1.139006in}}{\pgfqpoint{1.909389in}{1.146820in}}%
\pgfpathcurveto{\pgfqpoint{1.901576in}{1.154633in}}{\pgfqpoint{1.890977in}{1.159023in}}{\pgfqpoint{1.879926in}{1.159023in}}%
\pgfpathcurveto{\pgfqpoint{1.868876in}{1.159023in}}{\pgfqpoint{1.858277in}{1.154633in}}{\pgfqpoint{1.850464in}{1.146820in}}%
\pgfpathcurveto{\pgfqpoint{1.842650in}{1.139006in}}{\pgfqpoint{1.838260in}{1.128407in}}{\pgfqpoint{1.838260in}{1.117357in}}%
\pgfpathcurveto{\pgfqpoint{1.838260in}{1.106307in}}{\pgfqpoint{1.842650in}{1.095708in}}{\pgfqpoint{1.850464in}{1.087894in}}%
\pgfpathcurveto{\pgfqpoint{1.858277in}{1.080080in}}{\pgfqpoint{1.868876in}{1.075690in}}{\pgfqpoint{1.879926in}{1.075690in}}%
\pgfpathclose%
\pgfusepath{stroke,fill}%
\end{pgfscope}%
\begin{pgfscope}%
\pgfpathrectangle{\pgfqpoint{0.772069in}{0.515123in}}{\pgfqpoint{1.937500in}{1.347500in}}%
\pgfusepath{clip}%
\pgfsetbuttcap%
\pgfsetroundjoin%
\definecolor{currentfill}{rgb}{0.196078,0.803922,0.196078}%
\pgfsetfillcolor{currentfill}%
\pgfsetlinewidth{1.003750pt}%
\definecolor{currentstroke}{rgb}{0.196078,0.803922,0.196078}%
\pgfsetstrokecolor{currentstroke}%
\pgfsetdash{}{0pt}%
\pgfpathmoveto{\pgfqpoint{2.064073in}{1.166464in}}%
\pgfpathcurveto{\pgfqpoint{2.075123in}{1.166464in}}{\pgfqpoint{2.085723in}{1.170854in}}{\pgfqpoint{2.093536in}{1.178667in}}%
\pgfpathcurveto{\pgfqpoint{2.101350in}{1.186481in}}{\pgfqpoint{2.105740in}{1.197080in}}{\pgfqpoint{2.105740in}{1.208130in}}%
\pgfpathcurveto{\pgfqpoint{2.105740in}{1.219180in}}{\pgfqpoint{2.101350in}{1.229779in}}{\pgfqpoint{2.093536in}{1.237593in}}%
\pgfpathcurveto{\pgfqpoint{2.085723in}{1.245407in}}{\pgfqpoint{2.075123in}{1.249797in}}{\pgfqpoint{2.064073in}{1.249797in}}%
\pgfpathcurveto{\pgfqpoint{2.053023in}{1.249797in}}{\pgfqpoint{2.042424in}{1.245407in}}{\pgfqpoint{2.034611in}{1.237593in}}%
\pgfpathcurveto{\pgfqpoint{2.026797in}{1.229779in}}{\pgfqpoint{2.022407in}{1.219180in}}{\pgfqpoint{2.022407in}{1.208130in}}%
\pgfpathcurveto{\pgfqpoint{2.022407in}{1.197080in}}{\pgfqpoint{2.026797in}{1.186481in}}{\pgfqpoint{2.034611in}{1.178667in}}%
\pgfpathcurveto{\pgfqpoint{2.042424in}{1.170854in}}{\pgfqpoint{2.053023in}{1.166464in}}{\pgfqpoint{2.064073in}{1.166464in}}%
\pgfpathclose%
\pgfusepath{stroke,fill}%
\end{pgfscope}%
\begin{pgfscope}%
\pgfpathrectangle{\pgfqpoint{0.772069in}{0.515123in}}{\pgfqpoint{1.937500in}{1.347500in}}%
\pgfusepath{clip}%
\pgfsetbuttcap%
\pgfsetroundjoin%
\definecolor{currentfill}{rgb}{0.196078,0.803922,0.196078}%
\pgfsetfillcolor{currentfill}%
\pgfsetlinewidth{1.003750pt}%
\definecolor{currentstroke}{rgb}{0.196078,0.803922,0.196078}%
\pgfsetstrokecolor{currentstroke}%
\pgfsetdash{}{0pt}%
\pgfpathmoveto{\pgfqpoint{2.248220in}{1.253746in}}%
\pgfpathcurveto{\pgfqpoint{2.259270in}{1.253746in}}{\pgfqpoint{2.269869in}{1.258136in}}{\pgfqpoint{2.277683in}{1.265950in}}%
\pgfpathcurveto{\pgfqpoint{2.285497in}{1.273763in}}{\pgfqpoint{2.289887in}{1.284362in}}{\pgfqpoint{2.289887in}{1.295412in}}%
\pgfpathcurveto{\pgfqpoint{2.289887in}{1.306462in}}{\pgfqpoint{2.285497in}{1.317061in}}{\pgfqpoint{2.277683in}{1.324875in}}%
\pgfpathcurveto{\pgfqpoint{2.269869in}{1.332689in}}{\pgfqpoint{2.259270in}{1.337079in}}{\pgfqpoint{2.248220in}{1.337079in}}%
\pgfpathcurveto{\pgfqpoint{2.237170in}{1.337079in}}{\pgfqpoint{2.226571in}{1.332689in}}{\pgfqpoint{2.218757in}{1.324875in}}%
\pgfpathcurveto{\pgfqpoint{2.210944in}{1.317061in}}{\pgfqpoint{2.206554in}{1.306462in}}{\pgfqpoint{2.206554in}{1.295412in}}%
\pgfpathcurveto{\pgfqpoint{2.206554in}{1.284362in}}{\pgfqpoint{2.210944in}{1.273763in}}{\pgfqpoint{2.218757in}{1.265950in}}%
\pgfpathcurveto{\pgfqpoint{2.226571in}{1.258136in}}{\pgfqpoint{2.237170in}{1.253746in}}{\pgfqpoint{2.248220in}{1.253746in}}%
\pgfpathclose%
\pgfusepath{stroke,fill}%
\end{pgfscope}%
\begin{pgfscope}%
\pgfpathrectangle{\pgfqpoint{0.772069in}{0.515123in}}{\pgfqpoint{1.937500in}{1.347500in}}%
\pgfusepath{clip}%
\pgfsetbuttcap%
\pgfsetroundjoin%
\definecolor{currentfill}{rgb}{0.196078,0.803922,0.196078}%
\pgfsetfillcolor{currentfill}%
\pgfsetlinewidth{1.003750pt}%
\definecolor{currentstroke}{rgb}{0.196078,0.803922,0.196078}%
\pgfsetstrokecolor{currentstroke}%
\pgfsetdash{}{0pt}%
\pgfpathmoveto{\pgfqpoint{2.409349in}{1.339864in}}%
\pgfpathcurveto{\pgfqpoint{2.420399in}{1.339864in}}{\pgfqpoint{2.430998in}{1.344254in}}{\pgfqpoint{2.438812in}{1.352068in}}%
\pgfpathcurveto{\pgfqpoint{2.446625in}{1.359881in}}{\pgfqpoint{2.451015in}{1.370481in}}{\pgfqpoint{2.451015in}{1.381531in}}%
\pgfpathcurveto{\pgfqpoint{2.451015in}{1.392581in}}{\pgfqpoint{2.446625in}{1.403180in}}{\pgfqpoint{2.438812in}{1.410993in}}%
\pgfpathcurveto{\pgfqpoint{2.430998in}{1.418807in}}{\pgfqpoint{2.420399in}{1.423197in}}{\pgfqpoint{2.409349in}{1.423197in}}%
\pgfpathcurveto{\pgfqpoint{2.398299in}{1.423197in}}{\pgfqpoint{2.387700in}{1.418807in}}{\pgfqpoint{2.379886in}{1.410993in}}%
\pgfpathcurveto{\pgfqpoint{2.372072in}{1.403180in}}{\pgfqpoint{2.367682in}{1.392581in}}{\pgfqpoint{2.367682in}{1.381531in}}%
\pgfpathcurveto{\pgfqpoint{2.367682in}{1.370481in}}{\pgfqpoint{2.372072in}{1.359881in}}{\pgfqpoint{2.379886in}{1.352068in}}%
\pgfpathcurveto{\pgfqpoint{2.387700in}{1.344254in}}{\pgfqpoint{2.398299in}{1.339864in}}{\pgfqpoint{2.409349in}{1.339864in}}%
\pgfpathclose%
\pgfusepath{stroke,fill}%
\end{pgfscope}%
\begin{pgfscope}%
\pgfpathrectangle{\pgfqpoint{0.772069in}{0.515123in}}{\pgfqpoint{1.937500in}{1.347500in}}%
\pgfusepath{clip}%
\pgfsetbuttcap%
\pgfsetroundjoin%
\definecolor{currentfill}{rgb}{0.196078,0.803922,0.196078}%
\pgfsetfillcolor{currentfill}%
\pgfsetlinewidth{1.003750pt}%
\definecolor{currentstroke}{rgb}{0.196078,0.803922,0.196078}%
\pgfsetstrokecolor{currentstroke}%
\pgfsetdash{}{0pt}%
\pgfpathmoveto{\pgfqpoint{2.593496in}{1.428310in}}%
\pgfpathcurveto{\pgfqpoint{2.604546in}{1.428310in}}{\pgfqpoint{2.615145in}{1.432700in}}{\pgfqpoint{2.622958in}{1.440514in}}%
\pgfpathcurveto{\pgfqpoint{2.630772in}{1.448327in}}{\pgfqpoint{2.635162in}{1.458926in}}{\pgfqpoint{2.635162in}{1.469977in}}%
\pgfpathcurveto{\pgfqpoint{2.635162in}{1.481027in}}{\pgfqpoint{2.630772in}{1.491626in}}{\pgfqpoint{2.622958in}{1.499439in}}%
\pgfpathcurveto{\pgfqpoint{2.615145in}{1.507253in}}{\pgfqpoint{2.604546in}{1.511643in}}{\pgfqpoint{2.593496in}{1.511643in}}%
\pgfpathcurveto{\pgfqpoint{2.582446in}{1.511643in}}{\pgfqpoint{2.571847in}{1.507253in}}{\pgfqpoint{2.564033in}{1.499439in}}%
\pgfpathcurveto{\pgfqpoint{2.556219in}{1.491626in}}{\pgfqpoint{2.551829in}{1.481027in}}{\pgfqpoint{2.551829in}{1.469977in}}%
\pgfpathcurveto{\pgfqpoint{2.551829in}{1.458926in}}{\pgfqpoint{2.556219in}{1.448327in}}{\pgfqpoint{2.564033in}{1.440514in}}%
\pgfpathcurveto{\pgfqpoint{2.571847in}{1.432700in}}{\pgfqpoint{2.582446in}{1.428310in}}{\pgfqpoint{2.593496in}{1.428310in}}%
\pgfpathclose%
\pgfusepath{stroke,fill}%
\end{pgfscope}%
\begin{pgfscope}%
\pgfpathrectangle{\pgfqpoint{0.772069in}{0.515123in}}{\pgfqpoint{1.937500in}{1.347500in}}%
\pgfusepath{clip}%
\pgfsetbuttcap%
\pgfsetroundjoin%
\definecolor{currentfill}{rgb}{0.117647,0.564706,1.000000}%
\pgfsetfillcolor{currentfill}%
\pgfsetlinewidth{1.003750pt}%
\definecolor{currentstroke}{rgb}{0.117647,0.564706,1.000000}%
\pgfsetstrokecolor{currentstroke}%
\pgfsetdash{}{0pt}%
\pgfpathmoveto{\pgfqpoint{0.936174in}{0.672098in}}%
\pgfpathcurveto{\pgfqpoint{0.947224in}{0.672098in}}{\pgfqpoint{0.957823in}{0.676488in}}{\pgfqpoint{0.965636in}{0.684302in}}%
\pgfpathcurveto{\pgfqpoint{0.973450in}{0.692115in}}{\pgfqpoint{0.977840in}{0.702714in}}{\pgfqpoint{0.977840in}{0.713764in}}%
\pgfpathcurveto{\pgfqpoint{0.977840in}{0.724814in}}{\pgfqpoint{0.973450in}{0.735414in}}{\pgfqpoint{0.965636in}{0.743227in}}%
\pgfpathcurveto{\pgfqpoint{0.957823in}{0.751041in}}{\pgfqpoint{0.947224in}{0.755431in}}{\pgfqpoint{0.936174in}{0.755431in}}%
\pgfpathcurveto{\pgfqpoint{0.925123in}{0.755431in}}{\pgfqpoint{0.914524in}{0.751041in}}{\pgfqpoint{0.906711in}{0.743227in}}%
\pgfpathcurveto{\pgfqpoint{0.898897in}{0.735414in}}{\pgfqpoint{0.894507in}{0.724814in}}{\pgfqpoint{0.894507in}{0.713764in}}%
\pgfpathcurveto{\pgfqpoint{0.894507in}{0.702714in}}{\pgfqpoint{0.898897in}{0.692115in}}{\pgfqpoint{0.906711in}{0.684302in}}%
\pgfpathcurveto{\pgfqpoint{0.914524in}{0.676488in}}{\pgfqpoint{0.925123in}{0.672098in}}{\pgfqpoint{0.936174in}{0.672098in}}%
\pgfpathclose%
\pgfusepath{stroke,fill}%
\end{pgfscope}%
\begin{pgfscope}%
\pgfpathrectangle{\pgfqpoint{0.772069in}{0.515123in}}{\pgfqpoint{1.937500in}{1.347500in}}%
\pgfusepath{clip}%
\pgfsetbuttcap%
\pgfsetroundjoin%
\definecolor{currentfill}{rgb}{0.117647,0.564706,1.000000}%
\pgfsetfillcolor{currentfill}%
\pgfsetlinewidth{1.003750pt}%
\definecolor{currentstroke}{rgb}{0.117647,0.564706,1.000000}%
\pgfsetstrokecolor{currentstroke}%
\pgfsetdash{}{0pt}%
\pgfpathmoveto{\pgfqpoint{1.051265in}{0.784750in}}%
\pgfpathcurveto{\pgfqpoint{1.062316in}{0.784750in}}{\pgfqpoint{1.072915in}{0.789140in}}{\pgfqpoint{1.080728in}{0.796954in}}%
\pgfpathcurveto{\pgfqpoint{1.088542in}{0.804767in}}{\pgfqpoint{1.092932in}{0.815366in}}{\pgfqpoint{1.092932in}{0.826416in}}%
\pgfpathcurveto{\pgfqpoint{1.092932in}{0.837467in}}{\pgfqpoint{1.088542in}{0.848066in}}{\pgfqpoint{1.080728in}{0.855879in}}%
\pgfpathcurveto{\pgfqpoint{1.072915in}{0.863693in}}{\pgfqpoint{1.062316in}{0.868083in}}{\pgfqpoint{1.051265in}{0.868083in}}%
\pgfpathcurveto{\pgfqpoint{1.040215in}{0.868083in}}{\pgfqpoint{1.029616in}{0.863693in}}{\pgfqpoint{1.021803in}{0.855879in}}%
\pgfpathcurveto{\pgfqpoint{1.013989in}{0.848066in}}{\pgfqpoint{1.009599in}{0.837467in}}{\pgfqpoint{1.009599in}{0.826416in}}%
\pgfpathcurveto{\pgfqpoint{1.009599in}{0.815366in}}{\pgfqpoint{1.013989in}{0.804767in}}{\pgfqpoint{1.021803in}{0.796954in}}%
\pgfpathcurveto{\pgfqpoint{1.029616in}{0.789140in}}{\pgfqpoint{1.040215in}{0.784750in}}{\pgfqpoint{1.051265in}{0.784750in}}%
\pgfpathclose%
\pgfusepath{stroke,fill}%
\end{pgfscope}%
\begin{pgfscope}%
\pgfpathrectangle{\pgfqpoint{0.772069in}{0.515123in}}{\pgfqpoint{1.937500in}{1.347500in}}%
\pgfusepath{clip}%
\pgfsetbuttcap%
\pgfsetroundjoin%
\definecolor{currentfill}{rgb}{0.117647,0.564706,1.000000}%
\pgfsetfillcolor{currentfill}%
\pgfsetlinewidth{1.003750pt}%
\definecolor{currentstroke}{rgb}{0.117647,0.564706,1.000000}%
\pgfsetstrokecolor{currentstroke}%
\pgfsetdash{}{0pt}%
\pgfpathmoveto{\pgfqpoint{1.166357in}{0.905781in}}%
\pgfpathcurveto{\pgfqpoint{1.177407in}{0.905781in}}{\pgfqpoint{1.188006in}{0.910171in}}{\pgfqpoint{1.195820in}{0.917985in}}%
\pgfpathcurveto{\pgfqpoint{1.203634in}{0.925798in}}{\pgfqpoint{1.208024in}{0.936398in}}{\pgfqpoint{1.208024in}{0.947448in}}%
\pgfpathcurveto{\pgfqpoint{1.208024in}{0.958498in}}{\pgfqpoint{1.203634in}{0.969097in}}{\pgfqpoint{1.195820in}{0.976910in}}%
\pgfpathcurveto{\pgfqpoint{1.188006in}{0.984724in}}{\pgfqpoint{1.177407in}{0.989114in}}{\pgfqpoint{1.166357in}{0.989114in}}%
\pgfpathcurveto{\pgfqpoint{1.155307in}{0.989114in}}{\pgfqpoint{1.144708in}{0.984724in}}{\pgfqpoint{1.136894in}{0.976910in}}%
\pgfpathcurveto{\pgfqpoint{1.129081in}{0.969097in}}{\pgfqpoint{1.124691in}{0.958498in}}{\pgfqpoint{1.124691in}{0.947448in}}%
\pgfpathcurveto{\pgfqpoint{1.124691in}{0.936398in}}{\pgfqpoint{1.129081in}{0.925798in}}{\pgfqpoint{1.136894in}{0.917985in}}%
\pgfpathcurveto{\pgfqpoint{1.144708in}{0.910171in}}{\pgfqpoint{1.155307in}{0.905781in}}{\pgfqpoint{1.166357in}{0.905781in}}%
\pgfpathclose%
\pgfusepath{stroke,fill}%
\end{pgfscope}%
\begin{pgfscope}%
\pgfpathrectangle{\pgfqpoint{0.772069in}{0.515123in}}{\pgfqpoint{1.937500in}{1.347500in}}%
\pgfusepath{clip}%
\pgfsetbuttcap%
\pgfsetroundjoin%
\definecolor{currentfill}{rgb}{0.117647,0.564706,1.000000}%
\pgfsetfillcolor{currentfill}%
\pgfsetlinewidth{1.003750pt}%
\definecolor{currentstroke}{rgb}{0.117647,0.564706,1.000000}%
\pgfsetstrokecolor{currentstroke}%
\pgfsetdash{}{0pt}%
\pgfpathmoveto{\pgfqpoint{1.281449in}{1.024485in}}%
\pgfpathcurveto{\pgfqpoint{1.292499in}{1.024485in}}{\pgfqpoint{1.303098in}{1.028875in}}{\pgfqpoint{1.310912in}{1.036689in}}%
\pgfpathcurveto{\pgfqpoint{1.318725in}{1.044502in}}{\pgfqpoint{1.323116in}{1.055101in}}{\pgfqpoint{1.323116in}{1.066151in}}%
\pgfpathcurveto{\pgfqpoint{1.323116in}{1.077201in}}{\pgfqpoint{1.318725in}{1.087800in}}{\pgfqpoint{1.310912in}{1.095614in}}%
\pgfpathcurveto{\pgfqpoint{1.303098in}{1.103428in}}{\pgfqpoint{1.292499in}{1.107818in}}{\pgfqpoint{1.281449in}{1.107818in}}%
\pgfpathcurveto{\pgfqpoint{1.270399in}{1.107818in}}{\pgfqpoint{1.259800in}{1.103428in}}{\pgfqpoint{1.251986in}{1.095614in}}%
\pgfpathcurveto{\pgfqpoint{1.244173in}{1.087800in}}{\pgfqpoint{1.239782in}{1.077201in}}{\pgfqpoint{1.239782in}{1.066151in}}%
\pgfpathcurveto{\pgfqpoint{1.239782in}{1.055101in}}{\pgfqpoint{1.244173in}{1.044502in}}{\pgfqpoint{1.251986in}{1.036689in}}%
\pgfpathcurveto{\pgfqpoint{1.259800in}{1.028875in}}{\pgfqpoint{1.270399in}{1.024485in}}{\pgfqpoint{1.281449in}{1.024485in}}%
\pgfpathclose%
\pgfusepath{stroke,fill}%
\end{pgfscope}%
\begin{pgfscope}%
\pgfpathrectangle{\pgfqpoint{0.772069in}{0.515123in}}{\pgfqpoint{1.937500in}{1.347500in}}%
\pgfusepath{clip}%
\pgfsetbuttcap%
\pgfsetroundjoin%
\definecolor{currentfill}{rgb}{0.117647,0.564706,1.000000}%
\pgfsetfillcolor{currentfill}%
\pgfsetlinewidth{1.003750pt}%
\definecolor{currentstroke}{rgb}{0.117647,0.564706,1.000000}%
\pgfsetstrokecolor{currentstroke}%
\pgfsetdash{}{0pt}%
\pgfpathmoveto{\pgfqpoint{1.396541in}{1.143188in}}%
\pgfpathcurveto{\pgfqpoint{1.407591in}{1.143188in}}{\pgfqpoint{1.418190in}{1.147579in}}{\pgfqpoint{1.426004in}{1.155392in}}%
\pgfpathcurveto{\pgfqpoint{1.433817in}{1.163206in}}{\pgfqpoint{1.438208in}{1.173805in}}{\pgfqpoint{1.438208in}{1.184855in}}%
\pgfpathcurveto{\pgfqpoint{1.438208in}{1.195905in}}{\pgfqpoint{1.433817in}{1.206504in}}{\pgfqpoint{1.426004in}{1.214318in}}%
\pgfpathcurveto{\pgfqpoint{1.418190in}{1.222131in}}{\pgfqpoint{1.407591in}{1.226522in}}{\pgfqpoint{1.396541in}{1.226522in}}%
\pgfpathcurveto{\pgfqpoint{1.385491in}{1.226522in}}{\pgfqpoint{1.374892in}{1.222131in}}{\pgfqpoint{1.367078in}{1.214318in}}%
\pgfpathcurveto{\pgfqpoint{1.359264in}{1.206504in}}{\pgfqpoint{1.354874in}{1.195905in}}{\pgfqpoint{1.354874in}{1.184855in}}%
\pgfpathcurveto{\pgfqpoint{1.354874in}{1.173805in}}{\pgfqpoint{1.359264in}{1.163206in}}{\pgfqpoint{1.367078in}{1.155392in}}%
\pgfpathcurveto{\pgfqpoint{1.374892in}{1.147579in}}{\pgfqpoint{1.385491in}{1.143188in}}{\pgfqpoint{1.396541in}{1.143188in}}%
\pgfpathclose%
\pgfusepath{stroke,fill}%
\end{pgfscope}%
\begin{pgfscope}%
\pgfpathrectangle{\pgfqpoint{0.772069in}{0.515123in}}{\pgfqpoint{1.937500in}{1.347500in}}%
\pgfusepath{clip}%
\pgfsetbuttcap%
\pgfsetroundjoin%
\definecolor{currentfill}{rgb}{0.117647,0.564706,1.000000}%
\pgfsetfillcolor{currentfill}%
\pgfsetlinewidth{1.003750pt}%
\definecolor{currentstroke}{rgb}{0.117647,0.564706,1.000000}%
\pgfsetstrokecolor{currentstroke}%
\pgfsetdash{}{0pt}%
\pgfpathmoveto{\pgfqpoint{1.511633in}{1.256073in}}%
\pgfpathcurveto{\pgfqpoint{1.522683in}{1.256073in}}{\pgfqpoint{1.533282in}{1.260463in}}{\pgfqpoint{1.541095in}{1.268277in}}%
\pgfpathcurveto{\pgfqpoint{1.548909in}{1.276091in}}{\pgfqpoint{1.553299in}{1.286690in}}{\pgfqpoint{1.553299in}{1.297740in}}%
\pgfpathcurveto{\pgfqpoint{1.553299in}{1.308790in}}{\pgfqpoint{1.548909in}{1.319389in}}{\pgfqpoint{1.541095in}{1.327203in}}%
\pgfpathcurveto{\pgfqpoint{1.533282in}{1.335016in}}{\pgfqpoint{1.522683in}{1.339406in}}{\pgfqpoint{1.511633in}{1.339406in}}%
\pgfpathcurveto{\pgfqpoint{1.500583in}{1.339406in}}{\pgfqpoint{1.489983in}{1.335016in}}{\pgfqpoint{1.482170in}{1.327203in}}%
\pgfpathcurveto{\pgfqpoint{1.474356in}{1.319389in}}{\pgfqpoint{1.469966in}{1.308790in}}{\pgfqpoint{1.469966in}{1.297740in}}%
\pgfpathcurveto{\pgfqpoint{1.469966in}{1.286690in}}{\pgfqpoint{1.474356in}{1.276091in}}{\pgfqpoint{1.482170in}{1.268277in}}%
\pgfpathcurveto{\pgfqpoint{1.489983in}{1.260463in}}{\pgfqpoint{1.500583in}{1.256073in}}{\pgfqpoint{1.511633in}{1.256073in}}%
\pgfpathclose%
\pgfusepath{stroke,fill}%
\end{pgfscope}%
\begin{pgfscope}%
\pgfpathrectangle{\pgfqpoint{0.772069in}{0.515123in}}{\pgfqpoint{1.937500in}{1.347500in}}%
\pgfusepath{clip}%
\pgfsetbuttcap%
\pgfsetroundjoin%
\definecolor{currentfill}{rgb}{0.117647,0.564706,1.000000}%
\pgfsetfillcolor{currentfill}%
\pgfsetlinewidth{1.003750pt}%
\definecolor{currentstroke}{rgb}{0.117647,0.564706,1.000000}%
\pgfsetstrokecolor{currentstroke}%
\pgfsetdash{}{0pt}%
\pgfpathmoveto{\pgfqpoint{1.626724in}{1.378268in}}%
\pgfpathcurveto{\pgfqpoint{1.637775in}{1.378268in}}{\pgfqpoint{1.648374in}{1.382658in}}{\pgfqpoint{1.656187in}{1.390472in}}%
\pgfpathcurveto{\pgfqpoint{1.664001in}{1.398286in}}{\pgfqpoint{1.668391in}{1.408885in}}{\pgfqpoint{1.668391in}{1.419935in}}%
\pgfpathcurveto{\pgfqpoint{1.668391in}{1.430985in}}{\pgfqpoint{1.664001in}{1.441584in}}{\pgfqpoint{1.656187in}{1.449398in}}%
\pgfpathcurveto{\pgfqpoint{1.648374in}{1.457211in}}{\pgfqpoint{1.637775in}{1.461601in}}{\pgfqpoint{1.626724in}{1.461601in}}%
\pgfpathcurveto{\pgfqpoint{1.615674in}{1.461601in}}{\pgfqpoint{1.605075in}{1.457211in}}{\pgfqpoint{1.597262in}{1.449398in}}%
\pgfpathcurveto{\pgfqpoint{1.589448in}{1.441584in}}{\pgfqpoint{1.585058in}{1.430985in}}{\pgfqpoint{1.585058in}{1.419935in}}%
\pgfpathcurveto{\pgfqpoint{1.585058in}{1.408885in}}{\pgfqpoint{1.589448in}{1.398286in}}{\pgfqpoint{1.597262in}{1.390472in}}%
\pgfpathcurveto{\pgfqpoint{1.605075in}{1.382658in}}{\pgfqpoint{1.615674in}{1.378268in}}{\pgfqpoint{1.626724in}{1.378268in}}%
\pgfpathclose%
\pgfusepath{stroke,fill}%
\end{pgfscope}%
\begin{pgfscope}%
\pgfpathrectangle{\pgfqpoint{0.772069in}{0.515123in}}{\pgfqpoint{1.937500in}{1.347500in}}%
\pgfusepath{clip}%
\pgfsetbuttcap%
\pgfsetroundjoin%
\definecolor{currentfill}{rgb}{0.117647,0.564706,1.000000}%
\pgfsetfillcolor{currentfill}%
\pgfsetlinewidth{1.003750pt}%
\definecolor{currentstroke}{rgb}{0.117647,0.564706,1.000000}%
\pgfsetstrokecolor{currentstroke}%
\pgfsetdash{}{0pt}%
\pgfpathmoveto{\pgfqpoint{1.741816in}{1.495808in}}%
\pgfpathcurveto{\pgfqpoint{1.752866in}{1.495808in}}{\pgfqpoint{1.763465in}{1.500198in}}{\pgfqpoint{1.771279in}{1.508012in}}%
\pgfpathcurveto{\pgfqpoint{1.779093in}{1.515826in}}{\pgfqpoint{1.783483in}{1.526425in}}{\pgfqpoint{1.783483in}{1.537475in}}%
\pgfpathcurveto{\pgfqpoint{1.783483in}{1.548525in}}{\pgfqpoint{1.779093in}{1.559124in}}{\pgfqpoint{1.771279in}{1.566937in}}%
\pgfpathcurveto{\pgfqpoint{1.763465in}{1.574751in}}{\pgfqpoint{1.752866in}{1.579141in}}{\pgfqpoint{1.741816in}{1.579141in}}%
\pgfpathcurveto{\pgfqpoint{1.730766in}{1.579141in}}{\pgfqpoint{1.720167in}{1.574751in}}{\pgfqpoint{1.712353in}{1.566937in}}%
\pgfpathcurveto{\pgfqpoint{1.704540in}{1.559124in}}{\pgfqpoint{1.700150in}{1.548525in}}{\pgfqpoint{1.700150in}{1.537475in}}%
\pgfpathcurveto{\pgfqpoint{1.700150in}{1.526425in}}{\pgfqpoint{1.704540in}{1.515826in}}{\pgfqpoint{1.712353in}{1.508012in}}%
\pgfpathcurveto{\pgfqpoint{1.720167in}{1.500198in}}{\pgfqpoint{1.730766in}{1.495808in}}{\pgfqpoint{1.741816in}{1.495808in}}%
\pgfpathclose%
\pgfusepath{stroke,fill}%
\end{pgfscope}%
\begin{pgfscope}%
\pgfpathrectangle{\pgfqpoint{0.772069in}{0.515123in}}{\pgfqpoint{1.937500in}{1.347500in}}%
\pgfusepath{clip}%
\pgfsetbuttcap%
\pgfsetroundjoin%
\definecolor{currentfill}{rgb}{0.117647,0.564706,1.000000}%
\pgfsetfillcolor{currentfill}%
\pgfsetlinewidth{1.003750pt}%
\definecolor{currentstroke}{rgb}{0.117647,0.564706,1.000000}%
\pgfsetstrokecolor{currentstroke}%
\pgfsetdash{}{0pt}%
\pgfpathmoveto{\pgfqpoint{1.856908in}{1.611020in}}%
\pgfpathcurveto{\pgfqpoint{1.867958in}{1.611020in}}{\pgfqpoint{1.878557in}{1.615411in}}{\pgfqpoint{1.886371in}{1.623224in}}%
\pgfpathcurveto{\pgfqpoint{1.894184in}{1.631038in}}{\pgfqpoint{1.898575in}{1.641637in}}{\pgfqpoint{1.898575in}{1.652687in}}%
\pgfpathcurveto{\pgfqpoint{1.898575in}{1.663737in}}{\pgfqpoint{1.894184in}{1.674336in}}{\pgfqpoint{1.886371in}{1.682150in}}%
\pgfpathcurveto{\pgfqpoint{1.878557in}{1.689963in}}{\pgfqpoint{1.867958in}{1.694354in}}{\pgfqpoint{1.856908in}{1.694354in}}%
\pgfpathcurveto{\pgfqpoint{1.845858in}{1.694354in}}{\pgfqpoint{1.835259in}{1.689963in}}{\pgfqpoint{1.827445in}{1.682150in}}%
\pgfpathcurveto{\pgfqpoint{1.819632in}{1.674336in}}{\pgfqpoint{1.815241in}{1.663737in}}{\pgfqpoint{1.815241in}{1.652687in}}%
\pgfpathcurveto{\pgfqpoint{1.815241in}{1.641637in}}{\pgfqpoint{1.819632in}{1.631038in}}{\pgfqpoint{1.827445in}{1.623224in}}%
\pgfpathcurveto{\pgfqpoint{1.835259in}{1.615411in}}{\pgfqpoint{1.845858in}{1.611020in}}{\pgfqpoint{1.856908in}{1.611020in}}%
\pgfpathclose%
\pgfusepath{stroke,fill}%
\end{pgfscope}%
\begin{pgfscope}%
\pgfpathrectangle{\pgfqpoint{0.772069in}{0.515123in}}{\pgfqpoint{1.937500in}{1.347500in}}%
\pgfusepath{clip}%
\pgfsetbuttcap%
\pgfsetroundjoin%
\definecolor{currentfill}{rgb}{0.117647,0.564706,1.000000}%
\pgfsetfillcolor{currentfill}%
\pgfsetlinewidth{1.003750pt}%
\definecolor{currentstroke}{rgb}{0.117647,0.564706,1.000000}%
\pgfsetstrokecolor{currentstroke}%
\pgfsetdash{}{0pt}%
\pgfpathmoveto{\pgfqpoint{1.972000in}{1.729724in}}%
\pgfpathcurveto{\pgfqpoint{1.983050in}{1.729724in}}{\pgfqpoint{1.993649in}{1.734114in}}{\pgfqpoint{2.001463in}{1.741928in}}%
\pgfpathcurveto{\pgfqpoint{2.009276in}{1.749742in}}{\pgfqpoint{2.013667in}{1.760341in}}{\pgfqpoint{2.013667in}{1.771391in}}%
\pgfpathcurveto{\pgfqpoint{2.013667in}{1.782441in}}{\pgfqpoint{2.009276in}{1.793040in}}{\pgfqpoint{2.001463in}{1.800854in}}%
\pgfpathcurveto{\pgfqpoint{1.993649in}{1.808667in}}{\pgfqpoint{1.983050in}{1.813057in}}{\pgfqpoint{1.972000in}{1.813057in}}%
\pgfpathcurveto{\pgfqpoint{1.960950in}{1.813057in}}{\pgfqpoint{1.950351in}{1.808667in}}{\pgfqpoint{1.942537in}{1.800854in}}%
\pgfpathcurveto{\pgfqpoint{1.934723in}{1.793040in}}{\pgfqpoint{1.930333in}{1.782441in}}{\pgfqpoint{1.930333in}{1.771391in}}%
\pgfpathcurveto{\pgfqpoint{1.930333in}{1.760341in}}{\pgfqpoint{1.934723in}{1.749742in}}{\pgfqpoint{1.942537in}{1.741928in}}%
\pgfpathcurveto{\pgfqpoint{1.950351in}{1.734114in}}{\pgfqpoint{1.960950in}{1.729724in}}{\pgfqpoint{1.972000in}{1.729724in}}%
\pgfpathclose%
\pgfusepath{stroke,fill}%
\end{pgfscope}%
\begin{pgfscope}%
\pgfpathrectangle{\pgfqpoint{0.772069in}{0.515123in}}{\pgfqpoint{1.937500in}{1.347500in}}%
\pgfusepath{clip}%
\pgfsetbuttcap%
\pgfsetroundjoin%
\definecolor{currentfill}{rgb}{1.000000,0.627451,0.478431}%
\pgfsetfillcolor{currentfill}%
\pgfsetlinewidth{1.003750pt}%
\definecolor{currentstroke}{rgb}{1.000000,0.627451,0.478431}%
\pgfsetstrokecolor{currentstroke}%
\pgfsetdash{}{0pt}%
\pgfpathmoveto{\pgfqpoint{0.936174in}{0.567906in}}%
\pgfpathcurveto{\pgfqpoint{0.947224in}{0.567906in}}{\pgfqpoint{0.957823in}{0.572296in}}{\pgfqpoint{0.965636in}{0.580110in}}%
\pgfpathcurveto{\pgfqpoint{0.973450in}{0.587924in}}{\pgfqpoint{0.977840in}{0.598523in}}{\pgfqpoint{0.977840in}{0.609573in}}%
\pgfpathcurveto{\pgfqpoint{0.977840in}{0.620623in}}{\pgfqpoint{0.973450in}{0.631222in}}{\pgfqpoint{0.965636in}{0.639036in}}%
\pgfpathcurveto{\pgfqpoint{0.957823in}{0.646849in}}{\pgfqpoint{0.947224in}{0.651239in}}{\pgfqpoint{0.936174in}{0.651239in}}%
\pgfpathcurveto{\pgfqpoint{0.925123in}{0.651239in}}{\pgfqpoint{0.914524in}{0.646849in}}{\pgfqpoint{0.906711in}{0.639036in}}%
\pgfpathcurveto{\pgfqpoint{0.898897in}{0.631222in}}{\pgfqpoint{0.894507in}{0.620623in}}{\pgfqpoint{0.894507in}{0.609573in}}%
\pgfpathcurveto{\pgfqpoint{0.894507in}{0.598523in}}{\pgfqpoint{0.898897in}{0.587924in}}{\pgfqpoint{0.906711in}{0.580110in}}%
\pgfpathcurveto{\pgfqpoint{0.914524in}{0.572296in}}{\pgfqpoint{0.925123in}{0.567906in}}{\pgfqpoint{0.936174in}{0.567906in}}%
\pgfpathclose%
\pgfusepath{stroke,fill}%
\end{pgfscope}%
\begin{pgfscope}%
\pgfpathrectangle{\pgfqpoint{0.772069in}{0.515123in}}{\pgfqpoint{1.937500in}{1.347500in}}%
\pgfusepath{clip}%
\pgfsetbuttcap%
\pgfsetroundjoin%
\definecolor{currentfill}{rgb}{1.000000,0.627451,0.478431}%
\pgfsetfillcolor{currentfill}%
\pgfsetlinewidth{1.003750pt}%
\definecolor{currentstroke}{rgb}{1.000000,0.627451,0.478431}%
\pgfsetstrokecolor{currentstroke}%
\pgfsetdash{}{0pt}%
\pgfpathmoveto{\pgfqpoint{1.051265in}{0.582372in}}%
\pgfpathcurveto{\pgfqpoint{1.062316in}{0.582372in}}{\pgfqpoint{1.072915in}{0.586762in}}{\pgfqpoint{1.080728in}{0.594576in}}%
\pgfpathcurveto{\pgfqpoint{1.088542in}{0.602389in}}{\pgfqpoint{1.092932in}{0.612988in}}{\pgfqpoint{1.092932in}{0.624038in}}%
\pgfpathcurveto{\pgfqpoint{1.092932in}{0.635088in}}{\pgfqpoint{1.088542in}{0.645688in}}{\pgfqpoint{1.080728in}{0.653501in}}%
\pgfpathcurveto{\pgfqpoint{1.072915in}{0.661315in}}{\pgfqpoint{1.062316in}{0.665705in}}{\pgfqpoint{1.051265in}{0.665705in}}%
\pgfpathcurveto{\pgfqpoint{1.040215in}{0.665705in}}{\pgfqpoint{1.029616in}{0.661315in}}{\pgfqpoint{1.021803in}{0.653501in}}%
\pgfpathcurveto{\pgfqpoint{1.013989in}{0.645688in}}{\pgfqpoint{1.009599in}{0.635088in}}{\pgfqpoint{1.009599in}{0.624038in}}%
\pgfpathcurveto{\pgfqpoint{1.009599in}{0.612988in}}{\pgfqpoint{1.013989in}{0.602389in}}{\pgfqpoint{1.021803in}{0.594576in}}%
\pgfpathcurveto{\pgfqpoint{1.029616in}{0.586762in}}{\pgfqpoint{1.040215in}{0.582372in}}{\pgfqpoint{1.051265in}{0.582372in}}%
\pgfpathclose%
\pgfusepath{stroke,fill}%
\end{pgfscope}%
\begin{pgfscope}%
\pgfpathrectangle{\pgfqpoint{0.772069in}{0.515123in}}{\pgfqpoint{1.937500in}{1.347500in}}%
\pgfusepath{clip}%
\pgfsetbuttcap%
\pgfsetroundjoin%
\definecolor{currentfill}{rgb}{1.000000,0.627451,0.478431}%
\pgfsetfillcolor{currentfill}%
\pgfsetlinewidth{1.003750pt}%
\definecolor{currentstroke}{rgb}{1.000000,0.627451,0.478431}%
\pgfsetstrokecolor{currentstroke}%
\pgfsetdash{}{0pt}%
\pgfpathmoveto{\pgfqpoint{1.166357in}{0.586561in}}%
\pgfpathcurveto{\pgfqpoint{1.177407in}{0.586561in}}{\pgfqpoint{1.188006in}{0.590951in}}{\pgfqpoint{1.195820in}{0.598765in}}%
\pgfpathcurveto{\pgfqpoint{1.203634in}{0.606579in}}{\pgfqpoint{1.208024in}{0.617178in}}{\pgfqpoint{1.208024in}{0.628228in}}%
\pgfpathcurveto{\pgfqpoint{1.208024in}{0.639278in}}{\pgfqpoint{1.203634in}{0.649877in}}{\pgfqpoint{1.195820in}{0.657691in}}%
\pgfpathcurveto{\pgfqpoint{1.188006in}{0.665504in}}{\pgfqpoint{1.177407in}{0.669895in}}{\pgfqpoint{1.166357in}{0.669895in}}%
\pgfpathcurveto{\pgfqpoint{1.155307in}{0.669895in}}{\pgfqpoint{1.144708in}{0.665504in}}{\pgfqpoint{1.136894in}{0.657691in}}%
\pgfpathcurveto{\pgfqpoint{1.129081in}{0.649877in}}{\pgfqpoint{1.124691in}{0.639278in}}{\pgfqpoint{1.124691in}{0.628228in}}%
\pgfpathcurveto{\pgfqpoint{1.124691in}{0.617178in}}{\pgfqpoint{1.129081in}{0.606579in}}{\pgfqpoint{1.136894in}{0.598765in}}%
\pgfpathcurveto{\pgfqpoint{1.144708in}{0.590951in}}{\pgfqpoint{1.155307in}{0.586561in}}{\pgfqpoint{1.166357in}{0.586561in}}%
\pgfpathclose%
\pgfusepath{stroke,fill}%
\end{pgfscope}%
\begin{pgfscope}%
\pgfpathrectangle{\pgfqpoint{0.772069in}{0.515123in}}{\pgfqpoint{1.937500in}{1.347500in}}%
\pgfusepath{clip}%
\pgfsetbuttcap%
\pgfsetroundjoin%
\definecolor{currentfill}{rgb}{1.000000,0.627451,0.478431}%
\pgfsetfillcolor{currentfill}%
\pgfsetlinewidth{1.003750pt}%
\definecolor{currentstroke}{rgb}{1.000000,0.627451,0.478431}%
\pgfsetstrokecolor{currentstroke}%
\pgfsetdash{}{0pt}%
\pgfpathmoveto{\pgfqpoint{1.281449in}{0.613793in}}%
\pgfpathcurveto{\pgfqpoint{1.292499in}{0.613793in}}{\pgfqpoint{1.303098in}{0.618183in}}{\pgfqpoint{1.310912in}{0.625997in}}%
\pgfpathcurveto{\pgfqpoint{1.318725in}{0.633811in}}{\pgfqpoint{1.323116in}{0.644410in}}{\pgfqpoint{1.323116in}{0.655460in}}%
\pgfpathcurveto{\pgfqpoint{1.323116in}{0.666510in}}{\pgfqpoint{1.318725in}{0.677109in}}{\pgfqpoint{1.310912in}{0.684923in}}%
\pgfpathcurveto{\pgfqpoint{1.303098in}{0.692736in}}{\pgfqpoint{1.292499in}{0.697127in}}{\pgfqpoint{1.281449in}{0.697127in}}%
\pgfpathcurveto{\pgfqpoint{1.270399in}{0.697127in}}{\pgfqpoint{1.259800in}{0.692736in}}{\pgfqpoint{1.251986in}{0.684923in}}%
\pgfpathcurveto{\pgfqpoint{1.244173in}{0.677109in}}{\pgfqpoint{1.239782in}{0.666510in}}{\pgfqpoint{1.239782in}{0.655460in}}%
\pgfpathcurveto{\pgfqpoint{1.239782in}{0.644410in}}{\pgfqpoint{1.244173in}{0.633811in}}{\pgfqpoint{1.251986in}{0.625997in}}%
\pgfpathcurveto{\pgfqpoint{1.259800in}{0.618183in}}{\pgfqpoint{1.270399in}{0.613793in}}{\pgfqpoint{1.281449in}{0.613793in}}%
\pgfpathclose%
\pgfusepath{stroke,fill}%
\end{pgfscope}%
\begin{pgfscope}%
\pgfpathrectangle{\pgfqpoint{0.772069in}{0.515123in}}{\pgfqpoint{1.937500in}{1.347500in}}%
\pgfusepath{clip}%
\pgfsetbuttcap%
\pgfsetroundjoin%
\definecolor{currentfill}{rgb}{1.000000,0.627451,0.478431}%
\pgfsetfillcolor{currentfill}%
\pgfsetlinewidth{1.003750pt}%
\definecolor{currentstroke}{rgb}{1.000000,0.627451,0.478431}%
\pgfsetstrokecolor{currentstroke}%
\pgfsetdash{}{0pt}%
\pgfpathmoveto{\pgfqpoint{1.396541in}{0.629271in}}%
\pgfpathcurveto{\pgfqpoint{1.407591in}{0.629271in}}{\pgfqpoint{1.418190in}{0.633662in}}{\pgfqpoint{1.426004in}{0.641475in}}%
\pgfpathcurveto{\pgfqpoint{1.433817in}{0.649289in}}{\pgfqpoint{1.438208in}{0.659888in}}{\pgfqpoint{1.438208in}{0.670938in}}%
\pgfpathcurveto{\pgfqpoint{1.438208in}{0.681988in}}{\pgfqpoint{1.433817in}{0.692587in}}{\pgfqpoint{1.426004in}{0.700401in}}%
\pgfpathcurveto{\pgfqpoint{1.418190in}{0.708214in}}{\pgfqpoint{1.407591in}{0.712605in}}{\pgfqpoint{1.396541in}{0.712605in}}%
\pgfpathcurveto{\pgfqpoint{1.385491in}{0.712605in}}{\pgfqpoint{1.374892in}{0.708214in}}{\pgfqpoint{1.367078in}{0.700401in}}%
\pgfpathcurveto{\pgfqpoint{1.359264in}{0.692587in}}{\pgfqpoint{1.354874in}{0.681988in}}{\pgfqpoint{1.354874in}{0.670938in}}%
\pgfpathcurveto{\pgfqpoint{1.354874in}{0.659888in}}{\pgfqpoint{1.359264in}{0.649289in}}{\pgfqpoint{1.367078in}{0.641475in}}%
\pgfpathcurveto{\pgfqpoint{1.374892in}{0.633662in}}{\pgfqpoint{1.385491in}{0.629271in}}{\pgfqpoint{1.396541in}{0.629271in}}%
\pgfpathclose%
\pgfusepath{stroke,fill}%
\end{pgfscope}%
\begin{pgfscope}%
\pgfpathrectangle{\pgfqpoint{0.772069in}{0.515123in}}{\pgfqpoint{1.937500in}{1.347500in}}%
\pgfusepath{clip}%
\pgfsetbuttcap%
\pgfsetroundjoin%
\definecolor{currentfill}{rgb}{1.000000,0.627451,0.478431}%
\pgfsetfillcolor{currentfill}%
\pgfsetlinewidth{1.003750pt}%
\definecolor{currentstroke}{rgb}{1.000000,0.627451,0.478431}%
\pgfsetstrokecolor{currentstroke}%
\pgfsetdash{}{0pt}%
\pgfpathmoveto{\pgfqpoint{1.511633in}{0.644051in}}%
\pgfpathcurveto{\pgfqpoint{1.522683in}{0.644051in}}{\pgfqpoint{1.533282in}{0.648441in}}{\pgfqpoint{1.541095in}{0.656255in}}%
\pgfpathcurveto{\pgfqpoint{1.548909in}{0.664069in}}{\pgfqpoint{1.553299in}{0.674668in}}{\pgfqpoint{1.553299in}{0.685718in}}%
\pgfpathcurveto{\pgfqpoint{1.553299in}{0.696768in}}{\pgfqpoint{1.548909in}{0.707367in}}{\pgfqpoint{1.541095in}{0.715180in}}%
\pgfpathcurveto{\pgfqpoint{1.533282in}{0.722994in}}{\pgfqpoint{1.522683in}{0.727384in}}{\pgfqpoint{1.511633in}{0.727384in}}%
\pgfpathcurveto{\pgfqpoint{1.500583in}{0.727384in}}{\pgfqpoint{1.489983in}{0.722994in}}{\pgfqpoint{1.482170in}{0.715180in}}%
\pgfpathcurveto{\pgfqpoint{1.474356in}{0.707367in}}{\pgfqpoint{1.469966in}{0.696768in}}{\pgfqpoint{1.469966in}{0.685718in}}%
\pgfpathcurveto{\pgfqpoint{1.469966in}{0.674668in}}{\pgfqpoint{1.474356in}{0.664069in}}{\pgfqpoint{1.482170in}{0.656255in}}%
\pgfpathcurveto{\pgfqpoint{1.489983in}{0.648441in}}{\pgfqpoint{1.500583in}{0.644051in}}{\pgfqpoint{1.511633in}{0.644051in}}%
\pgfpathclose%
\pgfusepath{stroke,fill}%
\end{pgfscope}%
\begin{pgfscope}%
\pgfpathrectangle{\pgfqpoint{0.772069in}{0.515123in}}{\pgfqpoint{1.937500in}{1.347500in}}%
\pgfusepath{clip}%
\pgfsetbuttcap%
\pgfsetroundjoin%
\definecolor{currentfill}{rgb}{1.000000,0.627451,0.478431}%
\pgfsetfillcolor{currentfill}%
\pgfsetlinewidth{1.003750pt}%
\definecolor{currentstroke}{rgb}{1.000000,0.627451,0.478431}%
\pgfsetstrokecolor{currentstroke}%
\pgfsetdash{}{0pt}%
\pgfpathmoveto{\pgfqpoint{1.626724in}{0.660111in}}%
\pgfpathcurveto{\pgfqpoint{1.637775in}{0.660111in}}{\pgfqpoint{1.648374in}{0.664501in}}{\pgfqpoint{1.656187in}{0.672315in}}%
\pgfpathcurveto{\pgfqpoint{1.664001in}{0.680128in}}{\pgfqpoint{1.668391in}{0.690727in}}{\pgfqpoint{1.668391in}{0.701778in}}%
\pgfpathcurveto{\pgfqpoint{1.668391in}{0.712828in}}{\pgfqpoint{1.664001in}{0.723427in}}{\pgfqpoint{1.656187in}{0.731240in}}%
\pgfpathcurveto{\pgfqpoint{1.648374in}{0.739054in}}{\pgfqpoint{1.637775in}{0.743444in}}{\pgfqpoint{1.626724in}{0.743444in}}%
\pgfpathcurveto{\pgfqpoint{1.615674in}{0.743444in}}{\pgfqpoint{1.605075in}{0.739054in}}{\pgfqpoint{1.597262in}{0.731240in}}%
\pgfpathcurveto{\pgfqpoint{1.589448in}{0.723427in}}{\pgfqpoint{1.585058in}{0.712828in}}{\pgfqpoint{1.585058in}{0.701778in}}%
\pgfpathcurveto{\pgfqpoint{1.585058in}{0.690727in}}{\pgfqpoint{1.589448in}{0.680128in}}{\pgfqpoint{1.597262in}{0.672315in}}%
\pgfpathcurveto{\pgfqpoint{1.605075in}{0.664501in}}{\pgfqpoint{1.615674in}{0.660111in}}{\pgfqpoint{1.626724in}{0.660111in}}%
\pgfpathclose%
\pgfusepath{stroke,fill}%
\end{pgfscope}%
\begin{pgfscope}%
\pgfpathrectangle{\pgfqpoint{0.772069in}{0.515123in}}{\pgfqpoint{1.937500in}{1.347500in}}%
\pgfusepath{clip}%
\pgfsetbuttcap%
\pgfsetroundjoin%
\definecolor{currentfill}{rgb}{1.000000,0.627451,0.478431}%
\pgfsetfillcolor{currentfill}%
\pgfsetlinewidth{1.003750pt}%
\definecolor{currentstroke}{rgb}{1.000000,0.627451,0.478431}%
\pgfsetstrokecolor{currentstroke}%
\pgfsetdash{}{0pt}%
\pgfpathmoveto{\pgfqpoint{1.741816in}{0.675240in}}%
\pgfpathcurveto{\pgfqpoint{1.752866in}{0.675240in}}{\pgfqpoint{1.763465in}{0.679630in}}{\pgfqpoint{1.771279in}{0.687444in}}%
\pgfpathcurveto{\pgfqpoint{1.779093in}{0.695257in}}{\pgfqpoint{1.783483in}{0.705856in}}{\pgfqpoint{1.783483in}{0.716907in}}%
\pgfpathcurveto{\pgfqpoint{1.783483in}{0.727957in}}{\pgfqpoint{1.779093in}{0.738556in}}{\pgfqpoint{1.771279in}{0.746369in}}%
\pgfpathcurveto{\pgfqpoint{1.763465in}{0.754183in}}{\pgfqpoint{1.752866in}{0.758573in}}{\pgfqpoint{1.741816in}{0.758573in}}%
\pgfpathcurveto{\pgfqpoint{1.730766in}{0.758573in}}{\pgfqpoint{1.720167in}{0.754183in}}{\pgfqpoint{1.712353in}{0.746369in}}%
\pgfpathcurveto{\pgfqpoint{1.704540in}{0.738556in}}{\pgfqpoint{1.700150in}{0.727957in}}{\pgfqpoint{1.700150in}{0.716907in}}%
\pgfpathcurveto{\pgfqpoint{1.700150in}{0.705856in}}{\pgfqpoint{1.704540in}{0.695257in}}{\pgfqpoint{1.712353in}{0.687444in}}%
\pgfpathcurveto{\pgfqpoint{1.720167in}{0.679630in}}{\pgfqpoint{1.730766in}{0.675240in}}{\pgfqpoint{1.741816in}{0.675240in}}%
\pgfpathclose%
\pgfusepath{stroke,fill}%
\end{pgfscope}%
\begin{pgfscope}%
\pgfpathrectangle{\pgfqpoint{0.772069in}{0.515123in}}{\pgfqpoint{1.937500in}{1.347500in}}%
\pgfusepath{clip}%
\pgfsetbuttcap%
\pgfsetroundjoin%
\definecolor{currentfill}{rgb}{1.000000,0.627451,0.478431}%
\pgfsetfillcolor{currentfill}%
\pgfsetlinewidth{1.003750pt}%
\definecolor{currentstroke}{rgb}{1.000000,0.627451,0.478431}%
\pgfsetstrokecolor{currentstroke}%
\pgfsetdash{}{0pt}%
\pgfpathmoveto{\pgfqpoint{1.856908in}{0.690369in}}%
\pgfpathcurveto{\pgfqpoint{1.867958in}{0.690369in}}{\pgfqpoint{1.878557in}{0.694759in}}{\pgfqpoint{1.886371in}{0.702573in}}%
\pgfpathcurveto{\pgfqpoint{1.894184in}{0.710386in}}{\pgfqpoint{1.898575in}{0.720985in}}{\pgfqpoint{1.898575in}{0.732035in}}%
\pgfpathcurveto{\pgfqpoint{1.898575in}{0.743086in}}{\pgfqpoint{1.894184in}{0.753685in}}{\pgfqpoint{1.886371in}{0.761498in}}%
\pgfpathcurveto{\pgfqpoint{1.878557in}{0.769312in}}{\pgfqpoint{1.867958in}{0.773702in}}{\pgfqpoint{1.856908in}{0.773702in}}%
\pgfpathcurveto{\pgfqpoint{1.845858in}{0.773702in}}{\pgfqpoint{1.835259in}{0.769312in}}{\pgfqpoint{1.827445in}{0.761498in}}%
\pgfpathcurveto{\pgfqpoint{1.819632in}{0.753685in}}{\pgfqpoint{1.815241in}{0.743086in}}{\pgfqpoint{1.815241in}{0.732035in}}%
\pgfpathcurveto{\pgfqpoint{1.815241in}{0.720985in}}{\pgfqpoint{1.819632in}{0.710386in}}{\pgfqpoint{1.827445in}{0.702573in}}%
\pgfpathcurveto{\pgfqpoint{1.835259in}{0.694759in}}{\pgfqpoint{1.845858in}{0.690369in}}{\pgfqpoint{1.856908in}{0.690369in}}%
\pgfpathclose%
\pgfusepath{stroke,fill}%
\end{pgfscope}%
\begin{pgfscope}%
\pgfpathrectangle{\pgfqpoint{0.772069in}{0.515123in}}{\pgfqpoint{1.937500in}{1.347500in}}%
\pgfusepath{clip}%
\pgfsetbuttcap%
\pgfsetroundjoin%
\definecolor{currentfill}{rgb}{1.000000,0.627451,0.478431}%
\pgfsetfillcolor{currentfill}%
\pgfsetlinewidth{1.003750pt}%
\definecolor{currentstroke}{rgb}{1.000000,0.627451,0.478431}%
\pgfsetstrokecolor{currentstroke}%
\pgfsetdash{}{0pt}%
\pgfpathmoveto{\pgfqpoint{1.972000in}{0.705963in}}%
\pgfpathcurveto{\pgfqpoint{1.983050in}{0.705963in}}{\pgfqpoint{1.993649in}{0.710353in}}{\pgfqpoint{2.001463in}{0.718167in}}%
\pgfpathcurveto{\pgfqpoint{2.009276in}{0.725981in}}{\pgfqpoint{2.013667in}{0.736580in}}{\pgfqpoint{2.013667in}{0.747630in}}%
\pgfpathcurveto{\pgfqpoint{2.013667in}{0.758680in}}{\pgfqpoint{2.009276in}{0.769279in}}{\pgfqpoint{2.001463in}{0.777093in}}%
\pgfpathcurveto{\pgfqpoint{1.993649in}{0.784906in}}{\pgfqpoint{1.983050in}{0.789296in}}{\pgfqpoint{1.972000in}{0.789296in}}%
\pgfpathcurveto{\pgfqpoint{1.960950in}{0.789296in}}{\pgfqpoint{1.950351in}{0.784906in}}{\pgfqpoint{1.942537in}{0.777093in}}%
\pgfpathcurveto{\pgfqpoint{1.934723in}{0.769279in}}{\pgfqpoint{1.930333in}{0.758680in}}{\pgfqpoint{1.930333in}{0.747630in}}%
\pgfpathcurveto{\pgfqpoint{1.930333in}{0.736580in}}{\pgfqpoint{1.934723in}{0.725981in}}{\pgfqpoint{1.942537in}{0.718167in}}%
\pgfpathcurveto{\pgfqpoint{1.950351in}{0.710353in}}{\pgfqpoint{1.960950in}{0.705963in}}{\pgfqpoint{1.972000in}{0.705963in}}%
\pgfpathclose%
\pgfusepath{stroke,fill}%
\end{pgfscope}%
\begin{pgfscope}%
\pgfpathrectangle{\pgfqpoint{0.772069in}{0.515123in}}{\pgfqpoint{1.937500in}{1.347500in}}%
\pgfusepath{clip}%
\pgfsetbuttcap%
\pgfsetroundjoin%
\definecolor{currentfill}{rgb}{0.941176,0.901961,0.549020}%
\pgfsetfillcolor{currentfill}%
\pgfsetlinewidth{1.003750pt}%
\definecolor{currentstroke}{rgb}{0.941176,0.901961,0.549020}%
\pgfsetstrokecolor{currentstroke}%
\pgfsetdash{}{0pt}%
\pgfpathmoveto{\pgfqpoint{0.936174in}{0.564706in}}%
\pgfpathcurveto{\pgfqpoint{0.947224in}{0.564706in}}{\pgfqpoint{0.957823in}{0.569096in}}{\pgfqpoint{0.965636in}{0.576910in}}%
\pgfpathcurveto{\pgfqpoint{0.973450in}{0.584723in}}{\pgfqpoint{0.977840in}{0.595322in}}{\pgfqpoint{0.977840in}{0.606372in}}%
\pgfpathcurveto{\pgfqpoint{0.977840in}{0.617423in}}{\pgfqpoint{0.973450in}{0.628022in}}{\pgfqpoint{0.965636in}{0.635835in}}%
\pgfpathcurveto{\pgfqpoint{0.957823in}{0.643649in}}{\pgfqpoint{0.947224in}{0.648039in}}{\pgfqpoint{0.936174in}{0.648039in}}%
\pgfpathcurveto{\pgfqpoint{0.925123in}{0.648039in}}{\pgfqpoint{0.914524in}{0.643649in}}{\pgfqpoint{0.906711in}{0.635835in}}%
\pgfpathcurveto{\pgfqpoint{0.898897in}{0.628022in}}{\pgfqpoint{0.894507in}{0.617423in}}{\pgfqpoint{0.894507in}{0.606372in}}%
\pgfpathcurveto{\pgfqpoint{0.894507in}{0.595322in}}{\pgfqpoint{0.898897in}{0.584723in}}{\pgfqpoint{0.906711in}{0.576910in}}%
\pgfpathcurveto{\pgfqpoint{0.914524in}{0.569096in}}{\pgfqpoint{0.925123in}{0.564706in}}{\pgfqpoint{0.936174in}{0.564706in}}%
\pgfpathclose%
\pgfusepath{stroke,fill}%
\end{pgfscope}%
\begin{pgfscope}%
\pgfpathrectangle{\pgfqpoint{0.772069in}{0.515123in}}{\pgfqpoint{1.937500in}{1.347500in}}%
\pgfusepath{clip}%
\pgfsetbuttcap%
\pgfsetroundjoin%
\definecolor{currentfill}{rgb}{0.941176,0.901961,0.549020}%
\pgfsetfillcolor{currentfill}%
\pgfsetlinewidth{1.003750pt}%
\definecolor{currentstroke}{rgb}{0.941176,0.901961,0.549020}%
\pgfsetstrokecolor{currentstroke}%
\pgfsetdash{}{0pt}%
\pgfpathmoveto{\pgfqpoint{1.051265in}{0.576320in}}%
\pgfpathcurveto{\pgfqpoint{1.062316in}{0.576320in}}{\pgfqpoint{1.072915in}{0.580710in}}{\pgfqpoint{1.080728in}{0.588524in}}%
\pgfpathcurveto{\pgfqpoint{1.088542in}{0.596338in}}{\pgfqpoint{1.092932in}{0.606937in}}{\pgfqpoint{1.092932in}{0.617987in}}%
\pgfpathcurveto{\pgfqpoint{1.092932in}{0.629037in}}{\pgfqpoint{1.088542in}{0.639636in}}{\pgfqpoint{1.080728in}{0.647450in}}%
\pgfpathcurveto{\pgfqpoint{1.072915in}{0.655263in}}{\pgfqpoint{1.062316in}{0.659653in}}{\pgfqpoint{1.051265in}{0.659653in}}%
\pgfpathcurveto{\pgfqpoint{1.040215in}{0.659653in}}{\pgfqpoint{1.029616in}{0.655263in}}{\pgfqpoint{1.021803in}{0.647450in}}%
\pgfpathcurveto{\pgfqpoint{1.013989in}{0.639636in}}{\pgfqpoint{1.009599in}{0.629037in}}{\pgfqpoint{1.009599in}{0.617987in}}%
\pgfpathcurveto{\pgfqpoint{1.009599in}{0.606937in}}{\pgfqpoint{1.013989in}{0.596338in}}{\pgfqpoint{1.021803in}{0.588524in}}%
\pgfpathcurveto{\pgfqpoint{1.029616in}{0.580710in}}{\pgfqpoint{1.040215in}{0.576320in}}{\pgfqpoint{1.051265in}{0.576320in}}%
\pgfpathclose%
\pgfusepath{stroke,fill}%
\end{pgfscope}%
\begin{pgfscope}%
\pgfpathrectangle{\pgfqpoint{0.772069in}{0.515123in}}{\pgfqpoint{1.937500in}{1.347500in}}%
\pgfusepath{clip}%
\pgfsetbuttcap%
\pgfsetroundjoin%
\definecolor{currentfill}{rgb}{0.941176,0.901961,0.549020}%
\pgfsetfillcolor{currentfill}%
\pgfsetlinewidth{1.003750pt}%
\definecolor{currentstroke}{rgb}{0.941176,0.901961,0.549020}%
\pgfsetstrokecolor{currentstroke}%
\pgfsetdash{}{0pt}%
\pgfpathmoveto{\pgfqpoint{1.166357in}{0.588889in}}%
\pgfpathcurveto{\pgfqpoint{1.177407in}{0.588889in}}{\pgfqpoint{1.188006in}{0.593279in}}{\pgfqpoint{1.195820in}{0.601093in}}%
\pgfpathcurveto{\pgfqpoint{1.203634in}{0.608906in}}{\pgfqpoint{1.208024in}{0.619505in}}{\pgfqpoint{1.208024in}{0.630555in}}%
\pgfpathcurveto{\pgfqpoint{1.208024in}{0.641606in}}{\pgfqpoint{1.203634in}{0.652205in}}{\pgfqpoint{1.195820in}{0.660018in}}%
\pgfpathcurveto{\pgfqpoint{1.188006in}{0.667832in}}{\pgfqpoint{1.177407in}{0.672222in}}{\pgfqpoint{1.166357in}{0.672222in}}%
\pgfpathcurveto{\pgfqpoint{1.155307in}{0.672222in}}{\pgfqpoint{1.144708in}{0.667832in}}{\pgfqpoint{1.136894in}{0.660018in}}%
\pgfpathcurveto{\pgfqpoint{1.129081in}{0.652205in}}{\pgfqpoint{1.124691in}{0.641606in}}{\pgfqpoint{1.124691in}{0.630555in}}%
\pgfpathcurveto{\pgfqpoint{1.124691in}{0.619505in}}{\pgfqpoint{1.129081in}{0.608906in}}{\pgfqpoint{1.136894in}{0.601093in}}%
\pgfpathcurveto{\pgfqpoint{1.144708in}{0.593279in}}{\pgfqpoint{1.155307in}{0.588889in}}{\pgfqpoint{1.166357in}{0.588889in}}%
\pgfpathclose%
\pgfusepath{stroke,fill}%
\end{pgfscope}%
\begin{pgfscope}%
\pgfpathrectangle{\pgfqpoint{0.772069in}{0.515123in}}{\pgfqpoint{1.937500in}{1.347500in}}%
\pgfusepath{clip}%
\pgfsetbuttcap%
\pgfsetroundjoin%
\definecolor{currentfill}{rgb}{0.941176,0.901961,0.549020}%
\pgfsetfillcolor{currentfill}%
\pgfsetlinewidth{1.003750pt}%
\definecolor{currentstroke}{rgb}{0.941176,0.901961,0.549020}%
\pgfsetstrokecolor{currentstroke}%
\pgfsetdash{}{0pt}%
\pgfpathmoveto{\pgfqpoint{1.281449in}{0.601225in}}%
\pgfpathcurveto{\pgfqpoint{1.292499in}{0.601225in}}{\pgfqpoint{1.303098in}{0.605615in}}{\pgfqpoint{1.310912in}{0.613428in}}%
\pgfpathcurveto{\pgfqpoint{1.318725in}{0.621242in}}{\pgfqpoint{1.323116in}{0.631841in}}{\pgfqpoint{1.323116in}{0.642891in}}%
\pgfpathcurveto{\pgfqpoint{1.323116in}{0.653941in}}{\pgfqpoint{1.318725in}{0.664540in}}{\pgfqpoint{1.310912in}{0.672354in}}%
\pgfpathcurveto{\pgfqpoint{1.303098in}{0.680168in}}{\pgfqpoint{1.292499in}{0.684558in}}{\pgfqpoint{1.281449in}{0.684558in}}%
\pgfpathcurveto{\pgfqpoint{1.270399in}{0.684558in}}{\pgfqpoint{1.259800in}{0.680168in}}{\pgfqpoint{1.251986in}{0.672354in}}%
\pgfpathcurveto{\pgfqpoint{1.244173in}{0.664540in}}{\pgfqpoint{1.239782in}{0.653941in}}{\pgfqpoint{1.239782in}{0.642891in}}%
\pgfpathcurveto{\pgfqpoint{1.239782in}{0.631841in}}{\pgfqpoint{1.244173in}{0.621242in}}{\pgfqpoint{1.251986in}{0.613428in}}%
\pgfpathcurveto{\pgfqpoint{1.259800in}{0.605615in}}{\pgfqpoint{1.270399in}{0.601225in}}{\pgfqpoint{1.281449in}{0.601225in}}%
\pgfpathclose%
\pgfusepath{stroke,fill}%
\end{pgfscope}%
\begin{pgfscope}%
\pgfpathrectangle{\pgfqpoint{0.772069in}{0.515123in}}{\pgfqpoint{1.937500in}{1.347500in}}%
\pgfusepath{clip}%
\pgfsetbuttcap%
\pgfsetroundjoin%
\definecolor{currentfill}{rgb}{0.941176,0.901961,0.549020}%
\pgfsetfillcolor{currentfill}%
\pgfsetlinewidth{1.003750pt}%
\definecolor{currentstroke}{rgb}{0.941176,0.901961,0.549020}%
\pgfsetstrokecolor{currentstroke}%
\pgfsetdash{}{0pt}%
\pgfpathmoveto{\pgfqpoint{1.396541in}{0.613560in}}%
\pgfpathcurveto{\pgfqpoint{1.407591in}{0.613560in}}{\pgfqpoint{1.418190in}{0.617951in}}{\pgfqpoint{1.426004in}{0.625764in}}%
\pgfpathcurveto{\pgfqpoint{1.433817in}{0.633578in}}{\pgfqpoint{1.438208in}{0.644177in}}{\pgfqpoint{1.438208in}{0.655227in}}%
\pgfpathcurveto{\pgfqpoint{1.438208in}{0.666277in}}{\pgfqpoint{1.433817in}{0.676876in}}{\pgfqpoint{1.426004in}{0.684690in}}%
\pgfpathcurveto{\pgfqpoint{1.418190in}{0.692504in}}{\pgfqpoint{1.407591in}{0.696894in}}{\pgfqpoint{1.396541in}{0.696894in}}%
\pgfpathcurveto{\pgfqpoint{1.385491in}{0.696894in}}{\pgfqpoint{1.374892in}{0.692504in}}{\pgfqpoint{1.367078in}{0.684690in}}%
\pgfpathcurveto{\pgfqpoint{1.359264in}{0.676876in}}{\pgfqpoint{1.354874in}{0.666277in}}{\pgfqpoint{1.354874in}{0.655227in}}%
\pgfpathcurveto{\pgfqpoint{1.354874in}{0.644177in}}{\pgfqpoint{1.359264in}{0.633578in}}{\pgfqpoint{1.367078in}{0.625764in}}%
\pgfpathcurveto{\pgfqpoint{1.374892in}{0.617951in}}{\pgfqpoint{1.385491in}{0.613560in}}{\pgfqpoint{1.396541in}{0.613560in}}%
\pgfpathclose%
\pgfusepath{stroke,fill}%
\end{pgfscope}%
\begin{pgfscope}%
\pgfpathrectangle{\pgfqpoint{0.772069in}{0.515123in}}{\pgfqpoint{1.937500in}{1.347500in}}%
\pgfusepath{clip}%
\pgfsetbuttcap%
\pgfsetroundjoin%
\definecolor{currentfill}{rgb}{0.941176,0.901961,0.549020}%
\pgfsetfillcolor{currentfill}%
\pgfsetlinewidth{1.003750pt}%
\definecolor{currentstroke}{rgb}{0.941176,0.901961,0.549020}%
\pgfsetstrokecolor{currentstroke}%
\pgfsetdash{}{0pt}%
\pgfpathmoveto{\pgfqpoint{1.511633in}{0.625314in}}%
\pgfpathcurveto{\pgfqpoint{1.522683in}{0.625314in}}{\pgfqpoint{1.533282in}{0.629705in}}{\pgfqpoint{1.541095in}{0.637518in}}%
\pgfpathcurveto{\pgfqpoint{1.548909in}{0.645332in}}{\pgfqpoint{1.553299in}{0.655931in}}{\pgfqpoint{1.553299in}{0.666981in}}%
\pgfpathcurveto{\pgfqpoint{1.553299in}{0.678031in}}{\pgfqpoint{1.548909in}{0.688630in}}{\pgfqpoint{1.541095in}{0.696444in}}%
\pgfpathcurveto{\pgfqpoint{1.533282in}{0.704258in}}{\pgfqpoint{1.522683in}{0.708648in}}{\pgfqpoint{1.511633in}{0.708648in}}%
\pgfpathcurveto{\pgfqpoint{1.500583in}{0.708648in}}{\pgfqpoint{1.489983in}{0.704258in}}{\pgfqpoint{1.482170in}{0.696444in}}%
\pgfpathcurveto{\pgfqpoint{1.474356in}{0.688630in}}{\pgfqpoint{1.469966in}{0.678031in}}{\pgfqpoint{1.469966in}{0.666981in}}%
\pgfpathcurveto{\pgfqpoint{1.469966in}{0.655931in}}{\pgfqpoint{1.474356in}{0.645332in}}{\pgfqpoint{1.482170in}{0.637518in}}%
\pgfpathcurveto{\pgfqpoint{1.489983in}{0.629705in}}{\pgfqpoint{1.500583in}{0.625314in}}{\pgfqpoint{1.511633in}{0.625314in}}%
\pgfpathclose%
\pgfusepath{stroke,fill}%
\end{pgfscope}%
\begin{pgfscope}%
\pgfpathrectangle{\pgfqpoint{0.772069in}{0.515123in}}{\pgfqpoint{1.937500in}{1.347500in}}%
\pgfusepath{clip}%
\pgfsetbuttcap%
\pgfsetroundjoin%
\definecolor{currentfill}{rgb}{0.941176,0.901961,0.549020}%
\pgfsetfillcolor{currentfill}%
\pgfsetlinewidth{1.003750pt}%
\definecolor{currentstroke}{rgb}{0.941176,0.901961,0.549020}%
\pgfsetstrokecolor{currentstroke}%
\pgfsetdash{}{0pt}%
\pgfpathmoveto{\pgfqpoint{1.626724in}{0.644284in}}%
\pgfpathcurveto{\pgfqpoint{1.637775in}{0.644284in}}{\pgfqpoint{1.648374in}{0.648674in}}{\pgfqpoint{1.656187in}{0.656488in}}%
\pgfpathcurveto{\pgfqpoint{1.664001in}{0.664301in}}{\pgfqpoint{1.668391in}{0.674900in}}{\pgfqpoint{1.668391in}{0.685950in}}%
\pgfpathcurveto{\pgfqpoint{1.668391in}{0.697001in}}{\pgfqpoint{1.664001in}{0.707600in}}{\pgfqpoint{1.656187in}{0.715413in}}%
\pgfpathcurveto{\pgfqpoint{1.648374in}{0.723227in}}{\pgfqpoint{1.637775in}{0.727617in}}{\pgfqpoint{1.626724in}{0.727617in}}%
\pgfpathcurveto{\pgfqpoint{1.615674in}{0.727617in}}{\pgfqpoint{1.605075in}{0.723227in}}{\pgfqpoint{1.597262in}{0.715413in}}%
\pgfpathcurveto{\pgfqpoint{1.589448in}{0.707600in}}{\pgfqpoint{1.585058in}{0.697001in}}{\pgfqpoint{1.585058in}{0.685950in}}%
\pgfpathcurveto{\pgfqpoint{1.585058in}{0.674900in}}{\pgfqpoint{1.589448in}{0.664301in}}{\pgfqpoint{1.597262in}{0.656488in}}%
\pgfpathcurveto{\pgfqpoint{1.605075in}{0.648674in}}{\pgfqpoint{1.615674in}{0.644284in}}{\pgfqpoint{1.626724in}{0.644284in}}%
\pgfpathclose%
\pgfusepath{stroke,fill}%
\end{pgfscope}%
\begin{pgfscope}%
\pgfpathrectangle{\pgfqpoint{0.772069in}{0.515123in}}{\pgfqpoint{1.937500in}{1.347500in}}%
\pgfusepath{clip}%
\pgfsetbuttcap%
\pgfsetroundjoin%
\definecolor{currentfill}{rgb}{0.941176,0.901961,0.549020}%
\pgfsetfillcolor{currentfill}%
\pgfsetlinewidth{1.003750pt}%
\definecolor{currentstroke}{rgb}{0.941176,0.901961,0.549020}%
\pgfsetstrokecolor{currentstroke}%
\pgfsetdash{}{0pt}%
\pgfpathmoveto{\pgfqpoint{1.741816in}{0.650335in}}%
\pgfpathcurveto{\pgfqpoint{1.752866in}{0.650335in}}{\pgfqpoint{1.763465in}{0.654726in}}{\pgfqpoint{1.771279in}{0.662539in}}%
\pgfpathcurveto{\pgfqpoint{1.779093in}{0.670353in}}{\pgfqpoint{1.783483in}{0.680952in}}{\pgfqpoint{1.783483in}{0.692002in}}%
\pgfpathcurveto{\pgfqpoint{1.783483in}{0.703052in}}{\pgfqpoint{1.779093in}{0.713651in}}{\pgfqpoint{1.771279in}{0.721465in}}%
\pgfpathcurveto{\pgfqpoint{1.763465in}{0.729278in}}{\pgfqpoint{1.752866in}{0.733669in}}{\pgfqpoint{1.741816in}{0.733669in}}%
\pgfpathcurveto{\pgfqpoint{1.730766in}{0.733669in}}{\pgfqpoint{1.720167in}{0.729278in}}{\pgfqpoint{1.712353in}{0.721465in}}%
\pgfpathcurveto{\pgfqpoint{1.704540in}{0.713651in}}{\pgfqpoint{1.700150in}{0.703052in}}{\pgfqpoint{1.700150in}{0.692002in}}%
\pgfpathcurveto{\pgfqpoint{1.700150in}{0.680952in}}{\pgfqpoint{1.704540in}{0.670353in}}{\pgfqpoint{1.712353in}{0.662539in}}%
\pgfpathcurveto{\pgfqpoint{1.720167in}{0.654726in}}{\pgfqpoint{1.730766in}{0.650335in}}{\pgfqpoint{1.741816in}{0.650335in}}%
\pgfpathclose%
\pgfusepath{stroke,fill}%
\end{pgfscope}%
\begin{pgfscope}%
\pgfpathrectangle{\pgfqpoint{0.772069in}{0.515123in}}{\pgfqpoint{1.937500in}{1.347500in}}%
\pgfusepath{clip}%
\pgfsetbuttcap%
\pgfsetroundjoin%
\definecolor{currentfill}{rgb}{0.941176,0.901961,0.549020}%
\pgfsetfillcolor{currentfill}%
\pgfsetlinewidth{1.003750pt}%
\definecolor{currentstroke}{rgb}{0.941176,0.901961,0.549020}%
\pgfsetstrokecolor{currentstroke}%
\pgfsetdash{}{0pt}%
\pgfpathmoveto{\pgfqpoint{1.856908in}{0.662322in}}%
\pgfpathcurveto{\pgfqpoint{1.867958in}{0.662322in}}{\pgfqpoint{1.878557in}{0.666712in}}{\pgfqpoint{1.886371in}{0.674526in}}%
\pgfpathcurveto{\pgfqpoint{1.894184in}{0.682340in}}{\pgfqpoint{1.898575in}{0.692939in}}{\pgfqpoint{1.898575in}{0.703989in}}%
\pgfpathcurveto{\pgfqpoint{1.898575in}{0.715039in}}{\pgfqpoint{1.894184in}{0.725638in}}{\pgfqpoint{1.886371in}{0.733452in}}%
\pgfpathcurveto{\pgfqpoint{1.878557in}{0.741265in}}{\pgfqpoint{1.867958in}{0.745655in}}{\pgfqpoint{1.856908in}{0.745655in}}%
\pgfpathcurveto{\pgfqpoint{1.845858in}{0.745655in}}{\pgfqpoint{1.835259in}{0.741265in}}{\pgfqpoint{1.827445in}{0.733452in}}%
\pgfpathcurveto{\pgfqpoint{1.819632in}{0.725638in}}{\pgfqpoint{1.815241in}{0.715039in}}{\pgfqpoint{1.815241in}{0.703989in}}%
\pgfpathcurveto{\pgfqpoint{1.815241in}{0.692939in}}{\pgfqpoint{1.819632in}{0.682340in}}{\pgfqpoint{1.827445in}{0.674526in}}%
\pgfpathcurveto{\pgfqpoint{1.835259in}{0.666712in}}{\pgfqpoint{1.845858in}{0.662322in}}{\pgfqpoint{1.856908in}{0.662322in}}%
\pgfpathclose%
\pgfusepath{stroke,fill}%
\end{pgfscope}%
\begin{pgfscope}%
\pgfpathrectangle{\pgfqpoint{0.772069in}{0.515123in}}{\pgfqpoint{1.937500in}{1.347500in}}%
\pgfusepath{clip}%
\pgfsetbuttcap%
\pgfsetroundjoin%
\definecolor{currentfill}{rgb}{0.941176,0.901961,0.549020}%
\pgfsetfillcolor{currentfill}%
\pgfsetlinewidth{1.003750pt}%
\definecolor{currentstroke}{rgb}{0.941176,0.901961,0.549020}%
\pgfsetstrokecolor{currentstroke}%
\pgfsetdash{}{0pt}%
\pgfpathmoveto{\pgfqpoint{1.972000in}{0.674658in}}%
\pgfpathcurveto{\pgfqpoint{1.983050in}{0.674658in}}{\pgfqpoint{1.993649in}{0.679048in}}{\pgfqpoint{2.001463in}{0.686862in}}%
\pgfpathcurveto{\pgfqpoint{2.009276in}{0.694675in}}{\pgfqpoint{2.013667in}{0.705274in}}{\pgfqpoint{2.013667in}{0.716325in}}%
\pgfpathcurveto{\pgfqpoint{2.013667in}{0.727375in}}{\pgfqpoint{2.009276in}{0.737974in}}{\pgfqpoint{2.001463in}{0.745787in}}%
\pgfpathcurveto{\pgfqpoint{1.993649in}{0.753601in}}{\pgfqpoint{1.983050in}{0.757991in}}{\pgfqpoint{1.972000in}{0.757991in}}%
\pgfpathcurveto{\pgfqpoint{1.960950in}{0.757991in}}{\pgfqpoint{1.950351in}{0.753601in}}{\pgfqpoint{1.942537in}{0.745787in}}%
\pgfpathcurveto{\pgfqpoint{1.934723in}{0.737974in}}{\pgfqpoint{1.930333in}{0.727375in}}{\pgfqpoint{1.930333in}{0.716325in}}%
\pgfpathcurveto{\pgfqpoint{1.930333in}{0.705274in}}{\pgfqpoint{1.934723in}{0.694675in}}{\pgfqpoint{1.942537in}{0.686862in}}%
\pgfpathcurveto{\pgfqpoint{1.950351in}{0.679048in}}{\pgfqpoint{1.960950in}{0.674658in}}{\pgfqpoint{1.972000in}{0.674658in}}%
\pgfpathclose%
\pgfusepath{stroke,fill}%
\end{pgfscope}%
\begin{pgfscope}%
\pgfpathrectangle{\pgfqpoint{0.772069in}{0.515123in}}{\pgfqpoint{1.937500in}{1.347500in}}%
\pgfusepath{clip}%
\pgfsetbuttcap%
\pgfsetroundjoin%
\definecolor{currentfill}{rgb}{0.564706,0.933333,0.564706}%
\pgfsetfillcolor{currentfill}%
\pgfsetlinewidth{1.003750pt}%
\definecolor{currentstroke}{rgb}{0.564706,0.933333,0.564706}%
\pgfsetstrokecolor{currentstroke}%
\pgfsetdash{}{0pt}%
\pgfpathmoveto{\pgfqpoint{0.936174in}{0.641491in}}%
\pgfpathcurveto{\pgfqpoint{0.947224in}{0.641491in}}{\pgfqpoint{0.957823in}{0.645881in}}{\pgfqpoint{0.965636in}{0.653695in}}%
\pgfpathcurveto{\pgfqpoint{0.973450in}{0.661508in}}{\pgfqpoint{0.977840in}{0.672107in}}{\pgfqpoint{0.977840in}{0.683157in}}%
\pgfpathcurveto{\pgfqpoint{0.977840in}{0.694208in}}{\pgfqpoint{0.973450in}{0.704807in}}{\pgfqpoint{0.965636in}{0.712620in}}%
\pgfpathcurveto{\pgfqpoint{0.957823in}{0.720434in}}{\pgfqpoint{0.947224in}{0.724824in}}{\pgfqpoint{0.936174in}{0.724824in}}%
\pgfpathcurveto{\pgfqpoint{0.925123in}{0.724824in}}{\pgfqpoint{0.914524in}{0.720434in}}{\pgfqpoint{0.906711in}{0.712620in}}%
\pgfpathcurveto{\pgfqpoint{0.898897in}{0.704807in}}{\pgfqpoint{0.894507in}{0.694208in}}{\pgfqpoint{0.894507in}{0.683157in}}%
\pgfpathcurveto{\pgfqpoint{0.894507in}{0.672107in}}{\pgfqpoint{0.898897in}{0.661508in}}{\pgfqpoint{0.906711in}{0.653695in}}%
\pgfpathcurveto{\pgfqpoint{0.914524in}{0.645881in}}{\pgfqpoint{0.925123in}{0.641491in}}{\pgfqpoint{0.936174in}{0.641491in}}%
\pgfpathclose%
\pgfusepath{stroke,fill}%
\end{pgfscope}%
\begin{pgfscope}%
\pgfpathrectangle{\pgfqpoint{0.772069in}{0.515123in}}{\pgfqpoint{1.937500in}{1.347500in}}%
\pgfusepath{clip}%
\pgfsetbuttcap%
\pgfsetroundjoin%
\definecolor{currentfill}{rgb}{0.564706,0.933333,0.564706}%
\pgfsetfillcolor{currentfill}%
\pgfsetlinewidth{1.003750pt}%
\definecolor{currentstroke}{rgb}{0.564706,0.933333,0.564706}%
\pgfsetstrokecolor{currentstroke}%
\pgfsetdash{}{0pt}%
\pgfpathmoveto{\pgfqpoint{1.051265in}{0.725631in}}%
\pgfpathcurveto{\pgfqpoint{1.062316in}{0.725631in}}{\pgfqpoint{1.072915in}{0.730021in}}{\pgfqpoint{1.080728in}{0.737835in}}%
\pgfpathcurveto{\pgfqpoint{1.088542in}{0.745648in}}{\pgfqpoint{1.092932in}{0.756247in}}{\pgfqpoint{1.092932in}{0.767297in}}%
\pgfpathcurveto{\pgfqpoint{1.092932in}{0.778348in}}{\pgfqpoint{1.088542in}{0.788947in}}{\pgfqpoint{1.080728in}{0.796760in}}%
\pgfpathcurveto{\pgfqpoint{1.072915in}{0.804574in}}{\pgfqpoint{1.062316in}{0.808964in}}{\pgfqpoint{1.051265in}{0.808964in}}%
\pgfpathcurveto{\pgfqpoint{1.040215in}{0.808964in}}{\pgfqpoint{1.029616in}{0.804574in}}{\pgfqpoint{1.021803in}{0.796760in}}%
\pgfpathcurveto{\pgfqpoint{1.013989in}{0.788947in}}{\pgfqpoint{1.009599in}{0.778348in}}{\pgfqpoint{1.009599in}{0.767297in}}%
\pgfpathcurveto{\pgfqpoint{1.009599in}{0.756247in}}{\pgfqpoint{1.013989in}{0.745648in}}{\pgfqpoint{1.021803in}{0.737835in}}%
\pgfpathcurveto{\pgfqpoint{1.029616in}{0.730021in}}{\pgfqpoint{1.040215in}{0.725631in}}{\pgfqpoint{1.051265in}{0.725631in}}%
\pgfpathclose%
\pgfusepath{stroke,fill}%
\end{pgfscope}%
\begin{pgfscope}%
\pgfpathrectangle{\pgfqpoint{0.772069in}{0.515123in}}{\pgfqpoint{1.937500in}{1.347500in}}%
\pgfusepath{clip}%
\pgfsetbuttcap%
\pgfsetroundjoin%
\definecolor{currentfill}{rgb}{0.564706,0.933333,0.564706}%
\pgfsetfillcolor{currentfill}%
\pgfsetlinewidth{1.003750pt}%
\definecolor{currentstroke}{rgb}{0.564706,0.933333,0.564706}%
\pgfsetstrokecolor{currentstroke}%
\pgfsetdash{}{0pt}%
\pgfpathmoveto{\pgfqpoint{1.166357in}{0.813844in}}%
\pgfpathcurveto{\pgfqpoint{1.177407in}{0.813844in}}{\pgfqpoint{1.188006in}{0.818234in}}{\pgfqpoint{1.195820in}{0.826048in}}%
\pgfpathcurveto{\pgfqpoint{1.203634in}{0.833861in}}{\pgfqpoint{1.208024in}{0.844460in}}{\pgfqpoint{1.208024in}{0.855510in}}%
\pgfpathcurveto{\pgfqpoint{1.208024in}{0.866561in}}{\pgfqpoint{1.203634in}{0.877160in}}{\pgfqpoint{1.195820in}{0.884973in}}%
\pgfpathcurveto{\pgfqpoint{1.188006in}{0.892787in}}{\pgfqpoint{1.177407in}{0.897177in}}{\pgfqpoint{1.166357in}{0.897177in}}%
\pgfpathcurveto{\pgfqpoint{1.155307in}{0.897177in}}{\pgfqpoint{1.144708in}{0.892787in}}{\pgfqpoint{1.136894in}{0.884973in}}%
\pgfpathcurveto{\pgfqpoint{1.129081in}{0.877160in}}{\pgfqpoint{1.124691in}{0.866561in}}{\pgfqpoint{1.124691in}{0.855510in}}%
\pgfpathcurveto{\pgfqpoint{1.124691in}{0.844460in}}{\pgfqpoint{1.129081in}{0.833861in}}{\pgfqpoint{1.136894in}{0.826048in}}%
\pgfpathcurveto{\pgfqpoint{1.144708in}{0.818234in}}{\pgfqpoint{1.155307in}{0.813844in}}{\pgfqpoint{1.166357in}{0.813844in}}%
\pgfpathclose%
\pgfusepath{stroke,fill}%
\end{pgfscope}%
\begin{pgfscope}%
\pgfpathrectangle{\pgfqpoint{0.772069in}{0.515123in}}{\pgfqpoint{1.937500in}{1.347500in}}%
\pgfusepath{clip}%
\pgfsetbuttcap%
\pgfsetroundjoin%
\definecolor{currentfill}{rgb}{0.564706,0.933333,0.564706}%
\pgfsetfillcolor{currentfill}%
\pgfsetlinewidth{1.003750pt}%
\definecolor{currentstroke}{rgb}{0.564706,0.933333,0.564706}%
\pgfsetstrokecolor{currentstroke}%
\pgfsetdash{}{0pt}%
\pgfpathmoveto{\pgfqpoint{1.281449in}{0.903453in}}%
\pgfpathcurveto{\pgfqpoint{1.292499in}{0.903453in}}{\pgfqpoint{1.303098in}{0.907844in}}{\pgfqpoint{1.310912in}{0.915657in}}%
\pgfpathcurveto{\pgfqpoint{1.318725in}{0.923471in}}{\pgfqpoint{1.323116in}{0.934070in}}{\pgfqpoint{1.323116in}{0.945120in}}%
\pgfpathcurveto{\pgfqpoint{1.323116in}{0.956170in}}{\pgfqpoint{1.318725in}{0.966769in}}{\pgfqpoint{1.310912in}{0.974583in}}%
\pgfpathcurveto{\pgfqpoint{1.303098in}{0.982397in}}{\pgfqpoint{1.292499in}{0.986787in}}{\pgfqpoint{1.281449in}{0.986787in}}%
\pgfpathcurveto{\pgfqpoint{1.270399in}{0.986787in}}{\pgfqpoint{1.259800in}{0.982397in}}{\pgfqpoint{1.251986in}{0.974583in}}%
\pgfpathcurveto{\pgfqpoint{1.244173in}{0.966769in}}{\pgfqpoint{1.239782in}{0.956170in}}{\pgfqpoint{1.239782in}{0.945120in}}%
\pgfpathcurveto{\pgfqpoint{1.239782in}{0.934070in}}{\pgfqpoint{1.244173in}{0.923471in}}{\pgfqpoint{1.251986in}{0.915657in}}%
\pgfpathcurveto{\pgfqpoint{1.259800in}{0.907844in}}{\pgfqpoint{1.270399in}{0.903453in}}{\pgfqpoint{1.281449in}{0.903453in}}%
\pgfpathclose%
\pgfusepath{stroke,fill}%
\end{pgfscope}%
\begin{pgfscope}%
\pgfpathrectangle{\pgfqpoint{0.772069in}{0.515123in}}{\pgfqpoint{1.937500in}{1.347500in}}%
\pgfusepath{clip}%
\pgfsetbuttcap%
\pgfsetroundjoin%
\definecolor{currentfill}{rgb}{0.564706,0.933333,0.564706}%
\pgfsetfillcolor{currentfill}%
\pgfsetlinewidth{1.003750pt}%
\definecolor{currentstroke}{rgb}{0.564706,0.933333,0.564706}%
\pgfsetstrokecolor{currentstroke}%
\pgfsetdash{}{0pt}%
\pgfpathmoveto{\pgfqpoint{1.396541in}{0.991899in}}%
\pgfpathcurveto{\pgfqpoint{1.407591in}{0.991899in}}{\pgfqpoint{1.418190in}{0.996290in}}{\pgfqpoint{1.426004in}{1.004103in}}%
\pgfpathcurveto{\pgfqpoint{1.433817in}{1.011917in}}{\pgfqpoint{1.438208in}{1.022516in}}{\pgfqpoint{1.438208in}{1.033566in}}%
\pgfpathcurveto{\pgfqpoint{1.438208in}{1.044616in}}{\pgfqpoint{1.433817in}{1.055215in}}{\pgfqpoint{1.426004in}{1.063029in}}%
\pgfpathcurveto{\pgfqpoint{1.418190in}{1.070842in}}{\pgfqpoint{1.407591in}{1.075233in}}{\pgfqpoint{1.396541in}{1.075233in}}%
\pgfpathcurveto{\pgfqpoint{1.385491in}{1.075233in}}{\pgfqpoint{1.374892in}{1.070842in}}{\pgfqpoint{1.367078in}{1.063029in}}%
\pgfpathcurveto{\pgfqpoint{1.359264in}{1.055215in}}{\pgfqpoint{1.354874in}{1.044616in}}{\pgfqpoint{1.354874in}{1.033566in}}%
\pgfpathcurveto{\pgfqpoint{1.354874in}{1.022516in}}{\pgfqpoint{1.359264in}{1.011917in}}{\pgfqpoint{1.367078in}{1.004103in}}%
\pgfpathcurveto{\pgfqpoint{1.374892in}{0.996290in}}{\pgfqpoint{1.385491in}{0.991899in}}{\pgfqpoint{1.396541in}{0.991899in}}%
\pgfpathclose%
\pgfusepath{stroke,fill}%
\end{pgfscope}%
\begin{pgfscope}%
\pgfpathrectangle{\pgfqpoint{0.772069in}{0.515123in}}{\pgfqpoint{1.937500in}{1.347500in}}%
\pgfusepath{clip}%
\pgfsetbuttcap%
\pgfsetroundjoin%
\definecolor{currentfill}{rgb}{0.564706,0.933333,0.564706}%
\pgfsetfillcolor{currentfill}%
\pgfsetlinewidth{1.003750pt}%
\definecolor{currentstroke}{rgb}{0.564706,0.933333,0.564706}%
\pgfsetstrokecolor{currentstroke}%
\pgfsetdash{}{0pt}%
\pgfpathmoveto{\pgfqpoint{1.511633in}{1.075690in}}%
\pgfpathcurveto{\pgfqpoint{1.522683in}{1.075690in}}{\pgfqpoint{1.533282in}{1.080080in}}{\pgfqpoint{1.541095in}{1.087894in}}%
\pgfpathcurveto{\pgfqpoint{1.548909in}{1.095708in}}{\pgfqpoint{1.553299in}{1.106307in}}{\pgfqpoint{1.553299in}{1.117357in}}%
\pgfpathcurveto{\pgfqpoint{1.553299in}{1.128407in}}{\pgfqpoint{1.548909in}{1.139006in}}{\pgfqpoint{1.541095in}{1.146820in}}%
\pgfpathcurveto{\pgfqpoint{1.533282in}{1.154633in}}{\pgfqpoint{1.522683in}{1.159023in}}{\pgfqpoint{1.511633in}{1.159023in}}%
\pgfpathcurveto{\pgfqpoint{1.500583in}{1.159023in}}{\pgfqpoint{1.489983in}{1.154633in}}{\pgfqpoint{1.482170in}{1.146820in}}%
\pgfpathcurveto{\pgfqpoint{1.474356in}{1.139006in}}{\pgfqpoint{1.469966in}{1.128407in}}{\pgfqpoint{1.469966in}{1.117357in}}%
\pgfpathcurveto{\pgfqpoint{1.469966in}{1.106307in}}{\pgfqpoint{1.474356in}{1.095708in}}{\pgfqpoint{1.482170in}{1.087894in}}%
\pgfpathcurveto{\pgfqpoint{1.489983in}{1.080080in}}{\pgfqpoint{1.500583in}{1.075690in}}{\pgfqpoint{1.511633in}{1.075690in}}%
\pgfpathclose%
\pgfusepath{stroke,fill}%
\end{pgfscope}%
\begin{pgfscope}%
\pgfpathrectangle{\pgfqpoint{0.772069in}{0.515123in}}{\pgfqpoint{1.937500in}{1.347500in}}%
\pgfusepath{clip}%
\pgfsetbuttcap%
\pgfsetroundjoin%
\definecolor{currentfill}{rgb}{0.564706,0.933333,0.564706}%
\pgfsetfillcolor{currentfill}%
\pgfsetlinewidth{1.003750pt}%
\definecolor{currentstroke}{rgb}{0.564706,0.933333,0.564706}%
\pgfsetstrokecolor{currentstroke}%
\pgfsetdash{}{0pt}%
\pgfpathmoveto{\pgfqpoint{1.626724in}{1.166464in}}%
\pgfpathcurveto{\pgfqpoint{1.637775in}{1.166464in}}{\pgfqpoint{1.648374in}{1.170854in}}{\pgfqpoint{1.656187in}{1.178667in}}%
\pgfpathcurveto{\pgfqpoint{1.664001in}{1.186481in}}{\pgfqpoint{1.668391in}{1.197080in}}{\pgfqpoint{1.668391in}{1.208130in}}%
\pgfpathcurveto{\pgfqpoint{1.668391in}{1.219180in}}{\pgfqpoint{1.664001in}{1.229779in}}{\pgfqpoint{1.656187in}{1.237593in}}%
\pgfpathcurveto{\pgfqpoint{1.648374in}{1.245407in}}{\pgfqpoint{1.637775in}{1.249797in}}{\pgfqpoint{1.626724in}{1.249797in}}%
\pgfpathcurveto{\pgfqpoint{1.615674in}{1.249797in}}{\pgfqpoint{1.605075in}{1.245407in}}{\pgfqpoint{1.597262in}{1.237593in}}%
\pgfpathcurveto{\pgfqpoint{1.589448in}{1.229779in}}{\pgfqpoint{1.585058in}{1.219180in}}{\pgfqpoint{1.585058in}{1.208130in}}%
\pgfpathcurveto{\pgfqpoint{1.585058in}{1.197080in}}{\pgfqpoint{1.589448in}{1.186481in}}{\pgfqpoint{1.597262in}{1.178667in}}%
\pgfpathcurveto{\pgfqpoint{1.605075in}{1.170854in}}{\pgfqpoint{1.615674in}{1.166464in}}{\pgfqpoint{1.626724in}{1.166464in}}%
\pgfpathclose%
\pgfusepath{stroke,fill}%
\end{pgfscope}%
\begin{pgfscope}%
\pgfpathrectangle{\pgfqpoint{0.772069in}{0.515123in}}{\pgfqpoint{1.937500in}{1.347500in}}%
\pgfusepath{clip}%
\pgfsetbuttcap%
\pgfsetroundjoin%
\definecolor{currentfill}{rgb}{0.564706,0.933333,0.564706}%
\pgfsetfillcolor{currentfill}%
\pgfsetlinewidth{1.003750pt}%
\definecolor{currentstroke}{rgb}{0.564706,0.933333,0.564706}%
\pgfsetstrokecolor{currentstroke}%
\pgfsetdash{}{0pt}%
\pgfpathmoveto{\pgfqpoint{1.741816in}{1.253746in}}%
\pgfpathcurveto{\pgfqpoint{1.752866in}{1.253746in}}{\pgfqpoint{1.763465in}{1.258136in}}{\pgfqpoint{1.771279in}{1.265950in}}%
\pgfpathcurveto{\pgfqpoint{1.779093in}{1.273763in}}{\pgfqpoint{1.783483in}{1.284362in}}{\pgfqpoint{1.783483in}{1.295412in}}%
\pgfpathcurveto{\pgfqpoint{1.783483in}{1.306462in}}{\pgfqpoint{1.779093in}{1.317061in}}{\pgfqpoint{1.771279in}{1.324875in}}%
\pgfpathcurveto{\pgfqpoint{1.763465in}{1.332689in}}{\pgfqpoint{1.752866in}{1.337079in}}{\pgfqpoint{1.741816in}{1.337079in}}%
\pgfpathcurveto{\pgfqpoint{1.730766in}{1.337079in}}{\pgfqpoint{1.720167in}{1.332689in}}{\pgfqpoint{1.712353in}{1.324875in}}%
\pgfpathcurveto{\pgfqpoint{1.704540in}{1.317061in}}{\pgfqpoint{1.700150in}{1.306462in}}{\pgfqpoint{1.700150in}{1.295412in}}%
\pgfpathcurveto{\pgfqpoint{1.700150in}{1.284362in}}{\pgfqpoint{1.704540in}{1.273763in}}{\pgfqpoint{1.712353in}{1.265950in}}%
\pgfpathcurveto{\pgfqpoint{1.720167in}{1.258136in}}{\pgfqpoint{1.730766in}{1.253746in}}{\pgfqpoint{1.741816in}{1.253746in}}%
\pgfpathclose%
\pgfusepath{stroke,fill}%
\end{pgfscope}%
\begin{pgfscope}%
\pgfpathrectangle{\pgfqpoint{0.772069in}{0.515123in}}{\pgfqpoint{1.937500in}{1.347500in}}%
\pgfusepath{clip}%
\pgfsetbuttcap%
\pgfsetroundjoin%
\definecolor{currentfill}{rgb}{0.564706,0.933333,0.564706}%
\pgfsetfillcolor{currentfill}%
\pgfsetlinewidth{1.003750pt}%
\definecolor{currentstroke}{rgb}{0.564706,0.933333,0.564706}%
\pgfsetstrokecolor{currentstroke}%
\pgfsetdash{}{0pt}%
\pgfpathmoveto{\pgfqpoint{1.856908in}{1.339864in}}%
\pgfpathcurveto{\pgfqpoint{1.867958in}{1.339864in}}{\pgfqpoint{1.878557in}{1.344254in}}{\pgfqpoint{1.886371in}{1.352068in}}%
\pgfpathcurveto{\pgfqpoint{1.894184in}{1.359881in}}{\pgfqpoint{1.898575in}{1.370481in}}{\pgfqpoint{1.898575in}{1.381531in}}%
\pgfpathcurveto{\pgfqpoint{1.898575in}{1.392581in}}{\pgfqpoint{1.894184in}{1.403180in}}{\pgfqpoint{1.886371in}{1.410993in}}%
\pgfpathcurveto{\pgfqpoint{1.878557in}{1.418807in}}{\pgfqpoint{1.867958in}{1.423197in}}{\pgfqpoint{1.856908in}{1.423197in}}%
\pgfpathcurveto{\pgfqpoint{1.845858in}{1.423197in}}{\pgfqpoint{1.835259in}{1.418807in}}{\pgfqpoint{1.827445in}{1.410993in}}%
\pgfpathcurveto{\pgfqpoint{1.819632in}{1.403180in}}{\pgfqpoint{1.815241in}{1.392581in}}{\pgfqpoint{1.815241in}{1.381531in}}%
\pgfpathcurveto{\pgfqpoint{1.815241in}{1.370481in}}{\pgfqpoint{1.819632in}{1.359881in}}{\pgfqpoint{1.827445in}{1.352068in}}%
\pgfpathcurveto{\pgfqpoint{1.835259in}{1.344254in}}{\pgfqpoint{1.845858in}{1.339864in}}{\pgfqpoint{1.856908in}{1.339864in}}%
\pgfpathclose%
\pgfusepath{stroke,fill}%
\end{pgfscope}%
\begin{pgfscope}%
\pgfpathrectangle{\pgfqpoint{0.772069in}{0.515123in}}{\pgfqpoint{1.937500in}{1.347500in}}%
\pgfusepath{clip}%
\pgfsetbuttcap%
\pgfsetroundjoin%
\definecolor{currentfill}{rgb}{0.564706,0.933333,0.564706}%
\pgfsetfillcolor{currentfill}%
\pgfsetlinewidth{1.003750pt}%
\definecolor{currentstroke}{rgb}{0.564706,0.933333,0.564706}%
\pgfsetstrokecolor{currentstroke}%
\pgfsetdash{}{0pt}%
\pgfpathmoveto{\pgfqpoint{1.972000in}{1.428310in}}%
\pgfpathcurveto{\pgfqpoint{1.983050in}{1.428310in}}{\pgfqpoint{1.993649in}{1.432700in}}{\pgfqpoint{2.001463in}{1.440514in}}%
\pgfpathcurveto{\pgfqpoint{2.009276in}{1.448327in}}{\pgfqpoint{2.013667in}{1.458926in}}{\pgfqpoint{2.013667in}{1.469977in}}%
\pgfpathcurveto{\pgfqpoint{2.013667in}{1.481027in}}{\pgfqpoint{2.009276in}{1.491626in}}{\pgfqpoint{2.001463in}{1.499439in}}%
\pgfpathcurveto{\pgfqpoint{1.993649in}{1.507253in}}{\pgfqpoint{1.983050in}{1.511643in}}{\pgfqpoint{1.972000in}{1.511643in}}%
\pgfpathcurveto{\pgfqpoint{1.960950in}{1.511643in}}{\pgfqpoint{1.950351in}{1.507253in}}{\pgfqpoint{1.942537in}{1.499439in}}%
\pgfpathcurveto{\pgfqpoint{1.934723in}{1.491626in}}{\pgfqpoint{1.930333in}{1.481027in}}{\pgfqpoint{1.930333in}{1.469977in}}%
\pgfpathcurveto{\pgfqpoint{1.930333in}{1.458926in}}{\pgfqpoint{1.934723in}{1.448327in}}{\pgfqpoint{1.942537in}{1.440514in}}%
\pgfpathcurveto{\pgfqpoint{1.950351in}{1.432700in}}{\pgfqpoint{1.960950in}{1.428310in}}{\pgfqpoint{1.972000in}{1.428310in}}%
\pgfpathclose%
\pgfusepath{stroke,fill}%
\end{pgfscope}%
\begin{pgfscope}%
\pgfpathrectangle{\pgfqpoint{0.772069in}{0.515123in}}{\pgfqpoint{1.937500in}{1.347500in}}%
\pgfusepath{clip}%
\pgfsetbuttcap%
\pgfsetroundjoin%
\definecolor{currentfill}{rgb}{0.529412,0.807843,0.921569}%
\pgfsetfillcolor{currentfill}%
\pgfsetlinewidth{1.003750pt}%
\definecolor{currentstroke}{rgb}{0.529412,0.807843,0.921569}%
\pgfsetstrokecolor{currentstroke}%
\pgfsetdash{}{0pt}%
\pgfpathmoveto{\pgfqpoint{0.890137in}{0.672098in}}%
\pgfpathcurveto{\pgfqpoint{0.901187in}{0.672098in}}{\pgfqpoint{0.911786in}{0.676488in}}{\pgfqpoint{0.919600in}{0.684302in}}%
\pgfpathcurveto{\pgfqpoint{0.927413in}{0.692115in}}{\pgfqpoint{0.931804in}{0.702714in}}{\pgfqpoint{0.931804in}{0.713764in}}%
\pgfpathcurveto{\pgfqpoint{0.931804in}{0.724814in}}{\pgfqpoint{0.927413in}{0.735414in}}{\pgfqpoint{0.919600in}{0.743227in}}%
\pgfpathcurveto{\pgfqpoint{0.911786in}{0.751041in}}{\pgfqpoint{0.901187in}{0.755431in}}{\pgfqpoint{0.890137in}{0.755431in}}%
\pgfpathcurveto{\pgfqpoint{0.879087in}{0.755431in}}{\pgfqpoint{0.868488in}{0.751041in}}{\pgfqpoint{0.860674in}{0.743227in}}%
\pgfpathcurveto{\pgfqpoint{0.852860in}{0.735414in}}{\pgfqpoint{0.848470in}{0.724814in}}{\pgfqpoint{0.848470in}{0.713764in}}%
\pgfpathcurveto{\pgfqpoint{0.848470in}{0.702714in}}{\pgfqpoint{0.852860in}{0.692115in}}{\pgfqpoint{0.860674in}{0.684302in}}%
\pgfpathcurveto{\pgfqpoint{0.868488in}{0.676488in}}{\pgfqpoint{0.879087in}{0.672098in}}{\pgfqpoint{0.890137in}{0.672098in}}%
\pgfpathclose%
\pgfusepath{stroke,fill}%
\end{pgfscope}%
\begin{pgfscope}%
\pgfpathrectangle{\pgfqpoint{0.772069in}{0.515123in}}{\pgfqpoint{1.937500in}{1.347500in}}%
\pgfusepath{clip}%
\pgfsetbuttcap%
\pgfsetroundjoin%
\definecolor{currentfill}{rgb}{0.529412,0.807843,0.921569}%
\pgfsetfillcolor{currentfill}%
\pgfsetlinewidth{1.003750pt}%
\definecolor{currentstroke}{rgb}{0.529412,0.807843,0.921569}%
\pgfsetstrokecolor{currentstroke}%
\pgfsetdash{}{0pt}%
\pgfpathmoveto{\pgfqpoint{0.959192in}{0.784750in}}%
\pgfpathcurveto{\pgfqpoint{0.970242in}{0.784750in}}{\pgfqpoint{0.980841in}{0.789140in}}{\pgfqpoint{0.988655in}{0.796954in}}%
\pgfpathcurveto{\pgfqpoint{0.996468in}{0.804767in}}{\pgfqpoint{1.000859in}{0.815366in}}{\pgfqpoint{1.000859in}{0.826416in}}%
\pgfpathcurveto{\pgfqpoint{1.000859in}{0.837467in}}{\pgfqpoint{0.996468in}{0.848066in}}{\pgfqpoint{0.988655in}{0.855879in}}%
\pgfpathcurveto{\pgfqpoint{0.980841in}{0.863693in}}{\pgfqpoint{0.970242in}{0.868083in}}{\pgfqpoint{0.959192in}{0.868083in}}%
\pgfpathcurveto{\pgfqpoint{0.948142in}{0.868083in}}{\pgfqpoint{0.937543in}{0.863693in}}{\pgfqpoint{0.929729in}{0.855879in}}%
\pgfpathcurveto{\pgfqpoint{0.921916in}{0.848066in}}{\pgfqpoint{0.917525in}{0.837467in}}{\pgfqpoint{0.917525in}{0.826416in}}%
\pgfpathcurveto{\pgfqpoint{0.917525in}{0.815366in}}{\pgfqpoint{0.921916in}{0.804767in}}{\pgfqpoint{0.929729in}{0.796954in}}%
\pgfpathcurveto{\pgfqpoint{0.937543in}{0.789140in}}{\pgfqpoint{0.948142in}{0.784750in}}{\pgfqpoint{0.959192in}{0.784750in}}%
\pgfpathclose%
\pgfusepath{stroke,fill}%
\end{pgfscope}%
\begin{pgfscope}%
\pgfpathrectangle{\pgfqpoint{0.772069in}{0.515123in}}{\pgfqpoint{1.937500in}{1.347500in}}%
\pgfusepath{clip}%
\pgfsetbuttcap%
\pgfsetroundjoin%
\definecolor{currentfill}{rgb}{0.529412,0.807843,0.921569}%
\pgfsetfillcolor{currentfill}%
\pgfsetlinewidth{1.003750pt}%
\definecolor{currentstroke}{rgb}{0.529412,0.807843,0.921569}%
\pgfsetstrokecolor{currentstroke}%
\pgfsetdash{}{0pt}%
\pgfpathmoveto{\pgfqpoint{1.005229in}{0.905781in}}%
\pgfpathcurveto{\pgfqpoint{1.016279in}{0.905781in}}{\pgfqpoint{1.026878in}{0.910171in}}{\pgfqpoint{1.034691in}{0.917985in}}%
\pgfpathcurveto{\pgfqpoint{1.042505in}{0.925798in}}{\pgfqpoint{1.046895in}{0.936398in}}{\pgfqpoint{1.046895in}{0.947448in}}%
\pgfpathcurveto{\pgfqpoint{1.046895in}{0.958498in}}{\pgfqpoint{1.042505in}{0.969097in}}{\pgfqpoint{1.034691in}{0.976910in}}%
\pgfpathcurveto{\pgfqpoint{1.026878in}{0.984724in}}{\pgfqpoint{1.016279in}{0.989114in}}{\pgfqpoint{1.005229in}{0.989114in}}%
\pgfpathcurveto{\pgfqpoint{0.994179in}{0.989114in}}{\pgfqpoint{0.983580in}{0.984724in}}{\pgfqpoint{0.975766in}{0.976910in}}%
\pgfpathcurveto{\pgfqpoint{0.967952in}{0.969097in}}{\pgfqpoint{0.963562in}{0.958498in}}{\pgfqpoint{0.963562in}{0.947448in}}%
\pgfpathcurveto{\pgfqpoint{0.963562in}{0.936398in}}{\pgfqpoint{0.967952in}{0.925798in}}{\pgfqpoint{0.975766in}{0.917985in}}%
\pgfpathcurveto{\pgfqpoint{0.983580in}{0.910171in}}{\pgfqpoint{0.994179in}{0.905781in}}{\pgfqpoint{1.005229in}{0.905781in}}%
\pgfpathclose%
\pgfusepath{stroke,fill}%
\end{pgfscope}%
\begin{pgfscope}%
\pgfpathrectangle{\pgfqpoint{0.772069in}{0.515123in}}{\pgfqpoint{1.937500in}{1.347500in}}%
\pgfusepath{clip}%
\pgfsetbuttcap%
\pgfsetroundjoin%
\definecolor{currentfill}{rgb}{0.529412,0.807843,0.921569}%
\pgfsetfillcolor{currentfill}%
\pgfsetlinewidth{1.003750pt}%
\definecolor{currentstroke}{rgb}{0.529412,0.807843,0.921569}%
\pgfsetstrokecolor{currentstroke}%
\pgfsetdash{}{0pt}%
\pgfpathmoveto{\pgfqpoint{1.074284in}{1.024485in}}%
\pgfpathcurveto{\pgfqpoint{1.085334in}{1.024485in}}{\pgfqpoint{1.095933in}{1.028875in}}{\pgfqpoint{1.103747in}{1.036689in}}%
\pgfpathcurveto{\pgfqpoint{1.111560in}{1.044502in}}{\pgfqpoint{1.115950in}{1.055101in}}{\pgfqpoint{1.115950in}{1.066151in}}%
\pgfpathcurveto{\pgfqpoint{1.115950in}{1.077201in}}{\pgfqpoint{1.111560in}{1.087800in}}{\pgfqpoint{1.103747in}{1.095614in}}%
\pgfpathcurveto{\pgfqpoint{1.095933in}{1.103428in}}{\pgfqpoint{1.085334in}{1.107818in}}{\pgfqpoint{1.074284in}{1.107818in}}%
\pgfpathcurveto{\pgfqpoint{1.063234in}{1.107818in}}{\pgfqpoint{1.052635in}{1.103428in}}{\pgfqpoint{1.044821in}{1.095614in}}%
\pgfpathcurveto{\pgfqpoint{1.037007in}{1.087800in}}{\pgfqpoint{1.032617in}{1.077201in}}{\pgfqpoint{1.032617in}{1.066151in}}%
\pgfpathcurveto{\pgfqpoint{1.032617in}{1.055101in}}{\pgfqpoint{1.037007in}{1.044502in}}{\pgfqpoint{1.044821in}{1.036689in}}%
\pgfpathcurveto{\pgfqpoint{1.052635in}{1.028875in}}{\pgfqpoint{1.063234in}{1.024485in}}{\pgfqpoint{1.074284in}{1.024485in}}%
\pgfpathclose%
\pgfusepath{stroke,fill}%
\end{pgfscope}%
\begin{pgfscope}%
\pgfpathrectangle{\pgfqpoint{0.772069in}{0.515123in}}{\pgfqpoint{1.937500in}{1.347500in}}%
\pgfusepath{clip}%
\pgfsetbuttcap%
\pgfsetroundjoin%
\definecolor{currentfill}{rgb}{0.529412,0.807843,0.921569}%
\pgfsetfillcolor{currentfill}%
\pgfsetlinewidth{1.003750pt}%
\definecolor{currentstroke}{rgb}{0.529412,0.807843,0.921569}%
\pgfsetstrokecolor{currentstroke}%
\pgfsetdash{}{0pt}%
\pgfpathmoveto{\pgfqpoint{1.143339in}{1.143188in}}%
\pgfpathcurveto{\pgfqpoint{1.154389in}{1.143188in}}{\pgfqpoint{1.164988in}{1.147579in}}{\pgfqpoint{1.172802in}{1.155392in}}%
\pgfpathcurveto{\pgfqpoint{1.180615in}{1.163206in}}{\pgfqpoint{1.185006in}{1.173805in}}{\pgfqpoint{1.185006in}{1.184855in}}%
\pgfpathcurveto{\pgfqpoint{1.185006in}{1.195905in}}{\pgfqpoint{1.180615in}{1.206504in}}{\pgfqpoint{1.172802in}{1.214318in}}%
\pgfpathcurveto{\pgfqpoint{1.164988in}{1.222131in}}{\pgfqpoint{1.154389in}{1.226522in}}{\pgfqpoint{1.143339in}{1.226522in}}%
\pgfpathcurveto{\pgfqpoint{1.132289in}{1.226522in}}{\pgfqpoint{1.121690in}{1.222131in}}{\pgfqpoint{1.113876in}{1.214318in}}%
\pgfpathcurveto{\pgfqpoint{1.106062in}{1.206504in}}{\pgfqpoint{1.101672in}{1.195905in}}{\pgfqpoint{1.101672in}{1.184855in}}%
\pgfpathcurveto{\pgfqpoint{1.101672in}{1.173805in}}{\pgfqpoint{1.106062in}{1.163206in}}{\pgfqpoint{1.113876in}{1.155392in}}%
\pgfpathcurveto{\pgfqpoint{1.121690in}{1.147579in}}{\pgfqpoint{1.132289in}{1.143188in}}{\pgfqpoint{1.143339in}{1.143188in}}%
\pgfpathclose%
\pgfusepath{stroke,fill}%
\end{pgfscope}%
\begin{pgfscope}%
\pgfpathrectangle{\pgfqpoint{0.772069in}{0.515123in}}{\pgfqpoint{1.937500in}{1.347500in}}%
\pgfusepath{clip}%
\pgfsetbuttcap%
\pgfsetroundjoin%
\definecolor{currentfill}{rgb}{0.529412,0.807843,0.921569}%
\pgfsetfillcolor{currentfill}%
\pgfsetlinewidth{1.003750pt}%
\definecolor{currentstroke}{rgb}{0.529412,0.807843,0.921569}%
\pgfsetstrokecolor{currentstroke}%
\pgfsetdash{}{0pt}%
\pgfpathmoveto{\pgfqpoint{1.189376in}{1.256073in}}%
\pgfpathcurveto{\pgfqpoint{1.200426in}{1.256073in}}{\pgfqpoint{1.211025in}{1.260463in}}{\pgfqpoint{1.218838in}{1.268277in}}%
\pgfpathcurveto{\pgfqpoint{1.226652in}{1.276091in}}{\pgfqpoint{1.231042in}{1.286690in}}{\pgfqpoint{1.231042in}{1.297740in}}%
\pgfpathcurveto{\pgfqpoint{1.231042in}{1.308790in}}{\pgfqpoint{1.226652in}{1.319389in}}{\pgfqpoint{1.218838in}{1.327203in}}%
\pgfpathcurveto{\pgfqpoint{1.211025in}{1.335016in}}{\pgfqpoint{1.200426in}{1.339406in}}{\pgfqpoint{1.189376in}{1.339406in}}%
\pgfpathcurveto{\pgfqpoint{1.178325in}{1.339406in}}{\pgfqpoint{1.167726in}{1.335016in}}{\pgfqpoint{1.159913in}{1.327203in}}%
\pgfpathcurveto{\pgfqpoint{1.152099in}{1.319389in}}{\pgfqpoint{1.147709in}{1.308790in}}{\pgfqpoint{1.147709in}{1.297740in}}%
\pgfpathcurveto{\pgfqpoint{1.147709in}{1.286690in}}{\pgfqpoint{1.152099in}{1.276091in}}{\pgfqpoint{1.159913in}{1.268277in}}%
\pgfpathcurveto{\pgfqpoint{1.167726in}{1.260463in}}{\pgfqpoint{1.178325in}{1.256073in}}{\pgfqpoint{1.189376in}{1.256073in}}%
\pgfpathclose%
\pgfusepath{stroke,fill}%
\end{pgfscope}%
\begin{pgfscope}%
\pgfpathrectangle{\pgfqpoint{0.772069in}{0.515123in}}{\pgfqpoint{1.937500in}{1.347500in}}%
\pgfusepath{clip}%
\pgfsetbuttcap%
\pgfsetroundjoin%
\definecolor{currentfill}{rgb}{0.529412,0.807843,0.921569}%
\pgfsetfillcolor{currentfill}%
\pgfsetlinewidth{1.003750pt}%
\definecolor{currentstroke}{rgb}{0.529412,0.807843,0.921569}%
\pgfsetstrokecolor{currentstroke}%
\pgfsetdash{}{0pt}%
\pgfpathmoveto{\pgfqpoint{1.258431in}{1.378268in}}%
\pgfpathcurveto{\pgfqpoint{1.269481in}{1.378268in}}{\pgfqpoint{1.280080in}{1.382658in}}{\pgfqpoint{1.287893in}{1.390472in}}%
\pgfpathcurveto{\pgfqpoint{1.295707in}{1.398286in}}{\pgfqpoint{1.300097in}{1.408885in}}{\pgfqpoint{1.300097in}{1.419935in}}%
\pgfpathcurveto{\pgfqpoint{1.300097in}{1.430985in}}{\pgfqpoint{1.295707in}{1.441584in}}{\pgfqpoint{1.287893in}{1.449398in}}%
\pgfpathcurveto{\pgfqpoint{1.280080in}{1.457211in}}{\pgfqpoint{1.269481in}{1.461601in}}{\pgfqpoint{1.258431in}{1.461601in}}%
\pgfpathcurveto{\pgfqpoint{1.247381in}{1.461601in}}{\pgfqpoint{1.236781in}{1.457211in}}{\pgfqpoint{1.228968in}{1.449398in}}%
\pgfpathcurveto{\pgfqpoint{1.221154in}{1.441584in}}{\pgfqpoint{1.216764in}{1.430985in}}{\pgfqpoint{1.216764in}{1.419935in}}%
\pgfpathcurveto{\pgfqpoint{1.216764in}{1.408885in}}{\pgfqpoint{1.221154in}{1.398286in}}{\pgfqpoint{1.228968in}{1.390472in}}%
\pgfpathcurveto{\pgfqpoint{1.236781in}{1.382658in}}{\pgfqpoint{1.247381in}{1.378268in}}{\pgfqpoint{1.258431in}{1.378268in}}%
\pgfpathclose%
\pgfusepath{stroke,fill}%
\end{pgfscope}%
\begin{pgfscope}%
\pgfpathrectangle{\pgfqpoint{0.772069in}{0.515123in}}{\pgfqpoint{1.937500in}{1.347500in}}%
\pgfusepath{clip}%
\pgfsetbuttcap%
\pgfsetroundjoin%
\definecolor{currentfill}{rgb}{0.529412,0.807843,0.921569}%
\pgfsetfillcolor{currentfill}%
\pgfsetlinewidth{1.003750pt}%
\definecolor{currentstroke}{rgb}{0.529412,0.807843,0.921569}%
\pgfsetstrokecolor{currentstroke}%
\pgfsetdash{}{0pt}%
\pgfpathmoveto{\pgfqpoint{1.327486in}{1.495808in}}%
\pgfpathcurveto{\pgfqpoint{1.338536in}{1.495808in}}{\pgfqpoint{1.349135in}{1.500198in}}{\pgfqpoint{1.356949in}{1.508012in}}%
\pgfpathcurveto{\pgfqpoint{1.364762in}{1.515826in}}{\pgfqpoint{1.369152in}{1.526425in}}{\pgfqpoint{1.369152in}{1.537475in}}%
\pgfpathcurveto{\pgfqpoint{1.369152in}{1.548525in}}{\pgfqpoint{1.364762in}{1.559124in}}{\pgfqpoint{1.356949in}{1.566937in}}%
\pgfpathcurveto{\pgfqpoint{1.349135in}{1.574751in}}{\pgfqpoint{1.338536in}{1.579141in}}{\pgfqpoint{1.327486in}{1.579141in}}%
\pgfpathcurveto{\pgfqpoint{1.316436in}{1.579141in}}{\pgfqpoint{1.305837in}{1.574751in}}{\pgfqpoint{1.298023in}{1.566937in}}%
\pgfpathcurveto{\pgfqpoint{1.290209in}{1.559124in}}{\pgfqpoint{1.285819in}{1.548525in}}{\pgfqpoint{1.285819in}{1.537475in}}%
\pgfpathcurveto{\pgfqpoint{1.285819in}{1.526425in}}{\pgfqpoint{1.290209in}{1.515826in}}{\pgfqpoint{1.298023in}{1.508012in}}%
\pgfpathcurveto{\pgfqpoint{1.305837in}{1.500198in}}{\pgfqpoint{1.316436in}{1.495808in}}{\pgfqpoint{1.327486in}{1.495808in}}%
\pgfpathclose%
\pgfusepath{stroke,fill}%
\end{pgfscope}%
\begin{pgfscope}%
\pgfpathrectangle{\pgfqpoint{0.772069in}{0.515123in}}{\pgfqpoint{1.937500in}{1.347500in}}%
\pgfusepath{clip}%
\pgfsetbuttcap%
\pgfsetroundjoin%
\definecolor{currentfill}{rgb}{0.529412,0.807843,0.921569}%
\pgfsetfillcolor{currentfill}%
\pgfsetlinewidth{1.003750pt}%
\definecolor{currentstroke}{rgb}{0.529412,0.807843,0.921569}%
\pgfsetstrokecolor{currentstroke}%
\pgfsetdash{}{0pt}%
\pgfpathmoveto{\pgfqpoint{1.396541in}{1.611020in}}%
\pgfpathcurveto{\pgfqpoint{1.407591in}{1.611020in}}{\pgfqpoint{1.418190in}{1.615411in}}{\pgfqpoint{1.426004in}{1.623224in}}%
\pgfpathcurveto{\pgfqpoint{1.433817in}{1.631038in}}{\pgfqpoint{1.438208in}{1.641637in}}{\pgfqpoint{1.438208in}{1.652687in}}%
\pgfpathcurveto{\pgfqpoint{1.438208in}{1.663737in}}{\pgfqpoint{1.433817in}{1.674336in}}{\pgfqpoint{1.426004in}{1.682150in}}%
\pgfpathcurveto{\pgfqpoint{1.418190in}{1.689963in}}{\pgfqpoint{1.407591in}{1.694354in}}{\pgfqpoint{1.396541in}{1.694354in}}%
\pgfpathcurveto{\pgfqpoint{1.385491in}{1.694354in}}{\pgfqpoint{1.374892in}{1.689963in}}{\pgfqpoint{1.367078in}{1.682150in}}%
\pgfpathcurveto{\pgfqpoint{1.359264in}{1.674336in}}{\pgfqpoint{1.354874in}{1.663737in}}{\pgfqpoint{1.354874in}{1.652687in}}%
\pgfpathcurveto{\pgfqpoint{1.354874in}{1.641637in}}{\pgfqpoint{1.359264in}{1.631038in}}{\pgfqpoint{1.367078in}{1.623224in}}%
\pgfpathcurveto{\pgfqpoint{1.374892in}{1.615411in}}{\pgfqpoint{1.385491in}{1.611020in}}{\pgfqpoint{1.396541in}{1.611020in}}%
\pgfpathclose%
\pgfusepath{stroke,fill}%
\end{pgfscope}%
\begin{pgfscope}%
\pgfpathrectangle{\pgfqpoint{0.772069in}{0.515123in}}{\pgfqpoint{1.937500in}{1.347500in}}%
\pgfusepath{clip}%
\pgfsetbuttcap%
\pgfsetroundjoin%
\definecolor{currentfill}{rgb}{0.529412,0.807843,0.921569}%
\pgfsetfillcolor{currentfill}%
\pgfsetlinewidth{1.003750pt}%
\definecolor{currentstroke}{rgb}{0.529412,0.807843,0.921569}%
\pgfsetstrokecolor{currentstroke}%
\pgfsetdash{}{0pt}%
\pgfpathmoveto{\pgfqpoint{1.442578in}{1.729724in}}%
\pgfpathcurveto{\pgfqpoint{1.453628in}{1.729724in}}{\pgfqpoint{1.464227in}{1.734114in}}{\pgfqpoint{1.472040in}{1.741928in}}%
\pgfpathcurveto{\pgfqpoint{1.479854in}{1.749742in}}{\pgfqpoint{1.484244in}{1.760341in}}{\pgfqpoint{1.484244in}{1.771391in}}%
\pgfpathcurveto{\pgfqpoint{1.484244in}{1.782441in}}{\pgfqpoint{1.479854in}{1.793040in}}{\pgfqpoint{1.472040in}{1.800854in}}%
\pgfpathcurveto{\pgfqpoint{1.464227in}{1.808667in}}{\pgfqpoint{1.453628in}{1.813057in}}{\pgfqpoint{1.442578in}{1.813057in}}%
\pgfpathcurveto{\pgfqpoint{1.431527in}{1.813057in}}{\pgfqpoint{1.420928in}{1.808667in}}{\pgfqpoint{1.413115in}{1.800854in}}%
\pgfpathcurveto{\pgfqpoint{1.405301in}{1.793040in}}{\pgfqpoint{1.400911in}{1.782441in}}{\pgfqpoint{1.400911in}{1.771391in}}%
\pgfpathcurveto{\pgfqpoint{1.400911in}{1.760341in}}{\pgfqpoint{1.405301in}{1.749742in}}{\pgfqpoint{1.413115in}{1.741928in}}%
\pgfpathcurveto{\pgfqpoint{1.420928in}{1.734114in}}{\pgfqpoint{1.431527in}{1.729724in}}{\pgfqpoint{1.442578in}{1.729724in}}%
\pgfpathclose%
\pgfusepath{stroke,fill}%
\end{pgfscope}%
\begin{pgfscope}%
\pgfpathrectangle{\pgfqpoint{0.772069in}{0.515123in}}{\pgfqpoint{1.937500in}{1.347500in}}%
\pgfusepath{clip}%
\pgfsetbuttcap%
\pgfsetroundjoin%
\definecolor{currentfill}{rgb}{1.000000,0.854902,0.725490}%
\pgfsetfillcolor{currentfill}%
\pgfsetlinewidth{1.003750pt}%
\definecolor{currentstroke}{rgb}{1.000000,0.854902,0.725490}%
\pgfsetstrokecolor{currentstroke}%
\pgfsetdash{}{0pt}%
\pgfpathmoveto{\pgfqpoint{0.890137in}{0.567906in}}%
\pgfpathcurveto{\pgfqpoint{0.901187in}{0.567906in}}{\pgfqpoint{0.911786in}{0.572296in}}{\pgfqpoint{0.919600in}{0.580110in}}%
\pgfpathcurveto{\pgfqpoint{0.927413in}{0.587924in}}{\pgfqpoint{0.931804in}{0.598523in}}{\pgfqpoint{0.931804in}{0.609573in}}%
\pgfpathcurveto{\pgfqpoint{0.931804in}{0.620623in}}{\pgfqpoint{0.927413in}{0.631222in}}{\pgfqpoint{0.919600in}{0.639036in}}%
\pgfpathcurveto{\pgfqpoint{0.911786in}{0.646849in}}{\pgfqpoint{0.901187in}{0.651239in}}{\pgfqpoint{0.890137in}{0.651239in}}%
\pgfpathcurveto{\pgfqpoint{0.879087in}{0.651239in}}{\pgfqpoint{0.868488in}{0.646849in}}{\pgfqpoint{0.860674in}{0.639036in}}%
\pgfpathcurveto{\pgfqpoint{0.852860in}{0.631222in}}{\pgfqpoint{0.848470in}{0.620623in}}{\pgfqpoint{0.848470in}{0.609573in}}%
\pgfpathcurveto{\pgfqpoint{0.848470in}{0.598523in}}{\pgfqpoint{0.852860in}{0.587924in}}{\pgfqpoint{0.860674in}{0.580110in}}%
\pgfpathcurveto{\pgfqpoint{0.868488in}{0.572296in}}{\pgfqpoint{0.879087in}{0.567906in}}{\pgfqpoint{0.890137in}{0.567906in}}%
\pgfpathclose%
\pgfusepath{stroke,fill}%
\end{pgfscope}%
\begin{pgfscope}%
\pgfpathrectangle{\pgfqpoint{0.772069in}{0.515123in}}{\pgfqpoint{1.937500in}{1.347500in}}%
\pgfusepath{clip}%
\pgfsetbuttcap%
\pgfsetroundjoin%
\definecolor{currentfill}{rgb}{1.000000,0.854902,0.725490}%
\pgfsetfillcolor{currentfill}%
\pgfsetlinewidth{1.003750pt}%
\definecolor{currentstroke}{rgb}{1.000000,0.854902,0.725490}%
\pgfsetstrokecolor{currentstroke}%
\pgfsetdash{}{0pt}%
\pgfpathmoveto{\pgfqpoint{0.959192in}{0.582372in}}%
\pgfpathcurveto{\pgfqpoint{0.970242in}{0.582372in}}{\pgfqpoint{0.980841in}{0.586762in}}{\pgfqpoint{0.988655in}{0.594576in}}%
\pgfpathcurveto{\pgfqpoint{0.996468in}{0.602389in}}{\pgfqpoint{1.000859in}{0.612988in}}{\pgfqpoint{1.000859in}{0.624038in}}%
\pgfpathcurveto{\pgfqpoint{1.000859in}{0.635088in}}{\pgfqpoint{0.996468in}{0.645688in}}{\pgfqpoint{0.988655in}{0.653501in}}%
\pgfpathcurveto{\pgfqpoint{0.980841in}{0.661315in}}{\pgfqpoint{0.970242in}{0.665705in}}{\pgfqpoint{0.959192in}{0.665705in}}%
\pgfpathcurveto{\pgfqpoint{0.948142in}{0.665705in}}{\pgfqpoint{0.937543in}{0.661315in}}{\pgfqpoint{0.929729in}{0.653501in}}%
\pgfpathcurveto{\pgfqpoint{0.921916in}{0.645688in}}{\pgfqpoint{0.917525in}{0.635088in}}{\pgfqpoint{0.917525in}{0.624038in}}%
\pgfpathcurveto{\pgfqpoint{0.917525in}{0.612988in}}{\pgfqpoint{0.921916in}{0.602389in}}{\pgfqpoint{0.929729in}{0.594576in}}%
\pgfpathcurveto{\pgfqpoint{0.937543in}{0.586762in}}{\pgfqpoint{0.948142in}{0.582372in}}{\pgfqpoint{0.959192in}{0.582372in}}%
\pgfpathclose%
\pgfusepath{stroke,fill}%
\end{pgfscope}%
\begin{pgfscope}%
\pgfpathrectangle{\pgfqpoint{0.772069in}{0.515123in}}{\pgfqpoint{1.937500in}{1.347500in}}%
\pgfusepath{clip}%
\pgfsetbuttcap%
\pgfsetroundjoin%
\definecolor{currentfill}{rgb}{1.000000,0.854902,0.725490}%
\pgfsetfillcolor{currentfill}%
\pgfsetlinewidth{1.003750pt}%
\definecolor{currentstroke}{rgb}{1.000000,0.854902,0.725490}%
\pgfsetstrokecolor{currentstroke}%
\pgfsetdash{}{0pt}%
\pgfpathmoveto{\pgfqpoint{1.005229in}{0.586561in}}%
\pgfpathcurveto{\pgfqpoint{1.016279in}{0.586561in}}{\pgfqpoint{1.026878in}{0.590951in}}{\pgfqpoint{1.034691in}{0.598765in}}%
\pgfpathcurveto{\pgfqpoint{1.042505in}{0.606579in}}{\pgfqpoint{1.046895in}{0.617178in}}{\pgfqpoint{1.046895in}{0.628228in}}%
\pgfpathcurveto{\pgfqpoint{1.046895in}{0.639278in}}{\pgfqpoint{1.042505in}{0.649877in}}{\pgfqpoint{1.034691in}{0.657691in}}%
\pgfpathcurveto{\pgfqpoint{1.026878in}{0.665504in}}{\pgfqpoint{1.016279in}{0.669895in}}{\pgfqpoint{1.005229in}{0.669895in}}%
\pgfpathcurveto{\pgfqpoint{0.994179in}{0.669895in}}{\pgfqpoint{0.983580in}{0.665504in}}{\pgfqpoint{0.975766in}{0.657691in}}%
\pgfpathcurveto{\pgfqpoint{0.967952in}{0.649877in}}{\pgfqpoint{0.963562in}{0.639278in}}{\pgfqpoint{0.963562in}{0.628228in}}%
\pgfpathcurveto{\pgfqpoint{0.963562in}{0.617178in}}{\pgfqpoint{0.967952in}{0.606579in}}{\pgfqpoint{0.975766in}{0.598765in}}%
\pgfpathcurveto{\pgfqpoint{0.983580in}{0.590951in}}{\pgfqpoint{0.994179in}{0.586561in}}{\pgfqpoint{1.005229in}{0.586561in}}%
\pgfpathclose%
\pgfusepath{stroke,fill}%
\end{pgfscope}%
\begin{pgfscope}%
\pgfpathrectangle{\pgfqpoint{0.772069in}{0.515123in}}{\pgfqpoint{1.937500in}{1.347500in}}%
\pgfusepath{clip}%
\pgfsetbuttcap%
\pgfsetroundjoin%
\definecolor{currentfill}{rgb}{1.000000,0.854902,0.725490}%
\pgfsetfillcolor{currentfill}%
\pgfsetlinewidth{1.003750pt}%
\definecolor{currentstroke}{rgb}{1.000000,0.854902,0.725490}%
\pgfsetstrokecolor{currentstroke}%
\pgfsetdash{}{0pt}%
\pgfpathmoveto{\pgfqpoint{1.074284in}{0.613793in}}%
\pgfpathcurveto{\pgfqpoint{1.085334in}{0.613793in}}{\pgfqpoint{1.095933in}{0.618183in}}{\pgfqpoint{1.103747in}{0.625997in}}%
\pgfpathcurveto{\pgfqpoint{1.111560in}{0.633811in}}{\pgfqpoint{1.115950in}{0.644410in}}{\pgfqpoint{1.115950in}{0.655460in}}%
\pgfpathcurveto{\pgfqpoint{1.115950in}{0.666510in}}{\pgfqpoint{1.111560in}{0.677109in}}{\pgfqpoint{1.103747in}{0.684923in}}%
\pgfpathcurveto{\pgfqpoint{1.095933in}{0.692736in}}{\pgfqpoint{1.085334in}{0.697127in}}{\pgfqpoint{1.074284in}{0.697127in}}%
\pgfpathcurveto{\pgfqpoint{1.063234in}{0.697127in}}{\pgfqpoint{1.052635in}{0.692736in}}{\pgfqpoint{1.044821in}{0.684923in}}%
\pgfpathcurveto{\pgfqpoint{1.037007in}{0.677109in}}{\pgfqpoint{1.032617in}{0.666510in}}{\pgfqpoint{1.032617in}{0.655460in}}%
\pgfpathcurveto{\pgfqpoint{1.032617in}{0.644410in}}{\pgfqpoint{1.037007in}{0.633811in}}{\pgfqpoint{1.044821in}{0.625997in}}%
\pgfpathcurveto{\pgfqpoint{1.052635in}{0.618183in}}{\pgfqpoint{1.063234in}{0.613793in}}{\pgfqpoint{1.074284in}{0.613793in}}%
\pgfpathclose%
\pgfusepath{stroke,fill}%
\end{pgfscope}%
\begin{pgfscope}%
\pgfpathrectangle{\pgfqpoint{0.772069in}{0.515123in}}{\pgfqpoint{1.937500in}{1.347500in}}%
\pgfusepath{clip}%
\pgfsetbuttcap%
\pgfsetroundjoin%
\definecolor{currentfill}{rgb}{1.000000,0.854902,0.725490}%
\pgfsetfillcolor{currentfill}%
\pgfsetlinewidth{1.003750pt}%
\definecolor{currentstroke}{rgb}{1.000000,0.854902,0.725490}%
\pgfsetstrokecolor{currentstroke}%
\pgfsetdash{}{0pt}%
\pgfpathmoveto{\pgfqpoint{1.143339in}{0.629271in}}%
\pgfpathcurveto{\pgfqpoint{1.154389in}{0.629271in}}{\pgfqpoint{1.164988in}{0.633662in}}{\pgfqpoint{1.172802in}{0.641475in}}%
\pgfpathcurveto{\pgfqpoint{1.180615in}{0.649289in}}{\pgfqpoint{1.185006in}{0.659888in}}{\pgfqpoint{1.185006in}{0.670938in}}%
\pgfpathcurveto{\pgfqpoint{1.185006in}{0.681988in}}{\pgfqpoint{1.180615in}{0.692587in}}{\pgfqpoint{1.172802in}{0.700401in}}%
\pgfpathcurveto{\pgfqpoint{1.164988in}{0.708214in}}{\pgfqpoint{1.154389in}{0.712605in}}{\pgfqpoint{1.143339in}{0.712605in}}%
\pgfpathcurveto{\pgfqpoint{1.132289in}{0.712605in}}{\pgfqpoint{1.121690in}{0.708214in}}{\pgfqpoint{1.113876in}{0.700401in}}%
\pgfpathcurveto{\pgfqpoint{1.106062in}{0.692587in}}{\pgfqpoint{1.101672in}{0.681988in}}{\pgfqpoint{1.101672in}{0.670938in}}%
\pgfpathcurveto{\pgfqpoint{1.101672in}{0.659888in}}{\pgfqpoint{1.106062in}{0.649289in}}{\pgfqpoint{1.113876in}{0.641475in}}%
\pgfpathcurveto{\pgfqpoint{1.121690in}{0.633662in}}{\pgfqpoint{1.132289in}{0.629271in}}{\pgfqpoint{1.143339in}{0.629271in}}%
\pgfpathclose%
\pgfusepath{stroke,fill}%
\end{pgfscope}%
\begin{pgfscope}%
\pgfpathrectangle{\pgfqpoint{0.772069in}{0.515123in}}{\pgfqpoint{1.937500in}{1.347500in}}%
\pgfusepath{clip}%
\pgfsetbuttcap%
\pgfsetroundjoin%
\definecolor{currentfill}{rgb}{1.000000,0.854902,0.725490}%
\pgfsetfillcolor{currentfill}%
\pgfsetlinewidth{1.003750pt}%
\definecolor{currentstroke}{rgb}{1.000000,0.854902,0.725490}%
\pgfsetstrokecolor{currentstroke}%
\pgfsetdash{}{0pt}%
\pgfpathmoveto{\pgfqpoint{1.189376in}{0.644051in}}%
\pgfpathcurveto{\pgfqpoint{1.200426in}{0.644051in}}{\pgfqpoint{1.211025in}{0.648441in}}{\pgfqpoint{1.218838in}{0.656255in}}%
\pgfpathcurveto{\pgfqpoint{1.226652in}{0.664069in}}{\pgfqpoint{1.231042in}{0.674668in}}{\pgfqpoint{1.231042in}{0.685718in}}%
\pgfpathcurveto{\pgfqpoint{1.231042in}{0.696768in}}{\pgfqpoint{1.226652in}{0.707367in}}{\pgfqpoint{1.218838in}{0.715180in}}%
\pgfpathcurveto{\pgfqpoint{1.211025in}{0.722994in}}{\pgfqpoint{1.200426in}{0.727384in}}{\pgfqpoint{1.189376in}{0.727384in}}%
\pgfpathcurveto{\pgfqpoint{1.178325in}{0.727384in}}{\pgfqpoint{1.167726in}{0.722994in}}{\pgfqpoint{1.159913in}{0.715180in}}%
\pgfpathcurveto{\pgfqpoint{1.152099in}{0.707367in}}{\pgfqpoint{1.147709in}{0.696768in}}{\pgfqpoint{1.147709in}{0.685718in}}%
\pgfpathcurveto{\pgfqpoint{1.147709in}{0.674668in}}{\pgfqpoint{1.152099in}{0.664069in}}{\pgfqpoint{1.159913in}{0.656255in}}%
\pgfpathcurveto{\pgfqpoint{1.167726in}{0.648441in}}{\pgfqpoint{1.178325in}{0.644051in}}{\pgfqpoint{1.189376in}{0.644051in}}%
\pgfpathclose%
\pgfusepath{stroke,fill}%
\end{pgfscope}%
\begin{pgfscope}%
\pgfpathrectangle{\pgfqpoint{0.772069in}{0.515123in}}{\pgfqpoint{1.937500in}{1.347500in}}%
\pgfusepath{clip}%
\pgfsetbuttcap%
\pgfsetroundjoin%
\definecolor{currentfill}{rgb}{1.000000,0.854902,0.725490}%
\pgfsetfillcolor{currentfill}%
\pgfsetlinewidth{1.003750pt}%
\definecolor{currentstroke}{rgb}{1.000000,0.854902,0.725490}%
\pgfsetstrokecolor{currentstroke}%
\pgfsetdash{}{0pt}%
\pgfpathmoveto{\pgfqpoint{1.258431in}{0.660111in}}%
\pgfpathcurveto{\pgfqpoint{1.269481in}{0.660111in}}{\pgfqpoint{1.280080in}{0.664501in}}{\pgfqpoint{1.287893in}{0.672315in}}%
\pgfpathcurveto{\pgfqpoint{1.295707in}{0.680128in}}{\pgfqpoint{1.300097in}{0.690727in}}{\pgfqpoint{1.300097in}{0.701778in}}%
\pgfpathcurveto{\pgfqpoint{1.300097in}{0.712828in}}{\pgfqpoint{1.295707in}{0.723427in}}{\pgfqpoint{1.287893in}{0.731240in}}%
\pgfpathcurveto{\pgfqpoint{1.280080in}{0.739054in}}{\pgfqpoint{1.269481in}{0.743444in}}{\pgfqpoint{1.258431in}{0.743444in}}%
\pgfpathcurveto{\pgfqpoint{1.247381in}{0.743444in}}{\pgfqpoint{1.236781in}{0.739054in}}{\pgfqpoint{1.228968in}{0.731240in}}%
\pgfpathcurveto{\pgfqpoint{1.221154in}{0.723427in}}{\pgfqpoint{1.216764in}{0.712828in}}{\pgfqpoint{1.216764in}{0.701778in}}%
\pgfpathcurveto{\pgfqpoint{1.216764in}{0.690727in}}{\pgfqpoint{1.221154in}{0.680128in}}{\pgfqpoint{1.228968in}{0.672315in}}%
\pgfpathcurveto{\pgfqpoint{1.236781in}{0.664501in}}{\pgfqpoint{1.247381in}{0.660111in}}{\pgfqpoint{1.258431in}{0.660111in}}%
\pgfpathclose%
\pgfusepath{stroke,fill}%
\end{pgfscope}%
\begin{pgfscope}%
\pgfpathrectangle{\pgfqpoint{0.772069in}{0.515123in}}{\pgfqpoint{1.937500in}{1.347500in}}%
\pgfusepath{clip}%
\pgfsetbuttcap%
\pgfsetroundjoin%
\definecolor{currentfill}{rgb}{1.000000,0.854902,0.725490}%
\pgfsetfillcolor{currentfill}%
\pgfsetlinewidth{1.003750pt}%
\definecolor{currentstroke}{rgb}{1.000000,0.854902,0.725490}%
\pgfsetstrokecolor{currentstroke}%
\pgfsetdash{}{0pt}%
\pgfpathmoveto{\pgfqpoint{1.327486in}{0.675240in}}%
\pgfpathcurveto{\pgfqpoint{1.338536in}{0.675240in}}{\pgfqpoint{1.349135in}{0.679630in}}{\pgfqpoint{1.356949in}{0.687444in}}%
\pgfpathcurveto{\pgfqpoint{1.364762in}{0.695257in}}{\pgfqpoint{1.369152in}{0.705856in}}{\pgfqpoint{1.369152in}{0.716907in}}%
\pgfpathcurveto{\pgfqpoint{1.369152in}{0.727957in}}{\pgfqpoint{1.364762in}{0.738556in}}{\pgfqpoint{1.356949in}{0.746369in}}%
\pgfpathcurveto{\pgfqpoint{1.349135in}{0.754183in}}{\pgfqpoint{1.338536in}{0.758573in}}{\pgfqpoint{1.327486in}{0.758573in}}%
\pgfpathcurveto{\pgfqpoint{1.316436in}{0.758573in}}{\pgfqpoint{1.305837in}{0.754183in}}{\pgfqpoint{1.298023in}{0.746369in}}%
\pgfpathcurveto{\pgfqpoint{1.290209in}{0.738556in}}{\pgfqpoint{1.285819in}{0.727957in}}{\pgfqpoint{1.285819in}{0.716907in}}%
\pgfpathcurveto{\pgfqpoint{1.285819in}{0.705856in}}{\pgfqpoint{1.290209in}{0.695257in}}{\pgfqpoint{1.298023in}{0.687444in}}%
\pgfpathcurveto{\pgfqpoint{1.305837in}{0.679630in}}{\pgfqpoint{1.316436in}{0.675240in}}{\pgfqpoint{1.327486in}{0.675240in}}%
\pgfpathclose%
\pgfusepath{stroke,fill}%
\end{pgfscope}%
\begin{pgfscope}%
\pgfpathrectangle{\pgfqpoint{0.772069in}{0.515123in}}{\pgfqpoint{1.937500in}{1.347500in}}%
\pgfusepath{clip}%
\pgfsetbuttcap%
\pgfsetroundjoin%
\definecolor{currentfill}{rgb}{1.000000,0.854902,0.725490}%
\pgfsetfillcolor{currentfill}%
\pgfsetlinewidth{1.003750pt}%
\definecolor{currentstroke}{rgb}{1.000000,0.854902,0.725490}%
\pgfsetstrokecolor{currentstroke}%
\pgfsetdash{}{0pt}%
\pgfpathmoveto{\pgfqpoint{1.396541in}{0.690369in}}%
\pgfpathcurveto{\pgfqpoint{1.407591in}{0.690369in}}{\pgfqpoint{1.418190in}{0.694759in}}{\pgfqpoint{1.426004in}{0.702573in}}%
\pgfpathcurveto{\pgfqpoint{1.433817in}{0.710386in}}{\pgfqpoint{1.438208in}{0.720985in}}{\pgfqpoint{1.438208in}{0.732035in}}%
\pgfpathcurveto{\pgfqpoint{1.438208in}{0.743086in}}{\pgfqpoint{1.433817in}{0.753685in}}{\pgfqpoint{1.426004in}{0.761498in}}%
\pgfpathcurveto{\pgfqpoint{1.418190in}{0.769312in}}{\pgfqpoint{1.407591in}{0.773702in}}{\pgfqpoint{1.396541in}{0.773702in}}%
\pgfpathcurveto{\pgfqpoint{1.385491in}{0.773702in}}{\pgfqpoint{1.374892in}{0.769312in}}{\pgfqpoint{1.367078in}{0.761498in}}%
\pgfpathcurveto{\pgfqpoint{1.359264in}{0.753685in}}{\pgfqpoint{1.354874in}{0.743086in}}{\pgfqpoint{1.354874in}{0.732035in}}%
\pgfpathcurveto{\pgfqpoint{1.354874in}{0.720985in}}{\pgfqpoint{1.359264in}{0.710386in}}{\pgfqpoint{1.367078in}{0.702573in}}%
\pgfpathcurveto{\pgfqpoint{1.374892in}{0.694759in}}{\pgfqpoint{1.385491in}{0.690369in}}{\pgfqpoint{1.396541in}{0.690369in}}%
\pgfpathclose%
\pgfusepath{stroke,fill}%
\end{pgfscope}%
\begin{pgfscope}%
\pgfpathrectangle{\pgfqpoint{0.772069in}{0.515123in}}{\pgfqpoint{1.937500in}{1.347500in}}%
\pgfusepath{clip}%
\pgfsetbuttcap%
\pgfsetroundjoin%
\definecolor{currentfill}{rgb}{1.000000,0.854902,0.725490}%
\pgfsetfillcolor{currentfill}%
\pgfsetlinewidth{1.003750pt}%
\definecolor{currentstroke}{rgb}{1.000000,0.854902,0.725490}%
\pgfsetstrokecolor{currentstroke}%
\pgfsetdash{}{0pt}%
\pgfpathmoveto{\pgfqpoint{1.442578in}{0.705963in}}%
\pgfpathcurveto{\pgfqpoint{1.453628in}{0.705963in}}{\pgfqpoint{1.464227in}{0.710353in}}{\pgfqpoint{1.472040in}{0.718167in}}%
\pgfpathcurveto{\pgfqpoint{1.479854in}{0.725981in}}{\pgfqpoint{1.484244in}{0.736580in}}{\pgfqpoint{1.484244in}{0.747630in}}%
\pgfpathcurveto{\pgfqpoint{1.484244in}{0.758680in}}{\pgfqpoint{1.479854in}{0.769279in}}{\pgfqpoint{1.472040in}{0.777093in}}%
\pgfpathcurveto{\pgfqpoint{1.464227in}{0.784906in}}{\pgfqpoint{1.453628in}{0.789296in}}{\pgfqpoint{1.442578in}{0.789296in}}%
\pgfpathcurveto{\pgfqpoint{1.431527in}{0.789296in}}{\pgfqpoint{1.420928in}{0.784906in}}{\pgfqpoint{1.413115in}{0.777093in}}%
\pgfpathcurveto{\pgfqpoint{1.405301in}{0.769279in}}{\pgfqpoint{1.400911in}{0.758680in}}{\pgfqpoint{1.400911in}{0.747630in}}%
\pgfpathcurveto{\pgfqpoint{1.400911in}{0.736580in}}{\pgfqpoint{1.405301in}{0.725981in}}{\pgfqpoint{1.413115in}{0.718167in}}%
\pgfpathcurveto{\pgfqpoint{1.420928in}{0.710353in}}{\pgfqpoint{1.431527in}{0.705963in}}{\pgfqpoint{1.442578in}{0.705963in}}%
\pgfpathclose%
\pgfusepath{stroke,fill}%
\end{pgfscope}%
\begin{pgfscope}%
\pgfpathrectangle{\pgfqpoint{0.772069in}{0.515123in}}{\pgfqpoint{1.937500in}{1.347500in}}%
\pgfusepath{clip}%
\pgfsetbuttcap%
\pgfsetroundjoin%
\definecolor{currentfill}{rgb}{1.000000,0.980392,0.803922}%
\pgfsetfillcolor{currentfill}%
\pgfsetlinewidth{1.003750pt}%
\definecolor{currentstroke}{rgb}{1.000000,0.980392,0.803922}%
\pgfsetstrokecolor{currentstroke}%
\pgfsetdash{}{0pt}%
\pgfpathmoveto{\pgfqpoint{0.890137in}{0.564706in}}%
\pgfpathcurveto{\pgfqpoint{0.901187in}{0.564706in}}{\pgfqpoint{0.911786in}{0.569096in}}{\pgfqpoint{0.919600in}{0.576910in}}%
\pgfpathcurveto{\pgfqpoint{0.927413in}{0.584723in}}{\pgfqpoint{0.931804in}{0.595322in}}{\pgfqpoint{0.931804in}{0.606372in}}%
\pgfpathcurveto{\pgfqpoint{0.931804in}{0.617423in}}{\pgfqpoint{0.927413in}{0.628022in}}{\pgfqpoint{0.919600in}{0.635835in}}%
\pgfpathcurveto{\pgfqpoint{0.911786in}{0.643649in}}{\pgfqpoint{0.901187in}{0.648039in}}{\pgfqpoint{0.890137in}{0.648039in}}%
\pgfpathcurveto{\pgfqpoint{0.879087in}{0.648039in}}{\pgfqpoint{0.868488in}{0.643649in}}{\pgfqpoint{0.860674in}{0.635835in}}%
\pgfpathcurveto{\pgfqpoint{0.852860in}{0.628022in}}{\pgfqpoint{0.848470in}{0.617423in}}{\pgfqpoint{0.848470in}{0.606372in}}%
\pgfpathcurveto{\pgfqpoint{0.848470in}{0.595322in}}{\pgfqpoint{0.852860in}{0.584723in}}{\pgfqpoint{0.860674in}{0.576910in}}%
\pgfpathcurveto{\pgfqpoint{0.868488in}{0.569096in}}{\pgfqpoint{0.879087in}{0.564706in}}{\pgfqpoint{0.890137in}{0.564706in}}%
\pgfpathclose%
\pgfusepath{stroke,fill}%
\end{pgfscope}%
\begin{pgfscope}%
\pgfpathrectangle{\pgfqpoint{0.772069in}{0.515123in}}{\pgfqpoint{1.937500in}{1.347500in}}%
\pgfusepath{clip}%
\pgfsetbuttcap%
\pgfsetroundjoin%
\definecolor{currentfill}{rgb}{1.000000,0.980392,0.803922}%
\pgfsetfillcolor{currentfill}%
\pgfsetlinewidth{1.003750pt}%
\definecolor{currentstroke}{rgb}{1.000000,0.980392,0.803922}%
\pgfsetstrokecolor{currentstroke}%
\pgfsetdash{}{0pt}%
\pgfpathmoveto{\pgfqpoint{0.959192in}{0.576320in}}%
\pgfpathcurveto{\pgfqpoint{0.970242in}{0.576320in}}{\pgfqpoint{0.980841in}{0.580710in}}{\pgfqpoint{0.988655in}{0.588524in}}%
\pgfpathcurveto{\pgfqpoint{0.996468in}{0.596338in}}{\pgfqpoint{1.000859in}{0.606937in}}{\pgfqpoint{1.000859in}{0.617987in}}%
\pgfpathcurveto{\pgfqpoint{1.000859in}{0.629037in}}{\pgfqpoint{0.996468in}{0.639636in}}{\pgfqpoint{0.988655in}{0.647450in}}%
\pgfpathcurveto{\pgfqpoint{0.980841in}{0.655263in}}{\pgfqpoint{0.970242in}{0.659653in}}{\pgfqpoint{0.959192in}{0.659653in}}%
\pgfpathcurveto{\pgfqpoint{0.948142in}{0.659653in}}{\pgfqpoint{0.937543in}{0.655263in}}{\pgfqpoint{0.929729in}{0.647450in}}%
\pgfpathcurveto{\pgfqpoint{0.921916in}{0.639636in}}{\pgfqpoint{0.917525in}{0.629037in}}{\pgfqpoint{0.917525in}{0.617987in}}%
\pgfpathcurveto{\pgfqpoint{0.917525in}{0.606937in}}{\pgfqpoint{0.921916in}{0.596338in}}{\pgfqpoint{0.929729in}{0.588524in}}%
\pgfpathcurveto{\pgfqpoint{0.937543in}{0.580710in}}{\pgfqpoint{0.948142in}{0.576320in}}{\pgfqpoint{0.959192in}{0.576320in}}%
\pgfpathclose%
\pgfusepath{stroke,fill}%
\end{pgfscope}%
\begin{pgfscope}%
\pgfpathrectangle{\pgfqpoint{0.772069in}{0.515123in}}{\pgfqpoint{1.937500in}{1.347500in}}%
\pgfusepath{clip}%
\pgfsetbuttcap%
\pgfsetroundjoin%
\definecolor{currentfill}{rgb}{1.000000,0.980392,0.803922}%
\pgfsetfillcolor{currentfill}%
\pgfsetlinewidth{1.003750pt}%
\definecolor{currentstroke}{rgb}{1.000000,0.980392,0.803922}%
\pgfsetstrokecolor{currentstroke}%
\pgfsetdash{}{0pt}%
\pgfpathmoveto{\pgfqpoint{1.005229in}{0.588889in}}%
\pgfpathcurveto{\pgfqpoint{1.016279in}{0.588889in}}{\pgfqpoint{1.026878in}{0.593279in}}{\pgfqpoint{1.034691in}{0.601093in}}%
\pgfpathcurveto{\pgfqpoint{1.042505in}{0.608906in}}{\pgfqpoint{1.046895in}{0.619505in}}{\pgfqpoint{1.046895in}{0.630555in}}%
\pgfpathcurveto{\pgfqpoint{1.046895in}{0.641606in}}{\pgfqpoint{1.042505in}{0.652205in}}{\pgfqpoint{1.034691in}{0.660018in}}%
\pgfpathcurveto{\pgfqpoint{1.026878in}{0.667832in}}{\pgfqpoint{1.016279in}{0.672222in}}{\pgfqpoint{1.005229in}{0.672222in}}%
\pgfpathcurveto{\pgfqpoint{0.994179in}{0.672222in}}{\pgfqpoint{0.983580in}{0.667832in}}{\pgfqpoint{0.975766in}{0.660018in}}%
\pgfpathcurveto{\pgfqpoint{0.967952in}{0.652205in}}{\pgfqpoint{0.963562in}{0.641606in}}{\pgfqpoint{0.963562in}{0.630555in}}%
\pgfpathcurveto{\pgfqpoint{0.963562in}{0.619505in}}{\pgfqpoint{0.967952in}{0.608906in}}{\pgfqpoint{0.975766in}{0.601093in}}%
\pgfpathcurveto{\pgfqpoint{0.983580in}{0.593279in}}{\pgfqpoint{0.994179in}{0.588889in}}{\pgfqpoint{1.005229in}{0.588889in}}%
\pgfpathclose%
\pgfusepath{stroke,fill}%
\end{pgfscope}%
\begin{pgfscope}%
\pgfpathrectangle{\pgfqpoint{0.772069in}{0.515123in}}{\pgfqpoint{1.937500in}{1.347500in}}%
\pgfusepath{clip}%
\pgfsetbuttcap%
\pgfsetroundjoin%
\definecolor{currentfill}{rgb}{1.000000,0.980392,0.803922}%
\pgfsetfillcolor{currentfill}%
\pgfsetlinewidth{1.003750pt}%
\definecolor{currentstroke}{rgb}{1.000000,0.980392,0.803922}%
\pgfsetstrokecolor{currentstroke}%
\pgfsetdash{}{0pt}%
\pgfpathmoveto{\pgfqpoint{1.074284in}{0.601225in}}%
\pgfpathcurveto{\pgfqpoint{1.085334in}{0.601225in}}{\pgfqpoint{1.095933in}{0.605615in}}{\pgfqpoint{1.103747in}{0.613428in}}%
\pgfpathcurveto{\pgfqpoint{1.111560in}{0.621242in}}{\pgfqpoint{1.115950in}{0.631841in}}{\pgfqpoint{1.115950in}{0.642891in}}%
\pgfpathcurveto{\pgfqpoint{1.115950in}{0.653941in}}{\pgfqpoint{1.111560in}{0.664540in}}{\pgfqpoint{1.103747in}{0.672354in}}%
\pgfpathcurveto{\pgfqpoint{1.095933in}{0.680168in}}{\pgfqpoint{1.085334in}{0.684558in}}{\pgfqpoint{1.074284in}{0.684558in}}%
\pgfpathcurveto{\pgfqpoint{1.063234in}{0.684558in}}{\pgfqpoint{1.052635in}{0.680168in}}{\pgfqpoint{1.044821in}{0.672354in}}%
\pgfpathcurveto{\pgfqpoint{1.037007in}{0.664540in}}{\pgfqpoint{1.032617in}{0.653941in}}{\pgfqpoint{1.032617in}{0.642891in}}%
\pgfpathcurveto{\pgfqpoint{1.032617in}{0.631841in}}{\pgfqpoint{1.037007in}{0.621242in}}{\pgfqpoint{1.044821in}{0.613428in}}%
\pgfpathcurveto{\pgfqpoint{1.052635in}{0.605615in}}{\pgfqpoint{1.063234in}{0.601225in}}{\pgfqpoint{1.074284in}{0.601225in}}%
\pgfpathclose%
\pgfusepath{stroke,fill}%
\end{pgfscope}%
\begin{pgfscope}%
\pgfpathrectangle{\pgfqpoint{0.772069in}{0.515123in}}{\pgfqpoint{1.937500in}{1.347500in}}%
\pgfusepath{clip}%
\pgfsetbuttcap%
\pgfsetroundjoin%
\definecolor{currentfill}{rgb}{1.000000,0.980392,0.803922}%
\pgfsetfillcolor{currentfill}%
\pgfsetlinewidth{1.003750pt}%
\definecolor{currentstroke}{rgb}{1.000000,0.980392,0.803922}%
\pgfsetstrokecolor{currentstroke}%
\pgfsetdash{}{0pt}%
\pgfpathmoveto{\pgfqpoint{1.143339in}{0.613560in}}%
\pgfpathcurveto{\pgfqpoint{1.154389in}{0.613560in}}{\pgfqpoint{1.164988in}{0.617951in}}{\pgfqpoint{1.172802in}{0.625764in}}%
\pgfpathcurveto{\pgfqpoint{1.180615in}{0.633578in}}{\pgfqpoint{1.185006in}{0.644177in}}{\pgfqpoint{1.185006in}{0.655227in}}%
\pgfpathcurveto{\pgfqpoint{1.185006in}{0.666277in}}{\pgfqpoint{1.180615in}{0.676876in}}{\pgfqpoint{1.172802in}{0.684690in}}%
\pgfpathcurveto{\pgfqpoint{1.164988in}{0.692504in}}{\pgfqpoint{1.154389in}{0.696894in}}{\pgfqpoint{1.143339in}{0.696894in}}%
\pgfpathcurveto{\pgfqpoint{1.132289in}{0.696894in}}{\pgfqpoint{1.121690in}{0.692504in}}{\pgfqpoint{1.113876in}{0.684690in}}%
\pgfpathcurveto{\pgfqpoint{1.106062in}{0.676876in}}{\pgfqpoint{1.101672in}{0.666277in}}{\pgfqpoint{1.101672in}{0.655227in}}%
\pgfpathcurveto{\pgfqpoint{1.101672in}{0.644177in}}{\pgfqpoint{1.106062in}{0.633578in}}{\pgfqpoint{1.113876in}{0.625764in}}%
\pgfpathcurveto{\pgfqpoint{1.121690in}{0.617951in}}{\pgfqpoint{1.132289in}{0.613560in}}{\pgfqpoint{1.143339in}{0.613560in}}%
\pgfpathclose%
\pgfusepath{stroke,fill}%
\end{pgfscope}%
\begin{pgfscope}%
\pgfpathrectangle{\pgfqpoint{0.772069in}{0.515123in}}{\pgfqpoint{1.937500in}{1.347500in}}%
\pgfusepath{clip}%
\pgfsetbuttcap%
\pgfsetroundjoin%
\definecolor{currentfill}{rgb}{1.000000,0.980392,0.803922}%
\pgfsetfillcolor{currentfill}%
\pgfsetlinewidth{1.003750pt}%
\definecolor{currentstroke}{rgb}{1.000000,0.980392,0.803922}%
\pgfsetstrokecolor{currentstroke}%
\pgfsetdash{}{0pt}%
\pgfpathmoveto{\pgfqpoint{1.189376in}{0.625314in}}%
\pgfpathcurveto{\pgfqpoint{1.200426in}{0.625314in}}{\pgfqpoint{1.211025in}{0.629705in}}{\pgfqpoint{1.218838in}{0.637518in}}%
\pgfpathcurveto{\pgfqpoint{1.226652in}{0.645332in}}{\pgfqpoint{1.231042in}{0.655931in}}{\pgfqpoint{1.231042in}{0.666981in}}%
\pgfpathcurveto{\pgfqpoint{1.231042in}{0.678031in}}{\pgfqpoint{1.226652in}{0.688630in}}{\pgfqpoint{1.218838in}{0.696444in}}%
\pgfpathcurveto{\pgfqpoint{1.211025in}{0.704258in}}{\pgfqpoint{1.200426in}{0.708648in}}{\pgfqpoint{1.189376in}{0.708648in}}%
\pgfpathcurveto{\pgfqpoint{1.178325in}{0.708648in}}{\pgfqpoint{1.167726in}{0.704258in}}{\pgfqpoint{1.159913in}{0.696444in}}%
\pgfpathcurveto{\pgfqpoint{1.152099in}{0.688630in}}{\pgfqpoint{1.147709in}{0.678031in}}{\pgfqpoint{1.147709in}{0.666981in}}%
\pgfpathcurveto{\pgfqpoint{1.147709in}{0.655931in}}{\pgfqpoint{1.152099in}{0.645332in}}{\pgfqpoint{1.159913in}{0.637518in}}%
\pgfpathcurveto{\pgfqpoint{1.167726in}{0.629705in}}{\pgfqpoint{1.178325in}{0.625314in}}{\pgfqpoint{1.189376in}{0.625314in}}%
\pgfpathclose%
\pgfusepath{stroke,fill}%
\end{pgfscope}%
\begin{pgfscope}%
\pgfpathrectangle{\pgfqpoint{0.772069in}{0.515123in}}{\pgfqpoint{1.937500in}{1.347500in}}%
\pgfusepath{clip}%
\pgfsetbuttcap%
\pgfsetroundjoin%
\definecolor{currentfill}{rgb}{1.000000,0.980392,0.803922}%
\pgfsetfillcolor{currentfill}%
\pgfsetlinewidth{1.003750pt}%
\definecolor{currentstroke}{rgb}{1.000000,0.980392,0.803922}%
\pgfsetstrokecolor{currentstroke}%
\pgfsetdash{}{0pt}%
\pgfpathmoveto{\pgfqpoint{1.258431in}{0.644284in}}%
\pgfpathcurveto{\pgfqpoint{1.269481in}{0.644284in}}{\pgfqpoint{1.280080in}{0.648674in}}{\pgfqpoint{1.287893in}{0.656488in}}%
\pgfpathcurveto{\pgfqpoint{1.295707in}{0.664301in}}{\pgfqpoint{1.300097in}{0.674900in}}{\pgfqpoint{1.300097in}{0.685950in}}%
\pgfpathcurveto{\pgfqpoint{1.300097in}{0.697001in}}{\pgfqpoint{1.295707in}{0.707600in}}{\pgfqpoint{1.287893in}{0.715413in}}%
\pgfpathcurveto{\pgfqpoint{1.280080in}{0.723227in}}{\pgfqpoint{1.269481in}{0.727617in}}{\pgfqpoint{1.258431in}{0.727617in}}%
\pgfpathcurveto{\pgfqpoint{1.247381in}{0.727617in}}{\pgfqpoint{1.236781in}{0.723227in}}{\pgfqpoint{1.228968in}{0.715413in}}%
\pgfpathcurveto{\pgfqpoint{1.221154in}{0.707600in}}{\pgfqpoint{1.216764in}{0.697001in}}{\pgfqpoint{1.216764in}{0.685950in}}%
\pgfpathcurveto{\pgfqpoint{1.216764in}{0.674900in}}{\pgfqpoint{1.221154in}{0.664301in}}{\pgfqpoint{1.228968in}{0.656488in}}%
\pgfpathcurveto{\pgfqpoint{1.236781in}{0.648674in}}{\pgfqpoint{1.247381in}{0.644284in}}{\pgfqpoint{1.258431in}{0.644284in}}%
\pgfpathclose%
\pgfusepath{stroke,fill}%
\end{pgfscope}%
\begin{pgfscope}%
\pgfpathrectangle{\pgfqpoint{0.772069in}{0.515123in}}{\pgfqpoint{1.937500in}{1.347500in}}%
\pgfusepath{clip}%
\pgfsetbuttcap%
\pgfsetroundjoin%
\definecolor{currentfill}{rgb}{1.000000,0.980392,0.803922}%
\pgfsetfillcolor{currentfill}%
\pgfsetlinewidth{1.003750pt}%
\definecolor{currentstroke}{rgb}{1.000000,0.980392,0.803922}%
\pgfsetstrokecolor{currentstroke}%
\pgfsetdash{}{0pt}%
\pgfpathmoveto{\pgfqpoint{1.327486in}{0.650335in}}%
\pgfpathcurveto{\pgfqpoint{1.338536in}{0.650335in}}{\pgfqpoint{1.349135in}{0.654726in}}{\pgfqpoint{1.356949in}{0.662539in}}%
\pgfpathcurveto{\pgfqpoint{1.364762in}{0.670353in}}{\pgfqpoint{1.369152in}{0.680952in}}{\pgfqpoint{1.369152in}{0.692002in}}%
\pgfpathcurveto{\pgfqpoint{1.369152in}{0.703052in}}{\pgfqpoint{1.364762in}{0.713651in}}{\pgfqpoint{1.356949in}{0.721465in}}%
\pgfpathcurveto{\pgfqpoint{1.349135in}{0.729278in}}{\pgfqpoint{1.338536in}{0.733669in}}{\pgfqpoint{1.327486in}{0.733669in}}%
\pgfpathcurveto{\pgfqpoint{1.316436in}{0.733669in}}{\pgfqpoint{1.305837in}{0.729278in}}{\pgfqpoint{1.298023in}{0.721465in}}%
\pgfpathcurveto{\pgfqpoint{1.290209in}{0.713651in}}{\pgfqpoint{1.285819in}{0.703052in}}{\pgfqpoint{1.285819in}{0.692002in}}%
\pgfpathcurveto{\pgfqpoint{1.285819in}{0.680952in}}{\pgfqpoint{1.290209in}{0.670353in}}{\pgfqpoint{1.298023in}{0.662539in}}%
\pgfpathcurveto{\pgfqpoint{1.305837in}{0.654726in}}{\pgfqpoint{1.316436in}{0.650335in}}{\pgfqpoint{1.327486in}{0.650335in}}%
\pgfpathclose%
\pgfusepath{stroke,fill}%
\end{pgfscope}%
\begin{pgfscope}%
\pgfpathrectangle{\pgfqpoint{0.772069in}{0.515123in}}{\pgfqpoint{1.937500in}{1.347500in}}%
\pgfusepath{clip}%
\pgfsetbuttcap%
\pgfsetroundjoin%
\definecolor{currentfill}{rgb}{1.000000,0.980392,0.803922}%
\pgfsetfillcolor{currentfill}%
\pgfsetlinewidth{1.003750pt}%
\definecolor{currentstroke}{rgb}{1.000000,0.980392,0.803922}%
\pgfsetstrokecolor{currentstroke}%
\pgfsetdash{}{0pt}%
\pgfpathmoveto{\pgfqpoint{1.396541in}{0.662322in}}%
\pgfpathcurveto{\pgfqpoint{1.407591in}{0.662322in}}{\pgfqpoint{1.418190in}{0.666712in}}{\pgfqpoint{1.426004in}{0.674526in}}%
\pgfpathcurveto{\pgfqpoint{1.433817in}{0.682340in}}{\pgfqpoint{1.438208in}{0.692939in}}{\pgfqpoint{1.438208in}{0.703989in}}%
\pgfpathcurveto{\pgfqpoint{1.438208in}{0.715039in}}{\pgfqpoint{1.433817in}{0.725638in}}{\pgfqpoint{1.426004in}{0.733452in}}%
\pgfpathcurveto{\pgfqpoint{1.418190in}{0.741265in}}{\pgfqpoint{1.407591in}{0.745655in}}{\pgfqpoint{1.396541in}{0.745655in}}%
\pgfpathcurveto{\pgfqpoint{1.385491in}{0.745655in}}{\pgfqpoint{1.374892in}{0.741265in}}{\pgfqpoint{1.367078in}{0.733452in}}%
\pgfpathcurveto{\pgfqpoint{1.359264in}{0.725638in}}{\pgfqpoint{1.354874in}{0.715039in}}{\pgfqpoint{1.354874in}{0.703989in}}%
\pgfpathcurveto{\pgfqpoint{1.354874in}{0.692939in}}{\pgfqpoint{1.359264in}{0.682340in}}{\pgfqpoint{1.367078in}{0.674526in}}%
\pgfpathcurveto{\pgfqpoint{1.374892in}{0.666712in}}{\pgfqpoint{1.385491in}{0.662322in}}{\pgfqpoint{1.396541in}{0.662322in}}%
\pgfpathclose%
\pgfusepath{stroke,fill}%
\end{pgfscope}%
\begin{pgfscope}%
\pgfpathrectangle{\pgfqpoint{0.772069in}{0.515123in}}{\pgfqpoint{1.937500in}{1.347500in}}%
\pgfusepath{clip}%
\pgfsetbuttcap%
\pgfsetroundjoin%
\definecolor{currentfill}{rgb}{1.000000,0.980392,0.803922}%
\pgfsetfillcolor{currentfill}%
\pgfsetlinewidth{1.003750pt}%
\definecolor{currentstroke}{rgb}{1.000000,0.980392,0.803922}%
\pgfsetstrokecolor{currentstroke}%
\pgfsetdash{}{0pt}%
\pgfpathmoveto{\pgfqpoint{1.442578in}{0.674658in}}%
\pgfpathcurveto{\pgfqpoint{1.453628in}{0.674658in}}{\pgfqpoint{1.464227in}{0.679048in}}{\pgfqpoint{1.472040in}{0.686862in}}%
\pgfpathcurveto{\pgfqpoint{1.479854in}{0.694675in}}{\pgfqpoint{1.484244in}{0.705274in}}{\pgfqpoint{1.484244in}{0.716325in}}%
\pgfpathcurveto{\pgfqpoint{1.484244in}{0.727375in}}{\pgfqpoint{1.479854in}{0.737974in}}{\pgfqpoint{1.472040in}{0.745787in}}%
\pgfpathcurveto{\pgfqpoint{1.464227in}{0.753601in}}{\pgfqpoint{1.453628in}{0.757991in}}{\pgfqpoint{1.442578in}{0.757991in}}%
\pgfpathcurveto{\pgfqpoint{1.431527in}{0.757991in}}{\pgfqpoint{1.420928in}{0.753601in}}{\pgfqpoint{1.413115in}{0.745787in}}%
\pgfpathcurveto{\pgfqpoint{1.405301in}{0.737974in}}{\pgfqpoint{1.400911in}{0.727375in}}{\pgfqpoint{1.400911in}{0.716325in}}%
\pgfpathcurveto{\pgfqpoint{1.400911in}{0.705274in}}{\pgfqpoint{1.405301in}{0.694675in}}{\pgfqpoint{1.413115in}{0.686862in}}%
\pgfpathcurveto{\pgfqpoint{1.420928in}{0.679048in}}{\pgfqpoint{1.431527in}{0.674658in}}{\pgfqpoint{1.442578in}{0.674658in}}%
\pgfpathclose%
\pgfusepath{stroke,fill}%
\end{pgfscope}%
\begin{pgfscope}%
\pgfpathrectangle{\pgfqpoint{0.772069in}{0.515123in}}{\pgfqpoint{1.937500in}{1.347500in}}%
\pgfusepath{clip}%
\pgfsetbuttcap%
\pgfsetroundjoin%
\definecolor{currentfill}{rgb}{0.596078,0.984314,0.596078}%
\pgfsetfillcolor{currentfill}%
\pgfsetlinewidth{1.003750pt}%
\definecolor{currentstroke}{rgb}{0.596078,0.984314,0.596078}%
\pgfsetstrokecolor{currentstroke}%
\pgfsetdash{}{0pt}%
\pgfpathmoveto{\pgfqpoint{0.890137in}{0.641491in}}%
\pgfpathcurveto{\pgfqpoint{0.901187in}{0.641491in}}{\pgfqpoint{0.911786in}{0.645881in}}{\pgfqpoint{0.919600in}{0.653695in}}%
\pgfpathcurveto{\pgfqpoint{0.927413in}{0.661508in}}{\pgfqpoint{0.931804in}{0.672107in}}{\pgfqpoint{0.931804in}{0.683157in}}%
\pgfpathcurveto{\pgfqpoint{0.931804in}{0.694208in}}{\pgfqpoint{0.927413in}{0.704807in}}{\pgfqpoint{0.919600in}{0.712620in}}%
\pgfpathcurveto{\pgfqpoint{0.911786in}{0.720434in}}{\pgfqpoint{0.901187in}{0.724824in}}{\pgfqpoint{0.890137in}{0.724824in}}%
\pgfpathcurveto{\pgfqpoint{0.879087in}{0.724824in}}{\pgfqpoint{0.868488in}{0.720434in}}{\pgfqpoint{0.860674in}{0.712620in}}%
\pgfpathcurveto{\pgfqpoint{0.852860in}{0.704807in}}{\pgfqpoint{0.848470in}{0.694208in}}{\pgfqpoint{0.848470in}{0.683157in}}%
\pgfpathcurveto{\pgfqpoint{0.848470in}{0.672107in}}{\pgfqpoint{0.852860in}{0.661508in}}{\pgfqpoint{0.860674in}{0.653695in}}%
\pgfpathcurveto{\pgfqpoint{0.868488in}{0.645881in}}{\pgfqpoint{0.879087in}{0.641491in}}{\pgfqpoint{0.890137in}{0.641491in}}%
\pgfpathclose%
\pgfusepath{stroke,fill}%
\end{pgfscope}%
\begin{pgfscope}%
\pgfpathrectangle{\pgfqpoint{0.772069in}{0.515123in}}{\pgfqpoint{1.937500in}{1.347500in}}%
\pgfusepath{clip}%
\pgfsetbuttcap%
\pgfsetroundjoin%
\definecolor{currentfill}{rgb}{0.596078,0.984314,0.596078}%
\pgfsetfillcolor{currentfill}%
\pgfsetlinewidth{1.003750pt}%
\definecolor{currentstroke}{rgb}{0.596078,0.984314,0.596078}%
\pgfsetstrokecolor{currentstroke}%
\pgfsetdash{}{0pt}%
\pgfpathmoveto{\pgfqpoint{0.959192in}{0.725631in}}%
\pgfpathcurveto{\pgfqpoint{0.970242in}{0.725631in}}{\pgfqpoint{0.980841in}{0.730021in}}{\pgfqpoint{0.988655in}{0.737835in}}%
\pgfpathcurveto{\pgfqpoint{0.996468in}{0.745648in}}{\pgfqpoint{1.000859in}{0.756247in}}{\pgfqpoint{1.000859in}{0.767297in}}%
\pgfpathcurveto{\pgfqpoint{1.000859in}{0.778348in}}{\pgfqpoint{0.996468in}{0.788947in}}{\pgfqpoint{0.988655in}{0.796760in}}%
\pgfpathcurveto{\pgfqpoint{0.980841in}{0.804574in}}{\pgfqpoint{0.970242in}{0.808964in}}{\pgfqpoint{0.959192in}{0.808964in}}%
\pgfpathcurveto{\pgfqpoint{0.948142in}{0.808964in}}{\pgfqpoint{0.937543in}{0.804574in}}{\pgfqpoint{0.929729in}{0.796760in}}%
\pgfpathcurveto{\pgfqpoint{0.921916in}{0.788947in}}{\pgfqpoint{0.917525in}{0.778348in}}{\pgfqpoint{0.917525in}{0.767297in}}%
\pgfpathcurveto{\pgfqpoint{0.917525in}{0.756247in}}{\pgfqpoint{0.921916in}{0.745648in}}{\pgfqpoint{0.929729in}{0.737835in}}%
\pgfpathcurveto{\pgfqpoint{0.937543in}{0.730021in}}{\pgfqpoint{0.948142in}{0.725631in}}{\pgfqpoint{0.959192in}{0.725631in}}%
\pgfpathclose%
\pgfusepath{stroke,fill}%
\end{pgfscope}%
\begin{pgfscope}%
\pgfpathrectangle{\pgfqpoint{0.772069in}{0.515123in}}{\pgfqpoint{1.937500in}{1.347500in}}%
\pgfusepath{clip}%
\pgfsetbuttcap%
\pgfsetroundjoin%
\definecolor{currentfill}{rgb}{0.596078,0.984314,0.596078}%
\pgfsetfillcolor{currentfill}%
\pgfsetlinewidth{1.003750pt}%
\definecolor{currentstroke}{rgb}{0.596078,0.984314,0.596078}%
\pgfsetstrokecolor{currentstroke}%
\pgfsetdash{}{0pt}%
\pgfpathmoveto{\pgfqpoint{1.005229in}{0.813844in}}%
\pgfpathcurveto{\pgfqpoint{1.016279in}{0.813844in}}{\pgfqpoint{1.026878in}{0.818234in}}{\pgfqpoint{1.034691in}{0.826048in}}%
\pgfpathcurveto{\pgfqpoint{1.042505in}{0.833861in}}{\pgfqpoint{1.046895in}{0.844460in}}{\pgfqpoint{1.046895in}{0.855510in}}%
\pgfpathcurveto{\pgfqpoint{1.046895in}{0.866561in}}{\pgfqpoint{1.042505in}{0.877160in}}{\pgfqpoint{1.034691in}{0.884973in}}%
\pgfpathcurveto{\pgfqpoint{1.026878in}{0.892787in}}{\pgfqpoint{1.016279in}{0.897177in}}{\pgfqpoint{1.005229in}{0.897177in}}%
\pgfpathcurveto{\pgfqpoint{0.994179in}{0.897177in}}{\pgfqpoint{0.983580in}{0.892787in}}{\pgfqpoint{0.975766in}{0.884973in}}%
\pgfpathcurveto{\pgfqpoint{0.967952in}{0.877160in}}{\pgfqpoint{0.963562in}{0.866561in}}{\pgfqpoint{0.963562in}{0.855510in}}%
\pgfpathcurveto{\pgfqpoint{0.963562in}{0.844460in}}{\pgfqpoint{0.967952in}{0.833861in}}{\pgfqpoint{0.975766in}{0.826048in}}%
\pgfpathcurveto{\pgfqpoint{0.983580in}{0.818234in}}{\pgfqpoint{0.994179in}{0.813844in}}{\pgfqpoint{1.005229in}{0.813844in}}%
\pgfpathclose%
\pgfusepath{stroke,fill}%
\end{pgfscope}%
\begin{pgfscope}%
\pgfpathrectangle{\pgfqpoint{0.772069in}{0.515123in}}{\pgfqpoint{1.937500in}{1.347500in}}%
\pgfusepath{clip}%
\pgfsetbuttcap%
\pgfsetroundjoin%
\definecolor{currentfill}{rgb}{0.596078,0.984314,0.596078}%
\pgfsetfillcolor{currentfill}%
\pgfsetlinewidth{1.003750pt}%
\definecolor{currentstroke}{rgb}{0.596078,0.984314,0.596078}%
\pgfsetstrokecolor{currentstroke}%
\pgfsetdash{}{0pt}%
\pgfpathmoveto{\pgfqpoint{1.074284in}{0.903453in}}%
\pgfpathcurveto{\pgfqpoint{1.085334in}{0.903453in}}{\pgfqpoint{1.095933in}{0.907844in}}{\pgfqpoint{1.103747in}{0.915657in}}%
\pgfpathcurveto{\pgfqpoint{1.111560in}{0.923471in}}{\pgfqpoint{1.115950in}{0.934070in}}{\pgfqpoint{1.115950in}{0.945120in}}%
\pgfpathcurveto{\pgfqpoint{1.115950in}{0.956170in}}{\pgfqpoint{1.111560in}{0.966769in}}{\pgfqpoint{1.103747in}{0.974583in}}%
\pgfpathcurveto{\pgfqpoint{1.095933in}{0.982397in}}{\pgfqpoint{1.085334in}{0.986787in}}{\pgfqpoint{1.074284in}{0.986787in}}%
\pgfpathcurveto{\pgfqpoint{1.063234in}{0.986787in}}{\pgfqpoint{1.052635in}{0.982397in}}{\pgfqpoint{1.044821in}{0.974583in}}%
\pgfpathcurveto{\pgfqpoint{1.037007in}{0.966769in}}{\pgfqpoint{1.032617in}{0.956170in}}{\pgfqpoint{1.032617in}{0.945120in}}%
\pgfpathcurveto{\pgfqpoint{1.032617in}{0.934070in}}{\pgfqpoint{1.037007in}{0.923471in}}{\pgfqpoint{1.044821in}{0.915657in}}%
\pgfpathcurveto{\pgfqpoint{1.052635in}{0.907844in}}{\pgfqpoint{1.063234in}{0.903453in}}{\pgfqpoint{1.074284in}{0.903453in}}%
\pgfpathclose%
\pgfusepath{stroke,fill}%
\end{pgfscope}%
\begin{pgfscope}%
\pgfpathrectangle{\pgfqpoint{0.772069in}{0.515123in}}{\pgfqpoint{1.937500in}{1.347500in}}%
\pgfusepath{clip}%
\pgfsetbuttcap%
\pgfsetroundjoin%
\definecolor{currentfill}{rgb}{0.596078,0.984314,0.596078}%
\pgfsetfillcolor{currentfill}%
\pgfsetlinewidth{1.003750pt}%
\definecolor{currentstroke}{rgb}{0.596078,0.984314,0.596078}%
\pgfsetstrokecolor{currentstroke}%
\pgfsetdash{}{0pt}%
\pgfpathmoveto{\pgfqpoint{1.143339in}{0.991899in}}%
\pgfpathcurveto{\pgfqpoint{1.154389in}{0.991899in}}{\pgfqpoint{1.164988in}{0.996290in}}{\pgfqpoint{1.172802in}{1.004103in}}%
\pgfpathcurveto{\pgfqpoint{1.180615in}{1.011917in}}{\pgfqpoint{1.185006in}{1.022516in}}{\pgfqpoint{1.185006in}{1.033566in}}%
\pgfpathcurveto{\pgfqpoint{1.185006in}{1.044616in}}{\pgfqpoint{1.180615in}{1.055215in}}{\pgfqpoint{1.172802in}{1.063029in}}%
\pgfpathcurveto{\pgfqpoint{1.164988in}{1.070842in}}{\pgfqpoint{1.154389in}{1.075233in}}{\pgfqpoint{1.143339in}{1.075233in}}%
\pgfpathcurveto{\pgfqpoint{1.132289in}{1.075233in}}{\pgfqpoint{1.121690in}{1.070842in}}{\pgfqpoint{1.113876in}{1.063029in}}%
\pgfpathcurveto{\pgfqpoint{1.106062in}{1.055215in}}{\pgfqpoint{1.101672in}{1.044616in}}{\pgfqpoint{1.101672in}{1.033566in}}%
\pgfpathcurveto{\pgfqpoint{1.101672in}{1.022516in}}{\pgfqpoint{1.106062in}{1.011917in}}{\pgfqpoint{1.113876in}{1.004103in}}%
\pgfpathcurveto{\pgfqpoint{1.121690in}{0.996290in}}{\pgfqpoint{1.132289in}{0.991899in}}{\pgfqpoint{1.143339in}{0.991899in}}%
\pgfpathclose%
\pgfusepath{stroke,fill}%
\end{pgfscope}%
\begin{pgfscope}%
\pgfpathrectangle{\pgfqpoint{0.772069in}{0.515123in}}{\pgfqpoint{1.937500in}{1.347500in}}%
\pgfusepath{clip}%
\pgfsetbuttcap%
\pgfsetroundjoin%
\definecolor{currentfill}{rgb}{0.596078,0.984314,0.596078}%
\pgfsetfillcolor{currentfill}%
\pgfsetlinewidth{1.003750pt}%
\definecolor{currentstroke}{rgb}{0.596078,0.984314,0.596078}%
\pgfsetstrokecolor{currentstroke}%
\pgfsetdash{}{0pt}%
\pgfpathmoveto{\pgfqpoint{1.189376in}{1.075690in}}%
\pgfpathcurveto{\pgfqpoint{1.200426in}{1.075690in}}{\pgfqpoint{1.211025in}{1.080080in}}{\pgfqpoint{1.218838in}{1.087894in}}%
\pgfpathcurveto{\pgfqpoint{1.226652in}{1.095708in}}{\pgfqpoint{1.231042in}{1.106307in}}{\pgfqpoint{1.231042in}{1.117357in}}%
\pgfpathcurveto{\pgfqpoint{1.231042in}{1.128407in}}{\pgfqpoint{1.226652in}{1.139006in}}{\pgfqpoint{1.218838in}{1.146820in}}%
\pgfpathcurveto{\pgfqpoint{1.211025in}{1.154633in}}{\pgfqpoint{1.200426in}{1.159023in}}{\pgfqpoint{1.189376in}{1.159023in}}%
\pgfpathcurveto{\pgfqpoint{1.178325in}{1.159023in}}{\pgfqpoint{1.167726in}{1.154633in}}{\pgfqpoint{1.159913in}{1.146820in}}%
\pgfpathcurveto{\pgfqpoint{1.152099in}{1.139006in}}{\pgfqpoint{1.147709in}{1.128407in}}{\pgfqpoint{1.147709in}{1.117357in}}%
\pgfpathcurveto{\pgfqpoint{1.147709in}{1.106307in}}{\pgfqpoint{1.152099in}{1.095708in}}{\pgfqpoint{1.159913in}{1.087894in}}%
\pgfpathcurveto{\pgfqpoint{1.167726in}{1.080080in}}{\pgfqpoint{1.178325in}{1.075690in}}{\pgfqpoint{1.189376in}{1.075690in}}%
\pgfpathclose%
\pgfusepath{stroke,fill}%
\end{pgfscope}%
\begin{pgfscope}%
\pgfpathrectangle{\pgfqpoint{0.772069in}{0.515123in}}{\pgfqpoint{1.937500in}{1.347500in}}%
\pgfusepath{clip}%
\pgfsetbuttcap%
\pgfsetroundjoin%
\definecolor{currentfill}{rgb}{0.596078,0.984314,0.596078}%
\pgfsetfillcolor{currentfill}%
\pgfsetlinewidth{1.003750pt}%
\definecolor{currentstroke}{rgb}{0.596078,0.984314,0.596078}%
\pgfsetstrokecolor{currentstroke}%
\pgfsetdash{}{0pt}%
\pgfpathmoveto{\pgfqpoint{1.258431in}{1.166464in}}%
\pgfpathcurveto{\pgfqpoint{1.269481in}{1.166464in}}{\pgfqpoint{1.280080in}{1.170854in}}{\pgfqpoint{1.287893in}{1.178667in}}%
\pgfpathcurveto{\pgfqpoint{1.295707in}{1.186481in}}{\pgfqpoint{1.300097in}{1.197080in}}{\pgfqpoint{1.300097in}{1.208130in}}%
\pgfpathcurveto{\pgfqpoint{1.300097in}{1.219180in}}{\pgfqpoint{1.295707in}{1.229779in}}{\pgfqpoint{1.287893in}{1.237593in}}%
\pgfpathcurveto{\pgfqpoint{1.280080in}{1.245407in}}{\pgfqpoint{1.269481in}{1.249797in}}{\pgfqpoint{1.258431in}{1.249797in}}%
\pgfpathcurveto{\pgfqpoint{1.247381in}{1.249797in}}{\pgfqpoint{1.236781in}{1.245407in}}{\pgfqpoint{1.228968in}{1.237593in}}%
\pgfpathcurveto{\pgfqpoint{1.221154in}{1.229779in}}{\pgfqpoint{1.216764in}{1.219180in}}{\pgfqpoint{1.216764in}{1.208130in}}%
\pgfpathcurveto{\pgfqpoint{1.216764in}{1.197080in}}{\pgfqpoint{1.221154in}{1.186481in}}{\pgfqpoint{1.228968in}{1.178667in}}%
\pgfpathcurveto{\pgfqpoint{1.236781in}{1.170854in}}{\pgfqpoint{1.247381in}{1.166464in}}{\pgfqpoint{1.258431in}{1.166464in}}%
\pgfpathclose%
\pgfusepath{stroke,fill}%
\end{pgfscope}%
\begin{pgfscope}%
\pgfpathrectangle{\pgfqpoint{0.772069in}{0.515123in}}{\pgfqpoint{1.937500in}{1.347500in}}%
\pgfusepath{clip}%
\pgfsetbuttcap%
\pgfsetroundjoin%
\definecolor{currentfill}{rgb}{0.596078,0.984314,0.596078}%
\pgfsetfillcolor{currentfill}%
\pgfsetlinewidth{1.003750pt}%
\definecolor{currentstroke}{rgb}{0.596078,0.984314,0.596078}%
\pgfsetstrokecolor{currentstroke}%
\pgfsetdash{}{0pt}%
\pgfpathmoveto{\pgfqpoint{1.327486in}{1.253746in}}%
\pgfpathcurveto{\pgfqpoint{1.338536in}{1.253746in}}{\pgfqpoint{1.349135in}{1.258136in}}{\pgfqpoint{1.356949in}{1.265950in}}%
\pgfpathcurveto{\pgfqpoint{1.364762in}{1.273763in}}{\pgfqpoint{1.369152in}{1.284362in}}{\pgfqpoint{1.369152in}{1.295412in}}%
\pgfpathcurveto{\pgfqpoint{1.369152in}{1.306462in}}{\pgfqpoint{1.364762in}{1.317061in}}{\pgfqpoint{1.356949in}{1.324875in}}%
\pgfpathcurveto{\pgfqpoint{1.349135in}{1.332689in}}{\pgfqpoint{1.338536in}{1.337079in}}{\pgfqpoint{1.327486in}{1.337079in}}%
\pgfpathcurveto{\pgfqpoint{1.316436in}{1.337079in}}{\pgfqpoint{1.305837in}{1.332689in}}{\pgfqpoint{1.298023in}{1.324875in}}%
\pgfpathcurveto{\pgfqpoint{1.290209in}{1.317061in}}{\pgfqpoint{1.285819in}{1.306462in}}{\pgfqpoint{1.285819in}{1.295412in}}%
\pgfpathcurveto{\pgfqpoint{1.285819in}{1.284362in}}{\pgfqpoint{1.290209in}{1.273763in}}{\pgfqpoint{1.298023in}{1.265950in}}%
\pgfpathcurveto{\pgfqpoint{1.305837in}{1.258136in}}{\pgfqpoint{1.316436in}{1.253746in}}{\pgfqpoint{1.327486in}{1.253746in}}%
\pgfpathclose%
\pgfusepath{stroke,fill}%
\end{pgfscope}%
\begin{pgfscope}%
\pgfpathrectangle{\pgfqpoint{0.772069in}{0.515123in}}{\pgfqpoint{1.937500in}{1.347500in}}%
\pgfusepath{clip}%
\pgfsetbuttcap%
\pgfsetroundjoin%
\definecolor{currentfill}{rgb}{0.596078,0.984314,0.596078}%
\pgfsetfillcolor{currentfill}%
\pgfsetlinewidth{1.003750pt}%
\definecolor{currentstroke}{rgb}{0.596078,0.984314,0.596078}%
\pgfsetstrokecolor{currentstroke}%
\pgfsetdash{}{0pt}%
\pgfpathmoveto{\pgfqpoint{1.396541in}{1.339864in}}%
\pgfpathcurveto{\pgfqpoint{1.407591in}{1.339864in}}{\pgfqpoint{1.418190in}{1.344254in}}{\pgfqpoint{1.426004in}{1.352068in}}%
\pgfpathcurveto{\pgfqpoint{1.433817in}{1.359881in}}{\pgfqpoint{1.438208in}{1.370481in}}{\pgfqpoint{1.438208in}{1.381531in}}%
\pgfpathcurveto{\pgfqpoint{1.438208in}{1.392581in}}{\pgfqpoint{1.433817in}{1.403180in}}{\pgfqpoint{1.426004in}{1.410993in}}%
\pgfpathcurveto{\pgfqpoint{1.418190in}{1.418807in}}{\pgfqpoint{1.407591in}{1.423197in}}{\pgfqpoint{1.396541in}{1.423197in}}%
\pgfpathcurveto{\pgfqpoint{1.385491in}{1.423197in}}{\pgfqpoint{1.374892in}{1.418807in}}{\pgfqpoint{1.367078in}{1.410993in}}%
\pgfpathcurveto{\pgfqpoint{1.359264in}{1.403180in}}{\pgfqpoint{1.354874in}{1.392581in}}{\pgfqpoint{1.354874in}{1.381531in}}%
\pgfpathcurveto{\pgfqpoint{1.354874in}{1.370481in}}{\pgfqpoint{1.359264in}{1.359881in}}{\pgfqpoint{1.367078in}{1.352068in}}%
\pgfpathcurveto{\pgfqpoint{1.374892in}{1.344254in}}{\pgfqpoint{1.385491in}{1.339864in}}{\pgfqpoint{1.396541in}{1.339864in}}%
\pgfpathclose%
\pgfusepath{stroke,fill}%
\end{pgfscope}%
\begin{pgfscope}%
\pgfpathrectangle{\pgfqpoint{0.772069in}{0.515123in}}{\pgfqpoint{1.937500in}{1.347500in}}%
\pgfusepath{clip}%
\pgfsetbuttcap%
\pgfsetroundjoin%
\definecolor{currentfill}{rgb}{0.596078,0.984314,0.596078}%
\pgfsetfillcolor{currentfill}%
\pgfsetlinewidth{1.003750pt}%
\definecolor{currentstroke}{rgb}{0.596078,0.984314,0.596078}%
\pgfsetstrokecolor{currentstroke}%
\pgfsetdash{}{0pt}%
\pgfpathmoveto{\pgfqpoint{1.442578in}{1.428310in}}%
\pgfpathcurveto{\pgfqpoint{1.453628in}{1.428310in}}{\pgfqpoint{1.464227in}{1.432700in}}{\pgfqpoint{1.472040in}{1.440514in}}%
\pgfpathcurveto{\pgfqpoint{1.479854in}{1.448327in}}{\pgfqpoint{1.484244in}{1.458926in}}{\pgfqpoint{1.484244in}{1.469977in}}%
\pgfpathcurveto{\pgfqpoint{1.484244in}{1.481027in}}{\pgfqpoint{1.479854in}{1.491626in}}{\pgfqpoint{1.472040in}{1.499439in}}%
\pgfpathcurveto{\pgfqpoint{1.464227in}{1.507253in}}{\pgfqpoint{1.453628in}{1.511643in}}{\pgfqpoint{1.442578in}{1.511643in}}%
\pgfpathcurveto{\pgfqpoint{1.431527in}{1.511643in}}{\pgfqpoint{1.420928in}{1.507253in}}{\pgfqpoint{1.413115in}{1.499439in}}%
\pgfpathcurveto{\pgfqpoint{1.405301in}{1.491626in}}{\pgfqpoint{1.400911in}{1.481027in}}{\pgfqpoint{1.400911in}{1.469977in}}%
\pgfpathcurveto{\pgfqpoint{1.400911in}{1.458926in}}{\pgfqpoint{1.405301in}{1.448327in}}{\pgfqpoint{1.413115in}{1.440514in}}%
\pgfpathcurveto{\pgfqpoint{1.420928in}{1.432700in}}{\pgfqpoint{1.431527in}{1.428310in}}{\pgfqpoint{1.442578in}{1.428310in}}%
\pgfpathclose%
\pgfusepath{stroke,fill}%
\end{pgfscope}%
\begin{pgfscope}%
\pgfsetrectcap%
\pgfsetmiterjoin%
\pgfsetlinewidth{0.803000pt}%
\definecolor{currentstroke}{rgb}{0.000000,0.000000,0.000000}%
\pgfsetstrokecolor{currentstroke}%
\pgfsetdash{}{0pt}%
\pgfpathmoveto{\pgfqpoint{0.772069in}{0.515123in}}%
\pgfpathlineto{\pgfqpoint{0.772069in}{1.862623in}}%
\pgfusepath{stroke}%
\end{pgfscope}%
\begin{pgfscope}%
\pgfsetrectcap%
\pgfsetmiterjoin%
\pgfsetlinewidth{0.803000pt}%
\definecolor{currentstroke}{rgb}{0.000000,0.000000,0.000000}%
\pgfsetstrokecolor{currentstroke}%
\pgfsetdash{}{0pt}%
\pgfpathmoveto{\pgfqpoint{2.709569in}{0.515123in}}%
\pgfpathlineto{\pgfqpoint{2.709569in}{1.862623in}}%
\pgfusepath{stroke}%
\end{pgfscope}%
\begin{pgfscope}%
\pgfsetrectcap%
\pgfsetmiterjoin%
\pgfsetlinewidth{0.803000pt}%
\definecolor{currentstroke}{rgb}{0.000000,0.000000,0.000000}%
\pgfsetstrokecolor{currentstroke}%
\pgfsetdash{}{0pt}%
\pgfpathmoveto{\pgfqpoint{0.772069in}{0.515123in}}%
\pgfpathlineto{\pgfqpoint{2.709569in}{0.515123in}}%
\pgfusepath{stroke}%
\end{pgfscope}%
\begin{pgfscope}%
\pgfsetrectcap%
\pgfsetmiterjoin%
\pgfsetlinewidth{0.803000pt}%
\definecolor{currentstroke}{rgb}{0.000000,0.000000,0.000000}%
\pgfsetstrokecolor{currentstroke}%
\pgfsetdash{}{0pt}%
\pgfpathmoveto{\pgfqpoint{0.772069in}{1.862623in}}%
\pgfpathlineto{\pgfqpoint{2.709569in}{1.862623in}}%
\pgfusepath{stroke}%
\end{pgfscope}%
\end{pgfpicture}%
\makeatother%
\endgroup%

    \caption{Voltajes (\textcolor{Blue}{$V$}, \textcolor{Red}{$V_1$}, \textcolor{Yellow}{$V_{23}$}, \textcolor{Green}{$V_4$}) frente a intensidades (\textcolor{Black}{$I$}, \textcolor{DarkGrey}{$I_1$}, \textcolor{Grey}{$I_2$})}
    \label{graf:cmixmulti}
  \end{figure}

  \subsection{Ajuste por mínimos cuadrados}
  \label{v:mixto}

  Procedemos a aplicar el método de regresión lineal con el ajuste por mínimos cuadrados (\ref{sec:reglin}). Ajustamos las rectas y las representamos gráficamente. Para la gráfica \ref{graf:cmixinttotal} (voltaje frente intensidad total) los parámetros son $a = 8,47 \cdot 10^-2$ y $b = 1,32 \cdot 10^6$. El resto se pueden calcular análogamente con las fórmulas \ref{ec:a} y \ref{ec:b}. Para este ajuste obtenemos un coeficiente de regresión lineal utilizando la fórmula \ref{ec:r} de $r = 0,99993$ que, pese a ser el más bajo de las prácticas realizadas sigue teniendo cuatro nueves, lo cual hace que sea satisfactorio.

  \begin{figure}[H]
    %\centering
    \hspace{2.5em} %% Creator: Matplotlib, PGF backend
%%
%% To include the figure in your LaTeX document, write
%%   \input{<filename>.pgf}
%%
%% Make sure the required packages are loaded in your preamble
%%   \usepackage{pgf}
%%
%% Figures using additional raster images can only be included by \input if
%% they are in the same directory as the main LaTeX file. For loading figures
%% from other directories you can use the `import` package
%%   \usepackage{import}
%% and then include the figures with
%%   \import{<path to file>}{<filename>.pgf}
%%
%% Matplotlib used the following preamble
%%
\begingroup%
\makeatletter%
\begin{pgfpicture}%
\pgfpathrectangle{\pgfpointorigin}{\pgfqpoint{4.776953in}{3.310123in}}%
\pgfusepath{use as bounding box, clip}%
\begin{pgfscope}%
\pgfsetbuttcap%
\pgfsetmiterjoin%
\definecolor{currentfill}{rgb}{1.000000,1.000000,1.000000}%
\pgfsetfillcolor{currentfill}%
\pgfsetlinewidth{0.000000pt}%
\definecolor{currentstroke}{rgb}{1.000000,1.000000,1.000000}%
\pgfsetstrokecolor{currentstroke}%
\pgfsetdash{}{0pt}%
\pgfpathmoveto{\pgfqpoint{0.000000in}{0.000000in}}%
\pgfpathlineto{\pgfqpoint{4.776953in}{0.000000in}}%
\pgfpathlineto{\pgfqpoint{4.776953in}{3.310123in}}%
\pgfpathlineto{\pgfqpoint{0.000000in}{3.310123in}}%
\pgfpathclose%
\pgfusepath{fill}%
\end{pgfscope}%
\begin{pgfscope}%
\pgfsetbuttcap%
\pgfsetmiterjoin%
\definecolor{currentfill}{rgb}{1.000000,1.000000,1.000000}%
\pgfsetfillcolor{currentfill}%
\pgfsetlinewidth{0.000000pt}%
\definecolor{currentstroke}{rgb}{0.000000,0.000000,0.000000}%
\pgfsetstrokecolor{currentstroke}%
\pgfsetstrokeopacity{0.000000}%
\pgfsetdash{}{0pt}%
\pgfpathmoveto{\pgfqpoint{0.772069in}{0.515123in}}%
\pgfpathlineto{\pgfqpoint{4.647069in}{0.515123in}}%
\pgfpathlineto{\pgfqpoint{4.647069in}{3.210123in}}%
\pgfpathlineto{\pgfqpoint{0.772069in}{3.210123in}}%
\pgfpathclose%
\pgfusepath{fill}%
\end{pgfscope}%
\begin{pgfscope}%
\pgfpathrectangle{\pgfqpoint{0.772069in}{0.515123in}}{\pgfqpoint{3.875000in}{2.695000in}}%
\pgfusepath{clip}%
\pgfsetrectcap%
\pgfsetroundjoin%
\pgfsetlinewidth{1.505625pt}%
\definecolor{currentstroke}{rgb}{0.529412,0.807843,0.921569}%
\pgfsetstrokecolor{currentstroke}%
\pgfsetdash{}{0pt}%
\pgfpathmoveto{\pgfqpoint{0.977703in}{0.661170in}}%
\pgfpathlineto{\pgfqpoint{1.362562in}{0.927762in}}%
\pgfpathlineto{\pgfqpoint{1.747421in}{1.194354in}}%
\pgfpathlineto{\pgfqpoint{2.132280in}{1.460946in}}%
\pgfpathlineto{\pgfqpoint{2.517139in}{1.727538in}}%
\pgfpathlineto{\pgfqpoint{2.901998in}{1.994130in}}%
\pgfpathlineto{\pgfqpoint{3.286857in}{2.260722in}}%
\pgfpathlineto{\pgfqpoint{3.671716in}{2.527314in}}%
\pgfpathlineto{\pgfqpoint{4.056576in}{2.793906in}}%
\pgfpathlineto{\pgfqpoint{4.441435in}{3.060498in}}%
\pgfusepath{stroke}%
\end{pgfscope}%
\begin{pgfscope}%
\pgfsetbuttcap%
\pgfsetroundjoin%
\definecolor{currentfill}{rgb}{0.000000,0.000000,0.000000}%
\pgfsetfillcolor{currentfill}%
\pgfsetlinewidth{0.803000pt}%
\definecolor{currentstroke}{rgb}{0.000000,0.000000,0.000000}%
\pgfsetstrokecolor{currentstroke}%
\pgfsetdash{}{0pt}%
\pgfsys@defobject{currentmarker}{\pgfqpoint{0.000000in}{-0.048611in}}{\pgfqpoint{0.000000in}{0.000000in}}{%
\pgfpathmoveto{\pgfqpoint{0.000000in}{0.000000in}}%
\pgfpathlineto{\pgfqpoint{0.000000in}{-0.048611in}}%
\pgfusepath{stroke,fill}%
}%
\begin{pgfscope}%
\pgfsys@transformshift{1.128300in}{0.515123in}%
\pgfsys@useobject{currentmarker}{}%
\end{pgfscope}%
\end{pgfscope}%
\begin{pgfscope}%
\definecolor{textcolor}{rgb}{0.000000,0.000000,0.000000}%
\pgfsetstrokecolor{textcolor}%
\pgfsetfillcolor{textcolor}%
\pgftext[x=1.128300in,y=0.417901in,,top]{\color{textcolor}\rmfamily\fontsize{10.000000}{12.000000}\selectfont \(\displaystyle 1\)}%
\end{pgfscope}%
\begin{pgfscope}%
\pgfsetbuttcap%
\pgfsetroundjoin%
\definecolor{currentfill}{rgb}{0.000000,0.000000,0.000000}%
\pgfsetfillcolor{currentfill}%
\pgfsetlinewidth{0.803000pt}%
\definecolor{currentstroke}{rgb}{0.000000,0.000000,0.000000}%
\pgfsetstrokecolor{currentstroke}%
\pgfsetdash{}{0pt}%
\pgfsys@defobject{currentmarker}{\pgfqpoint{0.000000in}{-0.048611in}}{\pgfqpoint{0.000000in}{0.000000in}}{%
\pgfpathmoveto{\pgfqpoint{0.000000in}{0.000000in}}%
\pgfpathlineto{\pgfqpoint{0.000000in}{-0.048611in}}%
\pgfusepath{stroke,fill}%
}%
\begin{pgfscope}%
\pgfsys@transformshift{1.630290in}{0.515123in}%
\pgfsys@useobject{currentmarker}{}%
\end{pgfscope}%
\end{pgfscope}%
\begin{pgfscope}%
\definecolor{textcolor}{rgb}{0.000000,0.000000,0.000000}%
\pgfsetstrokecolor{textcolor}%
\pgfsetfillcolor{textcolor}%
\pgftext[x=1.630290in,y=0.417901in,,top]{\color{textcolor}\rmfamily\fontsize{10.000000}{12.000000}\selectfont \(\displaystyle 2\)}%
\end{pgfscope}%
\begin{pgfscope}%
\pgfsetbuttcap%
\pgfsetroundjoin%
\definecolor{currentfill}{rgb}{0.000000,0.000000,0.000000}%
\pgfsetfillcolor{currentfill}%
\pgfsetlinewidth{0.803000pt}%
\definecolor{currentstroke}{rgb}{0.000000,0.000000,0.000000}%
\pgfsetstrokecolor{currentstroke}%
\pgfsetdash{}{0pt}%
\pgfsys@defobject{currentmarker}{\pgfqpoint{0.000000in}{-0.048611in}}{\pgfqpoint{0.000000in}{0.000000in}}{%
\pgfpathmoveto{\pgfqpoint{0.000000in}{0.000000in}}%
\pgfpathlineto{\pgfqpoint{0.000000in}{-0.048611in}}%
\pgfusepath{stroke,fill}%
}%
\begin{pgfscope}%
\pgfsys@transformshift{2.132280in}{0.515123in}%
\pgfsys@useobject{currentmarker}{}%
\end{pgfscope}%
\end{pgfscope}%
\begin{pgfscope}%
\definecolor{textcolor}{rgb}{0.000000,0.000000,0.000000}%
\pgfsetstrokecolor{textcolor}%
\pgfsetfillcolor{textcolor}%
\pgftext[x=2.132280in,y=0.417901in,,top]{\color{textcolor}\rmfamily\fontsize{10.000000}{12.000000}\selectfont \(\displaystyle 3\)}%
\end{pgfscope}%
\begin{pgfscope}%
\pgfsetbuttcap%
\pgfsetroundjoin%
\definecolor{currentfill}{rgb}{0.000000,0.000000,0.000000}%
\pgfsetfillcolor{currentfill}%
\pgfsetlinewidth{0.803000pt}%
\definecolor{currentstroke}{rgb}{0.000000,0.000000,0.000000}%
\pgfsetstrokecolor{currentstroke}%
\pgfsetdash{}{0pt}%
\pgfsys@defobject{currentmarker}{\pgfqpoint{0.000000in}{-0.048611in}}{\pgfqpoint{0.000000in}{0.000000in}}{%
\pgfpathmoveto{\pgfqpoint{0.000000in}{0.000000in}}%
\pgfpathlineto{\pgfqpoint{0.000000in}{-0.048611in}}%
\pgfusepath{stroke,fill}%
}%
\begin{pgfscope}%
\pgfsys@transformshift{2.634270in}{0.515123in}%
\pgfsys@useobject{currentmarker}{}%
\end{pgfscope}%
\end{pgfscope}%
\begin{pgfscope}%
\definecolor{textcolor}{rgb}{0.000000,0.000000,0.000000}%
\pgfsetstrokecolor{textcolor}%
\pgfsetfillcolor{textcolor}%
\pgftext[x=2.634270in,y=0.417901in,,top]{\color{textcolor}\rmfamily\fontsize{10.000000}{12.000000}\selectfont \(\displaystyle 4\)}%
\end{pgfscope}%
\begin{pgfscope}%
\pgfsetbuttcap%
\pgfsetroundjoin%
\definecolor{currentfill}{rgb}{0.000000,0.000000,0.000000}%
\pgfsetfillcolor{currentfill}%
\pgfsetlinewidth{0.803000pt}%
\definecolor{currentstroke}{rgb}{0.000000,0.000000,0.000000}%
\pgfsetstrokecolor{currentstroke}%
\pgfsetdash{}{0pt}%
\pgfsys@defobject{currentmarker}{\pgfqpoint{0.000000in}{-0.048611in}}{\pgfqpoint{0.000000in}{0.000000in}}{%
\pgfpathmoveto{\pgfqpoint{0.000000in}{0.000000in}}%
\pgfpathlineto{\pgfqpoint{0.000000in}{-0.048611in}}%
\pgfusepath{stroke,fill}%
}%
\begin{pgfscope}%
\pgfsys@transformshift{3.136260in}{0.515123in}%
\pgfsys@useobject{currentmarker}{}%
\end{pgfscope}%
\end{pgfscope}%
\begin{pgfscope}%
\definecolor{textcolor}{rgb}{0.000000,0.000000,0.000000}%
\pgfsetstrokecolor{textcolor}%
\pgfsetfillcolor{textcolor}%
\pgftext[x=3.136260in,y=0.417901in,,top]{\color{textcolor}\rmfamily\fontsize{10.000000}{12.000000}\selectfont \(\displaystyle 5\)}%
\end{pgfscope}%
\begin{pgfscope}%
\pgfsetbuttcap%
\pgfsetroundjoin%
\definecolor{currentfill}{rgb}{0.000000,0.000000,0.000000}%
\pgfsetfillcolor{currentfill}%
\pgfsetlinewidth{0.803000pt}%
\definecolor{currentstroke}{rgb}{0.000000,0.000000,0.000000}%
\pgfsetstrokecolor{currentstroke}%
\pgfsetdash{}{0pt}%
\pgfsys@defobject{currentmarker}{\pgfqpoint{0.000000in}{-0.048611in}}{\pgfqpoint{0.000000in}{0.000000in}}{%
\pgfpathmoveto{\pgfqpoint{0.000000in}{0.000000in}}%
\pgfpathlineto{\pgfqpoint{0.000000in}{-0.048611in}}%
\pgfusepath{stroke,fill}%
}%
\begin{pgfscope}%
\pgfsys@transformshift{3.638250in}{0.515123in}%
\pgfsys@useobject{currentmarker}{}%
\end{pgfscope}%
\end{pgfscope}%
\begin{pgfscope}%
\definecolor{textcolor}{rgb}{0.000000,0.000000,0.000000}%
\pgfsetstrokecolor{textcolor}%
\pgfsetfillcolor{textcolor}%
\pgftext[x=3.638250in,y=0.417901in,,top]{\color{textcolor}\rmfamily\fontsize{10.000000}{12.000000}\selectfont \(\displaystyle 6\)}%
\end{pgfscope}%
\begin{pgfscope}%
\pgfsetbuttcap%
\pgfsetroundjoin%
\definecolor{currentfill}{rgb}{0.000000,0.000000,0.000000}%
\pgfsetfillcolor{currentfill}%
\pgfsetlinewidth{0.803000pt}%
\definecolor{currentstroke}{rgb}{0.000000,0.000000,0.000000}%
\pgfsetstrokecolor{currentstroke}%
\pgfsetdash{}{0pt}%
\pgfsys@defobject{currentmarker}{\pgfqpoint{0.000000in}{-0.048611in}}{\pgfqpoint{0.000000in}{0.000000in}}{%
\pgfpathmoveto{\pgfqpoint{0.000000in}{0.000000in}}%
\pgfpathlineto{\pgfqpoint{0.000000in}{-0.048611in}}%
\pgfusepath{stroke,fill}%
}%
\begin{pgfscope}%
\pgfsys@transformshift{4.140241in}{0.515123in}%
\pgfsys@useobject{currentmarker}{}%
\end{pgfscope}%
\end{pgfscope}%
\begin{pgfscope}%
\definecolor{textcolor}{rgb}{0.000000,0.000000,0.000000}%
\pgfsetstrokecolor{textcolor}%
\pgfsetfillcolor{textcolor}%
\pgftext[x=4.140241in,y=0.417901in,,top]{\color{textcolor}\rmfamily\fontsize{10.000000}{12.000000}\selectfont \(\displaystyle 7\)}%
\end{pgfscope}%
\begin{pgfscope}%
\pgfsetbuttcap%
\pgfsetroundjoin%
\definecolor{currentfill}{rgb}{0.000000,0.000000,0.000000}%
\pgfsetfillcolor{currentfill}%
\pgfsetlinewidth{0.803000pt}%
\definecolor{currentstroke}{rgb}{0.000000,0.000000,0.000000}%
\pgfsetstrokecolor{currentstroke}%
\pgfsetdash{}{0pt}%
\pgfsys@defobject{currentmarker}{\pgfqpoint{0.000000in}{-0.048611in}}{\pgfqpoint{0.000000in}{0.000000in}}{%
\pgfpathmoveto{\pgfqpoint{0.000000in}{0.000000in}}%
\pgfpathlineto{\pgfqpoint{0.000000in}{-0.048611in}}%
\pgfusepath{stroke,fill}%
}%
\begin{pgfscope}%
\pgfsys@transformshift{4.642231in}{0.515123in}%
\pgfsys@useobject{currentmarker}{}%
\end{pgfscope}%
\end{pgfscope}%
\begin{pgfscope}%
\definecolor{textcolor}{rgb}{0.000000,0.000000,0.000000}%
\pgfsetstrokecolor{textcolor}%
\pgfsetfillcolor{textcolor}%
\pgftext[x=4.642231in,y=0.417901in,,top]{\color{textcolor}\rmfamily\fontsize{10.000000}{12.000000}\selectfont \(\displaystyle 8\)}%
\end{pgfscope}%
\begin{pgfscope}%
\definecolor{textcolor}{rgb}{0.000000,0.000000,0.000000}%
\pgfsetstrokecolor{textcolor}%
\pgfsetfillcolor{textcolor}%
\pgftext[x=2.709569in,y=0.238889in,,top]{\color{textcolor}\rmfamily\fontsize{10.000000}{12.000000}\selectfont I(\(\displaystyle \mu\)A)}%
\end{pgfscope}%
\begin{pgfscope}%
\pgfsetbuttcap%
\pgfsetroundjoin%
\definecolor{currentfill}{rgb}{0.000000,0.000000,0.000000}%
\pgfsetfillcolor{currentfill}%
\pgfsetlinewidth{0.803000pt}%
\definecolor{currentstroke}{rgb}{0.000000,0.000000,0.000000}%
\pgfsetstrokecolor{currentstroke}%
\pgfsetdash{}{0pt}%
\pgfsys@defobject{currentmarker}{\pgfqpoint{-0.048611in}{0.000000in}}{\pgfqpoint{0.000000in}{0.000000in}}{%
\pgfpathmoveto{\pgfqpoint{0.000000in}{0.000000in}}%
\pgfpathlineto{\pgfqpoint{-0.048611in}{0.000000in}}%
\pgfusepath{stroke,fill}%
}%
\begin{pgfscope}%
\pgfsys@transformshift{0.772069in}{0.921704in}%
\pgfsys@useobject{currentmarker}{}%
\end{pgfscope}%
\end{pgfscope}%
\begin{pgfscope}%
\definecolor{textcolor}{rgb}{0.000000,0.000000,0.000000}%
\pgfsetstrokecolor{textcolor}%
\pgfsetfillcolor{textcolor}%
\pgftext[x=0.605402in,y=0.873478in,left,base]{\color{textcolor}\rmfamily\fontsize{10.000000}{12.000000}\selectfont \(\displaystyle 2\)}%
\end{pgfscope}%
\begin{pgfscope}%
\pgfsetbuttcap%
\pgfsetroundjoin%
\definecolor{currentfill}{rgb}{0.000000,0.000000,0.000000}%
\pgfsetfillcolor{currentfill}%
\pgfsetlinewidth{0.803000pt}%
\definecolor{currentstroke}{rgb}{0.000000,0.000000,0.000000}%
\pgfsetstrokecolor{currentstroke}%
\pgfsetdash{}{0pt}%
\pgfsys@defobject{currentmarker}{\pgfqpoint{-0.048611in}{0.000000in}}{\pgfqpoint{0.000000in}{0.000000in}}{%
\pgfpathmoveto{\pgfqpoint{0.000000in}{0.000000in}}%
\pgfpathlineto{\pgfqpoint{-0.048611in}{0.000000in}}%
\pgfusepath{stroke,fill}%
}%
\begin{pgfscope}%
\pgfsys@transformshift{0.772069in}{1.447945in}%
\pgfsys@useobject{currentmarker}{}%
\end{pgfscope}%
\end{pgfscope}%
\begin{pgfscope}%
\definecolor{textcolor}{rgb}{0.000000,0.000000,0.000000}%
\pgfsetstrokecolor{textcolor}%
\pgfsetfillcolor{textcolor}%
\pgftext[x=0.605402in,y=1.399720in,left,base]{\color{textcolor}\rmfamily\fontsize{10.000000}{12.000000}\selectfont \(\displaystyle 4\)}%
\end{pgfscope}%
\begin{pgfscope}%
\pgfsetbuttcap%
\pgfsetroundjoin%
\definecolor{currentfill}{rgb}{0.000000,0.000000,0.000000}%
\pgfsetfillcolor{currentfill}%
\pgfsetlinewidth{0.803000pt}%
\definecolor{currentstroke}{rgb}{0.000000,0.000000,0.000000}%
\pgfsetstrokecolor{currentstroke}%
\pgfsetdash{}{0pt}%
\pgfsys@defobject{currentmarker}{\pgfqpoint{-0.048611in}{0.000000in}}{\pgfqpoint{0.000000in}{0.000000in}}{%
\pgfpathmoveto{\pgfqpoint{0.000000in}{0.000000in}}%
\pgfpathlineto{\pgfqpoint{-0.048611in}{0.000000in}}%
\pgfusepath{stroke,fill}%
}%
\begin{pgfscope}%
\pgfsys@transformshift{0.772069in}{1.974187in}%
\pgfsys@useobject{currentmarker}{}%
\end{pgfscope}%
\end{pgfscope}%
\begin{pgfscope}%
\definecolor{textcolor}{rgb}{0.000000,0.000000,0.000000}%
\pgfsetstrokecolor{textcolor}%
\pgfsetfillcolor{textcolor}%
\pgftext[x=0.605402in,y=1.925961in,left,base]{\color{textcolor}\rmfamily\fontsize{10.000000}{12.000000}\selectfont \(\displaystyle 6\)}%
\end{pgfscope}%
\begin{pgfscope}%
\pgfsetbuttcap%
\pgfsetroundjoin%
\definecolor{currentfill}{rgb}{0.000000,0.000000,0.000000}%
\pgfsetfillcolor{currentfill}%
\pgfsetlinewidth{0.803000pt}%
\definecolor{currentstroke}{rgb}{0.000000,0.000000,0.000000}%
\pgfsetstrokecolor{currentstroke}%
\pgfsetdash{}{0pt}%
\pgfsys@defobject{currentmarker}{\pgfqpoint{-0.048611in}{0.000000in}}{\pgfqpoint{0.000000in}{0.000000in}}{%
\pgfpathmoveto{\pgfqpoint{0.000000in}{0.000000in}}%
\pgfpathlineto{\pgfqpoint{-0.048611in}{0.000000in}}%
\pgfusepath{stroke,fill}%
}%
\begin{pgfscope}%
\pgfsys@transformshift{0.772069in}{2.500428in}%
\pgfsys@useobject{currentmarker}{}%
\end{pgfscope}%
\end{pgfscope}%
\begin{pgfscope}%
\definecolor{textcolor}{rgb}{0.000000,0.000000,0.000000}%
\pgfsetstrokecolor{textcolor}%
\pgfsetfillcolor{textcolor}%
\pgftext[x=0.605402in,y=2.452203in,left,base]{\color{textcolor}\rmfamily\fontsize{10.000000}{12.000000}\selectfont \(\displaystyle 8\)}%
\end{pgfscope}%
\begin{pgfscope}%
\pgfsetbuttcap%
\pgfsetroundjoin%
\definecolor{currentfill}{rgb}{0.000000,0.000000,0.000000}%
\pgfsetfillcolor{currentfill}%
\pgfsetlinewidth{0.803000pt}%
\definecolor{currentstroke}{rgb}{0.000000,0.000000,0.000000}%
\pgfsetstrokecolor{currentstroke}%
\pgfsetdash{}{0pt}%
\pgfsys@defobject{currentmarker}{\pgfqpoint{-0.048611in}{0.000000in}}{\pgfqpoint{0.000000in}{0.000000in}}{%
\pgfpathmoveto{\pgfqpoint{0.000000in}{0.000000in}}%
\pgfpathlineto{\pgfqpoint{-0.048611in}{0.000000in}}%
\pgfusepath{stroke,fill}%
}%
\begin{pgfscope}%
\pgfsys@transformshift{0.772069in}{3.026669in}%
\pgfsys@useobject{currentmarker}{}%
\end{pgfscope}%
\end{pgfscope}%
\begin{pgfscope}%
\definecolor{textcolor}{rgb}{0.000000,0.000000,0.000000}%
\pgfsetstrokecolor{textcolor}%
\pgfsetfillcolor{textcolor}%
\pgftext[x=0.535957in,y=2.978444in,left,base]{\color{textcolor}\rmfamily\fontsize{10.000000}{12.000000}\selectfont \(\displaystyle 10\)}%
\end{pgfscope}%
\begin{pgfscope}%
\definecolor{textcolor}{rgb}{0.000000,0.000000,0.000000}%
\pgfsetstrokecolor{textcolor}%
\pgfsetfillcolor{textcolor}%
\pgftext[x=0.258179in,y=1.862623in,,bottom]{\color{textcolor}\rmfamily\fontsize{10.000000}{12.000000}\selectfont V(V)}%
\end{pgfscope}%
\begin{pgfscope}%
\pgfpathrectangle{\pgfqpoint{0.772069in}{0.515123in}}{\pgfqpoint{3.875000in}{2.695000in}}%
\pgfusepath{clip}%
\pgfsetbuttcap%
\pgfsetroundjoin%
\definecolor{currentfill}{rgb}{0.121569,0.466667,0.705882}%
\pgfsetfillcolor{currentfill}%
\pgfsetlinewidth{1.003750pt}%
\definecolor{currentstroke}{rgb}{0.121569,0.466667,0.705882}%
\pgfsetstrokecolor{currentstroke}%
\pgfsetdash{}{0pt}%
\pgfpathmoveto{\pgfqpoint{0.977703in}{0.625336in}}%
\pgfpathcurveto{\pgfqpoint{0.988753in}{0.625336in}}{\pgfqpoint{0.999352in}{0.629726in}}{\pgfqpoint{1.007165in}{0.637540in}}%
\pgfpathcurveto{\pgfqpoint{1.014979in}{0.645354in}}{\pgfqpoint{1.019369in}{0.655953in}}{\pgfqpoint{1.019369in}{0.667003in}}%
\pgfpathcurveto{\pgfqpoint{1.019369in}{0.678053in}}{\pgfqpoint{1.014979in}{0.688652in}}{\pgfqpoint{1.007165in}{0.696466in}}%
\pgfpathcurveto{\pgfqpoint{0.999352in}{0.704279in}}{\pgfqpoint{0.988753in}{0.708670in}}{\pgfqpoint{0.977703in}{0.708670in}}%
\pgfpathcurveto{\pgfqpoint{0.966653in}{0.708670in}}{\pgfqpoint{0.956053in}{0.704279in}}{\pgfqpoint{0.948240in}{0.696466in}}%
\pgfpathcurveto{\pgfqpoint{0.940426in}{0.688652in}}{\pgfqpoint{0.936036in}{0.678053in}}{\pgfqpoint{0.936036in}{0.667003in}}%
\pgfpathcurveto{\pgfqpoint{0.936036in}{0.655953in}}{\pgfqpoint{0.940426in}{0.645354in}}{\pgfqpoint{0.948240in}{0.637540in}}%
\pgfpathcurveto{\pgfqpoint{0.956053in}{0.629726in}}{\pgfqpoint{0.966653in}{0.625336in}}{\pgfqpoint{0.977703in}{0.625336in}}%
\pgfpathclose%
\pgfusepath{stroke,fill}%
\end{pgfscope}%
\begin{pgfscope}%
\pgfpathrectangle{\pgfqpoint{0.772069in}{0.515123in}}{\pgfqpoint{3.875000in}{2.695000in}}%
\pgfusepath{clip}%
\pgfsetbuttcap%
\pgfsetroundjoin%
\definecolor{currentfill}{rgb}{0.121569,0.466667,0.705882}%
\pgfsetfillcolor{currentfill}%
\pgfsetlinewidth{1.003750pt}%
\definecolor{currentstroke}{rgb}{0.121569,0.466667,0.705882}%
\pgfsetstrokecolor{currentstroke}%
\pgfsetdash{}{0pt}%
\pgfpathmoveto{\pgfqpoint{1.379295in}{0.880037in}}%
\pgfpathcurveto{\pgfqpoint{1.390345in}{0.880037in}}{\pgfqpoint{1.400944in}{0.884427in}}{\pgfqpoint{1.408758in}{0.892241in}}%
\pgfpathcurveto{\pgfqpoint{1.416571in}{0.900055in}}{\pgfqpoint{1.420961in}{0.910654in}}{\pgfqpoint{1.420961in}{0.921704in}}%
\pgfpathcurveto{\pgfqpoint{1.420961in}{0.932754in}}{\pgfqpoint{1.416571in}{0.943353in}}{\pgfqpoint{1.408758in}{0.951166in}}%
\pgfpathcurveto{\pgfqpoint{1.400944in}{0.958980in}}{\pgfqpoint{1.390345in}{0.963370in}}{\pgfqpoint{1.379295in}{0.963370in}}%
\pgfpathcurveto{\pgfqpoint{1.368245in}{0.963370in}}{\pgfqpoint{1.357646in}{0.958980in}}{\pgfqpoint{1.349832in}{0.951166in}}%
\pgfpathcurveto{\pgfqpoint{1.342018in}{0.943353in}}{\pgfqpoint{1.337628in}{0.932754in}}{\pgfqpoint{1.337628in}{0.921704in}}%
\pgfpathcurveto{\pgfqpoint{1.337628in}{0.910654in}}{\pgfqpoint{1.342018in}{0.900055in}}{\pgfqpoint{1.349832in}{0.892241in}}%
\pgfpathcurveto{\pgfqpoint{1.357646in}{0.884427in}}{\pgfqpoint{1.368245in}{0.880037in}}{\pgfqpoint{1.379295in}{0.880037in}}%
\pgfpathclose%
\pgfusepath{stroke,fill}%
\end{pgfscope}%
\begin{pgfscope}%
\pgfpathrectangle{\pgfqpoint{0.772069in}{0.515123in}}{\pgfqpoint{3.875000in}{2.695000in}}%
\pgfusepath{clip}%
\pgfsetbuttcap%
\pgfsetroundjoin%
\definecolor{currentfill}{rgb}{0.121569,0.466667,0.705882}%
\pgfsetfillcolor{currentfill}%
\pgfsetlinewidth{1.003750pt}%
\definecolor{currentstroke}{rgb}{0.121569,0.466667,0.705882}%
\pgfsetstrokecolor{currentstroke}%
\pgfsetdash{}{0pt}%
\pgfpathmoveto{\pgfqpoint{1.730688in}{1.153683in}}%
\pgfpathcurveto{\pgfqpoint{1.741738in}{1.153683in}}{\pgfqpoint{1.752337in}{1.158073in}}{\pgfqpoint{1.760151in}{1.165886in}}%
\pgfpathcurveto{\pgfqpoint{1.767964in}{1.173700in}}{\pgfqpoint{1.772355in}{1.184299in}}{\pgfqpoint{1.772355in}{1.195349in}}%
\pgfpathcurveto{\pgfqpoint{1.772355in}{1.206399in}}{\pgfqpoint{1.767964in}{1.216998in}}{\pgfqpoint{1.760151in}{1.224812in}}%
\pgfpathcurveto{\pgfqpoint{1.752337in}{1.232626in}}{\pgfqpoint{1.741738in}{1.237016in}}{\pgfqpoint{1.730688in}{1.237016in}}%
\pgfpathcurveto{\pgfqpoint{1.719638in}{1.237016in}}{\pgfqpoint{1.709039in}{1.232626in}}{\pgfqpoint{1.701225in}{1.224812in}}%
\pgfpathcurveto{\pgfqpoint{1.693411in}{1.216998in}}{\pgfqpoint{1.689021in}{1.206399in}}{\pgfqpoint{1.689021in}{1.195349in}}%
\pgfpathcurveto{\pgfqpoint{1.689021in}{1.184299in}}{\pgfqpoint{1.693411in}{1.173700in}}{\pgfqpoint{1.701225in}{1.165886in}}%
\pgfpathcurveto{\pgfqpoint{1.709039in}{1.158073in}}{\pgfqpoint{1.719638in}{1.153683in}}{\pgfqpoint{1.730688in}{1.153683in}}%
\pgfpathclose%
\pgfusepath{stroke,fill}%
\end{pgfscope}%
\begin{pgfscope}%
\pgfpathrectangle{\pgfqpoint{0.772069in}{0.515123in}}{\pgfqpoint{3.875000in}{2.695000in}}%
\pgfusepath{clip}%
\pgfsetbuttcap%
\pgfsetroundjoin%
\definecolor{currentfill}{rgb}{0.121569,0.466667,0.705882}%
\pgfsetfillcolor{currentfill}%
\pgfsetlinewidth{1.003750pt}%
\definecolor{currentstroke}{rgb}{0.121569,0.466667,0.705882}%
\pgfsetstrokecolor{currentstroke}%
\pgfsetdash{}{0pt}%
\pgfpathmoveto{\pgfqpoint{2.132280in}{1.422066in}}%
\pgfpathcurveto{\pgfqpoint{2.143330in}{1.422066in}}{\pgfqpoint{2.153929in}{1.426456in}}{\pgfqpoint{2.161743in}{1.434270in}}%
\pgfpathcurveto{\pgfqpoint{2.169556in}{1.442083in}}{\pgfqpoint{2.173947in}{1.452682in}}{\pgfqpoint{2.173947in}{1.463732in}}%
\pgfpathcurveto{\pgfqpoint{2.173947in}{1.474782in}}{\pgfqpoint{2.169556in}{1.485382in}}{\pgfqpoint{2.161743in}{1.493195in}}%
\pgfpathcurveto{\pgfqpoint{2.153929in}{1.501009in}}{\pgfqpoint{2.143330in}{1.505399in}}{\pgfqpoint{2.132280in}{1.505399in}}%
\pgfpathcurveto{\pgfqpoint{2.121230in}{1.505399in}}{\pgfqpoint{2.110631in}{1.501009in}}{\pgfqpoint{2.102817in}{1.493195in}}%
\pgfpathcurveto{\pgfqpoint{2.095004in}{1.485382in}}{\pgfqpoint{2.090613in}{1.474782in}}{\pgfqpoint{2.090613in}{1.463732in}}%
\pgfpathcurveto{\pgfqpoint{2.090613in}{1.452682in}}{\pgfqpoint{2.095004in}{1.442083in}}{\pgfqpoint{2.102817in}{1.434270in}}%
\pgfpathcurveto{\pgfqpoint{2.110631in}{1.426456in}}{\pgfqpoint{2.121230in}{1.422066in}}{\pgfqpoint{2.132280in}{1.422066in}}%
\pgfpathclose%
\pgfusepath{stroke,fill}%
\end{pgfscope}%
\begin{pgfscope}%
\pgfpathrectangle{\pgfqpoint{0.772069in}{0.515123in}}{\pgfqpoint{3.875000in}{2.695000in}}%
\pgfusepath{clip}%
\pgfsetbuttcap%
\pgfsetroundjoin%
\definecolor{currentfill}{rgb}{0.121569,0.466667,0.705882}%
\pgfsetfillcolor{currentfill}%
\pgfsetlinewidth{1.003750pt}%
\definecolor{currentstroke}{rgb}{0.121569,0.466667,0.705882}%
\pgfsetstrokecolor{currentstroke}%
\pgfsetdash{}{0pt}%
\pgfpathmoveto{\pgfqpoint{2.533872in}{1.690449in}}%
\pgfpathcurveto{\pgfqpoint{2.544922in}{1.690449in}}{\pgfqpoint{2.555521in}{1.694839in}}{\pgfqpoint{2.563335in}{1.702653in}}%
\pgfpathcurveto{\pgfqpoint{2.571149in}{1.710466in}}{\pgfqpoint{2.575539in}{1.721065in}}{\pgfqpoint{2.575539in}{1.732115in}}%
\pgfpathcurveto{\pgfqpoint{2.575539in}{1.743166in}}{\pgfqpoint{2.571149in}{1.753765in}}{\pgfqpoint{2.563335in}{1.761578in}}%
\pgfpathcurveto{\pgfqpoint{2.555521in}{1.769392in}}{\pgfqpoint{2.544922in}{1.773782in}}{\pgfqpoint{2.533872in}{1.773782in}}%
\pgfpathcurveto{\pgfqpoint{2.522822in}{1.773782in}}{\pgfqpoint{2.512223in}{1.769392in}}{\pgfqpoint{2.504409in}{1.761578in}}%
\pgfpathcurveto{\pgfqpoint{2.496596in}{1.753765in}}{\pgfqpoint{2.492205in}{1.743166in}}{\pgfqpoint{2.492205in}{1.732115in}}%
\pgfpathcurveto{\pgfqpoint{2.492205in}{1.721065in}}{\pgfqpoint{2.496596in}{1.710466in}}{\pgfqpoint{2.504409in}{1.702653in}}%
\pgfpathcurveto{\pgfqpoint{2.512223in}{1.694839in}}{\pgfqpoint{2.522822in}{1.690449in}}{\pgfqpoint{2.533872in}{1.690449in}}%
\pgfpathclose%
\pgfusepath{stroke,fill}%
\end{pgfscope}%
\begin{pgfscope}%
\pgfpathrectangle{\pgfqpoint{0.772069in}{0.515123in}}{\pgfqpoint{3.875000in}{2.695000in}}%
\pgfusepath{clip}%
\pgfsetbuttcap%
\pgfsetroundjoin%
\definecolor{currentfill}{rgb}{0.121569,0.466667,0.705882}%
\pgfsetfillcolor{currentfill}%
\pgfsetlinewidth{1.003750pt}%
\definecolor{currentstroke}{rgb}{0.121569,0.466667,0.705882}%
\pgfsetstrokecolor{currentstroke}%
\pgfsetdash{}{0pt}%
\pgfpathmoveto{\pgfqpoint{2.885265in}{1.945676in}}%
\pgfpathcurveto{\pgfqpoint{2.896315in}{1.945676in}}{\pgfqpoint{2.906914in}{1.950066in}}{\pgfqpoint{2.914728in}{1.957880in}}%
\pgfpathcurveto{\pgfqpoint{2.922542in}{1.965693in}}{\pgfqpoint{2.926932in}{1.976292in}}{\pgfqpoint{2.926932in}{1.987343in}}%
\pgfpathcurveto{\pgfqpoint{2.926932in}{1.998393in}}{\pgfqpoint{2.922542in}{2.008992in}}{\pgfqpoint{2.914728in}{2.016805in}}%
\pgfpathcurveto{\pgfqpoint{2.906914in}{2.024619in}}{\pgfqpoint{2.896315in}{2.029009in}}{\pgfqpoint{2.885265in}{2.029009in}}%
\pgfpathcurveto{\pgfqpoint{2.874215in}{2.029009in}}{\pgfqpoint{2.863616in}{2.024619in}}{\pgfqpoint{2.855802in}{2.016805in}}%
\pgfpathcurveto{\pgfqpoint{2.847989in}{2.008992in}}{\pgfqpoint{2.843599in}{1.998393in}}{\pgfqpoint{2.843599in}{1.987343in}}%
\pgfpathcurveto{\pgfqpoint{2.843599in}{1.976292in}}{\pgfqpoint{2.847989in}{1.965693in}}{\pgfqpoint{2.855802in}{1.957880in}}%
\pgfpathcurveto{\pgfqpoint{2.863616in}{1.950066in}}{\pgfqpoint{2.874215in}{1.945676in}}{\pgfqpoint{2.885265in}{1.945676in}}%
\pgfpathclose%
\pgfusepath{stroke,fill}%
\end{pgfscope}%
\begin{pgfscope}%
\pgfpathrectangle{\pgfqpoint{0.772069in}{0.515123in}}{\pgfqpoint{3.875000in}{2.695000in}}%
\pgfusepath{clip}%
\pgfsetbuttcap%
\pgfsetroundjoin%
\definecolor{currentfill}{rgb}{0.121569,0.466667,0.705882}%
\pgfsetfillcolor{currentfill}%
\pgfsetlinewidth{1.003750pt}%
\definecolor{currentstroke}{rgb}{0.121569,0.466667,0.705882}%
\pgfsetstrokecolor{currentstroke}%
\pgfsetdash{}{0pt}%
\pgfpathmoveto{\pgfqpoint{3.286857in}{2.221953in}}%
\pgfpathcurveto{\pgfqpoint{3.297907in}{2.221953in}}{\pgfqpoint{3.308507in}{2.226343in}}{\pgfqpoint{3.316320in}{2.234157in}}%
\pgfpathcurveto{\pgfqpoint{3.324134in}{2.241970in}}{\pgfqpoint{3.328524in}{2.252569in}}{\pgfqpoint{3.328524in}{2.263619in}}%
\pgfpathcurveto{\pgfqpoint{3.328524in}{2.274669in}}{\pgfqpoint{3.324134in}{2.285268in}}{\pgfqpoint{3.316320in}{2.293082in}}%
\pgfpathcurveto{\pgfqpoint{3.308507in}{2.300896in}}{\pgfqpoint{3.297907in}{2.305286in}}{\pgfqpoint{3.286857in}{2.305286in}}%
\pgfpathcurveto{\pgfqpoint{3.275807in}{2.305286in}}{\pgfqpoint{3.265208in}{2.300896in}}{\pgfqpoint{3.257395in}{2.293082in}}%
\pgfpathcurveto{\pgfqpoint{3.249581in}{2.285268in}}{\pgfqpoint{3.245191in}{2.274669in}}{\pgfqpoint{3.245191in}{2.263619in}}%
\pgfpathcurveto{\pgfqpoint{3.245191in}{2.252569in}}{\pgfqpoint{3.249581in}{2.241970in}}{\pgfqpoint{3.257395in}{2.234157in}}%
\pgfpathcurveto{\pgfqpoint{3.265208in}{2.226343in}}{\pgfqpoint{3.275807in}{2.221953in}}{\pgfqpoint{3.286857in}{2.221953in}}%
\pgfpathclose%
\pgfusepath{stroke,fill}%
\end{pgfscope}%
\begin{pgfscope}%
\pgfpathrectangle{\pgfqpoint{0.772069in}{0.515123in}}{\pgfqpoint{3.875000in}{2.695000in}}%
\pgfusepath{clip}%
\pgfsetbuttcap%
\pgfsetroundjoin%
\definecolor{currentfill}{rgb}{0.121569,0.466667,0.705882}%
\pgfsetfillcolor{currentfill}%
\pgfsetlinewidth{1.003750pt}%
\definecolor{currentstroke}{rgb}{0.121569,0.466667,0.705882}%
\pgfsetstrokecolor{currentstroke}%
\pgfsetdash{}{0pt}%
\pgfpathmoveto{\pgfqpoint{3.688449in}{2.487705in}}%
\pgfpathcurveto{\pgfqpoint{3.699500in}{2.487705in}}{\pgfqpoint{3.710099in}{2.492095in}}{\pgfqpoint{3.717912in}{2.499908in}}%
\pgfpathcurveto{\pgfqpoint{3.725726in}{2.507722in}}{\pgfqpoint{3.730116in}{2.518321in}}{\pgfqpoint{3.730116in}{2.529371in}}%
\pgfpathcurveto{\pgfqpoint{3.730116in}{2.540421in}}{\pgfqpoint{3.725726in}{2.551020in}}{\pgfqpoint{3.717912in}{2.558834in}}%
\pgfpathcurveto{\pgfqpoint{3.710099in}{2.566648in}}{\pgfqpoint{3.699500in}{2.571038in}}{\pgfqpoint{3.688449in}{2.571038in}}%
\pgfpathcurveto{\pgfqpoint{3.677399in}{2.571038in}}{\pgfqpoint{3.666800in}{2.566648in}}{\pgfqpoint{3.658987in}{2.558834in}}%
\pgfpathcurveto{\pgfqpoint{3.651173in}{2.551020in}}{\pgfqpoint{3.646783in}{2.540421in}}{\pgfqpoint{3.646783in}{2.529371in}}%
\pgfpathcurveto{\pgfqpoint{3.646783in}{2.518321in}}{\pgfqpoint{3.651173in}{2.507722in}}{\pgfqpoint{3.658987in}{2.499908in}}%
\pgfpathcurveto{\pgfqpoint{3.666800in}{2.492095in}}{\pgfqpoint{3.677399in}{2.487705in}}{\pgfqpoint{3.688449in}{2.487705in}}%
\pgfpathclose%
\pgfusepath{stroke,fill}%
\end{pgfscope}%
\begin{pgfscope}%
\pgfpathrectangle{\pgfqpoint{0.772069in}{0.515123in}}{\pgfqpoint{3.875000in}{2.695000in}}%
\pgfusepath{clip}%
\pgfsetbuttcap%
\pgfsetroundjoin%
\definecolor{currentfill}{rgb}{0.121569,0.466667,0.705882}%
\pgfsetfillcolor{currentfill}%
\pgfsetlinewidth{1.003750pt}%
\definecolor{currentstroke}{rgb}{0.121569,0.466667,0.705882}%
\pgfsetstrokecolor{currentstroke}%
\pgfsetdash{}{0pt}%
\pgfpathmoveto{\pgfqpoint{4.039843in}{2.748194in}}%
\pgfpathcurveto{\pgfqpoint{4.050893in}{2.748194in}}{\pgfqpoint{4.061492in}{2.752584in}}{\pgfqpoint{4.069305in}{2.760398in}}%
\pgfpathcurveto{\pgfqpoint{4.077119in}{2.768212in}}{\pgfqpoint{4.081509in}{2.778811in}}{\pgfqpoint{4.081509in}{2.789861in}}%
\pgfpathcurveto{\pgfqpoint{4.081509in}{2.800911in}}{\pgfqpoint{4.077119in}{2.811510in}}{\pgfqpoint{4.069305in}{2.819323in}}%
\pgfpathcurveto{\pgfqpoint{4.061492in}{2.827137in}}{\pgfqpoint{4.050893in}{2.831527in}}{\pgfqpoint{4.039843in}{2.831527in}}%
\pgfpathcurveto{\pgfqpoint{4.028792in}{2.831527in}}{\pgfqpoint{4.018193in}{2.827137in}}{\pgfqpoint{4.010380in}{2.819323in}}%
\pgfpathcurveto{\pgfqpoint{4.002566in}{2.811510in}}{\pgfqpoint{3.998176in}{2.800911in}}{\pgfqpoint{3.998176in}{2.789861in}}%
\pgfpathcurveto{\pgfqpoint{3.998176in}{2.778811in}}{\pgfqpoint{4.002566in}{2.768212in}}{\pgfqpoint{4.010380in}{2.760398in}}%
\pgfpathcurveto{\pgfqpoint{4.018193in}{2.752584in}}{\pgfqpoint{4.028792in}{2.748194in}}{\pgfqpoint{4.039843in}{2.748194in}}%
\pgfpathclose%
\pgfusepath{stroke,fill}%
\end{pgfscope}%
\begin{pgfscope}%
\pgfpathrectangle{\pgfqpoint{0.772069in}{0.515123in}}{\pgfqpoint{3.875000in}{2.695000in}}%
\pgfusepath{clip}%
\pgfsetbuttcap%
\pgfsetroundjoin%
\definecolor{currentfill}{rgb}{0.121569,0.466667,0.705882}%
\pgfsetfillcolor{currentfill}%
\pgfsetlinewidth{1.003750pt}%
\definecolor{currentstroke}{rgb}{0.121569,0.466667,0.705882}%
\pgfsetstrokecolor{currentstroke}%
\pgfsetdash{}{0pt}%
\pgfpathmoveto{\pgfqpoint{4.441435in}{3.016577in}}%
\pgfpathcurveto{\pgfqpoint{4.452485in}{3.016577in}}{\pgfqpoint{4.463084in}{3.020967in}}{\pgfqpoint{4.470898in}{3.028781in}}%
\pgfpathcurveto{\pgfqpoint{4.478711in}{3.036595in}}{\pgfqpoint{4.483101in}{3.047194in}}{\pgfqpoint{4.483101in}{3.058244in}}%
\pgfpathcurveto{\pgfqpoint{4.483101in}{3.069294in}}{\pgfqpoint{4.478711in}{3.079893in}}{\pgfqpoint{4.470898in}{3.087707in}}%
\pgfpathcurveto{\pgfqpoint{4.463084in}{3.095520in}}{\pgfqpoint{4.452485in}{3.099910in}}{\pgfqpoint{4.441435in}{3.099910in}}%
\pgfpathcurveto{\pgfqpoint{4.430385in}{3.099910in}}{\pgfqpoint{4.419786in}{3.095520in}}{\pgfqpoint{4.411972in}{3.087707in}}%
\pgfpathcurveto{\pgfqpoint{4.404158in}{3.079893in}}{\pgfqpoint{4.399768in}{3.069294in}}{\pgfqpoint{4.399768in}{3.058244in}}%
\pgfpathcurveto{\pgfqpoint{4.399768in}{3.047194in}}{\pgfqpoint{4.404158in}{3.036595in}}{\pgfqpoint{4.411972in}{3.028781in}}%
\pgfpathcurveto{\pgfqpoint{4.419786in}{3.020967in}}{\pgfqpoint{4.430385in}{3.016577in}}{\pgfqpoint{4.441435in}{3.016577in}}%
\pgfpathclose%
\pgfusepath{stroke,fill}%
\end{pgfscope}%
\begin{pgfscope}%
\pgfsetrectcap%
\pgfsetmiterjoin%
\pgfsetlinewidth{0.803000pt}%
\definecolor{currentstroke}{rgb}{0.000000,0.000000,0.000000}%
\pgfsetstrokecolor{currentstroke}%
\pgfsetdash{}{0pt}%
\pgfpathmoveto{\pgfqpoint{0.772069in}{0.515123in}}%
\pgfpathlineto{\pgfqpoint{0.772069in}{3.210123in}}%
\pgfusepath{stroke}%
\end{pgfscope}%
\begin{pgfscope}%
\pgfsetrectcap%
\pgfsetmiterjoin%
\pgfsetlinewidth{0.803000pt}%
\definecolor{currentstroke}{rgb}{0.000000,0.000000,0.000000}%
\pgfsetstrokecolor{currentstroke}%
\pgfsetdash{}{0pt}%
\pgfpathmoveto{\pgfqpoint{4.647069in}{0.515123in}}%
\pgfpathlineto{\pgfqpoint{4.647069in}{3.210123in}}%
\pgfusepath{stroke}%
\end{pgfscope}%
\begin{pgfscope}%
\pgfsetrectcap%
\pgfsetmiterjoin%
\pgfsetlinewidth{0.803000pt}%
\definecolor{currentstroke}{rgb}{0.000000,0.000000,0.000000}%
\pgfsetstrokecolor{currentstroke}%
\pgfsetdash{}{0pt}%
\pgfpathmoveto{\pgfqpoint{0.772069in}{0.515123in}}%
\pgfpathlineto{\pgfqpoint{4.647069in}{0.515123in}}%
\pgfusepath{stroke}%
\end{pgfscope}%
\begin{pgfscope}%
\pgfsetrectcap%
\pgfsetmiterjoin%
\pgfsetlinewidth{0.803000pt}%
\definecolor{currentstroke}{rgb}{0.000000,0.000000,0.000000}%
\pgfsetstrokecolor{currentstroke}%
\pgfsetdash{}{0pt}%
\pgfpathmoveto{\pgfqpoint{0.772069in}{3.210123in}}%
\pgfpathlineto{\pgfqpoint{4.647069in}{3.210123in}}%
\pgfusepath{stroke}%
\end{pgfscope}%
\end{pgfpicture}%
\makeatother%
\endgroup%

    \caption{Voltaje (V) frente a intensidad (I) con regresión lineal}
  \end{figure}

  Si comprobamos el valor de la resistencia equivalente calculado en \ref{sec:reseqmixto} ($R = 1,3210 \cdot 10^6$) y lo comparamos con la pendiente de la recta ($b = 1,32 \cdot 10^6$) vemos que son prácticamente la misma.

  Podemos probar a hacer la regresión de las diferentes gráficas de la figura \ref{graf:cmixmulti} simplemente cómo curiosidad, y para verificar que todos los datos siguen patrones lineales.

  \begin{figure}[H]
    \centering
    %% Creator: Matplotlib, PGF backend
%%
%% To include the figure in your LaTeX document, write
%%   \input{<filename>.pgf}
%%
%% Make sure the required packages are loaded in your preamble
%%   \usepackage{pgf}
%%
%% Figures using additional raster images can only be included by \input if
%% they are in the same directory as the main LaTeX file. For loading figures
%% from other directories you can use the `import` package
%%   \usepackage{import}
%% and then include the figures with
%%   \import{<path to file>}{<filename>.pgf}
%%
%% Matplotlib used the following preamble
%%
\begingroup%
\makeatletter%
\begin{pgfpicture}%
\pgfpathrectangle{\pgfpointorigin}{\pgfqpoint{2.824853in}{1.962623in}}%
\pgfusepath{use as bounding box, clip}%
\begin{pgfscope}%
\pgfsetbuttcap%
\pgfsetmiterjoin%
\definecolor{currentfill}{rgb}{1.000000,1.000000,1.000000}%
\pgfsetfillcolor{currentfill}%
\pgfsetlinewidth{0.000000pt}%
\definecolor{currentstroke}{rgb}{1.000000,1.000000,1.000000}%
\pgfsetstrokecolor{currentstroke}%
\pgfsetdash{}{0pt}%
\pgfpathmoveto{\pgfqpoint{0.000000in}{0.000000in}}%
\pgfpathlineto{\pgfqpoint{2.824853in}{0.000000in}}%
\pgfpathlineto{\pgfqpoint{2.824853in}{1.962623in}}%
\pgfpathlineto{\pgfqpoint{0.000000in}{1.962623in}}%
\pgfpathclose%
\pgfusepath{fill}%
\end{pgfscope}%
\begin{pgfscope}%
\pgfsetbuttcap%
\pgfsetmiterjoin%
\definecolor{currentfill}{rgb}{1.000000,1.000000,1.000000}%
\pgfsetfillcolor{currentfill}%
\pgfsetlinewidth{0.000000pt}%
\definecolor{currentstroke}{rgb}{0.000000,0.000000,0.000000}%
\pgfsetstrokecolor{currentstroke}%
\pgfsetstrokeopacity{0.000000}%
\pgfsetdash{}{0pt}%
\pgfpathmoveto{\pgfqpoint{0.772069in}{0.515123in}}%
\pgfpathlineto{\pgfqpoint{2.709569in}{0.515123in}}%
\pgfpathlineto{\pgfqpoint{2.709569in}{1.862623in}}%
\pgfpathlineto{\pgfqpoint{0.772069in}{1.862623in}}%
\pgfpathclose%
\pgfusepath{fill}%
\end{pgfscope}%
\begin{pgfscope}%
\pgfpathrectangle{\pgfqpoint{0.772069in}{0.515123in}}{\pgfqpoint{1.937500in}{1.347500in}}%
\pgfusepath{clip}%
\pgfsetrectcap%
\pgfsetroundjoin%
\pgfsetlinewidth{1.505625pt}%
\definecolor{currentstroke}{rgb}{0.529412,0.807843,0.921569}%
\pgfsetstrokecolor{currentstroke}%
\pgfsetdash{}{0pt}%
\pgfpathmoveto{\pgfqpoint{0.890137in}{0.701554in}}%
\pgfpathlineto{\pgfqpoint{1.079177in}{0.819870in}}%
\pgfpathlineto{\pgfqpoint{1.268218in}{0.938186in}}%
\pgfpathlineto{\pgfqpoint{1.457258in}{1.056503in}}%
\pgfpathlineto{\pgfqpoint{1.646298in}{1.174819in}}%
\pgfpathlineto{\pgfqpoint{1.835339in}{1.293135in}}%
\pgfpathlineto{\pgfqpoint{2.024379in}{1.411451in}}%
\pgfpathlineto{\pgfqpoint{2.213420in}{1.529767in}}%
\pgfpathlineto{\pgfqpoint{2.402460in}{1.648084in}}%
\pgfpathlineto{\pgfqpoint{2.591501in}{1.766400in}}%
\pgfusepath{stroke}%
\end{pgfscope}%
\begin{pgfscope}%
\pgfpathrectangle{\pgfqpoint{0.772069in}{0.515123in}}{\pgfqpoint{1.937500in}{1.347500in}}%
\pgfusepath{clip}%
\pgfsetrectcap%
\pgfsetroundjoin%
\pgfsetlinewidth{1.505625pt}%
\definecolor{currentstroke}{rgb}{1.000000,0.627451,0.478431}%
\pgfsetstrokecolor{currentstroke}%
\pgfsetdash{}{0pt}%
\pgfpathmoveto{\pgfqpoint{0.890137in}{0.607966in}}%
\pgfpathlineto{\pgfqpoint{1.079177in}{0.623781in}}%
\pgfpathlineto{\pgfqpoint{1.268218in}{0.639595in}}%
\pgfpathlineto{\pgfqpoint{1.457258in}{0.655410in}}%
\pgfpathlineto{\pgfqpoint{1.646298in}{0.671225in}}%
\pgfpathlineto{\pgfqpoint{1.835339in}{0.687040in}}%
\pgfpathlineto{\pgfqpoint{2.024379in}{0.702855in}}%
\pgfpathlineto{\pgfqpoint{2.213420in}{0.718670in}}%
\pgfpathlineto{\pgfqpoint{2.402460in}{0.734484in}}%
\pgfpathlineto{\pgfqpoint{2.591501in}{0.750299in}}%
\pgfusepath{stroke}%
\end{pgfscope}%
\begin{pgfscope}%
\pgfpathrectangle{\pgfqpoint{0.772069in}{0.515123in}}{\pgfqpoint{1.937500in}{1.347500in}}%
\pgfusepath{clip}%
\pgfsetrectcap%
\pgfsetroundjoin%
\pgfsetlinewidth{1.505625pt}%
\definecolor{currentstroke}{rgb}{1.000000,0.894118,0.709804}%
\pgfsetstrokecolor{currentstroke}%
\pgfsetdash{}{0pt}%
\pgfpathmoveto{\pgfqpoint{0.890137in}{0.604869in}}%
\pgfpathlineto{\pgfqpoint{1.079177in}{0.617292in}}%
\pgfpathlineto{\pgfqpoint{1.268218in}{0.629715in}}%
\pgfpathlineto{\pgfqpoint{1.457258in}{0.642138in}}%
\pgfpathlineto{\pgfqpoint{1.646298in}{0.654561in}}%
\pgfpathlineto{\pgfqpoint{1.835339in}{0.666983in}}%
\pgfpathlineto{\pgfqpoint{2.024379in}{0.679406in}}%
\pgfpathlineto{\pgfqpoint{2.213420in}{0.691829in}}%
\pgfpathlineto{\pgfqpoint{2.402460in}{0.704252in}}%
\pgfpathlineto{\pgfqpoint{2.591501in}{0.716675in}}%
\pgfusepath{stroke}%
\end{pgfscope}%
\begin{pgfscope}%
\pgfpathrectangle{\pgfqpoint{0.772069in}{0.515123in}}{\pgfqpoint{1.937500in}{1.347500in}}%
\pgfusepath{clip}%
\pgfsetrectcap%
\pgfsetroundjoin%
\pgfsetlinewidth{1.505625pt}%
\definecolor{currentstroke}{rgb}{0.564706,0.933333,0.564706}%
\pgfsetstrokecolor{currentstroke}%
\pgfsetdash{}{0pt}%
\pgfpathmoveto{\pgfqpoint{0.890137in}{0.673898in}}%
\pgfpathlineto{\pgfqpoint{1.079177in}{0.761925in}}%
\pgfpathlineto{\pgfqpoint{1.268218in}{0.849951in}}%
\pgfpathlineto{\pgfqpoint{1.457258in}{0.937977in}}%
\pgfpathlineto{\pgfqpoint{1.646298in}{1.026004in}}%
\pgfpathlineto{\pgfqpoint{1.835339in}{1.114030in}}%
\pgfpathlineto{\pgfqpoint{2.024379in}{1.202057in}}%
\pgfpathlineto{\pgfqpoint{2.213420in}{1.290083in}}%
\pgfpathlineto{\pgfqpoint{2.402460in}{1.378109in}}%
\pgfpathlineto{\pgfqpoint{2.591501in}{1.466136in}}%
\pgfusepath{stroke}%
\end{pgfscope}%
\begin{pgfscope}%
\pgfsetbuttcap%
\pgfsetroundjoin%
\definecolor{currentfill}{rgb}{0.000000,0.000000,0.000000}%
\pgfsetfillcolor{currentfill}%
\pgfsetlinewidth{0.803000pt}%
\definecolor{currentstroke}{rgb}{0.000000,0.000000,0.000000}%
\pgfsetstrokecolor{currentstroke}%
\pgfsetdash{}{0pt}%
\pgfsys@defobject{currentmarker}{\pgfqpoint{0.000000in}{-0.048611in}}{\pgfqpoint{0.000000in}{0.000000in}}{%
\pgfpathmoveto{\pgfqpoint{0.000000in}{0.000000in}}%
\pgfpathlineto{\pgfqpoint{0.000000in}{-0.048611in}}%
\pgfusepath{stroke,fill}%
}%
\begin{pgfscope}%
\pgfsys@transformshift{1.210684in}{0.515123in}%
\pgfsys@useobject{currentmarker}{}%
\end{pgfscope}%
\end{pgfscope}%
\begin{pgfscope}%
\definecolor{textcolor}{rgb}{0.000000,0.000000,0.000000}%
\pgfsetstrokecolor{textcolor}%
\pgfsetfillcolor{textcolor}%
\pgftext[x=1.210684in,y=0.417901in,,top]{\color{textcolor}\rmfamily\fontsize{10.000000}{12.000000}\selectfont \(\displaystyle 2\)}%
\end{pgfscope}%
\begin{pgfscope}%
\pgfsetbuttcap%
\pgfsetroundjoin%
\definecolor{currentfill}{rgb}{0.000000,0.000000,0.000000}%
\pgfsetfillcolor{currentfill}%
\pgfsetlinewidth{0.803000pt}%
\definecolor{currentstroke}{rgb}{0.000000,0.000000,0.000000}%
\pgfsetstrokecolor{currentstroke}%
\pgfsetdash{}{0pt}%
\pgfsys@defobject{currentmarker}{\pgfqpoint{0.000000in}{-0.048611in}}{\pgfqpoint{0.000000in}{0.000000in}}{%
\pgfpathmoveto{\pgfqpoint{0.000000in}{0.000000in}}%
\pgfpathlineto{\pgfqpoint{0.000000in}{-0.048611in}}%
\pgfusepath{stroke,fill}%
}%
\begin{pgfscope}%
\pgfsys@transformshift{1.703833in}{0.515123in}%
\pgfsys@useobject{currentmarker}{}%
\end{pgfscope}%
\end{pgfscope}%
\begin{pgfscope}%
\definecolor{textcolor}{rgb}{0.000000,0.000000,0.000000}%
\pgfsetstrokecolor{textcolor}%
\pgfsetfillcolor{textcolor}%
\pgftext[x=1.703833in,y=0.417901in,,top]{\color{textcolor}\rmfamily\fontsize{10.000000}{12.000000}\selectfont \(\displaystyle 4\)}%
\end{pgfscope}%
\begin{pgfscope}%
\pgfsetbuttcap%
\pgfsetroundjoin%
\definecolor{currentfill}{rgb}{0.000000,0.000000,0.000000}%
\pgfsetfillcolor{currentfill}%
\pgfsetlinewidth{0.803000pt}%
\definecolor{currentstroke}{rgb}{0.000000,0.000000,0.000000}%
\pgfsetstrokecolor{currentstroke}%
\pgfsetdash{}{0pt}%
\pgfsys@defobject{currentmarker}{\pgfqpoint{0.000000in}{-0.048611in}}{\pgfqpoint{0.000000in}{0.000000in}}{%
\pgfpathmoveto{\pgfqpoint{0.000000in}{0.000000in}}%
\pgfpathlineto{\pgfqpoint{0.000000in}{-0.048611in}}%
\pgfusepath{stroke,fill}%
}%
\begin{pgfscope}%
\pgfsys@transformshift{2.196981in}{0.515123in}%
\pgfsys@useobject{currentmarker}{}%
\end{pgfscope}%
\end{pgfscope}%
\begin{pgfscope}%
\definecolor{textcolor}{rgb}{0.000000,0.000000,0.000000}%
\pgfsetstrokecolor{textcolor}%
\pgfsetfillcolor{textcolor}%
\pgftext[x=2.196981in,y=0.417901in,,top]{\color{textcolor}\rmfamily\fontsize{10.000000}{12.000000}\selectfont \(\displaystyle 6\)}%
\end{pgfscope}%
\begin{pgfscope}%
\pgfsetbuttcap%
\pgfsetroundjoin%
\definecolor{currentfill}{rgb}{0.000000,0.000000,0.000000}%
\pgfsetfillcolor{currentfill}%
\pgfsetlinewidth{0.803000pt}%
\definecolor{currentstroke}{rgb}{0.000000,0.000000,0.000000}%
\pgfsetstrokecolor{currentstroke}%
\pgfsetdash{}{0pt}%
\pgfsys@defobject{currentmarker}{\pgfqpoint{0.000000in}{-0.048611in}}{\pgfqpoint{0.000000in}{0.000000in}}{%
\pgfpathmoveto{\pgfqpoint{0.000000in}{0.000000in}}%
\pgfpathlineto{\pgfqpoint{0.000000in}{-0.048611in}}%
\pgfusepath{stroke,fill}%
}%
\begin{pgfscope}%
\pgfsys@transformshift{2.690130in}{0.515123in}%
\pgfsys@useobject{currentmarker}{}%
\end{pgfscope}%
\end{pgfscope}%
\begin{pgfscope}%
\definecolor{textcolor}{rgb}{0.000000,0.000000,0.000000}%
\pgfsetstrokecolor{textcolor}%
\pgfsetfillcolor{textcolor}%
\pgftext[x=2.690130in,y=0.417901in,,top]{\color{textcolor}\rmfamily\fontsize{10.000000}{12.000000}\selectfont \(\displaystyle 8\)}%
\end{pgfscope}%
\begin{pgfscope}%
\definecolor{textcolor}{rgb}{0.000000,0.000000,0.000000}%
\pgfsetstrokecolor{textcolor}%
\pgfsetfillcolor{textcolor}%
\pgftext[x=1.740819in,y=0.238889in,,top]{\color{textcolor}\rmfamily\fontsize{10.000000}{12.000000}\selectfont I(\(\displaystyle \mu\)A)}%
\end{pgfscope}%
\begin{pgfscope}%
\pgfsetbuttcap%
\pgfsetroundjoin%
\definecolor{currentfill}{rgb}{0.000000,0.000000,0.000000}%
\pgfsetfillcolor{currentfill}%
\pgfsetlinewidth{0.803000pt}%
\definecolor{currentstroke}{rgb}{0.000000,0.000000,0.000000}%
\pgfsetstrokecolor{currentstroke}%
\pgfsetdash{}{0pt}%
\pgfsys@defobject{currentmarker}{\pgfqpoint{-0.048611in}{0.000000in}}{\pgfqpoint{0.000000in}{0.000000in}}{%
\pgfpathmoveto{\pgfqpoint{0.000000in}{0.000000in}}%
\pgfpathlineto{\pgfqpoint{-0.048611in}{0.000000in}}%
\pgfusepath{stroke,fill}%
}%
\begin{pgfscope}%
\pgfsys@transformshift{0.772069in}{0.593526in}%
\pgfsys@useobject{currentmarker}{}%
\end{pgfscope}%
\end{pgfscope}%
\begin{pgfscope}%
\definecolor{textcolor}{rgb}{0.000000,0.000000,0.000000}%
\pgfsetstrokecolor{textcolor}%
\pgfsetfillcolor{textcolor}%
\pgftext[x=0.605402in,y=0.545301in,left,base]{\color{textcolor}\rmfamily\fontsize{10.000000}{12.000000}\selectfont \(\displaystyle 0\)}%
\end{pgfscope}%
\begin{pgfscope}%
\pgfsetbuttcap%
\pgfsetroundjoin%
\definecolor{currentfill}{rgb}{0.000000,0.000000,0.000000}%
\pgfsetfillcolor{currentfill}%
\pgfsetlinewidth{0.803000pt}%
\definecolor{currentstroke}{rgb}{0.000000,0.000000,0.000000}%
\pgfsetstrokecolor{currentstroke}%
\pgfsetdash{}{0pt}%
\pgfsys@defobject{currentmarker}{\pgfqpoint{-0.048611in}{0.000000in}}{\pgfqpoint{0.000000in}{0.000000in}}{%
\pgfpathmoveto{\pgfqpoint{0.000000in}{0.000000in}}%
\pgfpathlineto{\pgfqpoint{-0.048611in}{0.000000in}}%
\pgfusepath{stroke,fill}%
}%
\begin{pgfscope}%
\pgfsys@transformshift{0.772069in}{1.177404in}%
\pgfsys@useobject{currentmarker}{}%
\end{pgfscope}%
\end{pgfscope}%
\begin{pgfscope}%
\definecolor{textcolor}{rgb}{0.000000,0.000000,0.000000}%
\pgfsetstrokecolor{textcolor}%
\pgfsetfillcolor{textcolor}%
\pgftext[x=0.605402in,y=1.129179in,left,base]{\color{textcolor}\rmfamily\fontsize{10.000000}{12.000000}\selectfont \(\displaystyle 5\)}%
\end{pgfscope}%
\begin{pgfscope}%
\pgfsetbuttcap%
\pgfsetroundjoin%
\definecolor{currentfill}{rgb}{0.000000,0.000000,0.000000}%
\pgfsetfillcolor{currentfill}%
\pgfsetlinewidth{0.803000pt}%
\definecolor{currentstroke}{rgb}{0.000000,0.000000,0.000000}%
\pgfsetstrokecolor{currentstroke}%
\pgfsetdash{}{0pt}%
\pgfsys@defobject{currentmarker}{\pgfqpoint{-0.048611in}{0.000000in}}{\pgfqpoint{0.000000in}{0.000000in}}{%
\pgfpathmoveto{\pgfqpoint{0.000000in}{0.000000in}}%
\pgfpathlineto{\pgfqpoint{-0.048611in}{0.000000in}}%
\pgfusepath{stroke,fill}%
}%
\begin{pgfscope}%
\pgfsys@transformshift{0.772069in}{1.761282in}%
\pgfsys@useobject{currentmarker}{}%
\end{pgfscope}%
\end{pgfscope}%
\begin{pgfscope}%
\definecolor{textcolor}{rgb}{0.000000,0.000000,0.000000}%
\pgfsetstrokecolor{textcolor}%
\pgfsetfillcolor{textcolor}%
\pgftext[x=0.535957in,y=1.713057in,left,base]{\color{textcolor}\rmfamily\fontsize{10.000000}{12.000000}\selectfont \(\displaystyle 10\)}%
\end{pgfscope}%
\begin{pgfscope}%
\definecolor{textcolor}{rgb}{0.000000,0.000000,0.000000}%
\pgfsetstrokecolor{textcolor}%
\pgfsetfillcolor{textcolor}%
\pgftext[x=0.258179in,y=1.188873in,,bottom]{\color{textcolor}\rmfamily\fontsize{10.000000}{12.000000}\selectfont V(V)}%
\end{pgfscope}%
\begin{pgfscope}%
\pgfpathrectangle{\pgfqpoint{0.772069in}{0.515123in}}{\pgfqpoint{1.937500in}{1.347500in}}%
\pgfusepath{clip}%
\pgfsetbuttcap%
\pgfsetroundjoin%
\definecolor{currentfill}{rgb}{0.121569,0.466667,0.705882}%
\pgfsetfillcolor{currentfill}%
\pgfsetlinewidth{1.003750pt}%
\definecolor{currentstroke}{rgb}{0.121569,0.466667,0.705882}%
\pgfsetstrokecolor{currentstroke}%
\pgfsetdash{}{0pt}%
\pgfpathmoveto{\pgfqpoint{0.890137in}{0.672372in}}%
\pgfpathcurveto{\pgfqpoint{0.901187in}{0.672372in}}{\pgfqpoint{0.911786in}{0.676762in}}{\pgfqpoint{0.919600in}{0.684576in}}%
\pgfpathcurveto{\pgfqpoint{0.927413in}{0.692389in}}{\pgfqpoint{0.931803in}{0.702988in}}{\pgfqpoint{0.931803in}{0.714039in}}%
\pgfpathcurveto{\pgfqpoint{0.931803in}{0.725089in}}{\pgfqpoint{0.927413in}{0.735688in}}{\pgfqpoint{0.919600in}{0.743501in}}%
\pgfpathcurveto{\pgfqpoint{0.911786in}{0.751315in}}{\pgfqpoint{0.901187in}{0.755705in}}{\pgfqpoint{0.890137in}{0.755705in}}%
\pgfpathcurveto{\pgfqpoint{0.879087in}{0.755705in}}{\pgfqpoint{0.868488in}{0.751315in}}{\pgfqpoint{0.860674in}{0.743501in}}%
\pgfpathcurveto{\pgfqpoint{0.852860in}{0.735688in}}{\pgfqpoint{0.848470in}{0.725089in}}{\pgfqpoint{0.848470in}{0.714039in}}%
\pgfpathcurveto{\pgfqpoint{0.848470in}{0.702988in}}{\pgfqpoint{0.852860in}{0.692389in}}{\pgfqpoint{0.860674in}{0.684576in}}%
\pgfpathcurveto{\pgfqpoint{0.868488in}{0.676762in}}{\pgfqpoint{0.879087in}{0.672372in}}{\pgfqpoint{0.890137in}{0.672372in}}%
\pgfpathclose%
\pgfusepath{stroke,fill}%
\end{pgfscope}%
\begin{pgfscope}%
\pgfpathrectangle{\pgfqpoint{0.772069in}{0.515123in}}{\pgfqpoint{1.937500in}{1.347500in}}%
\pgfusepath{clip}%
\pgfsetbuttcap%
\pgfsetroundjoin%
\definecolor{currentfill}{rgb}{0.121569,0.466667,0.705882}%
\pgfsetfillcolor{currentfill}%
\pgfsetlinewidth{1.003750pt}%
\definecolor{currentstroke}{rgb}{0.121569,0.466667,0.705882}%
\pgfsetstrokecolor{currentstroke}%
\pgfsetdash{}{0pt}%
\pgfpathmoveto{\pgfqpoint{1.087396in}{0.785411in}}%
\pgfpathcurveto{\pgfqpoint{1.098447in}{0.785411in}}{\pgfqpoint{1.109046in}{0.789801in}}{\pgfqpoint{1.116859in}{0.797615in}}%
\pgfpathcurveto{\pgfqpoint{1.124673in}{0.805428in}}{\pgfqpoint{1.129063in}{0.816027in}}{\pgfqpoint{1.129063in}{0.827077in}}%
\pgfpathcurveto{\pgfqpoint{1.129063in}{0.838128in}}{\pgfqpoint{1.124673in}{0.848727in}}{\pgfqpoint{1.116859in}{0.856540in}}%
\pgfpathcurveto{\pgfqpoint{1.109046in}{0.864354in}}{\pgfqpoint{1.098447in}{0.868744in}}{\pgfqpoint{1.087396in}{0.868744in}}%
\pgfpathcurveto{\pgfqpoint{1.076346in}{0.868744in}}{\pgfqpoint{1.065747in}{0.864354in}}{\pgfqpoint{1.057934in}{0.856540in}}%
\pgfpathcurveto{\pgfqpoint{1.050120in}{0.848727in}}{\pgfqpoint{1.045730in}{0.838128in}}{\pgfqpoint{1.045730in}{0.827077in}}%
\pgfpathcurveto{\pgfqpoint{1.045730in}{0.816027in}}{\pgfqpoint{1.050120in}{0.805428in}}{\pgfqpoint{1.057934in}{0.797615in}}%
\pgfpathcurveto{\pgfqpoint{1.065747in}{0.789801in}}{\pgfqpoint{1.076346in}{0.785411in}}{\pgfqpoint{1.087396in}{0.785411in}}%
\pgfpathclose%
\pgfusepath{stroke,fill}%
\end{pgfscope}%
\begin{pgfscope}%
\pgfpathrectangle{\pgfqpoint{0.772069in}{0.515123in}}{\pgfqpoint{1.937500in}{1.347500in}}%
\pgfusepath{clip}%
\pgfsetbuttcap%
\pgfsetroundjoin%
\definecolor{currentfill}{rgb}{0.121569,0.466667,0.705882}%
\pgfsetfillcolor{currentfill}%
\pgfsetlinewidth{1.003750pt}%
\definecolor{currentstroke}{rgb}{0.121569,0.466667,0.705882}%
\pgfsetstrokecolor{currentstroke}%
\pgfsetdash{}{0pt}%
\pgfpathmoveto{\pgfqpoint{1.259999in}{0.906857in}}%
\pgfpathcurveto{\pgfqpoint{1.271049in}{0.906857in}}{\pgfqpoint{1.281648in}{0.911248in}}{\pgfqpoint{1.289461in}{0.919061in}}%
\pgfpathcurveto{\pgfqpoint{1.297275in}{0.926875in}}{\pgfqpoint{1.301665in}{0.937474in}}{\pgfqpoint{1.301665in}{0.948524in}}%
\pgfpathcurveto{\pgfqpoint{1.301665in}{0.959574in}}{\pgfqpoint{1.297275in}{0.970173in}}{\pgfqpoint{1.289461in}{0.977987in}}%
\pgfpathcurveto{\pgfqpoint{1.281648in}{0.985800in}}{\pgfqpoint{1.271049in}{0.990191in}}{\pgfqpoint{1.259999in}{0.990191in}}%
\pgfpathcurveto{\pgfqpoint{1.248948in}{0.990191in}}{\pgfqpoint{1.238349in}{0.985800in}}{\pgfqpoint{1.230536in}{0.977987in}}%
\pgfpathcurveto{\pgfqpoint{1.222722in}{0.970173in}}{\pgfqpoint{1.218332in}{0.959574in}}{\pgfqpoint{1.218332in}{0.948524in}}%
\pgfpathcurveto{\pgfqpoint{1.218332in}{0.937474in}}{\pgfqpoint{1.222722in}{0.926875in}}{\pgfqpoint{1.230536in}{0.919061in}}%
\pgfpathcurveto{\pgfqpoint{1.238349in}{0.911248in}}{\pgfqpoint{1.248948in}{0.906857in}}{\pgfqpoint{1.259999in}{0.906857in}}%
\pgfpathclose%
\pgfusepath{stroke,fill}%
\end{pgfscope}%
\begin{pgfscope}%
\pgfpathrectangle{\pgfqpoint{0.772069in}{0.515123in}}{\pgfqpoint{1.937500in}{1.347500in}}%
\pgfusepath{clip}%
\pgfsetbuttcap%
\pgfsetroundjoin%
\definecolor{currentfill}{rgb}{0.121569,0.466667,0.705882}%
\pgfsetfillcolor{currentfill}%
\pgfsetlinewidth{1.003750pt}%
\definecolor{currentstroke}{rgb}{0.121569,0.466667,0.705882}%
\pgfsetstrokecolor{currentstroke}%
\pgfsetdash{}{0pt}%
\pgfpathmoveto{\pgfqpoint{1.457258in}{1.025969in}}%
\pgfpathcurveto{\pgfqpoint{1.468308in}{1.025969in}}{\pgfqpoint{1.478907in}{1.030359in}}{\pgfqpoint{1.486721in}{1.038172in}}%
\pgfpathcurveto{\pgfqpoint{1.494534in}{1.045986in}}{\pgfqpoint{1.498925in}{1.056585in}}{\pgfqpoint{1.498925in}{1.067635in}}%
\pgfpathcurveto{\pgfqpoint{1.498925in}{1.078685in}}{\pgfqpoint{1.494534in}{1.089284in}}{\pgfqpoint{1.486721in}{1.097098in}}%
\pgfpathcurveto{\pgfqpoint{1.478907in}{1.104912in}}{\pgfqpoint{1.468308in}{1.109302in}}{\pgfqpoint{1.457258in}{1.109302in}}%
\pgfpathcurveto{\pgfqpoint{1.446208in}{1.109302in}}{\pgfqpoint{1.435609in}{1.104912in}}{\pgfqpoint{1.427795in}{1.097098in}}%
\pgfpathcurveto{\pgfqpoint{1.419982in}{1.089284in}}{\pgfqpoint{1.415591in}{1.078685in}}{\pgfqpoint{1.415591in}{1.067635in}}%
\pgfpathcurveto{\pgfqpoint{1.415591in}{1.056585in}}{\pgfqpoint{1.419982in}{1.045986in}}{\pgfqpoint{1.427795in}{1.038172in}}%
\pgfpathcurveto{\pgfqpoint{1.435609in}{1.030359in}}{\pgfqpoint{1.446208in}{1.025969in}}{\pgfqpoint{1.457258in}{1.025969in}}%
\pgfpathclose%
\pgfusepath{stroke,fill}%
\end{pgfscope}%
\begin{pgfscope}%
\pgfpathrectangle{\pgfqpoint{0.772069in}{0.515123in}}{\pgfqpoint{1.937500in}{1.347500in}}%
\pgfusepath{clip}%
\pgfsetbuttcap%
\pgfsetroundjoin%
\definecolor{currentfill}{rgb}{0.121569,0.466667,0.705882}%
\pgfsetfillcolor{currentfill}%
\pgfsetlinewidth{1.003750pt}%
\definecolor{currentstroke}{rgb}{0.121569,0.466667,0.705882}%
\pgfsetstrokecolor{currentstroke}%
\pgfsetdash{}{0pt}%
\pgfpathmoveto{\pgfqpoint{1.654518in}{1.145080in}}%
\pgfpathcurveto{\pgfqpoint{1.665568in}{1.145080in}}{\pgfqpoint{1.676167in}{1.149470in}}{\pgfqpoint{1.683980in}{1.157284in}}%
\pgfpathcurveto{\pgfqpoint{1.691794in}{1.165097in}}{\pgfqpoint{1.696184in}{1.175696in}}{\pgfqpoint{1.696184in}{1.186746in}}%
\pgfpathcurveto{\pgfqpoint{1.696184in}{1.197796in}}{\pgfqpoint{1.691794in}{1.208396in}}{\pgfqpoint{1.683980in}{1.216209in}}%
\pgfpathcurveto{\pgfqpoint{1.676167in}{1.224023in}}{\pgfqpoint{1.665568in}{1.228413in}}{\pgfqpoint{1.654518in}{1.228413in}}%
\pgfpathcurveto{\pgfqpoint{1.643467in}{1.228413in}}{\pgfqpoint{1.632868in}{1.224023in}}{\pgfqpoint{1.625055in}{1.216209in}}%
\pgfpathcurveto{\pgfqpoint{1.617241in}{1.208396in}}{\pgfqpoint{1.612851in}{1.197796in}}{\pgfqpoint{1.612851in}{1.186746in}}%
\pgfpathcurveto{\pgfqpoint{1.612851in}{1.175696in}}{\pgfqpoint{1.617241in}{1.165097in}}{\pgfqpoint{1.625055in}{1.157284in}}%
\pgfpathcurveto{\pgfqpoint{1.632868in}{1.149470in}}{\pgfqpoint{1.643467in}{1.145080in}}{\pgfqpoint{1.654518in}{1.145080in}}%
\pgfpathclose%
\pgfusepath{stroke,fill}%
\end{pgfscope}%
\begin{pgfscope}%
\pgfpathrectangle{\pgfqpoint{0.772069in}{0.515123in}}{\pgfqpoint{1.937500in}{1.347500in}}%
\pgfusepath{clip}%
\pgfsetbuttcap%
\pgfsetroundjoin%
\definecolor{currentfill}{rgb}{0.121569,0.466667,0.705882}%
\pgfsetfillcolor{currentfill}%
\pgfsetlinewidth{1.003750pt}%
\definecolor{currentstroke}{rgb}{0.121569,0.466667,0.705882}%
\pgfsetstrokecolor{currentstroke}%
\pgfsetdash{}{0pt}%
\pgfpathmoveto{\pgfqpoint{1.827120in}{1.258352in}}%
\pgfpathcurveto{\pgfqpoint{1.838170in}{1.258352in}}{\pgfqpoint{1.848769in}{1.262742in}}{\pgfqpoint{1.856583in}{1.270556in}}%
\pgfpathcurveto{\pgfqpoint{1.864396in}{1.278370in}}{\pgfqpoint{1.868786in}{1.288969in}}{\pgfqpoint{1.868786in}{1.300019in}}%
\pgfpathcurveto{\pgfqpoint{1.868786in}{1.311069in}}{\pgfqpoint{1.864396in}{1.321668in}}{\pgfqpoint{1.856583in}{1.329481in}}%
\pgfpathcurveto{\pgfqpoint{1.848769in}{1.337295in}}{\pgfqpoint{1.838170in}{1.341685in}}{\pgfqpoint{1.827120in}{1.341685in}}%
\pgfpathcurveto{\pgfqpoint{1.816070in}{1.341685in}}{\pgfqpoint{1.805471in}{1.337295in}}{\pgfqpoint{1.797657in}{1.329481in}}%
\pgfpathcurveto{\pgfqpoint{1.789843in}{1.321668in}}{\pgfqpoint{1.785453in}{1.311069in}}{\pgfqpoint{1.785453in}{1.300019in}}%
\pgfpathcurveto{\pgfqpoint{1.785453in}{1.288969in}}{\pgfqpoint{1.789843in}{1.278370in}}{\pgfqpoint{1.797657in}{1.270556in}}%
\pgfpathcurveto{\pgfqpoint{1.805471in}{1.262742in}}{\pgfqpoint{1.816070in}{1.258352in}}{\pgfqpoint{1.827120in}{1.258352in}}%
\pgfpathclose%
\pgfusepath{stroke,fill}%
\end{pgfscope}%
\begin{pgfscope}%
\pgfpathrectangle{\pgfqpoint{0.772069in}{0.515123in}}{\pgfqpoint{1.937500in}{1.347500in}}%
\pgfusepath{clip}%
\pgfsetbuttcap%
\pgfsetroundjoin%
\definecolor{currentfill}{rgb}{0.121569,0.466667,0.705882}%
\pgfsetfillcolor{currentfill}%
\pgfsetlinewidth{1.003750pt}%
\definecolor{currentstroke}{rgb}{0.121569,0.466667,0.705882}%
\pgfsetstrokecolor{currentstroke}%
\pgfsetdash{}{0pt}%
\pgfpathmoveto{\pgfqpoint{2.024379in}{1.380966in}}%
\pgfpathcurveto{\pgfqpoint{2.035429in}{1.380966in}}{\pgfqpoint{2.046028in}{1.385357in}}{\pgfqpoint{2.053842in}{1.393170in}}%
\pgfpathcurveto{\pgfqpoint{2.061656in}{1.400984in}}{\pgfqpoint{2.066046in}{1.411583in}}{\pgfqpoint{2.066046in}{1.422633in}}%
\pgfpathcurveto{\pgfqpoint{2.066046in}{1.433683in}}{\pgfqpoint{2.061656in}{1.444282in}}{\pgfqpoint{2.053842in}{1.452096in}}%
\pgfpathcurveto{\pgfqpoint{2.046028in}{1.459910in}}{\pgfqpoint{2.035429in}{1.464300in}}{\pgfqpoint{2.024379in}{1.464300in}}%
\pgfpathcurveto{\pgfqpoint{2.013329in}{1.464300in}}{\pgfqpoint{2.002730in}{1.459910in}}{\pgfqpoint{1.994917in}{1.452096in}}%
\pgfpathcurveto{\pgfqpoint{1.987103in}{1.444282in}}{\pgfqpoint{1.982713in}{1.433683in}}{\pgfqpoint{1.982713in}{1.422633in}}%
\pgfpathcurveto{\pgfqpoint{1.982713in}{1.411583in}}{\pgfqpoint{1.987103in}{1.400984in}}{\pgfqpoint{1.994917in}{1.393170in}}%
\pgfpathcurveto{\pgfqpoint{2.002730in}{1.385357in}}{\pgfqpoint{2.013329in}{1.380966in}}{\pgfqpoint{2.024379in}{1.380966in}}%
\pgfpathclose%
\pgfusepath{stroke,fill}%
\end{pgfscope}%
\begin{pgfscope}%
\pgfpathrectangle{\pgfqpoint{0.772069in}{0.515123in}}{\pgfqpoint{1.937500in}{1.347500in}}%
\pgfusepath{clip}%
\pgfsetbuttcap%
\pgfsetroundjoin%
\definecolor{currentfill}{rgb}{0.121569,0.466667,0.705882}%
\pgfsetfillcolor{currentfill}%
\pgfsetlinewidth{1.003750pt}%
\definecolor{currentstroke}{rgb}{0.121569,0.466667,0.705882}%
\pgfsetstrokecolor{currentstroke}%
\pgfsetdash{}{0pt}%
\pgfpathmoveto{\pgfqpoint{2.221639in}{1.498910in}}%
\pgfpathcurveto{\pgfqpoint{2.232689in}{1.498910in}}{\pgfqpoint{2.243288in}{1.503300in}}{\pgfqpoint{2.251102in}{1.511114in}}%
\pgfpathcurveto{\pgfqpoint{2.258915in}{1.518927in}}{\pgfqpoint{2.263306in}{1.529526in}}{\pgfqpoint{2.263306in}{1.540576in}}%
\pgfpathcurveto{\pgfqpoint{2.263306in}{1.551627in}}{\pgfqpoint{2.258915in}{1.562226in}}{\pgfqpoint{2.251102in}{1.570039in}}%
\pgfpathcurveto{\pgfqpoint{2.243288in}{1.577853in}}{\pgfqpoint{2.232689in}{1.582243in}}{\pgfqpoint{2.221639in}{1.582243in}}%
\pgfpathcurveto{\pgfqpoint{2.210589in}{1.582243in}}{\pgfqpoint{2.199990in}{1.577853in}}{\pgfqpoint{2.192176in}{1.570039in}}%
\pgfpathcurveto{\pgfqpoint{2.184362in}{1.562226in}}{\pgfqpoint{2.179972in}{1.551627in}}{\pgfqpoint{2.179972in}{1.540576in}}%
\pgfpathcurveto{\pgfqpoint{2.179972in}{1.529526in}}{\pgfqpoint{2.184362in}{1.518927in}}{\pgfqpoint{2.192176in}{1.511114in}}%
\pgfpathcurveto{\pgfqpoint{2.199990in}{1.503300in}}{\pgfqpoint{2.210589in}{1.498910in}}{\pgfqpoint{2.221639in}{1.498910in}}%
\pgfpathclose%
\pgfusepath{stroke,fill}%
\end{pgfscope}%
\begin{pgfscope}%
\pgfpathrectangle{\pgfqpoint{0.772069in}{0.515123in}}{\pgfqpoint{1.937500in}{1.347500in}}%
\pgfusepath{clip}%
\pgfsetbuttcap%
\pgfsetroundjoin%
\definecolor{currentfill}{rgb}{0.121569,0.466667,0.705882}%
\pgfsetfillcolor{currentfill}%
\pgfsetlinewidth{1.003750pt}%
\definecolor{currentstroke}{rgb}{0.121569,0.466667,0.705882}%
\pgfsetstrokecolor{currentstroke}%
\pgfsetdash{}{0pt}%
\pgfpathmoveto{\pgfqpoint{2.394241in}{1.614518in}}%
\pgfpathcurveto{\pgfqpoint{2.405291in}{1.614518in}}{\pgfqpoint{2.415890in}{1.618908in}}{\pgfqpoint{2.423704in}{1.626722in}}%
\pgfpathcurveto{\pgfqpoint{2.431517in}{1.634535in}}{\pgfqpoint{2.435908in}{1.645134in}}{\pgfqpoint{2.435908in}{1.656184in}}%
\pgfpathcurveto{\pgfqpoint{2.435908in}{1.667234in}}{\pgfqpoint{2.431517in}{1.677834in}}{\pgfqpoint{2.423704in}{1.685647in}}%
\pgfpathcurveto{\pgfqpoint{2.415890in}{1.693461in}}{\pgfqpoint{2.405291in}{1.697851in}}{\pgfqpoint{2.394241in}{1.697851in}}%
\pgfpathcurveto{\pgfqpoint{2.383191in}{1.697851in}}{\pgfqpoint{2.372592in}{1.693461in}}{\pgfqpoint{2.364778in}{1.685647in}}%
\pgfpathcurveto{\pgfqpoint{2.356965in}{1.677834in}}{\pgfqpoint{2.352574in}{1.667234in}}{\pgfqpoint{2.352574in}{1.656184in}}%
\pgfpathcurveto{\pgfqpoint{2.352574in}{1.645134in}}{\pgfqpoint{2.356965in}{1.634535in}}{\pgfqpoint{2.364778in}{1.626722in}}%
\pgfpathcurveto{\pgfqpoint{2.372592in}{1.618908in}}{\pgfqpoint{2.383191in}{1.614518in}}{\pgfqpoint{2.394241in}{1.614518in}}%
\pgfpathclose%
\pgfusepath{stroke,fill}%
\end{pgfscope}%
\begin{pgfscope}%
\pgfpathrectangle{\pgfqpoint{0.772069in}{0.515123in}}{\pgfqpoint{1.937500in}{1.347500in}}%
\pgfusepath{clip}%
\pgfsetbuttcap%
\pgfsetroundjoin%
\definecolor{currentfill}{rgb}{0.121569,0.466667,0.705882}%
\pgfsetfillcolor{currentfill}%
\pgfsetlinewidth{1.003750pt}%
\definecolor{currentstroke}{rgb}{0.121569,0.466667,0.705882}%
\pgfsetstrokecolor{currentstroke}%
\pgfsetdash{}{0pt}%
\pgfpathmoveto{\pgfqpoint{2.591501in}{1.733629in}}%
\pgfpathcurveto{\pgfqpoint{2.602551in}{1.733629in}}{\pgfqpoint{2.613150in}{1.738019in}}{\pgfqpoint{2.620963in}{1.745833in}}%
\pgfpathcurveto{\pgfqpoint{2.628777in}{1.753646in}}{\pgfqpoint{2.633167in}{1.764245in}}{\pgfqpoint{2.633167in}{1.775295in}}%
\pgfpathcurveto{\pgfqpoint{2.633167in}{1.786346in}}{\pgfqpoint{2.628777in}{1.796945in}}{\pgfqpoint{2.620963in}{1.804758in}}%
\pgfpathcurveto{\pgfqpoint{2.613150in}{1.812572in}}{\pgfqpoint{2.602551in}{1.816962in}}{\pgfqpoint{2.591501in}{1.816962in}}%
\pgfpathcurveto{\pgfqpoint{2.580450in}{1.816962in}}{\pgfqpoint{2.569851in}{1.812572in}}{\pgfqpoint{2.562038in}{1.804758in}}%
\pgfpathcurveto{\pgfqpoint{2.554224in}{1.796945in}}{\pgfqpoint{2.549834in}{1.786346in}}{\pgfqpoint{2.549834in}{1.775295in}}%
\pgfpathcurveto{\pgfqpoint{2.549834in}{1.764245in}}{\pgfqpoint{2.554224in}{1.753646in}}{\pgfqpoint{2.562038in}{1.745833in}}%
\pgfpathcurveto{\pgfqpoint{2.569851in}{1.738019in}}{\pgfqpoint{2.580450in}{1.733629in}}{\pgfqpoint{2.591501in}{1.733629in}}%
\pgfpathclose%
\pgfusepath{stroke,fill}%
\end{pgfscope}%
\begin{pgfscope}%
\pgfpathrectangle{\pgfqpoint{0.772069in}{0.515123in}}{\pgfqpoint{1.937500in}{1.347500in}}%
\pgfusepath{clip}%
\pgfsetbuttcap%
\pgfsetroundjoin%
\definecolor{currentfill}{rgb}{1.000000,0.388235,0.278431}%
\pgfsetfillcolor{currentfill}%
\pgfsetlinewidth{1.003750pt}%
\definecolor{currentstroke}{rgb}{1.000000,0.388235,0.278431}%
\pgfsetstrokecolor{currentstroke}%
\pgfsetdash{}{0pt}%
\pgfpathmoveto{\pgfqpoint{0.890137in}{0.567823in}}%
\pgfpathcurveto{\pgfqpoint{0.901187in}{0.567823in}}{\pgfqpoint{0.911786in}{0.572213in}}{\pgfqpoint{0.919600in}{0.580027in}}%
\pgfpathcurveto{\pgfqpoint{0.927413in}{0.587840in}}{\pgfqpoint{0.931803in}{0.598439in}}{\pgfqpoint{0.931803in}{0.609489in}}%
\pgfpathcurveto{\pgfqpoint{0.931803in}{0.620540in}}{\pgfqpoint{0.927413in}{0.631139in}}{\pgfqpoint{0.919600in}{0.638952in}}%
\pgfpathcurveto{\pgfqpoint{0.911786in}{0.646766in}}{\pgfqpoint{0.901187in}{0.651156in}}{\pgfqpoint{0.890137in}{0.651156in}}%
\pgfpathcurveto{\pgfqpoint{0.879087in}{0.651156in}}{\pgfqpoint{0.868488in}{0.646766in}}{\pgfqpoint{0.860674in}{0.638952in}}%
\pgfpathcurveto{\pgfqpoint{0.852860in}{0.631139in}}{\pgfqpoint{0.848470in}{0.620540in}}{\pgfqpoint{0.848470in}{0.609489in}}%
\pgfpathcurveto{\pgfqpoint{0.848470in}{0.598439in}}{\pgfqpoint{0.852860in}{0.587840in}}{\pgfqpoint{0.860674in}{0.580027in}}%
\pgfpathcurveto{\pgfqpoint{0.868488in}{0.572213in}}{\pgfqpoint{0.879087in}{0.567823in}}{\pgfqpoint{0.890137in}{0.567823in}}%
\pgfpathclose%
\pgfusepath{stroke,fill}%
\end{pgfscope}%
\begin{pgfscope}%
\pgfpathrectangle{\pgfqpoint{0.772069in}{0.515123in}}{\pgfqpoint{1.937500in}{1.347500in}}%
\pgfusepath{clip}%
\pgfsetbuttcap%
\pgfsetroundjoin%
\definecolor{currentfill}{rgb}{1.000000,0.388235,0.278431}%
\pgfsetfillcolor{currentfill}%
\pgfsetlinewidth{1.003750pt}%
\definecolor{currentstroke}{rgb}{1.000000,0.388235,0.278431}%
\pgfsetstrokecolor{currentstroke}%
\pgfsetdash{}{0pt}%
\pgfpathmoveto{\pgfqpoint{1.087396in}{0.582338in}}%
\pgfpathcurveto{\pgfqpoint{1.098447in}{0.582338in}}{\pgfqpoint{1.109046in}{0.586728in}}{\pgfqpoint{1.116859in}{0.594542in}}%
\pgfpathcurveto{\pgfqpoint{1.124673in}{0.602355in}}{\pgfqpoint{1.129063in}{0.612954in}}{\pgfqpoint{1.129063in}{0.624005in}}%
\pgfpathcurveto{\pgfqpoint{1.129063in}{0.635055in}}{\pgfqpoint{1.124673in}{0.645654in}}{\pgfqpoint{1.116859in}{0.653467in}}%
\pgfpathcurveto{\pgfqpoint{1.109046in}{0.661281in}}{\pgfqpoint{1.098447in}{0.665671in}}{\pgfqpoint{1.087396in}{0.665671in}}%
\pgfpathcurveto{\pgfqpoint{1.076346in}{0.665671in}}{\pgfqpoint{1.065747in}{0.661281in}}{\pgfqpoint{1.057934in}{0.653467in}}%
\pgfpathcurveto{\pgfqpoint{1.050120in}{0.645654in}}{\pgfqpoint{1.045730in}{0.635055in}}{\pgfqpoint{1.045730in}{0.624005in}}%
\pgfpathcurveto{\pgfqpoint{1.045730in}{0.612954in}}{\pgfqpoint{1.050120in}{0.602355in}}{\pgfqpoint{1.057934in}{0.594542in}}%
\pgfpathcurveto{\pgfqpoint{1.065747in}{0.586728in}}{\pgfqpoint{1.076346in}{0.582338in}}{\pgfqpoint{1.087396in}{0.582338in}}%
\pgfpathclose%
\pgfusepath{stroke,fill}%
\end{pgfscope}%
\begin{pgfscope}%
\pgfpathrectangle{\pgfqpoint{0.772069in}{0.515123in}}{\pgfqpoint{1.937500in}{1.347500in}}%
\pgfusepath{clip}%
\pgfsetbuttcap%
\pgfsetroundjoin%
\definecolor{currentfill}{rgb}{1.000000,0.388235,0.278431}%
\pgfsetfillcolor{currentfill}%
\pgfsetlinewidth{1.003750pt}%
\definecolor{currentstroke}{rgb}{1.000000,0.388235,0.278431}%
\pgfsetstrokecolor{currentstroke}%
\pgfsetdash{}{0pt}%
\pgfpathmoveto{\pgfqpoint{1.259999in}{0.586542in}}%
\pgfpathcurveto{\pgfqpoint{1.271049in}{0.586542in}}{\pgfqpoint{1.281648in}{0.590932in}}{\pgfqpoint{1.289461in}{0.598746in}}%
\pgfpathcurveto{\pgfqpoint{1.297275in}{0.606559in}}{\pgfqpoint{1.301665in}{0.617158in}}{\pgfqpoint{1.301665in}{0.628209in}}%
\pgfpathcurveto{\pgfqpoint{1.301665in}{0.639259in}}{\pgfqpoint{1.297275in}{0.649858in}}{\pgfqpoint{1.289461in}{0.657671in}}%
\pgfpathcurveto{\pgfqpoint{1.281648in}{0.665485in}}{\pgfqpoint{1.271049in}{0.669875in}}{\pgfqpoint{1.259999in}{0.669875in}}%
\pgfpathcurveto{\pgfqpoint{1.248948in}{0.669875in}}{\pgfqpoint{1.238349in}{0.665485in}}{\pgfqpoint{1.230536in}{0.657671in}}%
\pgfpathcurveto{\pgfqpoint{1.222722in}{0.649858in}}{\pgfqpoint{1.218332in}{0.639259in}}{\pgfqpoint{1.218332in}{0.628209in}}%
\pgfpathcurveto{\pgfqpoint{1.218332in}{0.617158in}}{\pgfqpoint{1.222722in}{0.606559in}}{\pgfqpoint{1.230536in}{0.598746in}}%
\pgfpathcurveto{\pgfqpoint{1.238349in}{0.590932in}}{\pgfqpoint{1.248948in}{0.586542in}}{\pgfqpoint{1.259999in}{0.586542in}}%
\pgfpathclose%
\pgfusepath{stroke,fill}%
\end{pgfscope}%
\begin{pgfscope}%
\pgfpathrectangle{\pgfqpoint{0.772069in}{0.515123in}}{\pgfqpoint{1.937500in}{1.347500in}}%
\pgfusepath{clip}%
\pgfsetbuttcap%
\pgfsetroundjoin%
\definecolor{currentfill}{rgb}{1.000000,0.388235,0.278431}%
\pgfsetfillcolor{currentfill}%
\pgfsetlinewidth{1.003750pt}%
\definecolor{currentstroke}{rgb}{1.000000,0.388235,0.278431}%
\pgfsetstrokecolor{currentstroke}%
\pgfsetdash{}{0pt}%
\pgfpathmoveto{\pgfqpoint{1.457258in}{0.613867in}}%
\pgfpathcurveto{\pgfqpoint{1.468308in}{0.613867in}}{\pgfqpoint{1.478907in}{0.618258in}}{\pgfqpoint{1.486721in}{0.626071in}}%
\pgfpathcurveto{\pgfqpoint{1.494534in}{0.633885in}}{\pgfqpoint{1.498925in}{0.644484in}}{\pgfqpoint{1.498925in}{0.655534in}}%
\pgfpathcurveto{\pgfqpoint{1.498925in}{0.666584in}}{\pgfqpoint{1.494534in}{0.677183in}}{\pgfqpoint{1.486721in}{0.684997in}}%
\pgfpathcurveto{\pgfqpoint{1.478907in}{0.692810in}}{\pgfqpoint{1.468308in}{0.697201in}}{\pgfqpoint{1.457258in}{0.697201in}}%
\pgfpathcurveto{\pgfqpoint{1.446208in}{0.697201in}}{\pgfqpoint{1.435609in}{0.692810in}}{\pgfqpoint{1.427795in}{0.684997in}}%
\pgfpathcurveto{\pgfqpoint{1.419982in}{0.677183in}}{\pgfqpoint{1.415591in}{0.666584in}}{\pgfqpoint{1.415591in}{0.655534in}}%
\pgfpathcurveto{\pgfqpoint{1.415591in}{0.644484in}}{\pgfqpoint{1.419982in}{0.633885in}}{\pgfqpoint{1.427795in}{0.626071in}}%
\pgfpathcurveto{\pgfqpoint{1.435609in}{0.618258in}}{\pgfqpoint{1.446208in}{0.613867in}}{\pgfqpoint{1.457258in}{0.613867in}}%
\pgfpathclose%
\pgfusepath{stroke,fill}%
\end{pgfscope}%
\begin{pgfscope}%
\pgfpathrectangle{\pgfqpoint{0.772069in}{0.515123in}}{\pgfqpoint{1.937500in}{1.347500in}}%
\pgfusepath{clip}%
\pgfsetbuttcap%
\pgfsetroundjoin%
\definecolor{currentfill}{rgb}{1.000000,0.388235,0.278431}%
\pgfsetfillcolor{currentfill}%
\pgfsetlinewidth{1.003750pt}%
\definecolor{currentstroke}{rgb}{1.000000,0.388235,0.278431}%
\pgfsetstrokecolor{currentstroke}%
\pgfsetdash{}{0pt}%
\pgfpathmoveto{\pgfqpoint{1.654518in}{0.629399in}}%
\pgfpathcurveto{\pgfqpoint{1.665568in}{0.629399in}}{\pgfqpoint{1.676167in}{0.633789in}}{\pgfqpoint{1.683980in}{0.641602in}}%
\pgfpathcurveto{\pgfqpoint{1.691794in}{0.649416in}}{\pgfqpoint{1.696184in}{0.660015in}}{\pgfqpoint{1.696184in}{0.671065in}}%
\pgfpathcurveto{\pgfqpoint{1.696184in}{0.682115in}}{\pgfqpoint{1.691794in}{0.692714in}}{\pgfqpoint{1.683980in}{0.700528in}}%
\pgfpathcurveto{\pgfqpoint{1.676167in}{0.708342in}}{\pgfqpoint{1.665568in}{0.712732in}}{\pgfqpoint{1.654518in}{0.712732in}}%
\pgfpathcurveto{\pgfqpoint{1.643467in}{0.712732in}}{\pgfqpoint{1.632868in}{0.708342in}}{\pgfqpoint{1.625055in}{0.700528in}}%
\pgfpathcurveto{\pgfqpoint{1.617241in}{0.692714in}}{\pgfqpoint{1.612851in}{0.682115in}}{\pgfqpoint{1.612851in}{0.671065in}}%
\pgfpathcurveto{\pgfqpoint{1.612851in}{0.660015in}}{\pgfqpoint{1.617241in}{0.649416in}}{\pgfqpoint{1.625055in}{0.641602in}}%
\pgfpathcurveto{\pgfqpoint{1.632868in}{0.633789in}}{\pgfqpoint{1.643467in}{0.629399in}}{\pgfqpoint{1.654518in}{0.629399in}}%
\pgfpathclose%
\pgfusepath{stroke,fill}%
\end{pgfscope}%
\begin{pgfscope}%
\pgfpathrectangle{\pgfqpoint{0.772069in}{0.515123in}}{\pgfqpoint{1.937500in}{1.347500in}}%
\pgfusepath{clip}%
\pgfsetbuttcap%
\pgfsetroundjoin%
\definecolor{currentfill}{rgb}{1.000000,0.388235,0.278431}%
\pgfsetfillcolor{currentfill}%
\pgfsetlinewidth{1.003750pt}%
\definecolor{currentstroke}{rgb}{1.000000,0.388235,0.278431}%
\pgfsetstrokecolor{currentstroke}%
\pgfsetdash{}{0pt}%
\pgfpathmoveto{\pgfqpoint{1.827120in}{0.644229in}}%
\pgfpathcurveto{\pgfqpoint{1.838170in}{0.644229in}}{\pgfqpoint{1.848769in}{0.648619in}}{\pgfqpoint{1.856583in}{0.656433in}}%
\pgfpathcurveto{\pgfqpoint{1.864396in}{0.664247in}}{\pgfqpoint{1.868786in}{0.674846in}}{\pgfqpoint{1.868786in}{0.685896in}}%
\pgfpathcurveto{\pgfqpoint{1.868786in}{0.696946in}}{\pgfqpoint{1.864396in}{0.707545in}}{\pgfqpoint{1.856583in}{0.715358in}}%
\pgfpathcurveto{\pgfqpoint{1.848769in}{0.723172in}}{\pgfqpoint{1.838170in}{0.727562in}}{\pgfqpoint{1.827120in}{0.727562in}}%
\pgfpathcurveto{\pgfqpoint{1.816070in}{0.727562in}}{\pgfqpoint{1.805471in}{0.723172in}}{\pgfqpoint{1.797657in}{0.715358in}}%
\pgfpathcurveto{\pgfqpoint{1.789843in}{0.707545in}}{\pgfqpoint{1.785453in}{0.696946in}}{\pgfqpoint{1.785453in}{0.685896in}}%
\pgfpathcurveto{\pgfqpoint{1.785453in}{0.674846in}}{\pgfqpoint{1.789843in}{0.664247in}}{\pgfqpoint{1.797657in}{0.656433in}}%
\pgfpathcurveto{\pgfqpoint{1.805471in}{0.648619in}}{\pgfqpoint{1.816070in}{0.644229in}}{\pgfqpoint{1.827120in}{0.644229in}}%
\pgfpathclose%
\pgfusepath{stroke,fill}%
\end{pgfscope}%
\begin{pgfscope}%
\pgfpathrectangle{\pgfqpoint{0.772069in}{0.515123in}}{\pgfqpoint{1.937500in}{1.347500in}}%
\pgfusepath{clip}%
\pgfsetbuttcap%
\pgfsetroundjoin%
\definecolor{currentfill}{rgb}{1.000000,0.388235,0.278431}%
\pgfsetfillcolor{currentfill}%
\pgfsetlinewidth{1.003750pt}%
\definecolor{currentstroke}{rgb}{1.000000,0.388235,0.278431}%
\pgfsetstrokecolor{currentstroke}%
\pgfsetdash{}{0pt}%
\pgfpathmoveto{\pgfqpoint{2.024379in}{0.660344in}}%
\pgfpathcurveto{\pgfqpoint{2.035429in}{0.660344in}}{\pgfqpoint{2.046028in}{0.664734in}}{\pgfqpoint{2.053842in}{0.672548in}}%
\pgfpathcurveto{\pgfqpoint{2.061656in}{0.680362in}}{\pgfqpoint{2.066046in}{0.690961in}}{\pgfqpoint{2.066046in}{0.702011in}}%
\pgfpathcurveto{\pgfqpoint{2.066046in}{0.713061in}}{\pgfqpoint{2.061656in}{0.723660in}}{\pgfqpoint{2.053842in}{0.731474in}}%
\pgfpathcurveto{\pgfqpoint{2.046028in}{0.739287in}}{\pgfqpoint{2.035429in}{0.743677in}}{\pgfqpoint{2.024379in}{0.743677in}}%
\pgfpathcurveto{\pgfqpoint{2.013329in}{0.743677in}}{\pgfqpoint{2.002730in}{0.739287in}}{\pgfqpoint{1.994917in}{0.731474in}}%
\pgfpathcurveto{\pgfqpoint{1.987103in}{0.723660in}}{\pgfqpoint{1.982713in}{0.713061in}}{\pgfqpoint{1.982713in}{0.702011in}}%
\pgfpathcurveto{\pgfqpoint{1.982713in}{0.690961in}}{\pgfqpoint{1.987103in}{0.680362in}}{\pgfqpoint{1.994917in}{0.672548in}}%
\pgfpathcurveto{\pgfqpoint{2.002730in}{0.664734in}}{\pgfqpoint{2.013329in}{0.660344in}}{\pgfqpoint{2.024379in}{0.660344in}}%
\pgfpathclose%
\pgfusepath{stroke,fill}%
\end{pgfscope}%
\begin{pgfscope}%
\pgfpathrectangle{\pgfqpoint{0.772069in}{0.515123in}}{\pgfqpoint{1.937500in}{1.347500in}}%
\pgfusepath{clip}%
\pgfsetbuttcap%
\pgfsetroundjoin%
\definecolor{currentfill}{rgb}{1.000000,0.388235,0.278431}%
\pgfsetfillcolor{currentfill}%
\pgfsetlinewidth{1.003750pt}%
\definecolor{currentstroke}{rgb}{1.000000,0.388235,0.278431}%
\pgfsetstrokecolor{currentstroke}%
\pgfsetdash{}{0pt}%
\pgfpathmoveto{\pgfqpoint{2.221639in}{0.675525in}}%
\pgfpathcurveto{\pgfqpoint{2.232689in}{0.675525in}}{\pgfqpoint{2.243288in}{0.679915in}}{\pgfqpoint{2.251102in}{0.687729in}}%
\pgfpathcurveto{\pgfqpoint{2.258915in}{0.695542in}}{\pgfqpoint{2.263306in}{0.706141in}}{\pgfqpoint{2.263306in}{0.717192in}}%
\pgfpathcurveto{\pgfqpoint{2.263306in}{0.728242in}}{\pgfqpoint{2.258915in}{0.738841in}}{\pgfqpoint{2.251102in}{0.746654in}}%
\pgfpathcurveto{\pgfqpoint{2.243288in}{0.754468in}}{\pgfqpoint{2.232689in}{0.758858in}}{\pgfqpoint{2.221639in}{0.758858in}}%
\pgfpathcurveto{\pgfqpoint{2.210589in}{0.758858in}}{\pgfqpoint{2.199990in}{0.754468in}}{\pgfqpoint{2.192176in}{0.746654in}}%
\pgfpathcurveto{\pgfqpoint{2.184362in}{0.738841in}}{\pgfqpoint{2.179972in}{0.728242in}}{\pgfqpoint{2.179972in}{0.717192in}}%
\pgfpathcurveto{\pgfqpoint{2.179972in}{0.706141in}}{\pgfqpoint{2.184362in}{0.695542in}}{\pgfqpoint{2.192176in}{0.687729in}}%
\pgfpathcurveto{\pgfqpoint{2.199990in}{0.679915in}}{\pgfqpoint{2.210589in}{0.675525in}}{\pgfqpoint{2.221639in}{0.675525in}}%
\pgfpathclose%
\pgfusepath{stroke,fill}%
\end{pgfscope}%
\begin{pgfscope}%
\pgfpathrectangle{\pgfqpoint{0.772069in}{0.515123in}}{\pgfqpoint{1.937500in}{1.347500in}}%
\pgfusepath{clip}%
\pgfsetbuttcap%
\pgfsetroundjoin%
\definecolor{currentfill}{rgb}{1.000000,0.388235,0.278431}%
\pgfsetfillcolor{currentfill}%
\pgfsetlinewidth{1.003750pt}%
\definecolor{currentstroke}{rgb}{1.000000,0.388235,0.278431}%
\pgfsetstrokecolor{currentstroke}%
\pgfsetdash{}{0pt}%
\pgfpathmoveto{\pgfqpoint{2.394241in}{0.690706in}}%
\pgfpathcurveto{\pgfqpoint{2.405291in}{0.690706in}}{\pgfqpoint{2.415890in}{0.695096in}}{\pgfqpoint{2.423704in}{0.702910in}}%
\pgfpathcurveto{\pgfqpoint{2.431517in}{0.710723in}}{\pgfqpoint{2.435908in}{0.721322in}}{\pgfqpoint{2.435908in}{0.732372in}}%
\pgfpathcurveto{\pgfqpoint{2.435908in}{0.743423in}}{\pgfqpoint{2.431517in}{0.754022in}}{\pgfqpoint{2.423704in}{0.761835in}}%
\pgfpathcurveto{\pgfqpoint{2.415890in}{0.769649in}}{\pgfqpoint{2.405291in}{0.774039in}}{\pgfqpoint{2.394241in}{0.774039in}}%
\pgfpathcurveto{\pgfqpoint{2.383191in}{0.774039in}}{\pgfqpoint{2.372592in}{0.769649in}}{\pgfqpoint{2.364778in}{0.761835in}}%
\pgfpathcurveto{\pgfqpoint{2.356965in}{0.754022in}}{\pgfqpoint{2.352574in}{0.743423in}}{\pgfqpoint{2.352574in}{0.732372in}}%
\pgfpathcurveto{\pgfqpoint{2.352574in}{0.721322in}}{\pgfqpoint{2.356965in}{0.710723in}}{\pgfqpoint{2.364778in}{0.702910in}}%
\pgfpathcurveto{\pgfqpoint{2.372592in}{0.695096in}}{\pgfqpoint{2.383191in}{0.690706in}}{\pgfqpoint{2.394241in}{0.690706in}}%
\pgfpathclose%
\pgfusepath{stroke,fill}%
\end{pgfscope}%
\begin{pgfscope}%
\pgfpathrectangle{\pgfqpoint{0.772069in}{0.515123in}}{\pgfqpoint{1.937500in}{1.347500in}}%
\pgfusepath{clip}%
\pgfsetbuttcap%
\pgfsetroundjoin%
\definecolor{currentfill}{rgb}{1.000000,0.388235,0.278431}%
\pgfsetfillcolor{currentfill}%
\pgfsetlinewidth{1.003750pt}%
\definecolor{currentstroke}{rgb}{1.000000,0.388235,0.278431}%
\pgfsetstrokecolor{currentstroke}%
\pgfsetdash{}{0pt}%
\pgfpathmoveto{\pgfqpoint{2.591501in}{0.706354in}}%
\pgfpathcurveto{\pgfqpoint{2.602551in}{0.706354in}}{\pgfqpoint{2.613150in}{0.710744in}}{\pgfqpoint{2.620963in}{0.718558in}}%
\pgfpathcurveto{\pgfqpoint{2.628777in}{0.726371in}}{\pgfqpoint{2.633167in}{0.736970in}}{\pgfqpoint{2.633167in}{0.748020in}}%
\pgfpathcurveto{\pgfqpoint{2.633167in}{0.759070in}}{\pgfqpoint{2.628777in}{0.769669in}}{\pgfqpoint{2.620963in}{0.777483in}}%
\pgfpathcurveto{\pgfqpoint{2.613150in}{0.785297in}}{\pgfqpoint{2.602551in}{0.789687in}}{\pgfqpoint{2.591501in}{0.789687in}}%
\pgfpathcurveto{\pgfqpoint{2.580450in}{0.789687in}}{\pgfqpoint{2.569851in}{0.785297in}}{\pgfqpoint{2.562038in}{0.777483in}}%
\pgfpathcurveto{\pgfqpoint{2.554224in}{0.769669in}}{\pgfqpoint{2.549834in}{0.759070in}}{\pgfqpoint{2.549834in}{0.748020in}}%
\pgfpathcurveto{\pgfqpoint{2.549834in}{0.736970in}}{\pgfqpoint{2.554224in}{0.726371in}}{\pgfqpoint{2.562038in}{0.718558in}}%
\pgfpathcurveto{\pgfqpoint{2.569851in}{0.710744in}}{\pgfqpoint{2.580450in}{0.706354in}}{\pgfqpoint{2.591501in}{0.706354in}}%
\pgfpathclose%
\pgfusepath{stroke,fill}%
\end{pgfscope}%
\begin{pgfscope}%
\pgfpathrectangle{\pgfqpoint{0.772069in}{0.515123in}}{\pgfqpoint{1.937500in}{1.347500in}}%
\pgfusepath{clip}%
\pgfsetbuttcap%
\pgfsetroundjoin%
\definecolor{currentfill}{rgb}{1.000000,0.843137,0.000000}%
\pgfsetfillcolor{currentfill}%
\pgfsetlinewidth{1.003750pt}%
\definecolor{currentstroke}{rgb}{1.000000,0.843137,0.000000}%
\pgfsetstrokecolor{currentstroke}%
\pgfsetdash{}{0pt}%
\pgfpathmoveto{\pgfqpoint{0.890137in}{0.564611in}}%
\pgfpathcurveto{\pgfqpoint{0.901187in}{0.564611in}}{\pgfqpoint{0.911786in}{0.569002in}}{\pgfqpoint{0.919600in}{0.576815in}}%
\pgfpathcurveto{\pgfqpoint{0.927413in}{0.584629in}}{\pgfqpoint{0.931803in}{0.595228in}}{\pgfqpoint{0.931803in}{0.606278in}}%
\pgfpathcurveto{\pgfqpoint{0.931803in}{0.617328in}}{\pgfqpoint{0.927413in}{0.627927in}}{\pgfqpoint{0.919600in}{0.635741in}}%
\pgfpathcurveto{\pgfqpoint{0.911786in}{0.643554in}}{\pgfqpoint{0.901187in}{0.647945in}}{\pgfqpoint{0.890137in}{0.647945in}}%
\pgfpathcurveto{\pgfqpoint{0.879087in}{0.647945in}}{\pgfqpoint{0.868488in}{0.643554in}}{\pgfqpoint{0.860674in}{0.635741in}}%
\pgfpathcurveto{\pgfqpoint{0.852860in}{0.627927in}}{\pgfqpoint{0.848470in}{0.617328in}}{\pgfqpoint{0.848470in}{0.606278in}}%
\pgfpathcurveto{\pgfqpoint{0.848470in}{0.595228in}}{\pgfqpoint{0.852860in}{0.584629in}}{\pgfqpoint{0.860674in}{0.576815in}}%
\pgfpathcurveto{\pgfqpoint{0.868488in}{0.569002in}}{\pgfqpoint{0.879087in}{0.564611in}}{\pgfqpoint{0.890137in}{0.564611in}}%
\pgfpathclose%
\pgfusepath{stroke,fill}%
\end{pgfscope}%
\begin{pgfscope}%
\pgfpathrectangle{\pgfqpoint{0.772069in}{0.515123in}}{\pgfqpoint{1.937500in}{1.347500in}}%
\pgfusepath{clip}%
\pgfsetbuttcap%
\pgfsetroundjoin%
\definecolor{currentfill}{rgb}{1.000000,0.843137,0.000000}%
\pgfsetfillcolor{currentfill}%
\pgfsetlinewidth{1.003750pt}%
\definecolor{currentstroke}{rgb}{1.000000,0.843137,0.000000}%
\pgfsetstrokecolor{currentstroke}%
\pgfsetdash{}{0pt}%
\pgfpathmoveto{\pgfqpoint{1.087396in}{0.576266in}}%
\pgfpathcurveto{\pgfqpoint{1.098447in}{0.576266in}}{\pgfqpoint{1.109046in}{0.580656in}}{\pgfqpoint{1.116859in}{0.588469in}}%
\pgfpathcurveto{\pgfqpoint{1.124673in}{0.596283in}}{\pgfqpoint{1.129063in}{0.606882in}}{\pgfqpoint{1.129063in}{0.617932in}}%
\pgfpathcurveto{\pgfqpoint{1.129063in}{0.628982in}}{\pgfqpoint{1.124673in}{0.639581in}}{\pgfqpoint{1.116859in}{0.647395in}}%
\pgfpathcurveto{\pgfqpoint{1.109046in}{0.655209in}}{\pgfqpoint{1.098447in}{0.659599in}}{\pgfqpoint{1.087396in}{0.659599in}}%
\pgfpathcurveto{\pgfqpoint{1.076346in}{0.659599in}}{\pgfqpoint{1.065747in}{0.655209in}}{\pgfqpoint{1.057934in}{0.647395in}}%
\pgfpathcurveto{\pgfqpoint{1.050120in}{0.639581in}}{\pgfqpoint{1.045730in}{0.628982in}}{\pgfqpoint{1.045730in}{0.617932in}}%
\pgfpathcurveto{\pgfqpoint{1.045730in}{0.606882in}}{\pgfqpoint{1.050120in}{0.596283in}}{\pgfqpoint{1.057934in}{0.588469in}}%
\pgfpathcurveto{\pgfqpoint{1.065747in}{0.580656in}}{\pgfqpoint{1.076346in}{0.576266in}}{\pgfqpoint{1.087396in}{0.576266in}}%
\pgfpathclose%
\pgfusepath{stroke,fill}%
\end{pgfscope}%
\begin{pgfscope}%
\pgfpathrectangle{\pgfqpoint{0.772069in}{0.515123in}}{\pgfqpoint{1.937500in}{1.347500in}}%
\pgfusepath{clip}%
\pgfsetbuttcap%
\pgfsetroundjoin%
\definecolor{currentfill}{rgb}{1.000000,0.843137,0.000000}%
\pgfsetfillcolor{currentfill}%
\pgfsetlinewidth{1.003750pt}%
\definecolor{currentstroke}{rgb}{1.000000,0.843137,0.000000}%
\pgfsetstrokecolor{currentstroke}%
\pgfsetdash{}{0pt}%
\pgfpathmoveto{\pgfqpoint{1.259999in}{0.588877in}}%
\pgfpathcurveto{\pgfqpoint{1.271049in}{0.588877in}}{\pgfqpoint{1.281648in}{0.593268in}}{\pgfqpoint{1.289461in}{0.601081in}}%
\pgfpathcurveto{\pgfqpoint{1.297275in}{0.608895in}}{\pgfqpoint{1.301665in}{0.619494in}}{\pgfqpoint{1.301665in}{0.630544in}}%
\pgfpathcurveto{\pgfqpoint{1.301665in}{0.641594in}}{\pgfqpoint{1.297275in}{0.652193in}}{\pgfqpoint{1.289461in}{0.660007in}}%
\pgfpathcurveto{\pgfqpoint{1.281648in}{0.667820in}}{\pgfqpoint{1.271049in}{0.672211in}}{\pgfqpoint{1.259999in}{0.672211in}}%
\pgfpathcurveto{\pgfqpoint{1.248948in}{0.672211in}}{\pgfqpoint{1.238349in}{0.667820in}}{\pgfqpoint{1.230536in}{0.660007in}}%
\pgfpathcurveto{\pgfqpoint{1.222722in}{0.652193in}}{\pgfqpoint{1.218332in}{0.641594in}}{\pgfqpoint{1.218332in}{0.630544in}}%
\pgfpathcurveto{\pgfqpoint{1.218332in}{0.619494in}}{\pgfqpoint{1.222722in}{0.608895in}}{\pgfqpoint{1.230536in}{0.601081in}}%
\pgfpathcurveto{\pgfqpoint{1.238349in}{0.593268in}}{\pgfqpoint{1.248948in}{0.588877in}}{\pgfqpoint{1.259999in}{0.588877in}}%
\pgfpathclose%
\pgfusepath{stroke,fill}%
\end{pgfscope}%
\begin{pgfscope}%
\pgfpathrectangle{\pgfqpoint{0.772069in}{0.515123in}}{\pgfqpoint{1.937500in}{1.347500in}}%
\pgfusepath{clip}%
\pgfsetbuttcap%
\pgfsetroundjoin%
\definecolor{currentfill}{rgb}{1.000000,0.843137,0.000000}%
\pgfsetfillcolor{currentfill}%
\pgfsetlinewidth{1.003750pt}%
\definecolor{currentstroke}{rgb}{1.000000,0.843137,0.000000}%
\pgfsetstrokecolor{currentstroke}%
\pgfsetdash{}{0pt}%
\pgfpathmoveto{\pgfqpoint{1.457258in}{0.601256in}}%
\pgfpathcurveto{\pgfqpoint{1.468308in}{0.601256in}}{\pgfqpoint{1.478907in}{0.605646in}}{\pgfqpoint{1.486721in}{0.613459in}}%
\pgfpathcurveto{\pgfqpoint{1.494534in}{0.621273in}}{\pgfqpoint{1.498925in}{0.631872in}}{\pgfqpoint{1.498925in}{0.642922in}}%
\pgfpathcurveto{\pgfqpoint{1.498925in}{0.653972in}}{\pgfqpoint{1.494534in}{0.664571in}}{\pgfqpoint{1.486721in}{0.672385in}}%
\pgfpathcurveto{\pgfqpoint{1.478907in}{0.680199in}}{\pgfqpoint{1.468308in}{0.684589in}}{\pgfqpoint{1.457258in}{0.684589in}}%
\pgfpathcurveto{\pgfqpoint{1.446208in}{0.684589in}}{\pgfqpoint{1.435609in}{0.680199in}}{\pgfqpoint{1.427795in}{0.672385in}}%
\pgfpathcurveto{\pgfqpoint{1.419982in}{0.664571in}}{\pgfqpoint{1.415591in}{0.653972in}}{\pgfqpoint{1.415591in}{0.642922in}}%
\pgfpathcurveto{\pgfqpoint{1.415591in}{0.631872in}}{\pgfqpoint{1.419982in}{0.621273in}}{\pgfqpoint{1.427795in}{0.613459in}}%
\pgfpathcurveto{\pgfqpoint{1.435609in}{0.605646in}}{\pgfqpoint{1.446208in}{0.601256in}}{\pgfqpoint{1.457258in}{0.601256in}}%
\pgfpathclose%
\pgfusepath{stroke,fill}%
\end{pgfscope}%
\begin{pgfscope}%
\pgfpathrectangle{\pgfqpoint{0.772069in}{0.515123in}}{\pgfqpoint{1.937500in}{1.347500in}}%
\pgfusepath{clip}%
\pgfsetbuttcap%
\pgfsetroundjoin%
\definecolor{currentfill}{rgb}{1.000000,0.843137,0.000000}%
\pgfsetfillcolor{currentfill}%
\pgfsetlinewidth{1.003750pt}%
\definecolor{currentstroke}{rgb}{1.000000,0.843137,0.000000}%
\pgfsetstrokecolor{currentstroke}%
\pgfsetdash{}{0pt}%
\pgfpathmoveto{\pgfqpoint{1.654518in}{0.613634in}}%
\pgfpathcurveto{\pgfqpoint{1.665568in}{0.613634in}}{\pgfqpoint{1.676167in}{0.618024in}}{\pgfqpoint{1.683980in}{0.625838in}}%
\pgfpathcurveto{\pgfqpoint{1.691794in}{0.633651in}}{\pgfqpoint{1.696184in}{0.644250in}}{\pgfqpoint{1.696184in}{0.655300in}}%
\pgfpathcurveto{\pgfqpoint{1.696184in}{0.666351in}}{\pgfqpoint{1.691794in}{0.676950in}}{\pgfqpoint{1.683980in}{0.684763in}}%
\pgfpathcurveto{\pgfqpoint{1.676167in}{0.692577in}}{\pgfqpoint{1.665568in}{0.696967in}}{\pgfqpoint{1.654518in}{0.696967in}}%
\pgfpathcurveto{\pgfqpoint{1.643467in}{0.696967in}}{\pgfqpoint{1.632868in}{0.692577in}}{\pgfqpoint{1.625055in}{0.684763in}}%
\pgfpathcurveto{\pgfqpoint{1.617241in}{0.676950in}}{\pgfqpoint{1.612851in}{0.666351in}}{\pgfqpoint{1.612851in}{0.655300in}}%
\pgfpathcurveto{\pgfqpoint{1.612851in}{0.644250in}}{\pgfqpoint{1.617241in}{0.633651in}}{\pgfqpoint{1.625055in}{0.625838in}}%
\pgfpathcurveto{\pgfqpoint{1.632868in}{0.618024in}}{\pgfqpoint{1.643467in}{0.613634in}}{\pgfqpoint{1.654518in}{0.613634in}}%
\pgfpathclose%
\pgfusepath{stroke,fill}%
\end{pgfscope}%
\begin{pgfscope}%
\pgfpathrectangle{\pgfqpoint{0.772069in}{0.515123in}}{\pgfqpoint{1.937500in}{1.347500in}}%
\pgfusepath{clip}%
\pgfsetbuttcap%
\pgfsetroundjoin%
\definecolor{currentfill}{rgb}{1.000000,0.843137,0.000000}%
\pgfsetfillcolor{currentfill}%
\pgfsetlinewidth{1.003750pt}%
\definecolor{currentstroke}{rgb}{1.000000,0.843137,0.000000}%
\pgfsetstrokecolor{currentstroke}%
\pgfsetdash{}{0pt}%
\pgfpathmoveto{\pgfqpoint{1.827120in}{0.625428in}}%
\pgfpathcurveto{\pgfqpoint{1.838170in}{0.625428in}}{\pgfqpoint{1.848769in}{0.629818in}}{\pgfqpoint{1.856583in}{0.637632in}}%
\pgfpathcurveto{\pgfqpoint{1.864396in}{0.645446in}}{\pgfqpoint{1.868786in}{0.656045in}}{\pgfqpoint{1.868786in}{0.667095in}}%
\pgfpathcurveto{\pgfqpoint{1.868786in}{0.678145in}}{\pgfqpoint{1.864396in}{0.688744in}}{\pgfqpoint{1.856583in}{0.696558in}}%
\pgfpathcurveto{\pgfqpoint{1.848769in}{0.704371in}}{\pgfqpoint{1.838170in}{0.708761in}}{\pgfqpoint{1.827120in}{0.708761in}}%
\pgfpathcurveto{\pgfqpoint{1.816070in}{0.708761in}}{\pgfqpoint{1.805471in}{0.704371in}}{\pgfqpoint{1.797657in}{0.696558in}}%
\pgfpathcurveto{\pgfqpoint{1.789843in}{0.688744in}}{\pgfqpoint{1.785453in}{0.678145in}}{\pgfqpoint{1.785453in}{0.667095in}}%
\pgfpathcurveto{\pgfqpoint{1.785453in}{0.656045in}}{\pgfqpoint{1.789843in}{0.645446in}}{\pgfqpoint{1.797657in}{0.637632in}}%
\pgfpathcurveto{\pgfqpoint{1.805471in}{0.629818in}}{\pgfqpoint{1.816070in}{0.625428in}}{\pgfqpoint{1.827120in}{0.625428in}}%
\pgfpathclose%
\pgfusepath{stroke,fill}%
\end{pgfscope}%
\begin{pgfscope}%
\pgfpathrectangle{\pgfqpoint{0.772069in}{0.515123in}}{\pgfqpoint{1.937500in}{1.347500in}}%
\pgfusepath{clip}%
\pgfsetbuttcap%
\pgfsetroundjoin%
\definecolor{currentfill}{rgb}{1.000000,0.843137,0.000000}%
\pgfsetfillcolor{currentfill}%
\pgfsetlinewidth{1.003750pt}%
\definecolor{currentstroke}{rgb}{1.000000,0.843137,0.000000}%
\pgfsetstrokecolor{currentstroke}%
\pgfsetdash{}{0pt}%
\pgfpathmoveto{\pgfqpoint{2.024379in}{0.644463in}}%
\pgfpathcurveto{\pgfqpoint{2.035429in}{0.644463in}}{\pgfqpoint{2.046028in}{0.648853in}}{\pgfqpoint{2.053842in}{0.656666in}}%
\pgfpathcurveto{\pgfqpoint{2.061656in}{0.664480in}}{\pgfqpoint{2.066046in}{0.675079in}}{\pgfqpoint{2.066046in}{0.686129in}}%
\pgfpathcurveto{\pgfqpoint{2.066046in}{0.697179in}}{\pgfqpoint{2.061656in}{0.707778in}}{\pgfqpoint{2.053842in}{0.715592in}}%
\pgfpathcurveto{\pgfqpoint{2.046028in}{0.723406in}}{\pgfqpoint{2.035429in}{0.727796in}}{\pgfqpoint{2.024379in}{0.727796in}}%
\pgfpathcurveto{\pgfqpoint{2.013329in}{0.727796in}}{\pgfqpoint{2.002730in}{0.723406in}}{\pgfqpoint{1.994917in}{0.715592in}}%
\pgfpathcurveto{\pgfqpoint{1.987103in}{0.707778in}}{\pgfqpoint{1.982713in}{0.697179in}}{\pgfqpoint{1.982713in}{0.686129in}}%
\pgfpathcurveto{\pgfqpoint{1.982713in}{0.675079in}}{\pgfqpoint{1.987103in}{0.664480in}}{\pgfqpoint{1.994917in}{0.656666in}}%
\pgfpathcurveto{\pgfqpoint{2.002730in}{0.648853in}}{\pgfqpoint{2.013329in}{0.644463in}}{\pgfqpoint{2.024379in}{0.644463in}}%
\pgfpathclose%
\pgfusepath{stroke,fill}%
\end{pgfscope}%
\begin{pgfscope}%
\pgfpathrectangle{\pgfqpoint{0.772069in}{0.515123in}}{\pgfqpoint{1.937500in}{1.347500in}}%
\pgfusepath{clip}%
\pgfsetbuttcap%
\pgfsetroundjoin%
\definecolor{currentfill}{rgb}{1.000000,0.843137,0.000000}%
\pgfsetfillcolor{currentfill}%
\pgfsetlinewidth{1.003750pt}%
\definecolor{currentstroke}{rgb}{1.000000,0.843137,0.000000}%
\pgfsetstrokecolor{currentstroke}%
\pgfsetdash{}{0pt}%
\pgfpathmoveto{\pgfqpoint{2.221639in}{0.650535in}}%
\pgfpathcurveto{\pgfqpoint{2.232689in}{0.650535in}}{\pgfqpoint{2.243288in}{0.654925in}}{\pgfqpoint{2.251102in}{0.662739in}}%
\pgfpathcurveto{\pgfqpoint{2.258915in}{0.670552in}}{\pgfqpoint{2.263306in}{0.681151in}}{\pgfqpoint{2.263306in}{0.692202in}}%
\pgfpathcurveto{\pgfqpoint{2.263306in}{0.703252in}}{\pgfqpoint{2.258915in}{0.713851in}}{\pgfqpoint{2.251102in}{0.721664in}}%
\pgfpathcurveto{\pgfqpoint{2.243288in}{0.729478in}}{\pgfqpoint{2.232689in}{0.733868in}}{\pgfqpoint{2.221639in}{0.733868in}}%
\pgfpathcurveto{\pgfqpoint{2.210589in}{0.733868in}}{\pgfqpoint{2.199990in}{0.729478in}}{\pgfqpoint{2.192176in}{0.721664in}}%
\pgfpathcurveto{\pgfqpoint{2.184362in}{0.713851in}}{\pgfqpoint{2.179972in}{0.703252in}}{\pgfqpoint{2.179972in}{0.692202in}}%
\pgfpathcurveto{\pgfqpoint{2.179972in}{0.681151in}}{\pgfqpoint{2.184362in}{0.670552in}}{\pgfqpoint{2.192176in}{0.662739in}}%
\pgfpathcurveto{\pgfqpoint{2.199990in}{0.654925in}}{\pgfqpoint{2.210589in}{0.650535in}}{\pgfqpoint{2.221639in}{0.650535in}}%
\pgfpathclose%
\pgfusepath{stroke,fill}%
\end{pgfscope}%
\begin{pgfscope}%
\pgfpathrectangle{\pgfqpoint{0.772069in}{0.515123in}}{\pgfqpoint{1.937500in}{1.347500in}}%
\pgfusepath{clip}%
\pgfsetbuttcap%
\pgfsetroundjoin%
\definecolor{currentfill}{rgb}{1.000000,0.843137,0.000000}%
\pgfsetfillcolor{currentfill}%
\pgfsetlinewidth{1.003750pt}%
\definecolor{currentstroke}{rgb}{1.000000,0.843137,0.000000}%
\pgfsetstrokecolor{currentstroke}%
\pgfsetdash{}{0pt}%
\pgfpathmoveto{\pgfqpoint{2.394241in}{0.662563in}}%
\pgfpathcurveto{\pgfqpoint{2.405291in}{0.662563in}}{\pgfqpoint{2.415890in}{0.666953in}}{\pgfqpoint{2.423704in}{0.674767in}}%
\pgfpathcurveto{\pgfqpoint{2.431517in}{0.682580in}}{\pgfqpoint{2.435908in}{0.693179in}}{\pgfqpoint{2.435908in}{0.704229in}}%
\pgfpathcurveto{\pgfqpoint{2.435908in}{0.715280in}}{\pgfqpoint{2.431517in}{0.725879in}}{\pgfqpoint{2.423704in}{0.733692in}}%
\pgfpathcurveto{\pgfqpoint{2.415890in}{0.741506in}}{\pgfqpoint{2.405291in}{0.745896in}}{\pgfqpoint{2.394241in}{0.745896in}}%
\pgfpathcurveto{\pgfqpoint{2.383191in}{0.745896in}}{\pgfqpoint{2.372592in}{0.741506in}}{\pgfqpoint{2.364778in}{0.733692in}}%
\pgfpathcurveto{\pgfqpoint{2.356965in}{0.725879in}}{\pgfqpoint{2.352574in}{0.715280in}}{\pgfqpoint{2.352574in}{0.704229in}}%
\pgfpathcurveto{\pgfqpoint{2.352574in}{0.693179in}}{\pgfqpoint{2.356965in}{0.682580in}}{\pgfqpoint{2.364778in}{0.674767in}}%
\pgfpathcurveto{\pgfqpoint{2.372592in}{0.666953in}}{\pgfqpoint{2.383191in}{0.662563in}}{\pgfqpoint{2.394241in}{0.662563in}}%
\pgfpathclose%
\pgfusepath{stroke,fill}%
\end{pgfscope}%
\begin{pgfscope}%
\pgfpathrectangle{\pgfqpoint{0.772069in}{0.515123in}}{\pgfqpoint{1.937500in}{1.347500in}}%
\pgfusepath{clip}%
\pgfsetbuttcap%
\pgfsetroundjoin%
\definecolor{currentfill}{rgb}{1.000000,0.843137,0.000000}%
\pgfsetfillcolor{currentfill}%
\pgfsetlinewidth{1.003750pt}%
\definecolor{currentstroke}{rgb}{1.000000,0.843137,0.000000}%
\pgfsetstrokecolor{currentstroke}%
\pgfsetdash{}{0pt}%
\pgfpathmoveto{\pgfqpoint{2.591501in}{0.674941in}}%
\pgfpathcurveto{\pgfqpoint{2.602551in}{0.674941in}}{\pgfqpoint{2.613150in}{0.679331in}}{\pgfqpoint{2.620963in}{0.687145in}}%
\pgfpathcurveto{\pgfqpoint{2.628777in}{0.694959in}}{\pgfqpoint{2.633167in}{0.705558in}}{\pgfqpoint{2.633167in}{0.716608in}}%
\pgfpathcurveto{\pgfqpoint{2.633167in}{0.727658in}}{\pgfqpoint{2.628777in}{0.738257in}}{\pgfqpoint{2.620963in}{0.746070in}}%
\pgfpathcurveto{\pgfqpoint{2.613150in}{0.753884in}}{\pgfqpoint{2.602551in}{0.758274in}}{\pgfqpoint{2.591501in}{0.758274in}}%
\pgfpathcurveto{\pgfqpoint{2.580450in}{0.758274in}}{\pgfqpoint{2.569851in}{0.753884in}}{\pgfqpoint{2.562038in}{0.746070in}}%
\pgfpathcurveto{\pgfqpoint{2.554224in}{0.738257in}}{\pgfqpoint{2.549834in}{0.727658in}}{\pgfqpoint{2.549834in}{0.716608in}}%
\pgfpathcurveto{\pgfqpoint{2.549834in}{0.705558in}}{\pgfqpoint{2.554224in}{0.694959in}}{\pgfqpoint{2.562038in}{0.687145in}}%
\pgfpathcurveto{\pgfqpoint{2.569851in}{0.679331in}}{\pgfqpoint{2.580450in}{0.674941in}}{\pgfqpoint{2.591501in}{0.674941in}}%
\pgfpathclose%
\pgfusepath{stroke,fill}%
\end{pgfscope}%
\begin{pgfscope}%
\pgfpathrectangle{\pgfqpoint{0.772069in}{0.515123in}}{\pgfqpoint{1.937500in}{1.347500in}}%
\pgfusepath{clip}%
\pgfsetbuttcap%
\pgfsetroundjoin%
\definecolor{currentfill}{rgb}{0.196078,0.803922,0.196078}%
\pgfsetfillcolor{currentfill}%
\pgfsetlinewidth{1.003750pt}%
\definecolor{currentstroke}{rgb}{0.196078,0.803922,0.196078}%
\pgfsetstrokecolor{currentstroke}%
\pgfsetdash{}{0pt}%
\pgfpathmoveto{\pgfqpoint{0.890137in}{0.641660in}}%
\pgfpathcurveto{\pgfqpoint{0.901187in}{0.641660in}}{\pgfqpoint{0.911786in}{0.646050in}}{\pgfqpoint{0.919600in}{0.653864in}}%
\pgfpathcurveto{\pgfqpoint{0.927413in}{0.661677in}}{\pgfqpoint{0.931803in}{0.672277in}}{\pgfqpoint{0.931803in}{0.683327in}}%
\pgfpathcurveto{\pgfqpoint{0.931803in}{0.694377in}}{\pgfqpoint{0.927413in}{0.704976in}}{\pgfqpoint{0.919600in}{0.712789in}}%
\pgfpathcurveto{\pgfqpoint{0.911786in}{0.720603in}}{\pgfqpoint{0.901187in}{0.724993in}}{\pgfqpoint{0.890137in}{0.724993in}}%
\pgfpathcurveto{\pgfqpoint{0.879087in}{0.724993in}}{\pgfqpoint{0.868488in}{0.720603in}}{\pgfqpoint{0.860674in}{0.712789in}}%
\pgfpathcurveto{\pgfqpoint{0.852860in}{0.704976in}}{\pgfqpoint{0.848470in}{0.694377in}}{\pgfqpoint{0.848470in}{0.683327in}}%
\pgfpathcurveto{\pgfqpoint{0.848470in}{0.672277in}}{\pgfqpoint{0.852860in}{0.661677in}}{\pgfqpoint{0.860674in}{0.653864in}}%
\pgfpathcurveto{\pgfqpoint{0.868488in}{0.646050in}}{\pgfqpoint{0.879087in}{0.641660in}}{\pgfqpoint{0.890137in}{0.641660in}}%
\pgfpathclose%
\pgfusepath{stroke,fill}%
\end{pgfscope}%
\begin{pgfscope}%
\pgfpathrectangle{\pgfqpoint{0.772069in}{0.515123in}}{\pgfqpoint{1.937500in}{1.347500in}}%
\pgfusepath{clip}%
\pgfsetbuttcap%
\pgfsetroundjoin%
\definecolor{currentfill}{rgb}{0.196078,0.803922,0.196078}%
\pgfsetfillcolor{currentfill}%
\pgfsetlinewidth{1.003750pt}%
\definecolor{currentstroke}{rgb}{0.196078,0.803922,0.196078}%
\pgfsetstrokecolor{currentstroke}%
\pgfsetdash{}{0pt}%
\pgfpathmoveto{\pgfqpoint{1.087396in}{0.726089in}}%
\pgfpathcurveto{\pgfqpoint{1.098447in}{0.726089in}}{\pgfqpoint{1.109046in}{0.730479in}}{\pgfqpoint{1.116859in}{0.738293in}}%
\pgfpathcurveto{\pgfqpoint{1.124673in}{0.746106in}}{\pgfqpoint{1.129063in}{0.756705in}}{\pgfqpoint{1.129063in}{0.767755in}}%
\pgfpathcurveto{\pgfqpoint{1.129063in}{0.778806in}}{\pgfqpoint{1.124673in}{0.789405in}}{\pgfqpoint{1.116859in}{0.797218in}}%
\pgfpathcurveto{\pgfqpoint{1.109046in}{0.805032in}}{\pgfqpoint{1.098447in}{0.809422in}}{\pgfqpoint{1.087396in}{0.809422in}}%
\pgfpathcurveto{\pgfqpoint{1.076346in}{0.809422in}}{\pgfqpoint{1.065747in}{0.805032in}}{\pgfqpoint{1.057934in}{0.797218in}}%
\pgfpathcurveto{\pgfqpoint{1.050120in}{0.789405in}}{\pgfqpoint{1.045730in}{0.778806in}}{\pgfqpoint{1.045730in}{0.767755in}}%
\pgfpathcurveto{\pgfqpoint{1.045730in}{0.756705in}}{\pgfqpoint{1.050120in}{0.746106in}}{\pgfqpoint{1.057934in}{0.738293in}}%
\pgfpathcurveto{\pgfqpoint{1.065747in}{0.730479in}}{\pgfqpoint{1.076346in}{0.726089in}}{\pgfqpoint{1.087396in}{0.726089in}}%
\pgfpathclose%
\pgfusepath{stroke,fill}%
\end{pgfscope}%
\begin{pgfscope}%
\pgfpathrectangle{\pgfqpoint{0.772069in}{0.515123in}}{\pgfqpoint{1.937500in}{1.347500in}}%
\pgfusepath{clip}%
\pgfsetbuttcap%
\pgfsetroundjoin%
\definecolor{currentfill}{rgb}{0.196078,0.803922,0.196078}%
\pgfsetfillcolor{currentfill}%
\pgfsetlinewidth{1.003750pt}%
\definecolor{currentstroke}{rgb}{0.196078,0.803922,0.196078}%
\pgfsetstrokecolor{currentstroke}%
\pgfsetdash{}{0pt}%
\pgfpathmoveto{\pgfqpoint{1.259999in}{0.814605in}}%
\pgfpathcurveto{\pgfqpoint{1.271049in}{0.814605in}}{\pgfqpoint{1.281648in}{0.818995in}}{\pgfqpoint{1.289461in}{0.826809in}}%
\pgfpathcurveto{\pgfqpoint{1.297275in}{0.834622in}}{\pgfqpoint{1.301665in}{0.845221in}}{\pgfqpoint{1.301665in}{0.856271in}}%
\pgfpathcurveto{\pgfqpoint{1.301665in}{0.867321in}}{\pgfqpoint{1.297275in}{0.877920in}}{\pgfqpoint{1.289461in}{0.885734in}}%
\pgfpathcurveto{\pgfqpoint{1.281648in}{0.893548in}}{\pgfqpoint{1.271049in}{0.897938in}}{\pgfqpoint{1.259999in}{0.897938in}}%
\pgfpathcurveto{\pgfqpoint{1.248948in}{0.897938in}}{\pgfqpoint{1.238349in}{0.893548in}}{\pgfqpoint{1.230536in}{0.885734in}}%
\pgfpathcurveto{\pgfqpoint{1.222722in}{0.877920in}}{\pgfqpoint{1.218332in}{0.867321in}}{\pgfqpoint{1.218332in}{0.856271in}}%
\pgfpathcurveto{\pgfqpoint{1.218332in}{0.845221in}}{\pgfqpoint{1.222722in}{0.834622in}}{\pgfqpoint{1.230536in}{0.826809in}}%
\pgfpathcurveto{\pgfqpoint{1.238349in}{0.818995in}}{\pgfqpoint{1.248948in}{0.814605in}}{\pgfqpoint{1.259999in}{0.814605in}}%
\pgfpathclose%
\pgfusepath{stroke,fill}%
\end{pgfscope}%
\begin{pgfscope}%
\pgfpathrectangle{\pgfqpoint{0.772069in}{0.515123in}}{\pgfqpoint{1.937500in}{1.347500in}}%
\pgfusepath{clip}%
\pgfsetbuttcap%
\pgfsetroundjoin%
\definecolor{currentfill}{rgb}{0.196078,0.803922,0.196078}%
\pgfsetfillcolor{currentfill}%
\pgfsetlinewidth{1.003750pt}%
\definecolor{currentstroke}{rgb}{0.196078,0.803922,0.196078}%
\pgfsetstrokecolor{currentstroke}%
\pgfsetdash{}{0pt}%
\pgfpathmoveto{\pgfqpoint{1.457258in}{0.904522in}}%
\pgfpathcurveto{\pgfqpoint{1.468308in}{0.904522in}}{\pgfqpoint{1.478907in}{0.908912in}}{\pgfqpoint{1.486721in}{0.916726in}}%
\pgfpathcurveto{\pgfqpoint{1.494534in}{0.924539in}}{\pgfqpoint{1.498925in}{0.935138in}}{\pgfqpoint{1.498925in}{0.946189in}}%
\pgfpathcurveto{\pgfqpoint{1.498925in}{0.957239in}}{\pgfqpoint{1.494534in}{0.967838in}}{\pgfqpoint{1.486721in}{0.975651in}}%
\pgfpathcurveto{\pgfqpoint{1.478907in}{0.983465in}}{\pgfqpoint{1.468308in}{0.987855in}}{\pgfqpoint{1.457258in}{0.987855in}}%
\pgfpathcurveto{\pgfqpoint{1.446208in}{0.987855in}}{\pgfqpoint{1.435609in}{0.983465in}}{\pgfqpoint{1.427795in}{0.975651in}}%
\pgfpathcurveto{\pgfqpoint{1.419982in}{0.967838in}}{\pgfqpoint{1.415591in}{0.957239in}}{\pgfqpoint{1.415591in}{0.946189in}}%
\pgfpathcurveto{\pgfqpoint{1.415591in}{0.935138in}}{\pgfqpoint{1.419982in}{0.924539in}}{\pgfqpoint{1.427795in}{0.916726in}}%
\pgfpathcurveto{\pgfqpoint{1.435609in}{0.908912in}}{\pgfqpoint{1.446208in}{0.904522in}}{\pgfqpoint{1.457258in}{0.904522in}}%
\pgfpathclose%
\pgfusepath{stroke,fill}%
\end{pgfscope}%
\begin{pgfscope}%
\pgfpathrectangle{\pgfqpoint{0.772069in}{0.515123in}}{\pgfqpoint{1.937500in}{1.347500in}}%
\pgfusepath{clip}%
\pgfsetbuttcap%
\pgfsetroundjoin%
\definecolor{currentfill}{rgb}{0.196078,0.803922,0.196078}%
\pgfsetfillcolor{currentfill}%
\pgfsetlinewidth{1.003750pt}%
\definecolor{currentstroke}{rgb}{0.196078,0.803922,0.196078}%
\pgfsetstrokecolor{currentstroke}%
\pgfsetdash{}{0pt}%
\pgfpathmoveto{\pgfqpoint{1.654518in}{0.993271in}}%
\pgfpathcurveto{\pgfqpoint{1.665568in}{0.993271in}}{\pgfqpoint{1.676167in}{0.997662in}}{\pgfqpoint{1.683980in}{1.005475in}}%
\pgfpathcurveto{\pgfqpoint{1.691794in}{1.013289in}}{\pgfqpoint{1.696184in}{1.023888in}}{\pgfqpoint{1.696184in}{1.034938in}}%
\pgfpathcurveto{\pgfqpoint{1.696184in}{1.045988in}}{\pgfqpoint{1.691794in}{1.056587in}}{\pgfqpoint{1.683980in}{1.064401in}}%
\pgfpathcurveto{\pgfqpoint{1.676167in}{1.072214in}}{\pgfqpoint{1.665568in}{1.076605in}}{\pgfqpoint{1.654518in}{1.076605in}}%
\pgfpathcurveto{\pgfqpoint{1.643467in}{1.076605in}}{\pgfqpoint{1.632868in}{1.072214in}}{\pgfqpoint{1.625055in}{1.064401in}}%
\pgfpathcurveto{\pgfqpoint{1.617241in}{1.056587in}}{\pgfqpoint{1.612851in}{1.045988in}}{\pgfqpoint{1.612851in}{1.034938in}}%
\pgfpathcurveto{\pgfqpoint{1.612851in}{1.023888in}}{\pgfqpoint{1.617241in}{1.013289in}}{\pgfqpoint{1.625055in}{1.005475in}}%
\pgfpathcurveto{\pgfqpoint{1.632868in}{0.997662in}}{\pgfqpoint{1.643467in}{0.993271in}}{\pgfqpoint{1.654518in}{0.993271in}}%
\pgfpathclose%
\pgfusepath{stroke,fill}%
\end{pgfscope}%
\begin{pgfscope}%
\pgfpathrectangle{\pgfqpoint{0.772069in}{0.515123in}}{\pgfqpoint{1.937500in}{1.347500in}}%
\pgfusepath{clip}%
\pgfsetbuttcap%
\pgfsetroundjoin%
\definecolor{currentfill}{rgb}{0.196078,0.803922,0.196078}%
\pgfsetfillcolor{currentfill}%
\pgfsetlinewidth{1.003750pt}%
\definecolor{currentstroke}{rgb}{0.196078,0.803922,0.196078}%
\pgfsetstrokecolor{currentstroke}%
\pgfsetdash{}{0pt}%
\pgfpathmoveto{\pgfqpoint{1.827120in}{1.077350in}}%
\pgfpathcurveto{\pgfqpoint{1.838170in}{1.077350in}}{\pgfqpoint{1.848769in}{1.081740in}}{\pgfqpoint{1.856583in}{1.089554in}}%
\pgfpathcurveto{\pgfqpoint{1.864396in}{1.097367in}}{\pgfqpoint{1.868786in}{1.107966in}}{\pgfqpoint{1.868786in}{1.119016in}}%
\pgfpathcurveto{\pgfqpoint{1.868786in}{1.130067in}}{\pgfqpoint{1.864396in}{1.140666in}}{\pgfqpoint{1.856583in}{1.148479in}}%
\pgfpathcurveto{\pgfqpoint{1.848769in}{1.156293in}}{\pgfqpoint{1.838170in}{1.160683in}}{\pgfqpoint{1.827120in}{1.160683in}}%
\pgfpathcurveto{\pgfqpoint{1.816070in}{1.160683in}}{\pgfqpoint{1.805471in}{1.156293in}}{\pgfqpoint{1.797657in}{1.148479in}}%
\pgfpathcurveto{\pgfqpoint{1.789843in}{1.140666in}}{\pgfqpoint{1.785453in}{1.130067in}}{\pgfqpoint{1.785453in}{1.119016in}}%
\pgfpathcurveto{\pgfqpoint{1.785453in}{1.107966in}}{\pgfqpoint{1.789843in}{1.097367in}}{\pgfqpoint{1.797657in}{1.089554in}}%
\pgfpathcurveto{\pgfqpoint{1.805471in}{1.081740in}}{\pgfqpoint{1.816070in}{1.077350in}}{\pgfqpoint{1.827120in}{1.077350in}}%
\pgfpathclose%
\pgfusepath{stroke,fill}%
\end{pgfscope}%
\begin{pgfscope}%
\pgfpathrectangle{\pgfqpoint{0.772069in}{0.515123in}}{\pgfqpoint{1.937500in}{1.347500in}}%
\pgfusepath{clip}%
\pgfsetbuttcap%
\pgfsetroundjoin%
\definecolor{currentfill}{rgb}{0.196078,0.803922,0.196078}%
\pgfsetfillcolor{currentfill}%
\pgfsetlinewidth{1.003750pt}%
\definecolor{currentstroke}{rgb}{0.196078,0.803922,0.196078}%
\pgfsetstrokecolor{currentstroke}%
\pgfsetdash{}{0pt}%
\pgfpathmoveto{\pgfqpoint{2.024379in}{1.168435in}}%
\pgfpathcurveto{\pgfqpoint{2.035429in}{1.168435in}}{\pgfqpoint{2.046028in}{1.172825in}}{\pgfqpoint{2.053842in}{1.180639in}}%
\pgfpathcurveto{\pgfqpoint{2.061656in}{1.188452in}}{\pgfqpoint{2.066046in}{1.199051in}}{\pgfqpoint{2.066046in}{1.210101in}}%
\pgfpathcurveto{\pgfqpoint{2.066046in}{1.221152in}}{\pgfqpoint{2.061656in}{1.231751in}}{\pgfqpoint{2.053842in}{1.239564in}}%
\pgfpathcurveto{\pgfqpoint{2.046028in}{1.247378in}}{\pgfqpoint{2.035429in}{1.251768in}}{\pgfqpoint{2.024379in}{1.251768in}}%
\pgfpathcurveto{\pgfqpoint{2.013329in}{1.251768in}}{\pgfqpoint{2.002730in}{1.247378in}}{\pgfqpoint{1.994917in}{1.239564in}}%
\pgfpathcurveto{\pgfqpoint{1.987103in}{1.231751in}}{\pgfqpoint{1.982713in}{1.221152in}}{\pgfqpoint{1.982713in}{1.210101in}}%
\pgfpathcurveto{\pgfqpoint{1.982713in}{1.199051in}}{\pgfqpoint{1.987103in}{1.188452in}}{\pgfqpoint{1.994917in}{1.180639in}}%
\pgfpathcurveto{\pgfqpoint{2.002730in}{1.172825in}}{\pgfqpoint{2.013329in}{1.168435in}}{\pgfqpoint{2.024379in}{1.168435in}}%
\pgfpathclose%
\pgfusepath{stroke,fill}%
\end{pgfscope}%
\begin{pgfscope}%
\pgfpathrectangle{\pgfqpoint{0.772069in}{0.515123in}}{\pgfqpoint{1.937500in}{1.347500in}}%
\pgfusepath{clip}%
\pgfsetbuttcap%
\pgfsetroundjoin%
\definecolor{currentfill}{rgb}{0.196078,0.803922,0.196078}%
\pgfsetfillcolor{currentfill}%
\pgfsetlinewidth{1.003750pt}%
\definecolor{currentstroke}{rgb}{0.196078,0.803922,0.196078}%
\pgfsetstrokecolor{currentstroke}%
\pgfsetdash{}{0pt}%
\pgfpathmoveto{\pgfqpoint{2.221639in}{1.256017in}}%
\pgfpathcurveto{\pgfqpoint{2.232689in}{1.256017in}}{\pgfqpoint{2.243288in}{1.260407in}}{\pgfqpoint{2.251102in}{1.268220in}}%
\pgfpathcurveto{\pgfqpoint{2.258915in}{1.276034in}}{\pgfqpoint{2.263306in}{1.286633in}}{\pgfqpoint{2.263306in}{1.297683in}}%
\pgfpathcurveto{\pgfqpoint{2.263306in}{1.308733in}}{\pgfqpoint{2.258915in}{1.319332in}}{\pgfqpoint{2.251102in}{1.327146in}}%
\pgfpathcurveto{\pgfqpoint{2.243288in}{1.334960in}}{\pgfqpoint{2.232689in}{1.339350in}}{\pgfqpoint{2.221639in}{1.339350in}}%
\pgfpathcurveto{\pgfqpoint{2.210589in}{1.339350in}}{\pgfqpoint{2.199990in}{1.334960in}}{\pgfqpoint{2.192176in}{1.327146in}}%
\pgfpathcurveto{\pgfqpoint{2.184362in}{1.319332in}}{\pgfqpoint{2.179972in}{1.308733in}}{\pgfqpoint{2.179972in}{1.297683in}}%
\pgfpathcurveto{\pgfqpoint{2.179972in}{1.286633in}}{\pgfqpoint{2.184362in}{1.276034in}}{\pgfqpoint{2.192176in}{1.268220in}}%
\pgfpathcurveto{\pgfqpoint{2.199990in}{1.260407in}}{\pgfqpoint{2.210589in}{1.256017in}}{\pgfqpoint{2.221639in}{1.256017in}}%
\pgfpathclose%
\pgfusepath{stroke,fill}%
\end{pgfscope}%
\begin{pgfscope}%
\pgfpathrectangle{\pgfqpoint{0.772069in}{0.515123in}}{\pgfqpoint{1.937500in}{1.347500in}}%
\pgfusepath{clip}%
\pgfsetbuttcap%
\pgfsetroundjoin%
\definecolor{currentfill}{rgb}{0.196078,0.803922,0.196078}%
\pgfsetfillcolor{currentfill}%
\pgfsetlinewidth{1.003750pt}%
\definecolor{currentstroke}{rgb}{0.196078,0.803922,0.196078}%
\pgfsetstrokecolor{currentstroke}%
\pgfsetdash{}{0pt}%
\pgfpathmoveto{\pgfqpoint{2.394241in}{1.342430in}}%
\pgfpathcurveto{\pgfqpoint{2.405291in}{1.342430in}}{\pgfqpoint{2.415890in}{1.346821in}}{\pgfqpoint{2.423704in}{1.354634in}}%
\pgfpathcurveto{\pgfqpoint{2.431517in}{1.362448in}}{\pgfqpoint{2.435908in}{1.373047in}}{\pgfqpoint{2.435908in}{1.384097in}}%
\pgfpathcurveto{\pgfqpoint{2.435908in}{1.395147in}}{\pgfqpoint{2.431517in}{1.405746in}}{\pgfqpoint{2.423704in}{1.413560in}}%
\pgfpathcurveto{\pgfqpoint{2.415890in}{1.421374in}}{\pgfqpoint{2.405291in}{1.425764in}}{\pgfqpoint{2.394241in}{1.425764in}}%
\pgfpathcurveto{\pgfqpoint{2.383191in}{1.425764in}}{\pgfqpoint{2.372592in}{1.421374in}}{\pgfqpoint{2.364778in}{1.413560in}}%
\pgfpathcurveto{\pgfqpoint{2.356965in}{1.405746in}}{\pgfqpoint{2.352574in}{1.395147in}}{\pgfqpoint{2.352574in}{1.384097in}}%
\pgfpathcurveto{\pgfqpoint{2.352574in}{1.373047in}}{\pgfqpoint{2.356965in}{1.362448in}}{\pgfqpoint{2.364778in}{1.354634in}}%
\pgfpathcurveto{\pgfqpoint{2.372592in}{1.346821in}}{\pgfqpoint{2.383191in}{1.342430in}}{\pgfqpoint{2.394241in}{1.342430in}}%
\pgfpathclose%
\pgfusepath{stroke,fill}%
\end{pgfscope}%
\begin{pgfscope}%
\pgfpathrectangle{\pgfqpoint{0.772069in}{0.515123in}}{\pgfqpoint{1.937500in}{1.347500in}}%
\pgfusepath{clip}%
\pgfsetbuttcap%
\pgfsetroundjoin%
\definecolor{currentfill}{rgb}{0.196078,0.803922,0.196078}%
\pgfsetfillcolor{currentfill}%
\pgfsetlinewidth{1.003750pt}%
\definecolor{currentstroke}{rgb}{0.196078,0.803922,0.196078}%
\pgfsetstrokecolor{currentstroke}%
\pgfsetdash{}{0pt}%
\pgfpathmoveto{\pgfqpoint{2.591501in}{1.431180in}}%
\pgfpathcurveto{\pgfqpoint{2.602551in}{1.431180in}}{\pgfqpoint{2.613150in}{1.435570in}}{\pgfqpoint{2.620963in}{1.443384in}}%
\pgfpathcurveto{\pgfqpoint{2.628777in}{1.451197in}}{\pgfqpoint{2.633167in}{1.461796in}}{\pgfqpoint{2.633167in}{1.472847in}}%
\pgfpathcurveto{\pgfqpoint{2.633167in}{1.483897in}}{\pgfqpoint{2.628777in}{1.494496in}}{\pgfqpoint{2.620963in}{1.502309in}}%
\pgfpathcurveto{\pgfqpoint{2.613150in}{1.510123in}}{\pgfqpoint{2.602551in}{1.514513in}}{\pgfqpoint{2.591501in}{1.514513in}}%
\pgfpathcurveto{\pgfqpoint{2.580450in}{1.514513in}}{\pgfqpoint{2.569851in}{1.510123in}}{\pgfqpoint{2.562038in}{1.502309in}}%
\pgfpathcurveto{\pgfqpoint{2.554224in}{1.494496in}}{\pgfqpoint{2.549834in}{1.483897in}}{\pgfqpoint{2.549834in}{1.472847in}}%
\pgfpathcurveto{\pgfqpoint{2.549834in}{1.461796in}}{\pgfqpoint{2.554224in}{1.451197in}}{\pgfqpoint{2.562038in}{1.443384in}}%
\pgfpathcurveto{\pgfqpoint{2.569851in}{1.435570in}}{\pgfqpoint{2.580450in}{1.431180in}}{\pgfqpoint{2.591501in}{1.431180in}}%
\pgfpathclose%
\pgfusepath{stroke,fill}%
\end{pgfscope}%
\begin{pgfscope}%
\pgfsetrectcap%
\pgfsetmiterjoin%
\pgfsetlinewidth{0.803000pt}%
\definecolor{currentstroke}{rgb}{0.000000,0.000000,0.000000}%
\pgfsetstrokecolor{currentstroke}%
\pgfsetdash{}{0pt}%
\pgfpathmoveto{\pgfqpoint{0.772069in}{0.515123in}}%
\pgfpathlineto{\pgfqpoint{0.772069in}{1.862623in}}%
\pgfusepath{stroke}%
\end{pgfscope}%
\begin{pgfscope}%
\pgfsetrectcap%
\pgfsetmiterjoin%
\pgfsetlinewidth{0.803000pt}%
\definecolor{currentstroke}{rgb}{0.000000,0.000000,0.000000}%
\pgfsetstrokecolor{currentstroke}%
\pgfsetdash{}{0pt}%
\pgfpathmoveto{\pgfqpoint{2.709569in}{0.515123in}}%
\pgfpathlineto{\pgfqpoint{2.709569in}{1.862623in}}%
\pgfusepath{stroke}%
\end{pgfscope}%
\begin{pgfscope}%
\pgfsetrectcap%
\pgfsetmiterjoin%
\pgfsetlinewidth{0.803000pt}%
\definecolor{currentstroke}{rgb}{0.000000,0.000000,0.000000}%
\pgfsetstrokecolor{currentstroke}%
\pgfsetdash{}{0pt}%
\pgfpathmoveto{\pgfqpoint{0.772069in}{0.515123in}}%
\pgfpathlineto{\pgfqpoint{2.709569in}{0.515123in}}%
\pgfusepath{stroke}%
\end{pgfscope}%
\begin{pgfscope}%
\pgfsetrectcap%
\pgfsetmiterjoin%
\pgfsetlinewidth{0.803000pt}%
\definecolor{currentstroke}{rgb}{0.000000,0.000000,0.000000}%
\pgfsetstrokecolor{currentstroke}%
\pgfsetdash{}{0pt}%
\pgfpathmoveto{\pgfqpoint{0.772069in}{1.862623in}}%
\pgfpathlineto{\pgfqpoint{2.709569in}{1.862623in}}%
\pgfusepath{stroke}%
\end{pgfscope}%
\end{pgfpicture}%
\makeatother%
\endgroup%

    %% Creator: Matplotlib, PGF backend
%%
%% To include the figure in your LaTeX document, write
%%   \input{<filename>.pgf}
%%
%% Make sure the required packages are loaded in your preamble
%%   \usepackage{pgf}
%%
%% Figures using additional raster images can only be included by \input if
%% they are in the same directory as the main LaTeX file. For loading figures
%% from other directories you can use the `import` package
%%   \usepackage{import}
%% and then include the figures with
%%   \import{<path to file>}{<filename>.pgf}
%%
%% Matplotlib used the following preamble
%%
\begingroup%
\makeatletter%
\begin{pgfpicture}%
\pgfpathrectangle{\pgfpointorigin}{\pgfqpoint{2.809569in}{1.962623in}}%
\pgfusepath{use as bounding box, clip}%
\begin{pgfscope}%
\pgfsetbuttcap%
\pgfsetmiterjoin%
\definecolor{currentfill}{rgb}{1.000000,1.000000,1.000000}%
\pgfsetfillcolor{currentfill}%
\pgfsetlinewidth{0.000000pt}%
\definecolor{currentstroke}{rgb}{1.000000,1.000000,1.000000}%
\pgfsetstrokecolor{currentstroke}%
\pgfsetdash{}{0pt}%
\pgfpathmoveto{\pgfqpoint{0.000000in}{0.000000in}}%
\pgfpathlineto{\pgfqpoint{2.809569in}{0.000000in}}%
\pgfpathlineto{\pgfqpoint{2.809569in}{1.962623in}}%
\pgfpathlineto{\pgfqpoint{0.000000in}{1.962623in}}%
\pgfpathclose%
\pgfusepath{fill}%
\end{pgfscope}%
\begin{pgfscope}%
\pgfsetbuttcap%
\pgfsetmiterjoin%
\definecolor{currentfill}{rgb}{1.000000,1.000000,1.000000}%
\pgfsetfillcolor{currentfill}%
\pgfsetlinewidth{0.000000pt}%
\definecolor{currentstroke}{rgb}{0.000000,0.000000,0.000000}%
\pgfsetstrokecolor{currentstroke}%
\pgfsetstrokeopacity{0.000000}%
\pgfsetdash{}{0pt}%
\pgfpathmoveto{\pgfqpoint{0.772069in}{0.515123in}}%
\pgfpathlineto{\pgfqpoint{2.709569in}{0.515123in}}%
\pgfpathlineto{\pgfqpoint{2.709569in}{1.862623in}}%
\pgfpathlineto{\pgfqpoint{0.772069in}{1.862623in}}%
\pgfpathclose%
\pgfusepath{fill}%
\end{pgfscope}%
\begin{pgfscope}%
\pgfpathrectangle{\pgfqpoint{0.772069in}{0.515123in}}{\pgfqpoint{1.937500in}{1.347500in}}%
\pgfusepath{clip}%
\pgfsetrectcap%
\pgfsetroundjoin%
\pgfsetlinewidth{1.505625pt}%
\definecolor{currentstroke}{rgb}{0.529412,0.807843,0.921569}%
\pgfsetstrokecolor{currentstroke}%
\pgfsetdash{}{0pt}%
\pgfpathmoveto{\pgfqpoint{1.004823in}{0.598799in}}%
\pgfpathlineto{\pgfqpoint{1.181446in}{0.728007in}}%
\pgfpathlineto{\pgfqpoint{1.358068in}{0.857214in}}%
\pgfpathlineto{\pgfqpoint{1.534691in}{0.986421in}}%
\pgfpathlineto{\pgfqpoint{1.711313in}{1.115628in}}%
\pgfpathlineto{\pgfqpoint{1.887936in}{1.244835in}}%
\pgfpathlineto{\pgfqpoint{2.064558in}{1.374042in}}%
\pgfpathlineto{\pgfqpoint{2.241181in}{1.503250in}}%
\pgfpathlineto{\pgfqpoint{2.417803in}{1.632457in}}%
\pgfpathlineto{\pgfqpoint{2.594426in}{1.761664in}}%
\pgfusepath{stroke}%
\end{pgfscope}%
\begin{pgfscope}%
\pgfpathrectangle{\pgfqpoint{0.772069in}{0.515123in}}{\pgfqpoint{1.937500in}{1.347500in}}%
\pgfusepath{clip}%
\pgfsetrectcap%
\pgfsetroundjoin%
\pgfsetlinewidth{1.505625pt}%
\definecolor{currentstroke}{rgb}{0.529412,0.807843,0.921569}%
\pgfsetstrokecolor{currentstroke}%
\pgfsetdash{}{0pt}%
\pgfpathmoveto{\pgfqpoint{0.935710in}{0.584042in}}%
\pgfpathlineto{\pgfqpoint{1.050899in}{0.713060in}}%
\pgfpathlineto{\pgfqpoint{1.166087in}{0.842078in}}%
\pgfpathlineto{\pgfqpoint{1.281276in}{0.971096in}}%
\pgfpathlineto{\pgfqpoint{1.396464in}{1.100114in}}%
\pgfpathlineto{\pgfqpoint{1.511653in}{1.229132in}}%
\pgfpathlineto{\pgfqpoint{1.626842in}{1.358150in}}%
\pgfpathlineto{\pgfqpoint{1.742030in}{1.487168in}}%
\pgfpathlineto{\pgfqpoint{1.857219in}{1.616186in}}%
\pgfpathlineto{\pgfqpoint{1.972407in}{1.745204in}}%
\pgfusepath{stroke}%
\end{pgfscope}%
\begin{pgfscope}%
\pgfpathrectangle{\pgfqpoint{0.772069in}{0.515123in}}{\pgfqpoint{1.937500in}{1.347500in}}%
\pgfusepath{clip}%
\pgfsetrectcap%
\pgfsetroundjoin%
\pgfsetlinewidth{1.505625pt}%
\definecolor{currentstroke}{rgb}{0.529412,0.807843,0.921569}%
\pgfsetstrokecolor{currentstroke}%
\pgfsetdash{}{0pt}%
\pgfpathmoveto{\pgfqpoint{0.889634in}{0.576373in}}%
\pgfpathlineto{\pgfqpoint{0.951068in}{0.703768in}}%
\pgfpathlineto{\pgfqpoint{1.012502in}{0.831162in}}%
\pgfpathlineto{\pgfqpoint{1.073936in}{0.958556in}}%
\pgfpathlineto{\pgfqpoint{1.135370in}{1.085950in}}%
\pgfpathlineto{\pgfqpoint{1.196804in}{1.213345in}}%
\pgfpathlineto{\pgfqpoint{1.258238in}{1.340739in}}%
\pgfpathlineto{\pgfqpoint{1.319672in}{1.468133in}}%
\pgfpathlineto{\pgfqpoint{1.381106in}{1.595527in}}%
\pgfpathlineto{\pgfqpoint{1.442540in}{1.722922in}}%
\pgfusepath{stroke}%
\end{pgfscope}%
\begin{pgfscope}%
\pgfsetbuttcap%
\pgfsetroundjoin%
\definecolor{currentfill}{rgb}{0.000000,0.000000,0.000000}%
\pgfsetfillcolor{currentfill}%
\pgfsetlinewidth{0.803000pt}%
\definecolor{currentstroke}{rgb}{0.000000,0.000000,0.000000}%
\pgfsetstrokecolor{currentstroke}%
\pgfsetdash{}{0pt}%
\pgfsys@defobject{currentmarker}{\pgfqpoint{0.000000in}{-0.048611in}}{\pgfqpoint{0.000000in}{0.000000in}}{%
\pgfpathmoveto{\pgfqpoint{0.000000in}{0.000000in}}%
\pgfpathlineto{\pgfqpoint{0.000000in}{-0.048611in}}%
\pgfusepath{stroke,fill}%
}%
\begin{pgfscope}%
\pgfsys@transformshift{0.843559in}{0.515123in}%
\pgfsys@useobject{currentmarker}{}%
\end{pgfscope}%
\end{pgfscope}%
\begin{pgfscope}%
\definecolor{textcolor}{rgb}{0.000000,0.000000,0.000000}%
\pgfsetstrokecolor{textcolor}%
\pgfsetfillcolor{textcolor}%
\pgftext[x=0.843559in,y=0.417901in,,top]{\color{textcolor}\rmfamily\fontsize{10.000000}{12.000000}\selectfont \(\displaystyle 0.0\)}%
\end{pgfscope}%
\begin{pgfscope}%
\pgfsetbuttcap%
\pgfsetroundjoin%
\definecolor{currentfill}{rgb}{0.000000,0.000000,0.000000}%
\pgfsetfillcolor{currentfill}%
\pgfsetlinewidth{0.803000pt}%
\definecolor{currentstroke}{rgb}{0.000000,0.000000,0.000000}%
\pgfsetstrokecolor{currentstroke}%
\pgfsetdash{}{0pt}%
\pgfsys@defobject{currentmarker}{\pgfqpoint{0.000000in}{-0.048611in}}{\pgfqpoint{0.000000in}{0.000000in}}{%
\pgfpathmoveto{\pgfqpoint{0.000000in}{0.000000in}}%
\pgfpathlineto{\pgfqpoint{0.000000in}{-0.048611in}}%
\pgfusepath{stroke,fill}%
}%
\begin{pgfscope}%
\pgfsys@transformshift{1.419502in}{0.515123in}%
\pgfsys@useobject{currentmarker}{}%
\end{pgfscope}%
\end{pgfscope}%
\begin{pgfscope}%
\definecolor{textcolor}{rgb}{0.000000,0.000000,0.000000}%
\pgfsetstrokecolor{textcolor}%
\pgfsetfillcolor{textcolor}%
\pgftext[x=1.419502in,y=0.417901in,,top]{\color{textcolor}\rmfamily\fontsize{10.000000}{12.000000}\selectfont \(\displaystyle 2.5\)}%
\end{pgfscope}%
\begin{pgfscope}%
\pgfsetbuttcap%
\pgfsetroundjoin%
\definecolor{currentfill}{rgb}{0.000000,0.000000,0.000000}%
\pgfsetfillcolor{currentfill}%
\pgfsetlinewidth{0.803000pt}%
\definecolor{currentstroke}{rgb}{0.000000,0.000000,0.000000}%
\pgfsetstrokecolor{currentstroke}%
\pgfsetdash{}{0pt}%
\pgfsys@defobject{currentmarker}{\pgfqpoint{0.000000in}{-0.048611in}}{\pgfqpoint{0.000000in}{0.000000in}}{%
\pgfpathmoveto{\pgfqpoint{0.000000in}{0.000000in}}%
\pgfpathlineto{\pgfqpoint{0.000000in}{-0.048611in}}%
\pgfusepath{stroke,fill}%
}%
\begin{pgfscope}%
\pgfsys@transformshift{1.995445in}{0.515123in}%
\pgfsys@useobject{currentmarker}{}%
\end{pgfscope}%
\end{pgfscope}%
\begin{pgfscope}%
\definecolor{textcolor}{rgb}{0.000000,0.000000,0.000000}%
\pgfsetstrokecolor{textcolor}%
\pgfsetfillcolor{textcolor}%
\pgftext[x=1.995445in,y=0.417901in,,top]{\color{textcolor}\rmfamily\fontsize{10.000000}{12.000000}\selectfont \(\displaystyle 5.0\)}%
\end{pgfscope}%
\begin{pgfscope}%
\pgfsetbuttcap%
\pgfsetroundjoin%
\definecolor{currentfill}{rgb}{0.000000,0.000000,0.000000}%
\pgfsetfillcolor{currentfill}%
\pgfsetlinewidth{0.803000pt}%
\definecolor{currentstroke}{rgb}{0.000000,0.000000,0.000000}%
\pgfsetstrokecolor{currentstroke}%
\pgfsetdash{}{0pt}%
\pgfsys@defobject{currentmarker}{\pgfqpoint{0.000000in}{-0.048611in}}{\pgfqpoint{0.000000in}{0.000000in}}{%
\pgfpathmoveto{\pgfqpoint{0.000000in}{0.000000in}}%
\pgfpathlineto{\pgfqpoint{0.000000in}{-0.048611in}}%
\pgfusepath{stroke,fill}%
}%
\begin{pgfscope}%
\pgfsys@transformshift{2.571388in}{0.515123in}%
\pgfsys@useobject{currentmarker}{}%
\end{pgfscope}%
\end{pgfscope}%
\begin{pgfscope}%
\definecolor{textcolor}{rgb}{0.000000,0.000000,0.000000}%
\pgfsetstrokecolor{textcolor}%
\pgfsetfillcolor{textcolor}%
\pgftext[x=2.571388in,y=0.417901in,,top]{\color{textcolor}\rmfamily\fontsize{10.000000}{12.000000}\selectfont \(\displaystyle 7.5\)}%
\end{pgfscope}%
\begin{pgfscope}%
\definecolor{textcolor}{rgb}{0.000000,0.000000,0.000000}%
\pgfsetstrokecolor{textcolor}%
\pgfsetfillcolor{textcolor}%
\pgftext[x=1.740819in,y=0.238889in,,top]{\color{textcolor}\rmfamily\fontsize{10.000000}{12.000000}\selectfont I(\(\displaystyle \mu\)A)}%
\end{pgfscope}%
\begin{pgfscope}%
\pgfsetbuttcap%
\pgfsetroundjoin%
\definecolor{currentfill}{rgb}{0.000000,0.000000,0.000000}%
\pgfsetfillcolor{currentfill}%
\pgfsetlinewidth{0.803000pt}%
\definecolor{currentstroke}{rgb}{0.000000,0.000000,0.000000}%
\pgfsetstrokecolor{currentstroke}%
\pgfsetdash{}{0pt}%
\pgfsys@defobject{currentmarker}{\pgfqpoint{-0.048611in}{0.000000in}}{\pgfqpoint{0.000000in}{0.000000in}}{%
\pgfpathmoveto{\pgfqpoint{0.000000in}{0.000000in}}%
\pgfpathlineto{\pgfqpoint{-0.048611in}{0.000000in}}%
\pgfusepath{stroke,fill}%
}%
\begin{pgfscope}%
\pgfsys@transformshift{0.772069in}{1.118452in}%
\pgfsys@useobject{currentmarker}{}%
\end{pgfscope}%
\end{pgfscope}%
\begin{pgfscope}%
\definecolor{textcolor}{rgb}{0.000000,0.000000,0.000000}%
\pgfsetstrokecolor{textcolor}%
\pgfsetfillcolor{textcolor}%
\pgftext[x=0.605402in,y=1.070226in,left,base]{\color{textcolor}\rmfamily\fontsize{10.000000}{12.000000}\selectfont \(\displaystyle 5\)}%
\end{pgfscope}%
\begin{pgfscope}%
\pgfsetbuttcap%
\pgfsetroundjoin%
\definecolor{currentfill}{rgb}{0.000000,0.000000,0.000000}%
\pgfsetfillcolor{currentfill}%
\pgfsetlinewidth{0.803000pt}%
\definecolor{currentstroke}{rgb}{0.000000,0.000000,0.000000}%
\pgfsetstrokecolor{currentstroke}%
\pgfsetdash{}{0pt}%
\pgfsys@defobject{currentmarker}{\pgfqpoint{-0.048611in}{0.000000in}}{\pgfqpoint{0.000000in}{0.000000in}}{%
\pgfpathmoveto{\pgfqpoint{0.000000in}{0.000000in}}%
\pgfpathlineto{\pgfqpoint{-0.048611in}{0.000000in}}%
\pgfusepath{stroke,fill}%
}%
\begin{pgfscope}%
\pgfsys@transformshift{0.772069in}{1.756076in}%
\pgfsys@useobject{currentmarker}{}%
\end{pgfscope}%
\end{pgfscope}%
\begin{pgfscope}%
\definecolor{textcolor}{rgb}{0.000000,0.000000,0.000000}%
\pgfsetstrokecolor{textcolor}%
\pgfsetfillcolor{textcolor}%
\pgftext[x=0.535957in,y=1.707850in,left,base]{\color{textcolor}\rmfamily\fontsize{10.000000}{12.000000}\selectfont \(\displaystyle 10\)}%
\end{pgfscope}%
\begin{pgfscope}%
\definecolor{textcolor}{rgb}{0.000000,0.000000,0.000000}%
\pgfsetstrokecolor{textcolor}%
\pgfsetfillcolor{textcolor}%
\pgftext[x=0.258179in,y=1.188873in,,bottom]{\color{textcolor}\rmfamily\fontsize{10.000000}{12.000000}\selectfont V(V)}%
\end{pgfscope}%
\begin{pgfscope}%
\pgfpathrectangle{\pgfqpoint{0.772069in}{0.515123in}}{\pgfqpoint{1.937500in}{1.347500in}}%
\pgfusepath{clip}%
\pgfsetbuttcap%
\pgfsetroundjoin%
\definecolor{currentfill}{rgb}{0.121569,0.466667,0.705882}%
\pgfsetfillcolor{currentfill}%
\pgfsetlinewidth{1.003750pt}%
\definecolor{currentstroke}{rgb}{0.121569,0.466667,0.705882}%
\pgfsetstrokecolor{currentstroke}%
\pgfsetdash{}{0pt}%
\pgfpathmoveto{\pgfqpoint{1.004823in}{0.570767in}}%
\pgfpathcurveto{\pgfqpoint{1.015873in}{0.570767in}}{\pgfqpoint{1.026472in}{0.575157in}}{\pgfqpoint{1.034286in}{0.582970in}}%
\pgfpathcurveto{\pgfqpoint{1.042099in}{0.590784in}}{\pgfqpoint{1.046490in}{0.601383in}}{\pgfqpoint{1.046490in}{0.612433in}}%
\pgfpathcurveto{\pgfqpoint{1.046490in}{0.623483in}}{\pgfqpoint{1.042099in}{0.634082in}}{\pgfqpoint{1.034286in}{0.641896in}}%
\pgfpathcurveto{\pgfqpoint{1.026472in}{0.649710in}}{\pgfqpoint{1.015873in}{0.654100in}}{\pgfqpoint{1.004823in}{0.654100in}}%
\pgfpathcurveto{\pgfqpoint{0.993773in}{0.654100in}}{\pgfqpoint{0.983174in}{0.649710in}}{\pgfqpoint{0.975360in}{0.641896in}}%
\pgfpathcurveto{\pgfqpoint{0.967547in}{0.634082in}}{\pgfqpoint{0.963156in}{0.623483in}}{\pgfqpoint{0.963156in}{0.612433in}}%
\pgfpathcurveto{\pgfqpoint{0.963156in}{0.601383in}}{\pgfqpoint{0.967547in}{0.590784in}}{\pgfqpoint{0.975360in}{0.582970in}}%
\pgfpathcurveto{\pgfqpoint{0.983174in}{0.575157in}}{\pgfqpoint{0.993773in}{0.570767in}}{\pgfqpoint{1.004823in}{0.570767in}}%
\pgfpathclose%
\pgfusepath{stroke,fill}%
\end{pgfscope}%
\begin{pgfscope}%
\pgfpathrectangle{\pgfqpoint{0.772069in}{0.515123in}}{\pgfqpoint{1.937500in}{1.347500in}}%
\pgfusepath{clip}%
\pgfsetbuttcap%
\pgfsetroundjoin%
\definecolor{currentfill}{rgb}{0.121569,0.466667,0.705882}%
\pgfsetfillcolor{currentfill}%
\pgfsetlinewidth{1.003750pt}%
\definecolor{currentstroke}{rgb}{0.121569,0.466667,0.705882}%
\pgfsetstrokecolor{currentstroke}%
\pgfsetdash{}{0pt}%
\pgfpathmoveto{\pgfqpoint{1.189125in}{0.694211in}}%
\pgfpathcurveto{\pgfqpoint{1.200175in}{0.694211in}}{\pgfqpoint{1.210774in}{0.698601in}}{\pgfqpoint{1.218588in}{0.706414in}}%
\pgfpathcurveto{\pgfqpoint{1.226401in}{0.714228in}}{\pgfqpoint{1.230792in}{0.724827in}}{\pgfqpoint{1.230792in}{0.735877in}}%
\pgfpathcurveto{\pgfqpoint{1.230792in}{0.746927in}}{\pgfqpoint{1.226401in}{0.757526in}}{\pgfqpoint{1.218588in}{0.765340in}}%
\pgfpathcurveto{\pgfqpoint{1.210774in}{0.773154in}}{\pgfqpoint{1.200175in}{0.777544in}}{\pgfqpoint{1.189125in}{0.777544in}}%
\pgfpathcurveto{\pgfqpoint{1.178075in}{0.777544in}}{\pgfqpoint{1.167476in}{0.773154in}}{\pgfqpoint{1.159662in}{0.765340in}}%
\pgfpathcurveto{\pgfqpoint{1.151848in}{0.757526in}}{\pgfqpoint{1.147458in}{0.746927in}}{\pgfqpoint{1.147458in}{0.735877in}}%
\pgfpathcurveto{\pgfqpoint{1.147458in}{0.724827in}}{\pgfqpoint{1.151848in}{0.714228in}}{\pgfqpoint{1.159662in}{0.706414in}}%
\pgfpathcurveto{\pgfqpoint{1.167476in}{0.698601in}}{\pgfqpoint{1.178075in}{0.694211in}}{\pgfqpoint{1.189125in}{0.694211in}}%
\pgfpathclose%
\pgfusepath{stroke,fill}%
\end{pgfscope}%
\begin{pgfscope}%
\pgfpathrectangle{\pgfqpoint{0.772069in}{0.515123in}}{\pgfqpoint{1.937500in}{1.347500in}}%
\pgfusepath{clip}%
\pgfsetbuttcap%
\pgfsetroundjoin%
\definecolor{currentfill}{rgb}{0.121569,0.466667,0.705882}%
\pgfsetfillcolor{currentfill}%
\pgfsetlinewidth{1.003750pt}%
\definecolor{currentstroke}{rgb}{0.121569,0.466667,0.705882}%
\pgfsetstrokecolor{currentstroke}%
\pgfsetdash{}{0pt}%
\pgfpathmoveto{\pgfqpoint{1.350389in}{0.826836in}}%
\pgfpathcurveto{\pgfqpoint{1.361439in}{0.826836in}}{\pgfqpoint{1.372038in}{0.831227in}}{\pgfqpoint{1.379852in}{0.839040in}}%
\pgfpathcurveto{\pgfqpoint{1.387665in}{0.846854in}}{\pgfqpoint{1.392056in}{0.857453in}}{\pgfqpoint{1.392056in}{0.868503in}}%
\pgfpathcurveto{\pgfqpoint{1.392056in}{0.879553in}}{\pgfqpoint{1.387665in}{0.890152in}}{\pgfqpoint{1.379852in}{0.897966in}}%
\pgfpathcurveto{\pgfqpoint{1.372038in}{0.905779in}}{\pgfqpoint{1.361439in}{0.910170in}}{\pgfqpoint{1.350389in}{0.910170in}}%
\pgfpathcurveto{\pgfqpoint{1.339339in}{0.910170in}}{\pgfqpoint{1.328740in}{0.905779in}}{\pgfqpoint{1.320926in}{0.897966in}}%
\pgfpathcurveto{\pgfqpoint{1.313113in}{0.890152in}}{\pgfqpoint{1.308722in}{0.879553in}}{\pgfqpoint{1.308722in}{0.868503in}}%
\pgfpathcurveto{\pgfqpoint{1.308722in}{0.857453in}}{\pgfqpoint{1.313113in}{0.846854in}}{\pgfqpoint{1.320926in}{0.839040in}}%
\pgfpathcurveto{\pgfqpoint{1.328740in}{0.831227in}}{\pgfqpoint{1.339339in}{0.826836in}}{\pgfqpoint{1.350389in}{0.826836in}}%
\pgfpathclose%
\pgfusepath{stroke,fill}%
\end{pgfscope}%
\begin{pgfscope}%
\pgfpathrectangle{\pgfqpoint{0.772069in}{0.515123in}}{\pgfqpoint{1.937500in}{1.347500in}}%
\pgfusepath{clip}%
\pgfsetbuttcap%
\pgfsetroundjoin%
\definecolor{currentfill}{rgb}{0.121569,0.466667,0.705882}%
\pgfsetfillcolor{currentfill}%
\pgfsetlinewidth{1.003750pt}%
\definecolor{currentstroke}{rgb}{0.121569,0.466667,0.705882}%
\pgfsetstrokecolor{currentstroke}%
\pgfsetdash{}{0pt}%
\pgfpathmoveto{\pgfqpoint{1.534691in}{0.956912in}}%
\pgfpathcurveto{\pgfqpoint{1.545741in}{0.956912in}}{\pgfqpoint{1.556340in}{0.961302in}}{\pgfqpoint{1.564153in}{0.969116in}}%
\pgfpathcurveto{\pgfqpoint{1.571967in}{0.976929in}}{\pgfqpoint{1.576357in}{0.987528in}}{\pgfqpoint{1.576357in}{0.998578in}}%
\pgfpathcurveto{\pgfqpoint{1.576357in}{1.009628in}}{\pgfqpoint{1.571967in}{1.020227in}}{\pgfqpoint{1.564153in}{1.028041in}}%
\pgfpathcurveto{\pgfqpoint{1.556340in}{1.035855in}}{\pgfqpoint{1.545741in}{1.040245in}}{\pgfqpoint{1.534691in}{1.040245in}}%
\pgfpathcurveto{\pgfqpoint{1.523641in}{1.040245in}}{\pgfqpoint{1.513042in}{1.035855in}}{\pgfqpoint{1.505228in}{1.028041in}}%
\pgfpathcurveto{\pgfqpoint{1.497414in}{1.020227in}}{\pgfqpoint{1.493024in}{1.009628in}}{\pgfqpoint{1.493024in}{0.998578in}}%
\pgfpathcurveto{\pgfqpoint{1.493024in}{0.987528in}}{\pgfqpoint{1.497414in}{0.976929in}}{\pgfqpoint{1.505228in}{0.969116in}}%
\pgfpathcurveto{\pgfqpoint{1.513042in}{0.961302in}}{\pgfqpoint{1.523641in}{0.956912in}}{\pgfqpoint{1.534691in}{0.956912in}}%
\pgfpathclose%
\pgfusepath{stroke,fill}%
\end{pgfscope}%
\begin{pgfscope}%
\pgfpathrectangle{\pgfqpoint{0.772069in}{0.515123in}}{\pgfqpoint{1.937500in}{1.347500in}}%
\pgfusepath{clip}%
\pgfsetbuttcap%
\pgfsetroundjoin%
\definecolor{currentfill}{rgb}{0.121569,0.466667,0.705882}%
\pgfsetfillcolor{currentfill}%
\pgfsetlinewidth{1.003750pt}%
\definecolor{currentstroke}{rgb}{0.121569,0.466667,0.705882}%
\pgfsetstrokecolor{currentstroke}%
\pgfsetdash{}{0pt}%
\pgfpathmoveto{\pgfqpoint{1.718992in}{1.086987in}}%
\pgfpathcurveto{\pgfqpoint{1.730043in}{1.086987in}}{\pgfqpoint{1.740642in}{1.091377in}}{\pgfqpoint{1.748455in}{1.099191in}}%
\pgfpathcurveto{\pgfqpoint{1.756269in}{1.107004in}}{\pgfqpoint{1.760659in}{1.117604in}}{\pgfqpoint{1.760659in}{1.128654in}}%
\pgfpathcurveto{\pgfqpoint{1.760659in}{1.139704in}}{\pgfqpoint{1.756269in}{1.150303in}}{\pgfqpoint{1.748455in}{1.158116in}}%
\pgfpathcurveto{\pgfqpoint{1.740642in}{1.165930in}}{\pgfqpoint{1.730043in}{1.170320in}}{\pgfqpoint{1.718992in}{1.170320in}}%
\pgfpathcurveto{\pgfqpoint{1.707942in}{1.170320in}}{\pgfqpoint{1.697343in}{1.165930in}}{\pgfqpoint{1.689530in}{1.158116in}}%
\pgfpathcurveto{\pgfqpoint{1.681716in}{1.150303in}}{\pgfqpoint{1.677326in}{1.139704in}}{\pgfqpoint{1.677326in}{1.128654in}}%
\pgfpathcurveto{\pgfqpoint{1.677326in}{1.117604in}}{\pgfqpoint{1.681716in}{1.107004in}}{\pgfqpoint{1.689530in}{1.099191in}}%
\pgfpathcurveto{\pgfqpoint{1.697343in}{1.091377in}}{\pgfqpoint{1.707942in}{1.086987in}}{\pgfqpoint{1.718992in}{1.086987in}}%
\pgfpathclose%
\pgfusepath{stroke,fill}%
\end{pgfscope}%
\begin{pgfscope}%
\pgfpathrectangle{\pgfqpoint{0.772069in}{0.515123in}}{\pgfqpoint{1.937500in}{1.347500in}}%
\pgfusepath{clip}%
\pgfsetbuttcap%
\pgfsetroundjoin%
\definecolor{currentfill}{rgb}{0.121569,0.466667,0.705882}%
\pgfsetfillcolor{currentfill}%
\pgfsetlinewidth{1.003750pt}%
\definecolor{currentstroke}{rgb}{0.121569,0.466667,0.705882}%
\pgfsetstrokecolor{currentstroke}%
\pgfsetdash{}{0pt}%
\pgfpathmoveto{\pgfqpoint{1.880257in}{1.210686in}}%
\pgfpathcurveto{\pgfqpoint{1.891307in}{1.210686in}}{\pgfqpoint{1.901906in}{1.215076in}}{\pgfqpoint{1.909719in}{1.222890in}}%
\pgfpathcurveto{\pgfqpoint{1.917533in}{1.230704in}}{\pgfqpoint{1.921923in}{1.241303in}}{\pgfqpoint{1.921923in}{1.252353in}}%
\pgfpathcurveto{\pgfqpoint{1.921923in}{1.263403in}}{\pgfqpoint{1.917533in}{1.274002in}}{\pgfqpoint{1.909719in}{1.281815in}}%
\pgfpathcurveto{\pgfqpoint{1.901906in}{1.289629in}}{\pgfqpoint{1.891307in}{1.294019in}}{\pgfqpoint{1.880257in}{1.294019in}}%
\pgfpathcurveto{\pgfqpoint{1.869206in}{1.294019in}}{\pgfqpoint{1.858607in}{1.289629in}}{\pgfqpoint{1.850794in}{1.281815in}}%
\pgfpathcurveto{\pgfqpoint{1.842980in}{1.274002in}}{\pgfqpoint{1.838590in}{1.263403in}}{\pgfqpoint{1.838590in}{1.252353in}}%
\pgfpathcurveto{\pgfqpoint{1.838590in}{1.241303in}}{\pgfqpoint{1.842980in}{1.230704in}}{\pgfqpoint{1.850794in}{1.222890in}}%
\pgfpathcurveto{\pgfqpoint{1.858607in}{1.215076in}}{\pgfqpoint{1.869206in}{1.210686in}}{\pgfqpoint{1.880257in}{1.210686in}}%
\pgfpathclose%
\pgfusepath{stroke,fill}%
\end{pgfscope}%
\begin{pgfscope}%
\pgfpathrectangle{\pgfqpoint{0.772069in}{0.515123in}}{\pgfqpoint{1.937500in}{1.347500in}}%
\pgfusepath{clip}%
\pgfsetbuttcap%
\pgfsetroundjoin%
\definecolor{currentfill}{rgb}{0.121569,0.466667,0.705882}%
\pgfsetfillcolor{currentfill}%
\pgfsetlinewidth{1.003750pt}%
\definecolor{currentstroke}{rgb}{0.121569,0.466667,0.705882}%
\pgfsetstrokecolor{currentstroke}%
\pgfsetdash{}{0pt}%
\pgfpathmoveto{\pgfqpoint{2.064558in}{1.344587in}}%
\pgfpathcurveto{\pgfqpoint{2.075608in}{1.344587in}}{\pgfqpoint{2.086207in}{1.348977in}}{\pgfqpoint{2.094021in}{1.356791in}}%
\pgfpathcurveto{\pgfqpoint{2.101835in}{1.364605in}}{\pgfqpoint{2.106225in}{1.375204in}}{\pgfqpoint{2.106225in}{1.386254in}}%
\pgfpathcurveto{\pgfqpoint{2.106225in}{1.397304in}}{\pgfqpoint{2.101835in}{1.407903in}}{\pgfqpoint{2.094021in}{1.415717in}}%
\pgfpathcurveto{\pgfqpoint{2.086207in}{1.423530in}}{\pgfqpoint{2.075608in}{1.427920in}}{\pgfqpoint{2.064558in}{1.427920in}}%
\pgfpathcurveto{\pgfqpoint{2.053508in}{1.427920in}}{\pgfqpoint{2.042909in}{1.423530in}}{\pgfqpoint{2.035096in}{1.415717in}}%
\pgfpathcurveto{\pgfqpoint{2.027282in}{1.407903in}}{\pgfqpoint{2.022892in}{1.397304in}}{\pgfqpoint{2.022892in}{1.386254in}}%
\pgfpathcurveto{\pgfqpoint{2.022892in}{1.375204in}}{\pgfqpoint{2.027282in}{1.364605in}}{\pgfqpoint{2.035096in}{1.356791in}}%
\pgfpathcurveto{\pgfqpoint{2.042909in}{1.348977in}}{\pgfqpoint{2.053508in}{1.344587in}}{\pgfqpoint{2.064558in}{1.344587in}}%
\pgfpathclose%
\pgfusepath{stroke,fill}%
\end{pgfscope}%
\begin{pgfscope}%
\pgfpathrectangle{\pgfqpoint{0.772069in}{0.515123in}}{\pgfqpoint{1.937500in}{1.347500in}}%
\pgfusepath{clip}%
\pgfsetbuttcap%
\pgfsetroundjoin%
\definecolor{currentfill}{rgb}{0.121569,0.466667,0.705882}%
\pgfsetfillcolor{currentfill}%
\pgfsetlinewidth{1.003750pt}%
\definecolor{currentstroke}{rgb}{0.121569,0.466667,0.705882}%
\pgfsetstrokecolor{currentstroke}%
\pgfsetdash{}{0pt}%
\pgfpathmoveto{\pgfqpoint{2.248860in}{1.473387in}}%
\pgfpathcurveto{\pgfqpoint{2.259910in}{1.473387in}}{\pgfqpoint{2.270509in}{1.477777in}}{\pgfqpoint{2.278323in}{1.485591in}}%
\pgfpathcurveto{\pgfqpoint{2.286137in}{1.493405in}}{\pgfqpoint{2.290527in}{1.504004in}}{\pgfqpoint{2.290527in}{1.515054in}}%
\pgfpathcurveto{\pgfqpoint{2.290527in}{1.526104in}}{\pgfqpoint{2.286137in}{1.536703in}}{\pgfqpoint{2.278323in}{1.544517in}}%
\pgfpathcurveto{\pgfqpoint{2.270509in}{1.552330in}}{\pgfqpoint{2.259910in}{1.556720in}}{\pgfqpoint{2.248860in}{1.556720in}}%
\pgfpathcurveto{\pgfqpoint{2.237810in}{1.556720in}}{\pgfqpoint{2.227211in}{1.552330in}}{\pgfqpoint{2.219397in}{1.544517in}}%
\pgfpathcurveto{\pgfqpoint{2.211584in}{1.536703in}}{\pgfqpoint{2.207193in}{1.526104in}}{\pgfqpoint{2.207193in}{1.515054in}}%
\pgfpathcurveto{\pgfqpoint{2.207193in}{1.504004in}}{\pgfqpoint{2.211584in}{1.493405in}}{\pgfqpoint{2.219397in}{1.485591in}}%
\pgfpathcurveto{\pgfqpoint{2.227211in}{1.477777in}}{\pgfqpoint{2.237810in}{1.473387in}}{\pgfqpoint{2.248860in}{1.473387in}}%
\pgfpathclose%
\pgfusepath{stroke,fill}%
\end{pgfscope}%
\begin{pgfscope}%
\pgfpathrectangle{\pgfqpoint{0.772069in}{0.515123in}}{\pgfqpoint{1.937500in}{1.347500in}}%
\pgfusepath{clip}%
\pgfsetbuttcap%
\pgfsetroundjoin%
\definecolor{currentfill}{rgb}{0.121569,0.466667,0.705882}%
\pgfsetfillcolor{currentfill}%
\pgfsetlinewidth{1.003750pt}%
\definecolor{currentstroke}{rgb}{0.121569,0.466667,0.705882}%
\pgfsetstrokecolor{currentstroke}%
\pgfsetdash{}{0pt}%
\pgfpathmoveto{\pgfqpoint{2.410124in}{1.599637in}}%
\pgfpathcurveto{\pgfqpoint{2.421174in}{1.599637in}}{\pgfqpoint{2.431773in}{1.604027in}}{\pgfqpoint{2.439587in}{1.611841in}}%
\pgfpathcurveto{\pgfqpoint{2.447401in}{1.619654in}}{\pgfqpoint{2.451791in}{1.630253in}}{\pgfqpoint{2.451791in}{1.641303in}}%
\pgfpathcurveto{\pgfqpoint{2.451791in}{1.652353in}}{\pgfqpoint{2.447401in}{1.662953in}}{\pgfqpoint{2.439587in}{1.670766in}}%
\pgfpathcurveto{\pgfqpoint{2.431773in}{1.678580in}}{\pgfqpoint{2.421174in}{1.682970in}}{\pgfqpoint{2.410124in}{1.682970in}}%
\pgfpathcurveto{\pgfqpoint{2.399074in}{1.682970in}}{\pgfqpoint{2.388475in}{1.678580in}}{\pgfqpoint{2.380661in}{1.670766in}}%
\pgfpathcurveto{\pgfqpoint{2.372848in}{1.662953in}}{\pgfqpoint{2.368458in}{1.652353in}}{\pgfqpoint{2.368458in}{1.641303in}}%
\pgfpathcurveto{\pgfqpoint{2.368458in}{1.630253in}}{\pgfqpoint{2.372848in}{1.619654in}}{\pgfqpoint{2.380661in}{1.611841in}}%
\pgfpathcurveto{\pgfqpoint{2.388475in}{1.604027in}}{\pgfqpoint{2.399074in}{1.599637in}}{\pgfqpoint{2.410124in}{1.599637in}}%
\pgfpathclose%
\pgfusepath{stroke,fill}%
\end{pgfscope}%
\begin{pgfscope}%
\pgfpathrectangle{\pgfqpoint{0.772069in}{0.515123in}}{\pgfqpoint{1.937500in}{1.347500in}}%
\pgfusepath{clip}%
\pgfsetbuttcap%
\pgfsetroundjoin%
\definecolor{currentfill}{rgb}{0.121569,0.466667,0.705882}%
\pgfsetfillcolor{currentfill}%
\pgfsetlinewidth{1.003750pt}%
\definecolor{currentstroke}{rgb}{0.121569,0.466667,0.705882}%
\pgfsetstrokecolor{currentstroke}%
\pgfsetdash{}{0pt}%
\pgfpathmoveto{\pgfqpoint{2.594426in}{1.729712in}}%
\pgfpathcurveto{\pgfqpoint{2.605476in}{1.729712in}}{\pgfqpoint{2.616075in}{1.734102in}}{\pgfqpoint{2.623889in}{1.741916in}}%
\pgfpathcurveto{\pgfqpoint{2.631702in}{1.749729in}}{\pgfqpoint{2.636093in}{1.760329in}}{\pgfqpoint{2.636093in}{1.771379in}}%
\pgfpathcurveto{\pgfqpoint{2.636093in}{1.782429in}}{\pgfqpoint{2.631702in}{1.793028in}}{\pgfqpoint{2.623889in}{1.800841in}}%
\pgfpathcurveto{\pgfqpoint{2.616075in}{1.808655in}}{\pgfqpoint{2.605476in}{1.813045in}}{\pgfqpoint{2.594426in}{1.813045in}}%
\pgfpathcurveto{\pgfqpoint{2.583376in}{1.813045in}}{\pgfqpoint{2.572777in}{1.808655in}}{\pgfqpoint{2.564963in}{1.800841in}}%
\pgfpathcurveto{\pgfqpoint{2.557150in}{1.793028in}}{\pgfqpoint{2.552759in}{1.782429in}}{\pgfqpoint{2.552759in}{1.771379in}}%
\pgfpathcurveto{\pgfqpoint{2.552759in}{1.760329in}}{\pgfqpoint{2.557150in}{1.749729in}}{\pgfqpoint{2.564963in}{1.741916in}}%
\pgfpathcurveto{\pgfqpoint{2.572777in}{1.734102in}}{\pgfqpoint{2.583376in}{1.729712in}}{\pgfqpoint{2.594426in}{1.729712in}}%
\pgfpathclose%
\pgfusepath{stroke,fill}%
\end{pgfscope}%
\begin{pgfscope}%
\pgfpathrectangle{\pgfqpoint{0.772069in}{0.515123in}}{\pgfqpoint{1.937500in}{1.347500in}}%
\pgfusepath{clip}%
\pgfsetbuttcap%
\pgfsetroundjoin%
\definecolor{currentfill}{rgb}{0.117647,0.564706,1.000000}%
\pgfsetfillcolor{currentfill}%
\pgfsetlinewidth{1.003750pt}%
\definecolor{currentstroke}{rgb}{0.117647,0.564706,1.000000}%
\pgfsetstrokecolor{currentstroke}%
\pgfsetdash{}{0pt}%
\pgfpathmoveto{\pgfqpoint{0.935710in}{0.570767in}}%
\pgfpathcurveto{\pgfqpoint{0.946760in}{0.570767in}}{\pgfqpoint{0.957359in}{0.575157in}}{\pgfqpoint{0.965173in}{0.582970in}}%
\pgfpathcurveto{\pgfqpoint{0.972986in}{0.590784in}}{\pgfqpoint{0.977377in}{0.601383in}}{\pgfqpoint{0.977377in}{0.612433in}}%
\pgfpathcurveto{\pgfqpoint{0.977377in}{0.623483in}}{\pgfqpoint{0.972986in}{0.634082in}}{\pgfqpoint{0.965173in}{0.641896in}}%
\pgfpathcurveto{\pgfqpoint{0.957359in}{0.649710in}}{\pgfqpoint{0.946760in}{0.654100in}}{\pgfqpoint{0.935710in}{0.654100in}}%
\pgfpathcurveto{\pgfqpoint{0.924660in}{0.654100in}}{\pgfqpoint{0.914061in}{0.649710in}}{\pgfqpoint{0.906247in}{0.641896in}}%
\pgfpathcurveto{\pgfqpoint{0.898433in}{0.634082in}}{\pgfqpoint{0.894043in}{0.623483in}}{\pgfqpoint{0.894043in}{0.612433in}}%
\pgfpathcurveto{\pgfqpoint{0.894043in}{0.601383in}}{\pgfqpoint{0.898433in}{0.590784in}}{\pgfqpoint{0.906247in}{0.582970in}}%
\pgfpathcurveto{\pgfqpoint{0.914061in}{0.575157in}}{\pgfqpoint{0.924660in}{0.570767in}}{\pgfqpoint{0.935710in}{0.570767in}}%
\pgfpathclose%
\pgfusepath{stroke,fill}%
\end{pgfscope}%
\begin{pgfscope}%
\pgfpathrectangle{\pgfqpoint{0.772069in}{0.515123in}}{\pgfqpoint{1.937500in}{1.347500in}}%
\pgfusepath{clip}%
\pgfsetbuttcap%
\pgfsetroundjoin%
\definecolor{currentfill}{rgb}{0.117647,0.564706,1.000000}%
\pgfsetfillcolor{currentfill}%
\pgfsetlinewidth{1.003750pt}%
\definecolor{currentstroke}{rgb}{0.117647,0.564706,1.000000}%
\pgfsetstrokecolor{currentstroke}%
\pgfsetdash{}{0pt}%
\pgfpathmoveto{\pgfqpoint{1.050899in}{0.694211in}}%
\pgfpathcurveto{\pgfqpoint{1.061949in}{0.694211in}}{\pgfqpoint{1.072548in}{0.698601in}}{\pgfqpoint{1.080361in}{0.706414in}}%
\pgfpathcurveto{\pgfqpoint{1.088175in}{0.714228in}}{\pgfqpoint{1.092565in}{0.724827in}}{\pgfqpoint{1.092565in}{0.735877in}}%
\pgfpathcurveto{\pgfqpoint{1.092565in}{0.746927in}}{\pgfqpoint{1.088175in}{0.757526in}}{\pgfqpoint{1.080361in}{0.765340in}}%
\pgfpathcurveto{\pgfqpoint{1.072548in}{0.773154in}}{\pgfqpoint{1.061949in}{0.777544in}}{\pgfqpoint{1.050899in}{0.777544in}}%
\pgfpathcurveto{\pgfqpoint{1.039848in}{0.777544in}}{\pgfqpoint{1.029249in}{0.773154in}}{\pgfqpoint{1.021436in}{0.765340in}}%
\pgfpathcurveto{\pgfqpoint{1.013622in}{0.757526in}}{\pgfqpoint{1.009232in}{0.746927in}}{\pgfqpoint{1.009232in}{0.735877in}}%
\pgfpathcurveto{\pgfqpoint{1.009232in}{0.724827in}}{\pgfqpoint{1.013622in}{0.714228in}}{\pgfqpoint{1.021436in}{0.706414in}}%
\pgfpathcurveto{\pgfqpoint{1.029249in}{0.698601in}}{\pgfqpoint{1.039848in}{0.694211in}}{\pgfqpoint{1.050899in}{0.694211in}}%
\pgfpathclose%
\pgfusepath{stroke,fill}%
\end{pgfscope}%
\begin{pgfscope}%
\pgfpathrectangle{\pgfqpoint{0.772069in}{0.515123in}}{\pgfqpoint{1.937500in}{1.347500in}}%
\pgfusepath{clip}%
\pgfsetbuttcap%
\pgfsetroundjoin%
\definecolor{currentfill}{rgb}{0.117647,0.564706,1.000000}%
\pgfsetfillcolor{currentfill}%
\pgfsetlinewidth{1.003750pt}%
\definecolor{currentstroke}{rgb}{0.117647,0.564706,1.000000}%
\pgfsetstrokecolor{currentstroke}%
\pgfsetdash{}{0pt}%
\pgfpathmoveto{\pgfqpoint{1.166087in}{0.826836in}}%
\pgfpathcurveto{\pgfqpoint{1.177137in}{0.826836in}}{\pgfqpoint{1.187736in}{0.831227in}}{\pgfqpoint{1.195550in}{0.839040in}}%
\pgfpathcurveto{\pgfqpoint{1.203364in}{0.846854in}}{\pgfqpoint{1.207754in}{0.857453in}}{\pgfqpoint{1.207754in}{0.868503in}}%
\pgfpathcurveto{\pgfqpoint{1.207754in}{0.879553in}}{\pgfqpoint{1.203364in}{0.890152in}}{\pgfqpoint{1.195550in}{0.897966in}}%
\pgfpathcurveto{\pgfqpoint{1.187736in}{0.905779in}}{\pgfqpoint{1.177137in}{0.910170in}}{\pgfqpoint{1.166087in}{0.910170in}}%
\pgfpathcurveto{\pgfqpoint{1.155037in}{0.910170in}}{\pgfqpoint{1.144438in}{0.905779in}}{\pgfqpoint{1.136624in}{0.897966in}}%
\pgfpathcurveto{\pgfqpoint{1.128811in}{0.890152in}}{\pgfqpoint{1.124420in}{0.879553in}}{\pgfqpoint{1.124420in}{0.868503in}}%
\pgfpathcurveto{\pgfqpoint{1.124420in}{0.857453in}}{\pgfqpoint{1.128811in}{0.846854in}}{\pgfqpoint{1.136624in}{0.839040in}}%
\pgfpathcurveto{\pgfqpoint{1.144438in}{0.831227in}}{\pgfqpoint{1.155037in}{0.826836in}}{\pgfqpoint{1.166087in}{0.826836in}}%
\pgfpathclose%
\pgfusepath{stroke,fill}%
\end{pgfscope}%
\begin{pgfscope}%
\pgfpathrectangle{\pgfqpoint{0.772069in}{0.515123in}}{\pgfqpoint{1.937500in}{1.347500in}}%
\pgfusepath{clip}%
\pgfsetbuttcap%
\pgfsetroundjoin%
\definecolor{currentfill}{rgb}{0.117647,0.564706,1.000000}%
\pgfsetfillcolor{currentfill}%
\pgfsetlinewidth{1.003750pt}%
\definecolor{currentstroke}{rgb}{0.117647,0.564706,1.000000}%
\pgfsetstrokecolor{currentstroke}%
\pgfsetdash{}{0pt}%
\pgfpathmoveto{\pgfqpoint{1.281276in}{0.956912in}}%
\pgfpathcurveto{\pgfqpoint{1.292326in}{0.956912in}}{\pgfqpoint{1.302925in}{0.961302in}}{\pgfqpoint{1.310739in}{0.969116in}}%
\pgfpathcurveto{\pgfqpoint{1.318552in}{0.976929in}}{\pgfqpoint{1.322942in}{0.987528in}}{\pgfqpoint{1.322942in}{0.998578in}}%
\pgfpathcurveto{\pgfqpoint{1.322942in}{1.009628in}}{\pgfqpoint{1.318552in}{1.020227in}}{\pgfqpoint{1.310739in}{1.028041in}}%
\pgfpathcurveto{\pgfqpoint{1.302925in}{1.035855in}}{\pgfqpoint{1.292326in}{1.040245in}}{\pgfqpoint{1.281276in}{1.040245in}}%
\pgfpathcurveto{\pgfqpoint{1.270226in}{1.040245in}}{\pgfqpoint{1.259627in}{1.035855in}}{\pgfqpoint{1.251813in}{1.028041in}}%
\pgfpathcurveto{\pgfqpoint{1.243999in}{1.020227in}}{\pgfqpoint{1.239609in}{1.009628in}}{\pgfqpoint{1.239609in}{0.998578in}}%
\pgfpathcurveto{\pgfqpoint{1.239609in}{0.987528in}}{\pgfqpoint{1.243999in}{0.976929in}}{\pgfqpoint{1.251813in}{0.969116in}}%
\pgfpathcurveto{\pgfqpoint{1.259627in}{0.961302in}}{\pgfqpoint{1.270226in}{0.956912in}}{\pgfqpoint{1.281276in}{0.956912in}}%
\pgfpathclose%
\pgfusepath{stroke,fill}%
\end{pgfscope}%
\begin{pgfscope}%
\pgfpathrectangle{\pgfqpoint{0.772069in}{0.515123in}}{\pgfqpoint{1.937500in}{1.347500in}}%
\pgfusepath{clip}%
\pgfsetbuttcap%
\pgfsetroundjoin%
\definecolor{currentfill}{rgb}{0.117647,0.564706,1.000000}%
\pgfsetfillcolor{currentfill}%
\pgfsetlinewidth{1.003750pt}%
\definecolor{currentstroke}{rgb}{0.117647,0.564706,1.000000}%
\pgfsetstrokecolor{currentstroke}%
\pgfsetdash{}{0pt}%
\pgfpathmoveto{\pgfqpoint{1.396464in}{1.086987in}}%
\pgfpathcurveto{\pgfqpoint{1.407514in}{1.086987in}}{\pgfqpoint{1.418114in}{1.091377in}}{\pgfqpoint{1.425927in}{1.099191in}}%
\pgfpathcurveto{\pgfqpoint{1.433741in}{1.107004in}}{\pgfqpoint{1.438131in}{1.117604in}}{\pgfqpoint{1.438131in}{1.128654in}}%
\pgfpathcurveto{\pgfqpoint{1.438131in}{1.139704in}}{\pgfqpoint{1.433741in}{1.150303in}}{\pgfqpoint{1.425927in}{1.158116in}}%
\pgfpathcurveto{\pgfqpoint{1.418114in}{1.165930in}}{\pgfqpoint{1.407514in}{1.170320in}}{\pgfqpoint{1.396464in}{1.170320in}}%
\pgfpathcurveto{\pgfqpoint{1.385414in}{1.170320in}}{\pgfqpoint{1.374815in}{1.165930in}}{\pgfqpoint{1.367002in}{1.158116in}}%
\pgfpathcurveto{\pgfqpoint{1.359188in}{1.150303in}}{\pgfqpoint{1.354798in}{1.139704in}}{\pgfqpoint{1.354798in}{1.128654in}}%
\pgfpathcurveto{\pgfqpoint{1.354798in}{1.117604in}}{\pgfqpoint{1.359188in}{1.107004in}}{\pgfqpoint{1.367002in}{1.099191in}}%
\pgfpathcurveto{\pgfqpoint{1.374815in}{1.091377in}}{\pgfqpoint{1.385414in}{1.086987in}}{\pgfqpoint{1.396464in}{1.086987in}}%
\pgfpathclose%
\pgfusepath{stroke,fill}%
\end{pgfscope}%
\begin{pgfscope}%
\pgfpathrectangle{\pgfqpoint{0.772069in}{0.515123in}}{\pgfqpoint{1.937500in}{1.347500in}}%
\pgfusepath{clip}%
\pgfsetbuttcap%
\pgfsetroundjoin%
\definecolor{currentfill}{rgb}{0.117647,0.564706,1.000000}%
\pgfsetfillcolor{currentfill}%
\pgfsetlinewidth{1.003750pt}%
\definecolor{currentstroke}{rgb}{0.117647,0.564706,1.000000}%
\pgfsetstrokecolor{currentstroke}%
\pgfsetdash{}{0pt}%
\pgfpathmoveto{\pgfqpoint{1.511653in}{1.210686in}}%
\pgfpathcurveto{\pgfqpoint{1.522703in}{1.210686in}}{\pgfqpoint{1.533302in}{1.215076in}}{\pgfqpoint{1.541116in}{1.222890in}}%
\pgfpathcurveto{\pgfqpoint{1.548929in}{1.230704in}}{\pgfqpoint{1.553320in}{1.241303in}}{\pgfqpoint{1.553320in}{1.252353in}}%
\pgfpathcurveto{\pgfqpoint{1.553320in}{1.263403in}}{\pgfqpoint{1.548929in}{1.274002in}}{\pgfqpoint{1.541116in}{1.281815in}}%
\pgfpathcurveto{\pgfqpoint{1.533302in}{1.289629in}}{\pgfqpoint{1.522703in}{1.294019in}}{\pgfqpoint{1.511653in}{1.294019in}}%
\pgfpathcurveto{\pgfqpoint{1.500603in}{1.294019in}}{\pgfqpoint{1.490004in}{1.289629in}}{\pgfqpoint{1.482190in}{1.281815in}}%
\pgfpathcurveto{\pgfqpoint{1.474377in}{1.274002in}}{\pgfqpoint{1.469986in}{1.263403in}}{\pgfqpoint{1.469986in}{1.252353in}}%
\pgfpathcurveto{\pgfqpoint{1.469986in}{1.241303in}}{\pgfqpoint{1.474377in}{1.230704in}}{\pgfqpoint{1.482190in}{1.222890in}}%
\pgfpathcurveto{\pgfqpoint{1.490004in}{1.215076in}}{\pgfqpoint{1.500603in}{1.210686in}}{\pgfqpoint{1.511653in}{1.210686in}}%
\pgfpathclose%
\pgfusepath{stroke,fill}%
\end{pgfscope}%
\begin{pgfscope}%
\pgfpathrectangle{\pgfqpoint{0.772069in}{0.515123in}}{\pgfqpoint{1.937500in}{1.347500in}}%
\pgfusepath{clip}%
\pgfsetbuttcap%
\pgfsetroundjoin%
\definecolor{currentfill}{rgb}{0.117647,0.564706,1.000000}%
\pgfsetfillcolor{currentfill}%
\pgfsetlinewidth{1.003750pt}%
\definecolor{currentstroke}{rgb}{0.117647,0.564706,1.000000}%
\pgfsetstrokecolor{currentstroke}%
\pgfsetdash{}{0pt}%
\pgfpathmoveto{\pgfqpoint{1.626842in}{1.344587in}}%
\pgfpathcurveto{\pgfqpoint{1.637892in}{1.344587in}}{\pgfqpoint{1.648491in}{1.348977in}}{\pgfqpoint{1.656304in}{1.356791in}}%
\pgfpathcurveto{\pgfqpoint{1.664118in}{1.364605in}}{\pgfqpoint{1.668508in}{1.375204in}}{\pgfqpoint{1.668508in}{1.386254in}}%
\pgfpathcurveto{\pgfqpoint{1.668508in}{1.397304in}}{\pgfqpoint{1.664118in}{1.407903in}}{\pgfqpoint{1.656304in}{1.415717in}}%
\pgfpathcurveto{\pgfqpoint{1.648491in}{1.423530in}}{\pgfqpoint{1.637892in}{1.427920in}}{\pgfqpoint{1.626842in}{1.427920in}}%
\pgfpathcurveto{\pgfqpoint{1.615791in}{1.427920in}}{\pgfqpoint{1.605192in}{1.423530in}}{\pgfqpoint{1.597379in}{1.415717in}}%
\pgfpathcurveto{\pgfqpoint{1.589565in}{1.407903in}}{\pgfqpoint{1.585175in}{1.397304in}}{\pgfqpoint{1.585175in}{1.386254in}}%
\pgfpathcurveto{\pgfqpoint{1.585175in}{1.375204in}}{\pgfqpoint{1.589565in}{1.364605in}}{\pgfqpoint{1.597379in}{1.356791in}}%
\pgfpathcurveto{\pgfqpoint{1.605192in}{1.348977in}}{\pgfqpoint{1.615791in}{1.344587in}}{\pgfqpoint{1.626842in}{1.344587in}}%
\pgfpathclose%
\pgfusepath{stroke,fill}%
\end{pgfscope}%
\begin{pgfscope}%
\pgfpathrectangle{\pgfqpoint{0.772069in}{0.515123in}}{\pgfqpoint{1.937500in}{1.347500in}}%
\pgfusepath{clip}%
\pgfsetbuttcap%
\pgfsetroundjoin%
\definecolor{currentfill}{rgb}{0.117647,0.564706,1.000000}%
\pgfsetfillcolor{currentfill}%
\pgfsetlinewidth{1.003750pt}%
\definecolor{currentstroke}{rgb}{0.117647,0.564706,1.000000}%
\pgfsetstrokecolor{currentstroke}%
\pgfsetdash{}{0pt}%
\pgfpathmoveto{\pgfqpoint{1.742030in}{1.473387in}}%
\pgfpathcurveto{\pgfqpoint{1.753080in}{1.473387in}}{\pgfqpoint{1.763679in}{1.477777in}}{\pgfqpoint{1.771493in}{1.485591in}}%
\pgfpathcurveto{\pgfqpoint{1.779307in}{1.493405in}}{\pgfqpoint{1.783697in}{1.504004in}}{\pgfqpoint{1.783697in}{1.515054in}}%
\pgfpathcurveto{\pgfqpoint{1.783697in}{1.526104in}}{\pgfqpoint{1.779307in}{1.536703in}}{\pgfqpoint{1.771493in}{1.544517in}}%
\pgfpathcurveto{\pgfqpoint{1.763679in}{1.552330in}}{\pgfqpoint{1.753080in}{1.556720in}}{\pgfqpoint{1.742030in}{1.556720in}}%
\pgfpathcurveto{\pgfqpoint{1.730980in}{1.556720in}}{\pgfqpoint{1.720381in}{1.552330in}}{\pgfqpoint{1.712567in}{1.544517in}}%
\pgfpathcurveto{\pgfqpoint{1.704754in}{1.536703in}}{\pgfqpoint{1.700364in}{1.526104in}}{\pgfqpoint{1.700364in}{1.515054in}}%
\pgfpathcurveto{\pgfqpoint{1.700364in}{1.504004in}}{\pgfqpoint{1.704754in}{1.493405in}}{\pgfqpoint{1.712567in}{1.485591in}}%
\pgfpathcurveto{\pgfqpoint{1.720381in}{1.477777in}}{\pgfqpoint{1.730980in}{1.473387in}}{\pgfqpoint{1.742030in}{1.473387in}}%
\pgfpathclose%
\pgfusepath{stroke,fill}%
\end{pgfscope}%
\begin{pgfscope}%
\pgfpathrectangle{\pgfqpoint{0.772069in}{0.515123in}}{\pgfqpoint{1.937500in}{1.347500in}}%
\pgfusepath{clip}%
\pgfsetbuttcap%
\pgfsetroundjoin%
\definecolor{currentfill}{rgb}{0.117647,0.564706,1.000000}%
\pgfsetfillcolor{currentfill}%
\pgfsetlinewidth{1.003750pt}%
\definecolor{currentstroke}{rgb}{0.117647,0.564706,1.000000}%
\pgfsetstrokecolor{currentstroke}%
\pgfsetdash{}{0pt}%
\pgfpathmoveto{\pgfqpoint{1.857219in}{1.599637in}}%
\pgfpathcurveto{\pgfqpoint{1.868269in}{1.599637in}}{\pgfqpoint{1.878868in}{1.604027in}}{\pgfqpoint{1.886682in}{1.611841in}}%
\pgfpathcurveto{\pgfqpoint{1.894495in}{1.619654in}}{\pgfqpoint{1.898885in}{1.630253in}}{\pgfqpoint{1.898885in}{1.641303in}}%
\pgfpathcurveto{\pgfqpoint{1.898885in}{1.652353in}}{\pgfqpoint{1.894495in}{1.662953in}}{\pgfqpoint{1.886682in}{1.670766in}}%
\pgfpathcurveto{\pgfqpoint{1.878868in}{1.678580in}}{\pgfqpoint{1.868269in}{1.682970in}}{\pgfqpoint{1.857219in}{1.682970in}}%
\pgfpathcurveto{\pgfqpoint{1.846169in}{1.682970in}}{\pgfqpoint{1.835570in}{1.678580in}}{\pgfqpoint{1.827756in}{1.670766in}}%
\pgfpathcurveto{\pgfqpoint{1.819942in}{1.662953in}}{\pgfqpoint{1.815552in}{1.652353in}}{\pgfqpoint{1.815552in}{1.641303in}}%
\pgfpathcurveto{\pgfqpoint{1.815552in}{1.630253in}}{\pgfqpoint{1.819942in}{1.619654in}}{\pgfqpoint{1.827756in}{1.611841in}}%
\pgfpathcurveto{\pgfqpoint{1.835570in}{1.604027in}}{\pgfqpoint{1.846169in}{1.599637in}}{\pgfqpoint{1.857219in}{1.599637in}}%
\pgfpathclose%
\pgfusepath{stroke,fill}%
\end{pgfscope}%
\begin{pgfscope}%
\pgfpathrectangle{\pgfqpoint{0.772069in}{0.515123in}}{\pgfqpoint{1.937500in}{1.347500in}}%
\pgfusepath{clip}%
\pgfsetbuttcap%
\pgfsetroundjoin%
\definecolor{currentfill}{rgb}{0.117647,0.564706,1.000000}%
\pgfsetfillcolor{currentfill}%
\pgfsetlinewidth{1.003750pt}%
\definecolor{currentstroke}{rgb}{0.117647,0.564706,1.000000}%
\pgfsetstrokecolor{currentstroke}%
\pgfsetdash{}{0pt}%
\pgfpathmoveto{\pgfqpoint{1.972407in}{1.729712in}}%
\pgfpathcurveto{\pgfqpoint{1.983458in}{1.729712in}}{\pgfqpoint{1.994057in}{1.734102in}}{\pgfqpoint{2.001870in}{1.741916in}}%
\pgfpathcurveto{\pgfqpoint{2.009684in}{1.749729in}}{\pgfqpoint{2.014074in}{1.760329in}}{\pgfqpoint{2.014074in}{1.771379in}}%
\pgfpathcurveto{\pgfqpoint{2.014074in}{1.782429in}}{\pgfqpoint{2.009684in}{1.793028in}}{\pgfqpoint{2.001870in}{1.800841in}}%
\pgfpathcurveto{\pgfqpoint{1.994057in}{1.808655in}}{\pgfqpoint{1.983458in}{1.813045in}}{\pgfqpoint{1.972407in}{1.813045in}}%
\pgfpathcurveto{\pgfqpoint{1.961357in}{1.813045in}}{\pgfqpoint{1.950758in}{1.808655in}}{\pgfqpoint{1.942945in}{1.800841in}}%
\pgfpathcurveto{\pgfqpoint{1.935131in}{1.793028in}}{\pgfqpoint{1.930741in}{1.782429in}}{\pgfqpoint{1.930741in}{1.771379in}}%
\pgfpathcurveto{\pgfqpoint{1.930741in}{1.760329in}}{\pgfqpoint{1.935131in}{1.749729in}}{\pgfqpoint{1.942945in}{1.741916in}}%
\pgfpathcurveto{\pgfqpoint{1.950758in}{1.734102in}}{\pgfqpoint{1.961357in}{1.729712in}}{\pgfqpoint{1.972407in}{1.729712in}}%
\pgfpathclose%
\pgfusepath{stroke,fill}%
\end{pgfscope}%
\begin{pgfscope}%
\pgfpathrectangle{\pgfqpoint{0.772069in}{0.515123in}}{\pgfqpoint{1.937500in}{1.347500in}}%
\pgfusepath{clip}%
\pgfsetbuttcap%
\pgfsetroundjoin%
\definecolor{currentfill}{rgb}{0.529412,0.807843,0.921569}%
\pgfsetfillcolor{currentfill}%
\pgfsetlinewidth{1.003750pt}%
\definecolor{currentstroke}{rgb}{0.529412,0.807843,0.921569}%
\pgfsetstrokecolor{currentstroke}%
\pgfsetdash{}{0pt}%
\pgfpathmoveto{\pgfqpoint{0.889634in}{0.570767in}}%
\pgfpathcurveto{\pgfqpoint{0.900685in}{0.570767in}}{\pgfqpoint{0.911284in}{0.575157in}}{\pgfqpoint{0.919097in}{0.582970in}}%
\pgfpathcurveto{\pgfqpoint{0.926911in}{0.590784in}}{\pgfqpoint{0.931301in}{0.601383in}}{\pgfqpoint{0.931301in}{0.612433in}}%
\pgfpathcurveto{\pgfqpoint{0.931301in}{0.623483in}}{\pgfqpoint{0.926911in}{0.634082in}}{\pgfqpoint{0.919097in}{0.641896in}}%
\pgfpathcurveto{\pgfqpoint{0.911284in}{0.649710in}}{\pgfqpoint{0.900685in}{0.654100in}}{\pgfqpoint{0.889634in}{0.654100in}}%
\pgfpathcurveto{\pgfqpoint{0.878584in}{0.654100in}}{\pgfqpoint{0.867985in}{0.649710in}}{\pgfqpoint{0.860172in}{0.641896in}}%
\pgfpathcurveto{\pgfqpoint{0.852358in}{0.634082in}}{\pgfqpoint{0.847968in}{0.623483in}}{\pgfqpoint{0.847968in}{0.612433in}}%
\pgfpathcurveto{\pgfqpoint{0.847968in}{0.601383in}}{\pgfqpoint{0.852358in}{0.590784in}}{\pgfqpoint{0.860172in}{0.582970in}}%
\pgfpathcurveto{\pgfqpoint{0.867985in}{0.575157in}}{\pgfqpoint{0.878584in}{0.570767in}}{\pgfqpoint{0.889634in}{0.570767in}}%
\pgfpathclose%
\pgfusepath{stroke,fill}%
\end{pgfscope}%
\begin{pgfscope}%
\pgfpathrectangle{\pgfqpoint{0.772069in}{0.515123in}}{\pgfqpoint{1.937500in}{1.347500in}}%
\pgfusepath{clip}%
\pgfsetbuttcap%
\pgfsetroundjoin%
\definecolor{currentfill}{rgb}{0.529412,0.807843,0.921569}%
\pgfsetfillcolor{currentfill}%
\pgfsetlinewidth{1.003750pt}%
\definecolor{currentstroke}{rgb}{0.529412,0.807843,0.921569}%
\pgfsetstrokecolor{currentstroke}%
\pgfsetdash{}{0pt}%
\pgfpathmoveto{\pgfqpoint{0.958748in}{0.694211in}}%
\pgfpathcurveto{\pgfqpoint{0.969798in}{0.694211in}}{\pgfqpoint{0.980397in}{0.698601in}}{\pgfqpoint{0.988210in}{0.706414in}}%
\pgfpathcurveto{\pgfqpoint{0.996024in}{0.714228in}}{\pgfqpoint{1.000414in}{0.724827in}}{\pgfqpoint{1.000414in}{0.735877in}}%
\pgfpathcurveto{\pgfqpoint{1.000414in}{0.746927in}}{\pgfqpoint{0.996024in}{0.757526in}}{\pgfqpoint{0.988210in}{0.765340in}}%
\pgfpathcurveto{\pgfqpoint{0.980397in}{0.773154in}}{\pgfqpoint{0.969798in}{0.777544in}}{\pgfqpoint{0.958748in}{0.777544in}}%
\pgfpathcurveto{\pgfqpoint{0.947697in}{0.777544in}}{\pgfqpoint{0.937098in}{0.773154in}}{\pgfqpoint{0.929285in}{0.765340in}}%
\pgfpathcurveto{\pgfqpoint{0.921471in}{0.757526in}}{\pgfqpoint{0.917081in}{0.746927in}}{\pgfqpoint{0.917081in}{0.735877in}}%
\pgfpathcurveto{\pgfqpoint{0.917081in}{0.724827in}}{\pgfqpoint{0.921471in}{0.714228in}}{\pgfqpoint{0.929285in}{0.706414in}}%
\pgfpathcurveto{\pgfqpoint{0.937098in}{0.698601in}}{\pgfqpoint{0.947697in}{0.694211in}}{\pgfqpoint{0.958748in}{0.694211in}}%
\pgfpathclose%
\pgfusepath{stroke,fill}%
\end{pgfscope}%
\begin{pgfscope}%
\pgfpathrectangle{\pgfqpoint{0.772069in}{0.515123in}}{\pgfqpoint{1.937500in}{1.347500in}}%
\pgfusepath{clip}%
\pgfsetbuttcap%
\pgfsetroundjoin%
\definecolor{currentfill}{rgb}{0.529412,0.807843,0.921569}%
\pgfsetfillcolor{currentfill}%
\pgfsetlinewidth{1.003750pt}%
\definecolor{currentstroke}{rgb}{0.529412,0.807843,0.921569}%
\pgfsetstrokecolor{currentstroke}%
\pgfsetdash{}{0pt}%
\pgfpathmoveto{\pgfqpoint{1.004823in}{0.826836in}}%
\pgfpathcurveto{\pgfqpoint{1.015873in}{0.826836in}}{\pgfqpoint{1.026472in}{0.831227in}}{\pgfqpoint{1.034286in}{0.839040in}}%
\pgfpathcurveto{\pgfqpoint{1.042099in}{0.846854in}}{\pgfqpoint{1.046490in}{0.857453in}}{\pgfqpoint{1.046490in}{0.868503in}}%
\pgfpathcurveto{\pgfqpoint{1.046490in}{0.879553in}}{\pgfqpoint{1.042099in}{0.890152in}}{\pgfqpoint{1.034286in}{0.897966in}}%
\pgfpathcurveto{\pgfqpoint{1.026472in}{0.905779in}}{\pgfqpoint{1.015873in}{0.910170in}}{\pgfqpoint{1.004823in}{0.910170in}}%
\pgfpathcurveto{\pgfqpoint{0.993773in}{0.910170in}}{\pgfqpoint{0.983174in}{0.905779in}}{\pgfqpoint{0.975360in}{0.897966in}}%
\pgfpathcurveto{\pgfqpoint{0.967547in}{0.890152in}}{\pgfqpoint{0.963156in}{0.879553in}}{\pgfqpoint{0.963156in}{0.868503in}}%
\pgfpathcurveto{\pgfqpoint{0.963156in}{0.857453in}}{\pgfqpoint{0.967547in}{0.846854in}}{\pgfqpoint{0.975360in}{0.839040in}}%
\pgfpathcurveto{\pgfqpoint{0.983174in}{0.831227in}}{\pgfqpoint{0.993773in}{0.826836in}}{\pgfqpoint{1.004823in}{0.826836in}}%
\pgfpathclose%
\pgfusepath{stroke,fill}%
\end{pgfscope}%
\begin{pgfscope}%
\pgfpathrectangle{\pgfqpoint{0.772069in}{0.515123in}}{\pgfqpoint{1.937500in}{1.347500in}}%
\pgfusepath{clip}%
\pgfsetbuttcap%
\pgfsetroundjoin%
\definecolor{currentfill}{rgb}{0.529412,0.807843,0.921569}%
\pgfsetfillcolor{currentfill}%
\pgfsetlinewidth{1.003750pt}%
\definecolor{currentstroke}{rgb}{0.529412,0.807843,0.921569}%
\pgfsetstrokecolor{currentstroke}%
\pgfsetdash{}{0pt}%
\pgfpathmoveto{\pgfqpoint{1.073936in}{0.956912in}}%
\pgfpathcurveto{\pgfqpoint{1.084986in}{0.956912in}}{\pgfqpoint{1.095585in}{0.961302in}}{\pgfqpoint{1.103399in}{0.969116in}}%
\pgfpathcurveto{\pgfqpoint{1.111213in}{0.976929in}}{\pgfqpoint{1.115603in}{0.987528in}}{\pgfqpoint{1.115603in}{0.998578in}}%
\pgfpathcurveto{\pgfqpoint{1.115603in}{1.009628in}}{\pgfqpoint{1.111213in}{1.020227in}}{\pgfqpoint{1.103399in}{1.028041in}}%
\pgfpathcurveto{\pgfqpoint{1.095585in}{1.035855in}}{\pgfqpoint{1.084986in}{1.040245in}}{\pgfqpoint{1.073936in}{1.040245in}}%
\pgfpathcurveto{\pgfqpoint{1.062886in}{1.040245in}}{\pgfqpoint{1.052287in}{1.035855in}}{\pgfqpoint{1.044473in}{1.028041in}}%
\pgfpathcurveto{\pgfqpoint{1.036660in}{1.020227in}}{\pgfqpoint{1.032270in}{1.009628in}}{\pgfqpoint{1.032270in}{0.998578in}}%
\pgfpathcurveto{\pgfqpoint{1.032270in}{0.987528in}}{\pgfqpoint{1.036660in}{0.976929in}}{\pgfqpoint{1.044473in}{0.969116in}}%
\pgfpathcurveto{\pgfqpoint{1.052287in}{0.961302in}}{\pgfqpoint{1.062886in}{0.956912in}}{\pgfqpoint{1.073936in}{0.956912in}}%
\pgfpathclose%
\pgfusepath{stroke,fill}%
\end{pgfscope}%
\begin{pgfscope}%
\pgfpathrectangle{\pgfqpoint{0.772069in}{0.515123in}}{\pgfqpoint{1.937500in}{1.347500in}}%
\pgfusepath{clip}%
\pgfsetbuttcap%
\pgfsetroundjoin%
\definecolor{currentfill}{rgb}{0.529412,0.807843,0.921569}%
\pgfsetfillcolor{currentfill}%
\pgfsetlinewidth{1.003750pt}%
\definecolor{currentstroke}{rgb}{0.529412,0.807843,0.921569}%
\pgfsetstrokecolor{currentstroke}%
\pgfsetdash{}{0pt}%
\pgfpathmoveto{\pgfqpoint{1.143049in}{1.086987in}}%
\pgfpathcurveto{\pgfqpoint{1.154100in}{1.086987in}}{\pgfqpoint{1.164699in}{1.091377in}}{\pgfqpoint{1.172512in}{1.099191in}}%
\pgfpathcurveto{\pgfqpoint{1.180326in}{1.107004in}}{\pgfqpoint{1.184716in}{1.117604in}}{\pgfqpoint{1.184716in}{1.128654in}}%
\pgfpathcurveto{\pgfqpoint{1.184716in}{1.139704in}}{\pgfqpoint{1.180326in}{1.150303in}}{\pgfqpoint{1.172512in}{1.158116in}}%
\pgfpathcurveto{\pgfqpoint{1.164699in}{1.165930in}}{\pgfqpoint{1.154100in}{1.170320in}}{\pgfqpoint{1.143049in}{1.170320in}}%
\pgfpathcurveto{\pgfqpoint{1.131999in}{1.170320in}}{\pgfqpoint{1.121400in}{1.165930in}}{\pgfqpoint{1.113587in}{1.158116in}}%
\pgfpathcurveto{\pgfqpoint{1.105773in}{1.150303in}}{\pgfqpoint{1.101383in}{1.139704in}}{\pgfqpoint{1.101383in}{1.128654in}}%
\pgfpathcurveto{\pgfqpoint{1.101383in}{1.117604in}}{\pgfqpoint{1.105773in}{1.107004in}}{\pgfqpoint{1.113587in}{1.099191in}}%
\pgfpathcurveto{\pgfqpoint{1.121400in}{1.091377in}}{\pgfqpoint{1.131999in}{1.086987in}}{\pgfqpoint{1.143049in}{1.086987in}}%
\pgfpathclose%
\pgfusepath{stroke,fill}%
\end{pgfscope}%
\begin{pgfscope}%
\pgfpathrectangle{\pgfqpoint{0.772069in}{0.515123in}}{\pgfqpoint{1.937500in}{1.347500in}}%
\pgfusepath{clip}%
\pgfsetbuttcap%
\pgfsetroundjoin%
\definecolor{currentfill}{rgb}{0.529412,0.807843,0.921569}%
\pgfsetfillcolor{currentfill}%
\pgfsetlinewidth{1.003750pt}%
\definecolor{currentstroke}{rgb}{0.529412,0.807843,0.921569}%
\pgfsetstrokecolor{currentstroke}%
\pgfsetdash{}{0pt}%
\pgfpathmoveto{\pgfqpoint{1.189125in}{1.210686in}}%
\pgfpathcurveto{\pgfqpoint{1.200175in}{1.210686in}}{\pgfqpoint{1.210774in}{1.215076in}}{\pgfqpoint{1.218588in}{1.222890in}}%
\pgfpathcurveto{\pgfqpoint{1.226401in}{1.230704in}}{\pgfqpoint{1.230792in}{1.241303in}}{\pgfqpoint{1.230792in}{1.252353in}}%
\pgfpathcurveto{\pgfqpoint{1.230792in}{1.263403in}}{\pgfqpoint{1.226401in}{1.274002in}}{\pgfqpoint{1.218588in}{1.281815in}}%
\pgfpathcurveto{\pgfqpoint{1.210774in}{1.289629in}}{\pgfqpoint{1.200175in}{1.294019in}}{\pgfqpoint{1.189125in}{1.294019in}}%
\pgfpathcurveto{\pgfqpoint{1.178075in}{1.294019in}}{\pgfqpoint{1.167476in}{1.289629in}}{\pgfqpoint{1.159662in}{1.281815in}}%
\pgfpathcurveto{\pgfqpoint{1.151848in}{1.274002in}}{\pgfqpoint{1.147458in}{1.263403in}}{\pgfqpoint{1.147458in}{1.252353in}}%
\pgfpathcurveto{\pgfqpoint{1.147458in}{1.241303in}}{\pgfqpoint{1.151848in}{1.230704in}}{\pgfqpoint{1.159662in}{1.222890in}}%
\pgfpathcurveto{\pgfqpoint{1.167476in}{1.215076in}}{\pgfqpoint{1.178075in}{1.210686in}}{\pgfqpoint{1.189125in}{1.210686in}}%
\pgfpathclose%
\pgfusepath{stroke,fill}%
\end{pgfscope}%
\begin{pgfscope}%
\pgfpathrectangle{\pgfqpoint{0.772069in}{0.515123in}}{\pgfqpoint{1.937500in}{1.347500in}}%
\pgfusepath{clip}%
\pgfsetbuttcap%
\pgfsetroundjoin%
\definecolor{currentfill}{rgb}{0.529412,0.807843,0.921569}%
\pgfsetfillcolor{currentfill}%
\pgfsetlinewidth{1.003750pt}%
\definecolor{currentstroke}{rgb}{0.529412,0.807843,0.921569}%
\pgfsetstrokecolor{currentstroke}%
\pgfsetdash{}{0pt}%
\pgfpathmoveto{\pgfqpoint{1.258238in}{1.344587in}}%
\pgfpathcurveto{\pgfqpoint{1.269288in}{1.344587in}}{\pgfqpoint{1.279887in}{1.348977in}}{\pgfqpoint{1.287701in}{1.356791in}}%
\pgfpathcurveto{\pgfqpoint{1.295514in}{1.364605in}}{\pgfqpoint{1.299905in}{1.375204in}}{\pgfqpoint{1.299905in}{1.386254in}}%
\pgfpathcurveto{\pgfqpoint{1.299905in}{1.397304in}}{\pgfqpoint{1.295514in}{1.407903in}}{\pgfqpoint{1.287701in}{1.415717in}}%
\pgfpathcurveto{\pgfqpoint{1.279887in}{1.423530in}}{\pgfqpoint{1.269288in}{1.427920in}}{\pgfqpoint{1.258238in}{1.427920in}}%
\pgfpathcurveto{\pgfqpoint{1.247188in}{1.427920in}}{\pgfqpoint{1.236589in}{1.423530in}}{\pgfqpoint{1.228775in}{1.415717in}}%
\pgfpathcurveto{\pgfqpoint{1.220962in}{1.407903in}}{\pgfqpoint{1.216571in}{1.397304in}}{\pgfqpoint{1.216571in}{1.386254in}}%
\pgfpathcurveto{\pgfqpoint{1.216571in}{1.375204in}}{\pgfqpoint{1.220962in}{1.364605in}}{\pgfqpoint{1.228775in}{1.356791in}}%
\pgfpathcurveto{\pgfqpoint{1.236589in}{1.348977in}}{\pgfqpoint{1.247188in}{1.344587in}}{\pgfqpoint{1.258238in}{1.344587in}}%
\pgfpathclose%
\pgfusepath{stroke,fill}%
\end{pgfscope}%
\begin{pgfscope}%
\pgfpathrectangle{\pgfqpoint{0.772069in}{0.515123in}}{\pgfqpoint{1.937500in}{1.347500in}}%
\pgfusepath{clip}%
\pgfsetbuttcap%
\pgfsetroundjoin%
\definecolor{currentfill}{rgb}{0.529412,0.807843,0.921569}%
\pgfsetfillcolor{currentfill}%
\pgfsetlinewidth{1.003750pt}%
\definecolor{currentstroke}{rgb}{0.529412,0.807843,0.921569}%
\pgfsetstrokecolor{currentstroke}%
\pgfsetdash{}{0pt}%
\pgfpathmoveto{\pgfqpoint{1.327351in}{1.473387in}}%
\pgfpathcurveto{\pgfqpoint{1.338401in}{1.473387in}}{\pgfqpoint{1.349000in}{1.477777in}}{\pgfqpoint{1.356814in}{1.485591in}}%
\pgfpathcurveto{\pgfqpoint{1.364628in}{1.493405in}}{\pgfqpoint{1.369018in}{1.504004in}}{\pgfqpoint{1.369018in}{1.515054in}}%
\pgfpathcurveto{\pgfqpoint{1.369018in}{1.526104in}}{\pgfqpoint{1.364628in}{1.536703in}}{\pgfqpoint{1.356814in}{1.544517in}}%
\pgfpathcurveto{\pgfqpoint{1.349000in}{1.552330in}}{\pgfqpoint{1.338401in}{1.556720in}}{\pgfqpoint{1.327351in}{1.556720in}}%
\pgfpathcurveto{\pgfqpoint{1.316301in}{1.556720in}}{\pgfqpoint{1.305702in}{1.552330in}}{\pgfqpoint{1.297888in}{1.544517in}}%
\pgfpathcurveto{\pgfqpoint{1.290075in}{1.536703in}}{\pgfqpoint{1.285685in}{1.526104in}}{\pgfqpoint{1.285685in}{1.515054in}}%
\pgfpathcurveto{\pgfqpoint{1.285685in}{1.504004in}}{\pgfqpoint{1.290075in}{1.493405in}}{\pgfqpoint{1.297888in}{1.485591in}}%
\pgfpathcurveto{\pgfqpoint{1.305702in}{1.477777in}}{\pgfqpoint{1.316301in}{1.473387in}}{\pgfqpoint{1.327351in}{1.473387in}}%
\pgfpathclose%
\pgfusepath{stroke,fill}%
\end{pgfscope}%
\begin{pgfscope}%
\pgfpathrectangle{\pgfqpoint{0.772069in}{0.515123in}}{\pgfqpoint{1.937500in}{1.347500in}}%
\pgfusepath{clip}%
\pgfsetbuttcap%
\pgfsetroundjoin%
\definecolor{currentfill}{rgb}{0.529412,0.807843,0.921569}%
\pgfsetfillcolor{currentfill}%
\pgfsetlinewidth{1.003750pt}%
\definecolor{currentstroke}{rgb}{0.529412,0.807843,0.921569}%
\pgfsetstrokecolor{currentstroke}%
\pgfsetdash{}{0pt}%
\pgfpathmoveto{\pgfqpoint{1.396464in}{1.599637in}}%
\pgfpathcurveto{\pgfqpoint{1.407514in}{1.599637in}}{\pgfqpoint{1.418114in}{1.604027in}}{\pgfqpoint{1.425927in}{1.611841in}}%
\pgfpathcurveto{\pgfqpoint{1.433741in}{1.619654in}}{\pgfqpoint{1.438131in}{1.630253in}}{\pgfqpoint{1.438131in}{1.641303in}}%
\pgfpathcurveto{\pgfqpoint{1.438131in}{1.652353in}}{\pgfqpoint{1.433741in}{1.662953in}}{\pgfqpoint{1.425927in}{1.670766in}}%
\pgfpathcurveto{\pgfqpoint{1.418114in}{1.678580in}}{\pgfqpoint{1.407514in}{1.682970in}}{\pgfqpoint{1.396464in}{1.682970in}}%
\pgfpathcurveto{\pgfqpoint{1.385414in}{1.682970in}}{\pgfqpoint{1.374815in}{1.678580in}}{\pgfqpoint{1.367002in}{1.670766in}}%
\pgfpathcurveto{\pgfqpoint{1.359188in}{1.662953in}}{\pgfqpoint{1.354798in}{1.652353in}}{\pgfqpoint{1.354798in}{1.641303in}}%
\pgfpathcurveto{\pgfqpoint{1.354798in}{1.630253in}}{\pgfqpoint{1.359188in}{1.619654in}}{\pgfqpoint{1.367002in}{1.611841in}}%
\pgfpathcurveto{\pgfqpoint{1.374815in}{1.604027in}}{\pgfqpoint{1.385414in}{1.599637in}}{\pgfqpoint{1.396464in}{1.599637in}}%
\pgfpathclose%
\pgfusepath{stroke,fill}%
\end{pgfscope}%
\begin{pgfscope}%
\pgfpathrectangle{\pgfqpoint{0.772069in}{0.515123in}}{\pgfqpoint{1.937500in}{1.347500in}}%
\pgfusepath{clip}%
\pgfsetbuttcap%
\pgfsetroundjoin%
\definecolor{currentfill}{rgb}{0.529412,0.807843,0.921569}%
\pgfsetfillcolor{currentfill}%
\pgfsetlinewidth{1.003750pt}%
\definecolor{currentstroke}{rgb}{0.529412,0.807843,0.921569}%
\pgfsetstrokecolor{currentstroke}%
\pgfsetdash{}{0pt}%
\pgfpathmoveto{\pgfqpoint{1.442540in}{1.729712in}}%
\pgfpathcurveto{\pgfqpoint{1.453590in}{1.729712in}}{\pgfqpoint{1.464189in}{1.734102in}}{\pgfqpoint{1.472003in}{1.741916in}}%
\pgfpathcurveto{\pgfqpoint{1.479816in}{1.749729in}}{\pgfqpoint{1.484206in}{1.760329in}}{\pgfqpoint{1.484206in}{1.771379in}}%
\pgfpathcurveto{\pgfqpoint{1.484206in}{1.782429in}}{\pgfqpoint{1.479816in}{1.793028in}}{\pgfqpoint{1.472003in}{1.800841in}}%
\pgfpathcurveto{\pgfqpoint{1.464189in}{1.808655in}}{\pgfqpoint{1.453590in}{1.813045in}}{\pgfqpoint{1.442540in}{1.813045in}}%
\pgfpathcurveto{\pgfqpoint{1.431490in}{1.813045in}}{\pgfqpoint{1.420891in}{1.808655in}}{\pgfqpoint{1.413077in}{1.800841in}}%
\pgfpathcurveto{\pgfqpoint{1.405263in}{1.793028in}}{\pgfqpoint{1.400873in}{1.782429in}}{\pgfqpoint{1.400873in}{1.771379in}}%
\pgfpathcurveto{\pgfqpoint{1.400873in}{1.760329in}}{\pgfqpoint{1.405263in}{1.749729in}}{\pgfqpoint{1.413077in}{1.741916in}}%
\pgfpathcurveto{\pgfqpoint{1.420891in}{1.734102in}}{\pgfqpoint{1.431490in}{1.729712in}}{\pgfqpoint{1.442540in}{1.729712in}}%
\pgfpathclose%
\pgfusepath{stroke,fill}%
\end{pgfscope}%
\begin{pgfscope}%
\pgfsetrectcap%
\pgfsetmiterjoin%
\pgfsetlinewidth{0.803000pt}%
\definecolor{currentstroke}{rgb}{0.000000,0.000000,0.000000}%
\pgfsetstrokecolor{currentstroke}%
\pgfsetdash{}{0pt}%
\pgfpathmoveto{\pgfqpoint{0.772069in}{0.515123in}}%
\pgfpathlineto{\pgfqpoint{0.772069in}{1.862623in}}%
\pgfusepath{stroke}%
\end{pgfscope}%
\begin{pgfscope}%
\pgfsetrectcap%
\pgfsetmiterjoin%
\pgfsetlinewidth{0.803000pt}%
\definecolor{currentstroke}{rgb}{0.000000,0.000000,0.000000}%
\pgfsetstrokecolor{currentstroke}%
\pgfsetdash{}{0pt}%
\pgfpathmoveto{\pgfqpoint{2.709569in}{0.515123in}}%
\pgfpathlineto{\pgfqpoint{2.709569in}{1.862623in}}%
\pgfusepath{stroke}%
\end{pgfscope}%
\begin{pgfscope}%
\pgfsetrectcap%
\pgfsetmiterjoin%
\pgfsetlinewidth{0.803000pt}%
\definecolor{currentstroke}{rgb}{0.000000,0.000000,0.000000}%
\pgfsetstrokecolor{currentstroke}%
\pgfsetdash{}{0pt}%
\pgfpathmoveto{\pgfqpoint{0.772069in}{0.515123in}}%
\pgfpathlineto{\pgfqpoint{2.709569in}{0.515123in}}%
\pgfusepath{stroke}%
\end{pgfscope}%
\begin{pgfscope}%
\pgfsetrectcap%
\pgfsetmiterjoin%
\pgfsetlinewidth{0.803000pt}%
\definecolor{currentstroke}{rgb}{0.000000,0.000000,0.000000}%
\pgfsetstrokecolor{currentstroke}%
\pgfsetdash{}{0pt}%
\pgfpathmoveto{\pgfqpoint{0.772069in}{1.862623in}}%
\pgfpathlineto{\pgfqpoint{2.709569in}{1.862623in}}%
\pgfusepath{stroke}%
\end{pgfscope}%
\end{pgfpicture}%
\makeatother%
\endgroup%

    %% Creator: Matplotlib, PGF backend
%%
%% To include the figure in your LaTeX document, write
%%   \input{<filename>.pgf}
%%
%% Make sure the required packages are loaded in your preamble
%%   \usepackage{pgf}
%%
%% Figures using additional raster images can only be included by \input if
%% they are in the same directory as the main LaTeX file. For loading figures
%% from other directories you can use the `import` package
%%   \usepackage{import}
%% and then include the figures with
%%   \import{<path to file>}{<filename>.pgf}
%%
%% Matplotlib used the following preamble
%%
\begingroup%
\makeatletter%
\begin{pgfpicture}%
\pgfpathrectangle{\pgfpointorigin}{\pgfqpoint{2.809569in}{1.962623in}}%
\pgfusepath{use as bounding box, clip}%
\begin{pgfscope}%
\pgfsetbuttcap%
\pgfsetmiterjoin%
\definecolor{currentfill}{rgb}{1.000000,1.000000,1.000000}%
\pgfsetfillcolor{currentfill}%
\pgfsetlinewidth{0.000000pt}%
\definecolor{currentstroke}{rgb}{1.000000,1.000000,1.000000}%
\pgfsetstrokecolor{currentstroke}%
\pgfsetdash{}{0pt}%
\pgfpathmoveto{\pgfqpoint{0.000000in}{0.000000in}}%
\pgfpathlineto{\pgfqpoint{2.809569in}{0.000000in}}%
\pgfpathlineto{\pgfqpoint{2.809569in}{1.962623in}}%
\pgfpathlineto{\pgfqpoint{0.000000in}{1.962623in}}%
\pgfpathclose%
\pgfusepath{fill}%
\end{pgfscope}%
\begin{pgfscope}%
\pgfsetbuttcap%
\pgfsetmiterjoin%
\definecolor{currentfill}{rgb}{1.000000,1.000000,1.000000}%
\pgfsetfillcolor{currentfill}%
\pgfsetlinewidth{0.000000pt}%
\definecolor{currentstroke}{rgb}{0.000000,0.000000,0.000000}%
\pgfsetstrokecolor{currentstroke}%
\pgfsetstrokeopacity{0.000000}%
\pgfsetdash{}{0pt}%
\pgfpathmoveto{\pgfqpoint{0.772069in}{0.515123in}}%
\pgfpathlineto{\pgfqpoint{2.709569in}{0.515123in}}%
\pgfpathlineto{\pgfqpoint{2.709569in}{1.862623in}}%
\pgfpathlineto{\pgfqpoint{0.772069in}{1.862623in}}%
\pgfpathclose%
\pgfusepath{fill}%
\end{pgfscope}%
\begin{pgfscope}%
\pgfpathrectangle{\pgfqpoint{0.772069in}{0.515123in}}{\pgfqpoint{1.937500in}{1.347500in}}%
\pgfusepath{clip}%
\pgfsetrectcap%
\pgfsetroundjoin%
\pgfsetlinewidth{1.505625pt}%
\definecolor{currentstroke}{rgb}{0.529412,0.807843,0.921569}%
\pgfsetstrokecolor{currentstroke}%
\pgfsetdash{}{0pt}%
\pgfpathmoveto{\pgfqpoint{1.005229in}{0.711184in}}%
\pgfpathlineto{\pgfqpoint{1.181703in}{0.829095in}}%
\pgfpathlineto{\pgfqpoint{1.358177in}{0.947005in}}%
\pgfpathlineto{\pgfqpoint{1.534651in}{1.064915in}}%
\pgfpathlineto{\pgfqpoint{1.711125in}{1.182826in}}%
\pgfpathlineto{\pgfqpoint{1.887599in}{1.300736in}}%
\pgfpathlineto{\pgfqpoint{2.064073in}{1.418646in}}%
\pgfpathlineto{\pgfqpoint{2.240547in}{1.536557in}}%
\pgfpathlineto{\pgfqpoint{2.417021in}{1.654467in}}%
\pgfpathlineto{\pgfqpoint{2.593495in}{1.772377in}}%
\pgfusepath{stroke}%
\end{pgfscope}%
\begin{pgfscope}%
\pgfpathrectangle{\pgfqpoint{0.772069in}{0.515123in}}{\pgfqpoint{1.937500in}{1.347500in}}%
\pgfusepath{clip}%
\pgfsetrectcap%
\pgfsetroundjoin%
\pgfsetlinewidth{1.505625pt}%
\definecolor{currentstroke}{rgb}{1.000000,0.627451,0.478431}%
\pgfsetstrokecolor{currentstroke}%
\pgfsetdash{}{0pt}%
\pgfpathmoveto{\pgfqpoint{1.005229in}{0.606308in}}%
\pgfpathlineto{\pgfqpoint{1.181703in}{0.622068in}}%
\pgfpathlineto{\pgfqpoint{1.358177in}{0.637829in}}%
\pgfpathlineto{\pgfqpoint{1.534651in}{0.653590in}}%
\pgfpathlineto{\pgfqpoint{1.711125in}{0.669350in}}%
\pgfpathlineto{\pgfqpoint{1.887599in}{0.685111in}}%
\pgfpathlineto{\pgfqpoint{2.064073in}{0.700871in}}%
\pgfpathlineto{\pgfqpoint{2.240547in}{0.716632in}}%
\pgfpathlineto{\pgfqpoint{2.417021in}{0.732392in}}%
\pgfpathlineto{\pgfqpoint{2.593495in}{0.748153in}}%
\pgfusepath{stroke}%
\end{pgfscope}%
\begin{pgfscope}%
\pgfpathrectangle{\pgfqpoint{0.772069in}{0.515123in}}{\pgfqpoint{1.937500in}{1.347500in}}%
\pgfusepath{clip}%
\pgfsetrectcap%
\pgfsetroundjoin%
\pgfsetlinewidth{1.505625pt}%
\definecolor{currentstroke}{rgb}{1.000000,0.894118,0.709804}%
\pgfsetstrokecolor{currentstroke}%
\pgfsetdash{}{0pt}%
\pgfpathmoveto{\pgfqpoint{1.005229in}{0.606117in}}%
\pgfpathlineto{\pgfqpoint{1.181703in}{0.618497in}}%
\pgfpathlineto{\pgfqpoint{1.358177in}{0.630877in}}%
\pgfpathlineto{\pgfqpoint{1.534651in}{0.643258in}}%
\pgfpathlineto{\pgfqpoint{1.711125in}{0.655638in}}%
\pgfpathlineto{\pgfqpoint{1.887599in}{0.668018in}}%
\pgfpathlineto{\pgfqpoint{2.064073in}{0.680399in}}%
\pgfpathlineto{\pgfqpoint{2.240547in}{0.692779in}}%
\pgfpathlineto{\pgfqpoint{2.417021in}{0.705159in}}%
\pgfpathlineto{\pgfqpoint{2.593495in}{0.717540in}}%
\pgfusepath{stroke}%
\end{pgfscope}%
\begin{pgfscope}%
\pgfpathrectangle{\pgfqpoint{0.772069in}{0.515123in}}{\pgfqpoint{1.937500in}{1.347500in}}%
\pgfusepath{clip}%
\pgfsetrectcap%
\pgfsetroundjoin%
\pgfsetlinewidth{1.505625pt}%
\definecolor{currentstroke}{rgb}{0.564706,0.933333,0.564706}%
\pgfsetstrokecolor{currentstroke}%
\pgfsetdash{}{0pt}%
\pgfpathmoveto{\pgfqpoint{1.005229in}{0.680942in}}%
\pgfpathlineto{\pgfqpoint{1.181703in}{0.768667in}}%
\pgfpathlineto{\pgfqpoint{1.358177in}{0.856391in}}%
\pgfpathlineto{\pgfqpoint{1.534651in}{0.944115in}}%
\pgfpathlineto{\pgfqpoint{1.711125in}{1.031840in}}%
\pgfpathlineto{\pgfqpoint{1.887599in}{1.119564in}}%
\pgfpathlineto{\pgfqpoint{2.064073in}{1.207289in}}%
\pgfpathlineto{\pgfqpoint{2.240547in}{1.295013in}}%
\pgfpathlineto{\pgfqpoint{2.417021in}{1.382738in}}%
\pgfpathlineto{\pgfqpoint{2.593495in}{1.470462in}}%
\pgfusepath{stroke}%
\end{pgfscope}%
\begin{pgfscope}%
\pgfpathrectangle{\pgfqpoint{0.772069in}{0.515123in}}{\pgfqpoint{1.937500in}{1.347500in}}%
\pgfusepath{clip}%
\pgfsetrectcap%
\pgfsetroundjoin%
\pgfsetlinewidth{1.505625pt}%
\definecolor{currentstroke}{rgb}{0.529412,0.807843,0.921569}%
\pgfsetstrokecolor{currentstroke}%
\pgfsetdash{}{0pt}%
\pgfpathmoveto{\pgfqpoint{0.936174in}{0.711961in}}%
\pgfpathlineto{\pgfqpoint{1.051265in}{0.829699in}}%
\pgfpathlineto{\pgfqpoint{1.166357in}{0.947437in}}%
\pgfpathlineto{\pgfqpoint{1.281449in}{1.065174in}}%
\pgfpathlineto{\pgfqpoint{1.396541in}{1.182912in}}%
\pgfpathlineto{\pgfqpoint{1.511632in}{1.300650in}}%
\pgfpathlineto{\pgfqpoint{1.626724in}{1.418387in}}%
\pgfpathlineto{\pgfqpoint{1.741816in}{1.536125in}}%
\pgfpathlineto{\pgfqpoint{1.856908in}{1.653863in}}%
\pgfpathlineto{\pgfqpoint{1.971999in}{1.771601in}}%
\pgfusepath{stroke}%
\end{pgfscope}%
\begin{pgfscope}%
\pgfpathrectangle{\pgfqpoint{0.772069in}{0.515123in}}{\pgfqpoint{1.937500in}{1.347500in}}%
\pgfusepath{clip}%
\pgfsetrectcap%
\pgfsetroundjoin%
\pgfsetlinewidth{1.505625pt}%
\definecolor{currentstroke}{rgb}{0.529412,0.807843,0.921569}%
\pgfsetstrokecolor{currentstroke}%
\pgfsetdash{}{0pt}%
\pgfpathmoveto{\pgfqpoint{0.890137in}{0.714270in}}%
\pgfpathlineto{\pgfqpoint{0.951519in}{0.830526in}}%
\pgfpathlineto{\pgfqpoint{1.012901in}{0.946782in}}%
\pgfpathlineto{\pgfqpoint{1.074284in}{1.063038in}}%
\pgfpathlineto{\pgfqpoint{1.135666in}{1.179293in}}%
\pgfpathlineto{\pgfqpoint{1.197048in}{1.295549in}}%
\pgfpathlineto{\pgfqpoint{1.258430in}{1.411805in}}%
\pgfpathlineto{\pgfqpoint{1.319813in}{1.528061in}}%
\pgfpathlineto{\pgfqpoint{1.381195in}{1.644317in}}%
\pgfpathlineto{\pgfqpoint{1.442577in}{1.760573in}}%
\pgfusepath{stroke}%
\end{pgfscope}%
\begin{pgfscope}%
\pgfpathrectangle{\pgfqpoint{0.772069in}{0.515123in}}{\pgfqpoint{1.937500in}{1.347500in}}%
\pgfusepath{clip}%
\pgfsetrectcap%
\pgfsetroundjoin%
\pgfsetlinewidth{1.505625pt}%
\definecolor{currentstroke}{rgb}{1.000000,0.627451,0.478431}%
\pgfsetstrokecolor{currentstroke}%
\pgfsetdash{}{0pt}%
\pgfpathmoveto{\pgfqpoint{0.936174in}{0.606442in}}%
\pgfpathlineto{\pgfqpoint{1.051265in}{0.622172in}}%
\pgfpathlineto{\pgfqpoint{1.166357in}{0.637903in}}%
\pgfpathlineto{\pgfqpoint{1.281449in}{0.653634in}}%
\pgfpathlineto{\pgfqpoint{1.396541in}{0.669365in}}%
\pgfpathlineto{\pgfqpoint{1.511632in}{0.685096in}}%
\pgfpathlineto{\pgfqpoint{1.626724in}{0.700827in}}%
\pgfpathlineto{\pgfqpoint{1.741816in}{0.716558in}}%
\pgfpathlineto{\pgfqpoint{1.856908in}{0.732289in}}%
\pgfpathlineto{\pgfqpoint{1.971999in}{0.748019in}}%
\pgfusepath{stroke}%
\end{pgfscope}%
\begin{pgfscope}%
\pgfpathrectangle{\pgfqpoint{0.772069in}{0.515123in}}{\pgfqpoint{1.937500in}{1.347500in}}%
\pgfusepath{clip}%
\pgfsetrectcap%
\pgfsetroundjoin%
\pgfsetlinewidth{1.505625pt}%
\definecolor{currentstroke}{rgb}{1.000000,0.894118,0.709804}%
\pgfsetstrokecolor{currentstroke}%
\pgfsetdash{}{0pt}%
\pgfpathmoveto{\pgfqpoint{0.936174in}{0.606199in}}%
\pgfpathlineto{\pgfqpoint{1.051265in}{0.618561in}}%
\pgfpathlineto{\pgfqpoint{1.166357in}{0.630923in}}%
\pgfpathlineto{\pgfqpoint{1.281449in}{0.643285in}}%
\pgfpathlineto{\pgfqpoint{1.396541in}{0.655647in}}%
\pgfpathlineto{\pgfqpoint{1.511632in}{0.668009in}}%
\pgfpathlineto{\pgfqpoint{1.626724in}{0.680371in}}%
\pgfpathlineto{\pgfqpoint{1.741816in}{0.692733in}}%
\pgfpathlineto{\pgfqpoint{1.856908in}{0.705095in}}%
\pgfpathlineto{\pgfqpoint{1.971999in}{0.717457in}}%
\pgfusepath{stroke}%
\end{pgfscope}%
\begin{pgfscope}%
\pgfpathrectangle{\pgfqpoint{0.772069in}{0.515123in}}{\pgfqpoint{1.937500in}{1.347500in}}%
\pgfusepath{clip}%
\pgfsetrectcap%
\pgfsetroundjoin%
\pgfsetlinewidth{1.505625pt}%
\definecolor{currentstroke}{rgb}{0.564706,0.933333,0.564706}%
\pgfsetstrokecolor{currentstroke}%
\pgfsetdash{}{0pt}%
\pgfpathmoveto{\pgfqpoint{0.936174in}{0.681524in}}%
\pgfpathlineto{\pgfqpoint{1.051265in}{0.769119in}}%
\pgfpathlineto{\pgfqpoint{1.166357in}{0.856714in}}%
\pgfpathlineto{\pgfqpoint{1.281449in}{0.944309in}}%
\pgfpathlineto{\pgfqpoint{1.396541in}{1.031905in}}%
\pgfpathlineto{\pgfqpoint{1.511632in}{1.119500in}}%
\pgfpathlineto{\pgfqpoint{1.626724in}{1.207095in}}%
\pgfpathlineto{\pgfqpoint{1.741816in}{1.294690in}}%
\pgfpathlineto{\pgfqpoint{1.856908in}{1.382285in}}%
\pgfpathlineto{\pgfqpoint{1.971999in}{1.469880in}}%
\pgfusepath{stroke}%
\end{pgfscope}%
\begin{pgfscope}%
\pgfpathrectangle{\pgfqpoint{0.772069in}{0.515123in}}{\pgfqpoint{1.937500in}{1.347500in}}%
\pgfusepath{clip}%
\pgfsetrectcap%
\pgfsetroundjoin%
\pgfsetlinewidth{1.505625pt}%
\definecolor{currentstroke}{rgb}{1.000000,0.627451,0.478431}%
\pgfsetstrokecolor{currentstroke}%
\pgfsetdash{}{0pt}%
\pgfpathmoveto{\pgfqpoint{0.890137in}{0.606668in}}%
\pgfpathlineto{\pgfqpoint{0.951519in}{0.622219in}}%
\pgfpathlineto{\pgfqpoint{1.012901in}{0.637770in}}%
\pgfpathlineto{\pgfqpoint{1.074284in}{0.653321in}}%
\pgfpathlineto{\pgfqpoint{1.135666in}{0.668872in}}%
\pgfpathlineto{\pgfqpoint{1.197048in}{0.684423in}}%
\pgfpathlineto{\pgfqpoint{1.258430in}{0.699974in}}%
\pgfpathlineto{\pgfqpoint{1.319813in}{0.715525in}}%
\pgfpathlineto{\pgfqpoint{1.381195in}{0.731076in}}%
\pgfpathlineto{\pgfqpoint{1.442577in}{0.746627in}}%
\pgfusepath{stroke}%
\end{pgfscope}%
\begin{pgfscope}%
\pgfpathrectangle{\pgfqpoint{0.772069in}{0.515123in}}{\pgfqpoint{1.937500in}{1.347500in}}%
\pgfusepath{clip}%
\pgfsetrectcap%
\pgfsetroundjoin%
\pgfsetlinewidth{1.505625pt}%
\definecolor{currentstroke}{rgb}{1.000000,0.894118,0.709804}%
\pgfsetstrokecolor{currentstroke}%
\pgfsetdash{}{0pt}%
\pgfpathmoveto{\pgfqpoint{0.890137in}{0.606459in}}%
\pgfpathlineto{\pgfqpoint{0.951519in}{0.618662in}}%
\pgfpathlineto{\pgfqpoint{1.012901in}{0.630864in}}%
\pgfpathlineto{\pgfqpoint{1.074284in}{0.643067in}}%
\pgfpathlineto{\pgfqpoint{1.135666in}{0.655269in}}%
\pgfpathlineto{\pgfqpoint{1.197048in}{0.667472in}}%
\pgfpathlineto{\pgfqpoint{1.258430in}{0.679675in}}%
\pgfpathlineto{\pgfqpoint{1.319813in}{0.691877in}}%
\pgfpathlineto{\pgfqpoint{1.381195in}{0.704080in}}%
\pgfpathlineto{\pgfqpoint{1.442577in}{0.716282in}}%
\pgfusepath{stroke}%
\end{pgfscope}%
\begin{pgfscope}%
\pgfpathrectangle{\pgfqpoint{0.772069in}{0.515123in}}{\pgfqpoint{1.937500in}{1.347500in}}%
\pgfusepath{clip}%
\pgfsetrectcap%
\pgfsetroundjoin%
\pgfsetlinewidth{1.505625pt}%
\definecolor{currentstroke}{rgb}{0.564706,0.933333,0.564706}%
\pgfsetstrokecolor{currentstroke}%
\pgfsetdash{}{0pt}%
\pgfpathmoveto{\pgfqpoint{0.890137in}{0.683228in}}%
\pgfpathlineto{\pgfqpoint{0.951519in}{0.769724in}}%
\pgfpathlineto{\pgfqpoint{1.012901in}{0.856219in}}%
\pgfpathlineto{\pgfqpoint{1.074284in}{0.942715in}}%
\pgfpathlineto{\pgfqpoint{1.135666in}{1.029211in}}%
\pgfpathlineto{\pgfqpoint{1.197048in}{1.115706in}}%
\pgfpathlineto{\pgfqpoint{1.258430in}{1.202202in}}%
\pgfpathlineto{\pgfqpoint{1.319813in}{1.288698in}}%
\pgfpathlineto{\pgfqpoint{1.381195in}{1.375193in}}%
\pgfpathlineto{\pgfqpoint{1.442577in}{1.461689in}}%
\pgfusepath{stroke}%
\end{pgfscope}%
\begin{pgfscope}%
\pgfsetbuttcap%
\pgfsetroundjoin%
\definecolor{currentfill}{rgb}{0.000000,0.000000,0.000000}%
\pgfsetfillcolor{currentfill}%
\pgfsetlinewidth{0.803000pt}%
\definecolor{currentstroke}{rgb}{0.000000,0.000000,0.000000}%
\pgfsetstrokecolor{currentstroke}%
\pgfsetdash{}{0pt}%
\pgfsys@defobject{currentmarker}{\pgfqpoint{0.000000in}{-0.048611in}}{\pgfqpoint{0.000000in}{0.000000in}}{%
\pgfpathmoveto{\pgfqpoint{0.000000in}{0.000000in}}%
\pgfpathlineto{\pgfqpoint{0.000000in}{-0.048611in}}%
\pgfusepath{stroke,fill}%
}%
\begin{pgfscope}%
\pgfsys@transformshift{0.844100in}{0.515123in}%
\pgfsys@useobject{currentmarker}{}%
\end{pgfscope}%
\end{pgfscope}%
\begin{pgfscope}%
\definecolor{textcolor}{rgb}{0.000000,0.000000,0.000000}%
\pgfsetstrokecolor{textcolor}%
\pgfsetfillcolor{textcolor}%
\pgftext[x=0.844100in,y=0.417901in,,top]{\color{textcolor}\rmfamily\fontsize{10.000000}{12.000000}\selectfont \(\displaystyle 0.0\)}%
\end{pgfscope}%
\begin{pgfscope}%
\pgfsetbuttcap%
\pgfsetroundjoin%
\definecolor{currentfill}{rgb}{0.000000,0.000000,0.000000}%
\pgfsetfillcolor{currentfill}%
\pgfsetlinewidth{0.803000pt}%
\definecolor{currentstroke}{rgb}{0.000000,0.000000,0.000000}%
\pgfsetstrokecolor{currentstroke}%
\pgfsetdash{}{0pt}%
\pgfsys@defobject{currentmarker}{\pgfqpoint{0.000000in}{-0.048611in}}{\pgfqpoint{0.000000in}{0.000000in}}{%
\pgfpathmoveto{\pgfqpoint{0.000000in}{0.000000in}}%
\pgfpathlineto{\pgfqpoint{0.000000in}{-0.048611in}}%
\pgfusepath{stroke,fill}%
}%
\begin{pgfscope}%
\pgfsys@transformshift{1.419559in}{0.515123in}%
\pgfsys@useobject{currentmarker}{}%
\end{pgfscope}%
\end{pgfscope}%
\begin{pgfscope}%
\definecolor{textcolor}{rgb}{0.000000,0.000000,0.000000}%
\pgfsetstrokecolor{textcolor}%
\pgfsetfillcolor{textcolor}%
\pgftext[x=1.419559in,y=0.417901in,,top]{\color{textcolor}\rmfamily\fontsize{10.000000}{12.000000}\selectfont \(\displaystyle 2.5\)}%
\end{pgfscope}%
\begin{pgfscope}%
\pgfsetbuttcap%
\pgfsetroundjoin%
\definecolor{currentfill}{rgb}{0.000000,0.000000,0.000000}%
\pgfsetfillcolor{currentfill}%
\pgfsetlinewidth{0.803000pt}%
\definecolor{currentstroke}{rgb}{0.000000,0.000000,0.000000}%
\pgfsetstrokecolor{currentstroke}%
\pgfsetdash{}{0pt}%
\pgfsys@defobject{currentmarker}{\pgfqpoint{0.000000in}{-0.048611in}}{\pgfqpoint{0.000000in}{0.000000in}}{%
\pgfpathmoveto{\pgfqpoint{0.000000in}{0.000000in}}%
\pgfpathlineto{\pgfqpoint{0.000000in}{-0.048611in}}%
\pgfusepath{stroke,fill}%
}%
\begin{pgfscope}%
\pgfsys@transformshift{1.995018in}{0.515123in}%
\pgfsys@useobject{currentmarker}{}%
\end{pgfscope}%
\end{pgfscope}%
\begin{pgfscope}%
\definecolor{textcolor}{rgb}{0.000000,0.000000,0.000000}%
\pgfsetstrokecolor{textcolor}%
\pgfsetfillcolor{textcolor}%
\pgftext[x=1.995018in,y=0.417901in,,top]{\color{textcolor}\rmfamily\fontsize{10.000000}{12.000000}\selectfont \(\displaystyle 5.0\)}%
\end{pgfscope}%
\begin{pgfscope}%
\pgfsetbuttcap%
\pgfsetroundjoin%
\definecolor{currentfill}{rgb}{0.000000,0.000000,0.000000}%
\pgfsetfillcolor{currentfill}%
\pgfsetlinewidth{0.803000pt}%
\definecolor{currentstroke}{rgb}{0.000000,0.000000,0.000000}%
\pgfsetstrokecolor{currentstroke}%
\pgfsetdash{}{0pt}%
\pgfsys@defobject{currentmarker}{\pgfqpoint{0.000000in}{-0.048611in}}{\pgfqpoint{0.000000in}{0.000000in}}{%
\pgfpathmoveto{\pgfqpoint{0.000000in}{0.000000in}}%
\pgfpathlineto{\pgfqpoint{0.000000in}{-0.048611in}}%
\pgfusepath{stroke,fill}%
}%
\begin{pgfscope}%
\pgfsys@transformshift{2.570476in}{0.515123in}%
\pgfsys@useobject{currentmarker}{}%
\end{pgfscope}%
\end{pgfscope}%
\begin{pgfscope}%
\definecolor{textcolor}{rgb}{0.000000,0.000000,0.000000}%
\pgfsetstrokecolor{textcolor}%
\pgfsetfillcolor{textcolor}%
\pgftext[x=2.570476in,y=0.417901in,,top]{\color{textcolor}\rmfamily\fontsize{10.000000}{12.000000}\selectfont \(\displaystyle 7.5\)}%
\end{pgfscope}%
\begin{pgfscope}%
\definecolor{textcolor}{rgb}{0.000000,0.000000,0.000000}%
\pgfsetstrokecolor{textcolor}%
\pgfsetfillcolor{textcolor}%
\pgftext[x=1.740819in,y=0.238889in,,top]{\color{textcolor}\rmfamily\fontsize{10.000000}{12.000000}\selectfont I(\(\displaystyle \mu\)A)}%
\end{pgfscope}%
\begin{pgfscope}%
\pgfsetbuttcap%
\pgfsetroundjoin%
\definecolor{currentfill}{rgb}{0.000000,0.000000,0.000000}%
\pgfsetfillcolor{currentfill}%
\pgfsetlinewidth{0.803000pt}%
\definecolor{currentstroke}{rgb}{0.000000,0.000000,0.000000}%
\pgfsetstrokecolor{currentstroke}%
\pgfsetdash{}{0pt}%
\pgfsys@defobject{currentmarker}{\pgfqpoint{-0.048611in}{0.000000in}}{\pgfqpoint{0.000000in}{0.000000in}}{%
\pgfpathmoveto{\pgfqpoint{0.000000in}{0.000000in}}%
\pgfpathlineto{\pgfqpoint{-0.048611in}{0.000000in}}%
\pgfusepath{stroke,fill}%
}%
\begin{pgfscope}%
\pgfsys@transformshift{0.772069in}{0.593665in}%
\pgfsys@useobject{currentmarker}{}%
\end{pgfscope}%
\end{pgfscope}%
\begin{pgfscope}%
\definecolor{textcolor}{rgb}{0.000000,0.000000,0.000000}%
\pgfsetstrokecolor{textcolor}%
\pgfsetfillcolor{textcolor}%
\pgftext[x=0.605402in,y=0.545440in,left,base]{\color{textcolor}\rmfamily\fontsize{10.000000}{12.000000}\selectfont \(\displaystyle 0\)}%
\end{pgfscope}%
\begin{pgfscope}%
\pgfsetbuttcap%
\pgfsetroundjoin%
\definecolor{currentfill}{rgb}{0.000000,0.000000,0.000000}%
\pgfsetfillcolor{currentfill}%
\pgfsetlinewidth{0.803000pt}%
\definecolor{currentstroke}{rgb}{0.000000,0.000000,0.000000}%
\pgfsetstrokecolor{currentstroke}%
\pgfsetdash{}{0pt}%
\pgfsys@defobject{currentmarker}{\pgfqpoint{-0.048611in}{0.000000in}}{\pgfqpoint{0.000000in}{0.000000in}}{%
\pgfpathmoveto{\pgfqpoint{0.000000in}{0.000000in}}%
\pgfpathlineto{\pgfqpoint{-0.048611in}{0.000000in}}%
\pgfusepath{stroke,fill}%
}%
\begin{pgfscope}%
\pgfsys@transformshift{0.772069in}{1.175540in}%
\pgfsys@useobject{currentmarker}{}%
\end{pgfscope}%
\end{pgfscope}%
\begin{pgfscope}%
\definecolor{textcolor}{rgb}{0.000000,0.000000,0.000000}%
\pgfsetstrokecolor{textcolor}%
\pgfsetfillcolor{textcolor}%
\pgftext[x=0.605402in,y=1.127315in,left,base]{\color{textcolor}\rmfamily\fontsize{10.000000}{12.000000}\selectfont \(\displaystyle 5\)}%
\end{pgfscope}%
\begin{pgfscope}%
\pgfsetbuttcap%
\pgfsetroundjoin%
\definecolor{currentfill}{rgb}{0.000000,0.000000,0.000000}%
\pgfsetfillcolor{currentfill}%
\pgfsetlinewidth{0.803000pt}%
\definecolor{currentstroke}{rgb}{0.000000,0.000000,0.000000}%
\pgfsetstrokecolor{currentstroke}%
\pgfsetdash{}{0pt}%
\pgfsys@defobject{currentmarker}{\pgfqpoint{-0.048611in}{0.000000in}}{\pgfqpoint{0.000000in}{0.000000in}}{%
\pgfpathmoveto{\pgfqpoint{0.000000in}{0.000000in}}%
\pgfpathlineto{\pgfqpoint{-0.048611in}{0.000000in}}%
\pgfusepath{stroke,fill}%
}%
\begin{pgfscope}%
\pgfsys@transformshift{0.772069in}{1.757415in}%
\pgfsys@useobject{currentmarker}{}%
\end{pgfscope}%
\end{pgfscope}%
\begin{pgfscope}%
\definecolor{textcolor}{rgb}{0.000000,0.000000,0.000000}%
\pgfsetstrokecolor{textcolor}%
\pgfsetfillcolor{textcolor}%
\pgftext[x=0.535957in,y=1.709190in,left,base]{\color{textcolor}\rmfamily\fontsize{10.000000}{12.000000}\selectfont \(\displaystyle 10\)}%
\end{pgfscope}%
\begin{pgfscope}%
\definecolor{textcolor}{rgb}{0.000000,0.000000,0.000000}%
\pgfsetstrokecolor{textcolor}%
\pgfsetfillcolor{textcolor}%
\pgftext[x=0.258179in,y=1.188873in,,bottom]{\color{textcolor}\rmfamily\fontsize{10.000000}{12.000000}\selectfont V(V)}%
\end{pgfscope}%
\begin{pgfscope}%
\pgfpathrectangle{\pgfqpoint{0.772069in}{0.515123in}}{\pgfqpoint{1.937500in}{1.347500in}}%
\pgfusepath{clip}%
\pgfsetbuttcap%
\pgfsetroundjoin%
\definecolor{currentfill}{rgb}{0.121569,0.466667,0.705882}%
\pgfsetfillcolor{currentfill}%
\pgfsetlinewidth{1.003750pt}%
\definecolor{currentstroke}{rgb}{0.121569,0.466667,0.705882}%
\pgfsetstrokecolor{currentstroke}%
\pgfsetdash{}{0pt}%
\pgfpathmoveto{\pgfqpoint{1.005229in}{0.672097in}}%
\pgfpathcurveto{\pgfqpoint{1.016279in}{0.672097in}}{\pgfqpoint{1.026878in}{0.676488in}}{\pgfqpoint{1.034691in}{0.684301in}}%
\pgfpathcurveto{\pgfqpoint{1.042505in}{0.692115in}}{\pgfqpoint{1.046895in}{0.702714in}}{\pgfqpoint{1.046895in}{0.713764in}}%
\pgfpathcurveto{\pgfqpoint{1.046895in}{0.724814in}}{\pgfqpoint{1.042505in}{0.735413in}}{\pgfqpoint{1.034691in}{0.743227in}}%
\pgfpathcurveto{\pgfqpoint{1.026878in}{0.751041in}}{\pgfqpoint{1.016279in}{0.755431in}}{\pgfqpoint{1.005229in}{0.755431in}}%
\pgfpathcurveto{\pgfqpoint{0.994178in}{0.755431in}}{\pgfqpoint{0.983579in}{0.751041in}}{\pgfqpoint{0.975766in}{0.743227in}}%
\pgfpathcurveto{\pgfqpoint{0.967952in}{0.735413in}}{\pgfqpoint{0.963562in}{0.724814in}}{\pgfqpoint{0.963562in}{0.713764in}}%
\pgfpathcurveto{\pgfqpoint{0.963562in}{0.702714in}}{\pgfqpoint{0.967952in}{0.692115in}}{\pgfqpoint{0.975766in}{0.684301in}}%
\pgfpathcurveto{\pgfqpoint{0.983579in}{0.676488in}}{\pgfqpoint{0.994178in}{0.672097in}}{\pgfqpoint{1.005229in}{0.672097in}}%
\pgfpathclose%
\pgfusepath{stroke,fill}%
\end{pgfscope}%
\begin{pgfscope}%
\pgfpathrectangle{\pgfqpoint{0.772069in}{0.515123in}}{\pgfqpoint{1.937500in}{1.347500in}}%
\pgfusepath{clip}%
\pgfsetbuttcap%
\pgfsetroundjoin%
\definecolor{currentfill}{rgb}{0.121569,0.466667,0.705882}%
\pgfsetfillcolor{currentfill}%
\pgfsetlinewidth{1.003750pt}%
\definecolor{currentstroke}{rgb}{0.121569,0.466667,0.705882}%
\pgfsetstrokecolor{currentstroke}%
\pgfsetdash{}{0pt}%
\pgfpathmoveto{\pgfqpoint{1.189375in}{0.784748in}}%
\pgfpathcurveto{\pgfqpoint{1.200426in}{0.784748in}}{\pgfqpoint{1.211025in}{0.789139in}}{\pgfqpoint{1.218838in}{0.796952in}}%
\pgfpathcurveto{\pgfqpoint{1.226652in}{0.804766in}}{\pgfqpoint{1.231042in}{0.815365in}}{\pgfqpoint{1.231042in}{0.826415in}}%
\pgfpathcurveto{\pgfqpoint{1.231042in}{0.837465in}}{\pgfqpoint{1.226652in}{0.848064in}}{\pgfqpoint{1.218838in}{0.855878in}}%
\pgfpathcurveto{\pgfqpoint{1.211025in}{0.863692in}}{\pgfqpoint{1.200426in}{0.868082in}}{\pgfqpoint{1.189375in}{0.868082in}}%
\pgfpathcurveto{\pgfqpoint{1.178325in}{0.868082in}}{\pgfqpoint{1.167726in}{0.863692in}}{\pgfqpoint{1.159913in}{0.855878in}}%
\pgfpathcurveto{\pgfqpoint{1.152099in}{0.848064in}}{\pgfqpoint{1.147709in}{0.837465in}}{\pgfqpoint{1.147709in}{0.826415in}}%
\pgfpathcurveto{\pgfqpoint{1.147709in}{0.815365in}}{\pgfqpoint{1.152099in}{0.804766in}}{\pgfqpoint{1.159913in}{0.796952in}}%
\pgfpathcurveto{\pgfqpoint{1.167726in}{0.789139in}}{\pgfqpoint{1.178325in}{0.784748in}}{\pgfqpoint{1.189375in}{0.784748in}}%
\pgfpathclose%
\pgfusepath{stroke,fill}%
\end{pgfscope}%
\begin{pgfscope}%
\pgfpathrectangle{\pgfqpoint{0.772069in}{0.515123in}}{\pgfqpoint{1.937500in}{1.347500in}}%
\pgfusepath{clip}%
\pgfsetbuttcap%
\pgfsetroundjoin%
\definecolor{currentfill}{rgb}{0.121569,0.466667,0.705882}%
\pgfsetfillcolor{currentfill}%
\pgfsetlinewidth{1.003750pt}%
\definecolor{currentstroke}{rgb}{0.121569,0.466667,0.705882}%
\pgfsetstrokecolor{currentstroke}%
\pgfsetdash{}{0pt}%
\pgfpathmoveto{\pgfqpoint{1.350504in}{0.905779in}}%
\pgfpathcurveto{\pgfqpoint{1.361554in}{0.905779in}}{\pgfqpoint{1.372153in}{0.910169in}}{\pgfqpoint{1.379967in}{0.917982in}}%
\pgfpathcurveto{\pgfqpoint{1.387780in}{0.925796in}}{\pgfqpoint{1.392171in}{0.936395in}}{\pgfqpoint{1.392171in}{0.947445in}}%
\pgfpathcurveto{\pgfqpoint{1.392171in}{0.958495in}}{\pgfqpoint{1.387780in}{0.969094in}}{\pgfqpoint{1.379967in}{0.976908in}}%
\pgfpathcurveto{\pgfqpoint{1.372153in}{0.984722in}}{\pgfqpoint{1.361554in}{0.989112in}}{\pgfqpoint{1.350504in}{0.989112in}}%
\pgfpathcurveto{\pgfqpoint{1.339454in}{0.989112in}}{\pgfqpoint{1.328855in}{0.984722in}}{\pgfqpoint{1.321041in}{0.976908in}}%
\pgfpathcurveto{\pgfqpoint{1.313227in}{0.969094in}}{\pgfqpoint{1.308837in}{0.958495in}}{\pgfqpoint{1.308837in}{0.947445in}}%
\pgfpathcurveto{\pgfqpoint{1.308837in}{0.936395in}}{\pgfqpoint{1.313227in}{0.925796in}}{\pgfqpoint{1.321041in}{0.917982in}}%
\pgfpathcurveto{\pgfqpoint{1.328855in}{0.910169in}}{\pgfqpoint{1.339454in}{0.905779in}}{\pgfqpoint{1.350504in}{0.905779in}}%
\pgfpathclose%
\pgfusepath{stroke,fill}%
\end{pgfscope}%
\begin{pgfscope}%
\pgfpathrectangle{\pgfqpoint{0.772069in}{0.515123in}}{\pgfqpoint{1.937500in}{1.347500in}}%
\pgfusepath{clip}%
\pgfsetbuttcap%
\pgfsetroundjoin%
\definecolor{currentfill}{rgb}{0.121569,0.466667,0.705882}%
\pgfsetfillcolor{currentfill}%
\pgfsetlinewidth{1.003750pt}%
\definecolor{currentstroke}{rgb}{0.121569,0.466667,0.705882}%
\pgfsetstrokecolor{currentstroke}%
\pgfsetdash{}{0pt}%
\pgfpathmoveto{\pgfqpoint{1.534651in}{1.024481in}}%
\pgfpathcurveto{\pgfqpoint{1.545701in}{1.024481in}}{\pgfqpoint{1.556300in}{1.028871in}}{\pgfqpoint{1.564113in}{1.036685in}}%
\pgfpathcurveto{\pgfqpoint{1.571927in}{1.044499in}}{\pgfqpoint{1.576317in}{1.055098in}}{\pgfqpoint{1.576317in}{1.066148in}}%
\pgfpathcurveto{\pgfqpoint{1.576317in}{1.077198in}}{\pgfqpoint{1.571927in}{1.087797in}}{\pgfqpoint{1.564113in}{1.095611in}}%
\pgfpathcurveto{\pgfqpoint{1.556300in}{1.103424in}}{\pgfqpoint{1.545701in}{1.107814in}}{\pgfqpoint{1.534651in}{1.107814in}}%
\pgfpathcurveto{\pgfqpoint{1.523601in}{1.107814in}}{\pgfqpoint{1.513002in}{1.103424in}}{\pgfqpoint{1.505188in}{1.095611in}}%
\pgfpathcurveto{\pgfqpoint{1.497374in}{1.087797in}}{\pgfqpoint{1.492984in}{1.077198in}}{\pgfqpoint{1.492984in}{1.066148in}}%
\pgfpathcurveto{\pgfqpoint{1.492984in}{1.055098in}}{\pgfqpoint{1.497374in}{1.044499in}}{\pgfqpoint{1.505188in}{1.036685in}}%
\pgfpathcurveto{\pgfqpoint{1.513002in}{1.028871in}}{\pgfqpoint{1.523601in}{1.024481in}}{\pgfqpoint{1.534651in}{1.024481in}}%
\pgfpathclose%
\pgfusepath{stroke,fill}%
\end{pgfscope}%
\begin{pgfscope}%
\pgfpathrectangle{\pgfqpoint{0.772069in}{0.515123in}}{\pgfqpoint{1.937500in}{1.347500in}}%
\pgfusepath{clip}%
\pgfsetbuttcap%
\pgfsetroundjoin%
\definecolor{currentfill}{rgb}{0.121569,0.466667,0.705882}%
\pgfsetfillcolor{currentfill}%
\pgfsetlinewidth{1.003750pt}%
\definecolor{currentstroke}{rgb}{0.121569,0.466667,0.705882}%
\pgfsetstrokecolor{currentstroke}%
\pgfsetdash{}{0pt}%
\pgfpathmoveto{\pgfqpoint{1.718797in}{1.143184in}}%
\pgfpathcurveto{\pgfqpoint{1.729848in}{1.143184in}}{\pgfqpoint{1.740447in}{1.147574in}}{\pgfqpoint{1.748260in}{1.155387in}}%
\pgfpathcurveto{\pgfqpoint{1.756074in}{1.163201in}}{\pgfqpoint{1.760464in}{1.173800in}}{\pgfqpoint{1.760464in}{1.184850in}}%
\pgfpathcurveto{\pgfqpoint{1.760464in}{1.195900in}}{\pgfqpoint{1.756074in}{1.206499in}}{\pgfqpoint{1.748260in}{1.214313in}}%
\pgfpathcurveto{\pgfqpoint{1.740447in}{1.222127in}}{\pgfqpoint{1.729848in}{1.226517in}}{\pgfqpoint{1.718797in}{1.226517in}}%
\pgfpathcurveto{\pgfqpoint{1.707747in}{1.226517in}}{\pgfqpoint{1.697148in}{1.222127in}}{\pgfqpoint{1.689335in}{1.214313in}}%
\pgfpathcurveto{\pgfqpoint{1.681521in}{1.206499in}}{\pgfqpoint{1.677131in}{1.195900in}}{\pgfqpoint{1.677131in}{1.184850in}}%
\pgfpathcurveto{\pgfqpoint{1.677131in}{1.173800in}}{\pgfqpoint{1.681521in}{1.163201in}}{\pgfqpoint{1.689335in}{1.155387in}}%
\pgfpathcurveto{\pgfqpoint{1.697148in}{1.147574in}}{\pgfqpoint{1.707747in}{1.143184in}}{\pgfqpoint{1.718797in}{1.143184in}}%
\pgfpathclose%
\pgfusepath{stroke,fill}%
\end{pgfscope}%
\begin{pgfscope}%
\pgfpathrectangle{\pgfqpoint{0.772069in}{0.515123in}}{\pgfqpoint{1.937500in}{1.347500in}}%
\pgfusepath{clip}%
\pgfsetbuttcap%
\pgfsetroundjoin%
\definecolor{currentfill}{rgb}{0.121569,0.466667,0.705882}%
\pgfsetfillcolor{currentfill}%
\pgfsetlinewidth{1.003750pt}%
\definecolor{currentstroke}{rgb}{0.121569,0.466667,0.705882}%
\pgfsetstrokecolor{currentstroke}%
\pgfsetdash{}{0pt}%
\pgfpathmoveto{\pgfqpoint{1.879926in}{1.256067in}}%
\pgfpathcurveto{\pgfqpoint{1.890976in}{1.256067in}}{\pgfqpoint{1.901575in}{1.260458in}}{\pgfqpoint{1.909389in}{1.268271in}}%
\pgfpathcurveto{\pgfqpoint{1.917202in}{1.276085in}}{\pgfqpoint{1.921593in}{1.286684in}}{\pgfqpoint{1.921593in}{1.297734in}}%
\pgfpathcurveto{\pgfqpoint{1.921593in}{1.308784in}}{\pgfqpoint{1.917202in}{1.319383in}}{\pgfqpoint{1.909389in}{1.327197in}}%
\pgfpathcurveto{\pgfqpoint{1.901575in}{1.335010in}}{\pgfqpoint{1.890976in}{1.339401in}}{\pgfqpoint{1.879926in}{1.339401in}}%
\pgfpathcurveto{\pgfqpoint{1.868876in}{1.339401in}}{\pgfqpoint{1.858277in}{1.335010in}}{\pgfqpoint{1.850463in}{1.327197in}}%
\pgfpathcurveto{\pgfqpoint{1.842650in}{1.319383in}}{\pgfqpoint{1.838259in}{1.308784in}}{\pgfqpoint{1.838259in}{1.297734in}}%
\pgfpathcurveto{\pgfqpoint{1.838259in}{1.286684in}}{\pgfqpoint{1.842650in}{1.276085in}}{\pgfqpoint{1.850463in}{1.268271in}}%
\pgfpathcurveto{\pgfqpoint{1.858277in}{1.260458in}}{\pgfqpoint{1.868876in}{1.256067in}}{\pgfqpoint{1.879926in}{1.256067in}}%
\pgfpathclose%
\pgfusepath{stroke,fill}%
\end{pgfscope}%
\begin{pgfscope}%
\pgfpathrectangle{\pgfqpoint{0.772069in}{0.515123in}}{\pgfqpoint{1.937500in}{1.347500in}}%
\pgfusepath{clip}%
\pgfsetbuttcap%
\pgfsetroundjoin%
\definecolor{currentfill}{rgb}{0.121569,0.466667,0.705882}%
\pgfsetfillcolor{currentfill}%
\pgfsetlinewidth{1.003750pt}%
\definecolor{currentstroke}{rgb}{0.121569,0.466667,0.705882}%
\pgfsetstrokecolor{currentstroke}%
\pgfsetdash{}{0pt}%
\pgfpathmoveto{\pgfqpoint{2.064073in}{1.378261in}}%
\pgfpathcurveto{\pgfqpoint{2.075123in}{1.378261in}}{\pgfqpoint{2.085722in}{1.382651in}}{\pgfqpoint{2.093536in}{1.390465in}}%
\pgfpathcurveto{\pgfqpoint{2.101349in}{1.398279in}}{\pgfqpoint{2.105739in}{1.408878in}}{\pgfqpoint{2.105739in}{1.419928in}}%
\pgfpathcurveto{\pgfqpoint{2.105739in}{1.430978in}}{\pgfqpoint{2.101349in}{1.441577in}}{\pgfqpoint{2.093536in}{1.449391in}}%
\pgfpathcurveto{\pgfqpoint{2.085722in}{1.457204in}}{\pgfqpoint{2.075123in}{1.461595in}}{\pgfqpoint{2.064073in}{1.461595in}}%
\pgfpathcurveto{\pgfqpoint{2.053023in}{1.461595in}}{\pgfqpoint{2.042424in}{1.457204in}}{\pgfqpoint{2.034610in}{1.449391in}}%
\pgfpathcurveto{\pgfqpoint{2.026796in}{1.441577in}}{\pgfqpoint{2.022406in}{1.430978in}}{\pgfqpoint{2.022406in}{1.419928in}}%
\pgfpathcurveto{\pgfqpoint{2.022406in}{1.408878in}}{\pgfqpoint{2.026796in}{1.398279in}}{\pgfqpoint{2.034610in}{1.390465in}}%
\pgfpathcurveto{\pgfqpoint{2.042424in}{1.382651in}}{\pgfqpoint{2.053023in}{1.378261in}}{\pgfqpoint{2.064073in}{1.378261in}}%
\pgfpathclose%
\pgfusepath{stroke,fill}%
\end{pgfscope}%
\begin{pgfscope}%
\pgfpathrectangle{\pgfqpoint{0.772069in}{0.515123in}}{\pgfqpoint{1.937500in}{1.347500in}}%
\pgfusepath{clip}%
\pgfsetbuttcap%
\pgfsetroundjoin%
\definecolor{currentfill}{rgb}{0.121569,0.466667,0.705882}%
\pgfsetfillcolor{currentfill}%
\pgfsetlinewidth{1.003750pt}%
\definecolor{currentstroke}{rgb}{0.121569,0.466667,0.705882}%
\pgfsetstrokecolor{currentstroke}%
\pgfsetdash{}{0pt}%
\pgfpathmoveto{\pgfqpoint{2.248220in}{1.495800in}}%
\pgfpathcurveto{\pgfqpoint{2.259270in}{1.495800in}}{\pgfqpoint{2.269869in}{1.500190in}}{\pgfqpoint{2.277682in}{1.508004in}}%
\pgfpathcurveto{\pgfqpoint{2.285496in}{1.515817in}}{\pgfqpoint{2.289886in}{1.526417in}}{\pgfqpoint{2.289886in}{1.537467in}}%
\pgfpathcurveto{\pgfqpoint{2.289886in}{1.548517in}}{\pgfqpoint{2.285496in}{1.559116in}}{\pgfqpoint{2.277682in}{1.566929in}}%
\pgfpathcurveto{\pgfqpoint{2.269869in}{1.574743in}}{\pgfqpoint{2.259270in}{1.579133in}}{\pgfqpoint{2.248220in}{1.579133in}}%
\pgfpathcurveto{\pgfqpoint{2.237169in}{1.579133in}}{\pgfqpoint{2.226570in}{1.574743in}}{\pgfqpoint{2.218757in}{1.566929in}}%
\pgfpathcurveto{\pgfqpoint{2.210943in}{1.559116in}}{\pgfqpoint{2.206553in}{1.548517in}}{\pgfqpoint{2.206553in}{1.537467in}}%
\pgfpathcurveto{\pgfqpoint{2.206553in}{1.526417in}}{\pgfqpoint{2.210943in}{1.515817in}}{\pgfqpoint{2.218757in}{1.508004in}}%
\pgfpathcurveto{\pgfqpoint{2.226570in}{1.500190in}}{\pgfqpoint{2.237169in}{1.495800in}}{\pgfqpoint{2.248220in}{1.495800in}}%
\pgfpathclose%
\pgfusepath{stroke,fill}%
\end{pgfscope}%
\begin{pgfscope}%
\pgfpathrectangle{\pgfqpoint{0.772069in}{0.515123in}}{\pgfqpoint{1.937500in}{1.347500in}}%
\pgfusepath{clip}%
\pgfsetbuttcap%
\pgfsetroundjoin%
\definecolor{currentfill}{rgb}{0.121569,0.466667,0.705882}%
\pgfsetfillcolor{currentfill}%
\pgfsetlinewidth{1.003750pt}%
\definecolor{currentstroke}{rgb}{0.121569,0.466667,0.705882}%
\pgfsetstrokecolor{currentstroke}%
\pgfsetdash{}{0pt}%
\pgfpathmoveto{\pgfqpoint{2.409348in}{1.611011in}}%
\pgfpathcurveto{\pgfqpoint{2.420398in}{1.611011in}}{\pgfqpoint{2.430997in}{1.615402in}}{\pgfqpoint{2.438811in}{1.623215in}}%
\pgfpathcurveto{\pgfqpoint{2.446624in}{1.631029in}}{\pgfqpoint{2.451015in}{1.641628in}}{\pgfqpoint{2.451015in}{1.652678in}}%
\pgfpathcurveto{\pgfqpoint{2.451015in}{1.663728in}}{\pgfqpoint{2.446624in}{1.674327in}}{\pgfqpoint{2.438811in}{1.682141in}}%
\pgfpathcurveto{\pgfqpoint{2.430997in}{1.689954in}}{\pgfqpoint{2.420398in}{1.694345in}}{\pgfqpoint{2.409348in}{1.694345in}}%
\pgfpathcurveto{\pgfqpoint{2.398298in}{1.694345in}}{\pgfqpoint{2.387699in}{1.689954in}}{\pgfqpoint{2.379885in}{1.682141in}}%
\pgfpathcurveto{\pgfqpoint{2.372072in}{1.674327in}}{\pgfqpoint{2.367681in}{1.663728in}}{\pgfqpoint{2.367681in}{1.652678in}}%
\pgfpathcurveto{\pgfqpoint{2.367681in}{1.641628in}}{\pgfqpoint{2.372072in}{1.631029in}}{\pgfqpoint{2.379885in}{1.623215in}}%
\pgfpathcurveto{\pgfqpoint{2.387699in}{1.615402in}}{\pgfqpoint{2.398298in}{1.611011in}}{\pgfqpoint{2.409348in}{1.611011in}}%
\pgfpathclose%
\pgfusepath{stroke,fill}%
\end{pgfscope}%
\begin{pgfscope}%
\pgfpathrectangle{\pgfqpoint{0.772069in}{0.515123in}}{\pgfqpoint{1.937500in}{1.347500in}}%
\pgfusepath{clip}%
\pgfsetbuttcap%
\pgfsetroundjoin%
\definecolor{currentfill}{rgb}{0.121569,0.466667,0.705882}%
\pgfsetfillcolor{currentfill}%
\pgfsetlinewidth{1.003750pt}%
\definecolor{currentstroke}{rgb}{0.121569,0.466667,0.705882}%
\pgfsetstrokecolor{currentstroke}%
\pgfsetdash{}{0pt}%
\pgfpathmoveto{\pgfqpoint{2.593495in}{1.729714in}}%
\pgfpathcurveto{\pgfqpoint{2.604545in}{1.729714in}}{\pgfqpoint{2.615144in}{1.734104in}}{\pgfqpoint{2.622958in}{1.741918in}}%
\pgfpathcurveto{\pgfqpoint{2.630771in}{1.749731in}}{\pgfqpoint{2.635161in}{1.760330in}}{\pgfqpoint{2.635161in}{1.771380in}}%
\pgfpathcurveto{\pgfqpoint{2.635161in}{1.782431in}}{\pgfqpoint{2.630771in}{1.793030in}}{\pgfqpoint{2.622958in}{1.800843in}}%
\pgfpathcurveto{\pgfqpoint{2.615144in}{1.808657in}}{\pgfqpoint{2.604545in}{1.813047in}}{\pgfqpoint{2.593495in}{1.813047in}}%
\pgfpathcurveto{\pgfqpoint{2.582445in}{1.813047in}}{\pgfqpoint{2.571846in}{1.808657in}}{\pgfqpoint{2.564032in}{1.800843in}}%
\pgfpathcurveto{\pgfqpoint{2.556218in}{1.793030in}}{\pgfqpoint{2.551828in}{1.782431in}}{\pgfqpoint{2.551828in}{1.771380in}}%
\pgfpathcurveto{\pgfqpoint{2.551828in}{1.760330in}}{\pgfqpoint{2.556218in}{1.749731in}}{\pgfqpoint{2.564032in}{1.741918in}}%
\pgfpathcurveto{\pgfqpoint{2.571846in}{1.734104in}}{\pgfqpoint{2.582445in}{1.729714in}}{\pgfqpoint{2.593495in}{1.729714in}}%
\pgfpathclose%
\pgfusepath{stroke,fill}%
\end{pgfscope}%
\begin{pgfscope}%
\pgfpathrectangle{\pgfqpoint{0.772069in}{0.515123in}}{\pgfqpoint{1.937500in}{1.347500in}}%
\pgfusepath{clip}%
\pgfsetbuttcap%
\pgfsetroundjoin%
\definecolor{currentfill}{rgb}{1.000000,0.388235,0.278431}%
\pgfsetfillcolor{currentfill}%
\pgfsetlinewidth{1.003750pt}%
\definecolor{currentstroke}{rgb}{1.000000,0.388235,0.278431}%
\pgfsetstrokecolor{currentstroke}%
\pgfsetdash{}{0pt}%
\pgfpathmoveto{\pgfqpoint{1.005229in}{0.567907in}}%
\pgfpathcurveto{\pgfqpoint{1.016279in}{0.567907in}}{\pgfqpoint{1.026878in}{0.572297in}}{\pgfqpoint{1.034691in}{0.580111in}}%
\pgfpathcurveto{\pgfqpoint{1.042505in}{0.587924in}}{\pgfqpoint{1.046895in}{0.598523in}}{\pgfqpoint{1.046895in}{0.609574in}}%
\pgfpathcurveto{\pgfqpoint{1.046895in}{0.620624in}}{\pgfqpoint{1.042505in}{0.631223in}}{\pgfqpoint{1.034691in}{0.639036in}}%
\pgfpathcurveto{\pgfqpoint{1.026878in}{0.646850in}}{\pgfqpoint{1.016279in}{0.651240in}}{\pgfqpoint{1.005229in}{0.651240in}}%
\pgfpathcurveto{\pgfqpoint{0.994178in}{0.651240in}}{\pgfqpoint{0.983579in}{0.646850in}}{\pgfqpoint{0.975766in}{0.639036in}}%
\pgfpathcurveto{\pgfqpoint{0.967952in}{0.631223in}}{\pgfqpoint{0.963562in}{0.620624in}}{\pgfqpoint{0.963562in}{0.609574in}}%
\pgfpathcurveto{\pgfqpoint{0.963562in}{0.598523in}}{\pgfqpoint{0.967952in}{0.587924in}}{\pgfqpoint{0.975766in}{0.580111in}}%
\pgfpathcurveto{\pgfqpoint{0.983579in}{0.572297in}}{\pgfqpoint{0.994178in}{0.567907in}}{\pgfqpoint{1.005229in}{0.567907in}}%
\pgfpathclose%
\pgfusepath{stroke,fill}%
\end{pgfscope}%
\begin{pgfscope}%
\pgfpathrectangle{\pgfqpoint{0.772069in}{0.515123in}}{\pgfqpoint{1.937500in}{1.347500in}}%
\pgfusepath{clip}%
\pgfsetbuttcap%
\pgfsetroundjoin%
\definecolor{currentfill}{rgb}{1.000000,0.388235,0.278431}%
\pgfsetfillcolor{currentfill}%
\pgfsetlinewidth{1.003750pt}%
\definecolor{currentstroke}{rgb}{1.000000,0.388235,0.278431}%
\pgfsetstrokecolor{currentstroke}%
\pgfsetdash{}{0pt}%
\pgfpathmoveto{\pgfqpoint{1.189375in}{0.582372in}}%
\pgfpathcurveto{\pgfqpoint{1.200426in}{0.582372in}}{\pgfqpoint{1.211025in}{0.586763in}}{\pgfqpoint{1.218838in}{0.594576in}}%
\pgfpathcurveto{\pgfqpoint{1.226652in}{0.602390in}}{\pgfqpoint{1.231042in}{0.612989in}}{\pgfqpoint{1.231042in}{0.624039in}}%
\pgfpathcurveto{\pgfqpoint{1.231042in}{0.635089in}}{\pgfqpoint{1.226652in}{0.645688in}}{\pgfqpoint{1.218838in}{0.653502in}}%
\pgfpathcurveto{\pgfqpoint{1.211025in}{0.661315in}}{\pgfqpoint{1.200426in}{0.665706in}}{\pgfqpoint{1.189375in}{0.665706in}}%
\pgfpathcurveto{\pgfqpoint{1.178325in}{0.665706in}}{\pgfqpoint{1.167726in}{0.661315in}}{\pgfqpoint{1.159913in}{0.653502in}}%
\pgfpathcurveto{\pgfqpoint{1.152099in}{0.645688in}}{\pgfqpoint{1.147709in}{0.635089in}}{\pgfqpoint{1.147709in}{0.624039in}}%
\pgfpathcurveto{\pgfqpoint{1.147709in}{0.612989in}}{\pgfqpoint{1.152099in}{0.602390in}}{\pgfqpoint{1.159913in}{0.594576in}}%
\pgfpathcurveto{\pgfqpoint{1.167726in}{0.586763in}}{\pgfqpoint{1.178325in}{0.582372in}}{\pgfqpoint{1.189375in}{0.582372in}}%
\pgfpathclose%
\pgfusepath{stroke,fill}%
\end{pgfscope}%
\begin{pgfscope}%
\pgfpathrectangle{\pgfqpoint{0.772069in}{0.515123in}}{\pgfqpoint{1.937500in}{1.347500in}}%
\pgfusepath{clip}%
\pgfsetbuttcap%
\pgfsetroundjoin%
\definecolor{currentfill}{rgb}{1.000000,0.388235,0.278431}%
\pgfsetfillcolor{currentfill}%
\pgfsetlinewidth{1.003750pt}%
\definecolor{currentstroke}{rgb}{1.000000,0.388235,0.278431}%
\pgfsetstrokecolor{currentstroke}%
\pgfsetdash{}{0pt}%
\pgfpathmoveto{\pgfqpoint{1.350504in}{0.586562in}}%
\pgfpathcurveto{\pgfqpoint{1.361554in}{0.586562in}}{\pgfqpoint{1.372153in}{0.590952in}}{\pgfqpoint{1.379967in}{0.598766in}}%
\pgfpathcurveto{\pgfqpoint{1.387780in}{0.606579in}}{\pgfqpoint{1.392171in}{0.617178in}}{\pgfqpoint{1.392171in}{0.628228in}}%
\pgfpathcurveto{\pgfqpoint{1.392171in}{0.639279in}}{\pgfqpoint{1.387780in}{0.649878in}}{\pgfqpoint{1.379967in}{0.657691in}}%
\pgfpathcurveto{\pgfqpoint{1.372153in}{0.665505in}}{\pgfqpoint{1.361554in}{0.669895in}}{\pgfqpoint{1.350504in}{0.669895in}}%
\pgfpathcurveto{\pgfqpoint{1.339454in}{0.669895in}}{\pgfqpoint{1.328855in}{0.665505in}}{\pgfqpoint{1.321041in}{0.657691in}}%
\pgfpathcurveto{\pgfqpoint{1.313227in}{0.649878in}}{\pgfqpoint{1.308837in}{0.639279in}}{\pgfqpoint{1.308837in}{0.628228in}}%
\pgfpathcurveto{\pgfqpoint{1.308837in}{0.617178in}}{\pgfqpoint{1.313227in}{0.606579in}}{\pgfqpoint{1.321041in}{0.598766in}}%
\pgfpathcurveto{\pgfqpoint{1.328855in}{0.590952in}}{\pgfqpoint{1.339454in}{0.586562in}}{\pgfqpoint{1.350504in}{0.586562in}}%
\pgfpathclose%
\pgfusepath{stroke,fill}%
\end{pgfscope}%
\begin{pgfscope}%
\pgfpathrectangle{\pgfqpoint{0.772069in}{0.515123in}}{\pgfqpoint{1.937500in}{1.347500in}}%
\pgfusepath{clip}%
\pgfsetbuttcap%
\pgfsetroundjoin%
\definecolor{currentfill}{rgb}{1.000000,0.388235,0.278431}%
\pgfsetfillcolor{currentfill}%
\pgfsetlinewidth{1.003750pt}%
\definecolor{currentstroke}{rgb}{1.000000,0.388235,0.278431}%
\pgfsetstrokecolor{currentstroke}%
\pgfsetdash{}{0pt}%
\pgfpathmoveto{\pgfqpoint{1.534651in}{0.613794in}}%
\pgfpathcurveto{\pgfqpoint{1.545701in}{0.613794in}}{\pgfqpoint{1.556300in}{0.618184in}}{\pgfqpoint{1.564113in}{0.625997in}}%
\pgfpathcurveto{\pgfqpoint{1.571927in}{0.633811in}}{\pgfqpoint{1.576317in}{0.644410in}}{\pgfqpoint{1.576317in}{0.655460in}}%
\pgfpathcurveto{\pgfqpoint{1.576317in}{0.666510in}}{\pgfqpoint{1.571927in}{0.677109in}}{\pgfqpoint{1.564113in}{0.684923in}}%
\pgfpathcurveto{\pgfqpoint{1.556300in}{0.692737in}}{\pgfqpoint{1.545701in}{0.697127in}}{\pgfqpoint{1.534651in}{0.697127in}}%
\pgfpathcurveto{\pgfqpoint{1.523601in}{0.697127in}}{\pgfqpoint{1.513002in}{0.692737in}}{\pgfqpoint{1.505188in}{0.684923in}}%
\pgfpathcurveto{\pgfqpoint{1.497374in}{0.677109in}}{\pgfqpoint{1.492984in}{0.666510in}}{\pgfqpoint{1.492984in}{0.655460in}}%
\pgfpathcurveto{\pgfqpoint{1.492984in}{0.644410in}}{\pgfqpoint{1.497374in}{0.633811in}}{\pgfqpoint{1.505188in}{0.625997in}}%
\pgfpathcurveto{\pgfqpoint{1.513002in}{0.618184in}}{\pgfqpoint{1.523601in}{0.613794in}}{\pgfqpoint{1.534651in}{0.613794in}}%
\pgfpathclose%
\pgfusepath{stroke,fill}%
\end{pgfscope}%
\begin{pgfscope}%
\pgfpathrectangle{\pgfqpoint{0.772069in}{0.515123in}}{\pgfqpoint{1.937500in}{1.347500in}}%
\pgfusepath{clip}%
\pgfsetbuttcap%
\pgfsetroundjoin%
\definecolor{currentfill}{rgb}{1.000000,0.388235,0.278431}%
\pgfsetfillcolor{currentfill}%
\pgfsetlinewidth{1.003750pt}%
\definecolor{currentstroke}{rgb}{1.000000,0.388235,0.278431}%
\pgfsetstrokecolor{currentstroke}%
\pgfsetdash{}{0pt}%
\pgfpathmoveto{\pgfqpoint{1.718797in}{0.629271in}}%
\pgfpathcurveto{\pgfqpoint{1.729848in}{0.629271in}}{\pgfqpoint{1.740447in}{0.633662in}}{\pgfqpoint{1.748260in}{0.641475in}}%
\pgfpathcurveto{\pgfqpoint{1.756074in}{0.649289in}}{\pgfqpoint{1.760464in}{0.659888in}}{\pgfqpoint{1.760464in}{0.670938in}}%
\pgfpathcurveto{\pgfqpoint{1.760464in}{0.681988in}}{\pgfqpoint{1.756074in}{0.692587in}}{\pgfqpoint{1.748260in}{0.700401in}}%
\pgfpathcurveto{\pgfqpoint{1.740447in}{0.708214in}}{\pgfqpoint{1.729848in}{0.712605in}}{\pgfqpoint{1.718797in}{0.712605in}}%
\pgfpathcurveto{\pgfqpoint{1.707747in}{0.712605in}}{\pgfqpoint{1.697148in}{0.708214in}}{\pgfqpoint{1.689335in}{0.700401in}}%
\pgfpathcurveto{\pgfqpoint{1.681521in}{0.692587in}}{\pgfqpoint{1.677131in}{0.681988in}}{\pgfqpoint{1.677131in}{0.670938in}}%
\pgfpathcurveto{\pgfqpoint{1.677131in}{0.659888in}}{\pgfqpoint{1.681521in}{0.649289in}}{\pgfqpoint{1.689335in}{0.641475in}}%
\pgfpathcurveto{\pgfqpoint{1.697148in}{0.633662in}}{\pgfqpoint{1.707747in}{0.629271in}}{\pgfqpoint{1.718797in}{0.629271in}}%
\pgfpathclose%
\pgfusepath{stroke,fill}%
\end{pgfscope}%
\begin{pgfscope}%
\pgfpathrectangle{\pgfqpoint{0.772069in}{0.515123in}}{\pgfqpoint{1.937500in}{1.347500in}}%
\pgfusepath{clip}%
\pgfsetbuttcap%
\pgfsetroundjoin%
\definecolor{currentfill}{rgb}{1.000000,0.388235,0.278431}%
\pgfsetfillcolor{currentfill}%
\pgfsetlinewidth{1.003750pt}%
\definecolor{currentstroke}{rgb}{1.000000,0.388235,0.278431}%
\pgfsetstrokecolor{currentstroke}%
\pgfsetdash{}{0pt}%
\pgfpathmoveto{\pgfqpoint{1.879926in}{0.644051in}}%
\pgfpathcurveto{\pgfqpoint{1.890976in}{0.644051in}}{\pgfqpoint{1.901575in}{0.648441in}}{\pgfqpoint{1.909389in}{0.656255in}}%
\pgfpathcurveto{\pgfqpoint{1.917202in}{0.664069in}}{\pgfqpoint{1.921593in}{0.674668in}}{\pgfqpoint{1.921593in}{0.685718in}}%
\pgfpathcurveto{\pgfqpoint{1.921593in}{0.696768in}}{\pgfqpoint{1.917202in}{0.707367in}}{\pgfqpoint{1.909389in}{0.715181in}}%
\pgfpathcurveto{\pgfqpoint{1.901575in}{0.722994in}}{\pgfqpoint{1.890976in}{0.727384in}}{\pgfqpoint{1.879926in}{0.727384in}}%
\pgfpathcurveto{\pgfqpoint{1.868876in}{0.727384in}}{\pgfqpoint{1.858277in}{0.722994in}}{\pgfqpoint{1.850463in}{0.715181in}}%
\pgfpathcurveto{\pgfqpoint{1.842650in}{0.707367in}}{\pgfqpoint{1.838259in}{0.696768in}}{\pgfqpoint{1.838259in}{0.685718in}}%
\pgfpathcurveto{\pgfqpoint{1.838259in}{0.674668in}}{\pgfqpoint{1.842650in}{0.664069in}}{\pgfqpoint{1.850463in}{0.656255in}}%
\pgfpathcurveto{\pgfqpoint{1.858277in}{0.648441in}}{\pgfqpoint{1.868876in}{0.644051in}}{\pgfqpoint{1.879926in}{0.644051in}}%
\pgfpathclose%
\pgfusepath{stroke,fill}%
\end{pgfscope}%
\begin{pgfscope}%
\pgfpathrectangle{\pgfqpoint{0.772069in}{0.515123in}}{\pgfqpoint{1.937500in}{1.347500in}}%
\pgfusepath{clip}%
\pgfsetbuttcap%
\pgfsetroundjoin%
\definecolor{currentfill}{rgb}{1.000000,0.388235,0.278431}%
\pgfsetfillcolor{currentfill}%
\pgfsetlinewidth{1.003750pt}%
\definecolor{currentstroke}{rgb}{1.000000,0.388235,0.278431}%
\pgfsetstrokecolor{currentstroke}%
\pgfsetdash{}{0pt}%
\pgfpathmoveto{\pgfqpoint{2.064073in}{0.660111in}}%
\pgfpathcurveto{\pgfqpoint{2.075123in}{0.660111in}}{\pgfqpoint{2.085722in}{0.664501in}}{\pgfqpoint{2.093536in}{0.672315in}}%
\pgfpathcurveto{\pgfqpoint{2.101349in}{0.680128in}}{\pgfqpoint{2.105739in}{0.690727in}}{\pgfqpoint{2.105739in}{0.701777in}}%
\pgfpathcurveto{\pgfqpoint{2.105739in}{0.712828in}}{\pgfqpoint{2.101349in}{0.723427in}}{\pgfqpoint{2.093536in}{0.731240in}}%
\pgfpathcurveto{\pgfqpoint{2.085722in}{0.739054in}}{\pgfqpoint{2.075123in}{0.743444in}}{\pgfqpoint{2.064073in}{0.743444in}}%
\pgfpathcurveto{\pgfqpoint{2.053023in}{0.743444in}}{\pgfqpoint{2.042424in}{0.739054in}}{\pgfqpoint{2.034610in}{0.731240in}}%
\pgfpathcurveto{\pgfqpoint{2.026796in}{0.723427in}}{\pgfqpoint{2.022406in}{0.712828in}}{\pgfqpoint{2.022406in}{0.701777in}}%
\pgfpathcurveto{\pgfqpoint{2.022406in}{0.690727in}}{\pgfqpoint{2.026796in}{0.680128in}}{\pgfqpoint{2.034610in}{0.672315in}}%
\pgfpathcurveto{\pgfqpoint{2.042424in}{0.664501in}}{\pgfqpoint{2.053023in}{0.660111in}}{\pgfqpoint{2.064073in}{0.660111in}}%
\pgfpathclose%
\pgfusepath{stroke,fill}%
\end{pgfscope}%
\begin{pgfscope}%
\pgfpathrectangle{\pgfqpoint{0.772069in}{0.515123in}}{\pgfqpoint{1.937500in}{1.347500in}}%
\pgfusepath{clip}%
\pgfsetbuttcap%
\pgfsetroundjoin%
\definecolor{currentfill}{rgb}{1.000000,0.388235,0.278431}%
\pgfsetfillcolor{currentfill}%
\pgfsetlinewidth{1.003750pt}%
\definecolor{currentstroke}{rgb}{1.000000,0.388235,0.278431}%
\pgfsetstrokecolor{currentstroke}%
\pgfsetdash{}{0pt}%
\pgfpathmoveto{\pgfqpoint{2.248220in}{0.675240in}}%
\pgfpathcurveto{\pgfqpoint{2.259270in}{0.675240in}}{\pgfqpoint{2.269869in}{0.679630in}}{\pgfqpoint{2.277682in}{0.687443in}}%
\pgfpathcurveto{\pgfqpoint{2.285496in}{0.695257in}}{\pgfqpoint{2.289886in}{0.705856in}}{\pgfqpoint{2.289886in}{0.716906in}}%
\pgfpathcurveto{\pgfqpoint{2.289886in}{0.727956in}}{\pgfqpoint{2.285496in}{0.738555in}}{\pgfqpoint{2.277682in}{0.746369in}}%
\pgfpathcurveto{\pgfqpoint{2.269869in}{0.754183in}}{\pgfqpoint{2.259270in}{0.758573in}}{\pgfqpoint{2.248220in}{0.758573in}}%
\pgfpathcurveto{\pgfqpoint{2.237169in}{0.758573in}}{\pgfqpoint{2.226570in}{0.754183in}}{\pgfqpoint{2.218757in}{0.746369in}}%
\pgfpathcurveto{\pgfqpoint{2.210943in}{0.738555in}}{\pgfqpoint{2.206553in}{0.727956in}}{\pgfqpoint{2.206553in}{0.716906in}}%
\pgfpathcurveto{\pgfqpoint{2.206553in}{0.705856in}}{\pgfqpoint{2.210943in}{0.695257in}}{\pgfqpoint{2.218757in}{0.687443in}}%
\pgfpathcurveto{\pgfqpoint{2.226570in}{0.679630in}}{\pgfqpoint{2.237169in}{0.675240in}}{\pgfqpoint{2.248220in}{0.675240in}}%
\pgfpathclose%
\pgfusepath{stroke,fill}%
\end{pgfscope}%
\begin{pgfscope}%
\pgfpathrectangle{\pgfqpoint{0.772069in}{0.515123in}}{\pgfqpoint{1.937500in}{1.347500in}}%
\pgfusepath{clip}%
\pgfsetbuttcap%
\pgfsetroundjoin%
\definecolor{currentfill}{rgb}{1.000000,0.388235,0.278431}%
\pgfsetfillcolor{currentfill}%
\pgfsetlinewidth{1.003750pt}%
\definecolor{currentstroke}{rgb}{1.000000,0.388235,0.278431}%
\pgfsetstrokecolor{currentstroke}%
\pgfsetdash{}{0pt}%
\pgfpathmoveto{\pgfqpoint{2.409348in}{0.690368in}}%
\pgfpathcurveto{\pgfqpoint{2.420398in}{0.690368in}}{\pgfqpoint{2.430997in}{0.694759in}}{\pgfqpoint{2.438811in}{0.702572in}}%
\pgfpathcurveto{\pgfqpoint{2.446624in}{0.710386in}}{\pgfqpoint{2.451015in}{0.720985in}}{\pgfqpoint{2.451015in}{0.732035in}}%
\pgfpathcurveto{\pgfqpoint{2.451015in}{0.743085in}}{\pgfqpoint{2.446624in}{0.753684in}}{\pgfqpoint{2.438811in}{0.761498in}}%
\pgfpathcurveto{\pgfqpoint{2.430997in}{0.769311in}}{\pgfqpoint{2.420398in}{0.773702in}}{\pgfqpoint{2.409348in}{0.773702in}}%
\pgfpathcurveto{\pgfqpoint{2.398298in}{0.773702in}}{\pgfqpoint{2.387699in}{0.769311in}}{\pgfqpoint{2.379885in}{0.761498in}}%
\pgfpathcurveto{\pgfqpoint{2.372072in}{0.753684in}}{\pgfqpoint{2.367681in}{0.743085in}}{\pgfqpoint{2.367681in}{0.732035in}}%
\pgfpathcurveto{\pgfqpoint{2.367681in}{0.720985in}}{\pgfqpoint{2.372072in}{0.710386in}}{\pgfqpoint{2.379885in}{0.702572in}}%
\pgfpathcurveto{\pgfqpoint{2.387699in}{0.694759in}}{\pgfqpoint{2.398298in}{0.690368in}}{\pgfqpoint{2.409348in}{0.690368in}}%
\pgfpathclose%
\pgfusepath{stroke,fill}%
\end{pgfscope}%
\begin{pgfscope}%
\pgfpathrectangle{\pgfqpoint{0.772069in}{0.515123in}}{\pgfqpoint{1.937500in}{1.347500in}}%
\pgfusepath{clip}%
\pgfsetbuttcap%
\pgfsetroundjoin%
\definecolor{currentfill}{rgb}{1.000000,0.388235,0.278431}%
\pgfsetfillcolor{currentfill}%
\pgfsetlinewidth{1.003750pt}%
\definecolor{currentstroke}{rgb}{1.000000,0.388235,0.278431}%
\pgfsetstrokecolor{currentstroke}%
\pgfsetdash{}{0pt}%
\pgfpathmoveto{\pgfqpoint{2.593495in}{0.705963in}}%
\pgfpathcurveto{\pgfqpoint{2.604545in}{0.705963in}}{\pgfqpoint{2.615144in}{0.710353in}}{\pgfqpoint{2.622958in}{0.718166in}}%
\pgfpathcurveto{\pgfqpoint{2.630771in}{0.725980in}}{\pgfqpoint{2.635161in}{0.736579in}}{\pgfqpoint{2.635161in}{0.747629in}}%
\pgfpathcurveto{\pgfqpoint{2.635161in}{0.758679in}}{\pgfqpoint{2.630771in}{0.769278in}}{\pgfqpoint{2.622958in}{0.777092in}}%
\pgfpathcurveto{\pgfqpoint{2.615144in}{0.784906in}}{\pgfqpoint{2.604545in}{0.789296in}}{\pgfqpoint{2.593495in}{0.789296in}}%
\pgfpathcurveto{\pgfqpoint{2.582445in}{0.789296in}}{\pgfqpoint{2.571846in}{0.784906in}}{\pgfqpoint{2.564032in}{0.777092in}}%
\pgfpathcurveto{\pgfqpoint{2.556218in}{0.769278in}}{\pgfqpoint{2.551828in}{0.758679in}}{\pgfqpoint{2.551828in}{0.747629in}}%
\pgfpathcurveto{\pgfqpoint{2.551828in}{0.736579in}}{\pgfqpoint{2.556218in}{0.725980in}}{\pgfqpoint{2.564032in}{0.718166in}}%
\pgfpathcurveto{\pgfqpoint{2.571846in}{0.710353in}}{\pgfqpoint{2.582445in}{0.705963in}}{\pgfqpoint{2.593495in}{0.705963in}}%
\pgfpathclose%
\pgfusepath{stroke,fill}%
\end{pgfscope}%
\begin{pgfscope}%
\pgfpathrectangle{\pgfqpoint{0.772069in}{0.515123in}}{\pgfqpoint{1.937500in}{1.347500in}}%
\pgfusepath{clip}%
\pgfsetbuttcap%
\pgfsetroundjoin%
\definecolor{currentfill}{rgb}{1.000000,0.843137,0.000000}%
\pgfsetfillcolor{currentfill}%
\pgfsetlinewidth{1.003750pt}%
\definecolor{currentstroke}{rgb}{1.000000,0.843137,0.000000}%
\pgfsetstrokecolor{currentstroke}%
\pgfsetdash{}{0pt}%
\pgfpathmoveto{\pgfqpoint{1.005229in}{0.564707in}}%
\pgfpathcurveto{\pgfqpoint{1.016279in}{0.564707in}}{\pgfqpoint{1.026878in}{0.569097in}}{\pgfqpoint{1.034691in}{0.576910in}}%
\pgfpathcurveto{\pgfqpoint{1.042505in}{0.584724in}}{\pgfqpoint{1.046895in}{0.595323in}}{\pgfqpoint{1.046895in}{0.606373in}}%
\pgfpathcurveto{\pgfqpoint{1.046895in}{0.617423in}}{\pgfqpoint{1.042505in}{0.628022in}}{\pgfqpoint{1.034691in}{0.635836in}}%
\pgfpathcurveto{\pgfqpoint{1.026878in}{0.643650in}}{\pgfqpoint{1.016279in}{0.648040in}}{\pgfqpoint{1.005229in}{0.648040in}}%
\pgfpathcurveto{\pgfqpoint{0.994178in}{0.648040in}}{\pgfqpoint{0.983579in}{0.643650in}}{\pgfqpoint{0.975766in}{0.635836in}}%
\pgfpathcurveto{\pgfqpoint{0.967952in}{0.628022in}}{\pgfqpoint{0.963562in}{0.617423in}}{\pgfqpoint{0.963562in}{0.606373in}}%
\pgfpathcurveto{\pgfqpoint{0.963562in}{0.595323in}}{\pgfqpoint{0.967952in}{0.584724in}}{\pgfqpoint{0.975766in}{0.576910in}}%
\pgfpathcurveto{\pgfqpoint{0.983579in}{0.569097in}}{\pgfqpoint{0.994178in}{0.564707in}}{\pgfqpoint{1.005229in}{0.564707in}}%
\pgfpathclose%
\pgfusepath{stroke,fill}%
\end{pgfscope}%
\begin{pgfscope}%
\pgfpathrectangle{\pgfqpoint{0.772069in}{0.515123in}}{\pgfqpoint{1.937500in}{1.347500in}}%
\pgfusepath{clip}%
\pgfsetbuttcap%
\pgfsetroundjoin%
\definecolor{currentfill}{rgb}{1.000000,0.843137,0.000000}%
\pgfsetfillcolor{currentfill}%
\pgfsetlinewidth{1.003750pt}%
\definecolor{currentstroke}{rgb}{1.000000,0.843137,0.000000}%
\pgfsetstrokecolor{currentstroke}%
\pgfsetdash{}{0pt}%
\pgfpathmoveto{\pgfqpoint{1.189375in}{0.576321in}}%
\pgfpathcurveto{\pgfqpoint{1.200426in}{0.576321in}}{\pgfqpoint{1.211025in}{0.580711in}}{\pgfqpoint{1.218838in}{0.588525in}}%
\pgfpathcurveto{\pgfqpoint{1.226652in}{0.596338in}}{\pgfqpoint{1.231042in}{0.606937in}}{\pgfqpoint{1.231042in}{0.617987in}}%
\pgfpathcurveto{\pgfqpoint{1.231042in}{0.629038in}}{\pgfqpoint{1.226652in}{0.639637in}}{\pgfqpoint{1.218838in}{0.647450in}}%
\pgfpathcurveto{\pgfqpoint{1.211025in}{0.655264in}}{\pgfqpoint{1.200426in}{0.659654in}}{\pgfqpoint{1.189375in}{0.659654in}}%
\pgfpathcurveto{\pgfqpoint{1.178325in}{0.659654in}}{\pgfqpoint{1.167726in}{0.655264in}}{\pgfqpoint{1.159913in}{0.647450in}}%
\pgfpathcurveto{\pgfqpoint{1.152099in}{0.639637in}}{\pgfqpoint{1.147709in}{0.629038in}}{\pgfqpoint{1.147709in}{0.617987in}}%
\pgfpathcurveto{\pgfqpoint{1.147709in}{0.606937in}}{\pgfqpoint{1.152099in}{0.596338in}}{\pgfqpoint{1.159913in}{0.588525in}}%
\pgfpathcurveto{\pgfqpoint{1.167726in}{0.580711in}}{\pgfqpoint{1.178325in}{0.576321in}}{\pgfqpoint{1.189375in}{0.576321in}}%
\pgfpathclose%
\pgfusepath{stroke,fill}%
\end{pgfscope}%
\begin{pgfscope}%
\pgfpathrectangle{\pgfqpoint{0.772069in}{0.515123in}}{\pgfqpoint{1.937500in}{1.347500in}}%
\pgfusepath{clip}%
\pgfsetbuttcap%
\pgfsetroundjoin%
\definecolor{currentfill}{rgb}{1.000000,0.843137,0.000000}%
\pgfsetfillcolor{currentfill}%
\pgfsetlinewidth{1.003750pt}%
\definecolor{currentstroke}{rgb}{1.000000,0.843137,0.000000}%
\pgfsetstrokecolor{currentstroke}%
\pgfsetdash{}{0pt}%
\pgfpathmoveto{\pgfqpoint{1.350504in}{0.588889in}}%
\pgfpathcurveto{\pgfqpoint{1.361554in}{0.588889in}}{\pgfqpoint{1.372153in}{0.593280in}}{\pgfqpoint{1.379967in}{0.601093in}}%
\pgfpathcurveto{\pgfqpoint{1.387780in}{0.608907in}}{\pgfqpoint{1.392171in}{0.619506in}}{\pgfqpoint{1.392171in}{0.630556in}}%
\pgfpathcurveto{\pgfqpoint{1.392171in}{0.641606in}}{\pgfqpoint{1.387780in}{0.652205in}}{\pgfqpoint{1.379967in}{0.660019in}}%
\pgfpathcurveto{\pgfqpoint{1.372153in}{0.667832in}}{\pgfqpoint{1.361554in}{0.672223in}}{\pgfqpoint{1.350504in}{0.672223in}}%
\pgfpathcurveto{\pgfqpoint{1.339454in}{0.672223in}}{\pgfqpoint{1.328855in}{0.667832in}}{\pgfqpoint{1.321041in}{0.660019in}}%
\pgfpathcurveto{\pgfqpoint{1.313227in}{0.652205in}}{\pgfqpoint{1.308837in}{0.641606in}}{\pgfqpoint{1.308837in}{0.630556in}}%
\pgfpathcurveto{\pgfqpoint{1.308837in}{0.619506in}}{\pgfqpoint{1.313227in}{0.608907in}}{\pgfqpoint{1.321041in}{0.601093in}}%
\pgfpathcurveto{\pgfqpoint{1.328855in}{0.593280in}}{\pgfqpoint{1.339454in}{0.588889in}}{\pgfqpoint{1.350504in}{0.588889in}}%
\pgfpathclose%
\pgfusepath{stroke,fill}%
\end{pgfscope}%
\begin{pgfscope}%
\pgfpathrectangle{\pgfqpoint{0.772069in}{0.515123in}}{\pgfqpoint{1.937500in}{1.347500in}}%
\pgfusepath{clip}%
\pgfsetbuttcap%
\pgfsetroundjoin%
\definecolor{currentfill}{rgb}{1.000000,0.843137,0.000000}%
\pgfsetfillcolor{currentfill}%
\pgfsetlinewidth{1.003750pt}%
\definecolor{currentstroke}{rgb}{1.000000,0.843137,0.000000}%
\pgfsetstrokecolor{currentstroke}%
\pgfsetdash{}{0pt}%
\pgfpathmoveto{\pgfqpoint{1.534651in}{0.601225in}}%
\pgfpathcurveto{\pgfqpoint{1.545701in}{0.601225in}}{\pgfqpoint{1.556300in}{0.605615in}}{\pgfqpoint{1.564113in}{0.613429in}}%
\pgfpathcurveto{\pgfqpoint{1.571927in}{0.621243in}}{\pgfqpoint{1.576317in}{0.631842in}}{\pgfqpoint{1.576317in}{0.642892in}}%
\pgfpathcurveto{\pgfqpoint{1.576317in}{0.653942in}}{\pgfqpoint{1.571927in}{0.664541in}}{\pgfqpoint{1.564113in}{0.672354in}}%
\pgfpathcurveto{\pgfqpoint{1.556300in}{0.680168in}}{\pgfqpoint{1.545701in}{0.684558in}}{\pgfqpoint{1.534651in}{0.684558in}}%
\pgfpathcurveto{\pgfqpoint{1.523601in}{0.684558in}}{\pgfqpoint{1.513002in}{0.680168in}}{\pgfqpoint{1.505188in}{0.672354in}}%
\pgfpathcurveto{\pgfqpoint{1.497374in}{0.664541in}}{\pgfqpoint{1.492984in}{0.653942in}}{\pgfqpoint{1.492984in}{0.642892in}}%
\pgfpathcurveto{\pgfqpoint{1.492984in}{0.631842in}}{\pgfqpoint{1.497374in}{0.621243in}}{\pgfqpoint{1.505188in}{0.613429in}}%
\pgfpathcurveto{\pgfqpoint{1.513002in}{0.605615in}}{\pgfqpoint{1.523601in}{0.601225in}}{\pgfqpoint{1.534651in}{0.601225in}}%
\pgfpathclose%
\pgfusepath{stroke,fill}%
\end{pgfscope}%
\begin{pgfscope}%
\pgfpathrectangle{\pgfqpoint{0.772069in}{0.515123in}}{\pgfqpoint{1.937500in}{1.347500in}}%
\pgfusepath{clip}%
\pgfsetbuttcap%
\pgfsetroundjoin%
\definecolor{currentfill}{rgb}{1.000000,0.843137,0.000000}%
\pgfsetfillcolor{currentfill}%
\pgfsetlinewidth{1.003750pt}%
\definecolor{currentstroke}{rgb}{1.000000,0.843137,0.000000}%
\pgfsetstrokecolor{currentstroke}%
\pgfsetdash{}{0pt}%
\pgfpathmoveto{\pgfqpoint{1.718797in}{0.613561in}}%
\pgfpathcurveto{\pgfqpoint{1.729848in}{0.613561in}}{\pgfqpoint{1.740447in}{0.617951in}}{\pgfqpoint{1.748260in}{0.625765in}}%
\pgfpathcurveto{\pgfqpoint{1.756074in}{0.633578in}}{\pgfqpoint{1.760464in}{0.644177in}}{\pgfqpoint{1.760464in}{0.655227in}}%
\pgfpathcurveto{\pgfqpoint{1.760464in}{0.666278in}}{\pgfqpoint{1.756074in}{0.676877in}}{\pgfqpoint{1.748260in}{0.684690in}}%
\pgfpathcurveto{\pgfqpoint{1.740447in}{0.692504in}}{\pgfqpoint{1.729848in}{0.696894in}}{\pgfqpoint{1.718797in}{0.696894in}}%
\pgfpathcurveto{\pgfqpoint{1.707747in}{0.696894in}}{\pgfqpoint{1.697148in}{0.692504in}}{\pgfqpoint{1.689335in}{0.684690in}}%
\pgfpathcurveto{\pgfqpoint{1.681521in}{0.676877in}}{\pgfqpoint{1.677131in}{0.666278in}}{\pgfqpoint{1.677131in}{0.655227in}}%
\pgfpathcurveto{\pgfqpoint{1.677131in}{0.644177in}}{\pgfqpoint{1.681521in}{0.633578in}}{\pgfqpoint{1.689335in}{0.625765in}}%
\pgfpathcurveto{\pgfqpoint{1.697148in}{0.617951in}}{\pgfqpoint{1.707747in}{0.613561in}}{\pgfqpoint{1.718797in}{0.613561in}}%
\pgfpathclose%
\pgfusepath{stroke,fill}%
\end{pgfscope}%
\begin{pgfscope}%
\pgfpathrectangle{\pgfqpoint{0.772069in}{0.515123in}}{\pgfqpoint{1.937500in}{1.347500in}}%
\pgfusepath{clip}%
\pgfsetbuttcap%
\pgfsetroundjoin%
\definecolor{currentfill}{rgb}{1.000000,0.843137,0.000000}%
\pgfsetfillcolor{currentfill}%
\pgfsetlinewidth{1.003750pt}%
\definecolor{currentstroke}{rgb}{1.000000,0.843137,0.000000}%
\pgfsetstrokecolor{currentstroke}%
\pgfsetdash{}{0pt}%
\pgfpathmoveto{\pgfqpoint{1.879926in}{0.625315in}}%
\pgfpathcurveto{\pgfqpoint{1.890976in}{0.625315in}}{\pgfqpoint{1.901575in}{0.629705in}}{\pgfqpoint{1.909389in}{0.637519in}}%
\pgfpathcurveto{\pgfqpoint{1.917202in}{0.645332in}}{\pgfqpoint{1.921593in}{0.655931in}}{\pgfqpoint{1.921593in}{0.666981in}}%
\pgfpathcurveto{\pgfqpoint{1.921593in}{0.678031in}}{\pgfqpoint{1.917202in}{0.688631in}}{\pgfqpoint{1.909389in}{0.696444in}}%
\pgfpathcurveto{\pgfqpoint{1.901575in}{0.704258in}}{\pgfqpoint{1.890976in}{0.708648in}}{\pgfqpoint{1.879926in}{0.708648in}}%
\pgfpathcurveto{\pgfqpoint{1.868876in}{0.708648in}}{\pgfqpoint{1.858277in}{0.704258in}}{\pgfqpoint{1.850463in}{0.696444in}}%
\pgfpathcurveto{\pgfqpoint{1.842650in}{0.688631in}}{\pgfqpoint{1.838259in}{0.678031in}}{\pgfqpoint{1.838259in}{0.666981in}}%
\pgfpathcurveto{\pgfqpoint{1.838259in}{0.655931in}}{\pgfqpoint{1.842650in}{0.645332in}}{\pgfqpoint{1.850463in}{0.637519in}}%
\pgfpathcurveto{\pgfqpoint{1.858277in}{0.629705in}}{\pgfqpoint{1.868876in}{0.625315in}}{\pgfqpoint{1.879926in}{0.625315in}}%
\pgfpathclose%
\pgfusepath{stroke,fill}%
\end{pgfscope}%
\begin{pgfscope}%
\pgfpathrectangle{\pgfqpoint{0.772069in}{0.515123in}}{\pgfqpoint{1.937500in}{1.347500in}}%
\pgfusepath{clip}%
\pgfsetbuttcap%
\pgfsetroundjoin%
\definecolor{currentfill}{rgb}{1.000000,0.843137,0.000000}%
\pgfsetfillcolor{currentfill}%
\pgfsetlinewidth{1.003750pt}%
\definecolor{currentstroke}{rgb}{1.000000,0.843137,0.000000}%
\pgfsetstrokecolor{currentstroke}%
\pgfsetdash{}{0pt}%
\pgfpathmoveto{\pgfqpoint{2.064073in}{0.644284in}}%
\pgfpathcurveto{\pgfqpoint{2.075123in}{0.644284in}}{\pgfqpoint{2.085722in}{0.648674in}}{\pgfqpoint{2.093536in}{0.656488in}}%
\pgfpathcurveto{\pgfqpoint{2.101349in}{0.664301in}}{\pgfqpoint{2.105739in}{0.674900in}}{\pgfqpoint{2.105739in}{0.685950in}}%
\pgfpathcurveto{\pgfqpoint{2.105739in}{0.697001in}}{\pgfqpoint{2.101349in}{0.707600in}}{\pgfqpoint{2.093536in}{0.715413in}}%
\pgfpathcurveto{\pgfqpoint{2.085722in}{0.723227in}}{\pgfqpoint{2.075123in}{0.727617in}}{\pgfqpoint{2.064073in}{0.727617in}}%
\pgfpathcurveto{\pgfqpoint{2.053023in}{0.727617in}}{\pgfqpoint{2.042424in}{0.723227in}}{\pgfqpoint{2.034610in}{0.715413in}}%
\pgfpathcurveto{\pgfqpoint{2.026796in}{0.707600in}}{\pgfqpoint{2.022406in}{0.697001in}}{\pgfqpoint{2.022406in}{0.685950in}}%
\pgfpathcurveto{\pgfqpoint{2.022406in}{0.674900in}}{\pgfqpoint{2.026796in}{0.664301in}}{\pgfqpoint{2.034610in}{0.656488in}}%
\pgfpathcurveto{\pgfqpoint{2.042424in}{0.648674in}}{\pgfqpoint{2.053023in}{0.644284in}}{\pgfqpoint{2.064073in}{0.644284in}}%
\pgfpathclose%
\pgfusepath{stroke,fill}%
\end{pgfscope}%
\begin{pgfscope}%
\pgfpathrectangle{\pgfqpoint{0.772069in}{0.515123in}}{\pgfqpoint{1.937500in}{1.347500in}}%
\pgfusepath{clip}%
\pgfsetbuttcap%
\pgfsetroundjoin%
\definecolor{currentfill}{rgb}{1.000000,0.843137,0.000000}%
\pgfsetfillcolor{currentfill}%
\pgfsetlinewidth{1.003750pt}%
\definecolor{currentstroke}{rgb}{1.000000,0.843137,0.000000}%
\pgfsetstrokecolor{currentstroke}%
\pgfsetdash{}{0pt}%
\pgfpathmoveto{\pgfqpoint{2.248220in}{0.650335in}}%
\pgfpathcurveto{\pgfqpoint{2.259270in}{0.650335in}}{\pgfqpoint{2.269869in}{0.654726in}}{\pgfqpoint{2.277682in}{0.662539in}}%
\pgfpathcurveto{\pgfqpoint{2.285496in}{0.670353in}}{\pgfqpoint{2.289886in}{0.680952in}}{\pgfqpoint{2.289886in}{0.692002in}}%
\pgfpathcurveto{\pgfqpoint{2.289886in}{0.703052in}}{\pgfqpoint{2.285496in}{0.713651in}}{\pgfqpoint{2.277682in}{0.721465in}}%
\pgfpathcurveto{\pgfqpoint{2.269869in}{0.729278in}}{\pgfqpoint{2.259270in}{0.733669in}}{\pgfqpoint{2.248220in}{0.733669in}}%
\pgfpathcurveto{\pgfqpoint{2.237169in}{0.733669in}}{\pgfqpoint{2.226570in}{0.729278in}}{\pgfqpoint{2.218757in}{0.721465in}}%
\pgfpathcurveto{\pgfqpoint{2.210943in}{0.713651in}}{\pgfqpoint{2.206553in}{0.703052in}}{\pgfqpoint{2.206553in}{0.692002in}}%
\pgfpathcurveto{\pgfqpoint{2.206553in}{0.680952in}}{\pgfqpoint{2.210943in}{0.670353in}}{\pgfqpoint{2.218757in}{0.662539in}}%
\pgfpathcurveto{\pgfqpoint{2.226570in}{0.654726in}}{\pgfqpoint{2.237169in}{0.650335in}}{\pgfqpoint{2.248220in}{0.650335in}}%
\pgfpathclose%
\pgfusepath{stroke,fill}%
\end{pgfscope}%
\begin{pgfscope}%
\pgfpathrectangle{\pgfqpoint{0.772069in}{0.515123in}}{\pgfqpoint{1.937500in}{1.347500in}}%
\pgfusepath{clip}%
\pgfsetbuttcap%
\pgfsetroundjoin%
\definecolor{currentfill}{rgb}{1.000000,0.843137,0.000000}%
\pgfsetfillcolor{currentfill}%
\pgfsetlinewidth{1.003750pt}%
\definecolor{currentstroke}{rgb}{1.000000,0.843137,0.000000}%
\pgfsetstrokecolor{currentstroke}%
\pgfsetdash{}{0pt}%
\pgfpathmoveto{\pgfqpoint{2.409348in}{0.662322in}}%
\pgfpathcurveto{\pgfqpoint{2.420398in}{0.662322in}}{\pgfqpoint{2.430997in}{0.666712in}}{\pgfqpoint{2.438811in}{0.674526in}}%
\pgfpathcurveto{\pgfqpoint{2.446624in}{0.682339in}}{\pgfqpoint{2.451015in}{0.692938in}}{\pgfqpoint{2.451015in}{0.703989in}}%
\pgfpathcurveto{\pgfqpoint{2.451015in}{0.715039in}}{\pgfqpoint{2.446624in}{0.725638in}}{\pgfqpoint{2.438811in}{0.733451in}}%
\pgfpathcurveto{\pgfqpoint{2.430997in}{0.741265in}}{\pgfqpoint{2.420398in}{0.745655in}}{\pgfqpoint{2.409348in}{0.745655in}}%
\pgfpathcurveto{\pgfqpoint{2.398298in}{0.745655in}}{\pgfqpoint{2.387699in}{0.741265in}}{\pgfqpoint{2.379885in}{0.733451in}}%
\pgfpathcurveto{\pgfqpoint{2.372072in}{0.725638in}}{\pgfqpoint{2.367681in}{0.715039in}}{\pgfqpoint{2.367681in}{0.703989in}}%
\pgfpathcurveto{\pgfqpoint{2.367681in}{0.692938in}}{\pgfqpoint{2.372072in}{0.682339in}}{\pgfqpoint{2.379885in}{0.674526in}}%
\pgfpathcurveto{\pgfqpoint{2.387699in}{0.666712in}}{\pgfqpoint{2.398298in}{0.662322in}}{\pgfqpoint{2.409348in}{0.662322in}}%
\pgfpathclose%
\pgfusepath{stroke,fill}%
\end{pgfscope}%
\begin{pgfscope}%
\pgfpathrectangle{\pgfqpoint{0.772069in}{0.515123in}}{\pgfqpoint{1.937500in}{1.347500in}}%
\pgfusepath{clip}%
\pgfsetbuttcap%
\pgfsetroundjoin%
\definecolor{currentfill}{rgb}{1.000000,0.843137,0.000000}%
\pgfsetfillcolor{currentfill}%
\pgfsetlinewidth{1.003750pt}%
\definecolor{currentstroke}{rgb}{1.000000,0.843137,0.000000}%
\pgfsetstrokecolor{currentstroke}%
\pgfsetdash{}{0pt}%
\pgfpathmoveto{\pgfqpoint{2.593495in}{0.674658in}}%
\pgfpathcurveto{\pgfqpoint{2.604545in}{0.674658in}}{\pgfqpoint{2.615144in}{0.679048in}}{\pgfqpoint{2.622958in}{0.686862in}}%
\pgfpathcurveto{\pgfqpoint{2.630771in}{0.694675in}}{\pgfqpoint{2.635161in}{0.705274in}}{\pgfqpoint{2.635161in}{0.716324in}}%
\pgfpathcurveto{\pgfqpoint{2.635161in}{0.727374in}}{\pgfqpoint{2.630771in}{0.737974in}}{\pgfqpoint{2.622958in}{0.745787in}}%
\pgfpathcurveto{\pgfqpoint{2.615144in}{0.753601in}}{\pgfqpoint{2.604545in}{0.757991in}}{\pgfqpoint{2.593495in}{0.757991in}}%
\pgfpathcurveto{\pgfqpoint{2.582445in}{0.757991in}}{\pgfqpoint{2.571846in}{0.753601in}}{\pgfqpoint{2.564032in}{0.745787in}}%
\pgfpathcurveto{\pgfqpoint{2.556218in}{0.737974in}}{\pgfqpoint{2.551828in}{0.727374in}}{\pgfqpoint{2.551828in}{0.716324in}}%
\pgfpathcurveto{\pgfqpoint{2.551828in}{0.705274in}}{\pgfqpoint{2.556218in}{0.694675in}}{\pgfqpoint{2.564032in}{0.686862in}}%
\pgfpathcurveto{\pgfqpoint{2.571846in}{0.679048in}}{\pgfqpoint{2.582445in}{0.674658in}}{\pgfqpoint{2.593495in}{0.674658in}}%
\pgfpathclose%
\pgfusepath{stroke,fill}%
\end{pgfscope}%
\begin{pgfscope}%
\pgfpathrectangle{\pgfqpoint{0.772069in}{0.515123in}}{\pgfqpoint{1.937500in}{1.347500in}}%
\pgfusepath{clip}%
\pgfsetbuttcap%
\pgfsetroundjoin%
\definecolor{currentfill}{rgb}{0.196078,0.803922,0.196078}%
\pgfsetfillcolor{currentfill}%
\pgfsetlinewidth{1.003750pt}%
\definecolor{currentstroke}{rgb}{0.196078,0.803922,0.196078}%
\pgfsetstrokecolor{currentstroke}%
\pgfsetdash{}{0pt}%
\pgfpathmoveto{\pgfqpoint{1.005229in}{0.641491in}}%
\pgfpathcurveto{\pgfqpoint{1.016279in}{0.641491in}}{\pgfqpoint{1.026878in}{0.645881in}}{\pgfqpoint{1.034691in}{0.653695in}}%
\pgfpathcurveto{\pgfqpoint{1.042505in}{0.661508in}}{\pgfqpoint{1.046895in}{0.672107in}}{\pgfqpoint{1.046895in}{0.683157in}}%
\pgfpathcurveto{\pgfqpoint{1.046895in}{0.694208in}}{\pgfqpoint{1.042505in}{0.704807in}}{\pgfqpoint{1.034691in}{0.712620in}}%
\pgfpathcurveto{\pgfqpoint{1.026878in}{0.720434in}}{\pgfqpoint{1.016279in}{0.724824in}}{\pgfqpoint{1.005229in}{0.724824in}}%
\pgfpathcurveto{\pgfqpoint{0.994178in}{0.724824in}}{\pgfqpoint{0.983579in}{0.720434in}}{\pgfqpoint{0.975766in}{0.712620in}}%
\pgfpathcurveto{\pgfqpoint{0.967952in}{0.704807in}}{\pgfqpoint{0.963562in}{0.694208in}}{\pgfqpoint{0.963562in}{0.683157in}}%
\pgfpathcurveto{\pgfqpoint{0.963562in}{0.672107in}}{\pgfqpoint{0.967952in}{0.661508in}}{\pgfqpoint{0.975766in}{0.653695in}}%
\pgfpathcurveto{\pgfqpoint{0.983579in}{0.645881in}}{\pgfqpoint{0.994178in}{0.641491in}}{\pgfqpoint{1.005229in}{0.641491in}}%
\pgfpathclose%
\pgfusepath{stroke,fill}%
\end{pgfscope}%
\begin{pgfscope}%
\pgfpathrectangle{\pgfqpoint{0.772069in}{0.515123in}}{\pgfqpoint{1.937500in}{1.347500in}}%
\pgfusepath{clip}%
\pgfsetbuttcap%
\pgfsetroundjoin%
\definecolor{currentfill}{rgb}{0.196078,0.803922,0.196078}%
\pgfsetfillcolor{currentfill}%
\pgfsetlinewidth{1.003750pt}%
\definecolor{currentstroke}{rgb}{0.196078,0.803922,0.196078}%
\pgfsetstrokecolor{currentstroke}%
\pgfsetdash{}{0pt}%
\pgfpathmoveto{\pgfqpoint{1.189375in}{0.725630in}}%
\pgfpathcurveto{\pgfqpoint{1.200426in}{0.725630in}}{\pgfqpoint{1.211025in}{0.730020in}}{\pgfqpoint{1.218838in}{0.737834in}}%
\pgfpathcurveto{\pgfqpoint{1.226652in}{0.745647in}}{\pgfqpoint{1.231042in}{0.756246in}}{\pgfqpoint{1.231042in}{0.767297in}}%
\pgfpathcurveto{\pgfqpoint{1.231042in}{0.778347in}}{\pgfqpoint{1.226652in}{0.788946in}}{\pgfqpoint{1.218838in}{0.796759in}}%
\pgfpathcurveto{\pgfqpoint{1.211025in}{0.804573in}}{\pgfqpoint{1.200426in}{0.808963in}}{\pgfqpoint{1.189375in}{0.808963in}}%
\pgfpathcurveto{\pgfqpoint{1.178325in}{0.808963in}}{\pgfqpoint{1.167726in}{0.804573in}}{\pgfqpoint{1.159913in}{0.796759in}}%
\pgfpathcurveto{\pgfqpoint{1.152099in}{0.788946in}}{\pgfqpoint{1.147709in}{0.778347in}}{\pgfqpoint{1.147709in}{0.767297in}}%
\pgfpathcurveto{\pgfqpoint{1.147709in}{0.756246in}}{\pgfqpoint{1.152099in}{0.745647in}}{\pgfqpoint{1.159913in}{0.737834in}}%
\pgfpathcurveto{\pgfqpoint{1.167726in}{0.730020in}}{\pgfqpoint{1.178325in}{0.725630in}}{\pgfqpoint{1.189375in}{0.725630in}}%
\pgfpathclose%
\pgfusepath{stroke,fill}%
\end{pgfscope}%
\begin{pgfscope}%
\pgfpathrectangle{\pgfqpoint{0.772069in}{0.515123in}}{\pgfqpoint{1.937500in}{1.347500in}}%
\pgfusepath{clip}%
\pgfsetbuttcap%
\pgfsetroundjoin%
\definecolor{currentfill}{rgb}{0.196078,0.803922,0.196078}%
\pgfsetfillcolor{currentfill}%
\pgfsetlinewidth{1.003750pt}%
\definecolor{currentstroke}{rgb}{0.196078,0.803922,0.196078}%
\pgfsetstrokecolor{currentstroke}%
\pgfsetdash{}{0pt}%
\pgfpathmoveto{\pgfqpoint{1.350504in}{0.813842in}}%
\pgfpathcurveto{\pgfqpoint{1.361554in}{0.813842in}}{\pgfqpoint{1.372153in}{0.818233in}}{\pgfqpoint{1.379967in}{0.826046in}}%
\pgfpathcurveto{\pgfqpoint{1.387780in}{0.833860in}}{\pgfqpoint{1.392171in}{0.844459in}}{\pgfqpoint{1.392171in}{0.855509in}}%
\pgfpathcurveto{\pgfqpoint{1.392171in}{0.866559in}}{\pgfqpoint{1.387780in}{0.877158in}}{\pgfqpoint{1.379967in}{0.884972in}}%
\pgfpathcurveto{\pgfqpoint{1.372153in}{0.892785in}}{\pgfqpoint{1.361554in}{0.897176in}}{\pgfqpoint{1.350504in}{0.897176in}}%
\pgfpathcurveto{\pgfqpoint{1.339454in}{0.897176in}}{\pgfqpoint{1.328855in}{0.892785in}}{\pgfqpoint{1.321041in}{0.884972in}}%
\pgfpathcurveto{\pgfqpoint{1.313227in}{0.877158in}}{\pgfqpoint{1.308837in}{0.866559in}}{\pgfqpoint{1.308837in}{0.855509in}}%
\pgfpathcurveto{\pgfqpoint{1.308837in}{0.844459in}}{\pgfqpoint{1.313227in}{0.833860in}}{\pgfqpoint{1.321041in}{0.826046in}}%
\pgfpathcurveto{\pgfqpoint{1.328855in}{0.818233in}}{\pgfqpoint{1.339454in}{0.813842in}}{\pgfqpoint{1.350504in}{0.813842in}}%
\pgfpathclose%
\pgfusepath{stroke,fill}%
\end{pgfscope}%
\begin{pgfscope}%
\pgfpathrectangle{\pgfqpoint{0.772069in}{0.515123in}}{\pgfqpoint{1.937500in}{1.347500in}}%
\pgfusepath{clip}%
\pgfsetbuttcap%
\pgfsetroundjoin%
\definecolor{currentfill}{rgb}{0.196078,0.803922,0.196078}%
\pgfsetfillcolor{currentfill}%
\pgfsetlinewidth{1.003750pt}%
\definecolor{currentstroke}{rgb}{0.196078,0.803922,0.196078}%
\pgfsetstrokecolor{currentstroke}%
\pgfsetdash{}{0pt}%
\pgfpathmoveto{\pgfqpoint{1.534651in}{0.903451in}}%
\pgfpathcurveto{\pgfqpoint{1.545701in}{0.903451in}}{\pgfqpoint{1.556300in}{0.907841in}}{\pgfqpoint{1.564113in}{0.915655in}}%
\pgfpathcurveto{\pgfqpoint{1.571927in}{0.923469in}}{\pgfqpoint{1.576317in}{0.934068in}}{\pgfqpoint{1.576317in}{0.945118in}}%
\pgfpathcurveto{\pgfqpoint{1.576317in}{0.956168in}}{\pgfqpoint{1.571927in}{0.966767in}}{\pgfqpoint{1.564113in}{0.974580in}}%
\pgfpathcurveto{\pgfqpoint{1.556300in}{0.982394in}}{\pgfqpoint{1.545701in}{0.986784in}}{\pgfqpoint{1.534651in}{0.986784in}}%
\pgfpathcurveto{\pgfqpoint{1.523601in}{0.986784in}}{\pgfqpoint{1.513002in}{0.982394in}}{\pgfqpoint{1.505188in}{0.974580in}}%
\pgfpathcurveto{\pgfqpoint{1.497374in}{0.966767in}}{\pgfqpoint{1.492984in}{0.956168in}}{\pgfqpoint{1.492984in}{0.945118in}}%
\pgfpathcurveto{\pgfqpoint{1.492984in}{0.934068in}}{\pgfqpoint{1.497374in}{0.923469in}}{\pgfqpoint{1.505188in}{0.915655in}}%
\pgfpathcurveto{\pgfqpoint{1.513002in}{0.907841in}}{\pgfqpoint{1.523601in}{0.903451in}}{\pgfqpoint{1.534651in}{0.903451in}}%
\pgfpathclose%
\pgfusepath{stroke,fill}%
\end{pgfscope}%
\begin{pgfscope}%
\pgfpathrectangle{\pgfqpoint{0.772069in}{0.515123in}}{\pgfqpoint{1.937500in}{1.347500in}}%
\pgfusepath{clip}%
\pgfsetbuttcap%
\pgfsetroundjoin%
\definecolor{currentfill}{rgb}{0.196078,0.803922,0.196078}%
\pgfsetfillcolor{currentfill}%
\pgfsetlinewidth{1.003750pt}%
\definecolor{currentstroke}{rgb}{0.196078,0.803922,0.196078}%
\pgfsetstrokecolor{currentstroke}%
\pgfsetdash{}{0pt}%
\pgfpathmoveto{\pgfqpoint{1.718797in}{0.991896in}}%
\pgfpathcurveto{\pgfqpoint{1.729848in}{0.991896in}}{\pgfqpoint{1.740447in}{0.996286in}}{\pgfqpoint{1.748260in}{1.004100in}}%
\pgfpathcurveto{\pgfqpoint{1.756074in}{1.011914in}}{\pgfqpoint{1.760464in}{1.022513in}}{\pgfqpoint{1.760464in}{1.033563in}}%
\pgfpathcurveto{\pgfqpoint{1.760464in}{1.044613in}}{\pgfqpoint{1.756074in}{1.055212in}}{\pgfqpoint{1.748260in}{1.063026in}}%
\pgfpathcurveto{\pgfqpoint{1.740447in}{1.070839in}}{\pgfqpoint{1.729848in}{1.075229in}}{\pgfqpoint{1.718797in}{1.075229in}}%
\pgfpathcurveto{\pgfqpoint{1.707747in}{1.075229in}}{\pgfqpoint{1.697148in}{1.070839in}}{\pgfqpoint{1.689335in}{1.063026in}}%
\pgfpathcurveto{\pgfqpoint{1.681521in}{1.055212in}}{\pgfqpoint{1.677131in}{1.044613in}}{\pgfqpoint{1.677131in}{1.033563in}}%
\pgfpathcurveto{\pgfqpoint{1.677131in}{1.022513in}}{\pgfqpoint{1.681521in}{1.011914in}}{\pgfqpoint{1.689335in}{1.004100in}}%
\pgfpathcurveto{\pgfqpoint{1.697148in}{0.996286in}}{\pgfqpoint{1.707747in}{0.991896in}}{\pgfqpoint{1.718797in}{0.991896in}}%
\pgfpathclose%
\pgfusepath{stroke,fill}%
\end{pgfscope}%
\begin{pgfscope}%
\pgfpathrectangle{\pgfqpoint{0.772069in}{0.515123in}}{\pgfqpoint{1.937500in}{1.347500in}}%
\pgfusepath{clip}%
\pgfsetbuttcap%
\pgfsetroundjoin%
\definecolor{currentfill}{rgb}{0.196078,0.803922,0.196078}%
\pgfsetfillcolor{currentfill}%
\pgfsetlinewidth{1.003750pt}%
\definecolor{currentstroke}{rgb}{0.196078,0.803922,0.196078}%
\pgfsetstrokecolor{currentstroke}%
\pgfsetdash{}{0pt}%
\pgfpathmoveto{\pgfqpoint{1.879926in}{1.075686in}}%
\pgfpathcurveto{\pgfqpoint{1.890976in}{1.075686in}}{\pgfqpoint{1.901575in}{1.080076in}}{\pgfqpoint{1.909389in}{1.087890in}}%
\pgfpathcurveto{\pgfqpoint{1.917202in}{1.095704in}}{\pgfqpoint{1.921593in}{1.106303in}}{\pgfqpoint{1.921593in}{1.117353in}}%
\pgfpathcurveto{\pgfqpoint{1.921593in}{1.128403in}}{\pgfqpoint{1.917202in}{1.139002in}}{\pgfqpoint{1.909389in}{1.146816in}}%
\pgfpathcurveto{\pgfqpoint{1.901575in}{1.154629in}}{\pgfqpoint{1.890976in}{1.159019in}}{\pgfqpoint{1.879926in}{1.159019in}}%
\pgfpathcurveto{\pgfqpoint{1.868876in}{1.159019in}}{\pgfqpoint{1.858277in}{1.154629in}}{\pgfqpoint{1.850463in}{1.146816in}}%
\pgfpathcurveto{\pgfqpoint{1.842650in}{1.139002in}}{\pgfqpoint{1.838259in}{1.128403in}}{\pgfqpoint{1.838259in}{1.117353in}}%
\pgfpathcurveto{\pgfqpoint{1.838259in}{1.106303in}}{\pgfqpoint{1.842650in}{1.095704in}}{\pgfqpoint{1.850463in}{1.087890in}}%
\pgfpathcurveto{\pgfqpoint{1.858277in}{1.080076in}}{\pgfqpoint{1.868876in}{1.075686in}}{\pgfqpoint{1.879926in}{1.075686in}}%
\pgfpathclose%
\pgfusepath{stroke,fill}%
\end{pgfscope}%
\begin{pgfscope}%
\pgfpathrectangle{\pgfqpoint{0.772069in}{0.515123in}}{\pgfqpoint{1.937500in}{1.347500in}}%
\pgfusepath{clip}%
\pgfsetbuttcap%
\pgfsetroundjoin%
\definecolor{currentfill}{rgb}{0.196078,0.803922,0.196078}%
\pgfsetfillcolor{currentfill}%
\pgfsetlinewidth{1.003750pt}%
\definecolor{currentstroke}{rgb}{0.196078,0.803922,0.196078}%
\pgfsetstrokecolor{currentstroke}%
\pgfsetdash{}{0pt}%
\pgfpathmoveto{\pgfqpoint{2.064073in}{1.166459in}}%
\pgfpathcurveto{\pgfqpoint{2.075123in}{1.166459in}}{\pgfqpoint{2.085722in}{1.170849in}}{\pgfqpoint{2.093536in}{1.178663in}}%
\pgfpathcurveto{\pgfqpoint{2.101349in}{1.186476in}}{\pgfqpoint{2.105739in}{1.197075in}}{\pgfqpoint{2.105739in}{1.208125in}}%
\pgfpathcurveto{\pgfqpoint{2.105739in}{1.219175in}}{\pgfqpoint{2.101349in}{1.229774in}}{\pgfqpoint{2.093536in}{1.237588in}}%
\pgfpathcurveto{\pgfqpoint{2.085722in}{1.245402in}}{\pgfqpoint{2.075123in}{1.249792in}}{\pgfqpoint{2.064073in}{1.249792in}}%
\pgfpathcurveto{\pgfqpoint{2.053023in}{1.249792in}}{\pgfqpoint{2.042424in}{1.245402in}}{\pgfqpoint{2.034610in}{1.237588in}}%
\pgfpathcurveto{\pgfqpoint{2.026796in}{1.229774in}}{\pgfqpoint{2.022406in}{1.219175in}}{\pgfqpoint{2.022406in}{1.208125in}}%
\pgfpathcurveto{\pgfqpoint{2.022406in}{1.197075in}}{\pgfqpoint{2.026796in}{1.186476in}}{\pgfqpoint{2.034610in}{1.178663in}}%
\pgfpathcurveto{\pgfqpoint{2.042424in}{1.170849in}}{\pgfqpoint{2.053023in}{1.166459in}}{\pgfqpoint{2.064073in}{1.166459in}}%
\pgfpathclose%
\pgfusepath{stroke,fill}%
\end{pgfscope}%
\begin{pgfscope}%
\pgfpathrectangle{\pgfqpoint{0.772069in}{0.515123in}}{\pgfqpoint{1.937500in}{1.347500in}}%
\pgfusepath{clip}%
\pgfsetbuttcap%
\pgfsetroundjoin%
\definecolor{currentfill}{rgb}{0.196078,0.803922,0.196078}%
\pgfsetfillcolor{currentfill}%
\pgfsetlinewidth{1.003750pt}%
\definecolor{currentstroke}{rgb}{0.196078,0.803922,0.196078}%
\pgfsetstrokecolor{currentstroke}%
\pgfsetdash{}{0pt}%
\pgfpathmoveto{\pgfqpoint{2.248220in}{1.253740in}}%
\pgfpathcurveto{\pgfqpoint{2.259270in}{1.253740in}}{\pgfqpoint{2.269869in}{1.258130in}}{\pgfqpoint{2.277682in}{1.265944in}}%
\pgfpathcurveto{\pgfqpoint{2.285496in}{1.273757in}}{\pgfqpoint{2.289886in}{1.284356in}}{\pgfqpoint{2.289886in}{1.295407in}}%
\pgfpathcurveto{\pgfqpoint{2.289886in}{1.306457in}}{\pgfqpoint{2.285496in}{1.317056in}}{\pgfqpoint{2.277682in}{1.324869in}}%
\pgfpathcurveto{\pgfqpoint{2.269869in}{1.332683in}}{\pgfqpoint{2.259270in}{1.337073in}}{\pgfqpoint{2.248220in}{1.337073in}}%
\pgfpathcurveto{\pgfqpoint{2.237169in}{1.337073in}}{\pgfqpoint{2.226570in}{1.332683in}}{\pgfqpoint{2.218757in}{1.324869in}}%
\pgfpathcurveto{\pgfqpoint{2.210943in}{1.317056in}}{\pgfqpoint{2.206553in}{1.306457in}}{\pgfqpoint{2.206553in}{1.295407in}}%
\pgfpathcurveto{\pgfqpoint{2.206553in}{1.284356in}}{\pgfqpoint{2.210943in}{1.273757in}}{\pgfqpoint{2.218757in}{1.265944in}}%
\pgfpathcurveto{\pgfqpoint{2.226570in}{1.258130in}}{\pgfqpoint{2.237169in}{1.253740in}}{\pgfqpoint{2.248220in}{1.253740in}}%
\pgfpathclose%
\pgfusepath{stroke,fill}%
\end{pgfscope}%
\begin{pgfscope}%
\pgfpathrectangle{\pgfqpoint{0.772069in}{0.515123in}}{\pgfqpoint{1.937500in}{1.347500in}}%
\pgfusepath{clip}%
\pgfsetbuttcap%
\pgfsetroundjoin%
\definecolor{currentfill}{rgb}{0.196078,0.803922,0.196078}%
\pgfsetfillcolor{currentfill}%
\pgfsetlinewidth{1.003750pt}%
\definecolor{currentstroke}{rgb}{0.196078,0.803922,0.196078}%
\pgfsetstrokecolor{currentstroke}%
\pgfsetdash{}{0pt}%
\pgfpathmoveto{\pgfqpoint{2.409348in}{1.339857in}}%
\pgfpathcurveto{\pgfqpoint{2.420398in}{1.339857in}}{\pgfqpoint{2.430997in}{1.344248in}}{\pgfqpoint{2.438811in}{1.352061in}}%
\pgfpathcurveto{\pgfqpoint{2.446624in}{1.359875in}}{\pgfqpoint{2.451015in}{1.370474in}}{\pgfqpoint{2.451015in}{1.381524in}}%
\pgfpathcurveto{\pgfqpoint{2.451015in}{1.392574in}}{\pgfqpoint{2.446624in}{1.403173in}}{\pgfqpoint{2.438811in}{1.410987in}}%
\pgfpathcurveto{\pgfqpoint{2.430997in}{1.418801in}}{\pgfqpoint{2.420398in}{1.423191in}}{\pgfqpoint{2.409348in}{1.423191in}}%
\pgfpathcurveto{\pgfqpoint{2.398298in}{1.423191in}}{\pgfqpoint{2.387699in}{1.418801in}}{\pgfqpoint{2.379885in}{1.410987in}}%
\pgfpathcurveto{\pgfqpoint{2.372072in}{1.403173in}}{\pgfqpoint{2.367681in}{1.392574in}}{\pgfqpoint{2.367681in}{1.381524in}}%
\pgfpathcurveto{\pgfqpoint{2.367681in}{1.370474in}}{\pgfqpoint{2.372072in}{1.359875in}}{\pgfqpoint{2.379885in}{1.352061in}}%
\pgfpathcurveto{\pgfqpoint{2.387699in}{1.344248in}}{\pgfqpoint{2.398298in}{1.339857in}}{\pgfqpoint{2.409348in}{1.339857in}}%
\pgfpathclose%
\pgfusepath{stroke,fill}%
\end{pgfscope}%
\begin{pgfscope}%
\pgfpathrectangle{\pgfqpoint{0.772069in}{0.515123in}}{\pgfqpoint{1.937500in}{1.347500in}}%
\pgfusepath{clip}%
\pgfsetbuttcap%
\pgfsetroundjoin%
\definecolor{currentfill}{rgb}{0.196078,0.803922,0.196078}%
\pgfsetfillcolor{currentfill}%
\pgfsetlinewidth{1.003750pt}%
\definecolor{currentstroke}{rgb}{0.196078,0.803922,0.196078}%
\pgfsetstrokecolor{currentstroke}%
\pgfsetdash{}{0pt}%
\pgfpathmoveto{\pgfqpoint{2.593495in}{1.428302in}}%
\pgfpathcurveto{\pgfqpoint{2.604545in}{1.428302in}}{\pgfqpoint{2.615144in}{1.432693in}}{\pgfqpoint{2.622958in}{1.440506in}}%
\pgfpathcurveto{\pgfqpoint{2.630771in}{1.448320in}}{\pgfqpoint{2.635161in}{1.458919in}}{\pgfqpoint{2.635161in}{1.469969in}}%
\pgfpathcurveto{\pgfqpoint{2.635161in}{1.481019in}}{\pgfqpoint{2.630771in}{1.491618in}}{\pgfqpoint{2.622958in}{1.499432in}}%
\pgfpathcurveto{\pgfqpoint{2.615144in}{1.507246in}}{\pgfqpoint{2.604545in}{1.511636in}}{\pgfqpoint{2.593495in}{1.511636in}}%
\pgfpathcurveto{\pgfqpoint{2.582445in}{1.511636in}}{\pgfqpoint{2.571846in}{1.507246in}}{\pgfqpoint{2.564032in}{1.499432in}}%
\pgfpathcurveto{\pgfqpoint{2.556218in}{1.491618in}}{\pgfqpoint{2.551828in}{1.481019in}}{\pgfqpoint{2.551828in}{1.469969in}}%
\pgfpathcurveto{\pgfqpoint{2.551828in}{1.458919in}}{\pgfqpoint{2.556218in}{1.448320in}}{\pgfqpoint{2.564032in}{1.440506in}}%
\pgfpathcurveto{\pgfqpoint{2.571846in}{1.432693in}}{\pgfqpoint{2.582445in}{1.428302in}}{\pgfqpoint{2.593495in}{1.428302in}}%
\pgfpathclose%
\pgfusepath{stroke,fill}%
\end{pgfscope}%
\begin{pgfscope}%
\pgfpathrectangle{\pgfqpoint{0.772069in}{0.515123in}}{\pgfqpoint{1.937500in}{1.347500in}}%
\pgfusepath{clip}%
\pgfsetbuttcap%
\pgfsetroundjoin%
\definecolor{currentfill}{rgb}{0.117647,0.564706,1.000000}%
\pgfsetfillcolor{currentfill}%
\pgfsetlinewidth{1.003750pt}%
\definecolor{currentstroke}{rgb}{0.117647,0.564706,1.000000}%
\pgfsetstrokecolor{currentstroke}%
\pgfsetdash{}{0pt}%
\pgfpathmoveto{\pgfqpoint{0.936174in}{0.672097in}}%
\pgfpathcurveto{\pgfqpoint{0.947224in}{0.672097in}}{\pgfqpoint{0.957823in}{0.676488in}}{\pgfqpoint{0.965636in}{0.684301in}}%
\pgfpathcurveto{\pgfqpoint{0.973450in}{0.692115in}}{\pgfqpoint{0.977840in}{0.702714in}}{\pgfqpoint{0.977840in}{0.713764in}}%
\pgfpathcurveto{\pgfqpoint{0.977840in}{0.724814in}}{\pgfqpoint{0.973450in}{0.735413in}}{\pgfqpoint{0.965636in}{0.743227in}}%
\pgfpathcurveto{\pgfqpoint{0.957823in}{0.751041in}}{\pgfqpoint{0.947224in}{0.755431in}}{\pgfqpoint{0.936174in}{0.755431in}}%
\pgfpathcurveto{\pgfqpoint{0.925123in}{0.755431in}}{\pgfqpoint{0.914524in}{0.751041in}}{\pgfqpoint{0.906711in}{0.743227in}}%
\pgfpathcurveto{\pgfqpoint{0.898897in}{0.735413in}}{\pgfqpoint{0.894507in}{0.724814in}}{\pgfqpoint{0.894507in}{0.713764in}}%
\pgfpathcurveto{\pgfqpoint{0.894507in}{0.702714in}}{\pgfqpoint{0.898897in}{0.692115in}}{\pgfqpoint{0.906711in}{0.684301in}}%
\pgfpathcurveto{\pgfqpoint{0.914524in}{0.676488in}}{\pgfqpoint{0.925123in}{0.672097in}}{\pgfqpoint{0.936174in}{0.672097in}}%
\pgfpathclose%
\pgfusepath{stroke,fill}%
\end{pgfscope}%
\begin{pgfscope}%
\pgfpathrectangle{\pgfqpoint{0.772069in}{0.515123in}}{\pgfqpoint{1.937500in}{1.347500in}}%
\pgfusepath{clip}%
\pgfsetbuttcap%
\pgfsetroundjoin%
\definecolor{currentfill}{rgb}{0.117647,0.564706,1.000000}%
\pgfsetfillcolor{currentfill}%
\pgfsetlinewidth{1.003750pt}%
\definecolor{currentstroke}{rgb}{0.117647,0.564706,1.000000}%
\pgfsetstrokecolor{currentstroke}%
\pgfsetdash{}{0pt}%
\pgfpathmoveto{\pgfqpoint{1.051265in}{0.784748in}}%
\pgfpathcurveto{\pgfqpoint{1.062315in}{0.784748in}}{\pgfqpoint{1.072914in}{0.789139in}}{\pgfqpoint{1.080728in}{0.796952in}}%
\pgfpathcurveto{\pgfqpoint{1.088542in}{0.804766in}}{\pgfqpoint{1.092932in}{0.815365in}}{\pgfqpoint{1.092932in}{0.826415in}}%
\pgfpathcurveto{\pgfqpoint{1.092932in}{0.837465in}}{\pgfqpoint{1.088542in}{0.848064in}}{\pgfqpoint{1.080728in}{0.855878in}}%
\pgfpathcurveto{\pgfqpoint{1.072914in}{0.863692in}}{\pgfqpoint{1.062315in}{0.868082in}}{\pgfqpoint{1.051265in}{0.868082in}}%
\pgfpathcurveto{\pgfqpoint{1.040215in}{0.868082in}}{\pgfqpoint{1.029616in}{0.863692in}}{\pgfqpoint{1.021803in}{0.855878in}}%
\pgfpathcurveto{\pgfqpoint{1.013989in}{0.848064in}}{\pgfqpoint{1.009599in}{0.837465in}}{\pgfqpoint{1.009599in}{0.826415in}}%
\pgfpathcurveto{\pgfqpoint{1.009599in}{0.815365in}}{\pgfqpoint{1.013989in}{0.804766in}}{\pgfqpoint{1.021803in}{0.796952in}}%
\pgfpathcurveto{\pgfqpoint{1.029616in}{0.789139in}}{\pgfqpoint{1.040215in}{0.784748in}}{\pgfqpoint{1.051265in}{0.784748in}}%
\pgfpathclose%
\pgfusepath{stroke,fill}%
\end{pgfscope}%
\begin{pgfscope}%
\pgfpathrectangle{\pgfqpoint{0.772069in}{0.515123in}}{\pgfqpoint{1.937500in}{1.347500in}}%
\pgfusepath{clip}%
\pgfsetbuttcap%
\pgfsetroundjoin%
\definecolor{currentfill}{rgb}{0.117647,0.564706,1.000000}%
\pgfsetfillcolor{currentfill}%
\pgfsetlinewidth{1.003750pt}%
\definecolor{currentstroke}{rgb}{0.117647,0.564706,1.000000}%
\pgfsetstrokecolor{currentstroke}%
\pgfsetdash{}{0pt}%
\pgfpathmoveto{\pgfqpoint{1.166357in}{0.905779in}}%
\pgfpathcurveto{\pgfqpoint{1.177407in}{0.905779in}}{\pgfqpoint{1.188006in}{0.910169in}}{\pgfqpoint{1.195820in}{0.917982in}}%
\pgfpathcurveto{\pgfqpoint{1.203633in}{0.925796in}}{\pgfqpoint{1.208024in}{0.936395in}}{\pgfqpoint{1.208024in}{0.947445in}}%
\pgfpathcurveto{\pgfqpoint{1.208024in}{0.958495in}}{\pgfqpoint{1.203633in}{0.969094in}}{\pgfqpoint{1.195820in}{0.976908in}}%
\pgfpathcurveto{\pgfqpoint{1.188006in}{0.984722in}}{\pgfqpoint{1.177407in}{0.989112in}}{\pgfqpoint{1.166357in}{0.989112in}}%
\pgfpathcurveto{\pgfqpoint{1.155307in}{0.989112in}}{\pgfqpoint{1.144708in}{0.984722in}}{\pgfqpoint{1.136894in}{0.976908in}}%
\pgfpathcurveto{\pgfqpoint{1.129081in}{0.969094in}}{\pgfqpoint{1.124690in}{0.958495in}}{\pgfqpoint{1.124690in}{0.947445in}}%
\pgfpathcurveto{\pgfqpoint{1.124690in}{0.936395in}}{\pgfqpoint{1.129081in}{0.925796in}}{\pgfqpoint{1.136894in}{0.917982in}}%
\pgfpathcurveto{\pgfqpoint{1.144708in}{0.910169in}}{\pgfqpoint{1.155307in}{0.905779in}}{\pgfqpoint{1.166357in}{0.905779in}}%
\pgfpathclose%
\pgfusepath{stroke,fill}%
\end{pgfscope}%
\begin{pgfscope}%
\pgfpathrectangle{\pgfqpoint{0.772069in}{0.515123in}}{\pgfqpoint{1.937500in}{1.347500in}}%
\pgfusepath{clip}%
\pgfsetbuttcap%
\pgfsetroundjoin%
\definecolor{currentfill}{rgb}{0.117647,0.564706,1.000000}%
\pgfsetfillcolor{currentfill}%
\pgfsetlinewidth{1.003750pt}%
\definecolor{currentstroke}{rgb}{0.117647,0.564706,1.000000}%
\pgfsetstrokecolor{currentstroke}%
\pgfsetdash{}{0pt}%
\pgfpathmoveto{\pgfqpoint{1.281449in}{1.024481in}}%
\pgfpathcurveto{\pgfqpoint{1.292499in}{1.024481in}}{\pgfqpoint{1.303098in}{1.028871in}}{\pgfqpoint{1.310912in}{1.036685in}}%
\pgfpathcurveto{\pgfqpoint{1.318725in}{1.044499in}}{\pgfqpoint{1.323115in}{1.055098in}}{\pgfqpoint{1.323115in}{1.066148in}}%
\pgfpathcurveto{\pgfqpoint{1.323115in}{1.077198in}}{\pgfqpoint{1.318725in}{1.087797in}}{\pgfqpoint{1.310912in}{1.095611in}}%
\pgfpathcurveto{\pgfqpoint{1.303098in}{1.103424in}}{\pgfqpoint{1.292499in}{1.107814in}}{\pgfqpoint{1.281449in}{1.107814in}}%
\pgfpathcurveto{\pgfqpoint{1.270399in}{1.107814in}}{\pgfqpoint{1.259800in}{1.103424in}}{\pgfqpoint{1.251986in}{1.095611in}}%
\pgfpathcurveto{\pgfqpoint{1.244172in}{1.087797in}}{\pgfqpoint{1.239782in}{1.077198in}}{\pgfqpoint{1.239782in}{1.066148in}}%
\pgfpathcurveto{\pgfqpoint{1.239782in}{1.055098in}}{\pgfqpoint{1.244172in}{1.044499in}}{\pgfqpoint{1.251986in}{1.036685in}}%
\pgfpathcurveto{\pgfqpoint{1.259800in}{1.028871in}}{\pgfqpoint{1.270399in}{1.024481in}}{\pgfqpoint{1.281449in}{1.024481in}}%
\pgfpathclose%
\pgfusepath{stroke,fill}%
\end{pgfscope}%
\begin{pgfscope}%
\pgfpathrectangle{\pgfqpoint{0.772069in}{0.515123in}}{\pgfqpoint{1.937500in}{1.347500in}}%
\pgfusepath{clip}%
\pgfsetbuttcap%
\pgfsetroundjoin%
\definecolor{currentfill}{rgb}{0.117647,0.564706,1.000000}%
\pgfsetfillcolor{currentfill}%
\pgfsetlinewidth{1.003750pt}%
\definecolor{currentstroke}{rgb}{0.117647,0.564706,1.000000}%
\pgfsetstrokecolor{currentstroke}%
\pgfsetdash{}{0pt}%
\pgfpathmoveto{\pgfqpoint{1.396541in}{1.143184in}}%
\pgfpathcurveto{\pgfqpoint{1.407591in}{1.143184in}}{\pgfqpoint{1.418190in}{1.147574in}}{\pgfqpoint{1.426003in}{1.155387in}}%
\pgfpathcurveto{\pgfqpoint{1.433817in}{1.163201in}}{\pgfqpoint{1.438207in}{1.173800in}}{\pgfqpoint{1.438207in}{1.184850in}}%
\pgfpathcurveto{\pgfqpoint{1.438207in}{1.195900in}}{\pgfqpoint{1.433817in}{1.206499in}}{\pgfqpoint{1.426003in}{1.214313in}}%
\pgfpathcurveto{\pgfqpoint{1.418190in}{1.222127in}}{\pgfqpoint{1.407591in}{1.226517in}}{\pgfqpoint{1.396541in}{1.226517in}}%
\pgfpathcurveto{\pgfqpoint{1.385490in}{1.226517in}}{\pgfqpoint{1.374891in}{1.222127in}}{\pgfqpoint{1.367078in}{1.214313in}}%
\pgfpathcurveto{\pgfqpoint{1.359264in}{1.206499in}}{\pgfqpoint{1.354874in}{1.195900in}}{\pgfqpoint{1.354874in}{1.184850in}}%
\pgfpathcurveto{\pgfqpoint{1.354874in}{1.173800in}}{\pgfqpoint{1.359264in}{1.163201in}}{\pgfqpoint{1.367078in}{1.155387in}}%
\pgfpathcurveto{\pgfqpoint{1.374891in}{1.147574in}}{\pgfqpoint{1.385490in}{1.143184in}}{\pgfqpoint{1.396541in}{1.143184in}}%
\pgfpathclose%
\pgfusepath{stroke,fill}%
\end{pgfscope}%
\begin{pgfscope}%
\pgfpathrectangle{\pgfqpoint{0.772069in}{0.515123in}}{\pgfqpoint{1.937500in}{1.347500in}}%
\pgfusepath{clip}%
\pgfsetbuttcap%
\pgfsetroundjoin%
\definecolor{currentfill}{rgb}{0.117647,0.564706,1.000000}%
\pgfsetfillcolor{currentfill}%
\pgfsetlinewidth{1.003750pt}%
\definecolor{currentstroke}{rgb}{0.117647,0.564706,1.000000}%
\pgfsetstrokecolor{currentstroke}%
\pgfsetdash{}{0pt}%
\pgfpathmoveto{\pgfqpoint{1.511632in}{1.256067in}}%
\pgfpathcurveto{\pgfqpoint{1.522682in}{1.256067in}}{\pgfqpoint{1.533281in}{1.260458in}}{\pgfqpoint{1.541095in}{1.268271in}}%
\pgfpathcurveto{\pgfqpoint{1.548909in}{1.276085in}}{\pgfqpoint{1.553299in}{1.286684in}}{\pgfqpoint{1.553299in}{1.297734in}}%
\pgfpathcurveto{\pgfqpoint{1.553299in}{1.308784in}}{\pgfqpoint{1.548909in}{1.319383in}}{\pgfqpoint{1.541095in}{1.327197in}}%
\pgfpathcurveto{\pgfqpoint{1.533281in}{1.335010in}}{\pgfqpoint{1.522682in}{1.339401in}}{\pgfqpoint{1.511632in}{1.339401in}}%
\pgfpathcurveto{\pgfqpoint{1.500582in}{1.339401in}}{\pgfqpoint{1.489983in}{1.335010in}}{\pgfqpoint{1.482170in}{1.327197in}}%
\pgfpathcurveto{\pgfqpoint{1.474356in}{1.319383in}}{\pgfqpoint{1.469966in}{1.308784in}}{\pgfqpoint{1.469966in}{1.297734in}}%
\pgfpathcurveto{\pgfqpoint{1.469966in}{1.286684in}}{\pgfqpoint{1.474356in}{1.276085in}}{\pgfqpoint{1.482170in}{1.268271in}}%
\pgfpathcurveto{\pgfqpoint{1.489983in}{1.260458in}}{\pgfqpoint{1.500582in}{1.256067in}}{\pgfqpoint{1.511632in}{1.256067in}}%
\pgfpathclose%
\pgfusepath{stroke,fill}%
\end{pgfscope}%
\begin{pgfscope}%
\pgfpathrectangle{\pgfqpoint{0.772069in}{0.515123in}}{\pgfqpoint{1.937500in}{1.347500in}}%
\pgfusepath{clip}%
\pgfsetbuttcap%
\pgfsetroundjoin%
\definecolor{currentfill}{rgb}{0.117647,0.564706,1.000000}%
\pgfsetfillcolor{currentfill}%
\pgfsetlinewidth{1.003750pt}%
\definecolor{currentstroke}{rgb}{0.117647,0.564706,1.000000}%
\pgfsetstrokecolor{currentstroke}%
\pgfsetdash{}{0pt}%
\pgfpathmoveto{\pgfqpoint{1.626724in}{1.378261in}}%
\pgfpathcurveto{\pgfqpoint{1.637774in}{1.378261in}}{\pgfqpoint{1.648373in}{1.382651in}}{\pgfqpoint{1.656187in}{1.390465in}}%
\pgfpathcurveto{\pgfqpoint{1.664000in}{1.398279in}}{\pgfqpoint{1.668391in}{1.408878in}}{\pgfqpoint{1.668391in}{1.419928in}}%
\pgfpathcurveto{\pgfqpoint{1.668391in}{1.430978in}}{\pgfqpoint{1.664000in}{1.441577in}}{\pgfqpoint{1.656187in}{1.449391in}}%
\pgfpathcurveto{\pgfqpoint{1.648373in}{1.457204in}}{\pgfqpoint{1.637774in}{1.461595in}}{\pgfqpoint{1.626724in}{1.461595in}}%
\pgfpathcurveto{\pgfqpoint{1.615674in}{1.461595in}}{\pgfqpoint{1.605075in}{1.457204in}}{\pgfqpoint{1.597261in}{1.449391in}}%
\pgfpathcurveto{\pgfqpoint{1.589448in}{1.441577in}}{\pgfqpoint{1.585057in}{1.430978in}}{\pgfqpoint{1.585057in}{1.419928in}}%
\pgfpathcurveto{\pgfqpoint{1.585057in}{1.408878in}}{\pgfqpoint{1.589448in}{1.398279in}}{\pgfqpoint{1.597261in}{1.390465in}}%
\pgfpathcurveto{\pgfqpoint{1.605075in}{1.382651in}}{\pgfqpoint{1.615674in}{1.378261in}}{\pgfqpoint{1.626724in}{1.378261in}}%
\pgfpathclose%
\pgfusepath{stroke,fill}%
\end{pgfscope}%
\begin{pgfscope}%
\pgfpathrectangle{\pgfqpoint{0.772069in}{0.515123in}}{\pgfqpoint{1.937500in}{1.347500in}}%
\pgfusepath{clip}%
\pgfsetbuttcap%
\pgfsetroundjoin%
\definecolor{currentfill}{rgb}{0.117647,0.564706,1.000000}%
\pgfsetfillcolor{currentfill}%
\pgfsetlinewidth{1.003750pt}%
\definecolor{currentstroke}{rgb}{0.117647,0.564706,1.000000}%
\pgfsetstrokecolor{currentstroke}%
\pgfsetdash{}{0pt}%
\pgfpathmoveto{\pgfqpoint{1.741816in}{1.495800in}}%
\pgfpathcurveto{\pgfqpoint{1.752866in}{1.495800in}}{\pgfqpoint{1.763465in}{1.500190in}}{\pgfqpoint{1.771279in}{1.508004in}}%
\pgfpathcurveto{\pgfqpoint{1.779092in}{1.515817in}}{\pgfqpoint{1.783482in}{1.526417in}}{\pgfqpoint{1.783482in}{1.537467in}}%
\pgfpathcurveto{\pgfqpoint{1.783482in}{1.548517in}}{\pgfqpoint{1.779092in}{1.559116in}}{\pgfqpoint{1.771279in}{1.566929in}}%
\pgfpathcurveto{\pgfqpoint{1.763465in}{1.574743in}}{\pgfqpoint{1.752866in}{1.579133in}}{\pgfqpoint{1.741816in}{1.579133in}}%
\pgfpathcurveto{\pgfqpoint{1.730766in}{1.579133in}}{\pgfqpoint{1.720167in}{1.574743in}}{\pgfqpoint{1.712353in}{1.566929in}}%
\pgfpathcurveto{\pgfqpoint{1.704539in}{1.559116in}}{\pgfqpoint{1.700149in}{1.548517in}}{\pgfqpoint{1.700149in}{1.537467in}}%
\pgfpathcurveto{\pgfqpoint{1.700149in}{1.526417in}}{\pgfqpoint{1.704539in}{1.515817in}}{\pgfqpoint{1.712353in}{1.508004in}}%
\pgfpathcurveto{\pgfqpoint{1.720167in}{1.500190in}}{\pgfqpoint{1.730766in}{1.495800in}}{\pgfqpoint{1.741816in}{1.495800in}}%
\pgfpathclose%
\pgfusepath{stroke,fill}%
\end{pgfscope}%
\begin{pgfscope}%
\pgfpathrectangle{\pgfqpoint{0.772069in}{0.515123in}}{\pgfqpoint{1.937500in}{1.347500in}}%
\pgfusepath{clip}%
\pgfsetbuttcap%
\pgfsetroundjoin%
\definecolor{currentfill}{rgb}{0.117647,0.564706,1.000000}%
\pgfsetfillcolor{currentfill}%
\pgfsetlinewidth{1.003750pt}%
\definecolor{currentstroke}{rgb}{0.117647,0.564706,1.000000}%
\pgfsetstrokecolor{currentstroke}%
\pgfsetdash{}{0pt}%
\pgfpathmoveto{\pgfqpoint{1.856908in}{1.611011in}}%
\pgfpathcurveto{\pgfqpoint{1.867958in}{1.611011in}}{\pgfqpoint{1.878557in}{1.615402in}}{\pgfqpoint{1.886370in}{1.623215in}}%
\pgfpathcurveto{\pgfqpoint{1.894184in}{1.631029in}}{\pgfqpoint{1.898574in}{1.641628in}}{\pgfqpoint{1.898574in}{1.652678in}}%
\pgfpathcurveto{\pgfqpoint{1.898574in}{1.663728in}}{\pgfqpoint{1.894184in}{1.674327in}}{\pgfqpoint{1.886370in}{1.682141in}}%
\pgfpathcurveto{\pgfqpoint{1.878557in}{1.689954in}}{\pgfqpoint{1.867958in}{1.694345in}}{\pgfqpoint{1.856908in}{1.694345in}}%
\pgfpathcurveto{\pgfqpoint{1.845857in}{1.694345in}}{\pgfqpoint{1.835258in}{1.689954in}}{\pgfqpoint{1.827445in}{1.682141in}}%
\pgfpathcurveto{\pgfqpoint{1.819631in}{1.674327in}}{\pgfqpoint{1.815241in}{1.663728in}}{\pgfqpoint{1.815241in}{1.652678in}}%
\pgfpathcurveto{\pgfqpoint{1.815241in}{1.641628in}}{\pgfqpoint{1.819631in}{1.631029in}}{\pgfqpoint{1.827445in}{1.623215in}}%
\pgfpathcurveto{\pgfqpoint{1.835258in}{1.615402in}}{\pgfqpoint{1.845857in}{1.611011in}}{\pgfqpoint{1.856908in}{1.611011in}}%
\pgfpathclose%
\pgfusepath{stroke,fill}%
\end{pgfscope}%
\begin{pgfscope}%
\pgfpathrectangle{\pgfqpoint{0.772069in}{0.515123in}}{\pgfqpoint{1.937500in}{1.347500in}}%
\pgfusepath{clip}%
\pgfsetbuttcap%
\pgfsetroundjoin%
\definecolor{currentfill}{rgb}{0.117647,0.564706,1.000000}%
\pgfsetfillcolor{currentfill}%
\pgfsetlinewidth{1.003750pt}%
\definecolor{currentstroke}{rgb}{0.117647,0.564706,1.000000}%
\pgfsetstrokecolor{currentstroke}%
\pgfsetdash{}{0pt}%
\pgfpathmoveto{\pgfqpoint{1.971999in}{1.729714in}}%
\pgfpathcurveto{\pgfqpoint{1.983049in}{1.729714in}}{\pgfqpoint{1.993648in}{1.734104in}}{\pgfqpoint{2.001462in}{1.741918in}}%
\pgfpathcurveto{\pgfqpoint{2.009276in}{1.749731in}}{\pgfqpoint{2.013666in}{1.760330in}}{\pgfqpoint{2.013666in}{1.771380in}}%
\pgfpathcurveto{\pgfqpoint{2.013666in}{1.782431in}}{\pgfqpoint{2.009276in}{1.793030in}}{\pgfqpoint{2.001462in}{1.800843in}}%
\pgfpathcurveto{\pgfqpoint{1.993648in}{1.808657in}}{\pgfqpoint{1.983049in}{1.813047in}}{\pgfqpoint{1.971999in}{1.813047in}}%
\pgfpathcurveto{\pgfqpoint{1.960949in}{1.813047in}}{\pgfqpoint{1.950350in}{1.808657in}}{\pgfqpoint{1.942537in}{1.800843in}}%
\pgfpathcurveto{\pgfqpoint{1.934723in}{1.793030in}}{\pgfqpoint{1.930333in}{1.782431in}}{\pgfqpoint{1.930333in}{1.771380in}}%
\pgfpathcurveto{\pgfqpoint{1.930333in}{1.760330in}}{\pgfqpoint{1.934723in}{1.749731in}}{\pgfqpoint{1.942537in}{1.741918in}}%
\pgfpathcurveto{\pgfqpoint{1.950350in}{1.734104in}}{\pgfqpoint{1.960949in}{1.729714in}}{\pgfqpoint{1.971999in}{1.729714in}}%
\pgfpathclose%
\pgfusepath{stroke,fill}%
\end{pgfscope}%
\begin{pgfscope}%
\pgfpathrectangle{\pgfqpoint{0.772069in}{0.515123in}}{\pgfqpoint{1.937500in}{1.347500in}}%
\pgfusepath{clip}%
\pgfsetbuttcap%
\pgfsetroundjoin%
\definecolor{currentfill}{rgb}{0.529412,0.807843,0.921569}%
\pgfsetfillcolor{currentfill}%
\pgfsetlinewidth{1.003750pt}%
\definecolor{currentstroke}{rgb}{0.529412,0.807843,0.921569}%
\pgfsetstrokecolor{currentstroke}%
\pgfsetdash{}{0pt}%
\pgfpathmoveto{\pgfqpoint{0.890137in}{0.672097in}}%
\pgfpathcurveto{\pgfqpoint{0.901187in}{0.672097in}}{\pgfqpoint{0.911786in}{0.676488in}}{\pgfqpoint{0.919600in}{0.684301in}}%
\pgfpathcurveto{\pgfqpoint{0.927413in}{0.692115in}}{\pgfqpoint{0.931804in}{0.702714in}}{\pgfqpoint{0.931804in}{0.713764in}}%
\pgfpathcurveto{\pgfqpoint{0.931804in}{0.724814in}}{\pgfqpoint{0.927413in}{0.735413in}}{\pgfqpoint{0.919600in}{0.743227in}}%
\pgfpathcurveto{\pgfqpoint{0.911786in}{0.751041in}}{\pgfqpoint{0.901187in}{0.755431in}}{\pgfqpoint{0.890137in}{0.755431in}}%
\pgfpathcurveto{\pgfqpoint{0.879087in}{0.755431in}}{\pgfqpoint{0.868488in}{0.751041in}}{\pgfqpoint{0.860674in}{0.743227in}}%
\pgfpathcurveto{\pgfqpoint{0.852860in}{0.735413in}}{\pgfqpoint{0.848470in}{0.724814in}}{\pgfqpoint{0.848470in}{0.713764in}}%
\pgfpathcurveto{\pgfqpoint{0.848470in}{0.702714in}}{\pgfqpoint{0.852860in}{0.692115in}}{\pgfqpoint{0.860674in}{0.684301in}}%
\pgfpathcurveto{\pgfqpoint{0.868488in}{0.676488in}}{\pgfqpoint{0.879087in}{0.672097in}}{\pgfqpoint{0.890137in}{0.672097in}}%
\pgfpathclose%
\pgfusepath{stroke,fill}%
\end{pgfscope}%
\begin{pgfscope}%
\pgfpathrectangle{\pgfqpoint{0.772069in}{0.515123in}}{\pgfqpoint{1.937500in}{1.347500in}}%
\pgfusepath{clip}%
\pgfsetbuttcap%
\pgfsetroundjoin%
\definecolor{currentfill}{rgb}{0.529412,0.807843,0.921569}%
\pgfsetfillcolor{currentfill}%
\pgfsetlinewidth{1.003750pt}%
\definecolor{currentstroke}{rgb}{0.529412,0.807843,0.921569}%
\pgfsetstrokecolor{currentstroke}%
\pgfsetdash{}{0pt}%
\pgfpathmoveto{\pgfqpoint{0.959192in}{0.784748in}}%
\pgfpathcurveto{\pgfqpoint{0.970242in}{0.784748in}}{\pgfqpoint{0.980841in}{0.789139in}}{\pgfqpoint{0.988655in}{0.796952in}}%
\pgfpathcurveto{\pgfqpoint{0.996468in}{0.804766in}}{\pgfqpoint{1.000859in}{0.815365in}}{\pgfqpoint{1.000859in}{0.826415in}}%
\pgfpathcurveto{\pgfqpoint{1.000859in}{0.837465in}}{\pgfqpoint{0.996468in}{0.848064in}}{\pgfqpoint{0.988655in}{0.855878in}}%
\pgfpathcurveto{\pgfqpoint{0.980841in}{0.863692in}}{\pgfqpoint{0.970242in}{0.868082in}}{\pgfqpoint{0.959192in}{0.868082in}}%
\pgfpathcurveto{\pgfqpoint{0.948142in}{0.868082in}}{\pgfqpoint{0.937543in}{0.863692in}}{\pgfqpoint{0.929729in}{0.855878in}}%
\pgfpathcurveto{\pgfqpoint{0.921916in}{0.848064in}}{\pgfqpoint{0.917525in}{0.837465in}}{\pgfqpoint{0.917525in}{0.826415in}}%
\pgfpathcurveto{\pgfqpoint{0.917525in}{0.815365in}}{\pgfqpoint{0.921916in}{0.804766in}}{\pgfqpoint{0.929729in}{0.796952in}}%
\pgfpathcurveto{\pgfqpoint{0.937543in}{0.789139in}}{\pgfqpoint{0.948142in}{0.784748in}}{\pgfqpoint{0.959192in}{0.784748in}}%
\pgfpathclose%
\pgfusepath{stroke,fill}%
\end{pgfscope}%
\begin{pgfscope}%
\pgfpathrectangle{\pgfqpoint{0.772069in}{0.515123in}}{\pgfqpoint{1.937500in}{1.347500in}}%
\pgfusepath{clip}%
\pgfsetbuttcap%
\pgfsetroundjoin%
\definecolor{currentfill}{rgb}{0.529412,0.807843,0.921569}%
\pgfsetfillcolor{currentfill}%
\pgfsetlinewidth{1.003750pt}%
\definecolor{currentstroke}{rgb}{0.529412,0.807843,0.921569}%
\pgfsetstrokecolor{currentstroke}%
\pgfsetdash{}{0pt}%
\pgfpathmoveto{\pgfqpoint{1.005229in}{0.905779in}}%
\pgfpathcurveto{\pgfqpoint{1.016279in}{0.905779in}}{\pgfqpoint{1.026878in}{0.910169in}}{\pgfqpoint{1.034691in}{0.917982in}}%
\pgfpathcurveto{\pgfqpoint{1.042505in}{0.925796in}}{\pgfqpoint{1.046895in}{0.936395in}}{\pgfqpoint{1.046895in}{0.947445in}}%
\pgfpathcurveto{\pgfqpoint{1.046895in}{0.958495in}}{\pgfqpoint{1.042505in}{0.969094in}}{\pgfqpoint{1.034691in}{0.976908in}}%
\pgfpathcurveto{\pgfqpoint{1.026878in}{0.984722in}}{\pgfqpoint{1.016279in}{0.989112in}}{\pgfqpoint{1.005229in}{0.989112in}}%
\pgfpathcurveto{\pgfqpoint{0.994178in}{0.989112in}}{\pgfqpoint{0.983579in}{0.984722in}}{\pgfqpoint{0.975766in}{0.976908in}}%
\pgfpathcurveto{\pgfqpoint{0.967952in}{0.969094in}}{\pgfqpoint{0.963562in}{0.958495in}}{\pgfqpoint{0.963562in}{0.947445in}}%
\pgfpathcurveto{\pgfqpoint{0.963562in}{0.936395in}}{\pgfqpoint{0.967952in}{0.925796in}}{\pgfqpoint{0.975766in}{0.917982in}}%
\pgfpathcurveto{\pgfqpoint{0.983579in}{0.910169in}}{\pgfqpoint{0.994178in}{0.905779in}}{\pgfqpoint{1.005229in}{0.905779in}}%
\pgfpathclose%
\pgfusepath{stroke,fill}%
\end{pgfscope}%
\begin{pgfscope}%
\pgfpathrectangle{\pgfqpoint{0.772069in}{0.515123in}}{\pgfqpoint{1.937500in}{1.347500in}}%
\pgfusepath{clip}%
\pgfsetbuttcap%
\pgfsetroundjoin%
\definecolor{currentfill}{rgb}{0.529412,0.807843,0.921569}%
\pgfsetfillcolor{currentfill}%
\pgfsetlinewidth{1.003750pt}%
\definecolor{currentstroke}{rgb}{0.529412,0.807843,0.921569}%
\pgfsetstrokecolor{currentstroke}%
\pgfsetdash{}{0pt}%
\pgfpathmoveto{\pgfqpoint{1.074284in}{1.024481in}}%
\pgfpathcurveto{\pgfqpoint{1.085334in}{1.024481in}}{\pgfqpoint{1.095933in}{1.028871in}}{\pgfqpoint{1.103746in}{1.036685in}}%
\pgfpathcurveto{\pgfqpoint{1.111560in}{1.044499in}}{\pgfqpoint{1.115950in}{1.055098in}}{\pgfqpoint{1.115950in}{1.066148in}}%
\pgfpathcurveto{\pgfqpoint{1.115950in}{1.077198in}}{\pgfqpoint{1.111560in}{1.087797in}}{\pgfqpoint{1.103746in}{1.095611in}}%
\pgfpathcurveto{\pgfqpoint{1.095933in}{1.103424in}}{\pgfqpoint{1.085334in}{1.107814in}}{\pgfqpoint{1.074284in}{1.107814in}}%
\pgfpathcurveto{\pgfqpoint{1.063234in}{1.107814in}}{\pgfqpoint{1.052635in}{1.103424in}}{\pgfqpoint{1.044821in}{1.095611in}}%
\pgfpathcurveto{\pgfqpoint{1.037007in}{1.087797in}}{\pgfqpoint{1.032617in}{1.077198in}}{\pgfqpoint{1.032617in}{1.066148in}}%
\pgfpathcurveto{\pgfqpoint{1.032617in}{1.055098in}}{\pgfqpoint{1.037007in}{1.044499in}}{\pgfqpoint{1.044821in}{1.036685in}}%
\pgfpathcurveto{\pgfqpoint{1.052635in}{1.028871in}}{\pgfqpoint{1.063234in}{1.024481in}}{\pgfqpoint{1.074284in}{1.024481in}}%
\pgfpathclose%
\pgfusepath{stroke,fill}%
\end{pgfscope}%
\begin{pgfscope}%
\pgfpathrectangle{\pgfqpoint{0.772069in}{0.515123in}}{\pgfqpoint{1.937500in}{1.347500in}}%
\pgfusepath{clip}%
\pgfsetbuttcap%
\pgfsetroundjoin%
\definecolor{currentfill}{rgb}{0.529412,0.807843,0.921569}%
\pgfsetfillcolor{currentfill}%
\pgfsetlinewidth{1.003750pt}%
\definecolor{currentstroke}{rgb}{0.529412,0.807843,0.921569}%
\pgfsetstrokecolor{currentstroke}%
\pgfsetdash{}{0pt}%
\pgfpathmoveto{\pgfqpoint{1.143339in}{1.143184in}}%
\pgfpathcurveto{\pgfqpoint{1.154389in}{1.143184in}}{\pgfqpoint{1.164988in}{1.147574in}}{\pgfqpoint{1.172801in}{1.155387in}}%
\pgfpathcurveto{\pgfqpoint{1.180615in}{1.163201in}}{\pgfqpoint{1.185005in}{1.173800in}}{\pgfqpoint{1.185005in}{1.184850in}}%
\pgfpathcurveto{\pgfqpoint{1.185005in}{1.195900in}}{\pgfqpoint{1.180615in}{1.206499in}}{\pgfqpoint{1.172801in}{1.214313in}}%
\pgfpathcurveto{\pgfqpoint{1.164988in}{1.222127in}}{\pgfqpoint{1.154389in}{1.226517in}}{\pgfqpoint{1.143339in}{1.226517in}}%
\pgfpathcurveto{\pgfqpoint{1.132289in}{1.226517in}}{\pgfqpoint{1.121690in}{1.222127in}}{\pgfqpoint{1.113876in}{1.214313in}}%
\pgfpathcurveto{\pgfqpoint{1.106062in}{1.206499in}}{\pgfqpoint{1.101672in}{1.195900in}}{\pgfqpoint{1.101672in}{1.184850in}}%
\pgfpathcurveto{\pgfqpoint{1.101672in}{1.173800in}}{\pgfqpoint{1.106062in}{1.163201in}}{\pgfqpoint{1.113876in}{1.155387in}}%
\pgfpathcurveto{\pgfqpoint{1.121690in}{1.147574in}}{\pgfqpoint{1.132289in}{1.143184in}}{\pgfqpoint{1.143339in}{1.143184in}}%
\pgfpathclose%
\pgfusepath{stroke,fill}%
\end{pgfscope}%
\begin{pgfscope}%
\pgfpathrectangle{\pgfqpoint{0.772069in}{0.515123in}}{\pgfqpoint{1.937500in}{1.347500in}}%
\pgfusepath{clip}%
\pgfsetbuttcap%
\pgfsetroundjoin%
\definecolor{currentfill}{rgb}{0.529412,0.807843,0.921569}%
\pgfsetfillcolor{currentfill}%
\pgfsetlinewidth{1.003750pt}%
\definecolor{currentstroke}{rgb}{0.529412,0.807843,0.921569}%
\pgfsetstrokecolor{currentstroke}%
\pgfsetdash{}{0pt}%
\pgfpathmoveto{\pgfqpoint{1.189375in}{1.256067in}}%
\pgfpathcurveto{\pgfqpoint{1.200426in}{1.256067in}}{\pgfqpoint{1.211025in}{1.260458in}}{\pgfqpoint{1.218838in}{1.268271in}}%
\pgfpathcurveto{\pgfqpoint{1.226652in}{1.276085in}}{\pgfqpoint{1.231042in}{1.286684in}}{\pgfqpoint{1.231042in}{1.297734in}}%
\pgfpathcurveto{\pgfqpoint{1.231042in}{1.308784in}}{\pgfqpoint{1.226652in}{1.319383in}}{\pgfqpoint{1.218838in}{1.327197in}}%
\pgfpathcurveto{\pgfqpoint{1.211025in}{1.335010in}}{\pgfqpoint{1.200426in}{1.339401in}}{\pgfqpoint{1.189375in}{1.339401in}}%
\pgfpathcurveto{\pgfqpoint{1.178325in}{1.339401in}}{\pgfqpoint{1.167726in}{1.335010in}}{\pgfqpoint{1.159913in}{1.327197in}}%
\pgfpathcurveto{\pgfqpoint{1.152099in}{1.319383in}}{\pgfqpoint{1.147709in}{1.308784in}}{\pgfqpoint{1.147709in}{1.297734in}}%
\pgfpathcurveto{\pgfqpoint{1.147709in}{1.286684in}}{\pgfqpoint{1.152099in}{1.276085in}}{\pgfqpoint{1.159913in}{1.268271in}}%
\pgfpathcurveto{\pgfqpoint{1.167726in}{1.260458in}}{\pgfqpoint{1.178325in}{1.256067in}}{\pgfqpoint{1.189375in}{1.256067in}}%
\pgfpathclose%
\pgfusepath{stroke,fill}%
\end{pgfscope}%
\begin{pgfscope}%
\pgfpathrectangle{\pgfqpoint{0.772069in}{0.515123in}}{\pgfqpoint{1.937500in}{1.347500in}}%
\pgfusepath{clip}%
\pgfsetbuttcap%
\pgfsetroundjoin%
\definecolor{currentfill}{rgb}{0.529412,0.807843,0.921569}%
\pgfsetfillcolor{currentfill}%
\pgfsetlinewidth{1.003750pt}%
\definecolor{currentstroke}{rgb}{0.529412,0.807843,0.921569}%
\pgfsetstrokecolor{currentstroke}%
\pgfsetdash{}{0pt}%
\pgfpathmoveto{\pgfqpoint{1.258430in}{1.378261in}}%
\pgfpathcurveto{\pgfqpoint{1.269481in}{1.378261in}}{\pgfqpoint{1.280080in}{1.382651in}}{\pgfqpoint{1.287893in}{1.390465in}}%
\pgfpathcurveto{\pgfqpoint{1.295707in}{1.398279in}}{\pgfqpoint{1.300097in}{1.408878in}}{\pgfqpoint{1.300097in}{1.419928in}}%
\pgfpathcurveto{\pgfqpoint{1.300097in}{1.430978in}}{\pgfqpoint{1.295707in}{1.441577in}}{\pgfqpoint{1.287893in}{1.449391in}}%
\pgfpathcurveto{\pgfqpoint{1.280080in}{1.457204in}}{\pgfqpoint{1.269481in}{1.461595in}}{\pgfqpoint{1.258430in}{1.461595in}}%
\pgfpathcurveto{\pgfqpoint{1.247380in}{1.461595in}}{\pgfqpoint{1.236781in}{1.457204in}}{\pgfqpoint{1.228968in}{1.449391in}}%
\pgfpathcurveto{\pgfqpoint{1.221154in}{1.441577in}}{\pgfqpoint{1.216764in}{1.430978in}}{\pgfqpoint{1.216764in}{1.419928in}}%
\pgfpathcurveto{\pgfqpoint{1.216764in}{1.408878in}}{\pgfqpoint{1.221154in}{1.398279in}}{\pgfqpoint{1.228968in}{1.390465in}}%
\pgfpathcurveto{\pgfqpoint{1.236781in}{1.382651in}}{\pgfqpoint{1.247380in}{1.378261in}}{\pgfqpoint{1.258430in}{1.378261in}}%
\pgfpathclose%
\pgfusepath{stroke,fill}%
\end{pgfscope}%
\begin{pgfscope}%
\pgfpathrectangle{\pgfqpoint{0.772069in}{0.515123in}}{\pgfqpoint{1.937500in}{1.347500in}}%
\pgfusepath{clip}%
\pgfsetbuttcap%
\pgfsetroundjoin%
\definecolor{currentfill}{rgb}{0.529412,0.807843,0.921569}%
\pgfsetfillcolor{currentfill}%
\pgfsetlinewidth{1.003750pt}%
\definecolor{currentstroke}{rgb}{0.529412,0.807843,0.921569}%
\pgfsetstrokecolor{currentstroke}%
\pgfsetdash{}{0pt}%
\pgfpathmoveto{\pgfqpoint{1.327486in}{1.495800in}}%
\pgfpathcurveto{\pgfqpoint{1.338536in}{1.495800in}}{\pgfqpoint{1.349135in}{1.500190in}}{\pgfqpoint{1.356948in}{1.508004in}}%
\pgfpathcurveto{\pgfqpoint{1.364762in}{1.515817in}}{\pgfqpoint{1.369152in}{1.526417in}}{\pgfqpoint{1.369152in}{1.537467in}}%
\pgfpathcurveto{\pgfqpoint{1.369152in}{1.548517in}}{\pgfqpoint{1.364762in}{1.559116in}}{\pgfqpoint{1.356948in}{1.566929in}}%
\pgfpathcurveto{\pgfqpoint{1.349135in}{1.574743in}}{\pgfqpoint{1.338536in}{1.579133in}}{\pgfqpoint{1.327486in}{1.579133in}}%
\pgfpathcurveto{\pgfqpoint{1.316435in}{1.579133in}}{\pgfqpoint{1.305836in}{1.574743in}}{\pgfqpoint{1.298023in}{1.566929in}}%
\pgfpathcurveto{\pgfqpoint{1.290209in}{1.559116in}}{\pgfqpoint{1.285819in}{1.548517in}}{\pgfqpoint{1.285819in}{1.537467in}}%
\pgfpathcurveto{\pgfqpoint{1.285819in}{1.526417in}}{\pgfqpoint{1.290209in}{1.515817in}}{\pgfqpoint{1.298023in}{1.508004in}}%
\pgfpathcurveto{\pgfqpoint{1.305836in}{1.500190in}}{\pgfqpoint{1.316435in}{1.495800in}}{\pgfqpoint{1.327486in}{1.495800in}}%
\pgfpathclose%
\pgfusepath{stroke,fill}%
\end{pgfscope}%
\begin{pgfscope}%
\pgfpathrectangle{\pgfqpoint{0.772069in}{0.515123in}}{\pgfqpoint{1.937500in}{1.347500in}}%
\pgfusepath{clip}%
\pgfsetbuttcap%
\pgfsetroundjoin%
\definecolor{currentfill}{rgb}{0.529412,0.807843,0.921569}%
\pgfsetfillcolor{currentfill}%
\pgfsetlinewidth{1.003750pt}%
\definecolor{currentstroke}{rgb}{0.529412,0.807843,0.921569}%
\pgfsetstrokecolor{currentstroke}%
\pgfsetdash{}{0pt}%
\pgfpathmoveto{\pgfqpoint{1.396541in}{1.611011in}}%
\pgfpathcurveto{\pgfqpoint{1.407591in}{1.611011in}}{\pgfqpoint{1.418190in}{1.615402in}}{\pgfqpoint{1.426003in}{1.623215in}}%
\pgfpathcurveto{\pgfqpoint{1.433817in}{1.631029in}}{\pgfqpoint{1.438207in}{1.641628in}}{\pgfqpoint{1.438207in}{1.652678in}}%
\pgfpathcurveto{\pgfqpoint{1.438207in}{1.663728in}}{\pgfqpoint{1.433817in}{1.674327in}}{\pgfqpoint{1.426003in}{1.682141in}}%
\pgfpathcurveto{\pgfqpoint{1.418190in}{1.689954in}}{\pgfqpoint{1.407591in}{1.694345in}}{\pgfqpoint{1.396541in}{1.694345in}}%
\pgfpathcurveto{\pgfqpoint{1.385490in}{1.694345in}}{\pgfqpoint{1.374891in}{1.689954in}}{\pgfqpoint{1.367078in}{1.682141in}}%
\pgfpathcurveto{\pgfqpoint{1.359264in}{1.674327in}}{\pgfqpoint{1.354874in}{1.663728in}}{\pgfqpoint{1.354874in}{1.652678in}}%
\pgfpathcurveto{\pgfqpoint{1.354874in}{1.641628in}}{\pgfqpoint{1.359264in}{1.631029in}}{\pgfqpoint{1.367078in}{1.623215in}}%
\pgfpathcurveto{\pgfqpoint{1.374891in}{1.615402in}}{\pgfqpoint{1.385490in}{1.611011in}}{\pgfqpoint{1.396541in}{1.611011in}}%
\pgfpathclose%
\pgfusepath{stroke,fill}%
\end{pgfscope}%
\begin{pgfscope}%
\pgfpathrectangle{\pgfqpoint{0.772069in}{0.515123in}}{\pgfqpoint{1.937500in}{1.347500in}}%
\pgfusepath{clip}%
\pgfsetbuttcap%
\pgfsetroundjoin%
\definecolor{currentfill}{rgb}{0.529412,0.807843,0.921569}%
\pgfsetfillcolor{currentfill}%
\pgfsetlinewidth{1.003750pt}%
\definecolor{currentstroke}{rgb}{0.529412,0.807843,0.921569}%
\pgfsetstrokecolor{currentstroke}%
\pgfsetdash{}{0pt}%
\pgfpathmoveto{\pgfqpoint{1.442577in}{1.729714in}}%
\pgfpathcurveto{\pgfqpoint{1.453627in}{1.729714in}}{\pgfqpoint{1.464226in}{1.734104in}}{\pgfqpoint{1.472040in}{1.741918in}}%
\pgfpathcurveto{\pgfqpoint{1.479854in}{1.749731in}}{\pgfqpoint{1.484244in}{1.760330in}}{\pgfqpoint{1.484244in}{1.771380in}}%
\pgfpathcurveto{\pgfqpoint{1.484244in}{1.782431in}}{\pgfqpoint{1.479854in}{1.793030in}}{\pgfqpoint{1.472040in}{1.800843in}}%
\pgfpathcurveto{\pgfqpoint{1.464226in}{1.808657in}}{\pgfqpoint{1.453627in}{1.813047in}}{\pgfqpoint{1.442577in}{1.813047in}}%
\pgfpathcurveto{\pgfqpoint{1.431527in}{1.813047in}}{\pgfqpoint{1.420928in}{1.808657in}}{\pgfqpoint{1.413114in}{1.800843in}}%
\pgfpathcurveto{\pgfqpoint{1.405301in}{1.793030in}}{\pgfqpoint{1.400911in}{1.782431in}}{\pgfqpoint{1.400911in}{1.771380in}}%
\pgfpathcurveto{\pgfqpoint{1.400911in}{1.760330in}}{\pgfqpoint{1.405301in}{1.749731in}}{\pgfqpoint{1.413114in}{1.741918in}}%
\pgfpathcurveto{\pgfqpoint{1.420928in}{1.734104in}}{\pgfqpoint{1.431527in}{1.729714in}}{\pgfqpoint{1.442577in}{1.729714in}}%
\pgfpathclose%
\pgfusepath{stroke,fill}%
\end{pgfscope}%
\begin{pgfscope}%
\pgfpathrectangle{\pgfqpoint{0.772069in}{0.515123in}}{\pgfqpoint{1.937500in}{1.347500in}}%
\pgfusepath{clip}%
\pgfsetbuttcap%
\pgfsetroundjoin%
\definecolor{currentfill}{rgb}{1.000000,0.627451,0.478431}%
\pgfsetfillcolor{currentfill}%
\pgfsetlinewidth{1.003750pt}%
\definecolor{currentstroke}{rgb}{1.000000,0.627451,0.478431}%
\pgfsetstrokecolor{currentstroke}%
\pgfsetdash{}{0pt}%
\pgfpathmoveto{\pgfqpoint{0.936174in}{0.567907in}}%
\pgfpathcurveto{\pgfqpoint{0.947224in}{0.567907in}}{\pgfqpoint{0.957823in}{0.572297in}}{\pgfqpoint{0.965636in}{0.580111in}}%
\pgfpathcurveto{\pgfqpoint{0.973450in}{0.587924in}}{\pgfqpoint{0.977840in}{0.598523in}}{\pgfqpoint{0.977840in}{0.609574in}}%
\pgfpathcurveto{\pgfqpoint{0.977840in}{0.620624in}}{\pgfqpoint{0.973450in}{0.631223in}}{\pgfqpoint{0.965636in}{0.639036in}}%
\pgfpathcurveto{\pgfqpoint{0.957823in}{0.646850in}}{\pgfqpoint{0.947224in}{0.651240in}}{\pgfqpoint{0.936174in}{0.651240in}}%
\pgfpathcurveto{\pgfqpoint{0.925123in}{0.651240in}}{\pgfqpoint{0.914524in}{0.646850in}}{\pgfqpoint{0.906711in}{0.639036in}}%
\pgfpathcurveto{\pgfqpoint{0.898897in}{0.631223in}}{\pgfqpoint{0.894507in}{0.620624in}}{\pgfqpoint{0.894507in}{0.609574in}}%
\pgfpathcurveto{\pgfqpoint{0.894507in}{0.598523in}}{\pgfqpoint{0.898897in}{0.587924in}}{\pgfqpoint{0.906711in}{0.580111in}}%
\pgfpathcurveto{\pgfqpoint{0.914524in}{0.572297in}}{\pgfqpoint{0.925123in}{0.567907in}}{\pgfqpoint{0.936174in}{0.567907in}}%
\pgfpathclose%
\pgfusepath{stroke,fill}%
\end{pgfscope}%
\begin{pgfscope}%
\pgfpathrectangle{\pgfqpoint{0.772069in}{0.515123in}}{\pgfqpoint{1.937500in}{1.347500in}}%
\pgfusepath{clip}%
\pgfsetbuttcap%
\pgfsetroundjoin%
\definecolor{currentfill}{rgb}{1.000000,0.627451,0.478431}%
\pgfsetfillcolor{currentfill}%
\pgfsetlinewidth{1.003750pt}%
\definecolor{currentstroke}{rgb}{1.000000,0.627451,0.478431}%
\pgfsetstrokecolor{currentstroke}%
\pgfsetdash{}{0pt}%
\pgfpathmoveto{\pgfqpoint{1.051265in}{0.582372in}}%
\pgfpathcurveto{\pgfqpoint{1.062315in}{0.582372in}}{\pgfqpoint{1.072914in}{0.586763in}}{\pgfqpoint{1.080728in}{0.594576in}}%
\pgfpathcurveto{\pgfqpoint{1.088542in}{0.602390in}}{\pgfqpoint{1.092932in}{0.612989in}}{\pgfqpoint{1.092932in}{0.624039in}}%
\pgfpathcurveto{\pgfqpoint{1.092932in}{0.635089in}}{\pgfqpoint{1.088542in}{0.645688in}}{\pgfqpoint{1.080728in}{0.653502in}}%
\pgfpathcurveto{\pgfqpoint{1.072914in}{0.661315in}}{\pgfqpoint{1.062315in}{0.665706in}}{\pgfqpoint{1.051265in}{0.665706in}}%
\pgfpathcurveto{\pgfqpoint{1.040215in}{0.665706in}}{\pgfqpoint{1.029616in}{0.661315in}}{\pgfqpoint{1.021803in}{0.653502in}}%
\pgfpathcurveto{\pgfqpoint{1.013989in}{0.645688in}}{\pgfqpoint{1.009599in}{0.635089in}}{\pgfqpoint{1.009599in}{0.624039in}}%
\pgfpathcurveto{\pgfqpoint{1.009599in}{0.612989in}}{\pgfqpoint{1.013989in}{0.602390in}}{\pgfqpoint{1.021803in}{0.594576in}}%
\pgfpathcurveto{\pgfqpoint{1.029616in}{0.586763in}}{\pgfqpoint{1.040215in}{0.582372in}}{\pgfqpoint{1.051265in}{0.582372in}}%
\pgfpathclose%
\pgfusepath{stroke,fill}%
\end{pgfscope}%
\begin{pgfscope}%
\pgfpathrectangle{\pgfqpoint{0.772069in}{0.515123in}}{\pgfqpoint{1.937500in}{1.347500in}}%
\pgfusepath{clip}%
\pgfsetbuttcap%
\pgfsetroundjoin%
\definecolor{currentfill}{rgb}{1.000000,0.627451,0.478431}%
\pgfsetfillcolor{currentfill}%
\pgfsetlinewidth{1.003750pt}%
\definecolor{currentstroke}{rgb}{1.000000,0.627451,0.478431}%
\pgfsetstrokecolor{currentstroke}%
\pgfsetdash{}{0pt}%
\pgfpathmoveto{\pgfqpoint{1.166357in}{0.586562in}}%
\pgfpathcurveto{\pgfqpoint{1.177407in}{0.586562in}}{\pgfqpoint{1.188006in}{0.590952in}}{\pgfqpoint{1.195820in}{0.598766in}}%
\pgfpathcurveto{\pgfqpoint{1.203633in}{0.606579in}}{\pgfqpoint{1.208024in}{0.617178in}}{\pgfqpoint{1.208024in}{0.628228in}}%
\pgfpathcurveto{\pgfqpoint{1.208024in}{0.639279in}}{\pgfqpoint{1.203633in}{0.649878in}}{\pgfqpoint{1.195820in}{0.657691in}}%
\pgfpathcurveto{\pgfqpoint{1.188006in}{0.665505in}}{\pgfqpoint{1.177407in}{0.669895in}}{\pgfqpoint{1.166357in}{0.669895in}}%
\pgfpathcurveto{\pgfqpoint{1.155307in}{0.669895in}}{\pgfqpoint{1.144708in}{0.665505in}}{\pgfqpoint{1.136894in}{0.657691in}}%
\pgfpathcurveto{\pgfqpoint{1.129081in}{0.649878in}}{\pgfqpoint{1.124690in}{0.639279in}}{\pgfqpoint{1.124690in}{0.628228in}}%
\pgfpathcurveto{\pgfqpoint{1.124690in}{0.617178in}}{\pgfqpoint{1.129081in}{0.606579in}}{\pgfqpoint{1.136894in}{0.598766in}}%
\pgfpathcurveto{\pgfqpoint{1.144708in}{0.590952in}}{\pgfqpoint{1.155307in}{0.586562in}}{\pgfqpoint{1.166357in}{0.586562in}}%
\pgfpathclose%
\pgfusepath{stroke,fill}%
\end{pgfscope}%
\begin{pgfscope}%
\pgfpathrectangle{\pgfqpoint{0.772069in}{0.515123in}}{\pgfqpoint{1.937500in}{1.347500in}}%
\pgfusepath{clip}%
\pgfsetbuttcap%
\pgfsetroundjoin%
\definecolor{currentfill}{rgb}{1.000000,0.627451,0.478431}%
\pgfsetfillcolor{currentfill}%
\pgfsetlinewidth{1.003750pt}%
\definecolor{currentstroke}{rgb}{1.000000,0.627451,0.478431}%
\pgfsetstrokecolor{currentstroke}%
\pgfsetdash{}{0pt}%
\pgfpathmoveto{\pgfqpoint{1.281449in}{0.613794in}}%
\pgfpathcurveto{\pgfqpoint{1.292499in}{0.613794in}}{\pgfqpoint{1.303098in}{0.618184in}}{\pgfqpoint{1.310912in}{0.625997in}}%
\pgfpathcurveto{\pgfqpoint{1.318725in}{0.633811in}}{\pgfqpoint{1.323115in}{0.644410in}}{\pgfqpoint{1.323115in}{0.655460in}}%
\pgfpathcurveto{\pgfqpoint{1.323115in}{0.666510in}}{\pgfqpoint{1.318725in}{0.677109in}}{\pgfqpoint{1.310912in}{0.684923in}}%
\pgfpathcurveto{\pgfqpoint{1.303098in}{0.692737in}}{\pgfqpoint{1.292499in}{0.697127in}}{\pgfqpoint{1.281449in}{0.697127in}}%
\pgfpathcurveto{\pgfqpoint{1.270399in}{0.697127in}}{\pgfqpoint{1.259800in}{0.692737in}}{\pgfqpoint{1.251986in}{0.684923in}}%
\pgfpathcurveto{\pgfqpoint{1.244172in}{0.677109in}}{\pgfqpoint{1.239782in}{0.666510in}}{\pgfqpoint{1.239782in}{0.655460in}}%
\pgfpathcurveto{\pgfqpoint{1.239782in}{0.644410in}}{\pgfqpoint{1.244172in}{0.633811in}}{\pgfqpoint{1.251986in}{0.625997in}}%
\pgfpathcurveto{\pgfqpoint{1.259800in}{0.618184in}}{\pgfqpoint{1.270399in}{0.613794in}}{\pgfqpoint{1.281449in}{0.613794in}}%
\pgfpathclose%
\pgfusepath{stroke,fill}%
\end{pgfscope}%
\begin{pgfscope}%
\pgfpathrectangle{\pgfqpoint{0.772069in}{0.515123in}}{\pgfqpoint{1.937500in}{1.347500in}}%
\pgfusepath{clip}%
\pgfsetbuttcap%
\pgfsetroundjoin%
\definecolor{currentfill}{rgb}{1.000000,0.627451,0.478431}%
\pgfsetfillcolor{currentfill}%
\pgfsetlinewidth{1.003750pt}%
\definecolor{currentstroke}{rgb}{1.000000,0.627451,0.478431}%
\pgfsetstrokecolor{currentstroke}%
\pgfsetdash{}{0pt}%
\pgfpathmoveto{\pgfqpoint{1.396541in}{0.629271in}}%
\pgfpathcurveto{\pgfqpoint{1.407591in}{0.629271in}}{\pgfqpoint{1.418190in}{0.633662in}}{\pgfqpoint{1.426003in}{0.641475in}}%
\pgfpathcurveto{\pgfqpoint{1.433817in}{0.649289in}}{\pgfqpoint{1.438207in}{0.659888in}}{\pgfqpoint{1.438207in}{0.670938in}}%
\pgfpathcurveto{\pgfqpoint{1.438207in}{0.681988in}}{\pgfqpoint{1.433817in}{0.692587in}}{\pgfqpoint{1.426003in}{0.700401in}}%
\pgfpathcurveto{\pgfqpoint{1.418190in}{0.708214in}}{\pgfqpoint{1.407591in}{0.712605in}}{\pgfqpoint{1.396541in}{0.712605in}}%
\pgfpathcurveto{\pgfqpoint{1.385490in}{0.712605in}}{\pgfqpoint{1.374891in}{0.708214in}}{\pgfqpoint{1.367078in}{0.700401in}}%
\pgfpathcurveto{\pgfqpoint{1.359264in}{0.692587in}}{\pgfqpoint{1.354874in}{0.681988in}}{\pgfqpoint{1.354874in}{0.670938in}}%
\pgfpathcurveto{\pgfqpoint{1.354874in}{0.659888in}}{\pgfqpoint{1.359264in}{0.649289in}}{\pgfqpoint{1.367078in}{0.641475in}}%
\pgfpathcurveto{\pgfqpoint{1.374891in}{0.633662in}}{\pgfqpoint{1.385490in}{0.629271in}}{\pgfqpoint{1.396541in}{0.629271in}}%
\pgfpathclose%
\pgfusepath{stroke,fill}%
\end{pgfscope}%
\begin{pgfscope}%
\pgfpathrectangle{\pgfqpoint{0.772069in}{0.515123in}}{\pgfqpoint{1.937500in}{1.347500in}}%
\pgfusepath{clip}%
\pgfsetbuttcap%
\pgfsetroundjoin%
\definecolor{currentfill}{rgb}{1.000000,0.627451,0.478431}%
\pgfsetfillcolor{currentfill}%
\pgfsetlinewidth{1.003750pt}%
\definecolor{currentstroke}{rgb}{1.000000,0.627451,0.478431}%
\pgfsetstrokecolor{currentstroke}%
\pgfsetdash{}{0pt}%
\pgfpathmoveto{\pgfqpoint{1.511632in}{0.644051in}}%
\pgfpathcurveto{\pgfqpoint{1.522682in}{0.644051in}}{\pgfqpoint{1.533281in}{0.648441in}}{\pgfqpoint{1.541095in}{0.656255in}}%
\pgfpathcurveto{\pgfqpoint{1.548909in}{0.664069in}}{\pgfqpoint{1.553299in}{0.674668in}}{\pgfqpoint{1.553299in}{0.685718in}}%
\pgfpathcurveto{\pgfqpoint{1.553299in}{0.696768in}}{\pgfqpoint{1.548909in}{0.707367in}}{\pgfqpoint{1.541095in}{0.715181in}}%
\pgfpathcurveto{\pgfqpoint{1.533281in}{0.722994in}}{\pgfqpoint{1.522682in}{0.727384in}}{\pgfqpoint{1.511632in}{0.727384in}}%
\pgfpathcurveto{\pgfqpoint{1.500582in}{0.727384in}}{\pgfqpoint{1.489983in}{0.722994in}}{\pgfqpoint{1.482170in}{0.715181in}}%
\pgfpathcurveto{\pgfqpoint{1.474356in}{0.707367in}}{\pgfqpoint{1.469966in}{0.696768in}}{\pgfqpoint{1.469966in}{0.685718in}}%
\pgfpathcurveto{\pgfqpoint{1.469966in}{0.674668in}}{\pgfqpoint{1.474356in}{0.664069in}}{\pgfqpoint{1.482170in}{0.656255in}}%
\pgfpathcurveto{\pgfqpoint{1.489983in}{0.648441in}}{\pgfqpoint{1.500582in}{0.644051in}}{\pgfqpoint{1.511632in}{0.644051in}}%
\pgfpathclose%
\pgfusepath{stroke,fill}%
\end{pgfscope}%
\begin{pgfscope}%
\pgfpathrectangle{\pgfqpoint{0.772069in}{0.515123in}}{\pgfqpoint{1.937500in}{1.347500in}}%
\pgfusepath{clip}%
\pgfsetbuttcap%
\pgfsetroundjoin%
\definecolor{currentfill}{rgb}{1.000000,0.627451,0.478431}%
\pgfsetfillcolor{currentfill}%
\pgfsetlinewidth{1.003750pt}%
\definecolor{currentstroke}{rgb}{1.000000,0.627451,0.478431}%
\pgfsetstrokecolor{currentstroke}%
\pgfsetdash{}{0pt}%
\pgfpathmoveto{\pgfqpoint{1.626724in}{0.660111in}}%
\pgfpathcurveto{\pgfqpoint{1.637774in}{0.660111in}}{\pgfqpoint{1.648373in}{0.664501in}}{\pgfqpoint{1.656187in}{0.672315in}}%
\pgfpathcurveto{\pgfqpoint{1.664000in}{0.680128in}}{\pgfqpoint{1.668391in}{0.690727in}}{\pgfqpoint{1.668391in}{0.701777in}}%
\pgfpathcurveto{\pgfqpoint{1.668391in}{0.712828in}}{\pgfqpoint{1.664000in}{0.723427in}}{\pgfqpoint{1.656187in}{0.731240in}}%
\pgfpathcurveto{\pgfqpoint{1.648373in}{0.739054in}}{\pgfqpoint{1.637774in}{0.743444in}}{\pgfqpoint{1.626724in}{0.743444in}}%
\pgfpathcurveto{\pgfqpoint{1.615674in}{0.743444in}}{\pgfqpoint{1.605075in}{0.739054in}}{\pgfqpoint{1.597261in}{0.731240in}}%
\pgfpathcurveto{\pgfqpoint{1.589448in}{0.723427in}}{\pgfqpoint{1.585057in}{0.712828in}}{\pgfqpoint{1.585057in}{0.701777in}}%
\pgfpathcurveto{\pgfqpoint{1.585057in}{0.690727in}}{\pgfqpoint{1.589448in}{0.680128in}}{\pgfqpoint{1.597261in}{0.672315in}}%
\pgfpathcurveto{\pgfqpoint{1.605075in}{0.664501in}}{\pgfqpoint{1.615674in}{0.660111in}}{\pgfqpoint{1.626724in}{0.660111in}}%
\pgfpathclose%
\pgfusepath{stroke,fill}%
\end{pgfscope}%
\begin{pgfscope}%
\pgfpathrectangle{\pgfqpoint{0.772069in}{0.515123in}}{\pgfqpoint{1.937500in}{1.347500in}}%
\pgfusepath{clip}%
\pgfsetbuttcap%
\pgfsetroundjoin%
\definecolor{currentfill}{rgb}{1.000000,0.627451,0.478431}%
\pgfsetfillcolor{currentfill}%
\pgfsetlinewidth{1.003750pt}%
\definecolor{currentstroke}{rgb}{1.000000,0.627451,0.478431}%
\pgfsetstrokecolor{currentstroke}%
\pgfsetdash{}{0pt}%
\pgfpathmoveto{\pgfqpoint{1.741816in}{0.675240in}}%
\pgfpathcurveto{\pgfqpoint{1.752866in}{0.675240in}}{\pgfqpoint{1.763465in}{0.679630in}}{\pgfqpoint{1.771279in}{0.687443in}}%
\pgfpathcurveto{\pgfqpoint{1.779092in}{0.695257in}}{\pgfqpoint{1.783482in}{0.705856in}}{\pgfqpoint{1.783482in}{0.716906in}}%
\pgfpathcurveto{\pgfqpoint{1.783482in}{0.727956in}}{\pgfqpoint{1.779092in}{0.738555in}}{\pgfqpoint{1.771279in}{0.746369in}}%
\pgfpathcurveto{\pgfqpoint{1.763465in}{0.754183in}}{\pgfqpoint{1.752866in}{0.758573in}}{\pgfqpoint{1.741816in}{0.758573in}}%
\pgfpathcurveto{\pgfqpoint{1.730766in}{0.758573in}}{\pgfqpoint{1.720167in}{0.754183in}}{\pgfqpoint{1.712353in}{0.746369in}}%
\pgfpathcurveto{\pgfqpoint{1.704539in}{0.738555in}}{\pgfqpoint{1.700149in}{0.727956in}}{\pgfqpoint{1.700149in}{0.716906in}}%
\pgfpathcurveto{\pgfqpoint{1.700149in}{0.705856in}}{\pgfqpoint{1.704539in}{0.695257in}}{\pgfqpoint{1.712353in}{0.687443in}}%
\pgfpathcurveto{\pgfqpoint{1.720167in}{0.679630in}}{\pgfqpoint{1.730766in}{0.675240in}}{\pgfqpoint{1.741816in}{0.675240in}}%
\pgfpathclose%
\pgfusepath{stroke,fill}%
\end{pgfscope}%
\begin{pgfscope}%
\pgfpathrectangle{\pgfqpoint{0.772069in}{0.515123in}}{\pgfqpoint{1.937500in}{1.347500in}}%
\pgfusepath{clip}%
\pgfsetbuttcap%
\pgfsetroundjoin%
\definecolor{currentfill}{rgb}{1.000000,0.627451,0.478431}%
\pgfsetfillcolor{currentfill}%
\pgfsetlinewidth{1.003750pt}%
\definecolor{currentstroke}{rgb}{1.000000,0.627451,0.478431}%
\pgfsetstrokecolor{currentstroke}%
\pgfsetdash{}{0pt}%
\pgfpathmoveto{\pgfqpoint{1.856908in}{0.690368in}}%
\pgfpathcurveto{\pgfqpoint{1.867958in}{0.690368in}}{\pgfqpoint{1.878557in}{0.694759in}}{\pgfqpoint{1.886370in}{0.702572in}}%
\pgfpathcurveto{\pgfqpoint{1.894184in}{0.710386in}}{\pgfqpoint{1.898574in}{0.720985in}}{\pgfqpoint{1.898574in}{0.732035in}}%
\pgfpathcurveto{\pgfqpoint{1.898574in}{0.743085in}}{\pgfqpoint{1.894184in}{0.753684in}}{\pgfqpoint{1.886370in}{0.761498in}}%
\pgfpathcurveto{\pgfqpoint{1.878557in}{0.769311in}}{\pgfqpoint{1.867958in}{0.773702in}}{\pgfqpoint{1.856908in}{0.773702in}}%
\pgfpathcurveto{\pgfqpoint{1.845857in}{0.773702in}}{\pgfqpoint{1.835258in}{0.769311in}}{\pgfqpoint{1.827445in}{0.761498in}}%
\pgfpathcurveto{\pgfqpoint{1.819631in}{0.753684in}}{\pgfqpoint{1.815241in}{0.743085in}}{\pgfqpoint{1.815241in}{0.732035in}}%
\pgfpathcurveto{\pgfqpoint{1.815241in}{0.720985in}}{\pgfqpoint{1.819631in}{0.710386in}}{\pgfqpoint{1.827445in}{0.702572in}}%
\pgfpathcurveto{\pgfqpoint{1.835258in}{0.694759in}}{\pgfqpoint{1.845857in}{0.690368in}}{\pgfqpoint{1.856908in}{0.690368in}}%
\pgfpathclose%
\pgfusepath{stroke,fill}%
\end{pgfscope}%
\begin{pgfscope}%
\pgfpathrectangle{\pgfqpoint{0.772069in}{0.515123in}}{\pgfqpoint{1.937500in}{1.347500in}}%
\pgfusepath{clip}%
\pgfsetbuttcap%
\pgfsetroundjoin%
\definecolor{currentfill}{rgb}{1.000000,0.627451,0.478431}%
\pgfsetfillcolor{currentfill}%
\pgfsetlinewidth{1.003750pt}%
\definecolor{currentstroke}{rgb}{1.000000,0.627451,0.478431}%
\pgfsetstrokecolor{currentstroke}%
\pgfsetdash{}{0pt}%
\pgfpathmoveto{\pgfqpoint{1.971999in}{0.705963in}}%
\pgfpathcurveto{\pgfqpoint{1.983049in}{0.705963in}}{\pgfqpoint{1.993648in}{0.710353in}}{\pgfqpoint{2.001462in}{0.718166in}}%
\pgfpathcurveto{\pgfqpoint{2.009276in}{0.725980in}}{\pgfqpoint{2.013666in}{0.736579in}}{\pgfqpoint{2.013666in}{0.747629in}}%
\pgfpathcurveto{\pgfqpoint{2.013666in}{0.758679in}}{\pgfqpoint{2.009276in}{0.769278in}}{\pgfqpoint{2.001462in}{0.777092in}}%
\pgfpathcurveto{\pgfqpoint{1.993648in}{0.784906in}}{\pgfqpoint{1.983049in}{0.789296in}}{\pgfqpoint{1.971999in}{0.789296in}}%
\pgfpathcurveto{\pgfqpoint{1.960949in}{0.789296in}}{\pgfqpoint{1.950350in}{0.784906in}}{\pgfqpoint{1.942537in}{0.777092in}}%
\pgfpathcurveto{\pgfqpoint{1.934723in}{0.769278in}}{\pgfqpoint{1.930333in}{0.758679in}}{\pgfqpoint{1.930333in}{0.747629in}}%
\pgfpathcurveto{\pgfqpoint{1.930333in}{0.736579in}}{\pgfqpoint{1.934723in}{0.725980in}}{\pgfqpoint{1.942537in}{0.718166in}}%
\pgfpathcurveto{\pgfqpoint{1.950350in}{0.710353in}}{\pgfqpoint{1.960949in}{0.705963in}}{\pgfqpoint{1.971999in}{0.705963in}}%
\pgfpathclose%
\pgfusepath{stroke,fill}%
\end{pgfscope}%
\begin{pgfscope}%
\pgfpathrectangle{\pgfqpoint{0.772069in}{0.515123in}}{\pgfqpoint{1.937500in}{1.347500in}}%
\pgfusepath{clip}%
\pgfsetbuttcap%
\pgfsetroundjoin%
\definecolor{currentfill}{rgb}{0.941176,0.901961,0.549020}%
\pgfsetfillcolor{currentfill}%
\pgfsetlinewidth{1.003750pt}%
\definecolor{currentstroke}{rgb}{0.941176,0.901961,0.549020}%
\pgfsetstrokecolor{currentstroke}%
\pgfsetdash{}{0pt}%
\pgfpathmoveto{\pgfqpoint{0.936174in}{0.564707in}}%
\pgfpathcurveto{\pgfqpoint{0.947224in}{0.564707in}}{\pgfqpoint{0.957823in}{0.569097in}}{\pgfqpoint{0.965636in}{0.576910in}}%
\pgfpathcurveto{\pgfqpoint{0.973450in}{0.584724in}}{\pgfqpoint{0.977840in}{0.595323in}}{\pgfqpoint{0.977840in}{0.606373in}}%
\pgfpathcurveto{\pgfqpoint{0.977840in}{0.617423in}}{\pgfqpoint{0.973450in}{0.628022in}}{\pgfqpoint{0.965636in}{0.635836in}}%
\pgfpathcurveto{\pgfqpoint{0.957823in}{0.643650in}}{\pgfqpoint{0.947224in}{0.648040in}}{\pgfqpoint{0.936174in}{0.648040in}}%
\pgfpathcurveto{\pgfqpoint{0.925123in}{0.648040in}}{\pgfqpoint{0.914524in}{0.643650in}}{\pgfqpoint{0.906711in}{0.635836in}}%
\pgfpathcurveto{\pgfqpoint{0.898897in}{0.628022in}}{\pgfqpoint{0.894507in}{0.617423in}}{\pgfqpoint{0.894507in}{0.606373in}}%
\pgfpathcurveto{\pgfqpoint{0.894507in}{0.595323in}}{\pgfqpoint{0.898897in}{0.584724in}}{\pgfqpoint{0.906711in}{0.576910in}}%
\pgfpathcurveto{\pgfqpoint{0.914524in}{0.569097in}}{\pgfqpoint{0.925123in}{0.564707in}}{\pgfqpoint{0.936174in}{0.564707in}}%
\pgfpathclose%
\pgfusepath{stroke,fill}%
\end{pgfscope}%
\begin{pgfscope}%
\pgfpathrectangle{\pgfqpoint{0.772069in}{0.515123in}}{\pgfqpoint{1.937500in}{1.347500in}}%
\pgfusepath{clip}%
\pgfsetbuttcap%
\pgfsetroundjoin%
\definecolor{currentfill}{rgb}{0.941176,0.901961,0.549020}%
\pgfsetfillcolor{currentfill}%
\pgfsetlinewidth{1.003750pt}%
\definecolor{currentstroke}{rgb}{0.941176,0.901961,0.549020}%
\pgfsetstrokecolor{currentstroke}%
\pgfsetdash{}{0pt}%
\pgfpathmoveto{\pgfqpoint{1.051265in}{0.576321in}}%
\pgfpathcurveto{\pgfqpoint{1.062315in}{0.576321in}}{\pgfqpoint{1.072914in}{0.580711in}}{\pgfqpoint{1.080728in}{0.588525in}}%
\pgfpathcurveto{\pgfqpoint{1.088542in}{0.596338in}}{\pgfqpoint{1.092932in}{0.606937in}}{\pgfqpoint{1.092932in}{0.617987in}}%
\pgfpathcurveto{\pgfqpoint{1.092932in}{0.629038in}}{\pgfqpoint{1.088542in}{0.639637in}}{\pgfqpoint{1.080728in}{0.647450in}}%
\pgfpathcurveto{\pgfqpoint{1.072914in}{0.655264in}}{\pgfqpoint{1.062315in}{0.659654in}}{\pgfqpoint{1.051265in}{0.659654in}}%
\pgfpathcurveto{\pgfqpoint{1.040215in}{0.659654in}}{\pgfqpoint{1.029616in}{0.655264in}}{\pgfqpoint{1.021803in}{0.647450in}}%
\pgfpathcurveto{\pgfqpoint{1.013989in}{0.639637in}}{\pgfqpoint{1.009599in}{0.629038in}}{\pgfqpoint{1.009599in}{0.617987in}}%
\pgfpathcurveto{\pgfqpoint{1.009599in}{0.606937in}}{\pgfqpoint{1.013989in}{0.596338in}}{\pgfqpoint{1.021803in}{0.588525in}}%
\pgfpathcurveto{\pgfqpoint{1.029616in}{0.580711in}}{\pgfqpoint{1.040215in}{0.576321in}}{\pgfqpoint{1.051265in}{0.576321in}}%
\pgfpathclose%
\pgfusepath{stroke,fill}%
\end{pgfscope}%
\begin{pgfscope}%
\pgfpathrectangle{\pgfqpoint{0.772069in}{0.515123in}}{\pgfqpoint{1.937500in}{1.347500in}}%
\pgfusepath{clip}%
\pgfsetbuttcap%
\pgfsetroundjoin%
\definecolor{currentfill}{rgb}{0.941176,0.901961,0.549020}%
\pgfsetfillcolor{currentfill}%
\pgfsetlinewidth{1.003750pt}%
\definecolor{currentstroke}{rgb}{0.941176,0.901961,0.549020}%
\pgfsetstrokecolor{currentstroke}%
\pgfsetdash{}{0pt}%
\pgfpathmoveto{\pgfqpoint{1.166357in}{0.588889in}}%
\pgfpathcurveto{\pgfqpoint{1.177407in}{0.588889in}}{\pgfqpoint{1.188006in}{0.593280in}}{\pgfqpoint{1.195820in}{0.601093in}}%
\pgfpathcurveto{\pgfqpoint{1.203633in}{0.608907in}}{\pgfqpoint{1.208024in}{0.619506in}}{\pgfqpoint{1.208024in}{0.630556in}}%
\pgfpathcurveto{\pgfqpoint{1.208024in}{0.641606in}}{\pgfqpoint{1.203633in}{0.652205in}}{\pgfqpoint{1.195820in}{0.660019in}}%
\pgfpathcurveto{\pgfqpoint{1.188006in}{0.667832in}}{\pgfqpoint{1.177407in}{0.672223in}}{\pgfqpoint{1.166357in}{0.672223in}}%
\pgfpathcurveto{\pgfqpoint{1.155307in}{0.672223in}}{\pgfqpoint{1.144708in}{0.667832in}}{\pgfqpoint{1.136894in}{0.660019in}}%
\pgfpathcurveto{\pgfqpoint{1.129081in}{0.652205in}}{\pgfqpoint{1.124690in}{0.641606in}}{\pgfqpoint{1.124690in}{0.630556in}}%
\pgfpathcurveto{\pgfqpoint{1.124690in}{0.619506in}}{\pgfqpoint{1.129081in}{0.608907in}}{\pgfqpoint{1.136894in}{0.601093in}}%
\pgfpathcurveto{\pgfqpoint{1.144708in}{0.593280in}}{\pgfqpoint{1.155307in}{0.588889in}}{\pgfqpoint{1.166357in}{0.588889in}}%
\pgfpathclose%
\pgfusepath{stroke,fill}%
\end{pgfscope}%
\begin{pgfscope}%
\pgfpathrectangle{\pgfqpoint{0.772069in}{0.515123in}}{\pgfqpoint{1.937500in}{1.347500in}}%
\pgfusepath{clip}%
\pgfsetbuttcap%
\pgfsetroundjoin%
\definecolor{currentfill}{rgb}{0.941176,0.901961,0.549020}%
\pgfsetfillcolor{currentfill}%
\pgfsetlinewidth{1.003750pt}%
\definecolor{currentstroke}{rgb}{0.941176,0.901961,0.549020}%
\pgfsetstrokecolor{currentstroke}%
\pgfsetdash{}{0pt}%
\pgfpathmoveto{\pgfqpoint{1.281449in}{0.601225in}}%
\pgfpathcurveto{\pgfqpoint{1.292499in}{0.601225in}}{\pgfqpoint{1.303098in}{0.605615in}}{\pgfqpoint{1.310912in}{0.613429in}}%
\pgfpathcurveto{\pgfqpoint{1.318725in}{0.621243in}}{\pgfqpoint{1.323115in}{0.631842in}}{\pgfqpoint{1.323115in}{0.642892in}}%
\pgfpathcurveto{\pgfqpoint{1.323115in}{0.653942in}}{\pgfqpoint{1.318725in}{0.664541in}}{\pgfqpoint{1.310912in}{0.672354in}}%
\pgfpathcurveto{\pgfqpoint{1.303098in}{0.680168in}}{\pgfqpoint{1.292499in}{0.684558in}}{\pgfqpoint{1.281449in}{0.684558in}}%
\pgfpathcurveto{\pgfqpoint{1.270399in}{0.684558in}}{\pgfqpoint{1.259800in}{0.680168in}}{\pgfqpoint{1.251986in}{0.672354in}}%
\pgfpathcurveto{\pgfqpoint{1.244172in}{0.664541in}}{\pgfqpoint{1.239782in}{0.653942in}}{\pgfqpoint{1.239782in}{0.642892in}}%
\pgfpathcurveto{\pgfqpoint{1.239782in}{0.631842in}}{\pgfqpoint{1.244172in}{0.621243in}}{\pgfqpoint{1.251986in}{0.613429in}}%
\pgfpathcurveto{\pgfqpoint{1.259800in}{0.605615in}}{\pgfqpoint{1.270399in}{0.601225in}}{\pgfqpoint{1.281449in}{0.601225in}}%
\pgfpathclose%
\pgfusepath{stroke,fill}%
\end{pgfscope}%
\begin{pgfscope}%
\pgfpathrectangle{\pgfqpoint{0.772069in}{0.515123in}}{\pgfqpoint{1.937500in}{1.347500in}}%
\pgfusepath{clip}%
\pgfsetbuttcap%
\pgfsetroundjoin%
\definecolor{currentfill}{rgb}{0.941176,0.901961,0.549020}%
\pgfsetfillcolor{currentfill}%
\pgfsetlinewidth{1.003750pt}%
\definecolor{currentstroke}{rgb}{0.941176,0.901961,0.549020}%
\pgfsetstrokecolor{currentstroke}%
\pgfsetdash{}{0pt}%
\pgfpathmoveto{\pgfqpoint{1.396541in}{0.613561in}}%
\pgfpathcurveto{\pgfqpoint{1.407591in}{0.613561in}}{\pgfqpoint{1.418190in}{0.617951in}}{\pgfqpoint{1.426003in}{0.625765in}}%
\pgfpathcurveto{\pgfqpoint{1.433817in}{0.633578in}}{\pgfqpoint{1.438207in}{0.644177in}}{\pgfqpoint{1.438207in}{0.655227in}}%
\pgfpathcurveto{\pgfqpoint{1.438207in}{0.666278in}}{\pgfqpoint{1.433817in}{0.676877in}}{\pgfqpoint{1.426003in}{0.684690in}}%
\pgfpathcurveto{\pgfqpoint{1.418190in}{0.692504in}}{\pgfqpoint{1.407591in}{0.696894in}}{\pgfqpoint{1.396541in}{0.696894in}}%
\pgfpathcurveto{\pgfqpoint{1.385490in}{0.696894in}}{\pgfqpoint{1.374891in}{0.692504in}}{\pgfqpoint{1.367078in}{0.684690in}}%
\pgfpathcurveto{\pgfqpoint{1.359264in}{0.676877in}}{\pgfqpoint{1.354874in}{0.666278in}}{\pgfqpoint{1.354874in}{0.655227in}}%
\pgfpathcurveto{\pgfqpoint{1.354874in}{0.644177in}}{\pgfqpoint{1.359264in}{0.633578in}}{\pgfqpoint{1.367078in}{0.625765in}}%
\pgfpathcurveto{\pgfqpoint{1.374891in}{0.617951in}}{\pgfqpoint{1.385490in}{0.613561in}}{\pgfqpoint{1.396541in}{0.613561in}}%
\pgfpathclose%
\pgfusepath{stroke,fill}%
\end{pgfscope}%
\begin{pgfscope}%
\pgfpathrectangle{\pgfqpoint{0.772069in}{0.515123in}}{\pgfqpoint{1.937500in}{1.347500in}}%
\pgfusepath{clip}%
\pgfsetbuttcap%
\pgfsetroundjoin%
\definecolor{currentfill}{rgb}{0.941176,0.901961,0.549020}%
\pgfsetfillcolor{currentfill}%
\pgfsetlinewidth{1.003750pt}%
\definecolor{currentstroke}{rgb}{0.941176,0.901961,0.549020}%
\pgfsetstrokecolor{currentstroke}%
\pgfsetdash{}{0pt}%
\pgfpathmoveto{\pgfqpoint{1.511632in}{0.625315in}}%
\pgfpathcurveto{\pgfqpoint{1.522682in}{0.625315in}}{\pgfqpoint{1.533281in}{0.629705in}}{\pgfqpoint{1.541095in}{0.637519in}}%
\pgfpathcurveto{\pgfqpoint{1.548909in}{0.645332in}}{\pgfqpoint{1.553299in}{0.655931in}}{\pgfqpoint{1.553299in}{0.666981in}}%
\pgfpathcurveto{\pgfqpoint{1.553299in}{0.678031in}}{\pgfqpoint{1.548909in}{0.688631in}}{\pgfqpoint{1.541095in}{0.696444in}}%
\pgfpathcurveto{\pgfqpoint{1.533281in}{0.704258in}}{\pgfqpoint{1.522682in}{0.708648in}}{\pgfqpoint{1.511632in}{0.708648in}}%
\pgfpathcurveto{\pgfqpoint{1.500582in}{0.708648in}}{\pgfqpoint{1.489983in}{0.704258in}}{\pgfqpoint{1.482170in}{0.696444in}}%
\pgfpathcurveto{\pgfqpoint{1.474356in}{0.688631in}}{\pgfqpoint{1.469966in}{0.678031in}}{\pgfqpoint{1.469966in}{0.666981in}}%
\pgfpathcurveto{\pgfqpoint{1.469966in}{0.655931in}}{\pgfqpoint{1.474356in}{0.645332in}}{\pgfqpoint{1.482170in}{0.637519in}}%
\pgfpathcurveto{\pgfqpoint{1.489983in}{0.629705in}}{\pgfqpoint{1.500582in}{0.625315in}}{\pgfqpoint{1.511632in}{0.625315in}}%
\pgfpathclose%
\pgfusepath{stroke,fill}%
\end{pgfscope}%
\begin{pgfscope}%
\pgfpathrectangle{\pgfqpoint{0.772069in}{0.515123in}}{\pgfqpoint{1.937500in}{1.347500in}}%
\pgfusepath{clip}%
\pgfsetbuttcap%
\pgfsetroundjoin%
\definecolor{currentfill}{rgb}{0.941176,0.901961,0.549020}%
\pgfsetfillcolor{currentfill}%
\pgfsetlinewidth{1.003750pt}%
\definecolor{currentstroke}{rgb}{0.941176,0.901961,0.549020}%
\pgfsetstrokecolor{currentstroke}%
\pgfsetdash{}{0pt}%
\pgfpathmoveto{\pgfqpoint{1.626724in}{0.644284in}}%
\pgfpathcurveto{\pgfqpoint{1.637774in}{0.644284in}}{\pgfqpoint{1.648373in}{0.648674in}}{\pgfqpoint{1.656187in}{0.656488in}}%
\pgfpathcurveto{\pgfqpoint{1.664000in}{0.664301in}}{\pgfqpoint{1.668391in}{0.674900in}}{\pgfqpoint{1.668391in}{0.685950in}}%
\pgfpathcurveto{\pgfqpoint{1.668391in}{0.697001in}}{\pgfqpoint{1.664000in}{0.707600in}}{\pgfqpoint{1.656187in}{0.715413in}}%
\pgfpathcurveto{\pgfqpoint{1.648373in}{0.723227in}}{\pgfqpoint{1.637774in}{0.727617in}}{\pgfqpoint{1.626724in}{0.727617in}}%
\pgfpathcurveto{\pgfqpoint{1.615674in}{0.727617in}}{\pgfqpoint{1.605075in}{0.723227in}}{\pgfqpoint{1.597261in}{0.715413in}}%
\pgfpathcurveto{\pgfqpoint{1.589448in}{0.707600in}}{\pgfqpoint{1.585057in}{0.697001in}}{\pgfqpoint{1.585057in}{0.685950in}}%
\pgfpathcurveto{\pgfqpoint{1.585057in}{0.674900in}}{\pgfqpoint{1.589448in}{0.664301in}}{\pgfqpoint{1.597261in}{0.656488in}}%
\pgfpathcurveto{\pgfqpoint{1.605075in}{0.648674in}}{\pgfqpoint{1.615674in}{0.644284in}}{\pgfqpoint{1.626724in}{0.644284in}}%
\pgfpathclose%
\pgfusepath{stroke,fill}%
\end{pgfscope}%
\begin{pgfscope}%
\pgfpathrectangle{\pgfqpoint{0.772069in}{0.515123in}}{\pgfqpoint{1.937500in}{1.347500in}}%
\pgfusepath{clip}%
\pgfsetbuttcap%
\pgfsetroundjoin%
\definecolor{currentfill}{rgb}{0.941176,0.901961,0.549020}%
\pgfsetfillcolor{currentfill}%
\pgfsetlinewidth{1.003750pt}%
\definecolor{currentstroke}{rgb}{0.941176,0.901961,0.549020}%
\pgfsetstrokecolor{currentstroke}%
\pgfsetdash{}{0pt}%
\pgfpathmoveto{\pgfqpoint{1.741816in}{0.650335in}}%
\pgfpathcurveto{\pgfqpoint{1.752866in}{0.650335in}}{\pgfqpoint{1.763465in}{0.654726in}}{\pgfqpoint{1.771279in}{0.662539in}}%
\pgfpathcurveto{\pgfqpoint{1.779092in}{0.670353in}}{\pgfqpoint{1.783482in}{0.680952in}}{\pgfqpoint{1.783482in}{0.692002in}}%
\pgfpathcurveto{\pgfqpoint{1.783482in}{0.703052in}}{\pgfqpoint{1.779092in}{0.713651in}}{\pgfqpoint{1.771279in}{0.721465in}}%
\pgfpathcurveto{\pgfqpoint{1.763465in}{0.729278in}}{\pgfqpoint{1.752866in}{0.733669in}}{\pgfqpoint{1.741816in}{0.733669in}}%
\pgfpathcurveto{\pgfqpoint{1.730766in}{0.733669in}}{\pgfqpoint{1.720167in}{0.729278in}}{\pgfqpoint{1.712353in}{0.721465in}}%
\pgfpathcurveto{\pgfqpoint{1.704539in}{0.713651in}}{\pgfqpoint{1.700149in}{0.703052in}}{\pgfqpoint{1.700149in}{0.692002in}}%
\pgfpathcurveto{\pgfqpoint{1.700149in}{0.680952in}}{\pgfqpoint{1.704539in}{0.670353in}}{\pgfqpoint{1.712353in}{0.662539in}}%
\pgfpathcurveto{\pgfqpoint{1.720167in}{0.654726in}}{\pgfqpoint{1.730766in}{0.650335in}}{\pgfqpoint{1.741816in}{0.650335in}}%
\pgfpathclose%
\pgfusepath{stroke,fill}%
\end{pgfscope}%
\begin{pgfscope}%
\pgfpathrectangle{\pgfqpoint{0.772069in}{0.515123in}}{\pgfqpoint{1.937500in}{1.347500in}}%
\pgfusepath{clip}%
\pgfsetbuttcap%
\pgfsetroundjoin%
\definecolor{currentfill}{rgb}{0.941176,0.901961,0.549020}%
\pgfsetfillcolor{currentfill}%
\pgfsetlinewidth{1.003750pt}%
\definecolor{currentstroke}{rgb}{0.941176,0.901961,0.549020}%
\pgfsetstrokecolor{currentstroke}%
\pgfsetdash{}{0pt}%
\pgfpathmoveto{\pgfqpoint{1.856908in}{0.662322in}}%
\pgfpathcurveto{\pgfqpoint{1.867958in}{0.662322in}}{\pgfqpoint{1.878557in}{0.666712in}}{\pgfqpoint{1.886370in}{0.674526in}}%
\pgfpathcurveto{\pgfqpoint{1.894184in}{0.682339in}}{\pgfqpoint{1.898574in}{0.692938in}}{\pgfqpoint{1.898574in}{0.703989in}}%
\pgfpathcurveto{\pgfqpoint{1.898574in}{0.715039in}}{\pgfqpoint{1.894184in}{0.725638in}}{\pgfqpoint{1.886370in}{0.733451in}}%
\pgfpathcurveto{\pgfqpoint{1.878557in}{0.741265in}}{\pgfqpoint{1.867958in}{0.745655in}}{\pgfqpoint{1.856908in}{0.745655in}}%
\pgfpathcurveto{\pgfqpoint{1.845857in}{0.745655in}}{\pgfqpoint{1.835258in}{0.741265in}}{\pgfqpoint{1.827445in}{0.733451in}}%
\pgfpathcurveto{\pgfqpoint{1.819631in}{0.725638in}}{\pgfqpoint{1.815241in}{0.715039in}}{\pgfqpoint{1.815241in}{0.703989in}}%
\pgfpathcurveto{\pgfqpoint{1.815241in}{0.692938in}}{\pgfqpoint{1.819631in}{0.682339in}}{\pgfqpoint{1.827445in}{0.674526in}}%
\pgfpathcurveto{\pgfqpoint{1.835258in}{0.666712in}}{\pgfqpoint{1.845857in}{0.662322in}}{\pgfqpoint{1.856908in}{0.662322in}}%
\pgfpathclose%
\pgfusepath{stroke,fill}%
\end{pgfscope}%
\begin{pgfscope}%
\pgfpathrectangle{\pgfqpoint{0.772069in}{0.515123in}}{\pgfqpoint{1.937500in}{1.347500in}}%
\pgfusepath{clip}%
\pgfsetbuttcap%
\pgfsetroundjoin%
\definecolor{currentfill}{rgb}{0.941176,0.901961,0.549020}%
\pgfsetfillcolor{currentfill}%
\pgfsetlinewidth{1.003750pt}%
\definecolor{currentstroke}{rgb}{0.941176,0.901961,0.549020}%
\pgfsetstrokecolor{currentstroke}%
\pgfsetdash{}{0pt}%
\pgfpathmoveto{\pgfqpoint{1.971999in}{0.674658in}}%
\pgfpathcurveto{\pgfqpoint{1.983049in}{0.674658in}}{\pgfqpoint{1.993648in}{0.679048in}}{\pgfqpoint{2.001462in}{0.686862in}}%
\pgfpathcurveto{\pgfqpoint{2.009276in}{0.694675in}}{\pgfqpoint{2.013666in}{0.705274in}}{\pgfqpoint{2.013666in}{0.716324in}}%
\pgfpathcurveto{\pgfqpoint{2.013666in}{0.727374in}}{\pgfqpoint{2.009276in}{0.737974in}}{\pgfqpoint{2.001462in}{0.745787in}}%
\pgfpathcurveto{\pgfqpoint{1.993648in}{0.753601in}}{\pgfqpoint{1.983049in}{0.757991in}}{\pgfqpoint{1.971999in}{0.757991in}}%
\pgfpathcurveto{\pgfqpoint{1.960949in}{0.757991in}}{\pgfqpoint{1.950350in}{0.753601in}}{\pgfqpoint{1.942537in}{0.745787in}}%
\pgfpathcurveto{\pgfqpoint{1.934723in}{0.737974in}}{\pgfqpoint{1.930333in}{0.727374in}}{\pgfqpoint{1.930333in}{0.716324in}}%
\pgfpathcurveto{\pgfqpoint{1.930333in}{0.705274in}}{\pgfqpoint{1.934723in}{0.694675in}}{\pgfqpoint{1.942537in}{0.686862in}}%
\pgfpathcurveto{\pgfqpoint{1.950350in}{0.679048in}}{\pgfqpoint{1.960949in}{0.674658in}}{\pgfqpoint{1.971999in}{0.674658in}}%
\pgfpathclose%
\pgfusepath{stroke,fill}%
\end{pgfscope}%
\begin{pgfscope}%
\pgfpathrectangle{\pgfqpoint{0.772069in}{0.515123in}}{\pgfqpoint{1.937500in}{1.347500in}}%
\pgfusepath{clip}%
\pgfsetbuttcap%
\pgfsetroundjoin%
\definecolor{currentfill}{rgb}{0.564706,0.933333,0.564706}%
\pgfsetfillcolor{currentfill}%
\pgfsetlinewidth{1.003750pt}%
\definecolor{currentstroke}{rgb}{0.564706,0.933333,0.564706}%
\pgfsetstrokecolor{currentstroke}%
\pgfsetdash{}{0pt}%
\pgfpathmoveto{\pgfqpoint{0.936174in}{0.641491in}}%
\pgfpathcurveto{\pgfqpoint{0.947224in}{0.641491in}}{\pgfqpoint{0.957823in}{0.645881in}}{\pgfqpoint{0.965636in}{0.653695in}}%
\pgfpathcurveto{\pgfqpoint{0.973450in}{0.661508in}}{\pgfqpoint{0.977840in}{0.672107in}}{\pgfqpoint{0.977840in}{0.683157in}}%
\pgfpathcurveto{\pgfqpoint{0.977840in}{0.694208in}}{\pgfqpoint{0.973450in}{0.704807in}}{\pgfqpoint{0.965636in}{0.712620in}}%
\pgfpathcurveto{\pgfqpoint{0.957823in}{0.720434in}}{\pgfqpoint{0.947224in}{0.724824in}}{\pgfqpoint{0.936174in}{0.724824in}}%
\pgfpathcurveto{\pgfqpoint{0.925123in}{0.724824in}}{\pgfqpoint{0.914524in}{0.720434in}}{\pgfqpoint{0.906711in}{0.712620in}}%
\pgfpathcurveto{\pgfqpoint{0.898897in}{0.704807in}}{\pgfqpoint{0.894507in}{0.694208in}}{\pgfqpoint{0.894507in}{0.683157in}}%
\pgfpathcurveto{\pgfqpoint{0.894507in}{0.672107in}}{\pgfqpoint{0.898897in}{0.661508in}}{\pgfqpoint{0.906711in}{0.653695in}}%
\pgfpathcurveto{\pgfqpoint{0.914524in}{0.645881in}}{\pgfqpoint{0.925123in}{0.641491in}}{\pgfqpoint{0.936174in}{0.641491in}}%
\pgfpathclose%
\pgfusepath{stroke,fill}%
\end{pgfscope}%
\begin{pgfscope}%
\pgfpathrectangle{\pgfqpoint{0.772069in}{0.515123in}}{\pgfqpoint{1.937500in}{1.347500in}}%
\pgfusepath{clip}%
\pgfsetbuttcap%
\pgfsetroundjoin%
\definecolor{currentfill}{rgb}{0.564706,0.933333,0.564706}%
\pgfsetfillcolor{currentfill}%
\pgfsetlinewidth{1.003750pt}%
\definecolor{currentstroke}{rgb}{0.564706,0.933333,0.564706}%
\pgfsetstrokecolor{currentstroke}%
\pgfsetdash{}{0pt}%
\pgfpathmoveto{\pgfqpoint{1.051265in}{0.725630in}}%
\pgfpathcurveto{\pgfqpoint{1.062315in}{0.725630in}}{\pgfqpoint{1.072914in}{0.730020in}}{\pgfqpoint{1.080728in}{0.737834in}}%
\pgfpathcurveto{\pgfqpoint{1.088542in}{0.745647in}}{\pgfqpoint{1.092932in}{0.756246in}}{\pgfqpoint{1.092932in}{0.767297in}}%
\pgfpathcurveto{\pgfqpoint{1.092932in}{0.778347in}}{\pgfqpoint{1.088542in}{0.788946in}}{\pgfqpoint{1.080728in}{0.796759in}}%
\pgfpathcurveto{\pgfqpoint{1.072914in}{0.804573in}}{\pgfqpoint{1.062315in}{0.808963in}}{\pgfqpoint{1.051265in}{0.808963in}}%
\pgfpathcurveto{\pgfqpoint{1.040215in}{0.808963in}}{\pgfqpoint{1.029616in}{0.804573in}}{\pgfqpoint{1.021803in}{0.796759in}}%
\pgfpathcurveto{\pgfqpoint{1.013989in}{0.788946in}}{\pgfqpoint{1.009599in}{0.778347in}}{\pgfqpoint{1.009599in}{0.767297in}}%
\pgfpathcurveto{\pgfqpoint{1.009599in}{0.756246in}}{\pgfqpoint{1.013989in}{0.745647in}}{\pgfqpoint{1.021803in}{0.737834in}}%
\pgfpathcurveto{\pgfqpoint{1.029616in}{0.730020in}}{\pgfqpoint{1.040215in}{0.725630in}}{\pgfqpoint{1.051265in}{0.725630in}}%
\pgfpathclose%
\pgfusepath{stroke,fill}%
\end{pgfscope}%
\begin{pgfscope}%
\pgfpathrectangle{\pgfqpoint{0.772069in}{0.515123in}}{\pgfqpoint{1.937500in}{1.347500in}}%
\pgfusepath{clip}%
\pgfsetbuttcap%
\pgfsetroundjoin%
\definecolor{currentfill}{rgb}{0.564706,0.933333,0.564706}%
\pgfsetfillcolor{currentfill}%
\pgfsetlinewidth{1.003750pt}%
\definecolor{currentstroke}{rgb}{0.564706,0.933333,0.564706}%
\pgfsetstrokecolor{currentstroke}%
\pgfsetdash{}{0pt}%
\pgfpathmoveto{\pgfqpoint{1.166357in}{0.813842in}}%
\pgfpathcurveto{\pgfqpoint{1.177407in}{0.813842in}}{\pgfqpoint{1.188006in}{0.818233in}}{\pgfqpoint{1.195820in}{0.826046in}}%
\pgfpathcurveto{\pgfqpoint{1.203633in}{0.833860in}}{\pgfqpoint{1.208024in}{0.844459in}}{\pgfqpoint{1.208024in}{0.855509in}}%
\pgfpathcurveto{\pgfqpoint{1.208024in}{0.866559in}}{\pgfqpoint{1.203633in}{0.877158in}}{\pgfqpoint{1.195820in}{0.884972in}}%
\pgfpathcurveto{\pgfqpoint{1.188006in}{0.892785in}}{\pgfqpoint{1.177407in}{0.897176in}}{\pgfqpoint{1.166357in}{0.897176in}}%
\pgfpathcurveto{\pgfqpoint{1.155307in}{0.897176in}}{\pgfqpoint{1.144708in}{0.892785in}}{\pgfqpoint{1.136894in}{0.884972in}}%
\pgfpathcurveto{\pgfqpoint{1.129081in}{0.877158in}}{\pgfqpoint{1.124690in}{0.866559in}}{\pgfqpoint{1.124690in}{0.855509in}}%
\pgfpathcurveto{\pgfqpoint{1.124690in}{0.844459in}}{\pgfqpoint{1.129081in}{0.833860in}}{\pgfqpoint{1.136894in}{0.826046in}}%
\pgfpathcurveto{\pgfqpoint{1.144708in}{0.818233in}}{\pgfqpoint{1.155307in}{0.813842in}}{\pgfqpoint{1.166357in}{0.813842in}}%
\pgfpathclose%
\pgfusepath{stroke,fill}%
\end{pgfscope}%
\begin{pgfscope}%
\pgfpathrectangle{\pgfqpoint{0.772069in}{0.515123in}}{\pgfqpoint{1.937500in}{1.347500in}}%
\pgfusepath{clip}%
\pgfsetbuttcap%
\pgfsetroundjoin%
\definecolor{currentfill}{rgb}{0.564706,0.933333,0.564706}%
\pgfsetfillcolor{currentfill}%
\pgfsetlinewidth{1.003750pt}%
\definecolor{currentstroke}{rgb}{0.564706,0.933333,0.564706}%
\pgfsetstrokecolor{currentstroke}%
\pgfsetdash{}{0pt}%
\pgfpathmoveto{\pgfqpoint{1.281449in}{0.903451in}}%
\pgfpathcurveto{\pgfqpoint{1.292499in}{0.903451in}}{\pgfqpoint{1.303098in}{0.907841in}}{\pgfqpoint{1.310912in}{0.915655in}}%
\pgfpathcurveto{\pgfqpoint{1.318725in}{0.923469in}}{\pgfqpoint{1.323115in}{0.934068in}}{\pgfqpoint{1.323115in}{0.945118in}}%
\pgfpathcurveto{\pgfqpoint{1.323115in}{0.956168in}}{\pgfqpoint{1.318725in}{0.966767in}}{\pgfqpoint{1.310912in}{0.974580in}}%
\pgfpathcurveto{\pgfqpoint{1.303098in}{0.982394in}}{\pgfqpoint{1.292499in}{0.986784in}}{\pgfqpoint{1.281449in}{0.986784in}}%
\pgfpathcurveto{\pgfqpoint{1.270399in}{0.986784in}}{\pgfqpoint{1.259800in}{0.982394in}}{\pgfqpoint{1.251986in}{0.974580in}}%
\pgfpathcurveto{\pgfqpoint{1.244172in}{0.966767in}}{\pgfqpoint{1.239782in}{0.956168in}}{\pgfqpoint{1.239782in}{0.945118in}}%
\pgfpathcurveto{\pgfqpoint{1.239782in}{0.934068in}}{\pgfqpoint{1.244172in}{0.923469in}}{\pgfqpoint{1.251986in}{0.915655in}}%
\pgfpathcurveto{\pgfqpoint{1.259800in}{0.907841in}}{\pgfqpoint{1.270399in}{0.903451in}}{\pgfqpoint{1.281449in}{0.903451in}}%
\pgfpathclose%
\pgfusepath{stroke,fill}%
\end{pgfscope}%
\begin{pgfscope}%
\pgfpathrectangle{\pgfqpoint{0.772069in}{0.515123in}}{\pgfqpoint{1.937500in}{1.347500in}}%
\pgfusepath{clip}%
\pgfsetbuttcap%
\pgfsetroundjoin%
\definecolor{currentfill}{rgb}{0.564706,0.933333,0.564706}%
\pgfsetfillcolor{currentfill}%
\pgfsetlinewidth{1.003750pt}%
\definecolor{currentstroke}{rgb}{0.564706,0.933333,0.564706}%
\pgfsetstrokecolor{currentstroke}%
\pgfsetdash{}{0pt}%
\pgfpathmoveto{\pgfqpoint{1.396541in}{0.991896in}}%
\pgfpathcurveto{\pgfqpoint{1.407591in}{0.991896in}}{\pgfqpoint{1.418190in}{0.996286in}}{\pgfqpoint{1.426003in}{1.004100in}}%
\pgfpathcurveto{\pgfqpoint{1.433817in}{1.011914in}}{\pgfqpoint{1.438207in}{1.022513in}}{\pgfqpoint{1.438207in}{1.033563in}}%
\pgfpathcurveto{\pgfqpoint{1.438207in}{1.044613in}}{\pgfqpoint{1.433817in}{1.055212in}}{\pgfqpoint{1.426003in}{1.063026in}}%
\pgfpathcurveto{\pgfqpoint{1.418190in}{1.070839in}}{\pgfqpoint{1.407591in}{1.075229in}}{\pgfqpoint{1.396541in}{1.075229in}}%
\pgfpathcurveto{\pgfqpoint{1.385490in}{1.075229in}}{\pgfqpoint{1.374891in}{1.070839in}}{\pgfqpoint{1.367078in}{1.063026in}}%
\pgfpathcurveto{\pgfqpoint{1.359264in}{1.055212in}}{\pgfqpoint{1.354874in}{1.044613in}}{\pgfqpoint{1.354874in}{1.033563in}}%
\pgfpathcurveto{\pgfqpoint{1.354874in}{1.022513in}}{\pgfqpoint{1.359264in}{1.011914in}}{\pgfqpoint{1.367078in}{1.004100in}}%
\pgfpathcurveto{\pgfqpoint{1.374891in}{0.996286in}}{\pgfqpoint{1.385490in}{0.991896in}}{\pgfqpoint{1.396541in}{0.991896in}}%
\pgfpathclose%
\pgfusepath{stroke,fill}%
\end{pgfscope}%
\begin{pgfscope}%
\pgfpathrectangle{\pgfqpoint{0.772069in}{0.515123in}}{\pgfqpoint{1.937500in}{1.347500in}}%
\pgfusepath{clip}%
\pgfsetbuttcap%
\pgfsetroundjoin%
\definecolor{currentfill}{rgb}{0.564706,0.933333,0.564706}%
\pgfsetfillcolor{currentfill}%
\pgfsetlinewidth{1.003750pt}%
\definecolor{currentstroke}{rgb}{0.564706,0.933333,0.564706}%
\pgfsetstrokecolor{currentstroke}%
\pgfsetdash{}{0pt}%
\pgfpathmoveto{\pgfqpoint{1.511632in}{1.075686in}}%
\pgfpathcurveto{\pgfqpoint{1.522682in}{1.075686in}}{\pgfqpoint{1.533281in}{1.080076in}}{\pgfqpoint{1.541095in}{1.087890in}}%
\pgfpathcurveto{\pgfqpoint{1.548909in}{1.095704in}}{\pgfqpoint{1.553299in}{1.106303in}}{\pgfqpoint{1.553299in}{1.117353in}}%
\pgfpathcurveto{\pgfqpoint{1.553299in}{1.128403in}}{\pgfqpoint{1.548909in}{1.139002in}}{\pgfqpoint{1.541095in}{1.146816in}}%
\pgfpathcurveto{\pgfqpoint{1.533281in}{1.154629in}}{\pgfqpoint{1.522682in}{1.159019in}}{\pgfqpoint{1.511632in}{1.159019in}}%
\pgfpathcurveto{\pgfqpoint{1.500582in}{1.159019in}}{\pgfqpoint{1.489983in}{1.154629in}}{\pgfqpoint{1.482170in}{1.146816in}}%
\pgfpathcurveto{\pgfqpoint{1.474356in}{1.139002in}}{\pgfqpoint{1.469966in}{1.128403in}}{\pgfqpoint{1.469966in}{1.117353in}}%
\pgfpathcurveto{\pgfqpoint{1.469966in}{1.106303in}}{\pgfqpoint{1.474356in}{1.095704in}}{\pgfqpoint{1.482170in}{1.087890in}}%
\pgfpathcurveto{\pgfqpoint{1.489983in}{1.080076in}}{\pgfqpoint{1.500582in}{1.075686in}}{\pgfqpoint{1.511632in}{1.075686in}}%
\pgfpathclose%
\pgfusepath{stroke,fill}%
\end{pgfscope}%
\begin{pgfscope}%
\pgfpathrectangle{\pgfqpoint{0.772069in}{0.515123in}}{\pgfqpoint{1.937500in}{1.347500in}}%
\pgfusepath{clip}%
\pgfsetbuttcap%
\pgfsetroundjoin%
\definecolor{currentfill}{rgb}{0.564706,0.933333,0.564706}%
\pgfsetfillcolor{currentfill}%
\pgfsetlinewidth{1.003750pt}%
\definecolor{currentstroke}{rgb}{0.564706,0.933333,0.564706}%
\pgfsetstrokecolor{currentstroke}%
\pgfsetdash{}{0pt}%
\pgfpathmoveto{\pgfqpoint{1.626724in}{1.166459in}}%
\pgfpathcurveto{\pgfqpoint{1.637774in}{1.166459in}}{\pgfqpoint{1.648373in}{1.170849in}}{\pgfqpoint{1.656187in}{1.178663in}}%
\pgfpathcurveto{\pgfqpoint{1.664000in}{1.186476in}}{\pgfqpoint{1.668391in}{1.197075in}}{\pgfqpoint{1.668391in}{1.208125in}}%
\pgfpathcurveto{\pgfqpoint{1.668391in}{1.219175in}}{\pgfqpoint{1.664000in}{1.229774in}}{\pgfqpoint{1.656187in}{1.237588in}}%
\pgfpathcurveto{\pgfqpoint{1.648373in}{1.245402in}}{\pgfqpoint{1.637774in}{1.249792in}}{\pgfqpoint{1.626724in}{1.249792in}}%
\pgfpathcurveto{\pgfqpoint{1.615674in}{1.249792in}}{\pgfqpoint{1.605075in}{1.245402in}}{\pgfqpoint{1.597261in}{1.237588in}}%
\pgfpathcurveto{\pgfqpoint{1.589448in}{1.229774in}}{\pgfqpoint{1.585057in}{1.219175in}}{\pgfqpoint{1.585057in}{1.208125in}}%
\pgfpathcurveto{\pgfqpoint{1.585057in}{1.197075in}}{\pgfqpoint{1.589448in}{1.186476in}}{\pgfqpoint{1.597261in}{1.178663in}}%
\pgfpathcurveto{\pgfqpoint{1.605075in}{1.170849in}}{\pgfqpoint{1.615674in}{1.166459in}}{\pgfqpoint{1.626724in}{1.166459in}}%
\pgfpathclose%
\pgfusepath{stroke,fill}%
\end{pgfscope}%
\begin{pgfscope}%
\pgfpathrectangle{\pgfqpoint{0.772069in}{0.515123in}}{\pgfqpoint{1.937500in}{1.347500in}}%
\pgfusepath{clip}%
\pgfsetbuttcap%
\pgfsetroundjoin%
\definecolor{currentfill}{rgb}{0.564706,0.933333,0.564706}%
\pgfsetfillcolor{currentfill}%
\pgfsetlinewidth{1.003750pt}%
\definecolor{currentstroke}{rgb}{0.564706,0.933333,0.564706}%
\pgfsetstrokecolor{currentstroke}%
\pgfsetdash{}{0pt}%
\pgfpathmoveto{\pgfqpoint{1.741816in}{1.253740in}}%
\pgfpathcurveto{\pgfqpoint{1.752866in}{1.253740in}}{\pgfqpoint{1.763465in}{1.258130in}}{\pgfqpoint{1.771279in}{1.265944in}}%
\pgfpathcurveto{\pgfqpoint{1.779092in}{1.273757in}}{\pgfqpoint{1.783482in}{1.284356in}}{\pgfqpoint{1.783482in}{1.295407in}}%
\pgfpathcurveto{\pgfqpoint{1.783482in}{1.306457in}}{\pgfqpoint{1.779092in}{1.317056in}}{\pgfqpoint{1.771279in}{1.324869in}}%
\pgfpathcurveto{\pgfqpoint{1.763465in}{1.332683in}}{\pgfqpoint{1.752866in}{1.337073in}}{\pgfqpoint{1.741816in}{1.337073in}}%
\pgfpathcurveto{\pgfqpoint{1.730766in}{1.337073in}}{\pgfqpoint{1.720167in}{1.332683in}}{\pgfqpoint{1.712353in}{1.324869in}}%
\pgfpathcurveto{\pgfqpoint{1.704539in}{1.317056in}}{\pgfqpoint{1.700149in}{1.306457in}}{\pgfqpoint{1.700149in}{1.295407in}}%
\pgfpathcurveto{\pgfqpoint{1.700149in}{1.284356in}}{\pgfqpoint{1.704539in}{1.273757in}}{\pgfqpoint{1.712353in}{1.265944in}}%
\pgfpathcurveto{\pgfqpoint{1.720167in}{1.258130in}}{\pgfqpoint{1.730766in}{1.253740in}}{\pgfqpoint{1.741816in}{1.253740in}}%
\pgfpathclose%
\pgfusepath{stroke,fill}%
\end{pgfscope}%
\begin{pgfscope}%
\pgfpathrectangle{\pgfqpoint{0.772069in}{0.515123in}}{\pgfqpoint{1.937500in}{1.347500in}}%
\pgfusepath{clip}%
\pgfsetbuttcap%
\pgfsetroundjoin%
\definecolor{currentfill}{rgb}{0.564706,0.933333,0.564706}%
\pgfsetfillcolor{currentfill}%
\pgfsetlinewidth{1.003750pt}%
\definecolor{currentstroke}{rgb}{0.564706,0.933333,0.564706}%
\pgfsetstrokecolor{currentstroke}%
\pgfsetdash{}{0pt}%
\pgfpathmoveto{\pgfqpoint{1.856908in}{1.339857in}}%
\pgfpathcurveto{\pgfqpoint{1.867958in}{1.339857in}}{\pgfqpoint{1.878557in}{1.344248in}}{\pgfqpoint{1.886370in}{1.352061in}}%
\pgfpathcurveto{\pgfqpoint{1.894184in}{1.359875in}}{\pgfqpoint{1.898574in}{1.370474in}}{\pgfqpoint{1.898574in}{1.381524in}}%
\pgfpathcurveto{\pgfqpoint{1.898574in}{1.392574in}}{\pgfqpoint{1.894184in}{1.403173in}}{\pgfqpoint{1.886370in}{1.410987in}}%
\pgfpathcurveto{\pgfqpoint{1.878557in}{1.418801in}}{\pgfqpoint{1.867958in}{1.423191in}}{\pgfqpoint{1.856908in}{1.423191in}}%
\pgfpathcurveto{\pgfqpoint{1.845857in}{1.423191in}}{\pgfqpoint{1.835258in}{1.418801in}}{\pgfqpoint{1.827445in}{1.410987in}}%
\pgfpathcurveto{\pgfqpoint{1.819631in}{1.403173in}}{\pgfqpoint{1.815241in}{1.392574in}}{\pgfqpoint{1.815241in}{1.381524in}}%
\pgfpathcurveto{\pgfqpoint{1.815241in}{1.370474in}}{\pgfqpoint{1.819631in}{1.359875in}}{\pgfqpoint{1.827445in}{1.352061in}}%
\pgfpathcurveto{\pgfqpoint{1.835258in}{1.344248in}}{\pgfqpoint{1.845857in}{1.339857in}}{\pgfqpoint{1.856908in}{1.339857in}}%
\pgfpathclose%
\pgfusepath{stroke,fill}%
\end{pgfscope}%
\begin{pgfscope}%
\pgfpathrectangle{\pgfqpoint{0.772069in}{0.515123in}}{\pgfqpoint{1.937500in}{1.347500in}}%
\pgfusepath{clip}%
\pgfsetbuttcap%
\pgfsetroundjoin%
\definecolor{currentfill}{rgb}{0.564706,0.933333,0.564706}%
\pgfsetfillcolor{currentfill}%
\pgfsetlinewidth{1.003750pt}%
\definecolor{currentstroke}{rgb}{0.564706,0.933333,0.564706}%
\pgfsetstrokecolor{currentstroke}%
\pgfsetdash{}{0pt}%
\pgfpathmoveto{\pgfqpoint{1.971999in}{1.428302in}}%
\pgfpathcurveto{\pgfqpoint{1.983049in}{1.428302in}}{\pgfqpoint{1.993648in}{1.432693in}}{\pgfqpoint{2.001462in}{1.440506in}}%
\pgfpathcurveto{\pgfqpoint{2.009276in}{1.448320in}}{\pgfqpoint{2.013666in}{1.458919in}}{\pgfqpoint{2.013666in}{1.469969in}}%
\pgfpathcurveto{\pgfqpoint{2.013666in}{1.481019in}}{\pgfqpoint{2.009276in}{1.491618in}}{\pgfqpoint{2.001462in}{1.499432in}}%
\pgfpathcurveto{\pgfqpoint{1.993648in}{1.507246in}}{\pgfqpoint{1.983049in}{1.511636in}}{\pgfqpoint{1.971999in}{1.511636in}}%
\pgfpathcurveto{\pgfqpoint{1.960949in}{1.511636in}}{\pgfqpoint{1.950350in}{1.507246in}}{\pgfqpoint{1.942537in}{1.499432in}}%
\pgfpathcurveto{\pgfqpoint{1.934723in}{1.491618in}}{\pgfqpoint{1.930333in}{1.481019in}}{\pgfqpoint{1.930333in}{1.469969in}}%
\pgfpathcurveto{\pgfqpoint{1.930333in}{1.458919in}}{\pgfqpoint{1.934723in}{1.448320in}}{\pgfqpoint{1.942537in}{1.440506in}}%
\pgfpathcurveto{\pgfqpoint{1.950350in}{1.432693in}}{\pgfqpoint{1.960949in}{1.428302in}}{\pgfqpoint{1.971999in}{1.428302in}}%
\pgfpathclose%
\pgfusepath{stroke,fill}%
\end{pgfscope}%
\begin{pgfscope}%
\pgfpathrectangle{\pgfqpoint{0.772069in}{0.515123in}}{\pgfqpoint{1.937500in}{1.347500in}}%
\pgfusepath{clip}%
\pgfsetbuttcap%
\pgfsetroundjoin%
\definecolor{currentfill}{rgb}{1.000000,0.854902,0.725490}%
\pgfsetfillcolor{currentfill}%
\pgfsetlinewidth{1.003750pt}%
\definecolor{currentstroke}{rgb}{1.000000,0.854902,0.725490}%
\pgfsetstrokecolor{currentstroke}%
\pgfsetdash{}{0pt}%
\pgfpathmoveto{\pgfqpoint{0.890137in}{0.567907in}}%
\pgfpathcurveto{\pgfqpoint{0.901187in}{0.567907in}}{\pgfqpoint{0.911786in}{0.572297in}}{\pgfqpoint{0.919600in}{0.580111in}}%
\pgfpathcurveto{\pgfqpoint{0.927413in}{0.587924in}}{\pgfqpoint{0.931804in}{0.598523in}}{\pgfqpoint{0.931804in}{0.609574in}}%
\pgfpathcurveto{\pgfqpoint{0.931804in}{0.620624in}}{\pgfqpoint{0.927413in}{0.631223in}}{\pgfqpoint{0.919600in}{0.639036in}}%
\pgfpathcurveto{\pgfqpoint{0.911786in}{0.646850in}}{\pgfqpoint{0.901187in}{0.651240in}}{\pgfqpoint{0.890137in}{0.651240in}}%
\pgfpathcurveto{\pgfqpoint{0.879087in}{0.651240in}}{\pgfqpoint{0.868488in}{0.646850in}}{\pgfqpoint{0.860674in}{0.639036in}}%
\pgfpathcurveto{\pgfqpoint{0.852860in}{0.631223in}}{\pgfqpoint{0.848470in}{0.620624in}}{\pgfqpoint{0.848470in}{0.609574in}}%
\pgfpathcurveto{\pgfqpoint{0.848470in}{0.598523in}}{\pgfqpoint{0.852860in}{0.587924in}}{\pgfqpoint{0.860674in}{0.580111in}}%
\pgfpathcurveto{\pgfqpoint{0.868488in}{0.572297in}}{\pgfqpoint{0.879087in}{0.567907in}}{\pgfqpoint{0.890137in}{0.567907in}}%
\pgfpathclose%
\pgfusepath{stroke,fill}%
\end{pgfscope}%
\begin{pgfscope}%
\pgfpathrectangle{\pgfqpoint{0.772069in}{0.515123in}}{\pgfqpoint{1.937500in}{1.347500in}}%
\pgfusepath{clip}%
\pgfsetbuttcap%
\pgfsetroundjoin%
\definecolor{currentfill}{rgb}{1.000000,0.854902,0.725490}%
\pgfsetfillcolor{currentfill}%
\pgfsetlinewidth{1.003750pt}%
\definecolor{currentstroke}{rgb}{1.000000,0.854902,0.725490}%
\pgfsetstrokecolor{currentstroke}%
\pgfsetdash{}{0pt}%
\pgfpathmoveto{\pgfqpoint{0.959192in}{0.582372in}}%
\pgfpathcurveto{\pgfqpoint{0.970242in}{0.582372in}}{\pgfqpoint{0.980841in}{0.586763in}}{\pgfqpoint{0.988655in}{0.594576in}}%
\pgfpathcurveto{\pgfqpoint{0.996468in}{0.602390in}}{\pgfqpoint{1.000859in}{0.612989in}}{\pgfqpoint{1.000859in}{0.624039in}}%
\pgfpathcurveto{\pgfqpoint{1.000859in}{0.635089in}}{\pgfqpoint{0.996468in}{0.645688in}}{\pgfqpoint{0.988655in}{0.653502in}}%
\pgfpathcurveto{\pgfqpoint{0.980841in}{0.661315in}}{\pgfqpoint{0.970242in}{0.665706in}}{\pgfqpoint{0.959192in}{0.665706in}}%
\pgfpathcurveto{\pgfqpoint{0.948142in}{0.665706in}}{\pgfqpoint{0.937543in}{0.661315in}}{\pgfqpoint{0.929729in}{0.653502in}}%
\pgfpathcurveto{\pgfqpoint{0.921916in}{0.645688in}}{\pgfqpoint{0.917525in}{0.635089in}}{\pgfqpoint{0.917525in}{0.624039in}}%
\pgfpathcurveto{\pgfqpoint{0.917525in}{0.612989in}}{\pgfqpoint{0.921916in}{0.602390in}}{\pgfqpoint{0.929729in}{0.594576in}}%
\pgfpathcurveto{\pgfqpoint{0.937543in}{0.586763in}}{\pgfqpoint{0.948142in}{0.582372in}}{\pgfqpoint{0.959192in}{0.582372in}}%
\pgfpathclose%
\pgfusepath{stroke,fill}%
\end{pgfscope}%
\begin{pgfscope}%
\pgfpathrectangle{\pgfqpoint{0.772069in}{0.515123in}}{\pgfqpoint{1.937500in}{1.347500in}}%
\pgfusepath{clip}%
\pgfsetbuttcap%
\pgfsetroundjoin%
\definecolor{currentfill}{rgb}{1.000000,0.854902,0.725490}%
\pgfsetfillcolor{currentfill}%
\pgfsetlinewidth{1.003750pt}%
\definecolor{currentstroke}{rgb}{1.000000,0.854902,0.725490}%
\pgfsetstrokecolor{currentstroke}%
\pgfsetdash{}{0pt}%
\pgfpathmoveto{\pgfqpoint{1.005229in}{0.586562in}}%
\pgfpathcurveto{\pgfqpoint{1.016279in}{0.586562in}}{\pgfqpoint{1.026878in}{0.590952in}}{\pgfqpoint{1.034691in}{0.598766in}}%
\pgfpathcurveto{\pgfqpoint{1.042505in}{0.606579in}}{\pgfqpoint{1.046895in}{0.617178in}}{\pgfqpoint{1.046895in}{0.628228in}}%
\pgfpathcurveto{\pgfqpoint{1.046895in}{0.639279in}}{\pgfqpoint{1.042505in}{0.649878in}}{\pgfqpoint{1.034691in}{0.657691in}}%
\pgfpathcurveto{\pgfqpoint{1.026878in}{0.665505in}}{\pgfqpoint{1.016279in}{0.669895in}}{\pgfqpoint{1.005229in}{0.669895in}}%
\pgfpathcurveto{\pgfqpoint{0.994178in}{0.669895in}}{\pgfqpoint{0.983579in}{0.665505in}}{\pgfqpoint{0.975766in}{0.657691in}}%
\pgfpathcurveto{\pgfqpoint{0.967952in}{0.649878in}}{\pgfqpoint{0.963562in}{0.639279in}}{\pgfqpoint{0.963562in}{0.628228in}}%
\pgfpathcurveto{\pgfqpoint{0.963562in}{0.617178in}}{\pgfqpoint{0.967952in}{0.606579in}}{\pgfqpoint{0.975766in}{0.598766in}}%
\pgfpathcurveto{\pgfqpoint{0.983579in}{0.590952in}}{\pgfqpoint{0.994178in}{0.586562in}}{\pgfqpoint{1.005229in}{0.586562in}}%
\pgfpathclose%
\pgfusepath{stroke,fill}%
\end{pgfscope}%
\begin{pgfscope}%
\pgfpathrectangle{\pgfqpoint{0.772069in}{0.515123in}}{\pgfqpoint{1.937500in}{1.347500in}}%
\pgfusepath{clip}%
\pgfsetbuttcap%
\pgfsetroundjoin%
\definecolor{currentfill}{rgb}{1.000000,0.854902,0.725490}%
\pgfsetfillcolor{currentfill}%
\pgfsetlinewidth{1.003750pt}%
\definecolor{currentstroke}{rgb}{1.000000,0.854902,0.725490}%
\pgfsetstrokecolor{currentstroke}%
\pgfsetdash{}{0pt}%
\pgfpathmoveto{\pgfqpoint{1.074284in}{0.613794in}}%
\pgfpathcurveto{\pgfqpoint{1.085334in}{0.613794in}}{\pgfqpoint{1.095933in}{0.618184in}}{\pgfqpoint{1.103746in}{0.625997in}}%
\pgfpathcurveto{\pgfqpoint{1.111560in}{0.633811in}}{\pgfqpoint{1.115950in}{0.644410in}}{\pgfqpoint{1.115950in}{0.655460in}}%
\pgfpathcurveto{\pgfqpoint{1.115950in}{0.666510in}}{\pgfqpoint{1.111560in}{0.677109in}}{\pgfqpoint{1.103746in}{0.684923in}}%
\pgfpathcurveto{\pgfqpoint{1.095933in}{0.692737in}}{\pgfqpoint{1.085334in}{0.697127in}}{\pgfqpoint{1.074284in}{0.697127in}}%
\pgfpathcurveto{\pgfqpoint{1.063234in}{0.697127in}}{\pgfqpoint{1.052635in}{0.692737in}}{\pgfqpoint{1.044821in}{0.684923in}}%
\pgfpathcurveto{\pgfqpoint{1.037007in}{0.677109in}}{\pgfqpoint{1.032617in}{0.666510in}}{\pgfqpoint{1.032617in}{0.655460in}}%
\pgfpathcurveto{\pgfqpoint{1.032617in}{0.644410in}}{\pgfqpoint{1.037007in}{0.633811in}}{\pgfqpoint{1.044821in}{0.625997in}}%
\pgfpathcurveto{\pgfqpoint{1.052635in}{0.618184in}}{\pgfqpoint{1.063234in}{0.613794in}}{\pgfqpoint{1.074284in}{0.613794in}}%
\pgfpathclose%
\pgfusepath{stroke,fill}%
\end{pgfscope}%
\begin{pgfscope}%
\pgfpathrectangle{\pgfqpoint{0.772069in}{0.515123in}}{\pgfqpoint{1.937500in}{1.347500in}}%
\pgfusepath{clip}%
\pgfsetbuttcap%
\pgfsetroundjoin%
\definecolor{currentfill}{rgb}{1.000000,0.854902,0.725490}%
\pgfsetfillcolor{currentfill}%
\pgfsetlinewidth{1.003750pt}%
\definecolor{currentstroke}{rgb}{1.000000,0.854902,0.725490}%
\pgfsetstrokecolor{currentstroke}%
\pgfsetdash{}{0pt}%
\pgfpathmoveto{\pgfqpoint{1.143339in}{0.629271in}}%
\pgfpathcurveto{\pgfqpoint{1.154389in}{0.629271in}}{\pgfqpoint{1.164988in}{0.633662in}}{\pgfqpoint{1.172801in}{0.641475in}}%
\pgfpathcurveto{\pgfqpoint{1.180615in}{0.649289in}}{\pgfqpoint{1.185005in}{0.659888in}}{\pgfqpoint{1.185005in}{0.670938in}}%
\pgfpathcurveto{\pgfqpoint{1.185005in}{0.681988in}}{\pgfqpoint{1.180615in}{0.692587in}}{\pgfqpoint{1.172801in}{0.700401in}}%
\pgfpathcurveto{\pgfqpoint{1.164988in}{0.708214in}}{\pgfqpoint{1.154389in}{0.712605in}}{\pgfqpoint{1.143339in}{0.712605in}}%
\pgfpathcurveto{\pgfqpoint{1.132289in}{0.712605in}}{\pgfqpoint{1.121690in}{0.708214in}}{\pgfqpoint{1.113876in}{0.700401in}}%
\pgfpathcurveto{\pgfqpoint{1.106062in}{0.692587in}}{\pgfqpoint{1.101672in}{0.681988in}}{\pgfqpoint{1.101672in}{0.670938in}}%
\pgfpathcurveto{\pgfqpoint{1.101672in}{0.659888in}}{\pgfqpoint{1.106062in}{0.649289in}}{\pgfqpoint{1.113876in}{0.641475in}}%
\pgfpathcurveto{\pgfqpoint{1.121690in}{0.633662in}}{\pgfqpoint{1.132289in}{0.629271in}}{\pgfqpoint{1.143339in}{0.629271in}}%
\pgfpathclose%
\pgfusepath{stroke,fill}%
\end{pgfscope}%
\begin{pgfscope}%
\pgfpathrectangle{\pgfqpoint{0.772069in}{0.515123in}}{\pgfqpoint{1.937500in}{1.347500in}}%
\pgfusepath{clip}%
\pgfsetbuttcap%
\pgfsetroundjoin%
\definecolor{currentfill}{rgb}{1.000000,0.854902,0.725490}%
\pgfsetfillcolor{currentfill}%
\pgfsetlinewidth{1.003750pt}%
\definecolor{currentstroke}{rgb}{1.000000,0.854902,0.725490}%
\pgfsetstrokecolor{currentstroke}%
\pgfsetdash{}{0pt}%
\pgfpathmoveto{\pgfqpoint{1.189375in}{0.644051in}}%
\pgfpathcurveto{\pgfqpoint{1.200426in}{0.644051in}}{\pgfqpoint{1.211025in}{0.648441in}}{\pgfqpoint{1.218838in}{0.656255in}}%
\pgfpathcurveto{\pgfqpoint{1.226652in}{0.664069in}}{\pgfqpoint{1.231042in}{0.674668in}}{\pgfqpoint{1.231042in}{0.685718in}}%
\pgfpathcurveto{\pgfqpoint{1.231042in}{0.696768in}}{\pgfqpoint{1.226652in}{0.707367in}}{\pgfqpoint{1.218838in}{0.715181in}}%
\pgfpathcurveto{\pgfqpoint{1.211025in}{0.722994in}}{\pgfqpoint{1.200426in}{0.727384in}}{\pgfqpoint{1.189375in}{0.727384in}}%
\pgfpathcurveto{\pgfqpoint{1.178325in}{0.727384in}}{\pgfqpoint{1.167726in}{0.722994in}}{\pgfqpoint{1.159913in}{0.715181in}}%
\pgfpathcurveto{\pgfqpoint{1.152099in}{0.707367in}}{\pgfqpoint{1.147709in}{0.696768in}}{\pgfqpoint{1.147709in}{0.685718in}}%
\pgfpathcurveto{\pgfqpoint{1.147709in}{0.674668in}}{\pgfqpoint{1.152099in}{0.664069in}}{\pgfqpoint{1.159913in}{0.656255in}}%
\pgfpathcurveto{\pgfqpoint{1.167726in}{0.648441in}}{\pgfqpoint{1.178325in}{0.644051in}}{\pgfqpoint{1.189375in}{0.644051in}}%
\pgfpathclose%
\pgfusepath{stroke,fill}%
\end{pgfscope}%
\begin{pgfscope}%
\pgfpathrectangle{\pgfqpoint{0.772069in}{0.515123in}}{\pgfqpoint{1.937500in}{1.347500in}}%
\pgfusepath{clip}%
\pgfsetbuttcap%
\pgfsetroundjoin%
\definecolor{currentfill}{rgb}{1.000000,0.854902,0.725490}%
\pgfsetfillcolor{currentfill}%
\pgfsetlinewidth{1.003750pt}%
\definecolor{currentstroke}{rgb}{1.000000,0.854902,0.725490}%
\pgfsetstrokecolor{currentstroke}%
\pgfsetdash{}{0pt}%
\pgfpathmoveto{\pgfqpoint{1.258430in}{0.660111in}}%
\pgfpathcurveto{\pgfqpoint{1.269481in}{0.660111in}}{\pgfqpoint{1.280080in}{0.664501in}}{\pgfqpoint{1.287893in}{0.672315in}}%
\pgfpathcurveto{\pgfqpoint{1.295707in}{0.680128in}}{\pgfqpoint{1.300097in}{0.690727in}}{\pgfqpoint{1.300097in}{0.701777in}}%
\pgfpathcurveto{\pgfqpoint{1.300097in}{0.712828in}}{\pgfqpoint{1.295707in}{0.723427in}}{\pgfqpoint{1.287893in}{0.731240in}}%
\pgfpathcurveto{\pgfqpoint{1.280080in}{0.739054in}}{\pgfqpoint{1.269481in}{0.743444in}}{\pgfqpoint{1.258430in}{0.743444in}}%
\pgfpathcurveto{\pgfqpoint{1.247380in}{0.743444in}}{\pgfqpoint{1.236781in}{0.739054in}}{\pgfqpoint{1.228968in}{0.731240in}}%
\pgfpathcurveto{\pgfqpoint{1.221154in}{0.723427in}}{\pgfqpoint{1.216764in}{0.712828in}}{\pgfqpoint{1.216764in}{0.701777in}}%
\pgfpathcurveto{\pgfqpoint{1.216764in}{0.690727in}}{\pgfqpoint{1.221154in}{0.680128in}}{\pgfqpoint{1.228968in}{0.672315in}}%
\pgfpathcurveto{\pgfqpoint{1.236781in}{0.664501in}}{\pgfqpoint{1.247380in}{0.660111in}}{\pgfqpoint{1.258430in}{0.660111in}}%
\pgfpathclose%
\pgfusepath{stroke,fill}%
\end{pgfscope}%
\begin{pgfscope}%
\pgfpathrectangle{\pgfqpoint{0.772069in}{0.515123in}}{\pgfqpoint{1.937500in}{1.347500in}}%
\pgfusepath{clip}%
\pgfsetbuttcap%
\pgfsetroundjoin%
\definecolor{currentfill}{rgb}{1.000000,0.854902,0.725490}%
\pgfsetfillcolor{currentfill}%
\pgfsetlinewidth{1.003750pt}%
\definecolor{currentstroke}{rgb}{1.000000,0.854902,0.725490}%
\pgfsetstrokecolor{currentstroke}%
\pgfsetdash{}{0pt}%
\pgfpathmoveto{\pgfqpoint{1.327486in}{0.675240in}}%
\pgfpathcurveto{\pgfqpoint{1.338536in}{0.675240in}}{\pgfqpoint{1.349135in}{0.679630in}}{\pgfqpoint{1.356948in}{0.687443in}}%
\pgfpathcurveto{\pgfqpoint{1.364762in}{0.695257in}}{\pgfqpoint{1.369152in}{0.705856in}}{\pgfqpoint{1.369152in}{0.716906in}}%
\pgfpathcurveto{\pgfqpoint{1.369152in}{0.727956in}}{\pgfqpoint{1.364762in}{0.738555in}}{\pgfqpoint{1.356948in}{0.746369in}}%
\pgfpathcurveto{\pgfqpoint{1.349135in}{0.754183in}}{\pgfqpoint{1.338536in}{0.758573in}}{\pgfqpoint{1.327486in}{0.758573in}}%
\pgfpathcurveto{\pgfqpoint{1.316435in}{0.758573in}}{\pgfqpoint{1.305836in}{0.754183in}}{\pgfqpoint{1.298023in}{0.746369in}}%
\pgfpathcurveto{\pgfqpoint{1.290209in}{0.738555in}}{\pgfqpoint{1.285819in}{0.727956in}}{\pgfqpoint{1.285819in}{0.716906in}}%
\pgfpathcurveto{\pgfqpoint{1.285819in}{0.705856in}}{\pgfqpoint{1.290209in}{0.695257in}}{\pgfqpoint{1.298023in}{0.687443in}}%
\pgfpathcurveto{\pgfqpoint{1.305836in}{0.679630in}}{\pgfqpoint{1.316435in}{0.675240in}}{\pgfqpoint{1.327486in}{0.675240in}}%
\pgfpathclose%
\pgfusepath{stroke,fill}%
\end{pgfscope}%
\begin{pgfscope}%
\pgfpathrectangle{\pgfqpoint{0.772069in}{0.515123in}}{\pgfqpoint{1.937500in}{1.347500in}}%
\pgfusepath{clip}%
\pgfsetbuttcap%
\pgfsetroundjoin%
\definecolor{currentfill}{rgb}{1.000000,0.854902,0.725490}%
\pgfsetfillcolor{currentfill}%
\pgfsetlinewidth{1.003750pt}%
\definecolor{currentstroke}{rgb}{1.000000,0.854902,0.725490}%
\pgfsetstrokecolor{currentstroke}%
\pgfsetdash{}{0pt}%
\pgfpathmoveto{\pgfqpoint{1.396541in}{0.690368in}}%
\pgfpathcurveto{\pgfqpoint{1.407591in}{0.690368in}}{\pgfqpoint{1.418190in}{0.694759in}}{\pgfqpoint{1.426003in}{0.702572in}}%
\pgfpathcurveto{\pgfqpoint{1.433817in}{0.710386in}}{\pgfqpoint{1.438207in}{0.720985in}}{\pgfqpoint{1.438207in}{0.732035in}}%
\pgfpathcurveto{\pgfqpoint{1.438207in}{0.743085in}}{\pgfqpoint{1.433817in}{0.753684in}}{\pgfqpoint{1.426003in}{0.761498in}}%
\pgfpathcurveto{\pgfqpoint{1.418190in}{0.769311in}}{\pgfqpoint{1.407591in}{0.773702in}}{\pgfqpoint{1.396541in}{0.773702in}}%
\pgfpathcurveto{\pgfqpoint{1.385490in}{0.773702in}}{\pgfqpoint{1.374891in}{0.769311in}}{\pgfqpoint{1.367078in}{0.761498in}}%
\pgfpathcurveto{\pgfqpoint{1.359264in}{0.753684in}}{\pgfqpoint{1.354874in}{0.743085in}}{\pgfqpoint{1.354874in}{0.732035in}}%
\pgfpathcurveto{\pgfqpoint{1.354874in}{0.720985in}}{\pgfqpoint{1.359264in}{0.710386in}}{\pgfqpoint{1.367078in}{0.702572in}}%
\pgfpathcurveto{\pgfqpoint{1.374891in}{0.694759in}}{\pgfqpoint{1.385490in}{0.690368in}}{\pgfqpoint{1.396541in}{0.690368in}}%
\pgfpathclose%
\pgfusepath{stroke,fill}%
\end{pgfscope}%
\begin{pgfscope}%
\pgfpathrectangle{\pgfqpoint{0.772069in}{0.515123in}}{\pgfqpoint{1.937500in}{1.347500in}}%
\pgfusepath{clip}%
\pgfsetbuttcap%
\pgfsetroundjoin%
\definecolor{currentfill}{rgb}{1.000000,0.854902,0.725490}%
\pgfsetfillcolor{currentfill}%
\pgfsetlinewidth{1.003750pt}%
\definecolor{currentstroke}{rgb}{1.000000,0.854902,0.725490}%
\pgfsetstrokecolor{currentstroke}%
\pgfsetdash{}{0pt}%
\pgfpathmoveto{\pgfqpoint{1.442577in}{0.705963in}}%
\pgfpathcurveto{\pgfqpoint{1.453627in}{0.705963in}}{\pgfqpoint{1.464226in}{0.710353in}}{\pgfqpoint{1.472040in}{0.718166in}}%
\pgfpathcurveto{\pgfqpoint{1.479854in}{0.725980in}}{\pgfqpoint{1.484244in}{0.736579in}}{\pgfqpoint{1.484244in}{0.747629in}}%
\pgfpathcurveto{\pgfqpoint{1.484244in}{0.758679in}}{\pgfqpoint{1.479854in}{0.769278in}}{\pgfqpoint{1.472040in}{0.777092in}}%
\pgfpathcurveto{\pgfqpoint{1.464226in}{0.784906in}}{\pgfqpoint{1.453627in}{0.789296in}}{\pgfqpoint{1.442577in}{0.789296in}}%
\pgfpathcurveto{\pgfqpoint{1.431527in}{0.789296in}}{\pgfqpoint{1.420928in}{0.784906in}}{\pgfqpoint{1.413114in}{0.777092in}}%
\pgfpathcurveto{\pgfqpoint{1.405301in}{0.769278in}}{\pgfqpoint{1.400911in}{0.758679in}}{\pgfqpoint{1.400911in}{0.747629in}}%
\pgfpathcurveto{\pgfqpoint{1.400911in}{0.736579in}}{\pgfqpoint{1.405301in}{0.725980in}}{\pgfqpoint{1.413114in}{0.718166in}}%
\pgfpathcurveto{\pgfqpoint{1.420928in}{0.710353in}}{\pgfqpoint{1.431527in}{0.705963in}}{\pgfqpoint{1.442577in}{0.705963in}}%
\pgfpathclose%
\pgfusepath{stroke,fill}%
\end{pgfscope}%
\begin{pgfscope}%
\pgfpathrectangle{\pgfqpoint{0.772069in}{0.515123in}}{\pgfqpoint{1.937500in}{1.347500in}}%
\pgfusepath{clip}%
\pgfsetbuttcap%
\pgfsetroundjoin%
\definecolor{currentfill}{rgb}{1.000000,0.980392,0.803922}%
\pgfsetfillcolor{currentfill}%
\pgfsetlinewidth{1.003750pt}%
\definecolor{currentstroke}{rgb}{1.000000,0.980392,0.803922}%
\pgfsetstrokecolor{currentstroke}%
\pgfsetdash{}{0pt}%
\pgfpathmoveto{\pgfqpoint{0.890137in}{0.564707in}}%
\pgfpathcurveto{\pgfqpoint{0.901187in}{0.564707in}}{\pgfqpoint{0.911786in}{0.569097in}}{\pgfqpoint{0.919600in}{0.576910in}}%
\pgfpathcurveto{\pgfqpoint{0.927413in}{0.584724in}}{\pgfqpoint{0.931804in}{0.595323in}}{\pgfqpoint{0.931804in}{0.606373in}}%
\pgfpathcurveto{\pgfqpoint{0.931804in}{0.617423in}}{\pgfqpoint{0.927413in}{0.628022in}}{\pgfqpoint{0.919600in}{0.635836in}}%
\pgfpathcurveto{\pgfqpoint{0.911786in}{0.643650in}}{\pgfqpoint{0.901187in}{0.648040in}}{\pgfqpoint{0.890137in}{0.648040in}}%
\pgfpathcurveto{\pgfqpoint{0.879087in}{0.648040in}}{\pgfqpoint{0.868488in}{0.643650in}}{\pgfqpoint{0.860674in}{0.635836in}}%
\pgfpathcurveto{\pgfqpoint{0.852860in}{0.628022in}}{\pgfqpoint{0.848470in}{0.617423in}}{\pgfqpoint{0.848470in}{0.606373in}}%
\pgfpathcurveto{\pgfqpoint{0.848470in}{0.595323in}}{\pgfqpoint{0.852860in}{0.584724in}}{\pgfqpoint{0.860674in}{0.576910in}}%
\pgfpathcurveto{\pgfqpoint{0.868488in}{0.569097in}}{\pgfqpoint{0.879087in}{0.564707in}}{\pgfqpoint{0.890137in}{0.564707in}}%
\pgfpathclose%
\pgfusepath{stroke,fill}%
\end{pgfscope}%
\begin{pgfscope}%
\pgfpathrectangle{\pgfqpoint{0.772069in}{0.515123in}}{\pgfqpoint{1.937500in}{1.347500in}}%
\pgfusepath{clip}%
\pgfsetbuttcap%
\pgfsetroundjoin%
\definecolor{currentfill}{rgb}{1.000000,0.980392,0.803922}%
\pgfsetfillcolor{currentfill}%
\pgfsetlinewidth{1.003750pt}%
\definecolor{currentstroke}{rgb}{1.000000,0.980392,0.803922}%
\pgfsetstrokecolor{currentstroke}%
\pgfsetdash{}{0pt}%
\pgfpathmoveto{\pgfqpoint{0.959192in}{0.576321in}}%
\pgfpathcurveto{\pgfqpoint{0.970242in}{0.576321in}}{\pgfqpoint{0.980841in}{0.580711in}}{\pgfqpoint{0.988655in}{0.588525in}}%
\pgfpathcurveto{\pgfqpoint{0.996468in}{0.596338in}}{\pgfqpoint{1.000859in}{0.606937in}}{\pgfqpoint{1.000859in}{0.617987in}}%
\pgfpathcurveto{\pgfqpoint{1.000859in}{0.629038in}}{\pgfqpoint{0.996468in}{0.639637in}}{\pgfqpoint{0.988655in}{0.647450in}}%
\pgfpathcurveto{\pgfqpoint{0.980841in}{0.655264in}}{\pgfqpoint{0.970242in}{0.659654in}}{\pgfqpoint{0.959192in}{0.659654in}}%
\pgfpathcurveto{\pgfqpoint{0.948142in}{0.659654in}}{\pgfqpoint{0.937543in}{0.655264in}}{\pgfqpoint{0.929729in}{0.647450in}}%
\pgfpathcurveto{\pgfqpoint{0.921916in}{0.639637in}}{\pgfqpoint{0.917525in}{0.629038in}}{\pgfqpoint{0.917525in}{0.617987in}}%
\pgfpathcurveto{\pgfqpoint{0.917525in}{0.606937in}}{\pgfqpoint{0.921916in}{0.596338in}}{\pgfqpoint{0.929729in}{0.588525in}}%
\pgfpathcurveto{\pgfqpoint{0.937543in}{0.580711in}}{\pgfqpoint{0.948142in}{0.576321in}}{\pgfqpoint{0.959192in}{0.576321in}}%
\pgfpathclose%
\pgfusepath{stroke,fill}%
\end{pgfscope}%
\begin{pgfscope}%
\pgfpathrectangle{\pgfqpoint{0.772069in}{0.515123in}}{\pgfqpoint{1.937500in}{1.347500in}}%
\pgfusepath{clip}%
\pgfsetbuttcap%
\pgfsetroundjoin%
\definecolor{currentfill}{rgb}{1.000000,0.980392,0.803922}%
\pgfsetfillcolor{currentfill}%
\pgfsetlinewidth{1.003750pt}%
\definecolor{currentstroke}{rgb}{1.000000,0.980392,0.803922}%
\pgfsetstrokecolor{currentstroke}%
\pgfsetdash{}{0pt}%
\pgfpathmoveto{\pgfqpoint{1.005229in}{0.588889in}}%
\pgfpathcurveto{\pgfqpoint{1.016279in}{0.588889in}}{\pgfqpoint{1.026878in}{0.593280in}}{\pgfqpoint{1.034691in}{0.601093in}}%
\pgfpathcurveto{\pgfqpoint{1.042505in}{0.608907in}}{\pgfqpoint{1.046895in}{0.619506in}}{\pgfqpoint{1.046895in}{0.630556in}}%
\pgfpathcurveto{\pgfqpoint{1.046895in}{0.641606in}}{\pgfqpoint{1.042505in}{0.652205in}}{\pgfqpoint{1.034691in}{0.660019in}}%
\pgfpathcurveto{\pgfqpoint{1.026878in}{0.667832in}}{\pgfqpoint{1.016279in}{0.672223in}}{\pgfqpoint{1.005229in}{0.672223in}}%
\pgfpathcurveto{\pgfqpoint{0.994178in}{0.672223in}}{\pgfqpoint{0.983579in}{0.667832in}}{\pgfqpoint{0.975766in}{0.660019in}}%
\pgfpathcurveto{\pgfqpoint{0.967952in}{0.652205in}}{\pgfqpoint{0.963562in}{0.641606in}}{\pgfqpoint{0.963562in}{0.630556in}}%
\pgfpathcurveto{\pgfqpoint{0.963562in}{0.619506in}}{\pgfqpoint{0.967952in}{0.608907in}}{\pgfqpoint{0.975766in}{0.601093in}}%
\pgfpathcurveto{\pgfqpoint{0.983579in}{0.593280in}}{\pgfqpoint{0.994178in}{0.588889in}}{\pgfqpoint{1.005229in}{0.588889in}}%
\pgfpathclose%
\pgfusepath{stroke,fill}%
\end{pgfscope}%
\begin{pgfscope}%
\pgfpathrectangle{\pgfqpoint{0.772069in}{0.515123in}}{\pgfqpoint{1.937500in}{1.347500in}}%
\pgfusepath{clip}%
\pgfsetbuttcap%
\pgfsetroundjoin%
\definecolor{currentfill}{rgb}{1.000000,0.980392,0.803922}%
\pgfsetfillcolor{currentfill}%
\pgfsetlinewidth{1.003750pt}%
\definecolor{currentstroke}{rgb}{1.000000,0.980392,0.803922}%
\pgfsetstrokecolor{currentstroke}%
\pgfsetdash{}{0pt}%
\pgfpathmoveto{\pgfqpoint{1.074284in}{0.601225in}}%
\pgfpathcurveto{\pgfqpoint{1.085334in}{0.601225in}}{\pgfqpoint{1.095933in}{0.605615in}}{\pgfqpoint{1.103746in}{0.613429in}}%
\pgfpathcurveto{\pgfqpoint{1.111560in}{0.621243in}}{\pgfqpoint{1.115950in}{0.631842in}}{\pgfqpoint{1.115950in}{0.642892in}}%
\pgfpathcurveto{\pgfqpoint{1.115950in}{0.653942in}}{\pgfqpoint{1.111560in}{0.664541in}}{\pgfqpoint{1.103746in}{0.672354in}}%
\pgfpathcurveto{\pgfqpoint{1.095933in}{0.680168in}}{\pgfqpoint{1.085334in}{0.684558in}}{\pgfqpoint{1.074284in}{0.684558in}}%
\pgfpathcurveto{\pgfqpoint{1.063234in}{0.684558in}}{\pgfqpoint{1.052635in}{0.680168in}}{\pgfqpoint{1.044821in}{0.672354in}}%
\pgfpathcurveto{\pgfqpoint{1.037007in}{0.664541in}}{\pgfqpoint{1.032617in}{0.653942in}}{\pgfqpoint{1.032617in}{0.642892in}}%
\pgfpathcurveto{\pgfqpoint{1.032617in}{0.631842in}}{\pgfqpoint{1.037007in}{0.621243in}}{\pgfqpoint{1.044821in}{0.613429in}}%
\pgfpathcurveto{\pgfqpoint{1.052635in}{0.605615in}}{\pgfqpoint{1.063234in}{0.601225in}}{\pgfqpoint{1.074284in}{0.601225in}}%
\pgfpathclose%
\pgfusepath{stroke,fill}%
\end{pgfscope}%
\begin{pgfscope}%
\pgfpathrectangle{\pgfqpoint{0.772069in}{0.515123in}}{\pgfqpoint{1.937500in}{1.347500in}}%
\pgfusepath{clip}%
\pgfsetbuttcap%
\pgfsetroundjoin%
\definecolor{currentfill}{rgb}{1.000000,0.980392,0.803922}%
\pgfsetfillcolor{currentfill}%
\pgfsetlinewidth{1.003750pt}%
\definecolor{currentstroke}{rgb}{1.000000,0.980392,0.803922}%
\pgfsetstrokecolor{currentstroke}%
\pgfsetdash{}{0pt}%
\pgfpathmoveto{\pgfqpoint{1.143339in}{0.613561in}}%
\pgfpathcurveto{\pgfqpoint{1.154389in}{0.613561in}}{\pgfqpoint{1.164988in}{0.617951in}}{\pgfqpoint{1.172801in}{0.625765in}}%
\pgfpathcurveto{\pgfqpoint{1.180615in}{0.633578in}}{\pgfqpoint{1.185005in}{0.644177in}}{\pgfqpoint{1.185005in}{0.655227in}}%
\pgfpathcurveto{\pgfqpoint{1.185005in}{0.666278in}}{\pgfqpoint{1.180615in}{0.676877in}}{\pgfqpoint{1.172801in}{0.684690in}}%
\pgfpathcurveto{\pgfqpoint{1.164988in}{0.692504in}}{\pgfqpoint{1.154389in}{0.696894in}}{\pgfqpoint{1.143339in}{0.696894in}}%
\pgfpathcurveto{\pgfqpoint{1.132289in}{0.696894in}}{\pgfqpoint{1.121690in}{0.692504in}}{\pgfqpoint{1.113876in}{0.684690in}}%
\pgfpathcurveto{\pgfqpoint{1.106062in}{0.676877in}}{\pgfqpoint{1.101672in}{0.666278in}}{\pgfqpoint{1.101672in}{0.655227in}}%
\pgfpathcurveto{\pgfqpoint{1.101672in}{0.644177in}}{\pgfqpoint{1.106062in}{0.633578in}}{\pgfqpoint{1.113876in}{0.625765in}}%
\pgfpathcurveto{\pgfqpoint{1.121690in}{0.617951in}}{\pgfqpoint{1.132289in}{0.613561in}}{\pgfqpoint{1.143339in}{0.613561in}}%
\pgfpathclose%
\pgfusepath{stroke,fill}%
\end{pgfscope}%
\begin{pgfscope}%
\pgfpathrectangle{\pgfqpoint{0.772069in}{0.515123in}}{\pgfqpoint{1.937500in}{1.347500in}}%
\pgfusepath{clip}%
\pgfsetbuttcap%
\pgfsetroundjoin%
\definecolor{currentfill}{rgb}{1.000000,0.980392,0.803922}%
\pgfsetfillcolor{currentfill}%
\pgfsetlinewidth{1.003750pt}%
\definecolor{currentstroke}{rgb}{1.000000,0.980392,0.803922}%
\pgfsetstrokecolor{currentstroke}%
\pgfsetdash{}{0pt}%
\pgfpathmoveto{\pgfqpoint{1.189375in}{0.625315in}}%
\pgfpathcurveto{\pgfqpoint{1.200426in}{0.625315in}}{\pgfqpoint{1.211025in}{0.629705in}}{\pgfqpoint{1.218838in}{0.637519in}}%
\pgfpathcurveto{\pgfqpoint{1.226652in}{0.645332in}}{\pgfqpoint{1.231042in}{0.655931in}}{\pgfqpoint{1.231042in}{0.666981in}}%
\pgfpathcurveto{\pgfqpoint{1.231042in}{0.678031in}}{\pgfqpoint{1.226652in}{0.688631in}}{\pgfqpoint{1.218838in}{0.696444in}}%
\pgfpathcurveto{\pgfqpoint{1.211025in}{0.704258in}}{\pgfqpoint{1.200426in}{0.708648in}}{\pgfqpoint{1.189375in}{0.708648in}}%
\pgfpathcurveto{\pgfqpoint{1.178325in}{0.708648in}}{\pgfqpoint{1.167726in}{0.704258in}}{\pgfqpoint{1.159913in}{0.696444in}}%
\pgfpathcurveto{\pgfqpoint{1.152099in}{0.688631in}}{\pgfqpoint{1.147709in}{0.678031in}}{\pgfqpoint{1.147709in}{0.666981in}}%
\pgfpathcurveto{\pgfqpoint{1.147709in}{0.655931in}}{\pgfqpoint{1.152099in}{0.645332in}}{\pgfqpoint{1.159913in}{0.637519in}}%
\pgfpathcurveto{\pgfqpoint{1.167726in}{0.629705in}}{\pgfqpoint{1.178325in}{0.625315in}}{\pgfqpoint{1.189375in}{0.625315in}}%
\pgfpathclose%
\pgfusepath{stroke,fill}%
\end{pgfscope}%
\begin{pgfscope}%
\pgfpathrectangle{\pgfqpoint{0.772069in}{0.515123in}}{\pgfqpoint{1.937500in}{1.347500in}}%
\pgfusepath{clip}%
\pgfsetbuttcap%
\pgfsetroundjoin%
\definecolor{currentfill}{rgb}{1.000000,0.980392,0.803922}%
\pgfsetfillcolor{currentfill}%
\pgfsetlinewidth{1.003750pt}%
\definecolor{currentstroke}{rgb}{1.000000,0.980392,0.803922}%
\pgfsetstrokecolor{currentstroke}%
\pgfsetdash{}{0pt}%
\pgfpathmoveto{\pgfqpoint{1.258430in}{0.644284in}}%
\pgfpathcurveto{\pgfqpoint{1.269481in}{0.644284in}}{\pgfqpoint{1.280080in}{0.648674in}}{\pgfqpoint{1.287893in}{0.656488in}}%
\pgfpathcurveto{\pgfqpoint{1.295707in}{0.664301in}}{\pgfqpoint{1.300097in}{0.674900in}}{\pgfqpoint{1.300097in}{0.685950in}}%
\pgfpathcurveto{\pgfqpoint{1.300097in}{0.697001in}}{\pgfqpoint{1.295707in}{0.707600in}}{\pgfqpoint{1.287893in}{0.715413in}}%
\pgfpathcurveto{\pgfqpoint{1.280080in}{0.723227in}}{\pgfqpoint{1.269481in}{0.727617in}}{\pgfqpoint{1.258430in}{0.727617in}}%
\pgfpathcurveto{\pgfqpoint{1.247380in}{0.727617in}}{\pgfqpoint{1.236781in}{0.723227in}}{\pgfqpoint{1.228968in}{0.715413in}}%
\pgfpathcurveto{\pgfqpoint{1.221154in}{0.707600in}}{\pgfqpoint{1.216764in}{0.697001in}}{\pgfqpoint{1.216764in}{0.685950in}}%
\pgfpathcurveto{\pgfqpoint{1.216764in}{0.674900in}}{\pgfqpoint{1.221154in}{0.664301in}}{\pgfqpoint{1.228968in}{0.656488in}}%
\pgfpathcurveto{\pgfqpoint{1.236781in}{0.648674in}}{\pgfqpoint{1.247380in}{0.644284in}}{\pgfqpoint{1.258430in}{0.644284in}}%
\pgfpathclose%
\pgfusepath{stroke,fill}%
\end{pgfscope}%
\begin{pgfscope}%
\pgfpathrectangle{\pgfqpoint{0.772069in}{0.515123in}}{\pgfqpoint{1.937500in}{1.347500in}}%
\pgfusepath{clip}%
\pgfsetbuttcap%
\pgfsetroundjoin%
\definecolor{currentfill}{rgb}{1.000000,0.980392,0.803922}%
\pgfsetfillcolor{currentfill}%
\pgfsetlinewidth{1.003750pt}%
\definecolor{currentstroke}{rgb}{1.000000,0.980392,0.803922}%
\pgfsetstrokecolor{currentstroke}%
\pgfsetdash{}{0pt}%
\pgfpathmoveto{\pgfqpoint{1.327486in}{0.650335in}}%
\pgfpathcurveto{\pgfqpoint{1.338536in}{0.650335in}}{\pgfqpoint{1.349135in}{0.654726in}}{\pgfqpoint{1.356948in}{0.662539in}}%
\pgfpathcurveto{\pgfqpoint{1.364762in}{0.670353in}}{\pgfqpoint{1.369152in}{0.680952in}}{\pgfqpoint{1.369152in}{0.692002in}}%
\pgfpathcurveto{\pgfqpoint{1.369152in}{0.703052in}}{\pgfqpoint{1.364762in}{0.713651in}}{\pgfqpoint{1.356948in}{0.721465in}}%
\pgfpathcurveto{\pgfqpoint{1.349135in}{0.729278in}}{\pgfqpoint{1.338536in}{0.733669in}}{\pgfqpoint{1.327486in}{0.733669in}}%
\pgfpathcurveto{\pgfqpoint{1.316435in}{0.733669in}}{\pgfqpoint{1.305836in}{0.729278in}}{\pgfqpoint{1.298023in}{0.721465in}}%
\pgfpathcurveto{\pgfqpoint{1.290209in}{0.713651in}}{\pgfqpoint{1.285819in}{0.703052in}}{\pgfqpoint{1.285819in}{0.692002in}}%
\pgfpathcurveto{\pgfqpoint{1.285819in}{0.680952in}}{\pgfqpoint{1.290209in}{0.670353in}}{\pgfqpoint{1.298023in}{0.662539in}}%
\pgfpathcurveto{\pgfqpoint{1.305836in}{0.654726in}}{\pgfqpoint{1.316435in}{0.650335in}}{\pgfqpoint{1.327486in}{0.650335in}}%
\pgfpathclose%
\pgfusepath{stroke,fill}%
\end{pgfscope}%
\begin{pgfscope}%
\pgfpathrectangle{\pgfqpoint{0.772069in}{0.515123in}}{\pgfqpoint{1.937500in}{1.347500in}}%
\pgfusepath{clip}%
\pgfsetbuttcap%
\pgfsetroundjoin%
\definecolor{currentfill}{rgb}{1.000000,0.980392,0.803922}%
\pgfsetfillcolor{currentfill}%
\pgfsetlinewidth{1.003750pt}%
\definecolor{currentstroke}{rgb}{1.000000,0.980392,0.803922}%
\pgfsetstrokecolor{currentstroke}%
\pgfsetdash{}{0pt}%
\pgfpathmoveto{\pgfqpoint{1.396541in}{0.662322in}}%
\pgfpathcurveto{\pgfqpoint{1.407591in}{0.662322in}}{\pgfqpoint{1.418190in}{0.666712in}}{\pgfqpoint{1.426003in}{0.674526in}}%
\pgfpathcurveto{\pgfqpoint{1.433817in}{0.682339in}}{\pgfqpoint{1.438207in}{0.692938in}}{\pgfqpoint{1.438207in}{0.703989in}}%
\pgfpathcurveto{\pgfqpoint{1.438207in}{0.715039in}}{\pgfqpoint{1.433817in}{0.725638in}}{\pgfqpoint{1.426003in}{0.733451in}}%
\pgfpathcurveto{\pgfqpoint{1.418190in}{0.741265in}}{\pgfqpoint{1.407591in}{0.745655in}}{\pgfqpoint{1.396541in}{0.745655in}}%
\pgfpathcurveto{\pgfqpoint{1.385490in}{0.745655in}}{\pgfqpoint{1.374891in}{0.741265in}}{\pgfqpoint{1.367078in}{0.733451in}}%
\pgfpathcurveto{\pgfqpoint{1.359264in}{0.725638in}}{\pgfqpoint{1.354874in}{0.715039in}}{\pgfqpoint{1.354874in}{0.703989in}}%
\pgfpathcurveto{\pgfqpoint{1.354874in}{0.692938in}}{\pgfqpoint{1.359264in}{0.682339in}}{\pgfqpoint{1.367078in}{0.674526in}}%
\pgfpathcurveto{\pgfqpoint{1.374891in}{0.666712in}}{\pgfqpoint{1.385490in}{0.662322in}}{\pgfqpoint{1.396541in}{0.662322in}}%
\pgfpathclose%
\pgfusepath{stroke,fill}%
\end{pgfscope}%
\begin{pgfscope}%
\pgfpathrectangle{\pgfqpoint{0.772069in}{0.515123in}}{\pgfqpoint{1.937500in}{1.347500in}}%
\pgfusepath{clip}%
\pgfsetbuttcap%
\pgfsetroundjoin%
\definecolor{currentfill}{rgb}{1.000000,0.980392,0.803922}%
\pgfsetfillcolor{currentfill}%
\pgfsetlinewidth{1.003750pt}%
\definecolor{currentstroke}{rgb}{1.000000,0.980392,0.803922}%
\pgfsetstrokecolor{currentstroke}%
\pgfsetdash{}{0pt}%
\pgfpathmoveto{\pgfqpoint{1.442577in}{0.674658in}}%
\pgfpathcurveto{\pgfqpoint{1.453627in}{0.674658in}}{\pgfqpoint{1.464226in}{0.679048in}}{\pgfqpoint{1.472040in}{0.686862in}}%
\pgfpathcurveto{\pgfqpoint{1.479854in}{0.694675in}}{\pgfqpoint{1.484244in}{0.705274in}}{\pgfqpoint{1.484244in}{0.716324in}}%
\pgfpathcurveto{\pgfqpoint{1.484244in}{0.727374in}}{\pgfqpoint{1.479854in}{0.737974in}}{\pgfqpoint{1.472040in}{0.745787in}}%
\pgfpathcurveto{\pgfqpoint{1.464226in}{0.753601in}}{\pgfqpoint{1.453627in}{0.757991in}}{\pgfqpoint{1.442577in}{0.757991in}}%
\pgfpathcurveto{\pgfqpoint{1.431527in}{0.757991in}}{\pgfqpoint{1.420928in}{0.753601in}}{\pgfqpoint{1.413114in}{0.745787in}}%
\pgfpathcurveto{\pgfqpoint{1.405301in}{0.737974in}}{\pgfqpoint{1.400911in}{0.727374in}}{\pgfqpoint{1.400911in}{0.716324in}}%
\pgfpathcurveto{\pgfqpoint{1.400911in}{0.705274in}}{\pgfqpoint{1.405301in}{0.694675in}}{\pgfqpoint{1.413114in}{0.686862in}}%
\pgfpathcurveto{\pgfqpoint{1.420928in}{0.679048in}}{\pgfqpoint{1.431527in}{0.674658in}}{\pgfqpoint{1.442577in}{0.674658in}}%
\pgfpathclose%
\pgfusepath{stroke,fill}%
\end{pgfscope}%
\begin{pgfscope}%
\pgfpathrectangle{\pgfqpoint{0.772069in}{0.515123in}}{\pgfqpoint{1.937500in}{1.347500in}}%
\pgfusepath{clip}%
\pgfsetbuttcap%
\pgfsetroundjoin%
\definecolor{currentfill}{rgb}{0.596078,0.984314,0.596078}%
\pgfsetfillcolor{currentfill}%
\pgfsetlinewidth{1.003750pt}%
\definecolor{currentstroke}{rgb}{0.596078,0.984314,0.596078}%
\pgfsetstrokecolor{currentstroke}%
\pgfsetdash{}{0pt}%
\pgfpathmoveto{\pgfqpoint{0.890137in}{0.641491in}}%
\pgfpathcurveto{\pgfqpoint{0.901187in}{0.641491in}}{\pgfqpoint{0.911786in}{0.645881in}}{\pgfqpoint{0.919600in}{0.653695in}}%
\pgfpathcurveto{\pgfqpoint{0.927413in}{0.661508in}}{\pgfqpoint{0.931804in}{0.672107in}}{\pgfqpoint{0.931804in}{0.683157in}}%
\pgfpathcurveto{\pgfqpoint{0.931804in}{0.694208in}}{\pgfqpoint{0.927413in}{0.704807in}}{\pgfqpoint{0.919600in}{0.712620in}}%
\pgfpathcurveto{\pgfqpoint{0.911786in}{0.720434in}}{\pgfqpoint{0.901187in}{0.724824in}}{\pgfqpoint{0.890137in}{0.724824in}}%
\pgfpathcurveto{\pgfqpoint{0.879087in}{0.724824in}}{\pgfqpoint{0.868488in}{0.720434in}}{\pgfqpoint{0.860674in}{0.712620in}}%
\pgfpathcurveto{\pgfqpoint{0.852860in}{0.704807in}}{\pgfqpoint{0.848470in}{0.694208in}}{\pgfqpoint{0.848470in}{0.683157in}}%
\pgfpathcurveto{\pgfqpoint{0.848470in}{0.672107in}}{\pgfqpoint{0.852860in}{0.661508in}}{\pgfqpoint{0.860674in}{0.653695in}}%
\pgfpathcurveto{\pgfqpoint{0.868488in}{0.645881in}}{\pgfqpoint{0.879087in}{0.641491in}}{\pgfqpoint{0.890137in}{0.641491in}}%
\pgfpathclose%
\pgfusepath{stroke,fill}%
\end{pgfscope}%
\begin{pgfscope}%
\pgfpathrectangle{\pgfqpoint{0.772069in}{0.515123in}}{\pgfqpoint{1.937500in}{1.347500in}}%
\pgfusepath{clip}%
\pgfsetbuttcap%
\pgfsetroundjoin%
\definecolor{currentfill}{rgb}{0.596078,0.984314,0.596078}%
\pgfsetfillcolor{currentfill}%
\pgfsetlinewidth{1.003750pt}%
\definecolor{currentstroke}{rgb}{0.596078,0.984314,0.596078}%
\pgfsetstrokecolor{currentstroke}%
\pgfsetdash{}{0pt}%
\pgfpathmoveto{\pgfqpoint{0.959192in}{0.725630in}}%
\pgfpathcurveto{\pgfqpoint{0.970242in}{0.725630in}}{\pgfqpoint{0.980841in}{0.730020in}}{\pgfqpoint{0.988655in}{0.737834in}}%
\pgfpathcurveto{\pgfqpoint{0.996468in}{0.745647in}}{\pgfqpoint{1.000859in}{0.756246in}}{\pgfqpoint{1.000859in}{0.767297in}}%
\pgfpathcurveto{\pgfqpoint{1.000859in}{0.778347in}}{\pgfqpoint{0.996468in}{0.788946in}}{\pgfqpoint{0.988655in}{0.796759in}}%
\pgfpathcurveto{\pgfqpoint{0.980841in}{0.804573in}}{\pgfqpoint{0.970242in}{0.808963in}}{\pgfqpoint{0.959192in}{0.808963in}}%
\pgfpathcurveto{\pgfqpoint{0.948142in}{0.808963in}}{\pgfqpoint{0.937543in}{0.804573in}}{\pgfqpoint{0.929729in}{0.796759in}}%
\pgfpathcurveto{\pgfqpoint{0.921916in}{0.788946in}}{\pgfqpoint{0.917525in}{0.778347in}}{\pgfqpoint{0.917525in}{0.767297in}}%
\pgfpathcurveto{\pgfqpoint{0.917525in}{0.756246in}}{\pgfqpoint{0.921916in}{0.745647in}}{\pgfqpoint{0.929729in}{0.737834in}}%
\pgfpathcurveto{\pgfqpoint{0.937543in}{0.730020in}}{\pgfqpoint{0.948142in}{0.725630in}}{\pgfqpoint{0.959192in}{0.725630in}}%
\pgfpathclose%
\pgfusepath{stroke,fill}%
\end{pgfscope}%
\begin{pgfscope}%
\pgfpathrectangle{\pgfqpoint{0.772069in}{0.515123in}}{\pgfqpoint{1.937500in}{1.347500in}}%
\pgfusepath{clip}%
\pgfsetbuttcap%
\pgfsetroundjoin%
\definecolor{currentfill}{rgb}{0.596078,0.984314,0.596078}%
\pgfsetfillcolor{currentfill}%
\pgfsetlinewidth{1.003750pt}%
\definecolor{currentstroke}{rgb}{0.596078,0.984314,0.596078}%
\pgfsetstrokecolor{currentstroke}%
\pgfsetdash{}{0pt}%
\pgfpathmoveto{\pgfqpoint{1.005229in}{0.813842in}}%
\pgfpathcurveto{\pgfqpoint{1.016279in}{0.813842in}}{\pgfqpoint{1.026878in}{0.818233in}}{\pgfqpoint{1.034691in}{0.826046in}}%
\pgfpathcurveto{\pgfqpoint{1.042505in}{0.833860in}}{\pgfqpoint{1.046895in}{0.844459in}}{\pgfqpoint{1.046895in}{0.855509in}}%
\pgfpathcurveto{\pgfqpoint{1.046895in}{0.866559in}}{\pgfqpoint{1.042505in}{0.877158in}}{\pgfqpoint{1.034691in}{0.884972in}}%
\pgfpathcurveto{\pgfqpoint{1.026878in}{0.892785in}}{\pgfqpoint{1.016279in}{0.897176in}}{\pgfqpoint{1.005229in}{0.897176in}}%
\pgfpathcurveto{\pgfqpoint{0.994178in}{0.897176in}}{\pgfqpoint{0.983579in}{0.892785in}}{\pgfqpoint{0.975766in}{0.884972in}}%
\pgfpathcurveto{\pgfqpoint{0.967952in}{0.877158in}}{\pgfqpoint{0.963562in}{0.866559in}}{\pgfqpoint{0.963562in}{0.855509in}}%
\pgfpathcurveto{\pgfqpoint{0.963562in}{0.844459in}}{\pgfqpoint{0.967952in}{0.833860in}}{\pgfqpoint{0.975766in}{0.826046in}}%
\pgfpathcurveto{\pgfqpoint{0.983579in}{0.818233in}}{\pgfqpoint{0.994178in}{0.813842in}}{\pgfqpoint{1.005229in}{0.813842in}}%
\pgfpathclose%
\pgfusepath{stroke,fill}%
\end{pgfscope}%
\begin{pgfscope}%
\pgfpathrectangle{\pgfqpoint{0.772069in}{0.515123in}}{\pgfqpoint{1.937500in}{1.347500in}}%
\pgfusepath{clip}%
\pgfsetbuttcap%
\pgfsetroundjoin%
\definecolor{currentfill}{rgb}{0.596078,0.984314,0.596078}%
\pgfsetfillcolor{currentfill}%
\pgfsetlinewidth{1.003750pt}%
\definecolor{currentstroke}{rgb}{0.596078,0.984314,0.596078}%
\pgfsetstrokecolor{currentstroke}%
\pgfsetdash{}{0pt}%
\pgfpathmoveto{\pgfqpoint{1.074284in}{0.903451in}}%
\pgfpathcurveto{\pgfqpoint{1.085334in}{0.903451in}}{\pgfqpoint{1.095933in}{0.907841in}}{\pgfqpoint{1.103746in}{0.915655in}}%
\pgfpathcurveto{\pgfqpoint{1.111560in}{0.923469in}}{\pgfqpoint{1.115950in}{0.934068in}}{\pgfqpoint{1.115950in}{0.945118in}}%
\pgfpathcurveto{\pgfqpoint{1.115950in}{0.956168in}}{\pgfqpoint{1.111560in}{0.966767in}}{\pgfqpoint{1.103746in}{0.974580in}}%
\pgfpathcurveto{\pgfqpoint{1.095933in}{0.982394in}}{\pgfqpoint{1.085334in}{0.986784in}}{\pgfqpoint{1.074284in}{0.986784in}}%
\pgfpathcurveto{\pgfqpoint{1.063234in}{0.986784in}}{\pgfqpoint{1.052635in}{0.982394in}}{\pgfqpoint{1.044821in}{0.974580in}}%
\pgfpathcurveto{\pgfqpoint{1.037007in}{0.966767in}}{\pgfqpoint{1.032617in}{0.956168in}}{\pgfqpoint{1.032617in}{0.945118in}}%
\pgfpathcurveto{\pgfqpoint{1.032617in}{0.934068in}}{\pgfqpoint{1.037007in}{0.923469in}}{\pgfqpoint{1.044821in}{0.915655in}}%
\pgfpathcurveto{\pgfqpoint{1.052635in}{0.907841in}}{\pgfqpoint{1.063234in}{0.903451in}}{\pgfqpoint{1.074284in}{0.903451in}}%
\pgfpathclose%
\pgfusepath{stroke,fill}%
\end{pgfscope}%
\begin{pgfscope}%
\pgfpathrectangle{\pgfqpoint{0.772069in}{0.515123in}}{\pgfqpoint{1.937500in}{1.347500in}}%
\pgfusepath{clip}%
\pgfsetbuttcap%
\pgfsetroundjoin%
\definecolor{currentfill}{rgb}{0.596078,0.984314,0.596078}%
\pgfsetfillcolor{currentfill}%
\pgfsetlinewidth{1.003750pt}%
\definecolor{currentstroke}{rgb}{0.596078,0.984314,0.596078}%
\pgfsetstrokecolor{currentstroke}%
\pgfsetdash{}{0pt}%
\pgfpathmoveto{\pgfqpoint{1.143339in}{0.991896in}}%
\pgfpathcurveto{\pgfqpoint{1.154389in}{0.991896in}}{\pgfqpoint{1.164988in}{0.996286in}}{\pgfqpoint{1.172801in}{1.004100in}}%
\pgfpathcurveto{\pgfqpoint{1.180615in}{1.011914in}}{\pgfqpoint{1.185005in}{1.022513in}}{\pgfqpoint{1.185005in}{1.033563in}}%
\pgfpathcurveto{\pgfqpoint{1.185005in}{1.044613in}}{\pgfqpoint{1.180615in}{1.055212in}}{\pgfqpoint{1.172801in}{1.063026in}}%
\pgfpathcurveto{\pgfqpoint{1.164988in}{1.070839in}}{\pgfqpoint{1.154389in}{1.075229in}}{\pgfqpoint{1.143339in}{1.075229in}}%
\pgfpathcurveto{\pgfqpoint{1.132289in}{1.075229in}}{\pgfqpoint{1.121690in}{1.070839in}}{\pgfqpoint{1.113876in}{1.063026in}}%
\pgfpathcurveto{\pgfqpoint{1.106062in}{1.055212in}}{\pgfqpoint{1.101672in}{1.044613in}}{\pgfqpoint{1.101672in}{1.033563in}}%
\pgfpathcurveto{\pgfqpoint{1.101672in}{1.022513in}}{\pgfqpoint{1.106062in}{1.011914in}}{\pgfqpoint{1.113876in}{1.004100in}}%
\pgfpathcurveto{\pgfqpoint{1.121690in}{0.996286in}}{\pgfqpoint{1.132289in}{0.991896in}}{\pgfqpoint{1.143339in}{0.991896in}}%
\pgfpathclose%
\pgfusepath{stroke,fill}%
\end{pgfscope}%
\begin{pgfscope}%
\pgfpathrectangle{\pgfqpoint{0.772069in}{0.515123in}}{\pgfqpoint{1.937500in}{1.347500in}}%
\pgfusepath{clip}%
\pgfsetbuttcap%
\pgfsetroundjoin%
\definecolor{currentfill}{rgb}{0.596078,0.984314,0.596078}%
\pgfsetfillcolor{currentfill}%
\pgfsetlinewidth{1.003750pt}%
\definecolor{currentstroke}{rgb}{0.596078,0.984314,0.596078}%
\pgfsetstrokecolor{currentstroke}%
\pgfsetdash{}{0pt}%
\pgfpathmoveto{\pgfqpoint{1.189375in}{1.075686in}}%
\pgfpathcurveto{\pgfqpoint{1.200426in}{1.075686in}}{\pgfqpoint{1.211025in}{1.080076in}}{\pgfqpoint{1.218838in}{1.087890in}}%
\pgfpathcurveto{\pgfqpoint{1.226652in}{1.095704in}}{\pgfqpoint{1.231042in}{1.106303in}}{\pgfqpoint{1.231042in}{1.117353in}}%
\pgfpathcurveto{\pgfqpoint{1.231042in}{1.128403in}}{\pgfqpoint{1.226652in}{1.139002in}}{\pgfqpoint{1.218838in}{1.146816in}}%
\pgfpathcurveto{\pgfqpoint{1.211025in}{1.154629in}}{\pgfqpoint{1.200426in}{1.159019in}}{\pgfqpoint{1.189375in}{1.159019in}}%
\pgfpathcurveto{\pgfqpoint{1.178325in}{1.159019in}}{\pgfqpoint{1.167726in}{1.154629in}}{\pgfqpoint{1.159913in}{1.146816in}}%
\pgfpathcurveto{\pgfqpoint{1.152099in}{1.139002in}}{\pgfqpoint{1.147709in}{1.128403in}}{\pgfqpoint{1.147709in}{1.117353in}}%
\pgfpathcurveto{\pgfqpoint{1.147709in}{1.106303in}}{\pgfqpoint{1.152099in}{1.095704in}}{\pgfqpoint{1.159913in}{1.087890in}}%
\pgfpathcurveto{\pgfqpoint{1.167726in}{1.080076in}}{\pgfqpoint{1.178325in}{1.075686in}}{\pgfqpoint{1.189375in}{1.075686in}}%
\pgfpathclose%
\pgfusepath{stroke,fill}%
\end{pgfscope}%
\begin{pgfscope}%
\pgfpathrectangle{\pgfqpoint{0.772069in}{0.515123in}}{\pgfqpoint{1.937500in}{1.347500in}}%
\pgfusepath{clip}%
\pgfsetbuttcap%
\pgfsetroundjoin%
\definecolor{currentfill}{rgb}{0.596078,0.984314,0.596078}%
\pgfsetfillcolor{currentfill}%
\pgfsetlinewidth{1.003750pt}%
\definecolor{currentstroke}{rgb}{0.596078,0.984314,0.596078}%
\pgfsetstrokecolor{currentstroke}%
\pgfsetdash{}{0pt}%
\pgfpathmoveto{\pgfqpoint{1.258430in}{1.166459in}}%
\pgfpathcurveto{\pgfqpoint{1.269481in}{1.166459in}}{\pgfqpoint{1.280080in}{1.170849in}}{\pgfqpoint{1.287893in}{1.178663in}}%
\pgfpathcurveto{\pgfqpoint{1.295707in}{1.186476in}}{\pgfqpoint{1.300097in}{1.197075in}}{\pgfqpoint{1.300097in}{1.208125in}}%
\pgfpathcurveto{\pgfqpoint{1.300097in}{1.219175in}}{\pgfqpoint{1.295707in}{1.229774in}}{\pgfqpoint{1.287893in}{1.237588in}}%
\pgfpathcurveto{\pgfqpoint{1.280080in}{1.245402in}}{\pgfqpoint{1.269481in}{1.249792in}}{\pgfqpoint{1.258430in}{1.249792in}}%
\pgfpathcurveto{\pgfqpoint{1.247380in}{1.249792in}}{\pgfqpoint{1.236781in}{1.245402in}}{\pgfqpoint{1.228968in}{1.237588in}}%
\pgfpathcurveto{\pgfqpoint{1.221154in}{1.229774in}}{\pgfqpoint{1.216764in}{1.219175in}}{\pgfqpoint{1.216764in}{1.208125in}}%
\pgfpathcurveto{\pgfqpoint{1.216764in}{1.197075in}}{\pgfqpoint{1.221154in}{1.186476in}}{\pgfqpoint{1.228968in}{1.178663in}}%
\pgfpathcurveto{\pgfqpoint{1.236781in}{1.170849in}}{\pgfqpoint{1.247380in}{1.166459in}}{\pgfqpoint{1.258430in}{1.166459in}}%
\pgfpathclose%
\pgfusepath{stroke,fill}%
\end{pgfscope}%
\begin{pgfscope}%
\pgfpathrectangle{\pgfqpoint{0.772069in}{0.515123in}}{\pgfqpoint{1.937500in}{1.347500in}}%
\pgfusepath{clip}%
\pgfsetbuttcap%
\pgfsetroundjoin%
\definecolor{currentfill}{rgb}{0.596078,0.984314,0.596078}%
\pgfsetfillcolor{currentfill}%
\pgfsetlinewidth{1.003750pt}%
\definecolor{currentstroke}{rgb}{0.596078,0.984314,0.596078}%
\pgfsetstrokecolor{currentstroke}%
\pgfsetdash{}{0pt}%
\pgfpathmoveto{\pgfqpoint{1.327486in}{1.253740in}}%
\pgfpathcurveto{\pgfqpoint{1.338536in}{1.253740in}}{\pgfqpoint{1.349135in}{1.258130in}}{\pgfqpoint{1.356948in}{1.265944in}}%
\pgfpathcurveto{\pgfqpoint{1.364762in}{1.273757in}}{\pgfqpoint{1.369152in}{1.284356in}}{\pgfqpoint{1.369152in}{1.295407in}}%
\pgfpathcurveto{\pgfqpoint{1.369152in}{1.306457in}}{\pgfqpoint{1.364762in}{1.317056in}}{\pgfqpoint{1.356948in}{1.324869in}}%
\pgfpathcurveto{\pgfqpoint{1.349135in}{1.332683in}}{\pgfqpoint{1.338536in}{1.337073in}}{\pgfqpoint{1.327486in}{1.337073in}}%
\pgfpathcurveto{\pgfqpoint{1.316435in}{1.337073in}}{\pgfqpoint{1.305836in}{1.332683in}}{\pgfqpoint{1.298023in}{1.324869in}}%
\pgfpathcurveto{\pgfqpoint{1.290209in}{1.317056in}}{\pgfqpoint{1.285819in}{1.306457in}}{\pgfqpoint{1.285819in}{1.295407in}}%
\pgfpathcurveto{\pgfqpoint{1.285819in}{1.284356in}}{\pgfqpoint{1.290209in}{1.273757in}}{\pgfqpoint{1.298023in}{1.265944in}}%
\pgfpathcurveto{\pgfqpoint{1.305836in}{1.258130in}}{\pgfqpoint{1.316435in}{1.253740in}}{\pgfqpoint{1.327486in}{1.253740in}}%
\pgfpathclose%
\pgfusepath{stroke,fill}%
\end{pgfscope}%
\begin{pgfscope}%
\pgfpathrectangle{\pgfqpoint{0.772069in}{0.515123in}}{\pgfqpoint{1.937500in}{1.347500in}}%
\pgfusepath{clip}%
\pgfsetbuttcap%
\pgfsetroundjoin%
\definecolor{currentfill}{rgb}{0.596078,0.984314,0.596078}%
\pgfsetfillcolor{currentfill}%
\pgfsetlinewidth{1.003750pt}%
\definecolor{currentstroke}{rgb}{0.596078,0.984314,0.596078}%
\pgfsetstrokecolor{currentstroke}%
\pgfsetdash{}{0pt}%
\pgfpathmoveto{\pgfqpoint{1.396541in}{1.339857in}}%
\pgfpathcurveto{\pgfqpoint{1.407591in}{1.339857in}}{\pgfqpoint{1.418190in}{1.344248in}}{\pgfqpoint{1.426003in}{1.352061in}}%
\pgfpathcurveto{\pgfqpoint{1.433817in}{1.359875in}}{\pgfqpoint{1.438207in}{1.370474in}}{\pgfqpoint{1.438207in}{1.381524in}}%
\pgfpathcurveto{\pgfqpoint{1.438207in}{1.392574in}}{\pgfqpoint{1.433817in}{1.403173in}}{\pgfqpoint{1.426003in}{1.410987in}}%
\pgfpathcurveto{\pgfqpoint{1.418190in}{1.418801in}}{\pgfqpoint{1.407591in}{1.423191in}}{\pgfqpoint{1.396541in}{1.423191in}}%
\pgfpathcurveto{\pgfqpoint{1.385490in}{1.423191in}}{\pgfqpoint{1.374891in}{1.418801in}}{\pgfqpoint{1.367078in}{1.410987in}}%
\pgfpathcurveto{\pgfqpoint{1.359264in}{1.403173in}}{\pgfqpoint{1.354874in}{1.392574in}}{\pgfqpoint{1.354874in}{1.381524in}}%
\pgfpathcurveto{\pgfqpoint{1.354874in}{1.370474in}}{\pgfqpoint{1.359264in}{1.359875in}}{\pgfqpoint{1.367078in}{1.352061in}}%
\pgfpathcurveto{\pgfqpoint{1.374891in}{1.344248in}}{\pgfqpoint{1.385490in}{1.339857in}}{\pgfqpoint{1.396541in}{1.339857in}}%
\pgfpathclose%
\pgfusepath{stroke,fill}%
\end{pgfscope}%
\begin{pgfscope}%
\pgfpathrectangle{\pgfqpoint{0.772069in}{0.515123in}}{\pgfqpoint{1.937500in}{1.347500in}}%
\pgfusepath{clip}%
\pgfsetbuttcap%
\pgfsetroundjoin%
\definecolor{currentfill}{rgb}{0.596078,0.984314,0.596078}%
\pgfsetfillcolor{currentfill}%
\pgfsetlinewidth{1.003750pt}%
\definecolor{currentstroke}{rgb}{0.596078,0.984314,0.596078}%
\pgfsetstrokecolor{currentstroke}%
\pgfsetdash{}{0pt}%
\pgfpathmoveto{\pgfqpoint{1.442577in}{1.428302in}}%
\pgfpathcurveto{\pgfqpoint{1.453627in}{1.428302in}}{\pgfqpoint{1.464226in}{1.432693in}}{\pgfqpoint{1.472040in}{1.440506in}}%
\pgfpathcurveto{\pgfqpoint{1.479854in}{1.448320in}}{\pgfqpoint{1.484244in}{1.458919in}}{\pgfqpoint{1.484244in}{1.469969in}}%
\pgfpathcurveto{\pgfqpoint{1.484244in}{1.481019in}}{\pgfqpoint{1.479854in}{1.491618in}}{\pgfqpoint{1.472040in}{1.499432in}}%
\pgfpathcurveto{\pgfqpoint{1.464226in}{1.507246in}}{\pgfqpoint{1.453627in}{1.511636in}}{\pgfqpoint{1.442577in}{1.511636in}}%
\pgfpathcurveto{\pgfqpoint{1.431527in}{1.511636in}}{\pgfqpoint{1.420928in}{1.507246in}}{\pgfqpoint{1.413114in}{1.499432in}}%
\pgfpathcurveto{\pgfqpoint{1.405301in}{1.491618in}}{\pgfqpoint{1.400911in}{1.481019in}}{\pgfqpoint{1.400911in}{1.469969in}}%
\pgfpathcurveto{\pgfqpoint{1.400911in}{1.458919in}}{\pgfqpoint{1.405301in}{1.448320in}}{\pgfqpoint{1.413114in}{1.440506in}}%
\pgfpathcurveto{\pgfqpoint{1.420928in}{1.432693in}}{\pgfqpoint{1.431527in}{1.428302in}}{\pgfqpoint{1.442577in}{1.428302in}}%
\pgfpathclose%
\pgfusepath{stroke,fill}%
\end{pgfscope}%
\begin{pgfscope}%
\pgfsetrectcap%
\pgfsetmiterjoin%
\pgfsetlinewidth{0.803000pt}%
\definecolor{currentstroke}{rgb}{0.000000,0.000000,0.000000}%
\pgfsetstrokecolor{currentstroke}%
\pgfsetdash{}{0pt}%
\pgfpathmoveto{\pgfqpoint{0.772069in}{0.515123in}}%
\pgfpathlineto{\pgfqpoint{0.772069in}{1.862623in}}%
\pgfusepath{stroke}%
\end{pgfscope}%
\begin{pgfscope}%
\pgfsetrectcap%
\pgfsetmiterjoin%
\pgfsetlinewidth{0.803000pt}%
\definecolor{currentstroke}{rgb}{0.000000,0.000000,0.000000}%
\pgfsetstrokecolor{currentstroke}%
\pgfsetdash{}{0pt}%
\pgfpathmoveto{\pgfqpoint{2.709569in}{0.515123in}}%
\pgfpathlineto{\pgfqpoint{2.709569in}{1.862623in}}%
\pgfusepath{stroke}%
\end{pgfscope}%
\begin{pgfscope}%
\pgfsetrectcap%
\pgfsetmiterjoin%
\pgfsetlinewidth{0.803000pt}%
\definecolor{currentstroke}{rgb}{0.000000,0.000000,0.000000}%
\pgfsetstrokecolor{currentstroke}%
\pgfsetdash{}{0pt}%
\pgfpathmoveto{\pgfqpoint{0.772069in}{0.515123in}}%
\pgfpathlineto{\pgfqpoint{2.709569in}{0.515123in}}%
\pgfusepath{stroke}%
\end{pgfscope}%
\begin{pgfscope}%
\pgfsetrectcap%
\pgfsetmiterjoin%
\pgfsetlinewidth{0.803000pt}%
\definecolor{currentstroke}{rgb}{0.000000,0.000000,0.000000}%
\pgfsetstrokecolor{currentstroke}%
\pgfsetdash{}{0pt}%
\pgfpathmoveto{\pgfqpoint{0.772069in}{1.862623in}}%
\pgfpathlineto{\pgfqpoint{2.709569in}{1.862623in}}%
\pgfusepath{stroke}%
\end{pgfscope}%
\end{pgfpicture}%
\makeatother%
\endgroup%

    \caption{Voltajes (\textcolor{Blue}{$V$}, \textcolor{Red}{$V_1$}, \textcolor{Yellow}{$V_{23}$}, \textcolor{Green}{$V_4$}) frente a intensidades (\textcolor{Black}{$I$}, \textcolor{DarkGrey}{$I_1$}, \textcolor{Grey}{$I_2$}) con regresión lineal}
  \end{figure}

  En esta última regresión nos encontramos problemas que hasta ahora no habían aparecido. Pese a que alguna de las rectas tiene coeficientes de regresión de hasta $r = 0,99998$ ($V$ frente a $I_1$ y $V_4$ frente a $I_1$), otras no se encuentran tan próximas. Son varias las que sólo tienen un ajuste de dos nueves, cómo $V_1$ frente a $I$ ($r = 0,997$) o $V_{23}$ frente a $I_2$ ($r = 0,997$). Discutiremos esto en el apartado de conclusiones.


  \newpage
  \section{Conclusiones}

  \subsection{Medida de resistencias}

  Comprobamos que los valores nominales de las resistencias (\ref{tb:valnomres}) y las mediciones experimentales de las mismas (\ref{tab:res}) se encuentran dentro de la incertidumbre marcada por el fabricante.

  \subsection{Ley de Ohm}

  Hemos verificado la Ley de Ohm (\ref{ec:ohm}) en múltiples ocasiones, ya que en nuestras gráficas de Voltaje (\textit{V}) frente a Intensidad (\textit{I}) la linea obtenida mediante ajuste por mínimos cuadrados cumplía varias condiciones:

  \begin{itemize}[label=$-$]
    \item Es una recta que prácticamente pasa por el origen. Los términos \textit{a} (Ordenada en el origen) son despreciables frente a las magnitudes de \textit{V}. Lo calculamos manualmente en \ref{v:ohm} y \ref{v:serie} para los dos primeros circuitos, y en los apartados \ref{v:paralelo} y \ref{v:mixto} para los segundos. En todos los casos utilizamos la ecuación \ref{ec:a} para calcular el valor de \textit{a}.
    \item La pendiente de esta recta, \textit{b}, coincide con el valor teórico de la resistencia equivalente (\ref{ec:reseq} en serie y \ref{ec:resparalelo} en paralelo) y del valor de la medida del polímetro dentro del márgen de incertidumbre. El caso del circuito en serie (\ref{sec:reseqserie}) es una excepción, ya que el valor de la resistencia equivalente y la medida del polímetro no entran dentro del márgen de error, pero por un orden de magnitud de $10^2$, mientras que la medida es del orden de $10^5$. Consideraremos que se trata de un error en las condiciones del experimento, por ejemplo, a causa de la humedad, la temperatura, etc... En el resto de secciones se verifica perfectamente, y podemos ver los ajustes y valores de \textit{b} en las secciones \ref{sec:ajusteminres}, \ref{sec:ajusteminimoscuadradosserie}, \ref{v:paralelo} y \ref{v:mixto}.
  \end{itemize}

  \subsection{Leyes de Kirchhoff}

  También hemos verificados las Leyes de Kirchhoff (\ref{ec:kriser} en serie y \ref{ec:kripar} en paralelo) en los distintos circuitos:

  \begin{itemize}[label=$-$]
    \item En el circuito en serie, la suma de los voltajes en bornes de cada resistencia fue aproximadamente igual al voltaje total del circuito para cada medición, con incertidumbres atribuibles a los aparatos de medición o a los cables de conexión.
    \item En el circuito en paralelo, la suma de las intensidades en bornes de cada resistencia se aproximó a la intensidad total del circuito para cada medición, de nuevo con incertidumbres atribuibles a las técnicas de medición.
    \item En el circuito mixto, la suma de los voltajes en bornes de $R_1$, $R_{23}$ y $R_4$ coincide con el voltaje total, mientras que las intensidades en bornes de las resistencias $R_2$ y $R_3$ suman el valor de la intensidad total del circuito.
  \end{itemize}

  \subsection{Otras consideraciones}

  Por último, es necesario discutir una menor precisión de ajuste en el apartado \ref{v:mixto}. Mientras que el resto de regresiones lineales tenían un coeficiente (\ref{ec:r}) de al menos cuatro nueves, en algunas de las rectas que ajustamos en la última sección la precisión disminuyó a dos. Si bien la recta principal (V total frente a I total) cuenta con un $r = 0,99993$ que es bastante decente, otras cómo $V_1$ frente a $I$ o $V_{23}$ frente a $I_2$ tan solo llegan a $r = 0,997$. Esto se puede deber tanto a errores causados por las condiciones del laboratorio tras un periodo de tiempo prolongado (Mayor temperatura por la agrupación de tantos cuerpos, sudoración en las manos...), y también a algún error humano debido a la repetición de mediciones, pese a que tratamos de evitarlos en la medida de lo posible. De todas maneras, la desviación sobre el resultado no es crítica, y los datos siguen siendo coherentes (excepto el valor tres de $V_1$ que es menor que el de $V_23$ cuando debería de ser al revés), pero queríamos dejar constancia de esta disminución en la precisión del último experimento.

  En general, consideraríamos la práctica exitosa, ya que comprobamos tanto la \textbf{Ley de Ohm} cómo la \textbf{Ley de Kirchhoff} para la asociación de resistencias, y los resultados fueron satisfactorios.




  \newpage
  \part{Corriente Alterna}

  \section{Objetivos}
  \begin{itemize}[label=$-$]
    \item Comparar los valores teóricos y experimentales de la frecuencia de corte en un circuito RC.
    \item Medir la variación del módulo de la impedancia al variar la frecuencia.
    \item Obtener la fase de la impedancia utilizando el método de las dos trazas
  \end{itemize}

  \section{Circuito RC}

  En esta práctica trabajaremos con un circuito RC (Resitencia-Condensador) al que le suminstraremos corriente alterna desde un generador de señales senosoidales.

  \begin{figure}[H]
    \centering
    \begin{circuitikz}[european]
      \draw (0,0) to[sV, l=$V_{in}$] (0,3)
      to[R, l=R] (3,3)
      to[capacitor, l=C] (3,0);
      \draw (0,0) node[tlground]{};
      \draw (3,0) node[tlground]{};
      \draw (1.5,0) node[]{$V_{out}$};
    \end{circuitikz}
    \caption{Circuito RC}
  \end{figure}

  \subsection{Cálculo de $f_c$ y $T$}

  En este circuito se conectan una resistencia y un condensador en serie. El potencial total del circuito ($V(t)$) coincide con la suma de los dos potenciales en bornes:
  \begin{equation}
    V(t) = V_R(t) + V_C(t)
  \end{equation}
  Mientras tanto, la intensidad viene dada por una fórmula distinta:
  \begin{equation}
    I(t) = I_m sen(wt + \varphi)
  \end{equation}
  En esta ecuación la fase ($\varphi$), la amplitud de la intensidad ($I_m$) y el módulo de la magnitud compleja, también llamado Impedancia ($Z$), están definidas de la siguiente manera:
  \begin{gather}
    \varphi = arctg\left(-\frac{X_C}{R}\right) = arctg\left(-\frac{1}{\omega RC}\right) \label{ec:faseimp} \\
    I_m = \frac{V_m}{Z} \\
    \hat{Z} = R + \frac{1}{j\omega C} \label{ec:zhat} \\
    Z = \sqrt{R^2+X_C^2} = \sqrt{R^2+\frac{1}{w^2C^2}} \label{ec:moduloimp1}
  \end{gather}
  Con esto podemos determinar la \textbf{frecuencia de corte} ($f_c$), que podemos calcular utilizando la ecuación \ref{ec:zhat}. Para esta frecuencia $R = X_C$.
  \begin{equation}
    R = \frac{1}{\omega C} \Rightarrow \omega = \frac{1}{RC} \Rightarrow f_c = \frac{1}{2\pi RC} \label{ec:fcorte}
  \end{equation}
  También podemos definir la \textbf{constante de tiempo} (o tiempo de respuesta) del circuito:
  \begin{equation}
    T = RC \label{ec:Tresp}
  \end{equation}
  Además podemos deducir los valores del módulo (\ref{ec:moduloimp1}) y la fase (\ref{ec:faseimp}) de la impedancia para la frecuencia de corte:
  \begin{gather}
    Z = R\sqrt{2} = R\left(\frac{V_m}{V_mR}\right) \label{ec:moduloimp2} \\ \varphi = \frac{\pi}{4} \nonumber
  \end{gather}
  En este caso en particular, si seguimos las ecuaciones \ref{ec:fcorte} y \ref{ec:Tresp}, obtenemos los siguientes valores para la frecuencia de corte ($f_c$) y la constante de tiempo ($T$)
  \begin{gather}
    f_c = \frac{1}{2\pi \cdot 10^4 \cdot 12 \cdot 10^{-9}} = 1326,29 Hz \qquad T = 10^4 \cdot 12 \cdot 10^{-9} = 1,2 \cdot 10^{-4} s \nonumber
  \end{gather}

  \subsection{Procedimiento de medición}

  Ahora realizaremos una serie de medidas del potencial total del circuito ($V_m$) y de los potenciales en bornes de la resistencia ($V_{mR}$) y del condensador ($V_{mR}$) para distintas frecuencias. Para ello haremos uso de un \textbf{osciloscopio}. Utilizaremos dos canales, colocados de la manera que indica el siguiente gráfico. El cable positivo de CH1 siempre acompaña al positivo de la fuente de alimentación, mientras que su tierra se conecta con la conexión de tierra de la fuente. En cada medición, manteniendo la frecuencia constante, cambiaremos de lado los cuatro cables, intercambiando dos a dos. Por el contrario, sólo conectaremos el cable positivo de CH2, y se mantendrá siempre entre $R$ y $C$.

  \begin{figure}[H]
    \centering
    \begin{circuitikz}[european]
      \draw (-2,1.5) node[oscopeshape](os){CH1};
      \draw (os.in 1) to[short, *-] ++(0,-1.2) node[tlground]{};
      \draw (os.in 2) to[short, *-] ++(0,-0.6) -- (0,0.6) node[]{};

      \draw (3,3) node[oscopeshape](os2){CH2};

      \draw (0,0) to[sV, l=$V_{in}$] (0,3)
      to[R, l=R] (os2.left);
      \draw (os2.south) to[capacitor, l=C] (3,0);
      \draw (0,0) node[tlground]{};
      \draw (3,0) node[tlground]{};
      \draw (1.5,0) node[]{$V_{out}$};
    \end{circuitikz} \qquad

    \begin{circuitikz}[european]
      \draw (-2,1.5) node[oscopeshape](os){CH1};
      \draw (os.in 1) to[short, *-] ++(0,-1.2) node[tlground]{};
      \draw (os.in 2) to[short, *-] ++(0,-0.6) -- (0,0.6) node[]{};

      \draw (3,3) node[oscopeshape](os2){CH2};

      \draw (0,0) to[sV, l=$V_{in}$] (0,3)
      to[capacitor, l=C] (os2.left);
      \draw (os2.south) to[R, l=R] (3,0);
      \draw (0,0) node[tlground]{};
      \draw (3,0) node[tlground]{};
      \draw (1.5,0) node[]{$V_{out}$};
    \end{circuitikz}

    \caption{Medición del potencial total del circuito RC ($V_m$), y del potencial en bornes del condensador ($V_{mC}$, izquierda) y de la resistencia ($V_{mR}$, derecha).}
  \end{figure}

  \subsection{Medición experimental}

  Siguiendo el procedimiento que acabamos de mencionar, seleccionaremos una serie de frecuencias en torno a la frecuencia de corte y mediremos para ellas $V_m$, $V_{mR}$ y $V_{mC}$. Exponemos ahora un ejemplo de medición.

  \begin{table}[H]
  \centering
  \csvreader[
    tabular=|c|c|c|c|c|,
    table head=\hline Medida & $f (Hz)$  & $V_m (V)$ & $V_mR (V)$ & $V_mC (V)$ \\ \hline,
    late after last line=\\\hline,
    filter test=\ifnumless{\thecsvrow}{2},
    separator=semicolon
    ]{CA4.csv}
    {f=\f, Vm=\vm, VmR=\vmr, VmC=\vmc}
    {\thecsvrow & \f & \vm & \vmr & \vmc}
  \caption{Ejemplo de las mediciones de potenciales según la frecuencia}
  \end{table}

  Antes de mostrar todas las medidas, vamos a calcular otros campos para mostrar en la tabla. Primero, $\log f$, el logaritmo en base 10 de la frecuencia. Después calcularemos $Z$, el módulo de la impedancia, siguiendo la ecuación \ref{ec:moduloimp2}. Consideremos también la escala logarítmica de $Z$, en este caso, $20\log Z$, para poder representarlo posteriormente frente a la escala logarítmica de la frecuencia. Por último calcularemos el valor de $\frac{V_{mR}}{V_{mC}}$ para comprobar si es igual a 1 en la frecuencia de corte que habíamos predicho. También cabe hacer una aclaración. El potencial del circuito $V_{m}$ varió en su medición entre 20,40 y 20,60 Hz. Su valor oscilaba rápidamente entre ambos, impidiendo establecer uno u otro resultado como fiable. Esto se repitió a lo largo de toda la experiencia, por lo que consideramos utilizar 20,50 Hz cómo valor de $V_m$, en especial para evitar perturbaciones en las gráficas que realicemos. Discutiremos esto en el apartado de conclusiones con más detalle.

   Ahora sí, expondremos todos los resultados de la medición experimental:

  \begin{table}[H]
  \centering
  \resizebox{\columnwidth}{!}{
  \csvreader[
    tabular=|c|c|c|c|c|c|c|c|c|,
    table head=\hline Medida & $f (Hz)$ & $\log f$ & $V_m (V)$ & $V_mR (V)$ & $V_mC (V)$ & $Z (\Omega)$ & $20\log Z$ & $V_{mR}/V_{mC}$ \\ \hline,
    late after last line=\\\hline,
    separator=semicolon
    ]{CA4Alt.csv}
    {f=\f, logf=\logf, Vm=\vm, VmR=\vmr, VmC=\vmc, Z=\z, 20logZ=\logz, VmRVmC=\vmbyvm}
    {\thecsvrow & \f & \logf & \vm & \vmr & \vmc & \z & \logz & \vmbyvm}}
  \caption{Medición de potenciales frente a frecuencia}
  \label{tb:ac1}
  \end{table}

  Como podemos observar, y tal cómo se deduce de \ref{ec:moduloimp1}, el módulo de la impedancia ($Z$) decrece a medida que aumenta la frecuencia (ya que $f = \frac{w}{2\pi}$ y $\frac{1}{w^2C^2} = \frac{1}{4\pi^2f^2C^2}$).

  \subsection{Representación gráfica de Z frente a f}

  Si representamos el logaritmo de f frente al logaritmo de Z (escalado por un factor de 20) observaremos una parte de una hipérbola. Sus asíntotas, la horizontal debida a R y la oblicua debida a C, se pueden predecir de la ecuación \ref{ec:moduloimp2}, la primera imaginando que el circuito sólo consta de una resistencia y la segunda utilizando sólamente un condensador. En escala logarítmica se representan cómo rectas, y se intersecarán en el valor del eje de abscisas que se corresponde con la frecuencia de corte. Utilizaremos el siguiente código de \code{python} para la visualización de los datos:

  \begin{python}
    import matplotlib.pyplot as plt
    import pandas as pd
    import numpy as np

    d = pd.read_csv(name + ".csv", sep=';', decimal=',')

    r = 10**4
    f = d["f"]
    vm = d["Vm"]
    vmr = d["VmR"]

    logf = np.log10(f)
    logz = 20 * np.log10(r * (vm / vmr))

    plt.scatter(logf, logz)
  \end{python}

  \begin{figure}[H]
    %\centering
    \resizebox{\columnwidth}{!}{
    %% Creator: Matplotlib, PGF backend
%%
%% To include the figure in your LaTeX document, write
%%   \input{<filename>.pgf}
%%
%% Make sure the required packages are loaded in your preamble
%%   \usepackage{pgf}
%%
%% Figures using additional raster images can only be included by \input if
%% they are in the same directory as the main LaTeX file. For loading figures
%% from other directories you can use the `import` package
%%   \usepackage{import}
%% and then include the figures with
%%   \import{<path to file>}{<filename>.pgf}
%%
%% Matplotlib used the following preamble
%%
\begingroup%
\makeatletter%
\begin{pgfpicture}%
\pgfpathrectangle{\pgfpointorigin}{\pgfqpoint{4.840323in}{3.294691in}}%
\pgfusepath{use as bounding box, clip}%
\begin{pgfscope}%
\pgfsetbuttcap%
\pgfsetmiterjoin%
\definecolor{currentfill}{rgb}{1.000000,1.000000,1.000000}%
\pgfsetfillcolor{currentfill}%
\pgfsetlinewidth{0.000000pt}%
\definecolor{currentstroke}{rgb}{1.000000,1.000000,1.000000}%
\pgfsetstrokecolor{currentstroke}%
\pgfsetdash{}{0pt}%
\pgfpathmoveto{\pgfqpoint{0.000000in}{0.000000in}}%
\pgfpathlineto{\pgfqpoint{4.840323in}{0.000000in}}%
\pgfpathlineto{\pgfqpoint{4.840323in}{3.294691in}}%
\pgfpathlineto{\pgfqpoint{0.000000in}{3.294691in}}%
\pgfpathclose%
\pgfusepath{fill}%
\end{pgfscope}%
\begin{pgfscope}%
\pgfsetbuttcap%
\pgfsetmiterjoin%
\definecolor{currentfill}{rgb}{1.000000,1.000000,1.000000}%
\pgfsetfillcolor{currentfill}%
\pgfsetlinewidth{0.000000pt}%
\definecolor{currentstroke}{rgb}{0.000000,0.000000,0.000000}%
\pgfsetstrokecolor{currentstroke}%
\pgfsetstrokeopacity{0.000000}%
\pgfsetdash{}{0pt}%
\pgfpathmoveto{\pgfqpoint{0.837655in}{0.499691in}}%
\pgfpathlineto{\pgfqpoint{4.712655in}{0.499691in}}%
\pgfpathlineto{\pgfqpoint{4.712655in}{3.194691in}}%
\pgfpathlineto{\pgfqpoint{0.837655in}{3.194691in}}%
\pgfpathclose%
\pgfusepath{fill}%
\end{pgfscope}%
\begin{pgfscope}%
\pgfsetbuttcap%
\pgfsetroundjoin%
\definecolor{currentfill}{rgb}{0.000000,0.000000,0.000000}%
\pgfsetfillcolor{currentfill}%
\pgfsetlinewidth{0.803000pt}%
\definecolor{currentstroke}{rgb}{0.000000,0.000000,0.000000}%
\pgfsetstrokecolor{currentstroke}%
\pgfsetdash{}{0pt}%
\pgfsys@defobject{currentmarker}{\pgfqpoint{0.000000in}{-0.048611in}}{\pgfqpoint{0.000000in}{0.000000in}}{%
\pgfpathmoveto{\pgfqpoint{0.000000in}{0.000000in}}%
\pgfpathlineto{\pgfqpoint{0.000000in}{-0.048611in}}%
\pgfusepath{stroke,fill}%
}%
\begin{pgfscope}%
\pgfsys@transformshift{1.243244in}{0.499691in}%
\pgfsys@useobject{currentmarker}{}%
\end{pgfscope}%
\end{pgfscope}%
\begin{pgfscope}%
\definecolor{textcolor}{rgb}{0.000000,0.000000,0.000000}%
\pgfsetstrokecolor{textcolor}%
\pgfsetfillcolor{textcolor}%
\pgftext[x=1.243244in,y=0.402469in,,top]{\color{textcolor}\rmfamily\fontsize{10.000000}{12.000000}\selectfont \(\displaystyle 2.9\)}%
\end{pgfscope}%
\begin{pgfscope}%
\pgfsetbuttcap%
\pgfsetroundjoin%
\definecolor{currentfill}{rgb}{0.000000,0.000000,0.000000}%
\pgfsetfillcolor{currentfill}%
\pgfsetlinewidth{0.803000pt}%
\definecolor{currentstroke}{rgb}{0.000000,0.000000,0.000000}%
\pgfsetstrokecolor{currentstroke}%
\pgfsetdash{}{0pt}%
\pgfsys@defobject{currentmarker}{\pgfqpoint{0.000000in}{-0.048611in}}{\pgfqpoint{0.000000in}{0.000000in}}{%
\pgfpathmoveto{\pgfqpoint{0.000000in}{0.000000in}}%
\pgfpathlineto{\pgfqpoint{0.000000in}{-0.048611in}}%
\pgfusepath{stroke,fill}%
}%
\begin{pgfscope}%
\pgfsys@transformshift{2.095330in}{0.499691in}%
\pgfsys@useobject{currentmarker}{}%
\end{pgfscope}%
\end{pgfscope}%
\begin{pgfscope}%
\definecolor{textcolor}{rgb}{0.000000,0.000000,0.000000}%
\pgfsetstrokecolor{textcolor}%
\pgfsetfillcolor{textcolor}%
\pgftext[x=2.095330in,y=0.402469in,,top]{\color{textcolor}\rmfamily\fontsize{10.000000}{12.000000}\selectfont \(\displaystyle 3.0\)}%
\end{pgfscope}%
\begin{pgfscope}%
\pgfsetbuttcap%
\pgfsetroundjoin%
\definecolor{currentfill}{rgb}{0.000000,0.000000,0.000000}%
\pgfsetfillcolor{currentfill}%
\pgfsetlinewidth{0.803000pt}%
\definecolor{currentstroke}{rgb}{0.000000,0.000000,0.000000}%
\pgfsetstrokecolor{currentstroke}%
\pgfsetdash{}{0pt}%
\pgfsys@defobject{currentmarker}{\pgfqpoint{0.000000in}{-0.048611in}}{\pgfqpoint{0.000000in}{0.000000in}}{%
\pgfpathmoveto{\pgfqpoint{0.000000in}{0.000000in}}%
\pgfpathlineto{\pgfqpoint{0.000000in}{-0.048611in}}%
\pgfusepath{stroke,fill}%
}%
\begin{pgfscope}%
\pgfsys@transformshift{2.947416in}{0.499691in}%
\pgfsys@useobject{currentmarker}{}%
\end{pgfscope}%
\end{pgfscope}%
\begin{pgfscope}%
\definecolor{textcolor}{rgb}{0.000000,0.000000,0.000000}%
\pgfsetstrokecolor{textcolor}%
\pgfsetfillcolor{textcolor}%
\pgftext[x=2.947416in,y=0.402469in,,top]{\color{textcolor}\rmfamily\fontsize{10.000000}{12.000000}\selectfont \(\displaystyle 3.1\)}%
\end{pgfscope}%
\begin{pgfscope}%
\pgfsetbuttcap%
\pgfsetroundjoin%
\definecolor{currentfill}{rgb}{0.000000,0.000000,0.000000}%
\pgfsetfillcolor{currentfill}%
\pgfsetlinewidth{0.803000pt}%
\definecolor{currentstroke}{rgb}{0.000000,0.000000,0.000000}%
\pgfsetstrokecolor{currentstroke}%
\pgfsetdash{}{0pt}%
\pgfsys@defobject{currentmarker}{\pgfqpoint{0.000000in}{-0.048611in}}{\pgfqpoint{0.000000in}{0.000000in}}{%
\pgfpathmoveto{\pgfqpoint{0.000000in}{0.000000in}}%
\pgfpathlineto{\pgfqpoint{0.000000in}{-0.048611in}}%
\pgfusepath{stroke,fill}%
}%
\begin{pgfscope}%
\pgfsys@transformshift{3.799502in}{0.499691in}%
\pgfsys@useobject{currentmarker}{}%
\end{pgfscope}%
\end{pgfscope}%
\begin{pgfscope}%
\definecolor{textcolor}{rgb}{0.000000,0.000000,0.000000}%
\pgfsetstrokecolor{textcolor}%
\pgfsetfillcolor{textcolor}%
\pgftext[x=3.799502in,y=0.402469in,,top]{\color{textcolor}\rmfamily\fontsize{10.000000}{12.000000}\selectfont \(\displaystyle 3.2\)}%
\end{pgfscope}%
\begin{pgfscope}%
\pgfsetbuttcap%
\pgfsetroundjoin%
\definecolor{currentfill}{rgb}{0.000000,0.000000,0.000000}%
\pgfsetfillcolor{currentfill}%
\pgfsetlinewidth{0.803000pt}%
\definecolor{currentstroke}{rgb}{0.000000,0.000000,0.000000}%
\pgfsetstrokecolor{currentstroke}%
\pgfsetdash{}{0pt}%
\pgfsys@defobject{currentmarker}{\pgfqpoint{0.000000in}{-0.048611in}}{\pgfqpoint{0.000000in}{0.000000in}}{%
\pgfpathmoveto{\pgfqpoint{0.000000in}{0.000000in}}%
\pgfpathlineto{\pgfqpoint{0.000000in}{-0.048611in}}%
\pgfusepath{stroke,fill}%
}%
\begin{pgfscope}%
\pgfsys@transformshift{4.651588in}{0.499691in}%
\pgfsys@useobject{currentmarker}{}%
\end{pgfscope}%
\end{pgfscope}%
\begin{pgfscope}%
\definecolor{textcolor}{rgb}{0.000000,0.000000,0.000000}%
\pgfsetstrokecolor{textcolor}%
\pgfsetfillcolor{textcolor}%
\pgftext[x=4.651588in,y=0.402469in,,top]{\color{textcolor}\rmfamily\fontsize{10.000000}{12.000000}\selectfont \(\displaystyle 3.3\)}%
\end{pgfscope}%
\begin{pgfscope}%
\definecolor{textcolor}{rgb}{0.000000,0.000000,0.000000}%
\pgfsetstrokecolor{textcolor}%
\pgfsetfillcolor{textcolor}%
\pgftext[x=2.775155in,y=0.223457in,,top]{\color{textcolor}\rmfamily\fontsize{10.000000}{12.000000}\selectfont log f}%
\end{pgfscope}%
\begin{pgfscope}%
\pgfsetbuttcap%
\pgfsetroundjoin%
\definecolor{currentfill}{rgb}{0.000000,0.000000,0.000000}%
\pgfsetfillcolor{currentfill}%
\pgfsetlinewidth{0.803000pt}%
\definecolor{currentstroke}{rgb}{0.000000,0.000000,0.000000}%
\pgfsetstrokecolor{currentstroke}%
\pgfsetdash{}{0pt}%
\pgfsys@defobject{currentmarker}{\pgfqpoint{-0.048611in}{0.000000in}}{\pgfqpoint{0.000000in}{0.000000in}}{%
\pgfpathmoveto{\pgfqpoint{0.000000in}{0.000000in}}%
\pgfpathlineto{\pgfqpoint{-0.048611in}{0.000000in}}%
\pgfusepath{stroke,fill}%
}%
\begin{pgfscope}%
\pgfsys@transformshift{0.837655in}{0.665277in}%
\pgfsys@useobject{currentmarker}{}%
\end{pgfscope}%
\end{pgfscope}%
\begin{pgfscope}%
\definecolor{textcolor}{rgb}{0.000000,0.000000,0.000000}%
\pgfsetstrokecolor{textcolor}%
\pgfsetfillcolor{textcolor}%
\pgftext[x=0.601544in,y=0.617051in,left,base]{\color{textcolor}\rmfamily\fontsize{10.000000}{12.000000}\selectfont \(\displaystyle 82\)}%
\end{pgfscope}%
\begin{pgfscope}%
\pgfsetbuttcap%
\pgfsetroundjoin%
\definecolor{currentfill}{rgb}{0.000000,0.000000,0.000000}%
\pgfsetfillcolor{currentfill}%
\pgfsetlinewidth{0.803000pt}%
\definecolor{currentstroke}{rgb}{0.000000,0.000000,0.000000}%
\pgfsetstrokecolor{currentstroke}%
\pgfsetdash{}{0pt}%
\pgfsys@defobject{currentmarker}{\pgfqpoint{-0.048611in}{0.000000in}}{\pgfqpoint{0.000000in}{0.000000in}}{%
\pgfpathmoveto{\pgfqpoint{0.000000in}{0.000000in}}%
\pgfpathlineto{\pgfqpoint{-0.048611in}{0.000000in}}%
\pgfusepath{stroke,fill}%
}%
\begin{pgfscope}%
\pgfsys@transformshift{0.837655in}{1.256046in}%
\pgfsys@useobject{currentmarker}{}%
\end{pgfscope}%
\end{pgfscope}%
\begin{pgfscope}%
\definecolor{textcolor}{rgb}{0.000000,0.000000,0.000000}%
\pgfsetstrokecolor{textcolor}%
\pgfsetfillcolor{textcolor}%
\pgftext[x=0.601544in,y=1.207820in,left,base]{\color{textcolor}\rmfamily\fontsize{10.000000}{12.000000}\selectfont \(\displaystyle 83\)}%
\end{pgfscope}%
\begin{pgfscope}%
\pgfsetbuttcap%
\pgfsetroundjoin%
\definecolor{currentfill}{rgb}{0.000000,0.000000,0.000000}%
\pgfsetfillcolor{currentfill}%
\pgfsetlinewidth{0.803000pt}%
\definecolor{currentstroke}{rgb}{0.000000,0.000000,0.000000}%
\pgfsetstrokecolor{currentstroke}%
\pgfsetdash{}{0pt}%
\pgfsys@defobject{currentmarker}{\pgfqpoint{-0.048611in}{0.000000in}}{\pgfqpoint{0.000000in}{0.000000in}}{%
\pgfpathmoveto{\pgfqpoint{0.000000in}{0.000000in}}%
\pgfpathlineto{\pgfqpoint{-0.048611in}{0.000000in}}%
\pgfusepath{stroke,fill}%
}%
\begin{pgfscope}%
\pgfsys@transformshift{0.837655in}{1.846815in}%
\pgfsys@useobject{currentmarker}{}%
\end{pgfscope}%
\end{pgfscope}%
\begin{pgfscope}%
\definecolor{textcolor}{rgb}{0.000000,0.000000,0.000000}%
\pgfsetstrokecolor{textcolor}%
\pgfsetfillcolor{textcolor}%
\pgftext[x=0.601544in,y=1.798589in,left,base]{\color{textcolor}\rmfamily\fontsize{10.000000}{12.000000}\selectfont \(\displaystyle 84\)}%
\end{pgfscope}%
\begin{pgfscope}%
\pgfsetbuttcap%
\pgfsetroundjoin%
\definecolor{currentfill}{rgb}{0.000000,0.000000,0.000000}%
\pgfsetfillcolor{currentfill}%
\pgfsetlinewidth{0.803000pt}%
\definecolor{currentstroke}{rgb}{0.000000,0.000000,0.000000}%
\pgfsetstrokecolor{currentstroke}%
\pgfsetdash{}{0pt}%
\pgfsys@defobject{currentmarker}{\pgfqpoint{-0.048611in}{0.000000in}}{\pgfqpoint{0.000000in}{0.000000in}}{%
\pgfpathmoveto{\pgfqpoint{0.000000in}{0.000000in}}%
\pgfpathlineto{\pgfqpoint{-0.048611in}{0.000000in}}%
\pgfusepath{stroke,fill}%
}%
\begin{pgfscope}%
\pgfsys@transformshift{0.837655in}{2.437584in}%
\pgfsys@useobject{currentmarker}{}%
\end{pgfscope}%
\end{pgfscope}%
\begin{pgfscope}%
\definecolor{textcolor}{rgb}{0.000000,0.000000,0.000000}%
\pgfsetstrokecolor{textcolor}%
\pgfsetfillcolor{textcolor}%
\pgftext[x=0.601544in,y=2.389359in,left,base]{\color{textcolor}\rmfamily\fontsize{10.000000}{12.000000}\selectfont \(\displaystyle 85\)}%
\end{pgfscope}%
\begin{pgfscope}%
\pgfsetbuttcap%
\pgfsetroundjoin%
\definecolor{currentfill}{rgb}{0.000000,0.000000,0.000000}%
\pgfsetfillcolor{currentfill}%
\pgfsetlinewidth{0.803000pt}%
\definecolor{currentstroke}{rgb}{0.000000,0.000000,0.000000}%
\pgfsetstrokecolor{currentstroke}%
\pgfsetdash{}{0pt}%
\pgfsys@defobject{currentmarker}{\pgfqpoint{-0.048611in}{0.000000in}}{\pgfqpoint{0.000000in}{0.000000in}}{%
\pgfpathmoveto{\pgfqpoint{0.000000in}{0.000000in}}%
\pgfpathlineto{\pgfqpoint{-0.048611in}{0.000000in}}%
\pgfusepath{stroke,fill}%
}%
\begin{pgfscope}%
\pgfsys@transformshift{0.837655in}{3.028353in}%
\pgfsys@useobject{currentmarker}{}%
\end{pgfscope}%
\end{pgfscope}%
\begin{pgfscope}%
\definecolor{textcolor}{rgb}{0.000000,0.000000,0.000000}%
\pgfsetstrokecolor{textcolor}%
\pgfsetfillcolor{textcolor}%
\pgftext[x=0.601544in,y=2.980128in,left,base]{\color{textcolor}\rmfamily\fontsize{10.000000}{12.000000}\selectfont \(\displaystyle 86\)}%
\end{pgfscope}%
\begin{pgfscope}%
\definecolor{textcolor}{rgb}{0.000000,0.000000,0.000000}%
\pgfsetstrokecolor{textcolor}%
\pgfsetfillcolor{textcolor}%
\pgftext[x=0.323766in,y=1.847191in,,bottom]{\color{textcolor}\rmfamily\fontsize{10.000000}{12.000000}\selectfont 20log Z}%
\end{pgfscope}%
\begin{pgfscope}%
\pgfpathrectangle{\pgfqpoint{0.837655in}{0.499691in}}{\pgfqpoint{3.875000in}{2.695000in}}%
\pgfusepath{clip}%
\pgfsetbuttcap%
\pgfsetroundjoin%
\definecolor{currentfill}{rgb}{0.121569,0.466667,0.705882}%
\pgfsetfillcolor{currentfill}%
\pgfsetlinewidth{1.003750pt}%
\definecolor{currentstroke}{rgb}{0.121569,0.466667,0.705882}%
\pgfsetstrokecolor{currentstroke}%
\pgfsetdash{}{0pt}%
\pgfpathmoveto{\pgfqpoint{1.079760in}{3.023948in}}%
\pgfpathcurveto{\pgfqpoint{1.090810in}{3.023948in}}{\pgfqpoint{1.101409in}{3.028338in}}{\pgfqpoint{1.109222in}{3.036152in}}%
\pgfpathcurveto{\pgfqpoint{1.117036in}{3.043966in}}{\pgfqpoint{1.121426in}{3.054565in}}{\pgfqpoint{1.121426in}{3.065615in}}%
\pgfpathcurveto{\pgfqpoint{1.121426in}{3.076665in}}{\pgfqpoint{1.117036in}{3.087264in}}{\pgfqpoint{1.109222in}{3.095078in}}%
\pgfpathcurveto{\pgfqpoint{1.101409in}{3.102891in}}{\pgfqpoint{1.090810in}{3.107281in}}{\pgfqpoint{1.079760in}{3.107281in}}%
\pgfpathcurveto{\pgfqpoint{1.068709in}{3.107281in}}{\pgfqpoint{1.058110in}{3.102891in}}{\pgfqpoint{1.050297in}{3.095078in}}%
\pgfpathcurveto{\pgfqpoint{1.042483in}{3.087264in}}{\pgfqpoint{1.038093in}{3.076665in}}{\pgfqpoint{1.038093in}{3.065615in}}%
\pgfpathcurveto{\pgfqpoint{1.038093in}{3.054565in}}{\pgfqpoint{1.042483in}{3.043966in}}{\pgfqpoint{1.050297in}{3.036152in}}%
\pgfpathcurveto{\pgfqpoint{1.058110in}{3.028338in}}{\pgfqpoint{1.068709in}{3.023948in}}{\pgfqpoint{1.079760in}{3.023948in}}%
\pgfpathclose%
\pgfusepath{stroke,fill}%
\end{pgfscope}%
\begin{pgfscope}%
\pgfpathrectangle{\pgfqpoint{0.837655in}{0.499691in}}{\pgfqpoint{3.875000in}{2.695000in}}%
\pgfusepath{clip}%
\pgfsetbuttcap%
\pgfsetroundjoin%
\definecolor{currentfill}{rgb}{0.121569,0.466667,0.705882}%
\pgfsetfillcolor{currentfill}%
\pgfsetlinewidth{1.003750pt}%
\definecolor{currentstroke}{rgb}{0.121569,0.466667,0.705882}%
\pgfsetstrokecolor{currentstroke}%
\pgfsetdash{}{0pt}%
\pgfpathmoveto{\pgfqpoint{1.360950in}{2.730648in}}%
\pgfpathcurveto{\pgfqpoint{1.372000in}{2.730648in}}{\pgfqpoint{1.382599in}{2.735038in}}{\pgfqpoint{1.390413in}{2.742852in}}%
\pgfpathcurveto{\pgfqpoint{1.398227in}{2.750666in}}{\pgfqpoint{1.402617in}{2.761265in}}{\pgfqpoint{1.402617in}{2.772315in}}%
\pgfpathcurveto{\pgfqpoint{1.402617in}{2.783365in}}{\pgfqpoint{1.398227in}{2.793964in}}{\pgfqpoint{1.390413in}{2.801778in}}%
\pgfpathcurveto{\pgfqpoint{1.382599in}{2.809591in}}{\pgfqpoint{1.372000in}{2.813981in}}{\pgfqpoint{1.360950in}{2.813981in}}%
\pgfpathcurveto{\pgfqpoint{1.349900in}{2.813981in}}{\pgfqpoint{1.339301in}{2.809591in}}{\pgfqpoint{1.331487in}{2.801778in}}%
\pgfpathcurveto{\pgfqpoint{1.323674in}{2.793964in}}{\pgfqpoint{1.319283in}{2.783365in}}{\pgfqpoint{1.319283in}{2.772315in}}%
\pgfpathcurveto{\pgfqpoint{1.319283in}{2.761265in}}{\pgfqpoint{1.323674in}{2.750666in}}{\pgfqpoint{1.331487in}{2.742852in}}%
\pgfpathcurveto{\pgfqpoint{1.339301in}{2.735038in}}{\pgfqpoint{1.349900in}{2.730648in}}{\pgfqpoint{1.360950in}{2.730648in}}%
\pgfpathclose%
\pgfusepath{stroke,fill}%
\end{pgfscope}%
\begin{pgfscope}%
\pgfpathrectangle{\pgfqpoint{0.837655in}{0.499691in}}{\pgfqpoint{3.875000in}{2.695000in}}%
\pgfusepath{clip}%
\pgfsetbuttcap%
\pgfsetroundjoin%
\definecolor{currentfill}{rgb}{0.121569,0.466667,0.705882}%
\pgfsetfillcolor{currentfill}%
\pgfsetlinewidth{1.003750pt}%
\definecolor{currentstroke}{rgb}{0.121569,0.466667,0.705882}%
\pgfsetstrokecolor{currentstroke}%
\pgfsetdash{}{0pt}%
\pgfpathmoveto{\pgfqpoint{1.622275in}{2.544033in}}%
\pgfpathcurveto{\pgfqpoint{1.633325in}{2.544033in}}{\pgfqpoint{1.643924in}{2.548423in}}{\pgfqpoint{1.651738in}{2.556237in}}%
\pgfpathcurveto{\pgfqpoint{1.659551in}{2.564050in}}{\pgfqpoint{1.663942in}{2.574649in}}{\pgfqpoint{1.663942in}{2.585699in}}%
\pgfpathcurveto{\pgfqpoint{1.663942in}{2.596750in}}{\pgfqpoint{1.659551in}{2.607349in}}{\pgfqpoint{1.651738in}{2.615162in}}%
\pgfpathcurveto{\pgfqpoint{1.643924in}{2.622976in}}{\pgfqpoint{1.633325in}{2.627366in}}{\pgfqpoint{1.622275in}{2.627366in}}%
\pgfpathcurveto{\pgfqpoint{1.611225in}{2.627366in}}{\pgfqpoint{1.600626in}{2.622976in}}{\pgfqpoint{1.592812in}{2.615162in}}%
\pgfpathcurveto{\pgfqpoint{1.584998in}{2.607349in}}{\pgfqpoint{1.580608in}{2.596750in}}{\pgfqpoint{1.580608in}{2.585699in}}%
\pgfpathcurveto{\pgfqpoint{1.580608in}{2.574649in}}{\pgfqpoint{1.584998in}{2.564050in}}{\pgfqpoint{1.592812in}{2.556237in}}%
\pgfpathcurveto{\pgfqpoint{1.600626in}{2.548423in}}{\pgfqpoint{1.611225in}{2.544033in}}{\pgfqpoint{1.622275in}{2.544033in}}%
\pgfpathclose%
\pgfusepath{stroke,fill}%
\end{pgfscope}%
\begin{pgfscope}%
\pgfpathrectangle{\pgfqpoint{0.837655in}{0.499691in}}{\pgfqpoint{3.875000in}{2.695000in}}%
\pgfusepath{clip}%
\pgfsetbuttcap%
\pgfsetroundjoin%
\definecolor{currentfill}{rgb}{0.121569,0.466667,0.705882}%
\pgfsetfillcolor{currentfill}%
\pgfsetlinewidth{1.003750pt}%
\definecolor{currentstroke}{rgb}{0.121569,0.466667,0.705882}%
\pgfsetstrokecolor{currentstroke}%
\pgfsetdash{}{0pt}%
\pgfpathmoveto{\pgfqpoint{1.866356in}{2.276249in}}%
\pgfpathcurveto{\pgfqpoint{1.877407in}{2.276249in}}{\pgfqpoint{1.888006in}{2.280639in}}{\pgfqpoint{1.895819in}{2.288453in}}%
\pgfpathcurveto{\pgfqpoint{1.903633in}{2.296267in}}{\pgfqpoint{1.908023in}{2.306866in}}{\pgfqpoint{1.908023in}{2.317916in}}%
\pgfpathcurveto{\pgfqpoint{1.908023in}{2.328966in}}{\pgfqpoint{1.903633in}{2.339565in}}{\pgfqpoint{1.895819in}{2.347379in}}%
\pgfpathcurveto{\pgfqpoint{1.888006in}{2.355192in}}{\pgfqpoint{1.877407in}{2.359583in}}{\pgfqpoint{1.866356in}{2.359583in}}%
\pgfpathcurveto{\pgfqpoint{1.855306in}{2.359583in}}{\pgfqpoint{1.844707in}{2.355192in}}{\pgfqpoint{1.836894in}{2.347379in}}%
\pgfpathcurveto{\pgfqpoint{1.829080in}{2.339565in}}{\pgfqpoint{1.824690in}{2.328966in}}{\pgfqpoint{1.824690in}{2.317916in}}%
\pgfpathcurveto{\pgfqpoint{1.824690in}{2.306866in}}{\pgfqpoint{1.829080in}{2.296267in}}{\pgfqpoint{1.836894in}{2.288453in}}%
\pgfpathcurveto{\pgfqpoint{1.844707in}{2.280639in}}{\pgfqpoint{1.855306in}{2.276249in}}{\pgfqpoint{1.866356in}{2.276249in}}%
\pgfpathclose%
\pgfusepath{stroke,fill}%
\end{pgfscope}%
\begin{pgfscope}%
\pgfpathrectangle{\pgfqpoint{0.837655in}{0.499691in}}{\pgfqpoint{3.875000in}{2.695000in}}%
\pgfusepath{clip}%
\pgfsetbuttcap%
\pgfsetroundjoin%
\definecolor{currentfill}{rgb}{0.121569,0.466667,0.705882}%
\pgfsetfillcolor{currentfill}%
\pgfsetlinewidth{1.003750pt}%
\definecolor{currentstroke}{rgb}{0.121569,0.466667,0.705882}%
\pgfsetstrokecolor{currentstroke}%
\pgfsetdash{}{0pt}%
\pgfpathmoveto{\pgfqpoint{2.095330in}{2.105188in}}%
\pgfpathcurveto{\pgfqpoint{2.106380in}{2.105188in}}{\pgfqpoint{2.116979in}{2.109578in}}{\pgfqpoint{2.124793in}{2.117392in}}%
\pgfpathcurveto{\pgfqpoint{2.132607in}{2.125206in}}{\pgfqpoint{2.136997in}{2.135805in}}{\pgfqpoint{2.136997in}{2.146855in}}%
\pgfpathcurveto{\pgfqpoint{2.136997in}{2.157905in}}{\pgfqpoint{2.132607in}{2.168504in}}{\pgfqpoint{2.124793in}{2.176318in}}%
\pgfpathcurveto{\pgfqpoint{2.116979in}{2.184131in}}{\pgfqpoint{2.106380in}{2.188522in}}{\pgfqpoint{2.095330in}{2.188522in}}%
\pgfpathcurveto{\pgfqpoint{2.084280in}{2.188522in}}{\pgfqpoint{2.073681in}{2.184131in}}{\pgfqpoint{2.065868in}{2.176318in}}%
\pgfpathcurveto{\pgfqpoint{2.058054in}{2.168504in}}{\pgfqpoint{2.053664in}{2.157905in}}{\pgfqpoint{2.053664in}{2.146855in}}%
\pgfpathcurveto{\pgfqpoint{2.053664in}{2.135805in}}{\pgfqpoint{2.058054in}{2.125206in}}{\pgfqpoint{2.065868in}{2.117392in}}%
\pgfpathcurveto{\pgfqpoint{2.073681in}{2.109578in}}{\pgfqpoint{2.084280in}{2.105188in}}{\pgfqpoint{2.095330in}{2.105188in}}%
\pgfpathclose%
\pgfusepath{stroke,fill}%
\end{pgfscope}%
\begin{pgfscope}%
\pgfpathrectangle{\pgfqpoint{0.837655in}{0.499691in}}{\pgfqpoint{3.875000in}{2.695000in}}%
\pgfusepath{clip}%
\pgfsetbuttcap%
\pgfsetroundjoin%
\definecolor{currentfill}{rgb}{0.121569,0.466667,0.705882}%
\pgfsetfillcolor{currentfill}%
\pgfsetlinewidth{1.003750pt}%
\definecolor{currentstroke}{rgb}{0.121569,0.466667,0.705882}%
\pgfsetstrokecolor{currentstroke}%
\pgfsetdash{}{0pt}%
\pgfpathmoveto{\pgfqpoint{2.310958in}{1.858836in}}%
\pgfpathcurveto{\pgfqpoint{2.322008in}{1.858836in}}{\pgfqpoint{2.332607in}{1.863226in}}{\pgfqpoint{2.340421in}{1.871040in}}%
\pgfpathcurveto{\pgfqpoint{2.348234in}{1.878853in}}{\pgfqpoint{2.352625in}{1.889452in}}{\pgfqpoint{2.352625in}{1.900503in}}%
\pgfpathcurveto{\pgfqpoint{2.352625in}{1.911553in}}{\pgfqpoint{2.348234in}{1.922152in}}{\pgfqpoint{2.340421in}{1.929965in}}%
\pgfpathcurveto{\pgfqpoint{2.332607in}{1.937779in}}{\pgfqpoint{2.322008in}{1.942169in}}{\pgfqpoint{2.310958in}{1.942169in}}%
\pgfpathcurveto{\pgfqpoint{2.299908in}{1.942169in}}{\pgfqpoint{2.289309in}{1.937779in}}{\pgfqpoint{2.281495in}{1.929965in}}%
\pgfpathcurveto{\pgfqpoint{2.273682in}{1.922152in}}{\pgfqpoint{2.269291in}{1.911553in}}{\pgfqpoint{2.269291in}{1.900503in}}%
\pgfpathcurveto{\pgfqpoint{2.269291in}{1.889452in}}{\pgfqpoint{2.273682in}{1.878853in}}{\pgfqpoint{2.281495in}{1.871040in}}%
\pgfpathcurveto{\pgfqpoint{2.289309in}{1.863226in}}{\pgfqpoint{2.299908in}{1.858836in}}{\pgfqpoint{2.310958in}{1.858836in}}%
\pgfpathclose%
\pgfusepath{stroke,fill}%
\end{pgfscope}%
\begin{pgfscope}%
\pgfpathrectangle{\pgfqpoint{0.837655in}{0.499691in}}{\pgfqpoint{3.875000in}{2.695000in}}%
\pgfusepath{clip}%
\pgfsetbuttcap%
\pgfsetroundjoin%
\definecolor{currentfill}{rgb}{0.121569,0.466667,0.705882}%
\pgfsetfillcolor{currentfill}%
\pgfsetlinewidth{1.003750pt}%
\definecolor{currentstroke}{rgb}{0.121569,0.466667,0.705882}%
\pgfsetstrokecolor{currentstroke}%
\pgfsetdash{}{0pt}%
\pgfpathmoveto{\pgfqpoint{2.514710in}{1.700936in}}%
\pgfpathcurveto{\pgfqpoint{2.525760in}{1.700936in}}{\pgfqpoint{2.536359in}{1.705326in}}{\pgfqpoint{2.544173in}{1.713140in}}%
\pgfpathcurveto{\pgfqpoint{2.551987in}{1.720953in}}{\pgfqpoint{2.556377in}{1.731552in}}{\pgfqpoint{2.556377in}{1.742602in}}%
\pgfpathcurveto{\pgfqpoint{2.556377in}{1.753652in}}{\pgfqpoint{2.551987in}{1.764251in}}{\pgfqpoint{2.544173in}{1.772065in}}%
\pgfpathcurveto{\pgfqpoint{2.536359in}{1.779879in}}{\pgfqpoint{2.525760in}{1.784269in}}{\pgfqpoint{2.514710in}{1.784269in}}%
\pgfpathcurveto{\pgfqpoint{2.503660in}{1.784269in}}{\pgfqpoint{2.493061in}{1.779879in}}{\pgfqpoint{2.485247in}{1.772065in}}%
\pgfpathcurveto{\pgfqpoint{2.477434in}{1.764251in}}{\pgfqpoint{2.473044in}{1.753652in}}{\pgfqpoint{2.473044in}{1.742602in}}%
\pgfpathcurveto{\pgfqpoint{2.473044in}{1.731552in}}{\pgfqpoint{2.477434in}{1.720953in}}{\pgfqpoint{2.485247in}{1.713140in}}%
\pgfpathcurveto{\pgfqpoint{2.493061in}{1.705326in}}{\pgfqpoint{2.503660in}{1.700936in}}{\pgfqpoint{2.514710in}{1.700936in}}%
\pgfpathclose%
\pgfusepath{stroke,fill}%
\end{pgfscope}%
\begin{pgfscope}%
\pgfpathrectangle{\pgfqpoint{0.837655in}{0.499691in}}{\pgfqpoint{3.875000in}{2.695000in}}%
\pgfusepath{clip}%
\pgfsetbuttcap%
\pgfsetroundjoin%
\definecolor{currentfill}{rgb}{0.121569,0.466667,0.705882}%
\pgfsetfillcolor{currentfill}%
\pgfsetlinewidth{1.003750pt}%
\definecolor{currentstroke}{rgb}{0.121569,0.466667,0.705882}%
\pgfsetstrokecolor{currentstroke}%
\pgfsetdash{}{0pt}%
\pgfpathmoveto{\pgfqpoint{2.707827in}{1.585619in}}%
\pgfpathcurveto{\pgfqpoint{2.718877in}{1.585619in}}{\pgfqpoint{2.729476in}{1.590010in}}{\pgfqpoint{2.737290in}{1.597823in}}%
\pgfpathcurveto{\pgfqpoint{2.745103in}{1.605637in}}{\pgfqpoint{2.749494in}{1.616236in}}{\pgfqpoint{2.749494in}{1.627286in}}%
\pgfpathcurveto{\pgfqpoint{2.749494in}{1.638336in}}{\pgfqpoint{2.745103in}{1.648935in}}{\pgfqpoint{2.737290in}{1.656749in}}%
\pgfpathcurveto{\pgfqpoint{2.729476in}{1.664563in}}{\pgfqpoint{2.718877in}{1.668953in}}{\pgfqpoint{2.707827in}{1.668953in}}%
\pgfpathcurveto{\pgfqpoint{2.696777in}{1.668953in}}{\pgfqpoint{2.686178in}{1.664563in}}{\pgfqpoint{2.678364in}{1.656749in}}%
\pgfpathcurveto{\pgfqpoint{2.670550in}{1.648935in}}{\pgfqpoint{2.666160in}{1.638336in}}{\pgfqpoint{2.666160in}{1.627286in}}%
\pgfpathcurveto{\pgfqpoint{2.666160in}{1.616236in}}{\pgfqpoint{2.670550in}{1.605637in}}{\pgfqpoint{2.678364in}{1.597823in}}%
\pgfpathcurveto{\pgfqpoint{2.686178in}{1.590010in}}{\pgfqpoint{2.696777in}{1.585619in}}{\pgfqpoint{2.707827in}{1.585619in}}%
\pgfpathclose%
\pgfusepath{stroke,fill}%
\end{pgfscope}%
\begin{pgfscope}%
\pgfpathrectangle{\pgfqpoint{0.837655in}{0.499691in}}{\pgfqpoint{3.875000in}{2.695000in}}%
\pgfusepath{clip}%
\pgfsetbuttcap%
\pgfsetroundjoin%
\definecolor{currentfill}{rgb}{0.121569,0.466667,0.705882}%
\pgfsetfillcolor{currentfill}%
\pgfsetlinewidth{1.003750pt}%
\definecolor{currentstroke}{rgb}{0.121569,0.466667,0.705882}%
\pgfsetstrokecolor{currentstroke}%
\pgfsetdash{}{0pt}%
\pgfpathmoveto{\pgfqpoint{2.891363in}{1.435788in}}%
\pgfpathcurveto{\pgfqpoint{2.902414in}{1.435788in}}{\pgfqpoint{2.913013in}{1.440178in}}{\pgfqpoint{2.920826in}{1.447992in}}%
\pgfpathcurveto{\pgfqpoint{2.928640in}{1.455806in}}{\pgfqpoint{2.933030in}{1.466405in}}{\pgfqpoint{2.933030in}{1.477455in}}%
\pgfpathcurveto{\pgfqpoint{2.933030in}{1.488505in}}{\pgfqpoint{2.928640in}{1.499104in}}{\pgfqpoint{2.920826in}{1.506918in}}%
\pgfpathcurveto{\pgfqpoint{2.913013in}{1.514731in}}{\pgfqpoint{2.902414in}{1.519122in}}{\pgfqpoint{2.891363in}{1.519122in}}%
\pgfpathcurveto{\pgfqpoint{2.880313in}{1.519122in}}{\pgfqpoint{2.869714in}{1.514731in}}{\pgfqpoint{2.861901in}{1.506918in}}%
\pgfpathcurveto{\pgfqpoint{2.854087in}{1.499104in}}{\pgfqpoint{2.849697in}{1.488505in}}{\pgfqpoint{2.849697in}{1.477455in}}%
\pgfpathcurveto{\pgfqpoint{2.849697in}{1.466405in}}{\pgfqpoint{2.854087in}{1.455806in}}{\pgfqpoint{2.861901in}{1.447992in}}%
\pgfpathcurveto{\pgfqpoint{2.869714in}{1.440178in}}{\pgfqpoint{2.880313in}{1.435788in}}{\pgfqpoint{2.891363in}{1.435788in}}%
\pgfpathclose%
\pgfusepath{stroke,fill}%
\end{pgfscope}%
\begin{pgfscope}%
\pgfpathrectangle{\pgfqpoint{0.837655in}{0.499691in}}{\pgfqpoint{3.875000in}{2.695000in}}%
\pgfusepath{clip}%
\pgfsetbuttcap%
\pgfsetroundjoin%
\definecolor{currentfill}{rgb}{0.121569,0.466667,0.705882}%
\pgfsetfillcolor{currentfill}%
\pgfsetlinewidth{1.003750pt}%
\definecolor{currentstroke}{rgb}{0.121569,0.466667,0.705882}%
\pgfsetstrokecolor{currentstroke}%
\pgfsetdash{}{0pt}%
\pgfpathmoveto{\pgfqpoint{3.066226in}{1.399004in}}%
\pgfpathcurveto{\pgfqpoint{3.077276in}{1.399004in}}{\pgfqpoint{3.087875in}{1.403394in}}{\pgfqpoint{3.095689in}{1.411208in}}%
\pgfpathcurveto{\pgfqpoint{3.103502in}{1.419022in}}{\pgfqpoint{3.107892in}{1.429621in}}{\pgfqpoint{3.107892in}{1.440671in}}%
\pgfpathcurveto{\pgfqpoint{3.107892in}{1.451721in}}{\pgfqpoint{3.103502in}{1.462320in}}{\pgfqpoint{3.095689in}{1.470134in}}%
\pgfpathcurveto{\pgfqpoint{3.087875in}{1.477947in}}{\pgfqpoint{3.077276in}{1.482338in}}{\pgfqpoint{3.066226in}{1.482338in}}%
\pgfpathcurveto{\pgfqpoint{3.055176in}{1.482338in}}{\pgfqpoint{3.044577in}{1.477947in}}{\pgfqpoint{3.036763in}{1.470134in}}%
\pgfpathcurveto{\pgfqpoint{3.028949in}{1.462320in}}{\pgfqpoint{3.024559in}{1.451721in}}{\pgfqpoint{3.024559in}{1.440671in}}%
\pgfpathcurveto{\pgfqpoint{3.024559in}{1.429621in}}{\pgfqpoint{3.028949in}{1.419022in}}{\pgfqpoint{3.036763in}{1.411208in}}%
\pgfpathcurveto{\pgfqpoint{3.044577in}{1.403394in}}{\pgfqpoint{3.055176in}{1.399004in}}{\pgfqpoint{3.066226in}{1.399004in}}%
\pgfpathclose%
\pgfusepath{stroke,fill}%
\end{pgfscope}%
\begin{pgfscope}%
\pgfpathrectangle{\pgfqpoint{0.837655in}{0.499691in}}{\pgfqpoint{3.875000in}{2.695000in}}%
\pgfusepath{clip}%
\pgfsetbuttcap%
\pgfsetroundjoin%
\definecolor{currentfill}{rgb}{0.121569,0.466667,0.705882}%
\pgfsetfillcolor{currentfill}%
\pgfsetlinewidth{1.003750pt}%
\definecolor{currentstroke}{rgb}{0.121569,0.466667,0.705882}%
\pgfsetstrokecolor{currentstroke}%
\pgfsetdash{}{0pt}%
\pgfpathmoveto{\pgfqpoint{3.233197in}{1.254449in}}%
\pgfpathcurveto{\pgfqpoint{3.244247in}{1.254449in}}{\pgfqpoint{3.254846in}{1.258840in}}{\pgfqpoint{3.262660in}{1.266653in}}%
\pgfpathcurveto{\pgfqpoint{3.270473in}{1.274467in}}{\pgfqpoint{3.274863in}{1.285066in}}{\pgfqpoint{3.274863in}{1.296116in}}%
\pgfpathcurveto{\pgfqpoint{3.274863in}{1.307166in}}{\pgfqpoint{3.270473in}{1.317765in}}{\pgfqpoint{3.262660in}{1.325579in}}%
\pgfpathcurveto{\pgfqpoint{3.254846in}{1.333392in}}{\pgfqpoint{3.244247in}{1.337783in}}{\pgfqpoint{3.233197in}{1.337783in}}%
\pgfpathcurveto{\pgfqpoint{3.222147in}{1.337783in}}{\pgfqpoint{3.211548in}{1.333392in}}{\pgfqpoint{3.203734in}{1.325579in}}%
\pgfpathcurveto{\pgfqpoint{3.195920in}{1.317765in}}{\pgfqpoint{3.191530in}{1.307166in}}{\pgfqpoint{3.191530in}{1.296116in}}%
\pgfpathcurveto{\pgfqpoint{3.191530in}{1.285066in}}{\pgfqpoint{3.195920in}{1.274467in}}{\pgfqpoint{3.203734in}{1.266653in}}%
\pgfpathcurveto{\pgfqpoint{3.211548in}{1.258840in}}{\pgfqpoint{3.222147in}{1.254449in}}{\pgfqpoint{3.233197in}{1.254449in}}%
\pgfpathclose%
\pgfusepath{stroke,fill}%
\end{pgfscope}%
\begin{pgfscope}%
\pgfpathrectangle{\pgfqpoint{0.837655in}{0.499691in}}{\pgfqpoint{3.875000in}{2.695000in}}%
\pgfusepath{clip}%
\pgfsetbuttcap%
\pgfsetroundjoin%
\definecolor{currentfill}{rgb}{0.121569,0.466667,0.705882}%
\pgfsetfillcolor{currentfill}%
\pgfsetlinewidth{1.003750pt}%
\definecolor{currentstroke}{rgb}{0.121569,0.466667,0.705882}%
\pgfsetstrokecolor{currentstroke}%
\pgfsetdash{}{0pt}%
\pgfpathmoveto{\pgfqpoint{3.392958in}{1.183671in}}%
\pgfpathcurveto{\pgfqpoint{3.404008in}{1.183671in}}{\pgfqpoint{3.414607in}{1.188061in}}{\pgfqpoint{3.422421in}{1.195875in}}%
\pgfpathcurveto{\pgfqpoint{3.430234in}{1.203688in}}{\pgfqpoint{3.434625in}{1.214288in}}{\pgfqpoint{3.434625in}{1.225338in}}%
\pgfpathcurveto{\pgfqpoint{3.434625in}{1.236388in}}{\pgfqpoint{3.430234in}{1.246987in}}{\pgfqpoint{3.422421in}{1.254800in}}%
\pgfpathcurveto{\pgfqpoint{3.414607in}{1.262614in}}{\pgfqpoint{3.404008in}{1.267004in}}{\pgfqpoint{3.392958in}{1.267004in}}%
\pgfpathcurveto{\pgfqpoint{3.381908in}{1.267004in}}{\pgfqpoint{3.371309in}{1.262614in}}{\pgfqpoint{3.363495in}{1.254800in}}%
\pgfpathcurveto{\pgfqpoint{3.355682in}{1.246987in}}{\pgfqpoint{3.351291in}{1.236388in}}{\pgfqpoint{3.351291in}{1.225338in}}%
\pgfpathcurveto{\pgfqpoint{3.351291in}{1.214288in}}{\pgfqpoint{3.355682in}{1.203688in}}{\pgfqpoint{3.363495in}{1.195875in}}%
\pgfpathcurveto{\pgfqpoint{3.371309in}{1.188061in}}{\pgfqpoint{3.381908in}{1.183671in}}{\pgfqpoint{3.392958in}{1.183671in}}%
\pgfpathclose%
\pgfusepath{stroke,fill}%
\end{pgfscope}%
\begin{pgfscope}%
\pgfpathrectangle{\pgfqpoint{0.837655in}{0.499691in}}{\pgfqpoint{3.875000in}{2.695000in}}%
\pgfusepath{clip}%
\pgfsetbuttcap%
\pgfsetroundjoin%
\definecolor{currentfill}{rgb}{0.121569,0.466667,0.705882}%
\pgfsetfillcolor{currentfill}%
\pgfsetlinewidth{1.003750pt}%
\definecolor{currentstroke}{rgb}{0.121569,0.466667,0.705882}%
\pgfsetstrokecolor{currentstroke}%
\pgfsetdash{}{0pt}%
\pgfpathmoveto{\pgfqpoint{3.546107in}{1.113856in}}%
\pgfpathcurveto{\pgfqpoint{3.557157in}{1.113856in}}{\pgfqpoint{3.567756in}{1.118246in}}{\pgfqpoint{3.575569in}{1.126059in}}%
\pgfpathcurveto{\pgfqpoint{3.583383in}{1.133873in}}{\pgfqpoint{3.587773in}{1.144472in}}{\pgfqpoint{3.587773in}{1.155522in}}%
\pgfpathcurveto{\pgfqpoint{3.587773in}{1.166572in}}{\pgfqpoint{3.583383in}{1.177171in}}{\pgfqpoint{3.575569in}{1.184985in}}%
\pgfpathcurveto{\pgfqpoint{3.567756in}{1.192799in}}{\pgfqpoint{3.557157in}{1.197189in}}{\pgfqpoint{3.546107in}{1.197189in}}%
\pgfpathcurveto{\pgfqpoint{3.535057in}{1.197189in}}{\pgfqpoint{3.524457in}{1.192799in}}{\pgfqpoint{3.516644in}{1.184985in}}%
\pgfpathcurveto{\pgfqpoint{3.508830in}{1.177171in}}{\pgfqpoint{3.504440in}{1.166572in}}{\pgfqpoint{3.504440in}{1.155522in}}%
\pgfpathcurveto{\pgfqpoint{3.504440in}{1.144472in}}{\pgfqpoint{3.508830in}{1.133873in}}{\pgfqpoint{3.516644in}{1.126059in}}%
\pgfpathcurveto{\pgfqpoint{3.524457in}{1.118246in}}{\pgfqpoint{3.535057in}{1.113856in}}{\pgfqpoint{3.546107in}{1.113856in}}%
\pgfpathclose%
\pgfusepath{stroke,fill}%
\end{pgfscope}%
\begin{pgfscope}%
\pgfpathrectangle{\pgfqpoint{0.837655in}{0.499691in}}{\pgfqpoint{3.875000in}{2.695000in}}%
\pgfusepath{clip}%
\pgfsetbuttcap%
\pgfsetroundjoin%
\definecolor{currentfill}{rgb}{0.121569,0.466667,0.705882}%
\pgfsetfillcolor{currentfill}%
\pgfsetlinewidth{1.003750pt}%
\definecolor{currentstroke}{rgb}{0.121569,0.466667,0.705882}%
\pgfsetstrokecolor{currentstroke}%
\pgfsetdash{}{0pt}%
\pgfpathmoveto{\pgfqpoint{3.693168in}{0.977011in}}%
\pgfpathcurveto{\pgfqpoint{3.704218in}{0.977011in}}{\pgfqpoint{3.714817in}{0.981402in}}{\pgfqpoint{3.722631in}{0.989215in}}%
\pgfpathcurveto{\pgfqpoint{3.730445in}{0.997029in}}{\pgfqpoint{3.734835in}{1.007628in}}{\pgfqpoint{3.734835in}{1.018678in}}%
\pgfpathcurveto{\pgfqpoint{3.734835in}{1.029728in}}{\pgfqpoint{3.730445in}{1.040327in}}{\pgfqpoint{3.722631in}{1.048141in}}%
\pgfpathcurveto{\pgfqpoint{3.714817in}{1.055954in}}{\pgfqpoint{3.704218in}{1.060345in}}{\pgfqpoint{3.693168in}{1.060345in}}%
\pgfpathcurveto{\pgfqpoint{3.682118in}{1.060345in}}{\pgfqpoint{3.671519in}{1.055954in}}{\pgfqpoint{3.663705in}{1.048141in}}%
\pgfpathcurveto{\pgfqpoint{3.655892in}{1.040327in}}{\pgfqpoint{3.651502in}{1.029728in}}{\pgfqpoint{3.651502in}{1.018678in}}%
\pgfpathcurveto{\pgfqpoint{3.651502in}{1.007628in}}{\pgfqpoint{3.655892in}{0.997029in}}{\pgfqpoint{3.663705in}{0.989215in}}%
\pgfpathcurveto{\pgfqpoint{3.671519in}{0.981402in}}{\pgfqpoint{3.682118in}{0.977011in}}{\pgfqpoint{3.693168in}{0.977011in}}%
\pgfpathclose%
\pgfusepath{stroke,fill}%
\end{pgfscope}%
\begin{pgfscope}%
\pgfpathrectangle{\pgfqpoint{0.837655in}{0.499691in}}{\pgfqpoint{3.875000in}{2.695000in}}%
\pgfusepath{clip}%
\pgfsetbuttcap%
\pgfsetroundjoin%
\definecolor{currentfill}{rgb}{0.121569,0.466667,0.705882}%
\pgfsetfillcolor{currentfill}%
\pgfsetlinewidth{1.003750pt}%
\definecolor{currentstroke}{rgb}{0.121569,0.466667,0.705882}%
\pgfsetstrokecolor{currentstroke}%
\pgfsetdash{}{0pt}%
\pgfpathmoveto{\pgfqpoint{3.834608in}{0.843722in}}%
\pgfpathcurveto{\pgfqpoint{3.845658in}{0.843722in}}{\pgfqpoint{3.856257in}{0.848112in}}{\pgfqpoint{3.864071in}{0.855926in}}%
\pgfpathcurveto{\pgfqpoint{3.871885in}{0.863739in}}{\pgfqpoint{3.876275in}{0.874338in}}{\pgfqpoint{3.876275in}{0.885389in}}%
\pgfpathcurveto{\pgfqpoint{3.876275in}{0.896439in}}{\pgfqpoint{3.871885in}{0.907038in}}{\pgfqpoint{3.864071in}{0.914851in}}%
\pgfpathcurveto{\pgfqpoint{3.856257in}{0.922665in}}{\pgfqpoint{3.845658in}{0.927055in}}{\pgfqpoint{3.834608in}{0.927055in}}%
\pgfpathcurveto{\pgfqpoint{3.823558in}{0.927055in}}{\pgfqpoint{3.812959in}{0.922665in}}{\pgfqpoint{3.805145in}{0.914851in}}%
\pgfpathcurveto{\pgfqpoint{3.797332in}{0.907038in}}{\pgfqpoint{3.792942in}{0.896439in}}{\pgfqpoint{3.792942in}{0.885389in}}%
\pgfpathcurveto{\pgfqpoint{3.792942in}{0.874338in}}{\pgfqpoint{3.797332in}{0.863739in}}{\pgfqpoint{3.805145in}{0.855926in}}%
\pgfpathcurveto{\pgfqpoint{3.812959in}{0.848112in}}{\pgfqpoint{3.823558in}{0.843722in}}{\pgfqpoint{3.834608in}{0.843722in}}%
\pgfpathclose%
\pgfusepath{stroke,fill}%
\end{pgfscope}%
\begin{pgfscope}%
\pgfpathrectangle{\pgfqpoint{0.837655in}{0.499691in}}{\pgfqpoint{3.875000in}{2.695000in}}%
\pgfusepath{clip}%
\pgfsetbuttcap%
\pgfsetroundjoin%
\definecolor{currentfill}{rgb}{0.121569,0.466667,0.705882}%
\pgfsetfillcolor{currentfill}%
\pgfsetlinewidth{1.003750pt}%
\definecolor{currentstroke}{rgb}{0.121569,0.466667,0.705882}%
\pgfsetstrokecolor{currentstroke}%
\pgfsetdash{}{0pt}%
\pgfpathmoveto{\pgfqpoint{3.970841in}{0.843722in}}%
\pgfpathcurveto{\pgfqpoint{3.981891in}{0.843722in}}{\pgfqpoint{3.992490in}{0.848112in}}{\pgfqpoint{4.000303in}{0.855926in}}%
\pgfpathcurveto{\pgfqpoint{4.008117in}{0.863739in}}{\pgfqpoint{4.012507in}{0.874338in}}{\pgfqpoint{4.012507in}{0.885389in}}%
\pgfpathcurveto{\pgfqpoint{4.012507in}{0.896439in}}{\pgfqpoint{4.008117in}{0.907038in}}{\pgfqpoint{4.000303in}{0.914851in}}%
\pgfpathcurveto{\pgfqpoint{3.992490in}{0.922665in}}{\pgfqpoint{3.981891in}{0.927055in}}{\pgfqpoint{3.970841in}{0.927055in}}%
\pgfpathcurveto{\pgfqpoint{3.959791in}{0.927055in}}{\pgfqpoint{3.949191in}{0.922665in}}{\pgfqpoint{3.941378in}{0.914851in}}%
\pgfpathcurveto{\pgfqpoint{3.933564in}{0.907038in}}{\pgfqpoint{3.929174in}{0.896439in}}{\pgfqpoint{3.929174in}{0.885389in}}%
\pgfpathcurveto{\pgfqpoint{3.929174in}{0.874338in}}{\pgfqpoint{3.933564in}{0.863739in}}{\pgfqpoint{3.941378in}{0.855926in}}%
\pgfpathcurveto{\pgfqpoint{3.949191in}{0.848112in}}{\pgfqpoint{3.959791in}{0.843722in}}{\pgfqpoint{3.970841in}{0.843722in}}%
\pgfpathclose%
\pgfusepath{stroke,fill}%
\end{pgfscope}%
\begin{pgfscope}%
\pgfpathrectangle{\pgfqpoint{0.837655in}{0.499691in}}{\pgfqpoint{3.875000in}{2.695000in}}%
\pgfusepath{clip}%
\pgfsetbuttcap%
\pgfsetroundjoin%
\definecolor{currentfill}{rgb}{0.121569,0.466667,0.705882}%
\pgfsetfillcolor{currentfill}%
\pgfsetlinewidth{1.003750pt}%
\definecolor{currentstroke}{rgb}{0.121569,0.466667,0.705882}%
\pgfsetstrokecolor{currentstroke}%
\pgfsetdash{}{0pt}%
\pgfpathmoveto{\pgfqpoint{4.102235in}{0.713807in}}%
\pgfpathcurveto{\pgfqpoint{4.113286in}{0.713807in}}{\pgfqpoint{4.123885in}{0.718198in}}{\pgfqpoint{4.131698in}{0.726011in}}%
\pgfpathcurveto{\pgfqpoint{4.139512in}{0.733825in}}{\pgfqpoint{4.143902in}{0.744424in}}{\pgfqpoint{4.143902in}{0.755474in}}%
\pgfpathcurveto{\pgfqpoint{4.143902in}{0.766524in}}{\pgfqpoint{4.139512in}{0.777123in}}{\pgfqpoint{4.131698in}{0.784937in}}%
\pgfpathcurveto{\pgfqpoint{4.123885in}{0.792750in}}{\pgfqpoint{4.113286in}{0.797141in}}{\pgfqpoint{4.102235in}{0.797141in}}%
\pgfpathcurveto{\pgfqpoint{4.091185in}{0.797141in}}{\pgfqpoint{4.080586in}{0.792750in}}{\pgfqpoint{4.072773in}{0.784937in}}%
\pgfpathcurveto{\pgfqpoint{4.064959in}{0.777123in}}{\pgfqpoint{4.060569in}{0.766524in}}{\pgfqpoint{4.060569in}{0.755474in}}%
\pgfpathcurveto{\pgfqpoint{4.060569in}{0.744424in}}{\pgfqpoint{4.064959in}{0.733825in}}{\pgfqpoint{4.072773in}{0.726011in}}%
\pgfpathcurveto{\pgfqpoint{4.080586in}{0.718198in}}{\pgfqpoint{4.091185in}{0.713807in}}{\pgfqpoint{4.102235in}{0.713807in}}%
\pgfpathclose%
\pgfusepath{stroke,fill}%
\end{pgfscope}%
\begin{pgfscope}%
\pgfpathrectangle{\pgfqpoint{0.837655in}{0.499691in}}{\pgfqpoint{3.875000in}{2.695000in}}%
\pgfusepath{clip}%
\pgfsetbuttcap%
\pgfsetroundjoin%
\definecolor{currentfill}{rgb}{0.121569,0.466667,0.705882}%
\pgfsetfillcolor{currentfill}%
\pgfsetlinewidth{1.003750pt}%
\definecolor{currentstroke}{rgb}{0.121569,0.466667,0.705882}%
\pgfsetstrokecolor{currentstroke}%
\pgfsetdash{}{0pt}%
\pgfpathmoveto{\pgfqpoint{4.229124in}{0.713807in}}%
\pgfpathcurveto{\pgfqpoint{4.240174in}{0.713807in}}{\pgfqpoint{4.250773in}{0.718198in}}{\pgfqpoint{4.258587in}{0.726011in}}%
\pgfpathcurveto{\pgfqpoint{4.266401in}{0.733825in}}{\pgfqpoint{4.270791in}{0.744424in}}{\pgfqpoint{4.270791in}{0.755474in}}%
\pgfpathcurveto{\pgfqpoint{4.270791in}{0.766524in}}{\pgfqpoint{4.266401in}{0.777123in}}{\pgfqpoint{4.258587in}{0.784937in}}%
\pgfpathcurveto{\pgfqpoint{4.250773in}{0.792750in}}{\pgfqpoint{4.240174in}{0.797141in}}{\pgfqpoint{4.229124in}{0.797141in}}%
\pgfpathcurveto{\pgfqpoint{4.218074in}{0.797141in}}{\pgfqpoint{4.207475in}{0.792750in}}{\pgfqpoint{4.199661in}{0.784937in}}%
\pgfpathcurveto{\pgfqpoint{4.191848in}{0.777123in}}{\pgfqpoint{4.187458in}{0.766524in}}{\pgfqpoint{4.187458in}{0.755474in}}%
\pgfpathcurveto{\pgfqpoint{4.187458in}{0.744424in}}{\pgfqpoint{4.191848in}{0.733825in}}{\pgfqpoint{4.199661in}{0.726011in}}%
\pgfpathcurveto{\pgfqpoint{4.207475in}{0.718198in}}{\pgfqpoint{4.218074in}{0.713807in}}{\pgfqpoint{4.229124in}{0.713807in}}%
\pgfpathclose%
\pgfusepath{stroke,fill}%
\end{pgfscope}%
\begin{pgfscope}%
\pgfpathrectangle{\pgfqpoint{0.837655in}{0.499691in}}{\pgfqpoint{3.875000in}{2.695000in}}%
\pgfusepath{clip}%
\pgfsetbuttcap%
\pgfsetroundjoin%
\definecolor{currentfill}{rgb}{0.121569,0.466667,0.705882}%
\pgfsetfillcolor{currentfill}%
\pgfsetlinewidth{1.003750pt}%
\definecolor{currentstroke}{rgb}{0.121569,0.466667,0.705882}%
\pgfsetstrokecolor{currentstroke}%
\pgfsetdash{}{0pt}%
\pgfpathmoveto{\pgfqpoint{4.351806in}{0.587101in}}%
\pgfpathcurveto{\pgfqpoint{4.362856in}{0.587101in}}{\pgfqpoint{4.373455in}{0.591491in}}{\pgfqpoint{4.381269in}{0.599305in}}%
\pgfpathcurveto{\pgfqpoint{4.389082in}{0.607118in}}{\pgfqpoint{4.393473in}{0.617717in}}{\pgfqpoint{4.393473in}{0.628767in}}%
\pgfpathcurveto{\pgfqpoint{4.393473in}{0.639818in}}{\pgfqpoint{4.389082in}{0.650417in}}{\pgfqpoint{4.381269in}{0.658230in}}%
\pgfpathcurveto{\pgfqpoint{4.373455in}{0.666044in}}{\pgfqpoint{4.362856in}{0.670434in}}{\pgfqpoint{4.351806in}{0.670434in}}%
\pgfpathcurveto{\pgfqpoint{4.340756in}{0.670434in}}{\pgfqpoint{4.330157in}{0.666044in}}{\pgfqpoint{4.322343in}{0.658230in}}%
\pgfpathcurveto{\pgfqpoint{4.314530in}{0.650417in}}{\pgfqpoint{4.310139in}{0.639818in}}{\pgfqpoint{4.310139in}{0.628767in}}%
\pgfpathcurveto{\pgfqpoint{4.310139in}{0.617717in}}{\pgfqpoint{4.314530in}{0.607118in}}{\pgfqpoint{4.322343in}{0.599305in}}%
\pgfpathcurveto{\pgfqpoint{4.330157in}{0.591491in}}{\pgfqpoint{4.340756in}{0.587101in}}{\pgfqpoint{4.351806in}{0.587101in}}%
\pgfpathclose%
\pgfusepath{stroke,fill}%
\end{pgfscope}%
\begin{pgfscope}%
\pgfpathrectangle{\pgfqpoint{0.837655in}{0.499691in}}{\pgfqpoint{3.875000in}{2.695000in}}%
\pgfusepath{clip}%
\pgfsetbuttcap%
\pgfsetroundjoin%
\definecolor{currentfill}{rgb}{0.121569,0.466667,0.705882}%
\pgfsetfillcolor{currentfill}%
\pgfsetlinewidth{1.003750pt}%
\definecolor{currentstroke}{rgb}{0.121569,0.466667,0.705882}%
\pgfsetstrokecolor{currentstroke}%
\pgfsetdash{}{0pt}%
\pgfpathmoveto{\pgfqpoint{4.470551in}{0.587101in}}%
\pgfpathcurveto{\pgfqpoint{4.481601in}{0.587101in}}{\pgfqpoint{4.492200in}{0.591491in}}{\pgfqpoint{4.500014in}{0.599305in}}%
\pgfpathcurveto{\pgfqpoint{4.507827in}{0.607118in}}{\pgfqpoint{4.512218in}{0.617717in}}{\pgfqpoint{4.512218in}{0.628767in}}%
\pgfpathcurveto{\pgfqpoint{4.512218in}{0.639818in}}{\pgfqpoint{4.507827in}{0.650417in}}{\pgfqpoint{4.500014in}{0.658230in}}%
\pgfpathcurveto{\pgfqpoint{4.492200in}{0.666044in}}{\pgfqpoint{4.481601in}{0.670434in}}{\pgfqpoint{4.470551in}{0.670434in}}%
\pgfpathcurveto{\pgfqpoint{4.459501in}{0.670434in}}{\pgfqpoint{4.448902in}{0.666044in}}{\pgfqpoint{4.441088in}{0.658230in}}%
\pgfpathcurveto{\pgfqpoint{4.433274in}{0.650417in}}{\pgfqpoint{4.428884in}{0.639818in}}{\pgfqpoint{4.428884in}{0.628767in}}%
\pgfpathcurveto{\pgfqpoint{4.428884in}{0.617717in}}{\pgfqpoint{4.433274in}{0.607118in}}{\pgfqpoint{4.441088in}{0.599305in}}%
\pgfpathcurveto{\pgfqpoint{4.448902in}{0.591491in}}{\pgfqpoint{4.459501in}{0.587101in}}{\pgfqpoint{4.470551in}{0.587101in}}%
\pgfpathclose%
\pgfusepath{stroke,fill}%
\end{pgfscope}%
\begin{pgfscope}%
\pgfsetrectcap%
\pgfsetmiterjoin%
\pgfsetlinewidth{0.803000pt}%
\definecolor{currentstroke}{rgb}{0.000000,0.000000,0.000000}%
\pgfsetstrokecolor{currentstroke}%
\pgfsetdash{}{0pt}%
\pgfpathmoveto{\pgfqpoint{0.837655in}{0.499691in}}%
\pgfpathlineto{\pgfqpoint{0.837655in}{3.194691in}}%
\pgfusepath{stroke}%
\end{pgfscope}%
\begin{pgfscope}%
\pgfsetrectcap%
\pgfsetmiterjoin%
\pgfsetlinewidth{0.803000pt}%
\definecolor{currentstroke}{rgb}{0.000000,0.000000,0.000000}%
\pgfsetstrokecolor{currentstroke}%
\pgfsetdash{}{0pt}%
\pgfpathmoveto{\pgfqpoint{4.712655in}{0.499691in}}%
\pgfpathlineto{\pgfqpoint{4.712655in}{3.194691in}}%
\pgfusepath{stroke}%
\end{pgfscope}%
\begin{pgfscope}%
\pgfsetrectcap%
\pgfsetmiterjoin%
\pgfsetlinewidth{0.803000pt}%
\definecolor{currentstroke}{rgb}{0.000000,0.000000,0.000000}%
\pgfsetstrokecolor{currentstroke}%
\pgfsetdash{}{0pt}%
\pgfpathmoveto{\pgfqpoint{0.837655in}{0.499691in}}%
\pgfpathlineto{\pgfqpoint{4.712655in}{0.499691in}}%
\pgfusepath{stroke}%
\end{pgfscope}%
\begin{pgfscope}%
\pgfsetrectcap%
\pgfsetmiterjoin%
\pgfsetlinewidth{0.803000pt}%
\definecolor{currentstroke}{rgb}{0.000000,0.000000,0.000000}%
\pgfsetstrokecolor{currentstroke}%
\pgfsetdash{}{0pt}%
\pgfpathmoveto{\pgfqpoint{0.837655in}{3.194691in}}%
\pgfpathlineto{\pgfqpoint{4.712655in}{3.194691in}}%
\pgfusepath{stroke}%
\end{pgfscope}%
\end{pgfpicture}%
\makeatother%
\endgroup%

    %% Creator: Matplotlib, PGF backend
%%
%% To include the figure in your LaTeX document, write
%%   \input{<filename>.pgf}
%%
%% Make sure the required packages are loaded in your preamble
%%   \usepackage{pgf}
%%
%% Figures using additional raster images can only be included by \input if
%% they are in the same directory as the main LaTeX file. For loading figures
%% from other directories you can use the `import` package
%%   \usepackage{import}
%% and then include the figures with
%%   \import{<path to file>}{<filename>.pgf}
%%
%% Matplotlib used the following preamble
%%
\begingroup%
\makeatletter%
\begin{pgfpicture}%
\pgfpathrectangle{\pgfpointorigin}{\pgfqpoint{4.990125in}{3.294691in}}%
\pgfusepath{use as bounding box, clip}%
\begin{pgfscope}%
\pgfsetbuttcap%
\pgfsetmiterjoin%
\definecolor{currentfill}{rgb}{1.000000,1.000000,1.000000}%
\pgfsetfillcolor{currentfill}%
\pgfsetlinewidth{0.000000pt}%
\definecolor{currentstroke}{rgb}{1.000000,1.000000,1.000000}%
\pgfsetstrokecolor{currentstroke}%
\pgfsetdash{}{0pt}%
\pgfpathmoveto{\pgfqpoint{0.000000in}{0.000000in}}%
\pgfpathlineto{\pgfqpoint{4.990125in}{0.000000in}}%
\pgfpathlineto{\pgfqpoint{4.990125in}{3.294691in}}%
\pgfpathlineto{\pgfqpoint{0.000000in}{3.294691in}}%
\pgfpathclose%
\pgfusepath{fill}%
\end{pgfscope}%
\begin{pgfscope}%
\pgfsetbuttcap%
\pgfsetmiterjoin%
\definecolor{currentfill}{rgb}{1.000000,1.000000,1.000000}%
\pgfsetfillcolor{currentfill}%
\pgfsetlinewidth{0.000000pt}%
\definecolor{currentstroke}{rgb}{0.000000,0.000000,0.000000}%
\pgfsetstrokecolor{currentstroke}%
\pgfsetstrokeopacity{0.000000}%
\pgfsetdash{}{0pt}%
\pgfpathmoveto{\pgfqpoint{1.015125in}{0.499691in}}%
\pgfpathlineto{\pgfqpoint{4.890125in}{0.499691in}}%
\pgfpathlineto{\pgfqpoint{4.890125in}{3.194691in}}%
\pgfpathlineto{\pgfqpoint{1.015125in}{3.194691in}}%
\pgfpathclose%
\pgfusepath{fill}%
\end{pgfscope}%
\begin{pgfscope}%
\pgfpathrectangle{\pgfqpoint{1.015125in}{0.499691in}}{\pgfqpoint{3.875000in}{2.695000in}}%
\pgfusepath{clip}%
\pgfsetbuttcap%
\pgfsetroundjoin%
\definecolor{currentfill}{rgb}{0.529412,0.807843,0.921569}%
\pgfsetfillcolor{currentfill}%
\pgfsetlinewidth{1.003750pt}%
\definecolor{currentstroke}{rgb}{0.529412,0.807843,0.921569}%
\pgfsetstrokecolor{currentstroke}%
\pgfsetdash{}{0pt}%
\pgfpathmoveto{\pgfqpoint{1.200712in}{3.024145in}}%
\pgfpathcurveto{\pgfqpoint{1.211762in}{3.024145in}}{\pgfqpoint{1.222361in}{3.028535in}}{\pgfqpoint{1.230175in}{3.036349in}}%
\pgfpathcurveto{\pgfqpoint{1.237988in}{3.044162in}}{\pgfqpoint{1.242378in}{3.054762in}}{\pgfqpoint{1.242378in}{3.065812in}}%
\pgfpathcurveto{\pgfqpoint{1.242378in}{3.076862in}}{\pgfqpoint{1.237988in}{3.087461in}}{\pgfqpoint{1.230175in}{3.095274in}}%
\pgfpathcurveto{\pgfqpoint{1.222361in}{3.103088in}}{\pgfqpoint{1.211762in}{3.107478in}}{\pgfqpoint{1.200712in}{3.107478in}}%
\pgfpathcurveto{\pgfqpoint{1.189662in}{3.107478in}}{\pgfqpoint{1.179063in}{3.103088in}}{\pgfqpoint{1.171249in}{3.095274in}}%
\pgfpathcurveto{\pgfqpoint{1.163435in}{3.087461in}}{\pgfqpoint{1.159045in}{3.076862in}}{\pgfqpoint{1.159045in}{3.065812in}}%
\pgfpathcurveto{\pgfqpoint{1.159045in}{3.054762in}}{\pgfqpoint{1.163435in}{3.044162in}}{\pgfqpoint{1.171249in}{3.036349in}}%
\pgfpathcurveto{\pgfqpoint{1.179063in}{3.028535in}}{\pgfqpoint{1.189662in}{3.024145in}}{\pgfqpoint{1.200712in}{3.024145in}}%
\pgfpathclose%
\pgfusepath{stroke,fill}%
\end{pgfscope}%
\begin{pgfscope}%
\pgfpathrectangle{\pgfqpoint{1.015125in}{0.499691in}}{\pgfqpoint{3.875000in}{2.695000in}}%
\pgfusepath{clip}%
\pgfsetbuttcap%
\pgfsetroundjoin%
\definecolor{currentfill}{rgb}{0.529412,0.807843,0.921569}%
\pgfsetfillcolor{currentfill}%
\pgfsetlinewidth{1.003750pt}%
\definecolor{currentstroke}{rgb}{0.529412,0.807843,0.921569}%
\pgfsetstrokecolor{currentstroke}%
\pgfsetdash{}{0pt}%
\pgfpathmoveto{\pgfqpoint{1.748611in}{2.622893in}}%
\pgfpathcurveto{\pgfqpoint{1.759661in}{2.622893in}}{\pgfqpoint{1.770260in}{2.627283in}}{\pgfqpoint{1.778074in}{2.635097in}}%
\pgfpathcurveto{\pgfqpoint{1.785887in}{2.642910in}}{\pgfqpoint{1.790278in}{2.653509in}}{\pgfqpoint{1.790278in}{2.664559in}}%
\pgfpathcurveto{\pgfqpoint{1.790278in}{2.675610in}}{\pgfqpoint{1.785887in}{2.686209in}}{\pgfqpoint{1.778074in}{2.694022in}}%
\pgfpathcurveto{\pgfqpoint{1.770260in}{2.701836in}}{\pgfqpoint{1.759661in}{2.706226in}}{\pgfqpoint{1.748611in}{2.706226in}}%
\pgfpathcurveto{\pgfqpoint{1.737561in}{2.706226in}}{\pgfqpoint{1.726962in}{2.701836in}}{\pgfqpoint{1.719148in}{2.694022in}}%
\pgfpathcurveto{\pgfqpoint{1.711335in}{2.686209in}}{\pgfqpoint{1.706944in}{2.675610in}}{\pgfqpoint{1.706944in}{2.664559in}}%
\pgfpathcurveto{\pgfqpoint{1.706944in}{2.653509in}}{\pgfqpoint{1.711335in}{2.642910in}}{\pgfqpoint{1.719148in}{2.635097in}}%
\pgfpathcurveto{\pgfqpoint{1.726962in}{2.627283in}}{\pgfqpoint{1.737561in}{2.622893in}}{\pgfqpoint{1.748611in}{2.622893in}}%
\pgfpathclose%
\pgfusepath{stroke,fill}%
\end{pgfscope}%
\begin{pgfscope}%
\pgfpathrectangle{\pgfqpoint{1.015125in}{0.499691in}}{\pgfqpoint{3.875000in}{2.695000in}}%
\pgfusepath{clip}%
\pgfsetbuttcap%
\pgfsetroundjoin%
\definecolor{currentfill}{rgb}{0.529412,0.807843,0.921569}%
\pgfsetfillcolor{currentfill}%
\pgfsetlinewidth{1.003750pt}%
\definecolor{currentstroke}{rgb}{0.529412,0.807843,0.921569}%
\pgfsetstrokecolor{currentstroke}%
\pgfsetdash{}{0pt}%
\pgfpathmoveto{\pgfqpoint{2.119843in}{2.330170in}}%
\pgfpathcurveto{\pgfqpoint{2.130893in}{2.330170in}}{\pgfqpoint{2.141492in}{2.334560in}}{\pgfqpoint{2.149306in}{2.342374in}}%
\pgfpathcurveto{\pgfqpoint{2.157120in}{2.350187in}}{\pgfqpoint{2.161510in}{2.360786in}}{\pgfqpoint{2.161510in}{2.371836in}}%
\pgfpathcurveto{\pgfqpoint{2.161510in}{2.382886in}}{\pgfqpoint{2.157120in}{2.393485in}}{\pgfqpoint{2.149306in}{2.401299in}}%
\pgfpathcurveto{\pgfqpoint{2.141492in}{2.409113in}}{\pgfqpoint{2.130893in}{2.413503in}}{\pgfqpoint{2.119843in}{2.413503in}}%
\pgfpathcurveto{\pgfqpoint{2.108793in}{2.413503in}}{\pgfqpoint{2.098194in}{2.409113in}}{\pgfqpoint{2.090381in}{2.401299in}}%
\pgfpathcurveto{\pgfqpoint{2.082567in}{2.393485in}}{\pgfqpoint{2.078177in}{2.382886in}}{\pgfqpoint{2.078177in}{2.371836in}}%
\pgfpathcurveto{\pgfqpoint{2.078177in}{2.360786in}}{\pgfqpoint{2.082567in}{2.350187in}}{\pgfqpoint{2.090381in}{2.342374in}}%
\pgfpathcurveto{\pgfqpoint{2.098194in}{2.334560in}}{\pgfqpoint{2.108793in}{2.330170in}}{\pgfqpoint{2.119843in}{2.330170in}}%
\pgfpathclose%
\pgfusepath{stroke,fill}%
\end{pgfscope}%
\begin{pgfscope}%
\pgfpathrectangle{\pgfqpoint{1.015125in}{0.499691in}}{\pgfqpoint{3.875000in}{2.695000in}}%
\pgfusepath{clip}%
\pgfsetbuttcap%
\pgfsetroundjoin%
\definecolor{currentfill}{rgb}{0.529412,0.807843,0.921569}%
\pgfsetfillcolor{currentfill}%
\pgfsetlinewidth{1.003750pt}%
\definecolor{currentstroke}{rgb}{0.529412,0.807843,0.921569}%
\pgfsetstrokecolor{currentstroke}%
\pgfsetdash{}{0pt}%
\pgfpathmoveto{\pgfqpoint{2.400974in}{2.089251in}}%
\pgfpathcurveto{\pgfqpoint{2.412024in}{2.089251in}}{\pgfqpoint{2.422623in}{2.093642in}}{\pgfqpoint{2.430437in}{2.101455in}}%
\pgfpathcurveto{\pgfqpoint{2.438251in}{2.109269in}}{\pgfqpoint{2.442641in}{2.119868in}}{\pgfqpoint{2.442641in}{2.130918in}}%
\pgfpathcurveto{\pgfqpoint{2.442641in}{2.141968in}}{\pgfqpoint{2.438251in}{2.152567in}}{\pgfqpoint{2.430437in}{2.160381in}}%
\pgfpathcurveto{\pgfqpoint{2.422623in}{2.168194in}}{\pgfqpoint{2.412024in}{2.172585in}}{\pgfqpoint{2.400974in}{2.172585in}}%
\pgfpathcurveto{\pgfqpoint{2.389924in}{2.172585in}}{\pgfqpoint{2.379325in}{2.168194in}}{\pgfqpoint{2.371511in}{2.160381in}}%
\pgfpathcurveto{\pgfqpoint{2.363698in}{2.152567in}}{\pgfqpoint{2.359307in}{2.141968in}}{\pgfqpoint{2.359307in}{2.130918in}}%
\pgfpathcurveto{\pgfqpoint{2.359307in}{2.119868in}}{\pgfqpoint{2.363698in}{2.109269in}}{\pgfqpoint{2.371511in}{2.101455in}}%
\pgfpathcurveto{\pgfqpoint{2.379325in}{2.093642in}}{\pgfqpoint{2.389924in}{2.089251in}}{\pgfqpoint{2.400974in}{2.089251in}}%
\pgfpathclose%
\pgfusepath{stroke,fill}%
\end{pgfscope}%
\begin{pgfscope}%
\pgfpathrectangle{\pgfqpoint{1.015125in}{0.499691in}}{\pgfqpoint{3.875000in}{2.695000in}}%
\pgfusepath{clip}%
\pgfsetbuttcap%
\pgfsetroundjoin%
\definecolor{currentfill}{rgb}{0.529412,0.807843,0.921569}%
\pgfsetfillcolor{currentfill}%
\pgfsetlinewidth{1.003750pt}%
\definecolor{currentstroke}{rgb}{0.529412,0.807843,0.921569}%
\pgfsetstrokecolor{currentstroke}%
\pgfsetdash{}{0pt}%
\pgfpathmoveto{\pgfqpoint{2.627308in}{1.825366in}}%
\pgfpathcurveto{\pgfqpoint{2.638358in}{1.825366in}}{\pgfqpoint{2.648958in}{1.829756in}}{\pgfqpoint{2.656771in}{1.837570in}}%
\pgfpathcurveto{\pgfqpoint{2.664585in}{1.845384in}}{\pgfqpoint{2.668975in}{1.855983in}}{\pgfqpoint{2.668975in}{1.867033in}}%
\pgfpathcurveto{\pgfqpoint{2.668975in}{1.878083in}}{\pgfqpoint{2.664585in}{1.888682in}}{\pgfqpoint{2.656771in}{1.896496in}}%
\pgfpathcurveto{\pgfqpoint{2.648958in}{1.904309in}}{\pgfqpoint{2.638358in}{1.908700in}}{\pgfqpoint{2.627308in}{1.908700in}}%
\pgfpathcurveto{\pgfqpoint{2.616258in}{1.908700in}}{\pgfqpoint{2.605659in}{1.904309in}}{\pgfqpoint{2.597846in}{1.896496in}}%
\pgfpathcurveto{\pgfqpoint{2.590032in}{1.888682in}}{\pgfqpoint{2.585642in}{1.878083in}}{\pgfqpoint{2.585642in}{1.867033in}}%
\pgfpathcurveto{\pgfqpoint{2.585642in}{1.855983in}}{\pgfqpoint{2.590032in}{1.845384in}}{\pgfqpoint{2.597846in}{1.837570in}}%
\pgfpathcurveto{\pgfqpoint{2.605659in}{1.829756in}}{\pgfqpoint{2.616258in}{1.825366in}}{\pgfqpoint{2.627308in}{1.825366in}}%
\pgfpathclose%
\pgfusepath{stroke,fill}%
\end{pgfscope}%
\begin{pgfscope}%
\pgfpathrectangle{\pgfqpoint{1.015125in}{0.499691in}}{\pgfqpoint{3.875000in}{2.695000in}}%
\pgfusepath{clip}%
\pgfsetbuttcap%
\pgfsetroundjoin%
\definecolor{currentfill}{rgb}{0.529412,0.807843,0.921569}%
\pgfsetfillcolor{currentfill}%
\pgfsetlinewidth{1.003750pt}%
\definecolor{currentstroke}{rgb}{0.529412,0.807843,0.921569}%
\pgfsetstrokecolor{currentstroke}%
\pgfsetdash{}{0pt}%
\pgfpathmoveto{\pgfqpoint{2.816762in}{1.674620in}}%
\pgfpathcurveto{\pgfqpoint{2.827812in}{1.674620in}}{\pgfqpoint{2.838411in}{1.679010in}}{\pgfqpoint{2.846225in}{1.686824in}}%
\pgfpathcurveto{\pgfqpoint{2.854039in}{1.694637in}}{\pgfqpoint{2.858429in}{1.705236in}}{\pgfqpoint{2.858429in}{1.716287in}}%
\pgfpathcurveto{\pgfqpoint{2.858429in}{1.727337in}}{\pgfqpoint{2.854039in}{1.737936in}}{\pgfqpoint{2.846225in}{1.745749in}}%
\pgfpathcurveto{\pgfqpoint{2.838411in}{1.753563in}}{\pgfqpoint{2.827812in}{1.757953in}}{\pgfqpoint{2.816762in}{1.757953in}}%
\pgfpathcurveto{\pgfqpoint{2.805712in}{1.757953in}}{\pgfqpoint{2.795113in}{1.753563in}}{\pgfqpoint{2.787299in}{1.745749in}}%
\pgfpathcurveto{\pgfqpoint{2.779486in}{1.737936in}}{\pgfqpoint{2.775096in}{1.727337in}}{\pgfqpoint{2.775096in}{1.716287in}}%
\pgfpathcurveto{\pgfqpoint{2.775096in}{1.705236in}}{\pgfqpoint{2.779486in}{1.694637in}}{\pgfqpoint{2.787299in}{1.686824in}}%
\pgfpathcurveto{\pgfqpoint{2.795113in}{1.679010in}}{\pgfqpoint{2.805712in}{1.674620in}}{\pgfqpoint{2.816762in}{1.674620in}}%
\pgfpathclose%
\pgfusepath{stroke,fill}%
\end{pgfscope}%
\begin{pgfscope}%
\pgfpathrectangle{\pgfqpoint{1.015125in}{0.499691in}}{\pgfqpoint{3.875000in}{2.695000in}}%
\pgfusepath{clip}%
\pgfsetbuttcap%
\pgfsetroundjoin%
\definecolor{currentfill}{rgb}{0.529412,0.807843,0.921569}%
\pgfsetfillcolor{currentfill}%
\pgfsetlinewidth{1.003750pt}%
\definecolor{currentstroke}{rgb}{0.529412,0.807843,0.921569}%
\pgfsetstrokecolor{currentstroke}%
\pgfsetdash{}{0pt}%
\pgfpathmoveto{\pgfqpoint{2.979687in}{1.573824in}}%
\pgfpathcurveto{\pgfqpoint{2.990738in}{1.573824in}}{\pgfqpoint{3.001337in}{1.578215in}}{\pgfqpoint{3.009150in}{1.586028in}}%
\pgfpathcurveto{\pgfqpoint{3.016964in}{1.593842in}}{\pgfqpoint{3.021354in}{1.604441in}}{\pgfqpoint{3.021354in}{1.615491in}}%
\pgfpathcurveto{\pgfqpoint{3.021354in}{1.626541in}}{\pgfqpoint{3.016964in}{1.637140in}}{\pgfqpoint{3.009150in}{1.644954in}}%
\pgfpathcurveto{\pgfqpoint{3.001337in}{1.652767in}}{\pgfqpoint{2.990738in}{1.657158in}}{\pgfqpoint{2.979687in}{1.657158in}}%
\pgfpathcurveto{\pgfqpoint{2.968637in}{1.657158in}}{\pgfqpoint{2.958038in}{1.652767in}}{\pgfqpoint{2.950225in}{1.644954in}}%
\pgfpathcurveto{\pgfqpoint{2.942411in}{1.637140in}}{\pgfqpoint{2.938021in}{1.626541in}}{\pgfqpoint{2.938021in}{1.615491in}}%
\pgfpathcurveto{\pgfqpoint{2.938021in}{1.604441in}}{\pgfqpoint{2.942411in}{1.593842in}}{\pgfqpoint{2.950225in}{1.586028in}}%
\pgfpathcurveto{\pgfqpoint{2.958038in}{1.578215in}}{\pgfqpoint{2.968637in}{1.573824in}}{\pgfqpoint{2.979687in}{1.573824in}}%
\pgfpathclose%
\pgfusepath{stroke,fill}%
\end{pgfscope}%
\begin{pgfscope}%
\pgfpathrectangle{\pgfqpoint{1.015125in}{0.499691in}}{\pgfqpoint{3.875000in}{2.695000in}}%
\pgfusepath{clip}%
\pgfsetbuttcap%
\pgfsetroundjoin%
\definecolor{currentfill}{rgb}{0.529412,0.807843,0.921569}%
\pgfsetfillcolor{currentfill}%
\pgfsetlinewidth{1.003750pt}%
\definecolor{currentstroke}{rgb}{0.529412,0.807843,0.921569}%
\pgfsetstrokecolor{currentstroke}%
\pgfsetdash{}{0pt}%
\pgfpathmoveto{\pgfqpoint{3.122609in}{1.481557in}}%
\pgfpathcurveto{\pgfqpoint{3.133659in}{1.481557in}}{\pgfqpoint{3.144258in}{1.485948in}}{\pgfqpoint{3.152072in}{1.493761in}}%
\pgfpathcurveto{\pgfqpoint{3.159886in}{1.501575in}}{\pgfqpoint{3.164276in}{1.512174in}}{\pgfqpoint{3.164276in}{1.523224in}}%
\pgfpathcurveto{\pgfqpoint{3.164276in}{1.534274in}}{\pgfqpoint{3.159886in}{1.544873in}}{\pgfqpoint{3.152072in}{1.552687in}}%
\pgfpathcurveto{\pgfqpoint{3.144258in}{1.560501in}}{\pgfqpoint{3.133659in}{1.564891in}}{\pgfqpoint{3.122609in}{1.564891in}}%
\pgfpathcurveto{\pgfqpoint{3.111559in}{1.564891in}}{\pgfqpoint{3.100960in}{1.560501in}}{\pgfqpoint{3.093146in}{1.552687in}}%
\pgfpathcurveto{\pgfqpoint{3.085333in}{1.544873in}}{\pgfqpoint{3.080942in}{1.534274in}}{\pgfqpoint{3.080942in}{1.523224in}}%
\pgfpathcurveto{\pgfqpoint{3.080942in}{1.512174in}}{\pgfqpoint{3.085333in}{1.501575in}}{\pgfqpoint{3.093146in}{1.493761in}}%
\pgfpathcurveto{\pgfqpoint{3.100960in}{1.485948in}}{\pgfqpoint{3.111559in}{1.481557in}}{\pgfqpoint{3.122609in}{1.481557in}}%
\pgfpathclose%
\pgfusepath{stroke,fill}%
\end{pgfscope}%
\begin{pgfscope}%
\pgfpathrectangle{\pgfqpoint{1.015125in}{0.499691in}}{\pgfqpoint{3.875000in}{2.695000in}}%
\pgfusepath{clip}%
\pgfsetbuttcap%
\pgfsetroundjoin%
\definecolor{currentfill}{rgb}{0.529412,0.807843,0.921569}%
\pgfsetfillcolor{currentfill}%
\pgfsetlinewidth{1.003750pt}%
\definecolor{currentstroke}{rgb}{0.529412,0.807843,0.921569}%
\pgfsetstrokecolor{currentstroke}%
\pgfsetdash{}{0pt}%
\pgfpathmoveto{\pgfqpoint{3.249906in}{1.343249in}}%
\pgfpathcurveto{\pgfqpoint{3.260957in}{1.343249in}}{\pgfqpoint{3.271556in}{1.347639in}}{\pgfqpoint{3.279369in}{1.355453in}}%
\pgfpathcurveto{\pgfqpoint{3.287183in}{1.363267in}}{\pgfqpoint{3.291573in}{1.373866in}}{\pgfqpoint{3.291573in}{1.384916in}}%
\pgfpathcurveto{\pgfqpoint{3.291573in}{1.395966in}}{\pgfqpoint{3.287183in}{1.406565in}}{\pgfqpoint{3.279369in}{1.414379in}}%
\pgfpathcurveto{\pgfqpoint{3.271556in}{1.422192in}}{\pgfqpoint{3.260957in}{1.426582in}}{\pgfqpoint{3.249906in}{1.426582in}}%
\pgfpathcurveto{\pgfqpoint{3.238856in}{1.426582in}}{\pgfqpoint{3.228257in}{1.422192in}}{\pgfqpoint{3.220444in}{1.414379in}}%
\pgfpathcurveto{\pgfqpoint{3.212630in}{1.406565in}}{\pgfqpoint{3.208240in}{1.395966in}}{\pgfqpoint{3.208240in}{1.384916in}}%
\pgfpathcurveto{\pgfqpoint{3.208240in}{1.373866in}}{\pgfqpoint{3.212630in}{1.363267in}}{\pgfqpoint{3.220444in}{1.355453in}}%
\pgfpathcurveto{\pgfqpoint{3.228257in}{1.347639in}}{\pgfqpoint{3.238856in}{1.343249in}}{\pgfqpoint{3.249906in}{1.343249in}}%
\pgfpathclose%
\pgfusepath{stroke,fill}%
\end{pgfscope}%
\begin{pgfscope}%
\pgfpathrectangle{\pgfqpoint{1.015125in}{0.499691in}}{\pgfqpoint{3.875000in}{2.695000in}}%
\pgfusepath{clip}%
\pgfsetbuttcap%
\pgfsetroundjoin%
\definecolor{currentfill}{rgb}{0.529412,0.807843,0.921569}%
\pgfsetfillcolor{currentfill}%
\pgfsetlinewidth{1.003750pt}%
\definecolor{currentstroke}{rgb}{0.529412,0.807843,0.921569}%
\pgfsetstrokecolor{currentstroke}%
\pgfsetdash{}{0pt}%
\pgfpathmoveto{\pgfqpoint{3.364661in}{1.292488in}}%
\pgfpathcurveto{\pgfqpoint{3.375712in}{1.292488in}}{\pgfqpoint{3.386311in}{1.296878in}}{\pgfqpoint{3.394124in}{1.304691in}}%
\pgfpathcurveto{\pgfqpoint{3.401938in}{1.312505in}}{\pgfqpoint{3.406328in}{1.323104in}}{\pgfqpoint{3.406328in}{1.334154in}}%
\pgfpathcurveto{\pgfqpoint{3.406328in}{1.345204in}}{\pgfqpoint{3.401938in}{1.355803in}}{\pgfqpoint{3.394124in}{1.363617in}}%
\pgfpathcurveto{\pgfqpoint{3.386311in}{1.371431in}}{\pgfqpoint{3.375712in}{1.375821in}}{\pgfqpoint{3.364661in}{1.375821in}}%
\pgfpathcurveto{\pgfqpoint{3.353611in}{1.375821in}}{\pgfqpoint{3.343012in}{1.371431in}}{\pgfqpoint{3.335199in}{1.363617in}}%
\pgfpathcurveto{\pgfqpoint{3.327385in}{1.355803in}}{\pgfqpoint{3.322995in}{1.345204in}}{\pgfqpoint{3.322995in}{1.334154in}}%
\pgfpathcurveto{\pgfqpoint{3.322995in}{1.323104in}}{\pgfqpoint{3.327385in}{1.312505in}}{\pgfqpoint{3.335199in}{1.304691in}}%
\pgfpathcurveto{\pgfqpoint{3.343012in}{1.296878in}}{\pgfqpoint{3.353611in}{1.292488in}}{\pgfqpoint{3.364661in}{1.292488in}}%
\pgfpathclose%
\pgfusepath{stroke,fill}%
\end{pgfscope}%
\begin{pgfscope}%
\pgfpathrectangle{\pgfqpoint{1.015125in}{0.499691in}}{\pgfqpoint{3.875000in}{2.695000in}}%
\pgfusepath{clip}%
\pgfsetbuttcap%
\pgfsetroundjoin%
\definecolor{currentfill}{rgb}{0.529412,0.807843,0.921569}%
\pgfsetfillcolor{currentfill}%
\pgfsetlinewidth{1.003750pt}%
\definecolor{currentstroke}{rgb}{0.529412,0.807843,0.921569}%
\pgfsetstrokecolor{currentstroke}%
\pgfsetdash{}{0pt}%
\pgfpathmoveto{\pgfqpoint{3.469125in}{1.197543in}}%
\pgfpathcurveto{\pgfqpoint{3.480175in}{1.197543in}}{\pgfqpoint{3.490775in}{1.201933in}}{\pgfqpoint{3.498588in}{1.209747in}}%
\pgfpathcurveto{\pgfqpoint{3.506402in}{1.217560in}}{\pgfqpoint{3.510792in}{1.228159in}}{\pgfqpoint{3.510792in}{1.239210in}}%
\pgfpathcurveto{\pgfqpoint{3.510792in}{1.250260in}}{\pgfqpoint{3.506402in}{1.260859in}}{\pgfqpoint{3.498588in}{1.268672in}}%
\pgfpathcurveto{\pgfqpoint{3.490775in}{1.276486in}}{\pgfqpoint{3.480175in}{1.280876in}}{\pgfqpoint{3.469125in}{1.280876in}}%
\pgfpathcurveto{\pgfqpoint{3.458075in}{1.280876in}}{\pgfqpoint{3.447476in}{1.276486in}}{\pgfqpoint{3.439663in}{1.268672in}}%
\pgfpathcurveto{\pgfqpoint{3.431849in}{1.260859in}}{\pgfqpoint{3.427459in}{1.250260in}}{\pgfqpoint{3.427459in}{1.239210in}}%
\pgfpathcurveto{\pgfqpoint{3.427459in}{1.228159in}}{\pgfqpoint{3.431849in}{1.217560in}}{\pgfqpoint{3.439663in}{1.209747in}}%
\pgfpathcurveto{\pgfqpoint{3.447476in}{1.201933in}}{\pgfqpoint{3.458075in}{1.197543in}}{\pgfqpoint{3.469125in}{1.197543in}}%
\pgfpathclose%
\pgfusepath{stroke,fill}%
\end{pgfscope}%
\begin{pgfscope}%
\pgfpathrectangle{\pgfqpoint{1.015125in}{0.499691in}}{\pgfqpoint{3.875000in}{2.695000in}}%
\pgfusepath{clip}%
\pgfsetbuttcap%
\pgfsetroundjoin%
\definecolor{currentfill}{rgb}{0.529412,0.807843,0.921569}%
\pgfsetfillcolor{currentfill}%
\pgfsetlinewidth{1.003750pt}%
\definecolor{currentstroke}{rgb}{0.529412,0.807843,0.921569}%
\pgfsetstrokecolor{currentstroke}%
\pgfsetdash{}{0pt}%
\pgfpathmoveto{\pgfqpoint{3.564993in}{1.131391in}}%
\pgfpathcurveto{\pgfqpoint{3.576043in}{1.131391in}}{\pgfqpoint{3.586642in}{1.135781in}}{\pgfqpoint{3.594456in}{1.143595in}}%
\pgfpathcurveto{\pgfqpoint{3.602269in}{1.151408in}}{\pgfqpoint{3.606660in}{1.162007in}}{\pgfqpoint{3.606660in}{1.173057in}}%
\pgfpathcurveto{\pgfqpoint{3.606660in}{1.184108in}}{\pgfqpoint{3.602269in}{1.194707in}}{\pgfqpoint{3.594456in}{1.202520in}}%
\pgfpathcurveto{\pgfqpoint{3.586642in}{1.210334in}}{\pgfqpoint{3.576043in}{1.214724in}}{\pgfqpoint{3.564993in}{1.214724in}}%
\pgfpathcurveto{\pgfqpoint{3.553943in}{1.214724in}}{\pgfqpoint{3.543344in}{1.210334in}}{\pgfqpoint{3.535530in}{1.202520in}}%
\pgfpathcurveto{\pgfqpoint{3.527717in}{1.194707in}}{\pgfqpoint{3.523326in}{1.184108in}}{\pgfqpoint{3.523326in}{1.173057in}}%
\pgfpathcurveto{\pgfqpoint{3.523326in}{1.162007in}}{\pgfqpoint{3.527717in}{1.151408in}}{\pgfqpoint{3.535530in}{1.143595in}}%
\pgfpathcurveto{\pgfqpoint{3.543344in}{1.135781in}}{\pgfqpoint{3.553943in}{1.131391in}}{\pgfqpoint{3.564993in}{1.131391in}}%
\pgfpathclose%
\pgfusepath{stroke,fill}%
\end{pgfscope}%
\begin{pgfscope}%
\pgfpathrectangle{\pgfqpoint{1.015125in}{0.499691in}}{\pgfqpoint{3.875000in}{2.695000in}}%
\pgfusepath{clip}%
\pgfsetbuttcap%
\pgfsetroundjoin%
\definecolor{currentfill}{rgb}{0.529412,0.807843,0.921569}%
\pgfsetfillcolor{currentfill}%
\pgfsetlinewidth{1.003750pt}%
\definecolor{currentstroke}{rgb}{0.529412,0.807843,0.921569}%
\pgfsetstrokecolor{currentstroke}%
\pgfsetdash{}{0pt}%
\pgfpathmoveto{\pgfqpoint{3.653572in}{1.069021in}}%
\pgfpathcurveto{\pgfqpoint{3.664623in}{1.069021in}}{\pgfqpoint{3.675222in}{1.073411in}}{\pgfqpoint{3.683035in}{1.081225in}}%
\pgfpathcurveto{\pgfqpoint{3.690849in}{1.089038in}}{\pgfqpoint{3.695239in}{1.099638in}}{\pgfqpoint{3.695239in}{1.110688in}}%
\pgfpathcurveto{\pgfqpoint{3.695239in}{1.121738in}}{\pgfqpoint{3.690849in}{1.132337in}}{\pgfqpoint{3.683035in}{1.140150in}}%
\pgfpathcurveto{\pgfqpoint{3.675222in}{1.147964in}}{\pgfqpoint{3.664623in}{1.152354in}}{\pgfqpoint{3.653572in}{1.152354in}}%
\pgfpathcurveto{\pgfqpoint{3.642522in}{1.152354in}}{\pgfqpoint{3.631923in}{1.147964in}}{\pgfqpoint{3.624110in}{1.140150in}}%
\pgfpathcurveto{\pgfqpoint{3.616296in}{1.132337in}}{\pgfqpoint{3.611906in}{1.121738in}}{\pgfqpoint{3.611906in}{1.110688in}}%
\pgfpathcurveto{\pgfqpoint{3.611906in}{1.099638in}}{\pgfqpoint{3.616296in}{1.089038in}}{\pgfqpoint{3.624110in}{1.081225in}}%
\pgfpathcurveto{\pgfqpoint{3.631923in}{1.073411in}}{\pgfqpoint{3.642522in}{1.069021in}}{\pgfqpoint{3.653572in}{1.069021in}}%
\pgfpathclose%
\pgfusepath{stroke,fill}%
\end{pgfscope}%
\begin{pgfscope}%
\pgfpathrectangle{\pgfqpoint{1.015125in}{0.499691in}}{\pgfqpoint{3.875000in}{2.695000in}}%
\pgfusepath{clip}%
\pgfsetbuttcap%
\pgfsetroundjoin%
\definecolor{currentfill}{rgb}{0.529412,0.807843,0.921569}%
\pgfsetfillcolor{currentfill}%
\pgfsetlinewidth{1.003750pt}%
\definecolor{currentstroke}{rgb}{0.529412,0.807843,0.921569}%
\pgfsetstrokecolor{currentstroke}%
\pgfsetdash{}{0pt}%
\pgfpathmoveto{\pgfqpoint{3.735894in}{1.029337in}}%
\pgfpathcurveto{\pgfqpoint{3.746944in}{1.029337in}}{\pgfqpoint{3.757543in}{1.033728in}}{\pgfqpoint{3.765357in}{1.041541in}}%
\pgfpathcurveto{\pgfqpoint{3.773170in}{1.049355in}}{\pgfqpoint{3.777560in}{1.059954in}}{\pgfqpoint{3.777560in}{1.071004in}}%
\pgfpathcurveto{\pgfqpoint{3.777560in}{1.082054in}}{\pgfqpoint{3.773170in}{1.092653in}}{\pgfqpoint{3.765357in}{1.100467in}}%
\pgfpathcurveto{\pgfqpoint{3.757543in}{1.108281in}}{\pgfqpoint{3.746944in}{1.112671in}}{\pgfqpoint{3.735894in}{1.112671in}}%
\pgfpathcurveto{\pgfqpoint{3.724844in}{1.112671in}}{\pgfqpoint{3.714245in}{1.108281in}}{\pgfqpoint{3.706431in}{1.100467in}}%
\pgfpathcurveto{\pgfqpoint{3.698617in}{1.092653in}}{\pgfqpoint{3.694227in}{1.082054in}}{\pgfqpoint{3.694227in}{1.071004in}}%
\pgfpathcurveto{\pgfqpoint{3.694227in}{1.059954in}}{\pgfqpoint{3.698617in}{1.049355in}}{\pgfqpoint{3.706431in}{1.041541in}}%
\pgfpathcurveto{\pgfqpoint{3.714245in}{1.033728in}}{\pgfqpoint{3.724844in}{1.029337in}}{\pgfqpoint{3.735894in}{1.029337in}}%
\pgfpathclose%
\pgfusepath{stroke,fill}%
\end{pgfscope}%
\begin{pgfscope}%
\pgfpathrectangle{\pgfqpoint{1.015125in}{0.499691in}}{\pgfqpoint{3.875000in}{2.695000in}}%
\pgfusepath{clip}%
\pgfsetbuttcap%
\pgfsetroundjoin%
\definecolor{currentfill}{rgb}{0.529412,0.807843,0.921569}%
\pgfsetfillcolor{currentfill}%
\pgfsetlinewidth{1.003750pt}%
\definecolor{currentstroke}{rgb}{0.529412,0.807843,0.921569}%
\pgfsetstrokecolor{currentstroke}%
\pgfsetdash{}{0pt}%
\pgfpathmoveto{\pgfqpoint{3.812783in}{0.972394in}}%
\pgfpathcurveto{\pgfqpoint{3.823833in}{0.972394in}}{\pgfqpoint{3.834432in}{0.976784in}}{\pgfqpoint{3.842246in}{0.984598in}}%
\pgfpathcurveto{\pgfqpoint{3.850060in}{0.992411in}}{\pgfqpoint{3.854450in}{1.003010in}}{\pgfqpoint{3.854450in}{1.014060in}}%
\pgfpathcurveto{\pgfqpoint{3.854450in}{1.025110in}}{\pgfqpoint{3.850060in}{1.035710in}}{\pgfqpoint{3.842246in}{1.043523in}}%
\pgfpathcurveto{\pgfqpoint{3.834432in}{1.051337in}}{\pgfqpoint{3.823833in}{1.055727in}}{\pgfqpoint{3.812783in}{1.055727in}}%
\pgfpathcurveto{\pgfqpoint{3.801733in}{1.055727in}}{\pgfqpoint{3.791134in}{1.051337in}}{\pgfqpoint{3.783320in}{1.043523in}}%
\pgfpathcurveto{\pgfqpoint{3.775507in}{1.035710in}}{\pgfqpoint{3.771117in}{1.025110in}}{\pgfqpoint{3.771117in}{1.014060in}}%
\pgfpathcurveto{\pgfqpoint{3.771117in}{1.003010in}}{\pgfqpoint{3.775507in}{0.992411in}}{\pgfqpoint{3.783320in}{0.984598in}}%
\pgfpathcurveto{\pgfqpoint{3.791134in}{0.976784in}}{\pgfqpoint{3.801733in}{0.972394in}}{\pgfqpoint{3.812783in}{0.972394in}}%
\pgfpathclose%
\pgfusepath{stroke,fill}%
\end{pgfscope}%
\begin{pgfscope}%
\pgfpathrectangle{\pgfqpoint{1.015125in}{0.499691in}}{\pgfqpoint{3.875000in}{2.695000in}}%
\pgfusepath{clip}%
\pgfsetbuttcap%
\pgfsetroundjoin%
\definecolor{currentfill}{rgb}{0.529412,0.807843,0.921569}%
\pgfsetfillcolor{currentfill}%
\pgfsetlinewidth{1.003750pt}%
\definecolor{currentstroke}{rgb}{0.529412,0.807843,0.921569}%
\pgfsetstrokecolor{currentstroke}%
\pgfsetdash{}{0pt}%
\pgfpathmoveto{\pgfqpoint{3.884913in}{0.936018in}}%
\pgfpathcurveto{\pgfqpoint{3.895964in}{0.936018in}}{\pgfqpoint{3.906563in}{0.940408in}}{\pgfqpoint{3.914376in}{0.948222in}}%
\pgfpathcurveto{\pgfqpoint{3.922190in}{0.956035in}}{\pgfqpoint{3.926580in}{0.966634in}}{\pgfqpoint{3.926580in}{0.977684in}}%
\pgfpathcurveto{\pgfqpoint{3.926580in}{0.988735in}}{\pgfqpoint{3.922190in}{0.999334in}}{\pgfqpoint{3.914376in}{1.007147in}}%
\pgfpathcurveto{\pgfqpoint{3.906563in}{1.014961in}}{\pgfqpoint{3.895964in}{1.019351in}}{\pgfqpoint{3.884913in}{1.019351in}}%
\pgfpathcurveto{\pgfqpoint{3.873863in}{1.019351in}}{\pgfqpoint{3.863264in}{1.014961in}}{\pgfqpoint{3.855451in}{1.007147in}}%
\pgfpathcurveto{\pgfqpoint{3.847637in}{0.999334in}}{\pgfqpoint{3.843247in}{0.988735in}}{\pgfqpoint{3.843247in}{0.977684in}}%
\pgfpathcurveto{\pgfqpoint{3.843247in}{0.966634in}}{\pgfqpoint{3.847637in}{0.956035in}}{\pgfqpoint{3.855451in}{0.948222in}}%
\pgfpathcurveto{\pgfqpoint{3.863264in}{0.940408in}}{\pgfqpoint{3.873863in}{0.936018in}}{\pgfqpoint{3.884913in}{0.936018in}}%
\pgfpathclose%
\pgfusepath{stroke,fill}%
\end{pgfscope}%
\begin{pgfscope}%
\pgfpathrectangle{\pgfqpoint{1.015125in}{0.499691in}}{\pgfqpoint{3.875000in}{2.695000in}}%
\pgfusepath{clip}%
\pgfsetbuttcap%
\pgfsetroundjoin%
\definecolor{currentfill}{rgb}{0.529412,0.807843,0.921569}%
\pgfsetfillcolor{currentfill}%
\pgfsetlinewidth{1.003750pt}%
\definecolor{currentstroke}{rgb}{0.529412,0.807843,0.921569}%
\pgfsetstrokecolor{currentstroke}%
\pgfsetdash{}{0pt}%
\pgfpathmoveto{\pgfqpoint{3.952839in}{0.883631in}}%
\pgfpathcurveto{\pgfqpoint{3.963890in}{0.883631in}}{\pgfqpoint{3.974489in}{0.888022in}}{\pgfqpoint{3.982302in}{0.895835in}}%
\pgfpathcurveto{\pgfqpoint{3.990116in}{0.903649in}}{\pgfqpoint{3.994506in}{0.914248in}}{\pgfqpoint{3.994506in}{0.925298in}}%
\pgfpathcurveto{\pgfqpoint{3.994506in}{0.936348in}}{\pgfqpoint{3.990116in}{0.946947in}}{\pgfqpoint{3.982302in}{0.954761in}}%
\pgfpathcurveto{\pgfqpoint{3.974489in}{0.962574in}}{\pgfqpoint{3.963890in}{0.966965in}}{\pgfqpoint{3.952839in}{0.966965in}}%
\pgfpathcurveto{\pgfqpoint{3.941789in}{0.966965in}}{\pgfqpoint{3.931190in}{0.962574in}}{\pgfqpoint{3.923377in}{0.954761in}}%
\pgfpathcurveto{\pgfqpoint{3.915563in}{0.946947in}}{\pgfqpoint{3.911173in}{0.936348in}}{\pgfqpoint{3.911173in}{0.925298in}}%
\pgfpathcurveto{\pgfqpoint{3.911173in}{0.914248in}}{\pgfqpoint{3.915563in}{0.903649in}}{\pgfqpoint{3.923377in}{0.895835in}}%
\pgfpathcurveto{\pgfqpoint{3.931190in}{0.888022in}}{\pgfqpoint{3.941789in}{0.883631in}}{\pgfqpoint{3.952839in}{0.883631in}}%
\pgfpathclose%
\pgfusepath{stroke,fill}%
\end{pgfscope}%
\begin{pgfscope}%
\pgfpathrectangle{\pgfqpoint{1.015125in}{0.499691in}}{\pgfqpoint{3.875000in}{2.695000in}}%
\pgfusepath{clip}%
\pgfsetbuttcap%
\pgfsetroundjoin%
\definecolor{currentfill}{rgb}{0.529412,0.807843,0.921569}%
\pgfsetfillcolor{currentfill}%
\pgfsetlinewidth{1.003750pt}%
\definecolor{currentstroke}{rgb}{0.529412,0.807843,0.921569}%
\pgfsetstrokecolor{currentstroke}%
\pgfsetdash{}{0pt}%
\pgfpathmoveto{\pgfqpoint{4.017025in}{0.850054in}}%
\pgfpathcurveto{\pgfqpoint{4.028075in}{0.850054in}}{\pgfqpoint{4.038674in}{0.854444in}}{\pgfqpoint{4.046487in}{0.862258in}}%
\pgfpathcurveto{\pgfqpoint{4.054301in}{0.870072in}}{\pgfqpoint{4.058691in}{0.880671in}}{\pgfqpoint{4.058691in}{0.891721in}}%
\pgfpathcurveto{\pgfqpoint{4.058691in}{0.902771in}}{\pgfqpoint{4.054301in}{0.913370in}}{\pgfqpoint{4.046487in}{0.921183in}}%
\pgfpathcurveto{\pgfqpoint{4.038674in}{0.928997in}}{\pgfqpoint{4.028075in}{0.933387in}}{\pgfqpoint{4.017025in}{0.933387in}}%
\pgfpathcurveto{\pgfqpoint{4.005974in}{0.933387in}}{\pgfqpoint{3.995375in}{0.928997in}}{\pgfqpoint{3.987562in}{0.921183in}}%
\pgfpathcurveto{\pgfqpoint{3.979748in}{0.913370in}}{\pgfqpoint{3.975358in}{0.902771in}}{\pgfqpoint{3.975358in}{0.891721in}}%
\pgfpathcurveto{\pgfqpoint{3.975358in}{0.880671in}}{\pgfqpoint{3.979748in}{0.870072in}}{\pgfqpoint{3.987562in}{0.862258in}}%
\pgfpathcurveto{\pgfqpoint{3.995375in}{0.854444in}}{\pgfqpoint{4.005974in}{0.850054in}}{\pgfqpoint{4.017025in}{0.850054in}}%
\pgfpathclose%
\pgfusepath{stroke,fill}%
\end{pgfscope}%
\begin{pgfscope}%
\pgfpathrectangle{\pgfqpoint{1.015125in}{0.499691in}}{\pgfqpoint{3.875000in}{2.695000in}}%
\pgfusepath{clip}%
\pgfsetbuttcap%
\pgfsetroundjoin%
\definecolor{currentfill}{rgb}{0.529412,0.807843,0.921569}%
\pgfsetfillcolor{currentfill}%
\pgfsetlinewidth{1.003750pt}%
\definecolor{currentstroke}{rgb}{0.529412,0.807843,0.921569}%
\pgfsetstrokecolor{currentstroke}%
\pgfsetdash{}{0pt}%
\pgfpathmoveto{\pgfqpoint{4.077859in}{0.825532in}}%
\pgfpathcurveto{\pgfqpoint{4.088909in}{0.825532in}}{\pgfqpoint{4.099508in}{0.829922in}}{\pgfqpoint{4.107322in}{0.837736in}}%
\pgfpathcurveto{\pgfqpoint{4.115136in}{0.845550in}}{\pgfqpoint{4.119526in}{0.856149in}}{\pgfqpoint{4.119526in}{0.867199in}}%
\pgfpathcurveto{\pgfqpoint{4.119526in}{0.878249in}}{\pgfqpoint{4.115136in}{0.888848in}}{\pgfqpoint{4.107322in}{0.896662in}}%
\pgfpathcurveto{\pgfqpoint{4.099508in}{0.904475in}}{\pgfqpoint{4.088909in}{0.908866in}}{\pgfqpoint{4.077859in}{0.908866in}}%
\pgfpathcurveto{\pgfqpoint{4.066809in}{0.908866in}}{\pgfqpoint{4.056210in}{0.904475in}}{\pgfqpoint{4.048396in}{0.896662in}}%
\pgfpathcurveto{\pgfqpoint{4.040583in}{0.888848in}}{\pgfqpoint{4.036193in}{0.878249in}}{\pgfqpoint{4.036193in}{0.867199in}}%
\pgfpathcurveto{\pgfqpoint{4.036193in}{0.856149in}}{\pgfqpoint{4.040583in}{0.845550in}}{\pgfqpoint{4.048396in}{0.837736in}}%
\pgfpathcurveto{\pgfqpoint{4.056210in}{0.829922in}}{\pgfqpoint{4.066809in}{0.825532in}}{\pgfqpoint{4.077859in}{0.825532in}}%
\pgfpathclose%
\pgfusepath{stroke,fill}%
\end{pgfscope}%
\begin{pgfscope}%
\pgfpathrectangle{\pgfqpoint{1.015125in}{0.499691in}}{\pgfqpoint{3.875000in}{2.695000in}}%
\pgfusepath{clip}%
\pgfsetbuttcap%
\pgfsetroundjoin%
\definecolor{currentfill}{rgb}{0.529412,0.807843,0.921569}%
\pgfsetfillcolor{currentfill}%
\pgfsetlinewidth{1.003750pt}%
\definecolor{currentstroke}{rgb}{0.529412,0.807843,0.921569}%
\pgfsetstrokecolor{currentstroke}%
\pgfsetdash{}{0pt}%
\pgfpathmoveto{\pgfqpoint{4.135676in}{0.793671in}}%
\pgfpathcurveto{\pgfqpoint{4.146726in}{0.793671in}}{\pgfqpoint{4.157325in}{0.798061in}}{\pgfqpoint{4.165139in}{0.805875in}}%
\pgfpathcurveto{\pgfqpoint{4.172952in}{0.813688in}}{\pgfqpoint{4.177343in}{0.824287in}}{\pgfqpoint{4.177343in}{0.835337in}}%
\pgfpathcurveto{\pgfqpoint{4.177343in}{0.846388in}}{\pgfqpoint{4.172952in}{0.856987in}}{\pgfqpoint{4.165139in}{0.864800in}}%
\pgfpathcurveto{\pgfqpoint{4.157325in}{0.872614in}}{\pgfqpoint{4.146726in}{0.877004in}}{\pgfqpoint{4.135676in}{0.877004in}}%
\pgfpathcurveto{\pgfqpoint{4.124626in}{0.877004in}}{\pgfqpoint{4.114027in}{0.872614in}}{\pgfqpoint{4.106213in}{0.864800in}}%
\pgfpathcurveto{\pgfqpoint{4.098400in}{0.856987in}}{\pgfqpoint{4.094009in}{0.846388in}}{\pgfqpoint{4.094009in}{0.835337in}}%
\pgfpathcurveto{\pgfqpoint{4.094009in}{0.824287in}}{\pgfqpoint{4.098400in}{0.813688in}}{\pgfqpoint{4.106213in}{0.805875in}}%
\pgfpathcurveto{\pgfqpoint{4.114027in}{0.798061in}}{\pgfqpoint{4.124626in}{0.793671in}}{\pgfqpoint{4.135676in}{0.793671in}}%
\pgfpathclose%
\pgfusepath{stroke,fill}%
\end{pgfscope}%
\begin{pgfscope}%
\pgfpathrectangle{\pgfqpoint{1.015125in}{0.499691in}}{\pgfqpoint{3.875000in}{2.695000in}}%
\pgfusepath{clip}%
\pgfsetbuttcap%
\pgfsetroundjoin%
\definecolor{currentfill}{rgb}{0.529412,0.807843,0.921569}%
\pgfsetfillcolor{currentfill}%
\pgfsetlinewidth{1.003750pt}%
\definecolor{currentstroke}{rgb}{0.529412,0.807843,0.921569}%
\pgfsetstrokecolor{currentstroke}%
\pgfsetdash{}{0pt}%
\pgfpathmoveto{\pgfqpoint{4.190760in}{0.785849in}}%
\pgfpathcurveto{\pgfqpoint{4.201811in}{0.785849in}}{\pgfqpoint{4.212410in}{0.790239in}}{\pgfqpoint{4.220223in}{0.798053in}}%
\pgfpathcurveto{\pgfqpoint{4.228037in}{0.805866in}}{\pgfqpoint{4.232427in}{0.816465in}}{\pgfqpoint{4.232427in}{0.827515in}}%
\pgfpathcurveto{\pgfqpoint{4.232427in}{0.838566in}}{\pgfqpoint{4.228037in}{0.849165in}}{\pgfqpoint{4.220223in}{0.856978in}}%
\pgfpathcurveto{\pgfqpoint{4.212410in}{0.864792in}}{\pgfqpoint{4.201811in}{0.869182in}}{\pgfqpoint{4.190760in}{0.869182in}}%
\pgfpathcurveto{\pgfqpoint{4.179710in}{0.869182in}}{\pgfqpoint{4.169111in}{0.864792in}}{\pgfqpoint{4.161298in}{0.856978in}}%
\pgfpathcurveto{\pgfqpoint{4.153484in}{0.849165in}}{\pgfqpoint{4.149094in}{0.838566in}}{\pgfqpoint{4.149094in}{0.827515in}}%
\pgfpathcurveto{\pgfqpoint{4.149094in}{0.816465in}}{\pgfqpoint{4.153484in}{0.805866in}}{\pgfqpoint{4.161298in}{0.798053in}}%
\pgfpathcurveto{\pgfqpoint{4.169111in}{0.790239in}}{\pgfqpoint{4.179710in}{0.785849in}}{\pgfqpoint{4.190760in}{0.785849in}}%
\pgfpathclose%
\pgfusepath{stroke,fill}%
\end{pgfscope}%
\begin{pgfscope}%
\pgfpathrectangle{\pgfqpoint{1.015125in}{0.499691in}}{\pgfqpoint{3.875000in}{2.695000in}}%
\pgfusepath{clip}%
\pgfsetbuttcap%
\pgfsetroundjoin%
\definecolor{currentfill}{rgb}{0.529412,0.807843,0.921569}%
\pgfsetfillcolor{currentfill}%
\pgfsetlinewidth{1.003750pt}%
\definecolor{currentstroke}{rgb}{0.529412,0.807843,0.921569}%
\pgfsetstrokecolor{currentstroke}%
\pgfsetdash{}{0pt}%
\pgfpathmoveto{\pgfqpoint{4.243359in}{0.755109in}}%
\pgfpathcurveto{\pgfqpoint{4.254409in}{0.755109in}}{\pgfqpoint{4.265008in}{0.759500in}}{\pgfqpoint{4.272822in}{0.767313in}}%
\pgfpathcurveto{\pgfqpoint{4.280635in}{0.775127in}}{\pgfqpoint{4.285025in}{0.785726in}}{\pgfqpoint{4.285025in}{0.796776in}}%
\pgfpathcurveto{\pgfqpoint{4.285025in}{0.807826in}}{\pgfqpoint{4.280635in}{0.818425in}}{\pgfqpoint{4.272822in}{0.826239in}}%
\pgfpathcurveto{\pgfqpoint{4.265008in}{0.834052in}}{\pgfqpoint{4.254409in}{0.838443in}}{\pgfqpoint{4.243359in}{0.838443in}}%
\pgfpathcurveto{\pgfqpoint{4.232309in}{0.838443in}}{\pgfqpoint{4.221710in}{0.834052in}}{\pgfqpoint{4.213896in}{0.826239in}}%
\pgfpathcurveto{\pgfqpoint{4.206082in}{0.818425in}}{\pgfqpoint{4.201692in}{0.807826in}}{\pgfqpoint{4.201692in}{0.796776in}}%
\pgfpathcurveto{\pgfqpoint{4.201692in}{0.785726in}}{\pgfqpoint{4.206082in}{0.775127in}}{\pgfqpoint{4.213896in}{0.767313in}}%
\pgfpathcurveto{\pgfqpoint{4.221710in}{0.759500in}}{\pgfqpoint{4.232309in}{0.755109in}}{\pgfqpoint{4.243359in}{0.755109in}}%
\pgfpathclose%
\pgfusepath{stroke,fill}%
\end{pgfscope}%
\begin{pgfscope}%
\pgfpathrectangle{\pgfqpoint{1.015125in}{0.499691in}}{\pgfqpoint{3.875000in}{2.695000in}}%
\pgfusepath{clip}%
\pgfsetbuttcap%
\pgfsetroundjoin%
\definecolor{currentfill}{rgb}{0.529412,0.807843,0.921569}%
\pgfsetfillcolor{currentfill}%
\pgfsetlinewidth{1.003750pt}%
\definecolor{currentstroke}{rgb}{0.529412,0.807843,0.921569}%
\pgfsetstrokecolor{currentstroke}%
\pgfsetdash{}{0pt}%
\pgfpathmoveto{\pgfqpoint{4.293686in}{0.740058in}}%
\pgfpathcurveto{\pgfqpoint{4.304736in}{0.740058in}}{\pgfqpoint{4.315335in}{0.744449in}}{\pgfqpoint{4.323149in}{0.752262in}}%
\pgfpathcurveto{\pgfqpoint{4.330962in}{0.760076in}}{\pgfqpoint{4.335353in}{0.770675in}}{\pgfqpoint{4.335353in}{0.781725in}}%
\pgfpathcurveto{\pgfqpoint{4.335353in}{0.792775in}}{\pgfqpoint{4.330962in}{0.803374in}}{\pgfqpoint{4.323149in}{0.811188in}}%
\pgfpathcurveto{\pgfqpoint{4.315335in}{0.819001in}}{\pgfqpoint{4.304736in}{0.823392in}}{\pgfqpoint{4.293686in}{0.823392in}}%
\pgfpathcurveto{\pgfqpoint{4.282636in}{0.823392in}}{\pgfqpoint{4.272037in}{0.819001in}}{\pgfqpoint{4.264223in}{0.811188in}}%
\pgfpathcurveto{\pgfqpoint{4.256410in}{0.803374in}}{\pgfqpoint{4.252019in}{0.792775in}}{\pgfqpoint{4.252019in}{0.781725in}}%
\pgfpathcurveto{\pgfqpoint{4.252019in}{0.770675in}}{\pgfqpoint{4.256410in}{0.760076in}}{\pgfqpoint{4.264223in}{0.752262in}}%
\pgfpathcurveto{\pgfqpoint{4.272037in}{0.744449in}}{\pgfqpoint{4.282636in}{0.740058in}}{\pgfqpoint{4.293686in}{0.740058in}}%
\pgfpathclose%
\pgfusepath{stroke,fill}%
\end{pgfscope}%
\begin{pgfscope}%
\pgfpathrectangle{\pgfqpoint{1.015125in}{0.499691in}}{\pgfqpoint{3.875000in}{2.695000in}}%
\pgfusepath{clip}%
\pgfsetbuttcap%
\pgfsetroundjoin%
\definecolor{currentfill}{rgb}{0.529412,0.807843,0.921569}%
\pgfsetfillcolor{currentfill}%
\pgfsetlinewidth{1.003750pt}%
\definecolor{currentstroke}{rgb}{0.529412,0.807843,0.921569}%
\pgfsetstrokecolor{currentstroke}%
\pgfsetdash{}{0pt}%
\pgfpathmoveto{\pgfqpoint{4.341930in}{0.725212in}}%
\pgfpathcurveto{\pgfqpoint{4.352980in}{0.725212in}}{\pgfqpoint{4.363579in}{0.729603in}}{\pgfqpoint{4.371393in}{0.737416in}}%
\pgfpathcurveto{\pgfqpoint{4.379207in}{0.745230in}}{\pgfqpoint{4.383597in}{0.755829in}}{\pgfqpoint{4.383597in}{0.766879in}}%
\pgfpathcurveto{\pgfqpoint{4.383597in}{0.777929in}}{\pgfqpoint{4.379207in}{0.788528in}}{\pgfqpoint{4.371393in}{0.796342in}}%
\pgfpathcurveto{\pgfqpoint{4.363579in}{0.804155in}}{\pgfqpoint{4.352980in}{0.808546in}}{\pgfqpoint{4.341930in}{0.808546in}}%
\pgfpathcurveto{\pgfqpoint{4.330880in}{0.808546in}}{\pgfqpoint{4.320281in}{0.804155in}}{\pgfqpoint{4.312467in}{0.796342in}}%
\pgfpathcurveto{\pgfqpoint{4.304654in}{0.788528in}}{\pgfqpoint{4.300264in}{0.777929in}}{\pgfqpoint{4.300264in}{0.766879in}}%
\pgfpathcurveto{\pgfqpoint{4.300264in}{0.755829in}}{\pgfqpoint{4.304654in}{0.745230in}}{\pgfqpoint{4.312467in}{0.737416in}}%
\pgfpathcurveto{\pgfqpoint{4.320281in}{0.729603in}}{\pgfqpoint{4.330880in}{0.725212in}}{\pgfqpoint{4.341930in}{0.725212in}}%
\pgfpathclose%
\pgfusepath{stroke,fill}%
\end{pgfscope}%
\begin{pgfscope}%
\pgfpathrectangle{\pgfqpoint{1.015125in}{0.499691in}}{\pgfqpoint{3.875000in}{2.695000in}}%
\pgfusepath{clip}%
\pgfsetbuttcap%
\pgfsetroundjoin%
\definecolor{currentfill}{rgb}{0.529412,0.807843,0.921569}%
\pgfsetfillcolor{currentfill}%
\pgfsetlinewidth{1.003750pt}%
\definecolor{currentstroke}{rgb}{0.529412,0.807843,0.921569}%
\pgfsetstrokecolor{currentstroke}%
\pgfsetdash{}{0pt}%
\pgfpathmoveto{\pgfqpoint{4.388257in}{0.696113in}}%
\pgfpathcurveto{\pgfqpoint{4.399307in}{0.696113in}}{\pgfqpoint{4.409906in}{0.700503in}}{\pgfqpoint{4.417720in}{0.708316in}}%
\pgfpathcurveto{\pgfqpoint{4.425533in}{0.716130in}}{\pgfqpoint{4.429924in}{0.726729in}}{\pgfqpoint{4.429924in}{0.737779in}}%
\pgfpathcurveto{\pgfqpoint{4.429924in}{0.748829in}}{\pgfqpoint{4.425533in}{0.759428in}}{\pgfqpoint{4.417720in}{0.767242in}}%
\pgfpathcurveto{\pgfqpoint{4.409906in}{0.775056in}}{\pgfqpoint{4.399307in}{0.779446in}}{\pgfqpoint{4.388257in}{0.779446in}}%
\pgfpathcurveto{\pgfqpoint{4.377207in}{0.779446in}}{\pgfqpoint{4.366608in}{0.775056in}}{\pgfqpoint{4.358794in}{0.767242in}}%
\pgfpathcurveto{\pgfqpoint{4.350980in}{0.759428in}}{\pgfqpoint{4.346590in}{0.748829in}}{\pgfqpoint{4.346590in}{0.737779in}}%
\pgfpathcurveto{\pgfqpoint{4.346590in}{0.726729in}}{\pgfqpoint{4.350980in}{0.716130in}}{\pgfqpoint{4.358794in}{0.708316in}}%
\pgfpathcurveto{\pgfqpoint{4.366608in}{0.700503in}}{\pgfqpoint{4.377207in}{0.696113in}}{\pgfqpoint{4.388257in}{0.696113in}}%
\pgfpathclose%
\pgfusepath{stroke,fill}%
\end{pgfscope}%
\begin{pgfscope}%
\pgfpathrectangle{\pgfqpoint{1.015125in}{0.499691in}}{\pgfqpoint{3.875000in}{2.695000in}}%
\pgfusepath{clip}%
\pgfsetbuttcap%
\pgfsetroundjoin%
\definecolor{currentfill}{rgb}{0.529412,0.807843,0.921569}%
\pgfsetfillcolor{currentfill}%
\pgfsetlinewidth{1.003750pt}%
\definecolor{currentstroke}{rgb}{0.529412,0.807843,0.921569}%
\pgfsetstrokecolor{currentstroke}%
\pgfsetdash{}{0pt}%
\pgfpathmoveto{\pgfqpoint{4.432813in}{0.667769in}}%
\pgfpathcurveto{\pgfqpoint{4.443863in}{0.667769in}}{\pgfqpoint{4.454462in}{0.672159in}}{\pgfqpoint{4.462275in}{0.679973in}}%
\pgfpathcurveto{\pgfqpoint{4.470089in}{0.687786in}}{\pgfqpoint{4.474479in}{0.698385in}}{\pgfqpoint{4.474479in}{0.709435in}}%
\pgfpathcurveto{\pgfqpoint{4.474479in}{0.720486in}}{\pgfqpoint{4.470089in}{0.731085in}}{\pgfqpoint{4.462275in}{0.738898in}}%
\pgfpathcurveto{\pgfqpoint{4.454462in}{0.746712in}}{\pgfqpoint{4.443863in}{0.751102in}}{\pgfqpoint{4.432813in}{0.751102in}}%
\pgfpathcurveto{\pgfqpoint{4.421763in}{0.751102in}}{\pgfqpoint{4.411163in}{0.746712in}}{\pgfqpoint{4.403350in}{0.738898in}}%
\pgfpathcurveto{\pgfqpoint{4.395536in}{0.731085in}}{\pgfqpoint{4.391146in}{0.720486in}}{\pgfqpoint{4.391146in}{0.709435in}}%
\pgfpathcurveto{\pgfqpoint{4.391146in}{0.698385in}}{\pgfqpoint{4.395536in}{0.687786in}}{\pgfqpoint{4.403350in}{0.679973in}}%
\pgfpathcurveto{\pgfqpoint{4.411163in}{0.672159in}}{\pgfqpoint{4.421763in}{0.667769in}}{\pgfqpoint{4.432813in}{0.667769in}}%
\pgfpathclose%
\pgfusepath{stroke,fill}%
\end{pgfscope}%
\begin{pgfscope}%
\pgfpathrectangle{\pgfqpoint{1.015125in}{0.499691in}}{\pgfqpoint{3.875000in}{2.695000in}}%
\pgfusepath{clip}%
\pgfsetbuttcap%
\pgfsetroundjoin%
\definecolor{currentfill}{rgb}{0.529412,0.807843,0.921569}%
\pgfsetfillcolor{currentfill}%
\pgfsetlinewidth{1.003750pt}%
\definecolor{currentstroke}{rgb}{0.529412,0.807843,0.921569}%
\pgfsetstrokecolor{currentstroke}%
\pgfsetdash{}{0pt}%
\pgfpathmoveto{\pgfqpoint{4.475728in}{0.667769in}}%
\pgfpathcurveto{\pgfqpoint{4.486778in}{0.667769in}}{\pgfqpoint{4.497377in}{0.672159in}}{\pgfqpoint{4.505191in}{0.679973in}}%
\pgfpathcurveto{\pgfqpoint{4.513004in}{0.687786in}}{\pgfqpoint{4.517395in}{0.698385in}}{\pgfqpoint{4.517395in}{0.709435in}}%
\pgfpathcurveto{\pgfqpoint{4.517395in}{0.720486in}}{\pgfqpoint{4.513004in}{0.731085in}}{\pgfqpoint{4.505191in}{0.738898in}}%
\pgfpathcurveto{\pgfqpoint{4.497377in}{0.746712in}}{\pgfqpoint{4.486778in}{0.751102in}}{\pgfqpoint{4.475728in}{0.751102in}}%
\pgfpathcurveto{\pgfqpoint{4.464678in}{0.751102in}}{\pgfqpoint{4.454079in}{0.746712in}}{\pgfqpoint{4.446265in}{0.738898in}}%
\pgfpathcurveto{\pgfqpoint{4.438452in}{0.731085in}}{\pgfqpoint{4.434061in}{0.720486in}}{\pgfqpoint{4.434061in}{0.709435in}}%
\pgfpathcurveto{\pgfqpoint{4.434061in}{0.698385in}}{\pgfqpoint{4.438452in}{0.687786in}}{\pgfqpoint{4.446265in}{0.679973in}}%
\pgfpathcurveto{\pgfqpoint{4.454079in}{0.672159in}}{\pgfqpoint{4.464678in}{0.667769in}}{\pgfqpoint{4.475728in}{0.667769in}}%
\pgfpathclose%
\pgfusepath{stroke,fill}%
\end{pgfscope}%
\begin{pgfscope}%
\pgfpathrectangle{\pgfqpoint{1.015125in}{0.499691in}}{\pgfqpoint{3.875000in}{2.695000in}}%
\pgfusepath{clip}%
\pgfsetbuttcap%
\pgfsetroundjoin%
\definecolor{currentfill}{rgb}{0.529412,0.807843,0.921569}%
\pgfsetfillcolor{currentfill}%
\pgfsetlinewidth{1.003750pt}%
\definecolor{currentstroke}{rgb}{0.529412,0.807843,0.921569}%
\pgfsetstrokecolor{currentstroke}%
\pgfsetdash{}{0pt}%
\pgfpathmoveto{\pgfqpoint{4.517119in}{0.640143in}}%
\pgfpathcurveto{\pgfqpoint{4.528169in}{0.640143in}}{\pgfqpoint{4.538768in}{0.644533in}}{\pgfqpoint{4.546582in}{0.652346in}}%
\pgfpathcurveto{\pgfqpoint{4.554396in}{0.660160in}}{\pgfqpoint{4.558786in}{0.670759in}}{\pgfqpoint{4.558786in}{0.681809in}}%
\pgfpathcurveto{\pgfqpoint{4.558786in}{0.692859in}}{\pgfqpoint{4.554396in}{0.703458in}}{\pgfqpoint{4.546582in}{0.711272in}}%
\pgfpathcurveto{\pgfqpoint{4.538768in}{0.719086in}}{\pgfqpoint{4.528169in}{0.723476in}}{\pgfqpoint{4.517119in}{0.723476in}}%
\pgfpathcurveto{\pgfqpoint{4.506069in}{0.723476in}}{\pgfqpoint{4.495470in}{0.719086in}}{\pgfqpoint{4.487657in}{0.711272in}}%
\pgfpathcurveto{\pgfqpoint{4.479843in}{0.703458in}}{\pgfqpoint{4.475453in}{0.692859in}}{\pgfqpoint{4.475453in}{0.681809in}}%
\pgfpathcurveto{\pgfqpoint{4.475453in}{0.670759in}}{\pgfqpoint{4.479843in}{0.660160in}}{\pgfqpoint{4.487657in}{0.652346in}}%
\pgfpathcurveto{\pgfqpoint{4.495470in}{0.644533in}}{\pgfqpoint{4.506069in}{0.640143in}}{\pgfqpoint{4.517119in}{0.640143in}}%
\pgfpathclose%
\pgfusepath{stroke,fill}%
\end{pgfscope}%
\begin{pgfscope}%
\pgfpathrectangle{\pgfqpoint{1.015125in}{0.499691in}}{\pgfqpoint{3.875000in}{2.695000in}}%
\pgfusepath{clip}%
\pgfsetbuttcap%
\pgfsetroundjoin%
\definecolor{currentfill}{rgb}{0.529412,0.807843,0.921569}%
\pgfsetfillcolor{currentfill}%
\pgfsetlinewidth{1.003750pt}%
\definecolor{currentstroke}{rgb}{0.529412,0.807843,0.921569}%
\pgfsetstrokecolor{currentstroke}%
\pgfsetdash{}{0pt}%
\pgfpathmoveto{\pgfqpoint{4.557091in}{0.640143in}}%
\pgfpathcurveto{\pgfqpoint{4.568141in}{0.640143in}}{\pgfqpoint{4.578740in}{0.644533in}}{\pgfqpoint{4.586554in}{0.652346in}}%
\pgfpathcurveto{\pgfqpoint{4.594368in}{0.660160in}}{\pgfqpoint{4.598758in}{0.670759in}}{\pgfqpoint{4.598758in}{0.681809in}}%
\pgfpathcurveto{\pgfqpoint{4.598758in}{0.692859in}}{\pgfqpoint{4.594368in}{0.703458in}}{\pgfqpoint{4.586554in}{0.711272in}}%
\pgfpathcurveto{\pgfqpoint{4.578740in}{0.719086in}}{\pgfqpoint{4.568141in}{0.723476in}}{\pgfqpoint{4.557091in}{0.723476in}}%
\pgfpathcurveto{\pgfqpoint{4.546041in}{0.723476in}}{\pgfqpoint{4.535442in}{0.719086in}}{\pgfqpoint{4.527628in}{0.711272in}}%
\pgfpathcurveto{\pgfqpoint{4.519815in}{0.703458in}}{\pgfqpoint{4.515425in}{0.692859in}}{\pgfqpoint{4.515425in}{0.681809in}}%
\pgfpathcurveto{\pgfqpoint{4.515425in}{0.670759in}}{\pgfqpoint{4.519815in}{0.660160in}}{\pgfqpoint{4.527628in}{0.652346in}}%
\pgfpathcurveto{\pgfqpoint{4.535442in}{0.644533in}}{\pgfqpoint{4.546041in}{0.640143in}}{\pgfqpoint{4.557091in}{0.640143in}}%
\pgfpathclose%
\pgfusepath{stroke,fill}%
\end{pgfscope}%
\begin{pgfscope}%
\pgfpathrectangle{\pgfqpoint{1.015125in}{0.499691in}}{\pgfqpoint{3.875000in}{2.695000in}}%
\pgfusepath{clip}%
\pgfsetbuttcap%
\pgfsetroundjoin%
\definecolor{currentfill}{rgb}{0.529412,0.807843,0.921569}%
\pgfsetfillcolor{currentfill}%
\pgfsetlinewidth{1.003750pt}%
\definecolor{currentstroke}{rgb}{0.529412,0.807843,0.921569}%
\pgfsetstrokecolor{currentstroke}%
\pgfsetdash{}{0pt}%
\pgfpathmoveto{\pgfqpoint{4.595738in}{0.613199in}}%
\pgfpathcurveto{\pgfqpoint{4.606788in}{0.613199in}}{\pgfqpoint{4.617387in}{0.617589in}}{\pgfqpoint{4.625201in}{0.625402in}}%
\pgfpathcurveto{\pgfqpoint{4.633014in}{0.633216in}}{\pgfqpoint{4.637405in}{0.643815in}}{\pgfqpoint{4.637405in}{0.654865in}}%
\pgfpathcurveto{\pgfqpoint{4.637405in}{0.665915in}}{\pgfqpoint{4.633014in}{0.676514in}}{\pgfqpoint{4.625201in}{0.684328in}}%
\pgfpathcurveto{\pgfqpoint{4.617387in}{0.692142in}}{\pgfqpoint{4.606788in}{0.696532in}}{\pgfqpoint{4.595738in}{0.696532in}}%
\pgfpathcurveto{\pgfqpoint{4.584688in}{0.696532in}}{\pgfqpoint{4.574089in}{0.692142in}}{\pgfqpoint{4.566275in}{0.684328in}}%
\pgfpathcurveto{\pgfqpoint{4.558461in}{0.676514in}}{\pgfqpoint{4.554071in}{0.665915in}}{\pgfqpoint{4.554071in}{0.654865in}}%
\pgfpathcurveto{\pgfqpoint{4.554071in}{0.643815in}}{\pgfqpoint{4.558461in}{0.633216in}}{\pgfqpoint{4.566275in}{0.625402in}}%
\pgfpathcurveto{\pgfqpoint{4.574089in}{0.617589in}}{\pgfqpoint{4.584688in}{0.613199in}}{\pgfqpoint{4.595738in}{0.613199in}}%
\pgfpathclose%
\pgfusepath{stroke,fill}%
\end{pgfscope}%
\begin{pgfscope}%
\pgfpathrectangle{\pgfqpoint{1.015125in}{0.499691in}}{\pgfqpoint{3.875000in}{2.695000in}}%
\pgfusepath{clip}%
\pgfsetbuttcap%
\pgfsetroundjoin%
\definecolor{currentfill}{rgb}{0.529412,0.807843,0.921569}%
\pgfsetfillcolor{currentfill}%
\pgfsetlinewidth{1.003750pt}%
\definecolor{currentstroke}{rgb}{0.529412,0.807843,0.921569}%
\pgfsetstrokecolor{currentstroke}%
\pgfsetdash{}{0pt}%
\pgfpathmoveto{\pgfqpoint{4.633144in}{0.613199in}}%
\pgfpathcurveto{\pgfqpoint{4.644194in}{0.613199in}}{\pgfqpoint{4.654793in}{0.617589in}}{\pgfqpoint{4.662607in}{0.625402in}}%
\pgfpathcurveto{\pgfqpoint{4.670421in}{0.633216in}}{\pgfqpoint{4.674811in}{0.643815in}}{\pgfqpoint{4.674811in}{0.654865in}}%
\pgfpathcurveto{\pgfqpoint{4.674811in}{0.665915in}}{\pgfqpoint{4.670421in}{0.676514in}}{\pgfqpoint{4.662607in}{0.684328in}}%
\pgfpathcurveto{\pgfqpoint{4.654793in}{0.692142in}}{\pgfqpoint{4.644194in}{0.696532in}}{\pgfqpoint{4.633144in}{0.696532in}}%
\pgfpathcurveto{\pgfqpoint{4.622094in}{0.696532in}}{\pgfqpoint{4.611495in}{0.692142in}}{\pgfqpoint{4.603682in}{0.684328in}}%
\pgfpathcurveto{\pgfqpoint{4.595868in}{0.676514in}}{\pgfqpoint{4.591478in}{0.665915in}}{\pgfqpoint{4.591478in}{0.654865in}}%
\pgfpathcurveto{\pgfqpoint{4.591478in}{0.643815in}}{\pgfqpoint{4.595868in}{0.633216in}}{\pgfqpoint{4.603682in}{0.625402in}}%
\pgfpathcurveto{\pgfqpoint{4.611495in}{0.617589in}}{\pgfqpoint{4.622094in}{0.613199in}}{\pgfqpoint{4.633144in}{0.613199in}}%
\pgfpathclose%
\pgfusepath{stroke,fill}%
\end{pgfscope}%
\begin{pgfscope}%
\pgfpathrectangle{\pgfqpoint{1.015125in}{0.499691in}}{\pgfqpoint{3.875000in}{2.695000in}}%
\pgfusepath{clip}%
\pgfsetbuttcap%
\pgfsetroundjoin%
\definecolor{currentfill}{rgb}{0.529412,0.807843,0.921569}%
\pgfsetfillcolor{currentfill}%
\pgfsetlinewidth{1.003750pt}%
\definecolor{currentstroke}{rgb}{0.529412,0.807843,0.921569}%
\pgfsetstrokecolor{currentstroke}%
\pgfsetdash{}{0pt}%
\pgfpathmoveto{\pgfqpoint{4.669388in}{0.599972in}}%
\pgfpathcurveto{\pgfqpoint{4.680438in}{0.599972in}}{\pgfqpoint{4.691037in}{0.604362in}}{\pgfqpoint{4.698850in}{0.612176in}}%
\pgfpathcurveto{\pgfqpoint{4.706664in}{0.619990in}}{\pgfqpoint{4.711054in}{0.630589in}}{\pgfqpoint{4.711054in}{0.641639in}}%
\pgfpathcurveto{\pgfqpoint{4.711054in}{0.652689in}}{\pgfqpoint{4.706664in}{0.663288in}}{\pgfqpoint{4.698850in}{0.671101in}}%
\pgfpathcurveto{\pgfqpoint{4.691037in}{0.678915in}}{\pgfqpoint{4.680438in}{0.683305in}}{\pgfqpoint{4.669388in}{0.683305in}}%
\pgfpathcurveto{\pgfqpoint{4.658338in}{0.683305in}}{\pgfqpoint{4.647739in}{0.678915in}}{\pgfqpoint{4.639925in}{0.671101in}}%
\pgfpathcurveto{\pgfqpoint{4.632111in}{0.663288in}}{\pgfqpoint{4.627721in}{0.652689in}}{\pgfqpoint{4.627721in}{0.641639in}}%
\pgfpathcurveto{\pgfqpoint{4.627721in}{0.630589in}}{\pgfqpoint{4.632111in}{0.619990in}}{\pgfqpoint{4.639925in}{0.612176in}}%
\pgfpathcurveto{\pgfqpoint{4.647739in}{0.604362in}}{\pgfqpoint{4.658338in}{0.599972in}}{\pgfqpoint{4.669388in}{0.599972in}}%
\pgfpathclose%
\pgfusepath{stroke,fill}%
\end{pgfscope}%
\begin{pgfscope}%
\pgfpathrectangle{\pgfqpoint{1.015125in}{0.499691in}}{\pgfqpoint{3.875000in}{2.695000in}}%
\pgfusepath{clip}%
\pgfsetbuttcap%
\pgfsetroundjoin%
\definecolor{currentfill}{rgb}{0.529412,0.807843,0.921569}%
\pgfsetfillcolor{currentfill}%
\pgfsetlinewidth{1.003750pt}%
\definecolor{currentstroke}{rgb}{0.529412,0.807843,0.921569}%
\pgfsetstrokecolor{currentstroke}%
\pgfsetdash{}{0pt}%
\pgfpathmoveto{\pgfqpoint{4.704538in}{0.586904in}}%
\pgfpathcurveto{\pgfqpoint{4.715588in}{0.586904in}}{\pgfqpoint{4.726187in}{0.591294in}}{\pgfqpoint{4.734001in}{0.599108in}}%
\pgfpathcurveto{\pgfqpoint{4.741815in}{0.606921in}}{\pgfqpoint{4.746205in}{0.617520in}}{\pgfqpoint{4.746205in}{0.628571in}}%
\pgfpathcurveto{\pgfqpoint{4.746205in}{0.639621in}}{\pgfqpoint{4.741815in}{0.650220in}}{\pgfqpoint{4.734001in}{0.658033in}}%
\pgfpathcurveto{\pgfqpoint{4.726187in}{0.665847in}}{\pgfqpoint{4.715588in}{0.670237in}}{\pgfqpoint{4.704538in}{0.670237in}}%
\pgfpathcurveto{\pgfqpoint{4.693488in}{0.670237in}}{\pgfqpoint{4.682889in}{0.665847in}}{\pgfqpoint{4.675075in}{0.658033in}}%
\pgfpathcurveto{\pgfqpoint{4.667262in}{0.650220in}}{\pgfqpoint{4.662871in}{0.639621in}}{\pgfqpoint{4.662871in}{0.628571in}}%
\pgfpathcurveto{\pgfqpoint{4.662871in}{0.617520in}}{\pgfqpoint{4.667262in}{0.606921in}}{\pgfqpoint{4.675075in}{0.599108in}}%
\pgfpathcurveto{\pgfqpoint{4.682889in}{0.591294in}}{\pgfqpoint{4.693488in}{0.586904in}}{\pgfqpoint{4.704538in}{0.586904in}}%
\pgfpathclose%
\pgfusepath{stroke,fill}%
\end{pgfscope}%
\begin{pgfscope}%
\pgfsetbuttcap%
\pgfsetroundjoin%
\definecolor{currentfill}{rgb}{0.000000,0.000000,0.000000}%
\pgfsetfillcolor{currentfill}%
\pgfsetlinewidth{0.803000pt}%
\definecolor{currentstroke}{rgb}{0.000000,0.000000,0.000000}%
\pgfsetstrokecolor{currentstroke}%
\pgfsetdash{}{0pt}%
\pgfsys@defobject{currentmarker}{\pgfqpoint{0.000000in}{-0.048611in}}{\pgfqpoint{0.000000in}{0.000000in}}{%
\pgfpathmoveto{\pgfqpoint{0.000000in}{0.000000in}}%
\pgfpathlineto{\pgfqpoint{0.000000in}{-0.048611in}}%
\pgfusepath{stroke,fill}%
}%
\begin{pgfscope}%
\pgfsys@transformshift{1.200712in}{0.499691in}%
\pgfsys@useobject{currentmarker}{}%
\end{pgfscope}%
\end{pgfscope}%
\begin{pgfscope}%
\definecolor{textcolor}{rgb}{0.000000,0.000000,0.000000}%
\pgfsetstrokecolor{textcolor}%
\pgfsetfillcolor{textcolor}%
\pgftext[x=1.200712in,y=0.402469in,,top]{\color{textcolor}\rmfamily\fontsize{10.000000}{12.000000}\selectfont \(\displaystyle 2.0\)}%
\end{pgfscope}%
\begin{pgfscope}%
\pgfsetbuttcap%
\pgfsetroundjoin%
\definecolor{currentfill}{rgb}{0.000000,0.000000,0.000000}%
\pgfsetfillcolor{currentfill}%
\pgfsetlinewidth{0.803000pt}%
\definecolor{currentstroke}{rgb}{0.000000,0.000000,0.000000}%
\pgfsetstrokecolor{currentstroke}%
\pgfsetdash{}{0pt}%
\pgfsys@defobject{currentmarker}{\pgfqpoint{0.000000in}{-0.048611in}}{\pgfqpoint{0.000000in}{0.000000in}}{%
\pgfpathmoveto{\pgfqpoint{0.000000in}{0.000000in}}%
\pgfpathlineto{\pgfqpoint{0.000000in}{-0.048611in}}%
\pgfusepath{stroke,fill}%
}%
\begin{pgfscope}%
\pgfsys@transformshift{1.737552in}{0.499691in}%
\pgfsys@useobject{currentmarker}{}%
\end{pgfscope}%
\end{pgfscope}%
\begin{pgfscope}%
\definecolor{textcolor}{rgb}{0.000000,0.000000,0.000000}%
\pgfsetstrokecolor{textcolor}%
\pgfsetfillcolor{textcolor}%
\pgftext[x=1.737552in,y=0.402469in,,top]{\color{textcolor}\rmfamily\fontsize{10.000000}{12.000000}\selectfont \(\displaystyle 2.2\)}%
\end{pgfscope}%
\begin{pgfscope}%
\pgfsetbuttcap%
\pgfsetroundjoin%
\definecolor{currentfill}{rgb}{0.000000,0.000000,0.000000}%
\pgfsetfillcolor{currentfill}%
\pgfsetlinewidth{0.803000pt}%
\definecolor{currentstroke}{rgb}{0.000000,0.000000,0.000000}%
\pgfsetstrokecolor{currentstroke}%
\pgfsetdash{}{0pt}%
\pgfsys@defobject{currentmarker}{\pgfqpoint{0.000000in}{-0.048611in}}{\pgfqpoint{0.000000in}{0.000000in}}{%
\pgfpathmoveto{\pgfqpoint{0.000000in}{0.000000in}}%
\pgfpathlineto{\pgfqpoint{0.000000in}{-0.048611in}}%
\pgfusepath{stroke,fill}%
}%
\begin{pgfscope}%
\pgfsys@transformshift{2.274392in}{0.499691in}%
\pgfsys@useobject{currentmarker}{}%
\end{pgfscope}%
\end{pgfscope}%
\begin{pgfscope}%
\definecolor{textcolor}{rgb}{0.000000,0.000000,0.000000}%
\pgfsetstrokecolor{textcolor}%
\pgfsetfillcolor{textcolor}%
\pgftext[x=2.274392in,y=0.402469in,,top]{\color{textcolor}\rmfamily\fontsize{10.000000}{12.000000}\selectfont \(\displaystyle 2.4\)}%
\end{pgfscope}%
\begin{pgfscope}%
\pgfsetbuttcap%
\pgfsetroundjoin%
\definecolor{currentfill}{rgb}{0.000000,0.000000,0.000000}%
\pgfsetfillcolor{currentfill}%
\pgfsetlinewidth{0.803000pt}%
\definecolor{currentstroke}{rgb}{0.000000,0.000000,0.000000}%
\pgfsetstrokecolor{currentstroke}%
\pgfsetdash{}{0pt}%
\pgfsys@defobject{currentmarker}{\pgfqpoint{0.000000in}{-0.048611in}}{\pgfqpoint{0.000000in}{0.000000in}}{%
\pgfpathmoveto{\pgfqpoint{0.000000in}{0.000000in}}%
\pgfpathlineto{\pgfqpoint{0.000000in}{-0.048611in}}%
\pgfusepath{stroke,fill}%
}%
\begin{pgfscope}%
\pgfsys@transformshift{2.811233in}{0.499691in}%
\pgfsys@useobject{currentmarker}{}%
\end{pgfscope}%
\end{pgfscope}%
\begin{pgfscope}%
\definecolor{textcolor}{rgb}{0.000000,0.000000,0.000000}%
\pgfsetstrokecolor{textcolor}%
\pgfsetfillcolor{textcolor}%
\pgftext[x=2.811233in,y=0.402469in,,top]{\color{textcolor}\rmfamily\fontsize{10.000000}{12.000000}\selectfont \(\displaystyle 2.6\)}%
\end{pgfscope}%
\begin{pgfscope}%
\pgfsetbuttcap%
\pgfsetroundjoin%
\definecolor{currentfill}{rgb}{0.000000,0.000000,0.000000}%
\pgfsetfillcolor{currentfill}%
\pgfsetlinewidth{0.803000pt}%
\definecolor{currentstroke}{rgb}{0.000000,0.000000,0.000000}%
\pgfsetstrokecolor{currentstroke}%
\pgfsetdash{}{0pt}%
\pgfsys@defobject{currentmarker}{\pgfqpoint{0.000000in}{-0.048611in}}{\pgfqpoint{0.000000in}{0.000000in}}{%
\pgfpathmoveto{\pgfqpoint{0.000000in}{0.000000in}}%
\pgfpathlineto{\pgfqpoint{0.000000in}{-0.048611in}}%
\pgfusepath{stroke,fill}%
}%
\begin{pgfscope}%
\pgfsys@transformshift{3.348073in}{0.499691in}%
\pgfsys@useobject{currentmarker}{}%
\end{pgfscope}%
\end{pgfscope}%
\begin{pgfscope}%
\definecolor{textcolor}{rgb}{0.000000,0.000000,0.000000}%
\pgfsetstrokecolor{textcolor}%
\pgfsetfillcolor{textcolor}%
\pgftext[x=3.348073in,y=0.402469in,,top]{\color{textcolor}\rmfamily\fontsize{10.000000}{12.000000}\selectfont \(\displaystyle 2.8\)}%
\end{pgfscope}%
\begin{pgfscope}%
\pgfsetbuttcap%
\pgfsetroundjoin%
\definecolor{currentfill}{rgb}{0.000000,0.000000,0.000000}%
\pgfsetfillcolor{currentfill}%
\pgfsetlinewidth{0.803000pt}%
\definecolor{currentstroke}{rgb}{0.000000,0.000000,0.000000}%
\pgfsetstrokecolor{currentstroke}%
\pgfsetdash{}{0pt}%
\pgfsys@defobject{currentmarker}{\pgfqpoint{0.000000in}{-0.048611in}}{\pgfqpoint{0.000000in}{0.000000in}}{%
\pgfpathmoveto{\pgfqpoint{0.000000in}{0.000000in}}%
\pgfpathlineto{\pgfqpoint{0.000000in}{-0.048611in}}%
\pgfusepath{stroke,fill}%
}%
\begin{pgfscope}%
\pgfsys@transformshift{3.884913in}{0.499691in}%
\pgfsys@useobject{currentmarker}{}%
\end{pgfscope}%
\end{pgfscope}%
\begin{pgfscope}%
\definecolor{textcolor}{rgb}{0.000000,0.000000,0.000000}%
\pgfsetstrokecolor{textcolor}%
\pgfsetfillcolor{textcolor}%
\pgftext[x=3.884913in,y=0.402469in,,top]{\color{textcolor}\rmfamily\fontsize{10.000000}{12.000000}\selectfont \(\displaystyle 3.0\)}%
\end{pgfscope}%
\begin{pgfscope}%
\pgfsetbuttcap%
\pgfsetroundjoin%
\definecolor{currentfill}{rgb}{0.000000,0.000000,0.000000}%
\pgfsetfillcolor{currentfill}%
\pgfsetlinewidth{0.803000pt}%
\definecolor{currentstroke}{rgb}{0.000000,0.000000,0.000000}%
\pgfsetstrokecolor{currentstroke}%
\pgfsetdash{}{0pt}%
\pgfsys@defobject{currentmarker}{\pgfqpoint{0.000000in}{-0.048611in}}{\pgfqpoint{0.000000in}{0.000000in}}{%
\pgfpathmoveto{\pgfqpoint{0.000000in}{0.000000in}}%
\pgfpathlineto{\pgfqpoint{0.000000in}{-0.048611in}}%
\pgfusepath{stroke,fill}%
}%
\begin{pgfscope}%
\pgfsys@transformshift{4.421754in}{0.499691in}%
\pgfsys@useobject{currentmarker}{}%
\end{pgfscope}%
\end{pgfscope}%
\begin{pgfscope}%
\definecolor{textcolor}{rgb}{0.000000,0.000000,0.000000}%
\pgfsetstrokecolor{textcolor}%
\pgfsetfillcolor{textcolor}%
\pgftext[x=4.421754in,y=0.402469in,,top]{\color{textcolor}\rmfamily\fontsize{10.000000}{12.000000}\selectfont \(\displaystyle 3.2\)}%
\end{pgfscope}%
\begin{pgfscope}%
\definecolor{textcolor}{rgb}{0.000000,0.000000,0.000000}%
\pgfsetstrokecolor{textcolor}%
\pgfsetfillcolor{textcolor}%
\pgftext[x=2.952625in,y=0.223457in,,top]{\color{textcolor}\rmfamily\fontsize{10.000000}{12.000000}\selectfont log f}%
\end{pgfscope}%
\begin{pgfscope}%
\pgfsetbuttcap%
\pgfsetroundjoin%
\definecolor{currentfill}{rgb}{0.000000,0.000000,0.000000}%
\pgfsetfillcolor{currentfill}%
\pgfsetlinewidth{0.803000pt}%
\definecolor{currentstroke}{rgb}{0.000000,0.000000,0.000000}%
\pgfsetstrokecolor{currentstroke}%
\pgfsetdash{}{0pt}%
\pgfsys@defobject{currentmarker}{\pgfqpoint{-0.048611in}{0.000000in}}{\pgfqpoint{0.000000in}{0.000000in}}{%
\pgfpathmoveto{\pgfqpoint{0.000000in}{0.000000in}}%
\pgfpathlineto{\pgfqpoint{-0.048611in}{0.000000in}}%
\pgfusepath{stroke,fill}%
}%
\begin{pgfscope}%
\pgfsys@transformshift{1.015125in}{0.725442in}%
\pgfsys@useobject{currentmarker}{}%
\end{pgfscope}%
\end{pgfscope}%
\begin{pgfscope}%
\definecolor{textcolor}{rgb}{0.000000,0.000000,0.000000}%
\pgfsetstrokecolor{textcolor}%
\pgfsetfillcolor{textcolor}%
\pgftext[x=0.670988in,y=0.677217in,left,base]{\color{textcolor}\rmfamily\fontsize{10.000000}{12.000000}\selectfont \(\displaystyle 82.5\)}%
\end{pgfscope}%
\begin{pgfscope}%
\pgfsetbuttcap%
\pgfsetroundjoin%
\definecolor{currentfill}{rgb}{0.000000,0.000000,0.000000}%
\pgfsetfillcolor{currentfill}%
\pgfsetlinewidth{0.803000pt}%
\definecolor{currentstroke}{rgb}{0.000000,0.000000,0.000000}%
\pgfsetstrokecolor{currentstroke}%
\pgfsetdash{}{0pt}%
\pgfsys@defobject{currentmarker}{\pgfqpoint{-0.048611in}{0.000000in}}{\pgfqpoint{0.000000in}{0.000000in}}{%
\pgfpathmoveto{\pgfqpoint{0.000000in}{0.000000in}}%
\pgfpathlineto{\pgfqpoint{-0.048611in}{0.000000in}}%
\pgfusepath{stroke,fill}%
}%
\begin{pgfscope}%
\pgfsys@transformshift{1.015125in}{1.039508in}%
\pgfsys@useobject{currentmarker}{}%
\end{pgfscope}%
\end{pgfscope}%
\begin{pgfscope}%
\definecolor{textcolor}{rgb}{0.000000,0.000000,0.000000}%
\pgfsetstrokecolor{textcolor}%
\pgfsetfillcolor{textcolor}%
\pgftext[x=0.670988in,y=0.991282in,left,base]{\color{textcolor}\rmfamily\fontsize{10.000000}{12.000000}\selectfont \(\displaystyle 85.0\)}%
\end{pgfscope}%
\begin{pgfscope}%
\pgfsetbuttcap%
\pgfsetroundjoin%
\definecolor{currentfill}{rgb}{0.000000,0.000000,0.000000}%
\pgfsetfillcolor{currentfill}%
\pgfsetlinewidth{0.803000pt}%
\definecolor{currentstroke}{rgb}{0.000000,0.000000,0.000000}%
\pgfsetstrokecolor{currentstroke}%
\pgfsetdash{}{0pt}%
\pgfsys@defobject{currentmarker}{\pgfqpoint{-0.048611in}{0.000000in}}{\pgfqpoint{0.000000in}{0.000000in}}{%
\pgfpathmoveto{\pgfqpoint{0.000000in}{0.000000in}}%
\pgfpathlineto{\pgfqpoint{-0.048611in}{0.000000in}}%
\pgfusepath{stroke,fill}%
}%
\begin{pgfscope}%
\pgfsys@transformshift{1.015125in}{1.353573in}%
\pgfsys@useobject{currentmarker}{}%
\end{pgfscope}%
\end{pgfscope}%
\begin{pgfscope}%
\definecolor{textcolor}{rgb}{0.000000,0.000000,0.000000}%
\pgfsetstrokecolor{textcolor}%
\pgfsetfillcolor{textcolor}%
\pgftext[x=0.670988in,y=1.305348in,left,base]{\color{textcolor}\rmfamily\fontsize{10.000000}{12.000000}\selectfont \(\displaystyle 87.5\)}%
\end{pgfscope}%
\begin{pgfscope}%
\pgfsetbuttcap%
\pgfsetroundjoin%
\definecolor{currentfill}{rgb}{0.000000,0.000000,0.000000}%
\pgfsetfillcolor{currentfill}%
\pgfsetlinewidth{0.803000pt}%
\definecolor{currentstroke}{rgb}{0.000000,0.000000,0.000000}%
\pgfsetstrokecolor{currentstroke}%
\pgfsetdash{}{0pt}%
\pgfsys@defobject{currentmarker}{\pgfqpoint{-0.048611in}{0.000000in}}{\pgfqpoint{0.000000in}{0.000000in}}{%
\pgfpathmoveto{\pgfqpoint{0.000000in}{0.000000in}}%
\pgfpathlineto{\pgfqpoint{-0.048611in}{0.000000in}}%
\pgfusepath{stroke,fill}%
}%
\begin{pgfscope}%
\pgfsys@transformshift{1.015125in}{1.667639in}%
\pgfsys@useobject{currentmarker}{}%
\end{pgfscope}%
\end{pgfscope}%
\begin{pgfscope}%
\definecolor{textcolor}{rgb}{0.000000,0.000000,0.000000}%
\pgfsetstrokecolor{textcolor}%
\pgfsetfillcolor{textcolor}%
\pgftext[x=0.670988in,y=1.619413in,left,base]{\color{textcolor}\rmfamily\fontsize{10.000000}{12.000000}\selectfont \(\displaystyle 90.0\)}%
\end{pgfscope}%
\begin{pgfscope}%
\pgfsetbuttcap%
\pgfsetroundjoin%
\definecolor{currentfill}{rgb}{0.000000,0.000000,0.000000}%
\pgfsetfillcolor{currentfill}%
\pgfsetlinewidth{0.803000pt}%
\definecolor{currentstroke}{rgb}{0.000000,0.000000,0.000000}%
\pgfsetstrokecolor{currentstroke}%
\pgfsetdash{}{0pt}%
\pgfsys@defobject{currentmarker}{\pgfqpoint{-0.048611in}{0.000000in}}{\pgfqpoint{0.000000in}{0.000000in}}{%
\pgfpathmoveto{\pgfqpoint{0.000000in}{0.000000in}}%
\pgfpathlineto{\pgfqpoint{-0.048611in}{0.000000in}}%
\pgfusepath{stroke,fill}%
}%
\begin{pgfscope}%
\pgfsys@transformshift{1.015125in}{1.981704in}%
\pgfsys@useobject{currentmarker}{}%
\end{pgfscope}%
\end{pgfscope}%
\begin{pgfscope}%
\definecolor{textcolor}{rgb}{0.000000,0.000000,0.000000}%
\pgfsetstrokecolor{textcolor}%
\pgfsetfillcolor{textcolor}%
\pgftext[x=0.670988in,y=1.933479in,left,base]{\color{textcolor}\rmfamily\fontsize{10.000000}{12.000000}\selectfont \(\displaystyle 92.5\)}%
\end{pgfscope}%
\begin{pgfscope}%
\pgfsetbuttcap%
\pgfsetroundjoin%
\definecolor{currentfill}{rgb}{0.000000,0.000000,0.000000}%
\pgfsetfillcolor{currentfill}%
\pgfsetlinewidth{0.803000pt}%
\definecolor{currentstroke}{rgb}{0.000000,0.000000,0.000000}%
\pgfsetstrokecolor{currentstroke}%
\pgfsetdash{}{0pt}%
\pgfsys@defobject{currentmarker}{\pgfqpoint{-0.048611in}{0.000000in}}{\pgfqpoint{0.000000in}{0.000000in}}{%
\pgfpathmoveto{\pgfqpoint{0.000000in}{0.000000in}}%
\pgfpathlineto{\pgfqpoint{-0.048611in}{0.000000in}}%
\pgfusepath{stroke,fill}%
}%
\begin{pgfscope}%
\pgfsys@transformshift{1.015125in}{2.295770in}%
\pgfsys@useobject{currentmarker}{}%
\end{pgfscope}%
\end{pgfscope}%
\begin{pgfscope}%
\definecolor{textcolor}{rgb}{0.000000,0.000000,0.000000}%
\pgfsetstrokecolor{textcolor}%
\pgfsetfillcolor{textcolor}%
\pgftext[x=0.670988in,y=2.247544in,left,base]{\color{textcolor}\rmfamily\fontsize{10.000000}{12.000000}\selectfont \(\displaystyle 95.0\)}%
\end{pgfscope}%
\begin{pgfscope}%
\pgfsetbuttcap%
\pgfsetroundjoin%
\definecolor{currentfill}{rgb}{0.000000,0.000000,0.000000}%
\pgfsetfillcolor{currentfill}%
\pgfsetlinewidth{0.803000pt}%
\definecolor{currentstroke}{rgb}{0.000000,0.000000,0.000000}%
\pgfsetstrokecolor{currentstroke}%
\pgfsetdash{}{0pt}%
\pgfsys@defobject{currentmarker}{\pgfqpoint{-0.048611in}{0.000000in}}{\pgfqpoint{0.000000in}{0.000000in}}{%
\pgfpathmoveto{\pgfqpoint{0.000000in}{0.000000in}}%
\pgfpathlineto{\pgfqpoint{-0.048611in}{0.000000in}}%
\pgfusepath{stroke,fill}%
}%
\begin{pgfscope}%
\pgfsys@transformshift{1.015125in}{2.609835in}%
\pgfsys@useobject{currentmarker}{}%
\end{pgfscope}%
\end{pgfscope}%
\begin{pgfscope}%
\definecolor{textcolor}{rgb}{0.000000,0.000000,0.000000}%
\pgfsetstrokecolor{textcolor}%
\pgfsetfillcolor{textcolor}%
\pgftext[x=0.670988in,y=2.561610in,left,base]{\color{textcolor}\rmfamily\fontsize{10.000000}{12.000000}\selectfont \(\displaystyle 97.5\)}%
\end{pgfscope}%
\begin{pgfscope}%
\pgfsetbuttcap%
\pgfsetroundjoin%
\definecolor{currentfill}{rgb}{0.000000,0.000000,0.000000}%
\pgfsetfillcolor{currentfill}%
\pgfsetlinewidth{0.803000pt}%
\definecolor{currentstroke}{rgb}{0.000000,0.000000,0.000000}%
\pgfsetstrokecolor{currentstroke}%
\pgfsetdash{}{0pt}%
\pgfsys@defobject{currentmarker}{\pgfqpoint{-0.048611in}{0.000000in}}{\pgfqpoint{0.000000in}{0.000000in}}{%
\pgfpathmoveto{\pgfqpoint{0.000000in}{0.000000in}}%
\pgfpathlineto{\pgfqpoint{-0.048611in}{0.000000in}}%
\pgfusepath{stroke,fill}%
}%
\begin{pgfscope}%
\pgfsys@transformshift{1.015125in}{2.923901in}%
\pgfsys@useobject{currentmarker}{}%
\end{pgfscope}%
\end{pgfscope}%
\begin{pgfscope}%
\definecolor{textcolor}{rgb}{0.000000,0.000000,0.000000}%
\pgfsetstrokecolor{textcolor}%
\pgfsetfillcolor{textcolor}%
\pgftext[x=0.601544in,y=2.875676in,left,base]{\color{textcolor}\rmfamily\fontsize{10.000000}{12.000000}\selectfont \(\displaystyle 100.0\)}%
\end{pgfscope}%
\begin{pgfscope}%
\definecolor{textcolor}{rgb}{0.000000,0.000000,0.000000}%
\pgfsetstrokecolor{textcolor}%
\pgfsetfillcolor{textcolor}%
\pgftext[x=0.323766in,y=1.847191in,,bottom]{\color{textcolor}\rmfamily\fontsize{10.000000}{12.000000}\selectfont 20log Z}%
\end{pgfscope}%
\begin{pgfscope}%
\pgfpathrectangle{\pgfqpoint{1.015125in}{0.499691in}}{\pgfqpoint{3.875000in}{2.695000in}}%
\pgfusepath{clip}%
\pgfsetbuttcap%
\pgfsetroundjoin%
\definecolor{currentfill}{rgb}{0.121569,0.466667,0.705882}%
\pgfsetfillcolor{currentfill}%
\pgfsetlinewidth{1.003750pt}%
\definecolor{currentstroke}{rgb}{0.121569,0.466667,0.705882}%
\pgfsetstrokecolor{currentstroke}%
\pgfsetdash{}{0pt}%
\pgfpathmoveto{\pgfqpoint{3.564993in}{1.131391in}}%
\pgfpathcurveto{\pgfqpoint{3.576043in}{1.131391in}}{\pgfqpoint{3.586642in}{1.135781in}}{\pgfqpoint{3.594456in}{1.143595in}}%
\pgfpathcurveto{\pgfqpoint{3.602269in}{1.151408in}}{\pgfqpoint{3.606660in}{1.162007in}}{\pgfqpoint{3.606660in}{1.173057in}}%
\pgfpathcurveto{\pgfqpoint{3.606660in}{1.184108in}}{\pgfqpoint{3.602269in}{1.194707in}}{\pgfqpoint{3.594456in}{1.202520in}}%
\pgfpathcurveto{\pgfqpoint{3.586642in}{1.210334in}}{\pgfqpoint{3.576043in}{1.214724in}}{\pgfqpoint{3.564993in}{1.214724in}}%
\pgfpathcurveto{\pgfqpoint{3.553943in}{1.214724in}}{\pgfqpoint{3.543344in}{1.210334in}}{\pgfqpoint{3.535530in}{1.202520in}}%
\pgfpathcurveto{\pgfqpoint{3.527717in}{1.194707in}}{\pgfqpoint{3.523326in}{1.184108in}}{\pgfqpoint{3.523326in}{1.173057in}}%
\pgfpathcurveto{\pgfqpoint{3.523326in}{1.162007in}}{\pgfqpoint{3.527717in}{1.151408in}}{\pgfqpoint{3.535530in}{1.143595in}}%
\pgfpathcurveto{\pgfqpoint{3.543344in}{1.135781in}}{\pgfqpoint{3.553943in}{1.131391in}}{\pgfqpoint{3.564993in}{1.131391in}}%
\pgfpathclose%
\pgfusepath{stroke,fill}%
\end{pgfscope}%
\begin{pgfscope}%
\pgfpathrectangle{\pgfqpoint{1.015125in}{0.499691in}}{\pgfqpoint{3.875000in}{2.695000in}}%
\pgfusepath{clip}%
\pgfsetbuttcap%
\pgfsetroundjoin%
\definecolor{currentfill}{rgb}{0.121569,0.466667,0.705882}%
\pgfsetfillcolor{currentfill}%
\pgfsetlinewidth{1.003750pt}%
\definecolor{currentstroke}{rgb}{0.121569,0.466667,0.705882}%
\pgfsetstrokecolor{currentstroke}%
\pgfsetdash{}{0pt}%
\pgfpathmoveto{\pgfqpoint{3.653572in}{1.069021in}}%
\pgfpathcurveto{\pgfqpoint{3.664623in}{1.069021in}}{\pgfqpoint{3.675222in}{1.073411in}}{\pgfqpoint{3.683035in}{1.081225in}}%
\pgfpathcurveto{\pgfqpoint{3.690849in}{1.089038in}}{\pgfqpoint{3.695239in}{1.099638in}}{\pgfqpoint{3.695239in}{1.110688in}}%
\pgfpathcurveto{\pgfqpoint{3.695239in}{1.121738in}}{\pgfqpoint{3.690849in}{1.132337in}}{\pgfqpoint{3.683035in}{1.140150in}}%
\pgfpathcurveto{\pgfqpoint{3.675222in}{1.147964in}}{\pgfqpoint{3.664623in}{1.152354in}}{\pgfqpoint{3.653572in}{1.152354in}}%
\pgfpathcurveto{\pgfqpoint{3.642522in}{1.152354in}}{\pgfqpoint{3.631923in}{1.147964in}}{\pgfqpoint{3.624110in}{1.140150in}}%
\pgfpathcurveto{\pgfqpoint{3.616296in}{1.132337in}}{\pgfqpoint{3.611906in}{1.121738in}}{\pgfqpoint{3.611906in}{1.110688in}}%
\pgfpathcurveto{\pgfqpoint{3.611906in}{1.099638in}}{\pgfqpoint{3.616296in}{1.089038in}}{\pgfqpoint{3.624110in}{1.081225in}}%
\pgfpathcurveto{\pgfqpoint{3.631923in}{1.073411in}}{\pgfqpoint{3.642522in}{1.069021in}}{\pgfqpoint{3.653572in}{1.069021in}}%
\pgfpathclose%
\pgfusepath{stroke,fill}%
\end{pgfscope}%
\begin{pgfscope}%
\pgfpathrectangle{\pgfqpoint{1.015125in}{0.499691in}}{\pgfqpoint{3.875000in}{2.695000in}}%
\pgfusepath{clip}%
\pgfsetbuttcap%
\pgfsetroundjoin%
\definecolor{currentfill}{rgb}{0.121569,0.466667,0.705882}%
\pgfsetfillcolor{currentfill}%
\pgfsetlinewidth{1.003750pt}%
\definecolor{currentstroke}{rgb}{0.121569,0.466667,0.705882}%
\pgfsetstrokecolor{currentstroke}%
\pgfsetdash{}{0pt}%
\pgfpathmoveto{\pgfqpoint{3.735894in}{1.029337in}}%
\pgfpathcurveto{\pgfqpoint{3.746944in}{1.029337in}}{\pgfqpoint{3.757543in}{1.033728in}}{\pgfqpoint{3.765357in}{1.041541in}}%
\pgfpathcurveto{\pgfqpoint{3.773170in}{1.049355in}}{\pgfqpoint{3.777560in}{1.059954in}}{\pgfqpoint{3.777560in}{1.071004in}}%
\pgfpathcurveto{\pgfqpoint{3.777560in}{1.082054in}}{\pgfqpoint{3.773170in}{1.092653in}}{\pgfqpoint{3.765357in}{1.100467in}}%
\pgfpathcurveto{\pgfqpoint{3.757543in}{1.108281in}}{\pgfqpoint{3.746944in}{1.112671in}}{\pgfqpoint{3.735894in}{1.112671in}}%
\pgfpathcurveto{\pgfqpoint{3.724844in}{1.112671in}}{\pgfqpoint{3.714245in}{1.108281in}}{\pgfqpoint{3.706431in}{1.100467in}}%
\pgfpathcurveto{\pgfqpoint{3.698617in}{1.092653in}}{\pgfqpoint{3.694227in}{1.082054in}}{\pgfqpoint{3.694227in}{1.071004in}}%
\pgfpathcurveto{\pgfqpoint{3.694227in}{1.059954in}}{\pgfqpoint{3.698617in}{1.049355in}}{\pgfqpoint{3.706431in}{1.041541in}}%
\pgfpathcurveto{\pgfqpoint{3.714245in}{1.033728in}}{\pgfqpoint{3.724844in}{1.029337in}}{\pgfqpoint{3.735894in}{1.029337in}}%
\pgfpathclose%
\pgfusepath{stroke,fill}%
\end{pgfscope}%
\begin{pgfscope}%
\pgfpathrectangle{\pgfqpoint{1.015125in}{0.499691in}}{\pgfqpoint{3.875000in}{2.695000in}}%
\pgfusepath{clip}%
\pgfsetbuttcap%
\pgfsetroundjoin%
\definecolor{currentfill}{rgb}{0.121569,0.466667,0.705882}%
\pgfsetfillcolor{currentfill}%
\pgfsetlinewidth{1.003750pt}%
\definecolor{currentstroke}{rgb}{0.121569,0.466667,0.705882}%
\pgfsetstrokecolor{currentstroke}%
\pgfsetdash{}{0pt}%
\pgfpathmoveto{\pgfqpoint{3.812783in}{0.972394in}}%
\pgfpathcurveto{\pgfqpoint{3.823833in}{0.972394in}}{\pgfqpoint{3.834432in}{0.976784in}}{\pgfqpoint{3.842246in}{0.984598in}}%
\pgfpathcurveto{\pgfqpoint{3.850060in}{0.992411in}}{\pgfqpoint{3.854450in}{1.003010in}}{\pgfqpoint{3.854450in}{1.014060in}}%
\pgfpathcurveto{\pgfqpoint{3.854450in}{1.025110in}}{\pgfqpoint{3.850060in}{1.035710in}}{\pgfqpoint{3.842246in}{1.043523in}}%
\pgfpathcurveto{\pgfqpoint{3.834432in}{1.051337in}}{\pgfqpoint{3.823833in}{1.055727in}}{\pgfqpoint{3.812783in}{1.055727in}}%
\pgfpathcurveto{\pgfqpoint{3.801733in}{1.055727in}}{\pgfqpoint{3.791134in}{1.051337in}}{\pgfqpoint{3.783320in}{1.043523in}}%
\pgfpathcurveto{\pgfqpoint{3.775507in}{1.035710in}}{\pgfqpoint{3.771117in}{1.025110in}}{\pgfqpoint{3.771117in}{1.014060in}}%
\pgfpathcurveto{\pgfqpoint{3.771117in}{1.003010in}}{\pgfqpoint{3.775507in}{0.992411in}}{\pgfqpoint{3.783320in}{0.984598in}}%
\pgfpathcurveto{\pgfqpoint{3.791134in}{0.976784in}}{\pgfqpoint{3.801733in}{0.972394in}}{\pgfqpoint{3.812783in}{0.972394in}}%
\pgfpathclose%
\pgfusepath{stroke,fill}%
\end{pgfscope}%
\begin{pgfscope}%
\pgfpathrectangle{\pgfqpoint{1.015125in}{0.499691in}}{\pgfqpoint{3.875000in}{2.695000in}}%
\pgfusepath{clip}%
\pgfsetbuttcap%
\pgfsetroundjoin%
\definecolor{currentfill}{rgb}{0.121569,0.466667,0.705882}%
\pgfsetfillcolor{currentfill}%
\pgfsetlinewidth{1.003750pt}%
\definecolor{currentstroke}{rgb}{0.121569,0.466667,0.705882}%
\pgfsetstrokecolor{currentstroke}%
\pgfsetdash{}{0pt}%
\pgfpathmoveto{\pgfqpoint{3.884913in}{0.936018in}}%
\pgfpathcurveto{\pgfqpoint{3.895964in}{0.936018in}}{\pgfqpoint{3.906563in}{0.940408in}}{\pgfqpoint{3.914376in}{0.948222in}}%
\pgfpathcurveto{\pgfqpoint{3.922190in}{0.956035in}}{\pgfqpoint{3.926580in}{0.966634in}}{\pgfqpoint{3.926580in}{0.977684in}}%
\pgfpathcurveto{\pgfqpoint{3.926580in}{0.988735in}}{\pgfqpoint{3.922190in}{0.999334in}}{\pgfqpoint{3.914376in}{1.007147in}}%
\pgfpathcurveto{\pgfqpoint{3.906563in}{1.014961in}}{\pgfqpoint{3.895964in}{1.019351in}}{\pgfqpoint{3.884913in}{1.019351in}}%
\pgfpathcurveto{\pgfqpoint{3.873863in}{1.019351in}}{\pgfqpoint{3.863264in}{1.014961in}}{\pgfqpoint{3.855451in}{1.007147in}}%
\pgfpathcurveto{\pgfqpoint{3.847637in}{0.999334in}}{\pgfqpoint{3.843247in}{0.988735in}}{\pgfqpoint{3.843247in}{0.977684in}}%
\pgfpathcurveto{\pgfqpoint{3.843247in}{0.966634in}}{\pgfqpoint{3.847637in}{0.956035in}}{\pgfqpoint{3.855451in}{0.948222in}}%
\pgfpathcurveto{\pgfqpoint{3.863264in}{0.940408in}}{\pgfqpoint{3.873863in}{0.936018in}}{\pgfqpoint{3.884913in}{0.936018in}}%
\pgfpathclose%
\pgfusepath{stroke,fill}%
\end{pgfscope}%
\begin{pgfscope}%
\pgfpathrectangle{\pgfqpoint{1.015125in}{0.499691in}}{\pgfqpoint{3.875000in}{2.695000in}}%
\pgfusepath{clip}%
\pgfsetbuttcap%
\pgfsetroundjoin%
\definecolor{currentfill}{rgb}{0.121569,0.466667,0.705882}%
\pgfsetfillcolor{currentfill}%
\pgfsetlinewidth{1.003750pt}%
\definecolor{currentstroke}{rgb}{0.121569,0.466667,0.705882}%
\pgfsetstrokecolor{currentstroke}%
\pgfsetdash{}{0pt}%
\pgfpathmoveto{\pgfqpoint{3.952839in}{0.883631in}}%
\pgfpathcurveto{\pgfqpoint{3.963890in}{0.883631in}}{\pgfqpoint{3.974489in}{0.888022in}}{\pgfqpoint{3.982302in}{0.895835in}}%
\pgfpathcurveto{\pgfqpoint{3.990116in}{0.903649in}}{\pgfqpoint{3.994506in}{0.914248in}}{\pgfqpoint{3.994506in}{0.925298in}}%
\pgfpathcurveto{\pgfqpoint{3.994506in}{0.936348in}}{\pgfqpoint{3.990116in}{0.946947in}}{\pgfqpoint{3.982302in}{0.954761in}}%
\pgfpathcurveto{\pgfqpoint{3.974489in}{0.962574in}}{\pgfqpoint{3.963890in}{0.966965in}}{\pgfqpoint{3.952839in}{0.966965in}}%
\pgfpathcurveto{\pgfqpoint{3.941789in}{0.966965in}}{\pgfqpoint{3.931190in}{0.962574in}}{\pgfqpoint{3.923377in}{0.954761in}}%
\pgfpathcurveto{\pgfqpoint{3.915563in}{0.946947in}}{\pgfqpoint{3.911173in}{0.936348in}}{\pgfqpoint{3.911173in}{0.925298in}}%
\pgfpathcurveto{\pgfqpoint{3.911173in}{0.914248in}}{\pgfqpoint{3.915563in}{0.903649in}}{\pgfqpoint{3.923377in}{0.895835in}}%
\pgfpathcurveto{\pgfqpoint{3.931190in}{0.888022in}}{\pgfqpoint{3.941789in}{0.883631in}}{\pgfqpoint{3.952839in}{0.883631in}}%
\pgfpathclose%
\pgfusepath{stroke,fill}%
\end{pgfscope}%
\begin{pgfscope}%
\pgfpathrectangle{\pgfqpoint{1.015125in}{0.499691in}}{\pgfqpoint{3.875000in}{2.695000in}}%
\pgfusepath{clip}%
\pgfsetbuttcap%
\pgfsetroundjoin%
\definecolor{currentfill}{rgb}{0.121569,0.466667,0.705882}%
\pgfsetfillcolor{currentfill}%
\pgfsetlinewidth{1.003750pt}%
\definecolor{currentstroke}{rgb}{0.121569,0.466667,0.705882}%
\pgfsetstrokecolor{currentstroke}%
\pgfsetdash{}{0pt}%
\pgfpathmoveto{\pgfqpoint{4.017025in}{0.850054in}}%
\pgfpathcurveto{\pgfqpoint{4.028075in}{0.850054in}}{\pgfqpoint{4.038674in}{0.854444in}}{\pgfqpoint{4.046487in}{0.862258in}}%
\pgfpathcurveto{\pgfqpoint{4.054301in}{0.870072in}}{\pgfqpoint{4.058691in}{0.880671in}}{\pgfqpoint{4.058691in}{0.891721in}}%
\pgfpathcurveto{\pgfqpoint{4.058691in}{0.902771in}}{\pgfqpoint{4.054301in}{0.913370in}}{\pgfqpoint{4.046487in}{0.921183in}}%
\pgfpathcurveto{\pgfqpoint{4.038674in}{0.928997in}}{\pgfqpoint{4.028075in}{0.933387in}}{\pgfqpoint{4.017025in}{0.933387in}}%
\pgfpathcurveto{\pgfqpoint{4.005974in}{0.933387in}}{\pgfqpoint{3.995375in}{0.928997in}}{\pgfqpoint{3.987562in}{0.921183in}}%
\pgfpathcurveto{\pgfqpoint{3.979748in}{0.913370in}}{\pgfqpoint{3.975358in}{0.902771in}}{\pgfqpoint{3.975358in}{0.891721in}}%
\pgfpathcurveto{\pgfqpoint{3.975358in}{0.880671in}}{\pgfqpoint{3.979748in}{0.870072in}}{\pgfqpoint{3.987562in}{0.862258in}}%
\pgfpathcurveto{\pgfqpoint{3.995375in}{0.854444in}}{\pgfqpoint{4.005974in}{0.850054in}}{\pgfqpoint{4.017025in}{0.850054in}}%
\pgfpathclose%
\pgfusepath{stroke,fill}%
\end{pgfscope}%
\begin{pgfscope}%
\pgfpathrectangle{\pgfqpoint{1.015125in}{0.499691in}}{\pgfqpoint{3.875000in}{2.695000in}}%
\pgfusepath{clip}%
\pgfsetbuttcap%
\pgfsetroundjoin%
\definecolor{currentfill}{rgb}{0.121569,0.466667,0.705882}%
\pgfsetfillcolor{currentfill}%
\pgfsetlinewidth{1.003750pt}%
\definecolor{currentstroke}{rgb}{0.121569,0.466667,0.705882}%
\pgfsetstrokecolor{currentstroke}%
\pgfsetdash{}{0pt}%
\pgfpathmoveto{\pgfqpoint{4.077859in}{0.825532in}}%
\pgfpathcurveto{\pgfqpoint{4.088909in}{0.825532in}}{\pgfqpoint{4.099508in}{0.829922in}}{\pgfqpoint{4.107322in}{0.837736in}}%
\pgfpathcurveto{\pgfqpoint{4.115136in}{0.845550in}}{\pgfqpoint{4.119526in}{0.856149in}}{\pgfqpoint{4.119526in}{0.867199in}}%
\pgfpathcurveto{\pgfqpoint{4.119526in}{0.878249in}}{\pgfqpoint{4.115136in}{0.888848in}}{\pgfqpoint{4.107322in}{0.896662in}}%
\pgfpathcurveto{\pgfqpoint{4.099508in}{0.904475in}}{\pgfqpoint{4.088909in}{0.908866in}}{\pgfqpoint{4.077859in}{0.908866in}}%
\pgfpathcurveto{\pgfqpoint{4.066809in}{0.908866in}}{\pgfqpoint{4.056210in}{0.904475in}}{\pgfqpoint{4.048396in}{0.896662in}}%
\pgfpathcurveto{\pgfqpoint{4.040583in}{0.888848in}}{\pgfqpoint{4.036193in}{0.878249in}}{\pgfqpoint{4.036193in}{0.867199in}}%
\pgfpathcurveto{\pgfqpoint{4.036193in}{0.856149in}}{\pgfqpoint{4.040583in}{0.845550in}}{\pgfqpoint{4.048396in}{0.837736in}}%
\pgfpathcurveto{\pgfqpoint{4.056210in}{0.829922in}}{\pgfqpoint{4.066809in}{0.825532in}}{\pgfqpoint{4.077859in}{0.825532in}}%
\pgfpathclose%
\pgfusepath{stroke,fill}%
\end{pgfscope}%
\begin{pgfscope}%
\pgfpathrectangle{\pgfqpoint{1.015125in}{0.499691in}}{\pgfqpoint{3.875000in}{2.695000in}}%
\pgfusepath{clip}%
\pgfsetbuttcap%
\pgfsetroundjoin%
\definecolor{currentfill}{rgb}{0.121569,0.466667,0.705882}%
\pgfsetfillcolor{currentfill}%
\pgfsetlinewidth{1.003750pt}%
\definecolor{currentstroke}{rgb}{0.121569,0.466667,0.705882}%
\pgfsetstrokecolor{currentstroke}%
\pgfsetdash{}{0pt}%
\pgfpathmoveto{\pgfqpoint{4.135676in}{0.793671in}}%
\pgfpathcurveto{\pgfqpoint{4.146726in}{0.793671in}}{\pgfqpoint{4.157325in}{0.798061in}}{\pgfqpoint{4.165139in}{0.805875in}}%
\pgfpathcurveto{\pgfqpoint{4.172952in}{0.813688in}}{\pgfqpoint{4.177343in}{0.824287in}}{\pgfqpoint{4.177343in}{0.835337in}}%
\pgfpathcurveto{\pgfqpoint{4.177343in}{0.846388in}}{\pgfqpoint{4.172952in}{0.856987in}}{\pgfqpoint{4.165139in}{0.864800in}}%
\pgfpathcurveto{\pgfqpoint{4.157325in}{0.872614in}}{\pgfqpoint{4.146726in}{0.877004in}}{\pgfqpoint{4.135676in}{0.877004in}}%
\pgfpathcurveto{\pgfqpoint{4.124626in}{0.877004in}}{\pgfqpoint{4.114027in}{0.872614in}}{\pgfqpoint{4.106213in}{0.864800in}}%
\pgfpathcurveto{\pgfqpoint{4.098400in}{0.856987in}}{\pgfqpoint{4.094009in}{0.846388in}}{\pgfqpoint{4.094009in}{0.835337in}}%
\pgfpathcurveto{\pgfqpoint{4.094009in}{0.824287in}}{\pgfqpoint{4.098400in}{0.813688in}}{\pgfqpoint{4.106213in}{0.805875in}}%
\pgfpathcurveto{\pgfqpoint{4.114027in}{0.798061in}}{\pgfqpoint{4.124626in}{0.793671in}}{\pgfqpoint{4.135676in}{0.793671in}}%
\pgfpathclose%
\pgfusepath{stroke,fill}%
\end{pgfscope}%
\begin{pgfscope}%
\pgfpathrectangle{\pgfqpoint{1.015125in}{0.499691in}}{\pgfqpoint{3.875000in}{2.695000in}}%
\pgfusepath{clip}%
\pgfsetbuttcap%
\pgfsetroundjoin%
\definecolor{currentfill}{rgb}{0.121569,0.466667,0.705882}%
\pgfsetfillcolor{currentfill}%
\pgfsetlinewidth{1.003750pt}%
\definecolor{currentstroke}{rgb}{0.121569,0.466667,0.705882}%
\pgfsetstrokecolor{currentstroke}%
\pgfsetdash{}{0pt}%
\pgfpathmoveto{\pgfqpoint{4.190760in}{0.785849in}}%
\pgfpathcurveto{\pgfqpoint{4.201811in}{0.785849in}}{\pgfqpoint{4.212410in}{0.790239in}}{\pgfqpoint{4.220223in}{0.798053in}}%
\pgfpathcurveto{\pgfqpoint{4.228037in}{0.805866in}}{\pgfqpoint{4.232427in}{0.816465in}}{\pgfqpoint{4.232427in}{0.827515in}}%
\pgfpathcurveto{\pgfqpoint{4.232427in}{0.838566in}}{\pgfqpoint{4.228037in}{0.849165in}}{\pgfqpoint{4.220223in}{0.856978in}}%
\pgfpathcurveto{\pgfqpoint{4.212410in}{0.864792in}}{\pgfqpoint{4.201811in}{0.869182in}}{\pgfqpoint{4.190760in}{0.869182in}}%
\pgfpathcurveto{\pgfqpoint{4.179710in}{0.869182in}}{\pgfqpoint{4.169111in}{0.864792in}}{\pgfqpoint{4.161298in}{0.856978in}}%
\pgfpathcurveto{\pgfqpoint{4.153484in}{0.849165in}}{\pgfqpoint{4.149094in}{0.838566in}}{\pgfqpoint{4.149094in}{0.827515in}}%
\pgfpathcurveto{\pgfqpoint{4.149094in}{0.816465in}}{\pgfqpoint{4.153484in}{0.805866in}}{\pgfqpoint{4.161298in}{0.798053in}}%
\pgfpathcurveto{\pgfqpoint{4.169111in}{0.790239in}}{\pgfqpoint{4.179710in}{0.785849in}}{\pgfqpoint{4.190760in}{0.785849in}}%
\pgfpathclose%
\pgfusepath{stroke,fill}%
\end{pgfscope}%
\begin{pgfscope}%
\pgfpathrectangle{\pgfqpoint{1.015125in}{0.499691in}}{\pgfqpoint{3.875000in}{2.695000in}}%
\pgfusepath{clip}%
\pgfsetbuttcap%
\pgfsetroundjoin%
\definecolor{currentfill}{rgb}{0.121569,0.466667,0.705882}%
\pgfsetfillcolor{currentfill}%
\pgfsetlinewidth{1.003750pt}%
\definecolor{currentstroke}{rgb}{0.121569,0.466667,0.705882}%
\pgfsetstrokecolor{currentstroke}%
\pgfsetdash{}{0pt}%
\pgfpathmoveto{\pgfqpoint{4.243359in}{0.755109in}}%
\pgfpathcurveto{\pgfqpoint{4.254409in}{0.755109in}}{\pgfqpoint{4.265008in}{0.759500in}}{\pgfqpoint{4.272822in}{0.767313in}}%
\pgfpathcurveto{\pgfqpoint{4.280635in}{0.775127in}}{\pgfqpoint{4.285025in}{0.785726in}}{\pgfqpoint{4.285025in}{0.796776in}}%
\pgfpathcurveto{\pgfqpoint{4.285025in}{0.807826in}}{\pgfqpoint{4.280635in}{0.818425in}}{\pgfqpoint{4.272822in}{0.826239in}}%
\pgfpathcurveto{\pgfqpoint{4.265008in}{0.834052in}}{\pgfqpoint{4.254409in}{0.838443in}}{\pgfqpoint{4.243359in}{0.838443in}}%
\pgfpathcurveto{\pgfqpoint{4.232309in}{0.838443in}}{\pgfqpoint{4.221710in}{0.834052in}}{\pgfqpoint{4.213896in}{0.826239in}}%
\pgfpathcurveto{\pgfqpoint{4.206082in}{0.818425in}}{\pgfqpoint{4.201692in}{0.807826in}}{\pgfqpoint{4.201692in}{0.796776in}}%
\pgfpathcurveto{\pgfqpoint{4.201692in}{0.785726in}}{\pgfqpoint{4.206082in}{0.775127in}}{\pgfqpoint{4.213896in}{0.767313in}}%
\pgfpathcurveto{\pgfqpoint{4.221710in}{0.759500in}}{\pgfqpoint{4.232309in}{0.755109in}}{\pgfqpoint{4.243359in}{0.755109in}}%
\pgfpathclose%
\pgfusepath{stroke,fill}%
\end{pgfscope}%
\begin{pgfscope}%
\pgfpathrectangle{\pgfqpoint{1.015125in}{0.499691in}}{\pgfqpoint{3.875000in}{2.695000in}}%
\pgfusepath{clip}%
\pgfsetbuttcap%
\pgfsetroundjoin%
\definecolor{currentfill}{rgb}{0.121569,0.466667,0.705882}%
\pgfsetfillcolor{currentfill}%
\pgfsetlinewidth{1.003750pt}%
\definecolor{currentstroke}{rgb}{0.121569,0.466667,0.705882}%
\pgfsetstrokecolor{currentstroke}%
\pgfsetdash{}{0pt}%
\pgfpathmoveto{\pgfqpoint{4.293686in}{0.740058in}}%
\pgfpathcurveto{\pgfqpoint{4.304736in}{0.740058in}}{\pgfqpoint{4.315335in}{0.744449in}}{\pgfqpoint{4.323149in}{0.752262in}}%
\pgfpathcurveto{\pgfqpoint{4.330962in}{0.760076in}}{\pgfqpoint{4.335353in}{0.770675in}}{\pgfqpoint{4.335353in}{0.781725in}}%
\pgfpathcurveto{\pgfqpoint{4.335353in}{0.792775in}}{\pgfqpoint{4.330962in}{0.803374in}}{\pgfqpoint{4.323149in}{0.811188in}}%
\pgfpathcurveto{\pgfqpoint{4.315335in}{0.819001in}}{\pgfqpoint{4.304736in}{0.823392in}}{\pgfqpoint{4.293686in}{0.823392in}}%
\pgfpathcurveto{\pgfqpoint{4.282636in}{0.823392in}}{\pgfqpoint{4.272037in}{0.819001in}}{\pgfqpoint{4.264223in}{0.811188in}}%
\pgfpathcurveto{\pgfqpoint{4.256410in}{0.803374in}}{\pgfqpoint{4.252019in}{0.792775in}}{\pgfqpoint{4.252019in}{0.781725in}}%
\pgfpathcurveto{\pgfqpoint{4.252019in}{0.770675in}}{\pgfqpoint{4.256410in}{0.760076in}}{\pgfqpoint{4.264223in}{0.752262in}}%
\pgfpathcurveto{\pgfqpoint{4.272037in}{0.744449in}}{\pgfqpoint{4.282636in}{0.740058in}}{\pgfqpoint{4.293686in}{0.740058in}}%
\pgfpathclose%
\pgfusepath{stroke,fill}%
\end{pgfscope}%
\begin{pgfscope}%
\pgfpathrectangle{\pgfqpoint{1.015125in}{0.499691in}}{\pgfqpoint{3.875000in}{2.695000in}}%
\pgfusepath{clip}%
\pgfsetbuttcap%
\pgfsetroundjoin%
\definecolor{currentfill}{rgb}{0.121569,0.466667,0.705882}%
\pgfsetfillcolor{currentfill}%
\pgfsetlinewidth{1.003750pt}%
\definecolor{currentstroke}{rgb}{0.121569,0.466667,0.705882}%
\pgfsetstrokecolor{currentstroke}%
\pgfsetdash{}{0pt}%
\pgfpathmoveto{\pgfqpoint{4.341930in}{0.725212in}}%
\pgfpathcurveto{\pgfqpoint{4.352980in}{0.725212in}}{\pgfqpoint{4.363579in}{0.729603in}}{\pgfqpoint{4.371393in}{0.737416in}}%
\pgfpathcurveto{\pgfqpoint{4.379207in}{0.745230in}}{\pgfqpoint{4.383597in}{0.755829in}}{\pgfqpoint{4.383597in}{0.766879in}}%
\pgfpathcurveto{\pgfqpoint{4.383597in}{0.777929in}}{\pgfqpoint{4.379207in}{0.788528in}}{\pgfqpoint{4.371393in}{0.796342in}}%
\pgfpathcurveto{\pgfqpoint{4.363579in}{0.804155in}}{\pgfqpoint{4.352980in}{0.808546in}}{\pgfqpoint{4.341930in}{0.808546in}}%
\pgfpathcurveto{\pgfqpoint{4.330880in}{0.808546in}}{\pgfqpoint{4.320281in}{0.804155in}}{\pgfqpoint{4.312467in}{0.796342in}}%
\pgfpathcurveto{\pgfqpoint{4.304654in}{0.788528in}}{\pgfqpoint{4.300264in}{0.777929in}}{\pgfqpoint{4.300264in}{0.766879in}}%
\pgfpathcurveto{\pgfqpoint{4.300264in}{0.755829in}}{\pgfqpoint{4.304654in}{0.745230in}}{\pgfqpoint{4.312467in}{0.737416in}}%
\pgfpathcurveto{\pgfqpoint{4.320281in}{0.729603in}}{\pgfqpoint{4.330880in}{0.725212in}}{\pgfqpoint{4.341930in}{0.725212in}}%
\pgfpathclose%
\pgfusepath{stroke,fill}%
\end{pgfscope}%
\begin{pgfscope}%
\pgfpathrectangle{\pgfqpoint{1.015125in}{0.499691in}}{\pgfqpoint{3.875000in}{2.695000in}}%
\pgfusepath{clip}%
\pgfsetbuttcap%
\pgfsetroundjoin%
\definecolor{currentfill}{rgb}{0.121569,0.466667,0.705882}%
\pgfsetfillcolor{currentfill}%
\pgfsetlinewidth{1.003750pt}%
\definecolor{currentstroke}{rgb}{0.121569,0.466667,0.705882}%
\pgfsetstrokecolor{currentstroke}%
\pgfsetdash{}{0pt}%
\pgfpathmoveto{\pgfqpoint{4.388257in}{0.696113in}}%
\pgfpathcurveto{\pgfqpoint{4.399307in}{0.696113in}}{\pgfqpoint{4.409906in}{0.700503in}}{\pgfqpoint{4.417720in}{0.708316in}}%
\pgfpathcurveto{\pgfqpoint{4.425533in}{0.716130in}}{\pgfqpoint{4.429924in}{0.726729in}}{\pgfqpoint{4.429924in}{0.737779in}}%
\pgfpathcurveto{\pgfqpoint{4.429924in}{0.748829in}}{\pgfqpoint{4.425533in}{0.759428in}}{\pgfqpoint{4.417720in}{0.767242in}}%
\pgfpathcurveto{\pgfqpoint{4.409906in}{0.775056in}}{\pgfqpoint{4.399307in}{0.779446in}}{\pgfqpoint{4.388257in}{0.779446in}}%
\pgfpathcurveto{\pgfqpoint{4.377207in}{0.779446in}}{\pgfqpoint{4.366608in}{0.775056in}}{\pgfqpoint{4.358794in}{0.767242in}}%
\pgfpathcurveto{\pgfqpoint{4.350980in}{0.759428in}}{\pgfqpoint{4.346590in}{0.748829in}}{\pgfqpoint{4.346590in}{0.737779in}}%
\pgfpathcurveto{\pgfqpoint{4.346590in}{0.726729in}}{\pgfqpoint{4.350980in}{0.716130in}}{\pgfqpoint{4.358794in}{0.708316in}}%
\pgfpathcurveto{\pgfqpoint{4.366608in}{0.700503in}}{\pgfqpoint{4.377207in}{0.696113in}}{\pgfqpoint{4.388257in}{0.696113in}}%
\pgfpathclose%
\pgfusepath{stroke,fill}%
\end{pgfscope}%
\begin{pgfscope}%
\pgfpathrectangle{\pgfqpoint{1.015125in}{0.499691in}}{\pgfqpoint{3.875000in}{2.695000in}}%
\pgfusepath{clip}%
\pgfsetbuttcap%
\pgfsetroundjoin%
\definecolor{currentfill}{rgb}{0.121569,0.466667,0.705882}%
\pgfsetfillcolor{currentfill}%
\pgfsetlinewidth{1.003750pt}%
\definecolor{currentstroke}{rgb}{0.121569,0.466667,0.705882}%
\pgfsetstrokecolor{currentstroke}%
\pgfsetdash{}{0pt}%
\pgfpathmoveto{\pgfqpoint{4.432813in}{0.667769in}}%
\pgfpathcurveto{\pgfqpoint{4.443863in}{0.667769in}}{\pgfqpoint{4.454462in}{0.672159in}}{\pgfqpoint{4.462275in}{0.679973in}}%
\pgfpathcurveto{\pgfqpoint{4.470089in}{0.687786in}}{\pgfqpoint{4.474479in}{0.698385in}}{\pgfqpoint{4.474479in}{0.709435in}}%
\pgfpathcurveto{\pgfqpoint{4.474479in}{0.720486in}}{\pgfqpoint{4.470089in}{0.731085in}}{\pgfqpoint{4.462275in}{0.738898in}}%
\pgfpathcurveto{\pgfqpoint{4.454462in}{0.746712in}}{\pgfqpoint{4.443863in}{0.751102in}}{\pgfqpoint{4.432813in}{0.751102in}}%
\pgfpathcurveto{\pgfqpoint{4.421763in}{0.751102in}}{\pgfqpoint{4.411163in}{0.746712in}}{\pgfqpoint{4.403350in}{0.738898in}}%
\pgfpathcurveto{\pgfqpoint{4.395536in}{0.731085in}}{\pgfqpoint{4.391146in}{0.720486in}}{\pgfqpoint{4.391146in}{0.709435in}}%
\pgfpathcurveto{\pgfqpoint{4.391146in}{0.698385in}}{\pgfqpoint{4.395536in}{0.687786in}}{\pgfqpoint{4.403350in}{0.679973in}}%
\pgfpathcurveto{\pgfqpoint{4.411163in}{0.672159in}}{\pgfqpoint{4.421763in}{0.667769in}}{\pgfqpoint{4.432813in}{0.667769in}}%
\pgfpathclose%
\pgfusepath{stroke,fill}%
\end{pgfscope}%
\begin{pgfscope}%
\pgfpathrectangle{\pgfqpoint{1.015125in}{0.499691in}}{\pgfqpoint{3.875000in}{2.695000in}}%
\pgfusepath{clip}%
\pgfsetbuttcap%
\pgfsetroundjoin%
\definecolor{currentfill}{rgb}{0.121569,0.466667,0.705882}%
\pgfsetfillcolor{currentfill}%
\pgfsetlinewidth{1.003750pt}%
\definecolor{currentstroke}{rgb}{0.121569,0.466667,0.705882}%
\pgfsetstrokecolor{currentstroke}%
\pgfsetdash{}{0pt}%
\pgfpathmoveto{\pgfqpoint{4.475728in}{0.667769in}}%
\pgfpathcurveto{\pgfqpoint{4.486778in}{0.667769in}}{\pgfqpoint{4.497377in}{0.672159in}}{\pgfqpoint{4.505191in}{0.679973in}}%
\pgfpathcurveto{\pgfqpoint{4.513004in}{0.687786in}}{\pgfqpoint{4.517395in}{0.698385in}}{\pgfqpoint{4.517395in}{0.709435in}}%
\pgfpathcurveto{\pgfqpoint{4.517395in}{0.720486in}}{\pgfqpoint{4.513004in}{0.731085in}}{\pgfqpoint{4.505191in}{0.738898in}}%
\pgfpathcurveto{\pgfqpoint{4.497377in}{0.746712in}}{\pgfqpoint{4.486778in}{0.751102in}}{\pgfqpoint{4.475728in}{0.751102in}}%
\pgfpathcurveto{\pgfqpoint{4.464678in}{0.751102in}}{\pgfqpoint{4.454079in}{0.746712in}}{\pgfqpoint{4.446265in}{0.738898in}}%
\pgfpathcurveto{\pgfqpoint{4.438452in}{0.731085in}}{\pgfqpoint{4.434061in}{0.720486in}}{\pgfqpoint{4.434061in}{0.709435in}}%
\pgfpathcurveto{\pgfqpoint{4.434061in}{0.698385in}}{\pgfqpoint{4.438452in}{0.687786in}}{\pgfqpoint{4.446265in}{0.679973in}}%
\pgfpathcurveto{\pgfqpoint{4.454079in}{0.672159in}}{\pgfqpoint{4.464678in}{0.667769in}}{\pgfqpoint{4.475728in}{0.667769in}}%
\pgfpathclose%
\pgfusepath{stroke,fill}%
\end{pgfscope}%
\begin{pgfscope}%
\pgfpathrectangle{\pgfqpoint{1.015125in}{0.499691in}}{\pgfqpoint{3.875000in}{2.695000in}}%
\pgfusepath{clip}%
\pgfsetbuttcap%
\pgfsetroundjoin%
\definecolor{currentfill}{rgb}{0.121569,0.466667,0.705882}%
\pgfsetfillcolor{currentfill}%
\pgfsetlinewidth{1.003750pt}%
\definecolor{currentstroke}{rgb}{0.121569,0.466667,0.705882}%
\pgfsetstrokecolor{currentstroke}%
\pgfsetdash{}{0pt}%
\pgfpathmoveto{\pgfqpoint{4.517119in}{0.640143in}}%
\pgfpathcurveto{\pgfqpoint{4.528169in}{0.640143in}}{\pgfqpoint{4.538768in}{0.644533in}}{\pgfqpoint{4.546582in}{0.652346in}}%
\pgfpathcurveto{\pgfqpoint{4.554396in}{0.660160in}}{\pgfqpoint{4.558786in}{0.670759in}}{\pgfqpoint{4.558786in}{0.681809in}}%
\pgfpathcurveto{\pgfqpoint{4.558786in}{0.692859in}}{\pgfqpoint{4.554396in}{0.703458in}}{\pgfqpoint{4.546582in}{0.711272in}}%
\pgfpathcurveto{\pgfqpoint{4.538768in}{0.719086in}}{\pgfqpoint{4.528169in}{0.723476in}}{\pgfqpoint{4.517119in}{0.723476in}}%
\pgfpathcurveto{\pgfqpoint{4.506069in}{0.723476in}}{\pgfqpoint{4.495470in}{0.719086in}}{\pgfqpoint{4.487657in}{0.711272in}}%
\pgfpathcurveto{\pgfqpoint{4.479843in}{0.703458in}}{\pgfqpoint{4.475453in}{0.692859in}}{\pgfqpoint{4.475453in}{0.681809in}}%
\pgfpathcurveto{\pgfqpoint{4.475453in}{0.670759in}}{\pgfqpoint{4.479843in}{0.660160in}}{\pgfqpoint{4.487657in}{0.652346in}}%
\pgfpathcurveto{\pgfqpoint{4.495470in}{0.644533in}}{\pgfqpoint{4.506069in}{0.640143in}}{\pgfqpoint{4.517119in}{0.640143in}}%
\pgfpathclose%
\pgfusepath{stroke,fill}%
\end{pgfscope}%
\begin{pgfscope}%
\pgfpathrectangle{\pgfqpoint{1.015125in}{0.499691in}}{\pgfqpoint{3.875000in}{2.695000in}}%
\pgfusepath{clip}%
\pgfsetbuttcap%
\pgfsetroundjoin%
\definecolor{currentfill}{rgb}{0.121569,0.466667,0.705882}%
\pgfsetfillcolor{currentfill}%
\pgfsetlinewidth{1.003750pt}%
\definecolor{currentstroke}{rgb}{0.121569,0.466667,0.705882}%
\pgfsetstrokecolor{currentstroke}%
\pgfsetdash{}{0pt}%
\pgfpathmoveto{\pgfqpoint{4.557091in}{0.640143in}}%
\pgfpathcurveto{\pgfqpoint{4.568141in}{0.640143in}}{\pgfqpoint{4.578740in}{0.644533in}}{\pgfqpoint{4.586554in}{0.652346in}}%
\pgfpathcurveto{\pgfqpoint{4.594368in}{0.660160in}}{\pgfqpoint{4.598758in}{0.670759in}}{\pgfqpoint{4.598758in}{0.681809in}}%
\pgfpathcurveto{\pgfqpoint{4.598758in}{0.692859in}}{\pgfqpoint{4.594368in}{0.703458in}}{\pgfqpoint{4.586554in}{0.711272in}}%
\pgfpathcurveto{\pgfqpoint{4.578740in}{0.719086in}}{\pgfqpoint{4.568141in}{0.723476in}}{\pgfqpoint{4.557091in}{0.723476in}}%
\pgfpathcurveto{\pgfqpoint{4.546041in}{0.723476in}}{\pgfqpoint{4.535442in}{0.719086in}}{\pgfqpoint{4.527628in}{0.711272in}}%
\pgfpathcurveto{\pgfqpoint{4.519815in}{0.703458in}}{\pgfqpoint{4.515425in}{0.692859in}}{\pgfqpoint{4.515425in}{0.681809in}}%
\pgfpathcurveto{\pgfqpoint{4.515425in}{0.670759in}}{\pgfqpoint{4.519815in}{0.660160in}}{\pgfqpoint{4.527628in}{0.652346in}}%
\pgfpathcurveto{\pgfqpoint{4.535442in}{0.644533in}}{\pgfqpoint{4.546041in}{0.640143in}}{\pgfqpoint{4.557091in}{0.640143in}}%
\pgfpathclose%
\pgfusepath{stroke,fill}%
\end{pgfscope}%
\begin{pgfscope}%
\pgfpathrectangle{\pgfqpoint{1.015125in}{0.499691in}}{\pgfqpoint{3.875000in}{2.695000in}}%
\pgfusepath{clip}%
\pgfsetbuttcap%
\pgfsetroundjoin%
\definecolor{currentfill}{rgb}{0.121569,0.466667,0.705882}%
\pgfsetfillcolor{currentfill}%
\pgfsetlinewidth{1.003750pt}%
\definecolor{currentstroke}{rgb}{0.121569,0.466667,0.705882}%
\pgfsetstrokecolor{currentstroke}%
\pgfsetdash{}{0pt}%
\pgfpathmoveto{\pgfqpoint{4.595738in}{0.613199in}}%
\pgfpathcurveto{\pgfqpoint{4.606788in}{0.613199in}}{\pgfqpoint{4.617387in}{0.617589in}}{\pgfqpoint{4.625201in}{0.625402in}}%
\pgfpathcurveto{\pgfqpoint{4.633014in}{0.633216in}}{\pgfqpoint{4.637405in}{0.643815in}}{\pgfqpoint{4.637405in}{0.654865in}}%
\pgfpathcurveto{\pgfqpoint{4.637405in}{0.665915in}}{\pgfqpoint{4.633014in}{0.676514in}}{\pgfqpoint{4.625201in}{0.684328in}}%
\pgfpathcurveto{\pgfqpoint{4.617387in}{0.692142in}}{\pgfqpoint{4.606788in}{0.696532in}}{\pgfqpoint{4.595738in}{0.696532in}}%
\pgfpathcurveto{\pgfqpoint{4.584688in}{0.696532in}}{\pgfqpoint{4.574089in}{0.692142in}}{\pgfqpoint{4.566275in}{0.684328in}}%
\pgfpathcurveto{\pgfqpoint{4.558461in}{0.676514in}}{\pgfqpoint{4.554071in}{0.665915in}}{\pgfqpoint{4.554071in}{0.654865in}}%
\pgfpathcurveto{\pgfqpoint{4.554071in}{0.643815in}}{\pgfqpoint{4.558461in}{0.633216in}}{\pgfqpoint{4.566275in}{0.625402in}}%
\pgfpathcurveto{\pgfqpoint{4.574089in}{0.617589in}}{\pgfqpoint{4.584688in}{0.613199in}}{\pgfqpoint{4.595738in}{0.613199in}}%
\pgfpathclose%
\pgfusepath{stroke,fill}%
\end{pgfscope}%
\begin{pgfscope}%
\pgfpathrectangle{\pgfqpoint{1.015125in}{0.499691in}}{\pgfqpoint{3.875000in}{2.695000in}}%
\pgfusepath{clip}%
\pgfsetbuttcap%
\pgfsetroundjoin%
\definecolor{currentfill}{rgb}{0.121569,0.466667,0.705882}%
\pgfsetfillcolor{currentfill}%
\pgfsetlinewidth{1.003750pt}%
\definecolor{currentstroke}{rgb}{0.121569,0.466667,0.705882}%
\pgfsetstrokecolor{currentstroke}%
\pgfsetdash{}{0pt}%
\pgfpathmoveto{\pgfqpoint{4.633144in}{0.613199in}}%
\pgfpathcurveto{\pgfqpoint{4.644194in}{0.613199in}}{\pgfqpoint{4.654793in}{0.617589in}}{\pgfqpoint{4.662607in}{0.625402in}}%
\pgfpathcurveto{\pgfqpoint{4.670421in}{0.633216in}}{\pgfqpoint{4.674811in}{0.643815in}}{\pgfqpoint{4.674811in}{0.654865in}}%
\pgfpathcurveto{\pgfqpoint{4.674811in}{0.665915in}}{\pgfqpoint{4.670421in}{0.676514in}}{\pgfqpoint{4.662607in}{0.684328in}}%
\pgfpathcurveto{\pgfqpoint{4.654793in}{0.692142in}}{\pgfqpoint{4.644194in}{0.696532in}}{\pgfqpoint{4.633144in}{0.696532in}}%
\pgfpathcurveto{\pgfqpoint{4.622094in}{0.696532in}}{\pgfqpoint{4.611495in}{0.692142in}}{\pgfqpoint{4.603682in}{0.684328in}}%
\pgfpathcurveto{\pgfqpoint{4.595868in}{0.676514in}}{\pgfqpoint{4.591478in}{0.665915in}}{\pgfqpoint{4.591478in}{0.654865in}}%
\pgfpathcurveto{\pgfqpoint{4.591478in}{0.643815in}}{\pgfqpoint{4.595868in}{0.633216in}}{\pgfqpoint{4.603682in}{0.625402in}}%
\pgfpathcurveto{\pgfqpoint{4.611495in}{0.617589in}}{\pgfqpoint{4.622094in}{0.613199in}}{\pgfqpoint{4.633144in}{0.613199in}}%
\pgfpathclose%
\pgfusepath{stroke,fill}%
\end{pgfscope}%
\begin{pgfscope}%
\pgfsetrectcap%
\pgfsetmiterjoin%
\pgfsetlinewidth{0.803000pt}%
\definecolor{currentstroke}{rgb}{0.000000,0.000000,0.000000}%
\pgfsetstrokecolor{currentstroke}%
\pgfsetdash{}{0pt}%
\pgfpathmoveto{\pgfqpoint{1.015125in}{0.499691in}}%
\pgfpathlineto{\pgfqpoint{1.015125in}{3.194691in}}%
\pgfusepath{stroke}%
\end{pgfscope}%
\begin{pgfscope}%
\pgfsetrectcap%
\pgfsetmiterjoin%
\pgfsetlinewidth{0.803000pt}%
\definecolor{currentstroke}{rgb}{0.000000,0.000000,0.000000}%
\pgfsetstrokecolor{currentstroke}%
\pgfsetdash{}{0pt}%
\pgfpathmoveto{\pgfqpoint{4.890125in}{0.499691in}}%
\pgfpathlineto{\pgfqpoint{4.890125in}{3.194691in}}%
\pgfusepath{stroke}%
\end{pgfscope}%
\begin{pgfscope}%
\pgfsetrectcap%
\pgfsetmiterjoin%
\pgfsetlinewidth{0.803000pt}%
\definecolor{currentstroke}{rgb}{0.000000,0.000000,0.000000}%
\pgfsetstrokecolor{currentstroke}%
\pgfsetdash{}{0pt}%
\pgfpathmoveto{\pgfqpoint{1.015125in}{0.499691in}}%
\pgfpathlineto{\pgfqpoint{4.890125in}{0.499691in}}%
\pgfusepath{stroke}%
\end{pgfscope}%
\begin{pgfscope}%
\pgfsetrectcap%
\pgfsetmiterjoin%
\pgfsetlinewidth{0.803000pt}%
\definecolor{currentstroke}{rgb}{0.000000,0.000000,0.000000}%
\pgfsetstrokecolor{currentstroke}%
\pgfsetdash{}{0pt}%
\pgfpathmoveto{\pgfqpoint{1.015125in}{3.194691in}}%
\pgfpathlineto{\pgfqpoint{4.890125in}{3.194691in}}%
\pgfusepath{stroke}%
\end{pgfscope}%
\end{pgfpicture}%
\makeatother%
\endgroup%
}
    \caption{Frecuencia (en Hz, escala logarítmica) frente al módulo de la impedancia (En Ohmios, escala logarímica escalada por 20)}
  \end{figure}

  Como se observa en la gráfica, parece que los puntos trazan una curva, aunque hay bastante "ruído" para comprobarlo. Discutiremos esto en el apartado de conclusiones y hablaremos de la fiabilidad de nuestras medidas. Sin embargo, tomamos más de las 20 mediciones exigidas, ya que al principio medimos desde la frecuencia más baja hasta la de corte únicamente, y luego nos dimos cuenta de que teníamos que situar la frecuencia de corte en el medio de nuestros datos. Por ese despiste, contamos con puntos extra para visualizar nuestra gráfica, dibujados a la derecha en un tono más claro, y se aprecia mejor que no es una recta. Los datos utilizados para esta representación se encuentran en el anexo al final de las memorias (\ref{extra1}).

  Ahora calcularemos las dos asíntotas de la gráfica, la horizontal si no hubiera un condensador ($Z = R$) y la oblicua si no hubiera una resistencia ($Z = \frac{1}{wC} = \frac{1}{2\pi fC}$). Podemos hacerlo añadiendo la siguiente función a nuestro código anterior:

  \begin{python}
    c = 1.2 * 10**-8

    def asintotas(r, c, f, v1, v2):
        x = np.log10(f)

        #Curva R
        yr = 20 * np.log10(r) * np.ones(v1.shape[0])
        plt.plot(x, yr, color="tomato", zorder=0)

        #Curva C
        z = 1 / (2 * np.pi * f * c)
        yc = 20 * np.log10(z)
        plt.plot(x, yc, color="gold", zorder=0)

        #Interseccion
        m1, b1 = 0, yr[0]
        m2, b2 = np.polyfit(x, yc, 1)
        xint = (b2 - b1) / (m1 - m2)
        yint = m1 * xint + b1
        print("Interseccion: x", xint, "y", yint)
        plt.scatter(xint, yint, color="dodgerblue", zorder=3)

    asintotas(r, c, f, vm, vmr)
  \end{python}

  \begin{figure}[H]
    %\centering
    \resizebox{\columnwidth}{!}{
    %% Creator: Matplotlib, PGF backend
%%
%% To include the figure in your LaTeX document, write
%%   \input{<filename>.pgf}
%%
%% Make sure the required packages are loaded in your preamble
%%   \usepackage{pgf}
%%
%% Figures using additional raster images can only be included by \input if
%% they are in the same directory as the main LaTeX file. For loading figures
%% from other directories you can use the `import` package
%%   \usepackage{import}
%% and then include the figures with
%%   \import{<path to file>}{<filename>.pgf}
%%
%% Matplotlib used the following preamble
%%
\begingroup%
\makeatletter%
\begin{pgfpicture}%
\pgfpathrectangle{\pgfpointorigin}{\pgfqpoint{4.840323in}{3.294691in}}%
\pgfusepath{use as bounding box, clip}%
\begin{pgfscope}%
\pgfsetbuttcap%
\pgfsetmiterjoin%
\definecolor{currentfill}{rgb}{1.000000,1.000000,1.000000}%
\pgfsetfillcolor{currentfill}%
\pgfsetlinewidth{0.000000pt}%
\definecolor{currentstroke}{rgb}{1.000000,1.000000,1.000000}%
\pgfsetstrokecolor{currentstroke}%
\pgfsetdash{}{0pt}%
\pgfpathmoveto{\pgfqpoint{0.000000in}{0.000000in}}%
\pgfpathlineto{\pgfqpoint{4.840323in}{0.000000in}}%
\pgfpathlineto{\pgfqpoint{4.840323in}{3.294691in}}%
\pgfpathlineto{\pgfqpoint{0.000000in}{3.294691in}}%
\pgfpathclose%
\pgfusepath{fill}%
\end{pgfscope}%
\begin{pgfscope}%
\pgfsetbuttcap%
\pgfsetmiterjoin%
\definecolor{currentfill}{rgb}{1.000000,1.000000,1.000000}%
\pgfsetfillcolor{currentfill}%
\pgfsetlinewidth{0.000000pt}%
\definecolor{currentstroke}{rgb}{0.000000,0.000000,0.000000}%
\pgfsetstrokecolor{currentstroke}%
\pgfsetstrokeopacity{0.000000}%
\pgfsetdash{}{0pt}%
\pgfpathmoveto{\pgfqpoint{0.837655in}{0.499691in}}%
\pgfpathlineto{\pgfqpoint{4.712655in}{0.499691in}}%
\pgfpathlineto{\pgfqpoint{4.712655in}{3.194691in}}%
\pgfpathlineto{\pgfqpoint{0.837655in}{3.194691in}}%
\pgfpathclose%
\pgfusepath{fill}%
\end{pgfscope}%
\begin{pgfscope}%
\pgfpathrectangle{\pgfqpoint{0.837655in}{0.499691in}}{\pgfqpoint{3.875000in}{2.695000in}}%
\pgfusepath{clip}%
\pgfsetrectcap%
\pgfsetroundjoin%
\pgfsetlinewidth{1.505625pt}%
\definecolor{currentstroke}{rgb}{1.000000,0.388235,0.278431}%
\pgfsetstrokecolor{currentstroke}%
\pgfsetdash{}{0pt}%
\pgfpathmoveto{\pgfqpoint{1.079760in}{1.453988in}}%
\pgfpathlineto{\pgfqpoint{1.360950in}{1.453988in}}%
\pgfpathlineto{\pgfqpoint{1.622275in}{1.453988in}}%
\pgfpathlineto{\pgfqpoint{1.866356in}{1.453988in}}%
\pgfpathlineto{\pgfqpoint{2.095330in}{1.453988in}}%
\pgfpathlineto{\pgfqpoint{2.310958in}{1.453988in}}%
\pgfpathlineto{\pgfqpoint{2.514710in}{1.453988in}}%
\pgfpathlineto{\pgfqpoint{2.707827in}{1.453988in}}%
\pgfpathlineto{\pgfqpoint{2.891363in}{1.453988in}}%
\pgfpathlineto{\pgfqpoint{3.066226in}{1.453988in}}%
\pgfpathlineto{\pgfqpoint{3.233197in}{1.453988in}}%
\pgfpathlineto{\pgfqpoint{3.392958in}{1.453988in}}%
\pgfpathlineto{\pgfqpoint{3.546107in}{1.453988in}}%
\pgfpathlineto{\pgfqpoint{3.693168in}{1.453988in}}%
\pgfpathlineto{\pgfqpoint{3.834608in}{1.453988in}}%
\pgfpathlineto{\pgfqpoint{3.970841in}{1.453988in}}%
\pgfpathlineto{\pgfqpoint{4.102235in}{1.453988in}}%
\pgfpathlineto{\pgfqpoint{4.229124in}{1.453988in}}%
\pgfpathlineto{\pgfqpoint{4.351806in}{1.453988in}}%
\pgfpathlineto{\pgfqpoint{4.470551in}{1.453988in}}%
\pgfusepath{stroke}%
\end{pgfscope}%
\begin{pgfscope}%
\pgfpathrectangle{\pgfqpoint{0.837655in}{0.499691in}}{\pgfqpoint{3.875000in}{2.695000in}}%
\pgfusepath{clip}%
\pgfsetrectcap%
\pgfsetroundjoin%
\pgfsetlinewidth{1.505625pt}%
\definecolor{currentstroke}{rgb}{1.000000,0.843137,0.000000}%
\pgfsetstrokecolor{currentstroke}%
\pgfsetdash{}{0pt}%
\pgfpathmoveto{\pgfqpoint{1.079760in}{2.742461in}}%
\pgfpathlineto{\pgfqpoint{1.360950in}{2.566632in}}%
\pgfpathlineto{\pgfqpoint{1.622275in}{2.403225in}}%
\pgfpathlineto{\pgfqpoint{1.866356in}{2.250600in}}%
\pgfpathlineto{\pgfqpoint{2.095330in}{2.107422in}}%
\pgfpathlineto{\pgfqpoint{2.310958in}{1.972590in}}%
\pgfpathlineto{\pgfqpoint{2.514710in}{1.845183in}}%
\pgfpathlineto{\pgfqpoint{2.707827in}{1.724427in}}%
\pgfpathlineto{\pgfqpoint{2.891363in}{1.609661in}}%
\pgfpathlineto{\pgfqpoint{3.066226in}{1.500319in}}%
\pgfpathlineto{\pgfqpoint{3.233197in}{1.395912in}}%
\pgfpathlineto{\pgfqpoint{3.392958in}{1.296012in}}%
\pgfpathlineto{\pgfqpoint{3.546107in}{1.200248in}}%
\pgfpathlineto{\pgfqpoint{3.693168in}{1.108290in}}%
\pgfpathlineto{\pgfqpoint{3.834608in}{1.019848in}}%
\pgfpathlineto{\pgfqpoint{3.970841in}{0.934661in}}%
\pgfpathlineto{\pgfqpoint{4.102235in}{0.852500in}}%
\pgfpathlineto{\pgfqpoint{4.229124in}{0.773156in}}%
\pgfpathlineto{\pgfqpoint{4.351806in}{0.696443in}}%
\pgfpathlineto{\pgfqpoint{4.470551in}{0.622191in}}%
\pgfusepath{stroke}%
\end{pgfscope}%
\begin{pgfscope}%
\pgfsetbuttcap%
\pgfsetroundjoin%
\definecolor{currentfill}{rgb}{0.000000,0.000000,0.000000}%
\pgfsetfillcolor{currentfill}%
\pgfsetlinewidth{0.803000pt}%
\definecolor{currentstroke}{rgb}{0.000000,0.000000,0.000000}%
\pgfsetstrokecolor{currentstroke}%
\pgfsetdash{}{0pt}%
\pgfsys@defobject{currentmarker}{\pgfqpoint{0.000000in}{-0.048611in}}{\pgfqpoint{0.000000in}{0.000000in}}{%
\pgfpathmoveto{\pgfqpoint{0.000000in}{0.000000in}}%
\pgfpathlineto{\pgfqpoint{0.000000in}{-0.048611in}}%
\pgfusepath{stroke,fill}%
}%
\begin{pgfscope}%
\pgfsys@transformshift{1.243244in}{0.499691in}%
\pgfsys@useobject{currentmarker}{}%
\end{pgfscope}%
\end{pgfscope}%
\begin{pgfscope}%
\definecolor{textcolor}{rgb}{0.000000,0.000000,0.000000}%
\pgfsetstrokecolor{textcolor}%
\pgfsetfillcolor{textcolor}%
\pgftext[x=1.243244in,y=0.402469in,,top]{\color{textcolor}\rmfamily\fontsize{10.000000}{12.000000}\selectfont \(\displaystyle 2.9\)}%
\end{pgfscope}%
\begin{pgfscope}%
\pgfsetbuttcap%
\pgfsetroundjoin%
\definecolor{currentfill}{rgb}{0.000000,0.000000,0.000000}%
\pgfsetfillcolor{currentfill}%
\pgfsetlinewidth{0.803000pt}%
\definecolor{currentstroke}{rgb}{0.000000,0.000000,0.000000}%
\pgfsetstrokecolor{currentstroke}%
\pgfsetdash{}{0pt}%
\pgfsys@defobject{currentmarker}{\pgfqpoint{0.000000in}{-0.048611in}}{\pgfqpoint{0.000000in}{0.000000in}}{%
\pgfpathmoveto{\pgfqpoint{0.000000in}{0.000000in}}%
\pgfpathlineto{\pgfqpoint{0.000000in}{-0.048611in}}%
\pgfusepath{stroke,fill}%
}%
\begin{pgfscope}%
\pgfsys@transformshift{2.095330in}{0.499691in}%
\pgfsys@useobject{currentmarker}{}%
\end{pgfscope}%
\end{pgfscope}%
\begin{pgfscope}%
\definecolor{textcolor}{rgb}{0.000000,0.000000,0.000000}%
\pgfsetstrokecolor{textcolor}%
\pgfsetfillcolor{textcolor}%
\pgftext[x=2.095330in,y=0.402469in,,top]{\color{textcolor}\rmfamily\fontsize{10.000000}{12.000000}\selectfont \(\displaystyle 3.0\)}%
\end{pgfscope}%
\begin{pgfscope}%
\pgfsetbuttcap%
\pgfsetroundjoin%
\definecolor{currentfill}{rgb}{0.000000,0.000000,0.000000}%
\pgfsetfillcolor{currentfill}%
\pgfsetlinewidth{0.803000pt}%
\definecolor{currentstroke}{rgb}{0.000000,0.000000,0.000000}%
\pgfsetstrokecolor{currentstroke}%
\pgfsetdash{}{0pt}%
\pgfsys@defobject{currentmarker}{\pgfqpoint{0.000000in}{-0.048611in}}{\pgfqpoint{0.000000in}{0.000000in}}{%
\pgfpathmoveto{\pgfqpoint{0.000000in}{0.000000in}}%
\pgfpathlineto{\pgfqpoint{0.000000in}{-0.048611in}}%
\pgfusepath{stroke,fill}%
}%
\begin{pgfscope}%
\pgfsys@transformshift{2.947416in}{0.499691in}%
\pgfsys@useobject{currentmarker}{}%
\end{pgfscope}%
\end{pgfscope}%
\begin{pgfscope}%
\definecolor{textcolor}{rgb}{0.000000,0.000000,0.000000}%
\pgfsetstrokecolor{textcolor}%
\pgfsetfillcolor{textcolor}%
\pgftext[x=2.947416in,y=0.402469in,,top]{\color{textcolor}\rmfamily\fontsize{10.000000}{12.000000}\selectfont \(\displaystyle 3.1\)}%
\end{pgfscope}%
\begin{pgfscope}%
\pgfsetbuttcap%
\pgfsetroundjoin%
\definecolor{currentfill}{rgb}{0.000000,0.000000,0.000000}%
\pgfsetfillcolor{currentfill}%
\pgfsetlinewidth{0.803000pt}%
\definecolor{currentstroke}{rgb}{0.000000,0.000000,0.000000}%
\pgfsetstrokecolor{currentstroke}%
\pgfsetdash{}{0pt}%
\pgfsys@defobject{currentmarker}{\pgfqpoint{0.000000in}{-0.048611in}}{\pgfqpoint{0.000000in}{0.000000in}}{%
\pgfpathmoveto{\pgfqpoint{0.000000in}{0.000000in}}%
\pgfpathlineto{\pgfqpoint{0.000000in}{-0.048611in}}%
\pgfusepath{stroke,fill}%
}%
\begin{pgfscope}%
\pgfsys@transformshift{3.799502in}{0.499691in}%
\pgfsys@useobject{currentmarker}{}%
\end{pgfscope}%
\end{pgfscope}%
\begin{pgfscope}%
\definecolor{textcolor}{rgb}{0.000000,0.000000,0.000000}%
\pgfsetstrokecolor{textcolor}%
\pgfsetfillcolor{textcolor}%
\pgftext[x=3.799502in,y=0.402469in,,top]{\color{textcolor}\rmfamily\fontsize{10.000000}{12.000000}\selectfont \(\displaystyle 3.2\)}%
\end{pgfscope}%
\begin{pgfscope}%
\pgfsetbuttcap%
\pgfsetroundjoin%
\definecolor{currentfill}{rgb}{0.000000,0.000000,0.000000}%
\pgfsetfillcolor{currentfill}%
\pgfsetlinewidth{0.803000pt}%
\definecolor{currentstroke}{rgb}{0.000000,0.000000,0.000000}%
\pgfsetstrokecolor{currentstroke}%
\pgfsetdash{}{0pt}%
\pgfsys@defobject{currentmarker}{\pgfqpoint{0.000000in}{-0.048611in}}{\pgfqpoint{0.000000in}{0.000000in}}{%
\pgfpathmoveto{\pgfqpoint{0.000000in}{0.000000in}}%
\pgfpathlineto{\pgfqpoint{0.000000in}{-0.048611in}}%
\pgfusepath{stroke,fill}%
}%
\begin{pgfscope}%
\pgfsys@transformshift{4.651588in}{0.499691in}%
\pgfsys@useobject{currentmarker}{}%
\end{pgfscope}%
\end{pgfscope}%
\begin{pgfscope}%
\definecolor{textcolor}{rgb}{0.000000,0.000000,0.000000}%
\pgfsetstrokecolor{textcolor}%
\pgfsetfillcolor{textcolor}%
\pgftext[x=4.651588in,y=0.402469in,,top]{\color{textcolor}\rmfamily\fontsize{10.000000}{12.000000}\selectfont \(\displaystyle 3.3\)}%
\end{pgfscope}%
\begin{pgfscope}%
\definecolor{textcolor}{rgb}{0.000000,0.000000,0.000000}%
\pgfsetstrokecolor{textcolor}%
\pgfsetfillcolor{textcolor}%
\pgftext[x=2.775155in,y=0.223457in,,top]{\color{textcolor}\rmfamily\fontsize{10.000000}{12.000000}\selectfont log f}%
\end{pgfscope}%
\begin{pgfscope}%
\pgfsetbuttcap%
\pgfsetroundjoin%
\definecolor{currentfill}{rgb}{0.000000,0.000000,0.000000}%
\pgfsetfillcolor{currentfill}%
\pgfsetlinewidth{0.803000pt}%
\definecolor{currentstroke}{rgb}{0.000000,0.000000,0.000000}%
\pgfsetstrokecolor{currentstroke}%
\pgfsetdash{}{0pt}%
\pgfsys@defobject{currentmarker}{\pgfqpoint{-0.048611in}{0.000000in}}{\pgfqpoint{0.000000in}{0.000000in}}{%
\pgfpathmoveto{\pgfqpoint{0.000000in}{0.000000in}}%
\pgfpathlineto{\pgfqpoint{-0.048611in}{0.000000in}}%
\pgfusepath{stroke,fill}%
}%
\begin{pgfscope}%
\pgfsys@transformshift{0.837655in}{0.921177in}%
\pgfsys@useobject{currentmarker}{}%
\end{pgfscope}%
\end{pgfscope}%
\begin{pgfscope}%
\definecolor{textcolor}{rgb}{0.000000,0.000000,0.000000}%
\pgfsetstrokecolor{textcolor}%
\pgfsetfillcolor{textcolor}%
\pgftext[x=0.601544in,y=0.872951in,left,base]{\color{textcolor}\rmfamily\fontsize{10.000000}{12.000000}\selectfont \(\displaystyle 78\)}%
\end{pgfscope}%
\begin{pgfscope}%
\pgfsetbuttcap%
\pgfsetroundjoin%
\definecolor{currentfill}{rgb}{0.000000,0.000000,0.000000}%
\pgfsetfillcolor{currentfill}%
\pgfsetlinewidth{0.803000pt}%
\definecolor{currentstroke}{rgb}{0.000000,0.000000,0.000000}%
\pgfsetstrokecolor{currentstroke}%
\pgfsetdash{}{0pt}%
\pgfsys@defobject{currentmarker}{\pgfqpoint{-0.048611in}{0.000000in}}{\pgfqpoint{0.000000in}{0.000000in}}{%
\pgfpathmoveto{\pgfqpoint{0.000000in}{0.000000in}}%
\pgfpathlineto{\pgfqpoint{-0.048611in}{0.000000in}}%
\pgfusepath{stroke,fill}%
}%
\begin{pgfscope}%
\pgfsys@transformshift{0.837655in}{1.453988in}%
\pgfsys@useobject{currentmarker}{}%
\end{pgfscope}%
\end{pgfscope}%
\begin{pgfscope}%
\definecolor{textcolor}{rgb}{0.000000,0.000000,0.000000}%
\pgfsetstrokecolor{textcolor}%
\pgfsetfillcolor{textcolor}%
\pgftext[x=0.601544in,y=1.405763in,left,base]{\color{textcolor}\rmfamily\fontsize{10.000000}{12.000000}\selectfont \(\displaystyle 80\)}%
\end{pgfscope}%
\begin{pgfscope}%
\pgfsetbuttcap%
\pgfsetroundjoin%
\definecolor{currentfill}{rgb}{0.000000,0.000000,0.000000}%
\pgfsetfillcolor{currentfill}%
\pgfsetlinewidth{0.803000pt}%
\definecolor{currentstroke}{rgb}{0.000000,0.000000,0.000000}%
\pgfsetstrokecolor{currentstroke}%
\pgfsetdash{}{0pt}%
\pgfsys@defobject{currentmarker}{\pgfqpoint{-0.048611in}{0.000000in}}{\pgfqpoint{0.000000in}{0.000000in}}{%
\pgfpathmoveto{\pgfqpoint{0.000000in}{0.000000in}}%
\pgfpathlineto{\pgfqpoint{-0.048611in}{0.000000in}}%
\pgfusepath{stroke,fill}%
}%
\begin{pgfscope}%
\pgfsys@transformshift{0.837655in}{1.986800in}%
\pgfsys@useobject{currentmarker}{}%
\end{pgfscope}%
\end{pgfscope}%
\begin{pgfscope}%
\definecolor{textcolor}{rgb}{0.000000,0.000000,0.000000}%
\pgfsetstrokecolor{textcolor}%
\pgfsetfillcolor{textcolor}%
\pgftext[x=0.601544in,y=1.938574in,left,base]{\color{textcolor}\rmfamily\fontsize{10.000000}{12.000000}\selectfont \(\displaystyle 82\)}%
\end{pgfscope}%
\begin{pgfscope}%
\pgfsetbuttcap%
\pgfsetroundjoin%
\definecolor{currentfill}{rgb}{0.000000,0.000000,0.000000}%
\pgfsetfillcolor{currentfill}%
\pgfsetlinewidth{0.803000pt}%
\definecolor{currentstroke}{rgb}{0.000000,0.000000,0.000000}%
\pgfsetstrokecolor{currentstroke}%
\pgfsetdash{}{0pt}%
\pgfsys@defobject{currentmarker}{\pgfqpoint{-0.048611in}{0.000000in}}{\pgfqpoint{0.000000in}{0.000000in}}{%
\pgfpathmoveto{\pgfqpoint{0.000000in}{0.000000in}}%
\pgfpathlineto{\pgfqpoint{-0.048611in}{0.000000in}}%
\pgfusepath{stroke,fill}%
}%
\begin{pgfscope}%
\pgfsys@transformshift{0.837655in}{2.519611in}%
\pgfsys@useobject{currentmarker}{}%
\end{pgfscope}%
\end{pgfscope}%
\begin{pgfscope}%
\definecolor{textcolor}{rgb}{0.000000,0.000000,0.000000}%
\pgfsetstrokecolor{textcolor}%
\pgfsetfillcolor{textcolor}%
\pgftext[x=0.601544in,y=2.471386in,left,base]{\color{textcolor}\rmfamily\fontsize{10.000000}{12.000000}\selectfont \(\displaystyle 84\)}%
\end{pgfscope}%
\begin{pgfscope}%
\pgfsetbuttcap%
\pgfsetroundjoin%
\definecolor{currentfill}{rgb}{0.000000,0.000000,0.000000}%
\pgfsetfillcolor{currentfill}%
\pgfsetlinewidth{0.803000pt}%
\definecolor{currentstroke}{rgb}{0.000000,0.000000,0.000000}%
\pgfsetstrokecolor{currentstroke}%
\pgfsetdash{}{0pt}%
\pgfsys@defobject{currentmarker}{\pgfqpoint{-0.048611in}{0.000000in}}{\pgfqpoint{0.000000in}{0.000000in}}{%
\pgfpathmoveto{\pgfqpoint{0.000000in}{0.000000in}}%
\pgfpathlineto{\pgfqpoint{-0.048611in}{0.000000in}}%
\pgfusepath{stroke,fill}%
}%
\begin{pgfscope}%
\pgfsys@transformshift{0.837655in}{3.052422in}%
\pgfsys@useobject{currentmarker}{}%
\end{pgfscope}%
\end{pgfscope}%
\begin{pgfscope}%
\definecolor{textcolor}{rgb}{0.000000,0.000000,0.000000}%
\pgfsetstrokecolor{textcolor}%
\pgfsetfillcolor{textcolor}%
\pgftext[x=0.601544in,y=3.004197in,left,base]{\color{textcolor}\rmfamily\fontsize{10.000000}{12.000000}\selectfont \(\displaystyle 86\)}%
\end{pgfscope}%
\begin{pgfscope}%
\definecolor{textcolor}{rgb}{0.000000,0.000000,0.000000}%
\pgfsetstrokecolor{textcolor}%
\pgfsetfillcolor{textcolor}%
\pgftext[x=0.323766in,y=1.847191in,,bottom]{\color{textcolor}\rmfamily\fontsize{10.000000}{12.000000}\selectfont 20log Z}%
\end{pgfscope}%
\begin{pgfscope}%
\pgfpathrectangle{\pgfqpoint{0.837655in}{0.499691in}}{\pgfqpoint{3.875000in}{2.695000in}}%
\pgfusepath{clip}%
\pgfsetbuttcap%
\pgfsetroundjoin%
\definecolor{currentfill}{rgb}{0.121569,0.466667,0.705882}%
\pgfsetfillcolor{currentfill}%
\pgfsetlinewidth{1.003750pt}%
\definecolor{currentstroke}{rgb}{0.121569,0.466667,0.705882}%
\pgfsetstrokecolor{currentstroke}%
\pgfsetdash{}{0pt}%
\pgfpathmoveto{\pgfqpoint{1.079760in}{3.027559in}}%
\pgfpathcurveto{\pgfqpoint{1.090810in}{3.027559in}}{\pgfqpoint{1.101409in}{3.031949in}}{\pgfqpoint{1.109222in}{3.039763in}}%
\pgfpathcurveto{\pgfqpoint{1.117036in}{3.047576in}}{\pgfqpoint{1.121426in}{3.058175in}}{\pgfqpoint{1.121426in}{3.069226in}}%
\pgfpathcurveto{\pgfqpoint{1.121426in}{3.080276in}}{\pgfqpoint{1.117036in}{3.090875in}}{\pgfqpoint{1.109222in}{3.098688in}}%
\pgfpathcurveto{\pgfqpoint{1.101409in}{3.106502in}}{\pgfqpoint{1.090810in}{3.110892in}}{\pgfqpoint{1.079760in}{3.110892in}}%
\pgfpathcurveto{\pgfqpoint{1.068709in}{3.110892in}}{\pgfqpoint{1.058110in}{3.106502in}}{\pgfqpoint{1.050297in}{3.098688in}}%
\pgfpathcurveto{\pgfqpoint{1.042483in}{3.090875in}}{\pgfqpoint{1.038093in}{3.080276in}}{\pgfqpoint{1.038093in}{3.069226in}}%
\pgfpathcurveto{\pgfqpoint{1.038093in}{3.058175in}}{\pgfqpoint{1.042483in}{3.047576in}}{\pgfqpoint{1.050297in}{3.039763in}}%
\pgfpathcurveto{\pgfqpoint{1.058110in}{3.031949in}}{\pgfqpoint{1.068709in}{3.027559in}}{\pgfqpoint{1.079760in}{3.027559in}}%
\pgfpathclose%
\pgfusepath{stroke,fill}%
\end{pgfscope}%
\begin{pgfscope}%
\pgfpathrectangle{\pgfqpoint{0.837655in}{0.499691in}}{\pgfqpoint{3.875000in}{2.695000in}}%
\pgfusepath{clip}%
\pgfsetbuttcap%
\pgfsetroundjoin%
\definecolor{currentfill}{rgb}{0.121569,0.466667,0.705882}%
\pgfsetfillcolor{currentfill}%
\pgfsetlinewidth{1.003750pt}%
\definecolor{currentstroke}{rgb}{0.121569,0.466667,0.705882}%
\pgfsetstrokecolor{currentstroke}%
\pgfsetdash{}{0pt}%
\pgfpathmoveto{\pgfqpoint{1.360950in}{2.895296in}}%
\pgfpathcurveto{\pgfqpoint{1.372000in}{2.895296in}}{\pgfqpoint{1.382599in}{2.899686in}}{\pgfqpoint{1.390413in}{2.907500in}}%
\pgfpathcurveto{\pgfqpoint{1.398227in}{2.915314in}}{\pgfqpoint{1.402617in}{2.925913in}}{\pgfqpoint{1.402617in}{2.936963in}}%
\pgfpathcurveto{\pgfqpoint{1.402617in}{2.948013in}}{\pgfqpoint{1.398227in}{2.958612in}}{\pgfqpoint{1.390413in}{2.966425in}}%
\pgfpathcurveto{\pgfqpoint{1.382599in}{2.974239in}}{\pgfqpoint{1.372000in}{2.978629in}}{\pgfqpoint{1.360950in}{2.978629in}}%
\pgfpathcurveto{\pgfqpoint{1.349900in}{2.978629in}}{\pgfqpoint{1.339301in}{2.974239in}}{\pgfqpoint{1.331487in}{2.966425in}}%
\pgfpathcurveto{\pgfqpoint{1.323674in}{2.958612in}}{\pgfqpoint{1.319283in}{2.948013in}}{\pgfqpoint{1.319283in}{2.936963in}}%
\pgfpathcurveto{\pgfqpoint{1.319283in}{2.925913in}}{\pgfqpoint{1.323674in}{2.915314in}}{\pgfqpoint{1.331487in}{2.907500in}}%
\pgfpathcurveto{\pgfqpoint{1.339301in}{2.899686in}}{\pgfqpoint{1.349900in}{2.895296in}}{\pgfqpoint{1.360950in}{2.895296in}}%
\pgfpathclose%
\pgfusepath{stroke,fill}%
\end{pgfscope}%
\begin{pgfscope}%
\pgfpathrectangle{\pgfqpoint{0.837655in}{0.499691in}}{\pgfqpoint{3.875000in}{2.695000in}}%
\pgfusepath{clip}%
\pgfsetbuttcap%
\pgfsetroundjoin%
\definecolor{currentfill}{rgb}{0.121569,0.466667,0.705882}%
\pgfsetfillcolor{currentfill}%
\pgfsetlinewidth{1.003750pt}%
\definecolor{currentstroke}{rgb}{0.121569,0.466667,0.705882}%
\pgfsetstrokecolor{currentstroke}%
\pgfsetdash{}{0pt}%
\pgfpathmoveto{\pgfqpoint{1.622275in}{2.811142in}}%
\pgfpathcurveto{\pgfqpoint{1.633325in}{2.811142in}}{\pgfqpoint{1.643924in}{2.815533in}}{\pgfqpoint{1.651738in}{2.823346in}}%
\pgfpathcurveto{\pgfqpoint{1.659551in}{2.831160in}}{\pgfqpoint{1.663942in}{2.841759in}}{\pgfqpoint{1.663942in}{2.852809in}}%
\pgfpathcurveto{\pgfqpoint{1.663942in}{2.863859in}}{\pgfqpoint{1.659551in}{2.874458in}}{\pgfqpoint{1.651738in}{2.882272in}}%
\pgfpathcurveto{\pgfqpoint{1.643924in}{2.890085in}}{\pgfqpoint{1.633325in}{2.894476in}}{\pgfqpoint{1.622275in}{2.894476in}}%
\pgfpathcurveto{\pgfqpoint{1.611225in}{2.894476in}}{\pgfqpoint{1.600626in}{2.890085in}}{\pgfqpoint{1.592812in}{2.882272in}}%
\pgfpathcurveto{\pgfqpoint{1.584998in}{2.874458in}}{\pgfqpoint{1.580608in}{2.863859in}}{\pgfqpoint{1.580608in}{2.852809in}}%
\pgfpathcurveto{\pgfqpoint{1.580608in}{2.841759in}}{\pgfqpoint{1.584998in}{2.831160in}}{\pgfqpoint{1.592812in}{2.823346in}}%
\pgfpathcurveto{\pgfqpoint{1.600626in}{2.815533in}}{\pgfqpoint{1.611225in}{2.811142in}}{\pgfqpoint{1.622275in}{2.811142in}}%
\pgfpathclose%
\pgfusepath{stroke,fill}%
\end{pgfscope}%
\begin{pgfscope}%
\pgfpathrectangle{\pgfqpoint{0.837655in}{0.499691in}}{\pgfqpoint{3.875000in}{2.695000in}}%
\pgfusepath{clip}%
\pgfsetbuttcap%
\pgfsetroundjoin%
\definecolor{currentfill}{rgb}{0.121569,0.466667,0.705882}%
\pgfsetfillcolor{currentfill}%
\pgfsetlinewidth{1.003750pt}%
\definecolor{currentstroke}{rgb}{0.121569,0.466667,0.705882}%
\pgfsetstrokecolor{currentstroke}%
\pgfsetdash{}{0pt}%
\pgfpathmoveto{\pgfqpoint{1.866356in}{2.690386in}}%
\pgfpathcurveto{\pgfqpoint{1.877407in}{2.690386in}}{\pgfqpoint{1.888006in}{2.694776in}}{\pgfqpoint{1.895819in}{2.702590in}}%
\pgfpathcurveto{\pgfqpoint{1.903633in}{2.710404in}}{\pgfqpoint{1.908023in}{2.721003in}}{\pgfqpoint{1.908023in}{2.732053in}}%
\pgfpathcurveto{\pgfqpoint{1.908023in}{2.743103in}}{\pgfqpoint{1.903633in}{2.753702in}}{\pgfqpoint{1.895819in}{2.761516in}}%
\pgfpathcurveto{\pgfqpoint{1.888006in}{2.769329in}}{\pgfqpoint{1.877407in}{2.773719in}}{\pgfqpoint{1.866356in}{2.773719in}}%
\pgfpathcurveto{\pgfqpoint{1.855306in}{2.773719in}}{\pgfqpoint{1.844707in}{2.769329in}}{\pgfqpoint{1.836894in}{2.761516in}}%
\pgfpathcurveto{\pgfqpoint{1.829080in}{2.753702in}}{\pgfqpoint{1.824690in}{2.743103in}}{\pgfqpoint{1.824690in}{2.732053in}}%
\pgfpathcurveto{\pgfqpoint{1.824690in}{2.721003in}}{\pgfqpoint{1.829080in}{2.710404in}}{\pgfqpoint{1.836894in}{2.702590in}}%
\pgfpathcurveto{\pgfqpoint{1.844707in}{2.694776in}}{\pgfqpoint{1.855306in}{2.690386in}}{\pgfqpoint{1.866356in}{2.690386in}}%
\pgfpathclose%
\pgfusepath{stroke,fill}%
\end{pgfscope}%
\begin{pgfscope}%
\pgfpathrectangle{\pgfqpoint{0.837655in}{0.499691in}}{\pgfqpoint{3.875000in}{2.695000in}}%
\pgfusepath{clip}%
\pgfsetbuttcap%
\pgfsetroundjoin%
\definecolor{currentfill}{rgb}{0.121569,0.466667,0.705882}%
\pgfsetfillcolor{currentfill}%
\pgfsetlinewidth{1.003750pt}%
\definecolor{currentstroke}{rgb}{0.121569,0.466667,0.705882}%
\pgfsetstrokecolor{currentstroke}%
\pgfsetdash{}{0pt}%
\pgfpathmoveto{\pgfqpoint{2.095330in}{2.613247in}}%
\pgfpathcurveto{\pgfqpoint{2.106380in}{2.613247in}}{\pgfqpoint{2.116979in}{2.617637in}}{\pgfqpoint{2.124793in}{2.625450in}}%
\pgfpathcurveto{\pgfqpoint{2.132607in}{2.633264in}}{\pgfqpoint{2.136997in}{2.643863in}}{\pgfqpoint{2.136997in}{2.654913in}}%
\pgfpathcurveto{\pgfqpoint{2.136997in}{2.665963in}}{\pgfqpoint{2.132607in}{2.676562in}}{\pgfqpoint{2.124793in}{2.684376in}}%
\pgfpathcurveto{\pgfqpoint{2.116979in}{2.692190in}}{\pgfqpoint{2.106380in}{2.696580in}}{\pgfqpoint{2.095330in}{2.696580in}}%
\pgfpathcurveto{\pgfqpoint{2.084280in}{2.696580in}}{\pgfqpoint{2.073681in}{2.692190in}}{\pgfqpoint{2.065868in}{2.684376in}}%
\pgfpathcurveto{\pgfqpoint{2.058054in}{2.676562in}}{\pgfqpoint{2.053664in}{2.665963in}}{\pgfqpoint{2.053664in}{2.654913in}}%
\pgfpathcurveto{\pgfqpoint{2.053664in}{2.643863in}}{\pgfqpoint{2.058054in}{2.633264in}}{\pgfqpoint{2.065868in}{2.625450in}}%
\pgfpathcurveto{\pgfqpoint{2.073681in}{2.617637in}}{\pgfqpoint{2.084280in}{2.613247in}}{\pgfqpoint{2.095330in}{2.613247in}}%
\pgfpathclose%
\pgfusepath{stroke,fill}%
\end{pgfscope}%
\begin{pgfscope}%
\pgfpathrectangle{\pgfqpoint{0.837655in}{0.499691in}}{\pgfqpoint{3.875000in}{2.695000in}}%
\pgfusepath{clip}%
\pgfsetbuttcap%
\pgfsetroundjoin%
\definecolor{currentfill}{rgb}{0.121569,0.466667,0.705882}%
\pgfsetfillcolor{currentfill}%
\pgfsetlinewidth{1.003750pt}%
\definecolor{currentstroke}{rgb}{0.121569,0.466667,0.705882}%
\pgfsetstrokecolor{currentstroke}%
\pgfsetdash{}{0pt}%
\pgfpathmoveto{\pgfqpoint{2.310958in}{2.502155in}}%
\pgfpathcurveto{\pgfqpoint{2.322008in}{2.502155in}}{\pgfqpoint{2.332607in}{2.506545in}}{\pgfqpoint{2.340421in}{2.514359in}}%
\pgfpathcurveto{\pgfqpoint{2.348234in}{2.522172in}}{\pgfqpoint{2.352625in}{2.532771in}}{\pgfqpoint{2.352625in}{2.543821in}}%
\pgfpathcurveto{\pgfqpoint{2.352625in}{2.554871in}}{\pgfqpoint{2.348234in}{2.565470in}}{\pgfqpoint{2.340421in}{2.573284in}}%
\pgfpathcurveto{\pgfqpoint{2.332607in}{2.581098in}}{\pgfqpoint{2.322008in}{2.585488in}}{\pgfqpoint{2.310958in}{2.585488in}}%
\pgfpathcurveto{\pgfqpoint{2.299908in}{2.585488in}}{\pgfqpoint{2.289309in}{2.581098in}}{\pgfqpoint{2.281495in}{2.573284in}}%
\pgfpathcurveto{\pgfqpoint{2.273682in}{2.565470in}}{\pgfqpoint{2.269291in}{2.554871in}}{\pgfqpoint{2.269291in}{2.543821in}}%
\pgfpathcurveto{\pgfqpoint{2.269291in}{2.532771in}}{\pgfqpoint{2.273682in}{2.522172in}}{\pgfqpoint{2.281495in}{2.514359in}}%
\pgfpathcurveto{\pgfqpoint{2.289309in}{2.506545in}}{\pgfqpoint{2.299908in}{2.502155in}}{\pgfqpoint{2.310958in}{2.502155in}}%
\pgfpathclose%
\pgfusepath{stroke,fill}%
\end{pgfscope}%
\begin{pgfscope}%
\pgfpathrectangle{\pgfqpoint{0.837655in}{0.499691in}}{\pgfqpoint{3.875000in}{2.695000in}}%
\pgfusepath{clip}%
\pgfsetbuttcap%
\pgfsetroundjoin%
\definecolor{currentfill}{rgb}{0.121569,0.466667,0.705882}%
\pgfsetfillcolor{currentfill}%
\pgfsetlinewidth{1.003750pt}%
\definecolor{currentstroke}{rgb}{0.121569,0.466667,0.705882}%
\pgfsetstrokecolor{currentstroke}%
\pgfsetdash{}{0pt}%
\pgfpathmoveto{\pgfqpoint{2.514710in}{2.430950in}}%
\pgfpathcurveto{\pgfqpoint{2.525760in}{2.430950in}}{\pgfqpoint{2.536359in}{2.435340in}}{\pgfqpoint{2.544173in}{2.443154in}}%
\pgfpathcurveto{\pgfqpoint{2.551987in}{2.450967in}}{\pgfqpoint{2.556377in}{2.461566in}}{\pgfqpoint{2.556377in}{2.472617in}}%
\pgfpathcurveto{\pgfqpoint{2.556377in}{2.483667in}}{\pgfqpoint{2.551987in}{2.494266in}}{\pgfqpoint{2.544173in}{2.502079in}}%
\pgfpathcurveto{\pgfqpoint{2.536359in}{2.509893in}}{\pgfqpoint{2.525760in}{2.514283in}}{\pgfqpoint{2.514710in}{2.514283in}}%
\pgfpathcurveto{\pgfqpoint{2.503660in}{2.514283in}}{\pgfqpoint{2.493061in}{2.509893in}}{\pgfqpoint{2.485247in}{2.502079in}}%
\pgfpathcurveto{\pgfqpoint{2.477434in}{2.494266in}}{\pgfqpoint{2.473044in}{2.483667in}}{\pgfqpoint{2.473044in}{2.472617in}}%
\pgfpathcurveto{\pgfqpoint{2.473044in}{2.461566in}}{\pgfqpoint{2.477434in}{2.450967in}}{\pgfqpoint{2.485247in}{2.443154in}}%
\pgfpathcurveto{\pgfqpoint{2.493061in}{2.435340in}}{\pgfqpoint{2.503660in}{2.430950in}}{\pgfqpoint{2.514710in}{2.430950in}}%
\pgfpathclose%
\pgfusepath{stroke,fill}%
\end{pgfscope}%
\begin{pgfscope}%
\pgfpathrectangle{\pgfqpoint{0.837655in}{0.499691in}}{\pgfqpoint{3.875000in}{2.695000in}}%
\pgfusepath{clip}%
\pgfsetbuttcap%
\pgfsetroundjoin%
\definecolor{currentfill}{rgb}{0.121569,0.466667,0.705882}%
\pgfsetfillcolor{currentfill}%
\pgfsetlinewidth{1.003750pt}%
\definecolor{currentstroke}{rgb}{0.121569,0.466667,0.705882}%
\pgfsetstrokecolor{currentstroke}%
\pgfsetdash{}{0pt}%
\pgfpathmoveto{\pgfqpoint{2.707827in}{2.378948in}}%
\pgfpathcurveto{\pgfqpoint{2.718877in}{2.378948in}}{\pgfqpoint{2.729476in}{2.383339in}}{\pgfqpoint{2.737290in}{2.391152in}}%
\pgfpathcurveto{\pgfqpoint{2.745103in}{2.398966in}}{\pgfqpoint{2.749494in}{2.409565in}}{\pgfqpoint{2.749494in}{2.420615in}}%
\pgfpathcurveto{\pgfqpoint{2.749494in}{2.431665in}}{\pgfqpoint{2.745103in}{2.442264in}}{\pgfqpoint{2.737290in}{2.450078in}}%
\pgfpathcurveto{\pgfqpoint{2.729476in}{2.457891in}}{\pgfqpoint{2.718877in}{2.462282in}}{\pgfqpoint{2.707827in}{2.462282in}}%
\pgfpathcurveto{\pgfqpoint{2.696777in}{2.462282in}}{\pgfqpoint{2.686178in}{2.457891in}}{\pgfqpoint{2.678364in}{2.450078in}}%
\pgfpathcurveto{\pgfqpoint{2.670550in}{2.442264in}}{\pgfqpoint{2.666160in}{2.431665in}}{\pgfqpoint{2.666160in}{2.420615in}}%
\pgfpathcurveto{\pgfqpoint{2.666160in}{2.409565in}}{\pgfqpoint{2.670550in}{2.398966in}}{\pgfqpoint{2.678364in}{2.391152in}}%
\pgfpathcurveto{\pgfqpoint{2.686178in}{2.383339in}}{\pgfqpoint{2.696777in}{2.378948in}}{\pgfqpoint{2.707827in}{2.378948in}}%
\pgfpathclose%
\pgfusepath{stroke,fill}%
\end{pgfscope}%
\begin{pgfscope}%
\pgfpathrectangle{\pgfqpoint{0.837655in}{0.499691in}}{\pgfqpoint{3.875000in}{2.695000in}}%
\pgfusepath{clip}%
\pgfsetbuttcap%
\pgfsetroundjoin%
\definecolor{currentfill}{rgb}{0.121569,0.466667,0.705882}%
\pgfsetfillcolor{currentfill}%
\pgfsetlinewidth{1.003750pt}%
\definecolor{currentstroke}{rgb}{0.121569,0.466667,0.705882}%
\pgfsetstrokecolor{currentstroke}%
\pgfsetdash{}{0pt}%
\pgfpathmoveto{\pgfqpoint{2.891363in}{2.311382in}}%
\pgfpathcurveto{\pgfqpoint{2.902414in}{2.311382in}}{\pgfqpoint{2.913013in}{2.315773in}}{\pgfqpoint{2.920826in}{2.323586in}}%
\pgfpathcurveto{\pgfqpoint{2.928640in}{2.331400in}}{\pgfqpoint{2.933030in}{2.341999in}}{\pgfqpoint{2.933030in}{2.353049in}}%
\pgfpathcurveto{\pgfqpoint{2.933030in}{2.364099in}}{\pgfqpoint{2.928640in}{2.374698in}}{\pgfqpoint{2.920826in}{2.382512in}}%
\pgfpathcurveto{\pgfqpoint{2.913013in}{2.390325in}}{\pgfqpoint{2.902414in}{2.394716in}}{\pgfqpoint{2.891363in}{2.394716in}}%
\pgfpathcurveto{\pgfqpoint{2.880313in}{2.394716in}}{\pgfqpoint{2.869714in}{2.390325in}}{\pgfqpoint{2.861901in}{2.382512in}}%
\pgfpathcurveto{\pgfqpoint{2.854087in}{2.374698in}}{\pgfqpoint{2.849697in}{2.364099in}}{\pgfqpoint{2.849697in}{2.353049in}}%
\pgfpathcurveto{\pgfqpoint{2.849697in}{2.341999in}}{\pgfqpoint{2.854087in}{2.331400in}}{\pgfqpoint{2.861901in}{2.323586in}}%
\pgfpathcurveto{\pgfqpoint{2.869714in}{2.315773in}}{\pgfqpoint{2.880313in}{2.311382in}}{\pgfqpoint{2.891363in}{2.311382in}}%
\pgfpathclose%
\pgfusepath{stroke,fill}%
\end{pgfscope}%
\begin{pgfscope}%
\pgfpathrectangle{\pgfqpoint{0.837655in}{0.499691in}}{\pgfqpoint{3.875000in}{2.695000in}}%
\pgfusepath{clip}%
\pgfsetbuttcap%
\pgfsetroundjoin%
\definecolor{currentfill}{rgb}{0.121569,0.466667,0.705882}%
\pgfsetfillcolor{currentfill}%
\pgfsetlinewidth{1.003750pt}%
\definecolor{currentstroke}{rgb}{0.121569,0.466667,0.705882}%
\pgfsetstrokecolor{currentstroke}%
\pgfsetdash{}{0pt}%
\pgfpathmoveto{\pgfqpoint{3.066226in}{2.294795in}}%
\pgfpathcurveto{\pgfqpoint{3.077276in}{2.294795in}}{\pgfqpoint{3.087875in}{2.299185in}}{\pgfqpoint{3.095689in}{2.306999in}}%
\pgfpathcurveto{\pgfqpoint{3.103502in}{2.314812in}}{\pgfqpoint{3.107892in}{2.325411in}}{\pgfqpoint{3.107892in}{2.336461in}}%
\pgfpathcurveto{\pgfqpoint{3.107892in}{2.347512in}}{\pgfqpoint{3.103502in}{2.358111in}}{\pgfqpoint{3.095689in}{2.365924in}}%
\pgfpathcurveto{\pgfqpoint{3.087875in}{2.373738in}}{\pgfqpoint{3.077276in}{2.378128in}}{\pgfqpoint{3.066226in}{2.378128in}}%
\pgfpathcurveto{\pgfqpoint{3.055176in}{2.378128in}}{\pgfqpoint{3.044577in}{2.373738in}}{\pgfqpoint{3.036763in}{2.365924in}}%
\pgfpathcurveto{\pgfqpoint{3.028949in}{2.358111in}}{\pgfqpoint{3.024559in}{2.347512in}}{\pgfqpoint{3.024559in}{2.336461in}}%
\pgfpathcurveto{\pgfqpoint{3.024559in}{2.325411in}}{\pgfqpoint{3.028949in}{2.314812in}}{\pgfqpoint{3.036763in}{2.306999in}}%
\pgfpathcurveto{\pgfqpoint{3.044577in}{2.299185in}}{\pgfqpoint{3.055176in}{2.294795in}}{\pgfqpoint{3.066226in}{2.294795in}}%
\pgfpathclose%
\pgfusepath{stroke,fill}%
\end{pgfscope}%
\begin{pgfscope}%
\pgfpathrectangle{\pgfqpoint{0.837655in}{0.499691in}}{\pgfqpoint{3.875000in}{2.695000in}}%
\pgfusepath{clip}%
\pgfsetbuttcap%
\pgfsetroundjoin%
\definecolor{currentfill}{rgb}{0.121569,0.466667,0.705882}%
\pgfsetfillcolor{currentfill}%
\pgfsetlinewidth{1.003750pt}%
\definecolor{currentstroke}{rgb}{0.121569,0.466667,0.705882}%
\pgfsetstrokecolor{currentstroke}%
\pgfsetdash{}{0pt}%
\pgfpathmoveto{\pgfqpoint{3.233197in}{2.229608in}}%
\pgfpathcurveto{\pgfqpoint{3.244247in}{2.229608in}}{\pgfqpoint{3.254846in}{2.233998in}}{\pgfqpoint{3.262660in}{2.241812in}}%
\pgfpathcurveto{\pgfqpoint{3.270473in}{2.249626in}}{\pgfqpoint{3.274863in}{2.260225in}}{\pgfqpoint{3.274863in}{2.271275in}}%
\pgfpathcurveto{\pgfqpoint{3.274863in}{2.282325in}}{\pgfqpoint{3.270473in}{2.292924in}}{\pgfqpoint{3.262660in}{2.300738in}}%
\pgfpathcurveto{\pgfqpoint{3.254846in}{2.308551in}}{\pgfqpoint{3.244247in}{2.312941in}}{\pgfqpoint{3.233197in}{2.312941in}}%
\pgfpathcurveto{\pgfqpoint{3.222147in}{2.312941in}}{\pgfqpoint{3.211548in}{2.308551in}}{\pgfqpoint{3.203734in}{2.300738in}}%
\pgfpathcurveto{\pgfqpoint{3.195920in}{2.292924in}}{\pgfqpoint{3.191530in}{2.282325in}}{\pgfqpoint{3.191530in}{2.271275in}}%
\pgfpathcurveto{\pgfqpoint{3.191530in}{2.260225in}}{\pgfqpoint{3.195920in}{2.249626in}}{\pgfqpoint{3.203734in}{2.241812in}}%
\pgfpathcurveto{\pgfqpoint{3.211548in}{2.233998in}}{\pgfqpoint{3.222147in}{2.229608in}}{\pgfqpoint{3.233197in}{2.229608in}}%
\pgfpathclose%
\pgfusepath{stroke,fill}%
\end{pgfscope}%
\begin{pgfscope}%
\pgfpathrectangle{\pgfqpoint{0.837655in}{0.499691in}}{\pgfqpoint{3.875000in}{2.695000in}}%
\pgfusepath{clip}%
\pgfsetbuttcap%
\pgfsetroundjoin%
\definecolor{currentfill}{rgb}{0.121569,0.466667,0.705882}%
\pgfsetfillcolor{currentfill}%
\pgfsetlinewidth{1.003750pt}%
\definecolor{currentstroke}{rgb}{0.121569,0.466667,0.705882}%
\pgfsetstrokecolor{currentstroke}%
\pgfsetdash{}{0pt}%
\pgfpathmoveto{\pgfqpoint{3.392958in}{2.197691in}}%
\pgfpathcurveto{\pgfqpoint{3.404008in}{2.197691in}}{\pgfqpoint{3.414607in}{2.202081in}}{\pgfqpoint{3.422421in}{2.209895in}}%
\pgfpathcurveto{\pgfqpoint{3.430234in}{2.217708in}}{\pgfqpoint{3.434625in}{2.228307in}}{\pgfqpoint{3.434625in}{2.239357in}}%
\pgfpathcurveto{\pgfqpoint{3.434625in}{2.250408in}}{\pgfqpoint{3.430234in}{2.261007in}}{\pgfqpoint{3.422421in}{2.268820in}}%
\pgfpathcurveto{\pgfqpoint{3.414607in}{2.276634in}}{\pgfqpoint{3.404008in}{2.281024in}}{\pgfqpoint{3.392958in}{2.281024in}}%
\pgfpathcurveto{\pgfqpoint{3.381908in}{2.281024in}}{\pgfqpoint{3.371309in}{2.276634in}}{\pgfqpoint{3.363495in}{2.268820in}}%
\pgfpathcurveto{\pgfqpoint{3.355682in}{2.261007in}}{\pgfqpoint{3.351291in}{2.250408in}}{\pgfqpoint{3.351291in}{2.239357in}}%
\pgfpathcurveto{\pgfqpoint{3.351291in}{2.228307in}}{\pgfqpoint{3.355682in}{2.217708in}}{\pgfqpoint{3.363495in}{2.209895in}}%
\pgfpathcurveto{\pgfqpoint{3.371309in}{2.202081in}}{\pgfqpoint{3.381908in}{2.197691in}}{\pgfqpoint{3.392958in}{2.197691in}}%
\pgfpathclose%
\pgfusepath{stroke,fill}%
\end{pgfscope}%
\begin{pgfscope}%
\pgfpathrectangle{\pgfqpoint{0.837655in}{0.499691in}}{\pgfqpoint{3.875000in}{2.695000in}}%
\pgfusepath{clip}%
\pgfsetbuttcap%
\pgfsetroundjoin%
\definecolor{currentfill}{rgb}{0.121569,0.466667,0.705882}%
\pgfsetfillcolor{currentfill}%
\pgfsetlinewidth{1.003750pt}%
\definecolor{currentstroke}{rgb}{0.121569,0.466667,0.705882}%
\pgfsetstrokecolor{currentstroke}%
\pgfsetdash{}{0pt}%
\pgfpathmoveto{\pgfqpoint{3.546107in}{2.166208in}}%
\pgfpathcurveto{\pgfqpoint{3.557157in}{2.166208in}}{\pgfqpoint{3.567756in}{2.170598in}}{\pgfqpoint{3.575569in}{2.178412in}}%
\pgfpathcurveto{\pgfqpoint{3.583383in}{2.186225in}}{\pgfqpoint{3.587773in}{2.196824in}}{\pgfqpoint{3.587773in}{2.207874in}}%
\pgfpathcurveto{\pgfqpoint{3.587773in}{2.218925in}}{\pgfqpoint{3.583383in}{2.229524in}}{\pgfqpoint{3.575569in}{2.237337in}}%
\pgfpathcurveto{\pgfqpoint{3.567756in}{2.245151in}}{\pgfqpoint{3.557157in}{2.249541in}}{\pgfqpoint{3.546107in}{2.249541in}}%
\pgfpathcurveto{\pgfqpoint{3.535057in}{2.249541in}}{\pgfqpoint{3.524457in}{2.245151in}}{\pgfqpoint{3.516644in}{2.237337in}}%
\pgfpathcurveto{\pgfqpoint{3.508830in}{2.229524in}}{\pgfqpoint{3.504440in}{2.218925in}}{\pgfqpoint{3.504440in}{2.207874in}}%
\pgfpathcurveto{\pgfqpoint{3.504440in}{2.196824in}}{\pgfqpoint{3.508830in}{2.186225in}}{\pgfqpoint{3.516644in}{2.178412in}}%
\pgfpathcurveto{\pgfqpoint{3.524457in}{2.170598in}}{\pgfqpoint{3.535057in}{2.166208in}}{\pgfqpoint{3.546107in}{2.166208in}}%
\pgfpathclose%
\pgfusepath{stroke,fill}%
\end{pgfscope}%
\begin{pgfscope}%
\pgfpathrectangle{\pgfqpoint{0.837655in}{0.499691in}}{\pgfqpoint{3.875000in}{2.695000in}}%
\pgfusepath{clip}%
\pgfsetbuttcap%
\pgfsetroundjoin%
\definecolor{currentfill}{rgb}{0.121569,0.466667,0.705882}%
\pgfsetfillcolor{currentfill}%
\pgfsetlinewidth{1.003750pt}%
\definecolor{currentstroke}{rgb}{0.121569,0.466667,0.705882}%
\pgfsetstrokecolor{currentstroke}%
\pgfsetdash{}{0pt}%
\pgfpathmoveto{\pgfqpoint{3.693168in}{2.104498in}}%
\pgfpathcurveto{\pgfqpoint{3.704218in}{2.104498in}}{\pgfqpoint{3.714817in}{2.108888in}}{\pgfqpoint{3.722631in}{2.116702in}}%
\pgfpathcurveto{\pgfqpoint{3.730445in}{2.124516in}}{\pgfqpoint{3.734835in}{2.135115in}}{\pgfqpoint{3.734835in}{2.146165in}}%
\pgfpathcurveto{\pgfqpoint{3.734835in}{2.157215in}}{\pgfqpoint{3.730445in}{2.167814in}}{\pgfqpoint{3.722631in}{2.175628in}}%
\pgfpathcurveto{\pgfqpoint{3.714817in}{2.183441in}}{\pgfqpoint{3.704218in}{2.187832in}}{\pgfqpoint{3.693168in}{2.187832in}}%
\pgfpathcurveto{\pgfqpoint{3.682118in}{2.187832in}}{\pgfqpoint{3.671519in}{2.183441in}}{\pgfqpoint{3.663705in}{2.175628in}}%
\pgfpathcurveto{\pgfqpoint{3.655892in}{2.167814in}}{\pgfqpoint{3.651502in}{2.157215in}}{\pgfqpoint{3.651502in}{2.146165in}}%
\pgfpathcurveto{\pgfqpoint{3.651502in}{2.135115in}}{\pgfqpoint{3.655892in}{2.124516in}}{\pgfqpoint{3.663705in}{2.116702in}}%
\pgfpathcurveto{\pgfqpoint{3.671519in}{2.108888in}}{\pgfqpoint{3.682118in}{2.104498in}}{\pgfqpoint{3.693168in}{2.104498in}}%
\pgfpathclose%
\pgfusepath{stroke,fill}%
\end{pgfscope}%
\begin{pgfscope}%
\pgfpathrectangle{\pgfqpoint{0.837655in}{0.499691in}}{\pgfqpoint{3.875000in}{2.695000in}}%
\pgfusepath{clip}%
\pgfsetbuttcap%
\pgfsetroundjoin%
\definecolor{currentfill}{rgb}{0.121569,0.466667,0.705882}%
\pgfsetfillcolor{currentfill}%
\pgfsetlinewidth{1.003750pt}%
\definecolor{currentstroke}{rgb}{0.121569,0.466667,0.705882}%
\pgfsetstrokecolor{currentstroke}%
\pgfsetdash{}{0pt}%
\pgfpathmoveto{\pgfqpoint{3.834608in}{2.044392in}}%
\pgfpathcurveto{\pgfqpoint{3.845658in}{2.044392in}}{\pgfqpoint{3.856257in}{2.048782in}}{\pgfqpoint{3.864071in}{2.056596in}}%
\pgfpathcurveto{\pgfqpoint{3.871885in}{2.064409in}}{\pgfqpoint{3.876275in}{2.075008in}}{\pgfqpoint{3.876275in}{2.086058in}}%
\pgfpathcurveto{\pgfqpoint{3.876275in}{2.097108in}}{\pgfqpoint{3.871885in}{2.107708in}}{\pgfqpoint{3.864071in}{2.115521in}}%
\pgfpathcurveto{\pgfqpoint{3.856257in}{2.123335in}}{\pgfqpoint{3.845658in}{2.127725in}}{\pgfqpoint{3.834608in}{2.127725in}}%
\pgfpathcurveto{\pgfqpoint{3.823558in}{2.127725in}}{\pgfqpoint{3.812959in}{2.123335in}}{\pgfqpoint{3.805145in}{2.115521in}}%
\pgfpathcurveto{\pgfqpoint{3.797332in}{2.107708in}}{\pgfqpoint{3.792942in}{2.097108in}}{\pgfqpoint{3.792942in}{2.086058in}}%
\pgfpathcurveto{\pgfqpoint{3.792942in}{2.075008in}}{\pgfqpoint{3.797332in}{2.064409in}}{\pgfqpoint{3.805145in}{2.056596in}}%
\pgfpathcurveto{\pgfqpoint{3.812959in}{2.048782in}}{\pgfqpoint{3.823558in}{2.044392in}}{\pgfqpoint{3.834608in}{2.044392in}}%
\pgfpathclose%
\pgfusepath{stroke,fill}%
\end{pgfscope}%
\begin{pgfscope}%
\pgfpathrectangle{\pgfqpoint{0.837655in}{0.499691in}}{\pgfqpoint{3.875000in}{2.695000in}}%
\pgfusepath{clip}%
\pgfsetbuttcap%
\pgfsetroundjoin%
\definecolor{currentfill}{rgb}{0.121569,0.466667,0.705882}%
\pgfsetfillcolor{currentfill}%
\pgfsetlinewidth{1.003750pt}%
\definecolor{currentstroke}{rgb}{0.121569,0.466667,0.705882}%
\pgfsetstrokecolor{currentstroke}%
\pgfsetdash{}{0pt}%
\pgfpathmoveto{\pgfqpoint{3.970841in}{2.044392in}}%
\pgfpathcurveto{\pgfqpoint{3.981891in}{2.044392in}}{\pgfqpoint{3.992490in}{2.048782in}}{\pgfqpoint{4.000303in}{2.056596in}}%
\pgfpathcurveto{\pgfqpoint{4.008117in}{2.064409in}}{\pgfqpoint{4.012507in}{2.075008in}}{\pgfqpoint{4.012507in}{2.086058in}}%
\pgfpathcurveto{\pgfqpoint{4.012507in}{2.097108in}}{\pgfqpoint{4.008117in}{2.107708in}}{\pgfqpoint{4.000303in}{2.115521in}}%
\pgfpathcurveto{\pgfqpoint{3.992490in}{2.123335in}}{\pgfqpoint{3.981891in}{2.127725in}}{\pgfqpoint{3.970841in}{2.127725in}}%
\pgfpathcurveto{\pgfqpoint{3.959791in}{2.127725in}}{\pgfqpoint{3.949191in}{2.123335in}}{\pgfqpoint{3.941378in}{2.115521in}}%
\pgfpathcurveto{\pgfqpoint{3.933564in}{2.107708in}}{\pgfqpoint{3.929174in}{2.097108in}}{\pgfqpoint{3.929174in}{2.086058in}}%
\pgfpathcurveto{\pgfqpoint{3.929174in}{2.075008in}}{\pgfqpoint{3.933564in}{2.064409in}}{\pgfqpoint{3.941378in}{2.056596in}}%
\pgfpathcurveto{\pgfqpoint{3.949191in}{2.048782in}}{\pgfqpoint{3.959791in}{2.044392in}}{\pgfqpoint{3.970841in}{2.044392in}}%
\pgfpathclose%
\pgfusepath{stroke,fill}%
\end{pgfscope}%
\begin{pgfscope}%
\pgfpathrectangle{\pgfqpoint{0.837655in}{0.499691in}}{\pgfqpoint{3.875000in}{2.695000in}}%
\pgfusepath{clip}%
\pgfsetbuttcap%
\pgfsetroundjoin%
\definecolor{currentfill}{rgb}{0.121569,0.466667,0.705882}%
\pgfsetfillcolor{currentfill}%
\pgfsetlinewidth{1.003750pt}%
\definecolor{currentstroke}{rgb}{0.121569,0.466667,0.705882}%
\pgfsetstrokecolor{currentstroke}%
\pgfsetdash{}{0pt}%
\pgfpathmoveto{\pgfqpoint{4.102235in}{1.985807in}}%
\pgfpathcurveto{\pgfqpoint{4.113286in}{1.985807in}}{\pgfqpoint{4.123885in}{1.990197in}}{\pgfqpoint{4.131698in}{1.998011in}}%
\pgfpathcurveto{\pgfqpoint{4.139512in}{2.005825in}}{\pgfqpoint{4.143902in}{2.016424in}}{\pgfqpoint{4.143902in}{2.027474in}}%
\pgfpathcurveto{\pgfqpoint{4.143902in}{2.038524in}}{\pgfqpoint{4.139512in}{2.049123in}}{\pgfqpoint{4.131698in}{2.056936in}}%
\pgfpathcurveto{\pgfqpoint{4.123885in}{2.064750in}}{\pgfqpoint{4.113286in}{2.069140in}}{\pgfqpoint{4.102235in}{2.069140in}}%
\pgfpathcurveto{\pgfqpoint{4.091185in}{2.069140in}}{\pgfqpoint{4.080586in}{2.064750in}}{\pgfqpoint{4.072773in}{2.056936in}}%
\pgfpathcurveto{\pgfqpoint{4.064959in}{2.049123in}}{\pgfqpoint{4.060569in}{2.038524in}}{\pgfqpoint{4.060569in}{2.027474in}}%
\pgfpathcurveto{\pgfqpoint{4.060569in}{2.016424in}}{\pgfqpoint{4.064959in}{2.005825in}}{\pgfqpoint{4.072773in}{1.998011in}}%
\pgfpathcurveto{\pgfqpoint{4.080586in}{1.990197in}}{\pgfqpoint{4.091185in}{1.985807in}}{\pgfqpoint{4.102235in}{1.985807in}}%
\pgfpathclose%
\pgfusepath{stroke,fill}%
\end{pgfscope}%
\begin{pgfscope}%
\pgfpathrectangle{\pgfqpoint{0.837655in}{0.499691in}}{\pgfqpoint{3.875000in}{2.695000in}}%
\pgfusepath{clip}%
\pgfsetbuttcap%
\pgfsetroundjoin%
\definecolor{currentfill}{rgb}{0.121569,0.466667,0.705882}%
\pgfsetfillcolor{currentfill}%
\pgfsetlinewidth{1.003750pt}%
\definecolor{currentstroke}{rgb}{0.121569,0.466667,0.705882}%
\pgfsetstrokecolor{currentstroke}%
\pgfsetdash{}{0pt}%
\pgfpathmoveto{\pgfqpoint{4.229124in}{1.985807in}}%
\pgfpathcurveto{\pgfqpoint{4.240174in}{1.985807in}}{\pgfqpoint{4.250773in}{1.990197in}}{\pgfqpoint{4.258587in}{1.998011in}}%
\pgfpathcurveto{\pgfqpoint{4.266401in}{2.005825in}}{\pgfqpoint{4.270791in}{2.016424in}}{\pgfqpoint{4.270791in}{2.027474in}}%
\pgfpathcurveto{\pgfqpoint{4.270791in}{2.038524in}}{\pgfqpoint{4.266401in}{2.049123in}}{\pgfqpoint{4.258587in}{2.056936in}}%
\pgfpathcurveto{\pgfqpoint{4.250773in}{2.064750in}}{\pgfqpoint{4.240174in}{2.069140in}}{\pgfqpoint{4.229124in}{2.069140in}}%
\pgfpathcurveto{\pgfqpoint{4.218074in}{2.069140in}}{\pgfqpoint{4.207475in}{2.064750in}}{\pgfqpoint{4.199661in}{2.056936in}}%
\pgfpathcurveto{\pgfqpoint{4.191848in}{2.049123in}}{\pgfqpoint{4.187458in}{2.038524in}}{\pgfqpoint{4.187458in}{2.027474in}}%
\pgfpathcurveto{\pgfqpoint{4.187458in}{2.016424in}}{\pgfqpoint{4.191848in}{2.005825in}}{\pgfqpoint{4.199661in}{1.998011in}}%
\pgfpathcurveto{\pgfqpoint{4.207475in}{1.990197in}}{\pgfqpoint{4.218074in}{1.985807in}}{\pgfqpoint{4.229124in}{1.985807in}}%
\pgfpathclose%
\pgfusepath{stroke,fill}%
\end{pgfscope}%
\begin{pgfscope}%
\pgfpathrectangle{\pgfqpoint{0.837655in}{0.499691in}}{\pgfqpoint{3.875000in}{2.695000in}}%
\pgfusepath{clip}%
\pgfsetbuttcap%
\pgfsetroundjoin%
\definecolor{currentfill}{rgb}{0.121569,0.466667,0.705882}%
\pgfsetfillcolor{currentfill}%
\pgfsetlinewidth{1.003750pt}%
\definecolor{currentstroke}{rgb}{0.121569,0.466667,0.705882}%
\pgfsetstrokecolor{currentstroke}%
\pgfsetdash{}{0pt}%
\pgfpathmoveto{\pgfqpoint{4.351806in}{1.928669in}}%
\pgfpathcurveto{\pgfqpoint{4.362856in}{1.928669in}}{\pgfqpoint{4.373455in}{1.933059in}}{\pgfqpoint{4.381269in}{1.940873in}}%
\pgfpathcurveto{\pgfqpoint{4.389082in}{1.948687in}}{\pgfqpoint{4.393473in}{1.959286in}}{\pgfqpoint{4.393473in}{1.970336in}}%
\pgfpathcurveto{\pgfqpoint{4.393473in}{1.981386in}}{\pgfqpoint{4.389082in}{1.991985in}}{\pgfqpoint{4.381269in}{1.999799in}}%
\pgfpathcurveto{\pgfqpoint{4.373455in}{2.007612in}}{\pgfqpoint{4.362856in}{2.012002in}}{\pgfqpoint{4.351806in}{2.012002in}}%
\pgfpathcurveto{\pgfqpoint{4.340756in}{2.012002in}}{\pgfqpoint{4.330157in}{2.007612in}}{\pgfqpoint{4.322343in}{1.999799in}}%
\pgfpathcurveto{\pgfqpoint{4.314530in}{1.991985in}}{\pgfqpoint{4.310139in}{1.981386in}}{\pgfqpoint{4.310139in}{1.970336in}}%
\pgfpathcurveto{\pgfqpoint{4.310139in}{1.959286in}}{\pgfqpoint{4.314530in}{1.948687in}}{\pgfqpoint{4.322343in}{1.940873in}}%
\pgfpathcurveto{\pgfqpoint{4.330157in}{1.933059in}}{\pgfqpoint{4.340756in}{1.928669in}}{\pgfqpoint{4.351806in}{1.928669in}}%
\pgfpathclose%
\pgfusepath{stroke,fill}%
\end{pgfscope}%
\begin{pgfscope}%
\pgfpathrectangle{\pgfqpoint{0.837655in}{0.499691in}}{\pgfqpoint{3.875000in}{2.695000in}}%
\pgfusepath{clip}%
\pgfsetbuttcap%
\pgfsetroundjoin%
\definecolor{currentfill}{rgb}{0.121569,0.466667,0.705882}%
\pgfsetfillcolor{currentfill}%
\pgfsetlinewidth{1.003750pt}%
\definecolor{currentstroke}{rgb}{0.121569,0.466667,0.705882}%
\pgfsetstrokecolor{currentstroke}%
\pgfsetdash{}{0pt}%
\pgfpathmoveto{\pgfqpoint{4.470551in}{1.928669in}}%
\pgfpathcurveto{\pgfqpoint{4.481601in}{1.928669in}}{\pgfqpoint{4.492200in}{1.933059in}}{\pgfqpoint{4.500014in}{1.940873in}}%
\pgfpathcurveto{\pgfqpoint{4.507827in}{1.948687in}}{\pgfqpoint{4.512218in}{1.959286in}}{\pgfqpoint{4.512218in}{1.970336in}}%
\pgfpathcurveto{\pgfqpoint{4.512218in}{1.981386in}}{\pgfqpoint{4.507827in}{1.991985in}}{\pgfqpoint{4.500014in}{1.999799in}}%
\pgfpathcurveto{\pgfqpoint{4.492200in}{2.007612in}}{\pgfqpoint{4.481601in}{2.012002in}}{\pgfqpoint{4.470551in}{2.012002in}}%
\pgfpathcurveto{\pgfqpoint{4.459501in}{2.012002in}}{\pgfqpoint{4.448902in}{2.007612in}}{\pgfqpoint{4.441088in}{1.999799in}}%
\pgfpathcurveto{\pgfqpoint{4.433274in}{1.991985in}}{\pgfqpoint{4.428884in}{1.981386in}}{\pgfqpoint{4.428884in}{1.970336in}}%
\pgfpathcurveto{\pgfqpoint{4.428884in}{1.959286in}}{\pgfqpoint{4.433274in}{1.948687in}}{\pgfqpoint{4.441088in}{1.940873in}}%
\pgfpathcurveto{\pgfqpoint{4.448902in}{1.933059in}}{\pgfqpoint{4.459501in}{1.928669in}}{\pgfqpoint{4.470551in}{1.928669in}}%
\pgfpathclose%
\pgfusepath{stroke,fill}%
\end{pgfscope}%
\begin{pgfscope}%
\pgfsetrectcap%
\pgfsetmiterjoin%
\pgfsetlinewidth{0.803000pt}%
\definecolor{currentstroke}{rgb}{0.000000,0.000000,0.000000}%
\pgfsetstrokecolor{currentstroke}%
\pgfsetdash{}{0pt}%
\pgfpathmoveto{\pgfqpoint{0.837655in}{0.499691in}}%
\pgfpathlineto{\pgfqpoint{0.837655in}{3.194691in}}%
\pgfusepath{stroke}%
\end{pgfscope}%
\begin{pgfscope}%
\pgfsetrectcap%
\pgfsetmiterjoin%
\pgfsetlinewidth{0.803000pt}%
\definecolor{currentstroke}{rgb}{0.000000,0.000000,0.000000}%
\pgfsetstrokecolor{currentstroke}%
\pgfsetdash{}{0pt}%
\pgfpathmoveto{\pgfqpoint{4.712655in}{0.499691in}}%
\pgfpathlineto{\pgfqpoint{4.712655in}{3.194691in}}%
\pgfusepath{stroke}%
\end{pgfscope}%
\begin{pgfscope}%
\pgfsetrectcap%
\pgfsetmiterjoin%
\pgfsetlinewidth{0.803000pt}%
\definecolor{currentstroke}{rgb}{0.000000,0.000000,0.000000}%
\pgfsetstrokecolor{currentstroke}%
\pgfsetdash{}{0pt}%
\pgfpathmoveto{\pgfqpoint{0.837655in}{0.499691in}}%
\pgfpathlineto{\pgfqpoint{4.712655in}{0.499691in}}%
\pgfusepath{stroke}%
\end{pgfscope}%
\begin{pgfscope}%
\pgfsetrectcap%
\pgfsetmiterjoin%
\pgfsetlinewidth{0.803000pt}%
\definecolor{currentstroke}{rgb}{0.000000,0.000000,0.000000}%
\pgfsetstrokecolor{currentstroke}%
\pgfsetdash{}{0pt}%
\pgfpathmoveto{\pgfqpoint{0.837655in}{3.194691in}}%
\pgfpathlineto{\pgfqpoint{4.712655in}{3.194691in}}%
\pgfusepath{stroke}%
\end{pgfscope}%
\begin{pgfscope}%
\pgfpathrectangle{\pgfqpoint{0.837655in}{0.499691in}}{\pgfqpoint{3.875000in}{2.695000in}}%
\pgfusepath{clip}%
\pgfsetbuttcap%
\pgfsetroundjoin%
\definecolor{currentfill}{rgb}{0.117647,0.564706,1.000000}%
\pgfsetfillcolor{currentfill}%
\pgfsetlinewidth{1.003750pt}%
\definecolor{currentstroke}{rgb}{0.117647,0.564706,1.000000}%
\pgfsetstrokecolor{currentstroke}%
\pgfsetdash{}{0pt}%
\pgfpathmoveto{\pgfqpoint{3.140319in}{1.412321in}}%
\pgfpathcurveto{\pgfqpoint{3.151369in}{1.412321in}}{\pgfqpoint{3.161968in}{1.416712in}}{\pgfqpoint{3.169782in}{1.424525in}}%
\pgfpathcurveto{\pgfqpoint{3.177596in}{1.432339in}}{\pgfqpoint{3.181986in}{1.442938in}}{\pgfqpoint{3.181986in}{1.453988in}}%
\pgfpathcurveto{\pgfqpoint{3.181986in}{1.465038in}}{\pgfqpoint{3.177596in}{1.475637in}}{\pgfqpoint{3.169782in}{1.483451in}}%
\pgfpathcurveto{\pgfqpoint{3.161968in}{1.491265in}}{\pgfqpoint{3.151369in}{1.495655in}}{\pgfqpoint{3.140319in}{1.495655in}}%
\pgfpathcurveto{\pgfqpoint{3.129269in}{1.495655in}}{\pgfqpoint{3.118670in}{1.491265in}}{\pgfqpoint{3.110856in}{1.483451in}}%
\pgfpathcurveto{\pgfqpoint{3.103043in}{1.475637in}}{\pgfqpoint{3.098652in}{1.465038in}}{\pgfqpoint{3.098652in}{1.453988in}}%
\pgfpathcurveto{\pgfqpoint{3.098652in}{1.442938in}}{\pgfqpoint{3.103043in}{1.432339in}}{\pgfqpoint{3.110856in}{1.424525in}}%
\pgfpathcurveto{\pgfqpoint{3.118670in}{1.416712in}}{\pgfqpoint{3.129269in}{1.412321in}}{\pgfqpoint{3.140319in}{1.412321in}}%
\pgfpathclose%
\pgfusepath{stroke,fill}%
\end{pgfscope}%
\end{pgfpicture}%
\makeatother%
\endgroup%

    %% Creator: Matplotlib, PGF backend
%%
%% To include the figure in your LaTeX document, write
%%   \input{<filename>.pgf}
%%
%% Make sure the required packages are loaded in your preamble
%%   \usepackage{pgf}
%%
%% Figures using additional raster images can only be included by \input if
%% they are in the same directory as the main LaTeX file. For loading figures
%% from other directories you can use the `import` package
%%   \usepackage{import}
%% and then include the figures with
%%   \import{<path to file>}{<filename>.pgf}
%%
%% Matplotlib used the following preamble
%%
\begingroup%
\makeatletter%
\begin{pgfpicture}%
\pgfpathrectangle{\pgfpointorigin}{\pgfqpoint{4.882100in}{3.294691in}}%
\pgfusepath{use as bounding box, clip}%
\begin{pgfscope}%
\pgfsetbuttcap%
\pgfsetmiterjoin%
\definecolor{currentfill}{rgb}{1.000000,1.000000,1.000000}%
\pgfsetfillcolor{currentfill}%
\pgfsetlinewidth{0.000000pt}%
\definecolor{currentstroke}{rgb}{1.000000,1.000000,1.000000}%
\pgfsetstrokecolor{currentstroke}%
\pgfsetdash{}{0pt}%
\pgfpathmoveto{\pgfqpoint{0.000000in}{0.000000in}}%
\pgfpathlineto{\pgfqpoint{4.882100in}{0.000000in}}%
\pgfpathlineto{\pgfqpoint{4.882100in}{3.294691in}}%
\pgfpathlineto{\pgfqpoint{0.000000in}{3.294691in}}%
\pgfpathclose%
\pgfusepath{fill}%
\end{pgfscope}%
\begin{pgfscope}%
\pgfsetbuttcap%
\pgfsetmiterjoin%
\definecolor{currentfill}{rgb}{1.000000,1.000000,1.000000}%
\pgfsetfillcolor{currentfill}%
\pgfsetlinewidth{0.000000pt}%
\definecolor{currentstroke}{rgb}{0.000000,0.000000,0.000000}%
\pgfsetstrokecolor{currentstroke}%
\pgfsetstrokeopacity{0.000000}%
\pgfsetdash{}{0pt}%
\pgfpathmoveto{\pgfqpoint{0.907100in}{0.499691in}}%
\pgfpathlineto{\pgfqpoint{4.782100in}{0.499691in}}%
\pgfpathlineto{\pgfqpoint{4.782100in}{3.194691in}}%
\pgfpathlineto{\pgfqpoint{0.907100in}{3.194691in}}%
\pgfpathclose%
\pgfusepath{fill}%
\end{pgfscope}%
\begin{pgfscope}%
\pgfpathrectangle{\pgfqpoint{0.907100in}{0.499691in}}{\pgfqpoint{3.875000in}{2.695000in}}%
\pgfusepath{clip}%
\pgfsetrectcap%
\pgfsetroundjoin%
\pgfsetlinewidth{1.505625pt}%
\definecolor{currentstroke}{rgb}{1.000000,0.388235,0.278431}%
\pgfsetstrokecolor{currentstroke}%
\pgfsetdash{}{0pt}%
\pgfpathmoveto{\pgfqpoint{1.092687in}{0.965122in}}%
\pgfpathlineto{\pgfqpoint{1.640586in}{0.965122in}}%
\pgfpathlineto{\pgfqpoint{2.011818in}{0.965122in}}%
\pgfpathlineto{\pgfqpoint{2.292949in}{0.965122in}}%
\pgfpathlineto{\pgfqpoint{2.519283in}{0.965122in}}%
\pgfpathlineto{\pgfqpoint{2.708737in}{0.965122in}}%
\pgfpathlineto{\pgfqpoint{2.871662in}{0.965122in}}%
\pgfpathlineto{\pgfqpoint{3.014584in}{0.965122in}}%
\pgfpathlineto{\pgfqpoint{3.141881in}{0.965122in}}%
\pgfpathlineto{\pgfqpoint{3.256636in}{0.965122in}}%
\pgfpathlineto{\pgfqpoint{3.361100in}{0.965122in}}%
\pgfpathlineto{\pgfqpoint{3.456968in}{0.965122in}}%
\pgfpathlineto{\pgfqpoint{3.545547in}{0.965122in}}%
\pgfpathlineto{\pgfqpoint{3.627869in}{0.965122in}}%
\pgfpathlineto{\pgfqpoint{3.704758in}{0.965122in}}%
\pgfpathlineto{\pgfqpoint{3.776888in}{0.965122in}}%
\pgfpathlineto{\pgfqpoint{3.844814in}{0.965122in}}%
\pgfpathlineto{\pgfqpoint{3.909000in}{0.965122in}}%
\pgfpathlineto{\pgfqpoint{3.969834in}{0.965122in}}%
\pgfpathlineto{\pgfqpoint{4.027651in}{0.965122in}}%
\pgfpathlineto{\pgfqpoint{4.082735in}{0.965122in}}%
\pgfpathlineto{\pgfqpoint{4.135334in}{0.965122in}}%
\pgfpathlineto{\pgfqpoint{4.185661in}{0.965122in}}%
\pgfpathlineto{\pgfqpoint{4.233905in}{0.965122in}}%
\pgfpathlineto{\pgfqpoint{4.280232in}{0.965122in}}%
\pgfpathlineto{\pgfqpoint{4.324788in}{0.965122in}}%
\pgfpathlineto{\pgfqpoint{4.367703in}{0.965122in}}%
\pgfpathlineto{\pgfqpoint{4.409094in}{0.965122in}}%
\pgfpathlineto{\pgfqpoint{4.449066in}{0.965122in}}%
\pgfpathlineto{\pgfqpoint{4.487713in}{0.965122in}}%
\pgfpathlineto{\pgfqpoint{4.525119in}{0.965122in}}%
\pgfpathlineto{\pgfqpoint{4.561363in}{0.965122in}}%
\pgfpathlineto{\pgfqpoint{4.596513in}{0.965122in}}%
\pgfusepath{stroke}%
\end{pgfscope}%
\begin{pgfscope}%
\pgfpathrectangle{\pgfqpoint{0.907100in}{0.499691in}}{\pgfqpoint{3.875000in}{2.695000in}}%
\pgfusepath{clip}%
\pgfsetrectcap%
\pgfsetroundjoin%
\pgfsetlinewidth{1.505625pt}%
\definecolor{currentstroke}{rgb}{1.000000,0.843137,0.000000}%
\pgfsetstrokecolor{currentstroke}%
\pgfsetdash{}{0pt}%
\pgfpathmoveto{\pgfqpoint{1.092687in}{3.072191in}}%
\pgfpathlineto{\pgfqpoint{1.640586in}{2.689081in}}%
\pgfpathlineto{\pgfqpoint{2.011818in}{2.429502in}}%
\pgfpathlineto{\pgfqpoint{2.292949in}{2.232925in}}%
\pgfpathlineto{\pgfqpoint{2.519283in}{2.074664in}}%
\pgfpathlineto{\pgfqpoint{2.708737in}{1.942191in}}%
\pgfpathlineto{\pgfqpoint{2.871662in}{1.828268in}}%
\pgfpathlineto{\pgfqpoint{3.014584in}{1.728332in}}%
\pgfpathlineto{\pgfqpoint{3.141881in}{1.639321in}}%
\pgfpathlineto{\pgfqpoint{3.256636in}{1.559081in}}%
\pgfpathlineto{\pgfqpoint{3.361100in}{1.486036in}}%
\pgfpathlineto{\pgfqpoint{3.456968in}{1.419002in}}%
\pgfpathlineto{\pgfqpoint{3.545547in}{1.357064in}}%
\pgfpathlineto{\pgfqpoint{3.627869in}{1.299502in}}%
\pgfpathlineto{\pgfqpoint{3.704758in}{1.245738in}}%
\pgfpathlineto{\pgfqpoint{3.776888in}{1.195302in}}%
\pgfpathlineto{\pgfqpoint{3.844814in}{1.147806in}}%
\pgfpathlineto{\pgfqpoint{3.909000in}{1.102925in}}%
\pgfpathlineto{\pgfqpoint{3.969834in}{1.060387in}}%
\pgfpathlineto{\pgfqpoint{4.027651in}{1.019960in}}%
\pgfpathlineto{\pgfqpoint{4.082735in}{0.981443in}}%
\pgfpathlineto{\pgfqpoint{4.135334in}{0.944664in}}%
\pgfpathlineto{\pgfqpoint{4.185661in}{0.909473in}}%
\pgfpathlineto{\pgfqpoint{4.233905in}{0.875739in}}%
\pgfpathlineto{\pgfqpoint{4.280232in}{0.843346in}}%
\pgfpathlineto{\pgfqpoint{4.324788in}{0.812191in}}%
\pgfpathlineto{\pgfqpoint{4.367703in}{0.782183in}}%
\pgfpathlineto{\pgfqpoint{4.409094in}{0.753241in}}%
\pgfpathlineto{\pgfqpoint{4.449066in}{0.725291in}}%
\pgfpathlineto{\pgfqpoint{4.487713in}{0.698268in}}%
\pgfpathlineto{\pgfqpoint{4.525119in}{0.672112in}}%
\pgfpathlineto{\pgfqpoint{4.561363in}{0.646770in}}%
\pgfpathlineto{\pgfqpoint{4.596513in}{0.622191in}}%
\pgfusepath{stroke}%
\end{pgfscope}%
\begin{pgfscope}%
\pgfpathrectangle{\pgfqpoint{0.907100in}{0.499691in}}{\pgfqpoint{3.875000in}{2.695000in}}%
\pgfusepath{clip}%
\pgfsetbuttcap%
\pgfsetroundjoin%
\definecolor{currentfill}{rgb}{0.529412,0.807843,0.921569}%
\pgfsetfillcolor{currentfill}%
\pgfsetlinewidth{1.003750pt}%
\definecolor{currentstroke}{rgb}{0.529412,0.807843,0.921569}%
\pgfsetstrokecolor{currentstroke}%
\pgfsetdash{}{0pt}%
\pgfpathmoveto{\pgfqpoint{1.092687in}{2.906354in}}%
\pgfpathcurveto{\pgfqpoint{1.103737in}{2.906354in}}{\pgfqpoint{1.114336in}{2.910744in}}{\pgfqpoint{1.122150in}{2.918558in}}%
\pgfpathcurveto{\pgfqpoint{1.129963in}{2.926372in}}{\pgfqpoint{1.134353in}{2.936971in}}{\pgfqpoint{1.134353in}{2.948021in}}%
\pgfpathcurveto{\pgfqpoint{1.134353in}{2.959071in}}{\pgfqpoint{1.129963in}{2.969670in}}{\pgfqpoint{1.122150in}{2.977484in}}%
\pgfpathcurveto{\pgfqpoint{1.114336in}{2.985297in}}{\pgfqpoint{1.103737in}{2.989687in}}{\pgfqpoint{1.092687in}{2.989687in}}%
\pgfpathcurveto{\pgfqpoint{1.081637in}{2.989687in}}{\pgfqpoint{1.071038in}{2.985297in}}{\pgfqpoint{1.063224in}{2.977484in}}%
\pgfpathcurveto{\pgfqpoint{1.055410in}{2.969670in}}{\pgfqpoint{1.051020in}{2.959071in}}{\pgfqpoint{1.051020in}{2.948021in}}%
\pgfpathcurveto{\pgfqpoint{1.051020in}{2.936971in}}{\pgfqpoint{1.055410in}{2.926372in}}{\pgfqpoint{1.063224in}{2.918558in}}%
\pgfpathcurveto{\pgfqpoint{1.071038in}{2.910744in}}{\pgfqpoint{1.081637in}{2.906354in}}{\pgfqpoint{1.092687in}{2.906354in}}%
\pgfpathclose%
\pgfusepath{stroke,fill}%
\end{pgfscope}%
\begin{pgfscope}%
\pgfpathrectangle{\pgfqpoint{0.907100in}{0.499691in}}{\pgfqpoint{3.875000in}{2.695000in}}%
\pgfusepath{clip}%
\pgfsetbuttcap%
\pgfsetroundjoin%
\definecolor{currentfill}{rgb}{0.529412,0.807843,0.921569}%
\pgfsetfillcolor{currentfill}%
\pgfsetlinewidth{1.003750pt}%
\definecolor{currentstroke}{rgb}{0.529412,0.807843,0.921569}%
\pgfsetstrokecolor{currentstroke}%
\pgfsetdash{}{0pt}%
\pgfpathmoveto{\pgfqpoint{1.640586in}{2.606613in}}%
\pgfpathcurveto{\pgfqpoint{1.651636in}{2.606613in}}{\pgfqpoint{1.662235in}{2.611004in}}{\pgfqpoint{1.670049in}{2.618817in}}%
\pgfpathcurveto{\pgfqpoint{1.677862in}{2.626631in}}{\pgfqpoint{1.682253in}{2.637230in}}{\pgfqpoint{1.682253in}{2.648280in}}%
\pgfpathcurveto{\pgfqpoint{1.682253in}{2.659330in}}{\pgfqpoint{1.677862in}{2.669929in}}{\pgfqpoint{1.670049in}{2.677743in}}%
\pgfpathcurveto{\pgfqpoint{1.662235in}{2.685556in}}{\pgfqpoint{1.651636in}{2.689947in}}{\pgfqpoint{1.640586in}{2.689947in}}%
\pgfpathcurveto{\pgfqpoint{1.629536in}{2.689947in}}{\pgfqpoint{1.618937in}{2.685556in}}{\pgfqpoint{1.611123in}{2.677743in}}%
\pgfpathcurveto{\pgfqpoint{1.603310in}{2.669929in}}{\pgfqpoint{1.598919in}{2.659330in}}{\pgfqpoint{1.598919in}{2.648280in}}%
\pgfpathcurveto{\pgfqpoint{1.598919in}{2.637230in}}{\pgfqpoint{1.603310in}{2.626631in}}{\pgfqpoint{1.611123in}{2.618817in}}%
\pgfpathcurveto{\pgfqpoint{1.618937in}{2.611004in}}{\pgfqpoint{1.629536in}{2.606613in}}{\pgfqpoint{1.640586in}{2.606613in}}%
\pgfpathclose%
\pgfusepath{stroke,fill}%
\end{pgfscope}%
\begin{pgfscope}%
\pgfpathrectangle{\pgfqpoint{0.907100in}{0.499691in}}{\pgfqpoint{3.875000in}{2.695000in}}%
\pgfusepath{clip}%
\pgfsetbuttcap%
\pgfsetroundjoin%
\definecolor{currentfill}{rgb}{0.529412,0.807843,0.921569}%
\pgfsetfillcolor{currentfill}%
\pgfsetlinewidth{1.003750pt}%
\definecolor{currentstroke}{rgb}{0.529412,0.807843,0.921569}%
\pgfsetstrokecolor{currentstroke}%
\pgfsetdash{}{0pt}%
\pgfpathmoveto{\pgfqpoint{2.011818in}{2.387945in}}%
\pgfpathcurveto{\pgfqpoint{2.022868in}{2.387945in}}{\pgfqpoint{2.033467in}{2.392335in}}{\pgfqpoint{2.041281in}{2.400149in}}%
\pgfpathcurveto{\pgfqpoint{2.049095in}{2.407963in}}{\pgfqpoint{2.053485in}{2.418562in}}{\pgfqpoint{2.053485in}{2.429612in}}%
\pgfpathcurveto{\pgfqpoint{2.053485in}{2.440662in}}{\pgfqpoint{2.049095in}{2.451261in}}{\pgfqpoint{2.041281in}{2.459075in}}%
\pgfpathcurveto{\pgfqpoint{2.033467in}{2.466888in}}{\pgfqpoint{2.022868in}{2.471279in}}{\pgfqpoint{2.011818in}{2.471279in}}%
\pgfpathcurveto{\pgfqpoint{2.000768in}{2.471279in}}{\pgfqpoint{1.990169in}{2.466888in}}{\pgfqpoint{1.982356in}{2.459075in}}%
\pgfpathcurveto{\pgfqpoint{1.974542in}{2.451261in}}{\pgfqpoint{1.970152in}{2.440662in}}{\pgfqpoint{1.970152in}{2.429612in}}%
\pgfpathcurveto{\pgfqpoint{1.970152in}{2.418562in}}{\pgfqpoint{1.974542in}{2.407963in}}{\pgfqpoint{1.982356in}{2.400149in}}%
\pgfpathcurveto{\pgfqpoint{1.990169in}{2.392335in}}{\pgfqpoint{2.000768in}{2.387945in}}{\pgfqpoint{2.011818in}{2.387945in}}%
\pgfpathclose%
\pgfusepath{stroke,fill}%
\end{pgfscope}%
\begin{pgfscope}%
\pgfpathrectangle{\pgfqpoint{0.907100in}{0.499691in}}{\pgfqpoint{3.875000in}{2.695000in}}%
\pgfusepath{clip}%
\pgfsetbuttcap%
\pgfsetroundjoin%
\definecolor{currentfill}{rgb}{0.529412,0.807843,0.921569}%
\pgfsetfillcolor{currentfill}%
\pgfsetlinewidth{1.003750pt}%
\definecolor{currentstroke}{rgb}{0.529412,0.807843,0.921569}%
\pgfsetstrokecolor{currentstroke}%
\pgfsetdash{}{0pt}%
\pgfpathmoveto{\pgfqpoint{2.292949in}{2.207976in}}%
\pgfpathcurveto{\pgfqpoint{2.303999in}{2.207976in}}{\pgfqpoint{2.314598in}{2.212366in}}{\pgfqpoint{2.322412in}{2.220180in}}%
\pgfpathcurveto{\pgfqpoint{2.330226in}{2.227994in}}{\pgfqpoint{2.334616in}{2.238593in}}{\pgfqpoint{2.334616in}{2.249643in}}%
\pgfpathcurveto{\pgfqpoint{2.334616in}{2.260693in}}{\pgfqpoint{2.330226in}{2.271292in}}{\pgfqpoint{2.322412in}{2.279106in}}%
\pgfpathcurveto{\pgfqpoint{2.314598in}{2.286919in}}{\pgfqpoint{2.303999in}{2.291309in}}{\pgfqpoint{2.292949in}{2.291309in}}%
\pgfpathcurveto{\pgfqpoint{2.281899in}{2.291309in}}{\pgfqpoint{2.271300in}{2.286919in}}{\pgfqpoint{2.263486in}{2.279106in}}%
\pgfpathcurveto{\pgfqpoint{2.255673in}{2.271292in}}{\pgfqpoint{2.251282in}{2.260693in}}{\pgfqpoint{2.251282in}{2.249643in}}%
\pgfpathcurveto{\pgfqpoint{2.251282in}{2.238593in}}{\pgfqpoint{2.255673in}{2.227994in}}{\pgfqpoint{2.263486in}{2.220180in}}%
\pgfpathcurveto{\pgfqpoint{2.271300in}{2.212366in}}{\pgfqpoint{2.281899in}{2.207976in}}{\pgfqpoint{2.292949in}{2.207976in}}%
\pgfpathclose%
\pgfusepath{stroke,fill}%
\end{pgfscope}%
\begin{pgfscope}%
\pgfpathrectangle{\pgfqpoint{0.907100in}{0.499691in}}{\pgfqpoint{3.875000in}{2.695000in}}%
\pgfusepath{clip}%
\pgfsetbuttcap%
\pgfsetroundjoin%
\definecolor{currentfill}{rgb}{0.529412,0.807843,0.921569}%
\pgfsetfillcolor{currentfill}%
\pgfsetlinewidth{1.003750pt}%
\definecolor{currentstroke}{rgb}{0.529412,0.807843,0.921569}%
\pgfsetstrokecolor{currentstroke}%
\pgfsetdash{}{0pt}%
\pgfpathmoveto{\pgfqpoint{2.519283in}{2.010850in}}%
\pgfpathcurveto{\pgfqpoint{2.530333in}{2.010850in}}{\pgfqpoint{2.540933in}{2.015241in}}{\pgfqpoint{2.548746in}{2.023054in}}%
\pgfpathcurveto{\pgfqpoint{2.556560in}{2.030868in}}{\pgfqpoint{2.560950in}{2.041467in}}{\pgfqpoint{2.560950in}{2.052517in}}%
\pgfpathcurveto{\pgfqpoint{2.560950in}{2.063567in}}{\pgfqpoint{2.556560in}{2.074166in}}{\pgfqpoint{2.548746in}{2.081980in}}%
\pgfpathcurveto{\pgfqpoint{2.540933in}{2.089793in}}{\pgfqpoint{2.530333in}{2.094184in}}{\pgfqpoint{2.519283in}{2.094184in}}%
\pgfpathcurveto{\pgfqpoint{2.508233in}{2.094184in}}{\pgfqpoint{2.497634in}{2.089793in}}{\pgfqpoint{2.489821in}{2.081980in}}%
\pgfpathcurveto{\pgfqpoint{2.482007in}{2.074166in}}{\pgfqpoint{2.477617in}{2.063567in}}{\pgfqpoint{2.477617in}{2.052517in}}%
\pgfpathcurveto{\pgfqpoint{2.477617in}{2.041467in}}{\pgfqpoint{2.482007in}{2.030868in}}{\pgfqpoint{2.489821in}{2.023054in}}%
\pgfpathcurveto{\pgfqpoint{2.497634in}{2.015241in}}{\pgfqpoint{2.508233in}{2.010850in}}{\pgfqpoint{2.519283in}{2.010850in}}%
\pgfpathclose%
\pgfusepath{stroke,fill}%
\end{pgfscope}%
\begin{pgfscope}%
\pgfpathrectangle{\pgfqpoint{0.907100in}{0.499691in}}{\pgfqpoint{3.875000in}{2.695000in}}%
\pgfusepath{clip}%
\pgfsetbuttcap%
\pgfsetroundjoin%
\definecolor{currentfill}{rgb}{0.529412,0.807843,0.921569}%
\pgfsetfillcolor{currentfill}%
\pgfsetlinewidth{1.003750pt}%
\definecolor{currentstroke}{rgb}{0.529412,0.807843,0.921569}%
\pgfsetstrokecolor{currentstroke}%
\pgfsetdash{}{0pt}%
\pgfpathmoveto{\pgfqpoint{2.708737in}{1.898241in}}%
\pgfpathcurveto{\pgfqpoint{2.719787in}{1.898241in}}{\pgfqpoint{2.730386in}{1.902631in}}{\pgfqpoint{2.738200in}{1.910445in}}%
\pgfpathcurveto{\pgfqpoint{2.746014in}{1.918258in}}{\pgfqpoint{2.750404in}{1.928857in}}{\pgfqpoint{2.750404in}{1.939908in}}%
\pgfpathcurveto{\pgfqpoint{2.750404in}{1.950958in}}{\pgfqpoint{2.746014in}{1.961557in}}{\pgfqpoint{2.738200in}{1.969370in}}%
\pgfpathcurveto{\pgfqpoint{2.730386in}{1.977184in}}{\pgfqpoint{2.719787in}{1.981574in}}{\pgfqpoint{2.708737in}{1.981574in}}%
\pgfpathcurveto{\pgfqpoint{2.697687in}{1.981574in}}{\pgfqpoint{2.687088in}{1.977184in}}{\pgfqpoint{2.679274in}{1.969370in}}%
\pgfpathcurveto{\pgfqpoint{2.671461in}{1.961557in}}{\pgfqpoint{2.667071in}{1.950958in}}{\pgfqpoint{2.667071in}{1.939908in}}%
\pgfpathcurveto{\pgfqpoint{2.667071in}{1.928857in}}{\pgfqpoint{2.671461in}{1.918258in}}{\pgfqpoint{2.679274in}{1.910445in}}%
\pgfpathcurveto{\pgfqpoint{2.687088in}{1.902631in}}{\pgfqpoint{2.697687in}{1.898241in}}{\pgfqpoint{2.708737in}{1.898241in}}%
\pgfpathclose%
\pgfusepath{stroke,fill}%
\end{pgfscope}%
\begin{pgfscope}%
\pgfpathrectangle{\pgfqpoint{0.907100in}{0.499691in}}{\pgfqpoint{3.875000in}{2.695000in}}%
\pgfusepath{clip}%
\pgfsetbuttcap%
\pgfsetroundjoin%
\definecolor{currentfill}{rgb}{0.529412,0.807843,0.921569}%
\pgfsetfillcolor{currentfill}%
\pgfsetlinewidth{1.003750pt}%
\definecolor{currentstroke}{rgb}{0.529412,0.807843,0.921569}%
\pgfsetstrokecolor{currentstroke}%
\pgfsetdash{}{0pt}%
\pgfpathmoveto{\pgfqpoint{2.871662in}{1.822945in}}%
\pgfpathcurveto{\pgfqpoint{2.882713in}{1.822945in}}{\pgfqpoint{2.893312in}{1.827336in}}{\pgfqpoint{2.901125in}{1.835149in}}%
\pgfpathcurveto{\pgfqpoint{2.908939in}{1.842963in}}{\pgfqpoint{2.913329in}{1.853562in}}{\pgfqpoint{2.913329in}{1.864612in}}%
\pgfpathcurveto{\pgfqpoint{2.913329in}{1.875662in}}{\pgfqpoint{2.908939in}{1.886261in}}{\pgfqpoint{2.901125in}{1.894075in}}%
\pgfpathcurveto{\pgfqpoint{2.893312in}{1.901888in}}{\pgfqpoint{2.882713in}{1.906279in}}{\pgfqpoint{2.871662in}{1.906279in}}%
\pgfpathcurveto{\pgfqpoint{2.860612in}{1.906279in}}{\pgfqpoint{2.850013in}{1.901888in}}{\pgfqpoint{2.842200in}{1.894075in}}%
\pgfpathcurveto{\pgfqpoint{2.834386in}{1.886261in}}{\pgfqpoint{2.829996in}{1.875662in}}{\pgfqpoint{2.829996in}{1.864612in}}%
\pgfpathcurveto{\pgfqpoint{2.829996in}{1.853562in}}{\pgfqpoint{2.834386in}{1.842963in}}{\pgfqpoint{2.842200in}{1.835149in}}%
\pgfpathcurveto{\pgfqpoint{2.850013in}{1.827336in}}{\pgfqpoint{2.860612in}{1.822945in}}{\pgfqpoint{2.871662in}{1.822945in}}%
\pgfpathclose%
\pgfusepath{stroke,fill}%
\end{pgfscope}%
\begin{pgfscope}%
\pgfpathrectangle{\pgfqpoint{0.907100in}{0.499691in}}{\pgfqpoint{3.875000in}{2.695000in}}%
\pgfusepath{clip}%
\pgfsetbuttcap%
\pgfsetroundjoin%
\definecolor{currentfill}{rgb}{0.529412,0.807843,0.921569}%
\pgfsetfillcolor{currentfill}%
\pgfsetlinewidth{1.003750pt}%
\definecolor{currentstroke}{rgb}{0.529412,0.807843,0.921569}%
\pgfsetstrokecolor{currentstroke}%
\pgfsetdash{}{0pt}%
\pgfpathmoveto{\pgfqpoint{3.014584in}{1.754021in}}%
\pgfpathcurveto{\pgfqpoint{3.025634in}{1.754021in}}{\pgfqpoint{3.036233in}{1.758411in}}{\pgfqpoint{3.044047in}{1.766224in}}%
\pgfpathcurveto{\pgfqpoint{3.051861in}{1.774038in}}{\pgfqpoint{3.056251in}{1.784637in}}{\pgfqpoint{3.056251in}{1.795687in}}%
\pgfpathcurveto{\pgfqpoint{3.056251in}{1.806737in}}{\pgfqpoint{3.051861in}{1.817336in}}{\pgfqpoint{3.044047in}{1.825150in}}%
\pgfpathcurveto{\pgfqpoint{3.036233in}{1.832964in}}{\pgfqpoint{3.025634in}{1.837354in}}{\pgfqpoint{3.014584in}{1.837354in}}%
\pgfpathcurveto{\pgfqpoint{3.003534in}{1.837354in}}{\pgfqpoint{2.992935in}{1.832964in}}{\pgfqpoint{2.985121in}{1.825150in}}%
\pgfpathcurveto{\pgfqpoint{2.977308in}{1.817336in}}{\pgfqpoint{2.972917in}{1.806737in}}{\pgfqpoint{2.972917in}{1.795687in}}%
\pgfpathcurveto{\pgfqpoint{2.972917in}{1.784637in}}{\pgfqpoint{2.977308in}{1.774038in}}{\pgfqpoint{2.985121in}{1.766224in}}%
\pgfpathcurveto{\pgfqpoint{2.992935in}{1.758411in}}{\pgfqpoint{3.003534in}{1.754021in}}{\pgfqpoint{3.014584in}{1.754021in}}%
\pgfpathclose%
\pgfusepath{stroke,fill}%
\end{pgfscope}%
\begin{pgfscope}%
\pgfpathrectangle{\pgfqpoint{0.907100in}{0.499691in}}{\pgfqpoint{3.875000in}{2.695000in}}%
\pgfusepath{clip}%
\pgfsetbuttcap%
\pgfsetroundjoin%
\definecolor{currentfill}{rgb}{0.529412,0.807843,0.921569}%
\pgfsetfillcolor{currentfill}%
\pgfsetlinewidth{1.003750pt}%
\definecolor{currentstroke}{rgb}{0.529412,0.807843,0.921569}%
\pgfsetstrokecolor{currentstroke}%
\pgfsetdash{}{0pt}%
\pgfpathmoveto{\pgfqpoint{3.141881in}{1.650702in}}%
\pgfpathcurveto{\pgfqpoint{3.152932in}{1.650702in}}{\pgfqpoint{3.163531in}{1.655093in}}{\pgfqpoint{3.171344in}{1.662906in}}%
\pgfpathcurveto{\pgfqpoint{3.179158in}{1.670720in}}{\pgfqpoint{3.183548in}{1.681319in}}{\pgfqpoint{3.183548in}{1.692369in}}%
\pgfpathcurveto{\pgfqpoint{3.183548in}{1.703419in}}{\pgfqpoint{3.179158in}{1.714018in}}{\pgfqpoint{3.171344in}{1.721832in}}%
\pgfpathcurveto{\pgfqpoint{3.163531in}{1.729645in}}{\pgfqpoint{3.152932in}{1.734036in}}{\pgfqpoint{3.141881in}{1.734036in}}%
\pgfpathcurveto{\pgfqpoint{3.130831in}{1.734036in}}{\pgfqpoint{3.120232in}{1.729645in}}{\pgfqpoint{3.112419in}{1.721832in}}%
\pgfpathcurveto{\pgfqpoint{3.104605in}{1.714018in}}{\pgfqpoint{3.100215in}{1.703419in}}{\pgfqpoint{3.100215in}{1.692369in}}%
\pgfpathcurveto{\pgfqpoint{3.100215in}{1.681319in}}{\pgfqpoint{3.104605in}{1.670720in}}{\pgfqpoint{3.112419in}{1.662906in}}%
\pgfpathcurveto{\pgfqpoint{3.120232in}{1.655093in}}{\pgfqpoint{3.130831in}{1.650702in}}{\pgfqpoint{3.141881in}{1.650702in}}%
\pgfpathclose%
\pgfusepath{stroke,fill}%
\end{pgfscope}%
\begin{pgfscope}%
\pgfpathrectangle{\pgfqpoint{0.907100in}{0.499691in}}{\pgfqpoint{3.875000in}{2.695000in}}%
\pgfusepath{clip}%
\pgfsetbuttcap%
\pgfsetroundjoin%
\definecolor{currentfill}{rgb}{0.529412,0.807843,0.921569}%
\pgfsetfillcolor{currentfill}%
\pgfsetlinewidth{1.003750pt}%
\definecolor{currentstroke}{rgb}{0.529412,0.807843,0.921569}%
\pgfsetstrokecolor{currentstroke}%
\pgfsetdash{}{0pt}%
\pgfpathmoveto{\pgfqpoint{3.256636in}{1.612783in}}%
\pgfpathcurveto{\pgfqpoint{3.267687in}{1.612783in}}{\pgfqpoint{3.278286in}{1.617173in}}{\pgfqpoint{3.286099in}{1.624987in}}%
\pgfpathcurveto{\pgfqpoint{3.293913in}{1.632800in}}{\pgfqpoint{3.298303in}{1.643399in}}{\pgfqpoint{3.298303in}{1.654450in}}%
\pgfpathcurveto{\pgfqpoint{3.298303in}{1.665500in}}{\pgfqpoint{3.293913in}{1.676099in}}{\pgfqpoint{3.286099in}{1.683912in}}%
\pgfpathcurveto{\pgfqpoint{3.278286in}{1.691726in}}{\pgfqpoint{3.267687in}{1.696116in}}{\pgfqpoint{3.256636in}{1.696116in}}%
\pgfpathcurveto{\pgfqpoint{3.245586in}{1.696116in}}{\pgfqpoint{3.234987in}{1.691726in}}{\pgfqpoint{3.227174in}{1.683912in}}%
\pgfpathcurveto{\pgfqpoint{3.219360in}{1.676099in}}{\pgfqpoint{3.214970in}{1.665500in}}{\pgfqpoint{3.214970in}{1.654450in}}%
\pgfpathcurveto{\pgfqpoint{3.214970in}{1.643399in}}{\pgfqpoint{3.219360in}{1.632800in}}{\pgfqpoint{3.227174in}{1.624987in}}%
\pgfpathcurveto{\pgfqpoint{3.234987in}{1.617173in}}{\pgfqpoint{3.245586in}{1.612783in}}{\pgfqpoint{3.256636in}{1.612783in}}%
\pgfpathclose%
\pgfusepath{stroke,fill}%
\end{pgfscope}%
\begin{pgfscope}%
\pgfpathrectangle{\pgfqpoint{0.907100in}{0.499691in}}{\pgfqpoint{3.875000in}{2.695000in}}%
\pgfusepath{clip}%
\pgfsetbuttcap%
\pgfsetroundjoin%
\definecolor{currentfill}{rgb}{0.529412,0.807843,0.921569}%
\pgfsetfillcolor{currentfill}%
\pgfsetlinewidth{1.003750pt}%
\definecolor{currentstroke}{rgb}{0.529412,0.807843,0.921569}%
\pgfsetstrokecolor{currentstroke}%
\pgfsetdash{}{0pt}%
\pgfpathmoveto{\pgfqpoint{3.361100in}{1.541858in}}%
\pgfpathcurveto{\pgfqpoint{3.372150in}{1.541858in}}{\pgfqpoint{3.382750in}{1.546248in}}{\pgfqpoint{3.390563in}{1.554062in}}%
\pgfpathcurveto{\pgfqpoint{3.398377in}{1.561875in}}{\pgfqpoint{3.402767in}{1.572474in}}{\pgfqpoint{3.402767in}{1.583525in}}%
\pgfpathcurveto{\pgfqpoint{3.402767in}{1.594575in}}{\pgfqpoint{3.398377in}{1.605174in}}{\pgfqpoint{3.390563in}{1.612987in}}%
\pgfpathcurveto{\pgfqpoint{3.382750in}{1.620801in}}{\pgfqpoint{3.372150in}{1.625191in}}{\pgfqpoint{3.361100in}{1.625191in}}%
\pgfpathcurveto{\pgfqpoint{3.350050in}{1.625191in}}{\pgfqpoint{3.339451in}{1.620801in}}{\pgfqpoint{3.331638in}{1.612987in}}%
\pgfpathcurveto{\pgfqpoint{3.323824in}{1.605174in}}{\pgfqpoint{3.319434in}{1.594575in}}{\pgfqpoint{3.319434in}{1.583525in}}%
\pgfpathcurveto{\pgfqpoint{3.319434in}{1.572474in}}{\pgfqpoint{3.323824in}{1.561875in}}{\pgfqpoint{3.331638in}{1.554062in}}%
\pgfpathcurveto{\pgfqpoint{3.339451in}{1.546248in}}{\pgfqpoint{3.350050in}{1.541858in}}{\pgfqpoint{3.361100in}{1.541858in}}%
\pgfpathclose%
\pgfusepath{stroke,fill}%
\end{pgfscope}%
\begin{pgfscope}%
\pgfpathrectangle{\pgfqpoint{0.907100in}{0.499691in}}{\pgfqpoint{3.875000in}{2.695000in}}%
\pgfusepath{clip}%
\pgfsetbuttcap%
\pgfsetroundjoin%
\definecolor{currentfill}{rgb}{0.529412,0.807843,0.921569}%
\pgfsetfillcolor{currentfill}%
\pgfsetlinewidth{1.003750pt}%
\definecolor{currentstroke}{rgb}{0.529412,0.807843,0.921569}%
\pgfsetstrokecolor{currentstroke}%
\pgfsetdash{}{0pt}%
\pgfpathmoveto{\pgfqpoint{3.456968in}{1.492441in}}%
\pgfpathcurveto{\pgfqpoint{3.468018in}{1.492441in}}{\pgfqpoint{3.478617in}{1.496832in}}{\pgfqpoint{3.486431in}{1.504645in}}%
\pgfpathcurveto{\pgfqpoint{3.494244in}{1.512459in}}{\pgfqpoint{3.498635in}{1.523058in}}{\pgfqpoint{3.498635in}{1.534108in}}%
\pgfpathcurveto{\pgfqpoint{3.498635in}{1.545158in}}{\pgfqpoint{3.494244in}{1.555757in}}{\pgfqpoint{3.486431in}{1.563571in}}%
\pgfpathcurveto{\pgfqpoint{3.478617in}{1.571385in}}{\pgfqpoint{3.468018in}{1.575775in}}{\pgfqpoint{3.456968in}{1.575775in}}%
\pgfpathcurveto{\pgfqpoint{3.445918in}{1.575775in}}{\pgfqpoint{3.435319in}{1.571385in}}{\pgfqpoint{3.427505in}{1.563571in}}%
\pgfpathcurveto{\pgfqpoint{3.419692in}{1.555757in}}{\pgfqpoint{3.415301in}{1.545158in}}{\pgfqpoint{3.415301in}{1.534108in}}%
\pgfpathcurveto{\pgfqpoint{3.415301in}{1.523058in}}{\pgfqpoint{3.419692in}{1.512459in}}{\pgfqpoint{3.427505in}{1.504645in}}%
\pgfpathcurveto{\pgfqpoint{3.435319in}{1.496832in}}{\pgfqpoint{3.445918in}{1.492441in}}{\pgfqpoint{3.456968in}{1.492441in}}%
\pgfpathclose%
\pgfusepath{stroke,fill}%
\end{pgfscope}%
\begin{pgfscope}%
\pgfpathrectangle{\pgfqpoint{0.907100in}{0.499691in}}{\pgfqpoint{3.875000in}{2.695000in}}%
\pgfusepath{clip}%
\pgfsetbuttcap%
\pgfsetroundjoin%
\definecolor{currentfill}{rgb}{0.529412,0.807843,0.921569}%
\pgfsetfillcolor{currentfill}%
\pgfsetlinewidth{1.003750pt}%
\definecolor{currentstroke}{rgb}{0.529412,0.807843,0.921569}%
\pgfsetstrokecolor{currentstroke}%
\pgfsetdash{}{0pt}%
\pgfpathmoveto{\pgfqpoint{3.545547in}{1.445850in}}%
\pgfpathcurveto{\pgfqpoint{3.556598in}{1.445850in}}{\pgfqpoint{3.567197in}{1.450241in}}{\pgfqpoint{3.575010in}{1.458054in}}%
\pgfpathcurveto{\pgfqpoint{3.582824in}{1.465868in}}{\pgfqpoint{3.587214in}{1.476467in}}{\pgfqpoint{3.587214in}{1.487517in}}%
\pgfpathcurveto{\pgfqpoint{3.587214in}{1.498567in}}{\pgfqpoint{3.582824in}{1.509166in}}{\pgfqpoint{3.575010in}{1.516980in}}%
\pgfpathcurveto{\pgfqpoint{3.567197in}{1.524793in}}{\pgfqpoint{3.556598in}{1.529184in}}{\pgfqpoint{3.545547in}{1.529184in}}%
\pgfpathcurveto{\pgfqpoint{3.534497in}{1.529184in}}{\pgfqpoint{3.523898in}{1.524793in}}{\pgfqpoint{3.516085in}{1.516980in}}%
\pgfpathcurveto{\pgfqpoint{3.508271in}{1.509166in}}{\pgfqpoint{3.503881in}{1.498567in}}{\pgfqpoint{3.503881in}{1.487517in}}%
\pgfpathcurveto{\pgfqpoint{3.503881in}{1.476467in}}{\pgfqpoint{3.508271in}{1.465868in}}{\pgfqpoint{3.516085in}{1.458054in}}%
\pgfpathcurveto{\pgfqpoint{3.523898in}{1.450241in}}{\pgfqpoint{3.534497in}{1.445850in}}{\pgfqpoint{3.545547in}{1.445850in}}%
\pgfpathclose%
\pgfusepath{stroke,fill}%
\end{pgfscope}%
\begin{pgfscope}%
\pgfpathrectangle{\pgfqpoint{0.907100in}{0.499691in}}{\pgfqpoint{3.875000in}{2.695000in}}%
\pgfusepath{clip}%
\pgfsetbuttcap%
\pgfsetroundjoin%
\definecolor{currentfill}{rgb}{0.529412,0.807843,0.921569}%
\pgfsetfillcolor{currentfill}%
\pgfsetlinewidth{1.003750pt}%
\definecolor{currentstroke}{rgb}{0.529412,0.807843,0.921569}%
\pgfsetstrokecolor{currentstroke}%
\pgfsetdash{}{0pt}%
\pgfpathmoveto{\pgfqpoint{3.627869in}{1.416206in}}%
\pgfpathcurveto{\pgfqpoint{3.638919in}{1.416206in}}{\pgfqpoint{3.649518in}{1.420597in}}{\pgfqpoint{3.657332in}{1.428410in}}%
\pgfpathcurveto{\pgfqpoint{3.665145in}{1.436224in}}{\pgfqpoint{3.669535in}{1.446823in}}{\pgfqpoint{3.669535in}{1.457873in}}%
\pgfpathcurveto{\pgfqpoint{3.669535in}{1.468923in}}{\pgfqpoint{3.665145in}{1.479522in}}{\pgfqpoint{3.657332in}{1.487336in}}%
\pgfpathcurveto{\pgfqpoint{3.649518in}{1.495149in}}{\pgfqpoint{3.638919in}{1.499540in}}{\pgfqpoint{3.627869in}{1.499540in}}%
\pgfpathcurveto{\pgfqpoint{3.616819in}{1.499540in}}{\pgfqpoint{3.606220in}{1.495149in}}{\pgfqpoint{3.598406in}{1.487336in}}%
\pgfpathcurveto{\pgfqpoint{3.590592in}{1.479522in}}{\pgfqpoint{3.586202in}{1.468923in}}{\pgfqpoint{3.586202in}{1.457873in}}%
\pgfpathcurveto{\pgfqpoint{3.586202in}{1.446823in}}{\pgfqpoint{3.590592in}{1.436224in}}{\pgfqpoint{3.598406in}{1.428410in}}%
\pgfpathcurveto{\pgfqpoint{3.606220in}{1.420597in}}{\pgfqpoint{3.616819in}{1.416206in}}{\pgfqpoint{3.627869in}{1.416206in}}%
\pgfpathclose%
\pgfusepath{stroke,fill}%
\end{pgfscope}%
\begin{pgfscope}%
\pgfpathrectangle{\pgfqpoint{0.907100in}{0.499691in}}{\pgfqpoint{3.875000in}{2.695000in}}%
\pgfusepath{clip}%
\pgfsetbuttcap%
\pgfsetroundjoin%
\definecolor{currentfill}{rgb}{0.529412,0.807843,0.921569}%
\pgfsetfillcolor{currentfill}%
\pgfsetlinewidth{1.003750pt}%
\definecolor{currentstroke}{rgb}{0.529412,0.807843,0.921569}%
\pgfsetstrokecolor{currentstroke}%
\pgfsetdash{}{0pt}%
\pgfpathmoveto{\pgfqpoint{3.704758in}{1.373668in}}%
\pgfpathcurveto{\pgfqpoint{3.715808in}{1.373668in}}{\pgfqpoint{3.726407in}{1.378059in}}{\pgfqpoint{3.734221in}{1.385872in}}%
\pgfpathcurveto{\pgfqpoint{3.742035in}{1.393686in}}{\pgfqpoint{3.746425in}{1.404285in}}{\pgfqpoint{3.746425in}{1.415335in}}%
\pgfpathcurveto{\pgfqpoint{3.746425in}{1.426385in}}{\pgfqpoint{3.742035in}{1.436984in}}{\pgfqpoint{3.734221in}{1.444798in}}%
\pgfpathcurveto{\pgfqpoint{3.726407in}{1.452612in}}{\pgfqpoint{3.715808in}{1.457002in}}{\pgfqpoint{3.704758in}{1.457002in}}%
\pgfpathcurveto{\pgfqpoint{3.693708in}{1.457002in}}{\pgfqpoint{3.683109in}{1.452612in}}{\pgfqpoint{3.675295in}{1.444798in}}%
\pgfpathcurveto{\pgfqpoint{3.667482in}{1.436984in}}{\pgfqpoint{3.663092in}{1.426385in}}{\pgfqpoint{3.663092in}{1.415335in}}%
\pgfpathcurveto{\pgfqpoint{3.663092in}{1.404285in}}{\pgfqpoint{3.667482in}{1.393686in}}{\pgfqpoint{3.675295in}{1.385872in}}%
\pgfpathcurveto{\pgfqpoint{3.683109in}{1.378059in}}{\pgfqpoint{3.693708in}{1.373668in}}{\pgfqpoint{3.704758in}{1.373668in}}%
\pgfpathclose%
\pgfusepath{stroke,fill}%
\end{pgfscope}%
\begin{pgfscope}%
\pgfpathrectangle{\pgfqpoint{0.907100in}{0.499691in}}{\pgfqpoint{3.875000in}{2.695000in}}%
\pgfusepath{clip}%
\pgfsetbuttcap%
\pgfsetroundjoin%
\definecolor{currentfill}{rgb}{0.529412,0.807843,0.921569}%
\pgfsetfillcolor{currentfill}%
\pgfsetlinewidth{1.003750pt}%
\definecolor{currentstroke}{rgb}{0.529412,0.807843,0.921569}%
\pgfsetstrokecolor{currentstroke}%
\pgfsetdash{}{0pt}%
\pgfpathmoveto{\pgfqpoint{3.776888in}{1.346495in}}%
\pgfpathcurveto{\pgfqpoint{3.787939in}{1.346495in}}{\pgfqpoint{3.798538in}{1.350885in}}{\pgfqpoint{3.806351in}{1.358699in}}%
\pgfpathcurveto{\pgfqpoint{3.814165in}{1.366513in}}{\pgfqpoint{3.818555in}{1.377112in}}{\pgfqpoint{3.818555in}{1.388162in}}%
\pgfpathcurveto{\pgfqpoint{3.818555in}{1.399212in}}{\pgfqpoint{3.814165in}{1.409811in}}{\pgfqpoint{3.806351in}{1.417625in}}%
\pgfpathcurveto{\pgfqpoint{3.798538in}{1.425438in}}{\pgfqpoint{3.787939in}{1.429829in}}{\pgfqpoint{3.776888in}{1.429829in}}%
\pgfpathcurveto{\pgfqpoint{3.765838in}{1.429829in}}{\pgfqpoint{3.755239in}{1.425438in}}{\pgfqpoint{3.747426in}{1.417625in}}%
\pgfpathcurveto{\pgfqpoint{3.739612in}{1.409811in}}{\pgfqpoint{3.735222in}{1.399212in}}{\pgfqpoint{3.735222in}{1.388162in}}%
\pgfpathcurveto{\pgfqpoint{3.735222in}{1.377112in}}{\pgfqpoint{3.739612in}{1.366513in}}{\pgfqpoint{3.747426in}{1.358699in}}%
\pgfpathcurveto{\pgfqpoint{3.755239in}{1.350885in}}{\pgfqpoint{3.765838in}{1.346495in}}{\pgfqpoint{3.776888in}{1.346495in}}%
\pgfpathclose%
\pgfusepath{stroke,fill}%
\end{pgfscope}%
\begin{pgfscope}%
\pgfpathrectangle{\pgfqpoint{0.907100in}{0.499691in}}{\pgfqpoint{3.875000in}{2.695000in}}%
\pgfusepath{clip}%
\pgfsetbuttcap%
\pgfsetroundjoin%
\definecolor{currentfill}{rgb}{0.529412,0.807843,0.921569}%
\pgfsetfillcolor{currentfill}%
\pgfsetlinewidth{1.003750pt}%
\definecolor{currentstroke}{rgb}{0.529412,0.807843,0.921569}%
\pgfsetstrokecolor{currentstroke}%
\pgfsetdash{}{0pt}%
\pgfpathmoveto{\pgfqpoint{3.844814in}{1.307362in}}%
\pgfpathcurveto{\pgfqpoint{3.855865in}{1.307362in}}{\pgfqpoint{3.866464in}{1.311752in}}{\pgfqpoint{3.874277in}{1.319566in}}%
\pgfpathcurveto{\pgfqpoint{3.882091in}{1.327379in}}{\pgfqpoint{3.886481in}{1.337978in}}{\pgfqpoint{3.886481in}{1.349028in}}%
\pgfpathcurveto{\pgfqpoint{3.886481in}{1.360079in}}{\pgfqpoint{3.882091in}{1.370678in}}{\pgfqpoint{3.874277in}{1.378491in}}%
\pgfpathcurveto{\pgfqpoint{3.866464in}{1.386305in}}{\pgfqpoint{3.855865in}{1.390695in}}{\pgfqpoint{3.844814in}{1.390695in}}%
\pgfpathcurveto{\pgfqpoint{3.833764in}{1.390695in}}{\pgfqpoint{3.823165in}{1.386305in}}{\pgfqpoint{3.815352in}{1.378491in}}%
\pgfpathcurveto{\pgfqpoint{3.807538in}{1.370678in}}{\pgfqpoint{3.803148in}{1.360079in}}{\pgfqpoint{3.803148in}{1.349028in}}%
\pgfpathcurveto{\pgfqpoint{3.803148in}{1.337978in}}{\pgfqpoint{3.807538in}{1.327379in}}{\pgfqpoint{3.815352in}{1.319566in}}%
\pgfpathcurveto{\pgfqpoint{3.823165in}{1.311752in}}{\pgfqpoint{3.833764in}{1.307362in}}{\pgfqpoint{3.844814in}{1.307362in}}%
\pgfpathclose%
\pgfusepath{stroke,fill}%
\end{pgfscope}%
\begin{pgfscope}%
\pgfpathrectangle{\pgfqpoint{0.907100in}{0.499691in}}{\pgfqpoint{3.875000in}{2.695000in}}%
\pgfusepath{clip}%
\pgfsetbuttcap%
\pgfsetroundjoin%
\definecolor{currentfill}{rgb}{0.529412,0.807843,0.921569}%
\pgfsetfillcolor{currentfill}%
\pgfsetlinewidth{1.003750pt}%
\definecolor{currentstroke}{rgb}{0.529412,0.807843,0.921569}%
\pgfsetstrokecolor{currentstroke}%
\pgfsetdash{}{0pt}%
\pgfpathmoveto{\pgfqpoint{3.909000in}{1.282279in}}%
\pgfpathcurveto{\pgfqpoint{3.920050in}{1.282279in}}{\pgfqpoint{3.930649in}{1.286669in}}{\pgfqpoint{3.938462in}{1.294483in}}%
\pgfpathcurveto{\pgfqpoint{3.946276in}{1.302297in}}{\pgfqpoint{3.950666in}{1.312896in}}{\pgfqpoint{3.950666in}{1.323946in}}%
\pgfpathcurveto{\pgfqpoint{3.950666in}{1.334996in}}{\pgfqpoint{3.946276in}{1.345595in}}{\pgfqpoint{3.938462in}{1.353409in}}%
\pgfpathcurveto{\pgfqpoint{3.930649in}{1.361222in}}{\pgfqpoint{3.920050in}{1.365612in}}{\pgfqpoint{3.909000in}{1.365612in}}%
\pgfpathcurveto{\pgfqpoint{3.897949in}{1.365612in}}{\pgfqpoint{3.887350in}{1.361222in}}{\pgfqpoint{3.879537in}{1.353409in}}%
\pgfpathcurveto{\pgfqpoint{3.871723in}{1.345595in}}{\pgfqpoint{3.867333in}{1.334996in}}{\pgfqpoint{3.867333in}{1.323946in}}%
\pgfpathcurveto{\pgfqpoint{3.867333in}{1.312896in}}{\pgfqpoint{3.871723in}{1.302297in}}{\pgfqpoint{3.879537in}{1.294483in}}%
\pgfpathcurveto{\pgfqpoint{3.887350in}{1.286669in}}{\pgfqpoint{3.897949in}{1.282279in}}{\pgfqpoint{3.909000in}{1.282279in}}%
\pgfpathclose%
\pgfusepath{stroke,fill}%
\end{pgfscope}%
\begin{pgfscope}%
\pgfpathrectangle{\pgfqpoint{0.907100in}{0.499691in}}{\pgfqpoint{3.875000in}{2.695000in}}%
\pgfusepath{clip}%
\pgfsetbuttcap%
\pgfsetroundjoin%
\definecolor{currentfill}{rgb}{0.529412,0.807843,0.921569}%
\pgfsetfillcolor{currentfill}%
\pgfsetlinewidth{1.003750pt}%
\definecolor{currentstroke}{rgb}{0.529412,0.807843,0.921569}%
\pgfsetstrokecolor{currentstroke}%
\pgfsetdash{}{0pt}%
\pgfpathmoveto{\pgfqpoint{3.969834in}{1.263961in}}%
\pgfpathcurveto{\pgfqpoint{3.980884in}{1.263961in}}{\pgfqpoint{3.991483in}{1.268351in}}{\pgfqpoint{3.999297in}{1.276165in}}%
\pgfpathcurveto{\pgfqpoint{4.007111in}{1.283978in}}{\pgfqpoint{4.011501in}{1.294578in}}{\pgfqpoint{4.011501in}{1.305628in}}%
\pgfpathcurveto{\pgfqpoint{4.011501in}{1.316678in}}{\pgfqpoint{4.007111in}{1.327277in}}{\pgfqpoint{3.999297in}{1.335090in}}%
\pgfpathcurveto{\pgfqpoint{3.991483in}{1.342904in}}{\pgfqpoint{3.980884in}{1.347294in}}{\pgfqpoint{3.969834in}{1.347294in}}%
\pgfpathcurveto{\pgfqpoint{3.958784in}{1.347294in}}{\pgfqpoint{3.948185in}{1.342904in}}{\pgfqpoint{3.940371in}{1.335090in}}%
\pgfpathcurveto{\pgfqpoint{3.932558in}{1.327277in}}{\pgfqpoint{3.928168in}{1.316678in}}{\pgfqpoint{3.928168in}{1.305628in}}%
\pgfpathcurveto{\pgfqpoint{3.928168in}{1.294578in}}{\pgfqpoint{3.932558in}{1.283978in}}{\pgfqpoint{3.940371in}{1.276165in}}%
\pgfpathcurveto{\pgfqpoint{3.948185in}{1.268351in}}{\pgfqpoint{3.958784in}{1.263961in}}{\pgfqpoint{3.969834in}{1.263961in}}%
\pgfpathclose%
\pgfusepath{stroke,fill}%
\end{pgfscope}%
\begin{pgfscope}%
\pgfpathrectangle{\pgfqpoint{0.907100in}{0.499691in}}{\pgfqpoint{3.875000in}{2.695000in}}%
\pgfusepath{clip}%
\pgfsetbuttcap%
\pgfsetroundjoin%
\definecolor{currentfill}{rgb}{0.529412,0.807843,0.921569}%
\pgfsetfillcolor{currentfill}%
\pgfsetlinewidth{1.003750pt}%
\definecolor{currentstroke}{rgb}{0.529412,0.807843,0.921569}%
\pgfsetstrokecolor{currentstroke}%
\pgfsetdash{}{0pt}%
\pgfpathmoveto{\pgfqpoint{4.027651in}{1.240160in}}%
\pgfpathcurveto{\pgfqpoint{4.038701in}{1.240160in}}{\pgfqpoint{4.049300in}{1.244550in}}{\pgfqpoint{4.057114in}{1.252364in}}%
\pgfpathcurveto{\pgfqpoint{4.064927in}{1.260178in}}{\pgfqpoint{4.069318in}{1.270777in}}{\pgfqpoint{4.069318in}{1.281827in}}%
\pgfpathcurveto{\pgfqpoint{4.069318in}{1.292877in}}{\pgfqpoint{4.064927in}{1.303476in}}{\pgfqpoint{4.057114in}{1.311290in}}%
\pgfpathcurveto{\pgfqpoint{4.049300in}{1.319103in}}{\pgfqpoint{4.038701in}{1.323493in}}{\pgfqpoint{4.027651in}{1.323493in}}%
\pgfpathcurveto{\pgfqpoint{4.016601in}{1.323493in}}{\pgfqpoint{4.006002in}{1.319103in}}{\pgfqpoint{3.998188in}{1.311290in}}%
\pgfpathcurveto{\pgfqpoint{3.990375in}{1.303476in}}{\pgfqpoint{3.985984in}{1.292877in}}{\pgfqpoint{3.985984in}{1.281827in}}%
\pgfpathcurveto{\pgfqpoint{3.985984in}{1.270777in}}{\pgfqpoint{3.990375in}{1.260178in}}{\pgfqpoint{3.998188in}{1.252364in}}%
\pgfpathcurveto{\pgfqpoint{4.006002in}{1.244550in}}{\pgfqpoint{4.016601in}{1.240160in}}{\pgfqpoint{4.027651in}{1.240160in}}%
\pgfpathclose%
\pgfusepath{stroke,fill}%
\end{pgfscope}%
\begin{pgfscope}%
\pgfpathrectangle{\pgfqpoint{0.907100in}{0.499691in}}{\pgfqpoint{3.875000in}{2.695000in}}%
\pgfusepath{clip}%
\pgfsetbuttcap%
\pgfsetroundjoin%
\definecolor{currentfill}{rgb}{0.529412,0.807843,0.921569}%
\pgfsetfillcolor{currentfill}%
\pgfsetlinewidth{1.003750pt}%
\definecolor{currentstroke}{rgb}{0.529412,0.807843,0.921569}%
\pgfsetstrokecolor{currentstroke}%
\pgfsetdash{}{0pt}%
\pgfpathmoveto{\pgfqpoint{4.082735in}{1.234317in}}%
\pgfpathcurveto{\pgfqpoint{4.093786in}{1.234317in}}{\pgfqpoint{4.104385in}{1.238707in}}{\pgfqpoint{4.112198in}{1.246521in}}%
\pgfpathcurveto{\pgfqpoint{4.120012in}{1.254334in}}{\pgfqpoint{4.124402in}{1.264933in}}{\pgfqpoint{4.124402in}{1.275984in}}%
\pgfpathcurveto{\pgfqpoint{4.124402in}{1.287034in}}{\pgfqpoint{4.120012in}{1.297633in}}{\pgfqpoint{4.112198in}{1.305446in}}%
\pgfpathcurveto{\pgfqpoint{4.104385in}{1.313260in}}{\pgfqpoint{4.093786in}{1.317650in}}{\pgfqpoint{4.082735in}{1.317650in}}%
\pgfpathcurveto{\pgfqpoint{4.071685in}{1.317650in}}{\pgfqpoint{4.061086in}{1.313260in}}{\pgfqpoint{4.053273in}{1.305446in}}%
\pgfpathcurveto{\pgfqpoint{4.045459in}{1.297633in}}{\pgfqpoint{4.041069in}{1.287034in}}{\pgfqpoint{4.041069in}{1.275984in}}%
\pgfpathcurveto{\pgfqpoint{4.041069in}{1.264933in}}{\pgfqpoint{4.045459in}{1.254334in}}{\pgfqpoint{4.053273in}{1.246521in}}%
\pgfpathcurveto{\pgfqpoint{4.061086in}{1.238707in}}{\pgfqpoint{4.071685in}{1.234317in}}{\pgfqpoint{4.082735in}{1.234317in}}%
\pgfpathclose%
\pgfusepath{stroke,fill}%
\end{pgfscope}%
\begin{pgfscope}%
\pgfpathrectangle{\pgfqpoint{0.907100in}{0.499691in}}{\pgfqpoint{3.875000in}{2.695000in}}%
\pgfusepath{clip}%
\pgfsetbuttcap%
\pgfsetroundjoin%
\definecolor{currentfill}{rgb}{0.529412,0.807843,0.921569}%
\pgfsetfillcolor{currentfill}%
\pgfsetlinewidth{1.003750pt}%
\definecolor{currentstroke}{rgb}{0.529412,0.807843,0.921569}%
\pgfsetstrokecolor{currentstroke}%
\pgfsetdash{}{0pt}%
\pgfpathmoveto{\pgfqpoint{4.135334in}{1.211354in}}%
\pgfpathcurveto{\pgfqpoint{4.146384in}{1.211354in}}{\pgfqpoint{4.156983in}{1.215744in}}{\pgfqpoint{4.164797in}{1.223558in}}%
\pgfpathcurveto{\pgfqpoint{4.172610in}{1.231372in}}{\pgfqpoint{4.177000in}{1.241971in}}{\pgfqpoint{4.177000in}{1.253021in}}%
\pgfpathcurveto{\pgfqpoint{4.177000in}{1.264071in}}{\pgfqpoint{4.172610in}{1.274670in}}{\pgfqpoint{4.164797in}{1.282484in}}%
\pgfpathcurveto{\pgfqpoint{4.156983in}{1.290297in}}{\pgfqpoint{4.146384in}{1.294687in}}{\pgfqpoint{4.135334in}{1.294687in}}%
\pgfpathcurveto{\pgfqpoint{4.124284in}{1.294687in}}{\pgfqpoint{4.113685in}{1.290297in}}{\pgfqpoint{4.105871in}{1.282484in}}%
\pgfpathcurveto{\pgfqpoint{4.098057in}{1.274670in}}{\pgfqpoint{4.093667in}{1.264071in}}{\pgfqpoint{4.093667in}{1.253021in}}%
\pgfpathcurveto{\pgfqpoint{4.093667in}{1.241971in}}{\pgfqpoint{4.098057in}{1.231372in}}{\pgfqpoint{4.105871in}{1.223558in}}%
\pgfpathcurveto{\pgfqpoint{4.113685in}{1.215744in}}{\pgfqpoint{4.124284in}{1.211354in}}{\pgfqpoint{4.135334in}{1.211354in}}%
\pgfpathclose%
\pgfusepath{stroke,fill}%
\end{pgfscope}%
\begin{pgfscope}%
\pgfpathrectangle{\pgfqpoint{0.907100in}{0.499691in}}{\pgfqpoint{3.875000in}{2.695000in}}%
\pgfusepath{clip}%
\pgfsetbuttcap%
\pgfsetroundjoin%
\definecolor{currentfill}{rgb}{0.529412,0.807843,0.921569}%
\pgfsetfillcolor{currentfill}%
\pgfsetlinewidth{1.003750pt}%
\definecolor{currentstroke}{rgb}{0.529412,0.807843,0.921569}%
\pgfsetstrokecolor{currentstroke}%
\pgfsetdash{}{0pt}%
\pgfpathmoveto{\pgfqpoint{4.185661in}{1.200111in}}%
\pgfpathcurveto{\pgfqpoint{4.196711in}{1.200111in}}{\pgfqpoint{4.207310in}{1.204501in}}{\pgfqpoint{4.215124in}{1.212315in}}%
\pgfpathcurveto{\pgfqpoint{4.222937in}{1.220128in}}{\pgfqpoint{4.227328in}{1.230727in}}{\pgfqpoint{4.227328in}{1.241778in}}%
\pgfpathcurveto{\pgfqpoint{4.227328in}{1.252828in}}{\pgfqpoint{4.222937in}{1.263427in}}{\pgfqpoint{4.215124in}{1.271240in}}%
\pgfpathcurveto{\pgfqpoint{4.207310in}{1.279054in}}{\pgfqpoint{4.196711in}{1.283444in}}{\pgfqpoint{4.185661in}{1.283444in}}%
\pgfpathcurveto{\pgfqpoint{4.174611in}{1.283444in}}{\pgfqpoint{4.164012in}{1.279054in}}{\pgfqpoint{4.156198in}{1.271240in}}%
\pgfpathcurveto{\pgfqpoint{4.148385in}{1.263427in}}{\pgfqpoint{4.143994in}{1.252828in}}{\pgfqpoint{4.143994in}{1.241778in}}%
\pgfpathcurveto{\pgfqpoint{4.143994in}{1.230727in}}{\pgfqpoint{4.148385in}{1.220128in}}{\pgfqpoint{4.156198in}{1.212315in}}%
\pgfpathcurveto{\pgfqpoint{4.164012in}{1.204501in}}{\pgfqpoint{4.174611in}{1.200111in}}{\pgfqpoint{4.185661in}{1.200111in}}%
\pgfpathclose%
\pgfusepath{stroke,fill}%
\end{pgfscope}%
\begin{pgfscope}%
\pgfpathrectangle{\pgfqpoint{0.907100in}{0.499691in}}{\pgfqpoint{3.875000in}{2.695000in}}%
\pgfusepath{clip}%
\pgfsetbuttcap%
\pgfsetroundjoin%
\definecolor{currentfill}{rgb}{0.529412,0.807843,0.921569}%
\pgfsetfillcolor{currentfill}%
\pgfsetlinewidth{1.003750pt}%
\definecolor{currentstroke}{rgb}{0.529412,0.807843,0.921569}%
\pgfsetstrokecolor{currentstroke}%
\pgfsetdash{}{0pt}%
\pgfpathmoveto{\pgfqpoint{4.233905in}{1.189021in}}%
\pgfpathcurveto{\pgfqpoint{4.244955in}{1.189021in}}{\pgfqpoint{4.255554in}{1.193411in}}{\pgfqpoint{4.263368in}{1.201225in}}%
\pgfpathcurveto{\pgfqpoint{4.271182in}{1.209038in}}{\pgfqpoint{4.275572in}{1.219637in}}{\pgfqpoint{4.275572in}{1.230687in}}%
\pgfpathcurveto{\pgfqpoint{4.275572in}{1.241737in}}{\pgfqpoint{4.271182in}{1.252336in}}{\pgfqpoint{4.263368in}{1.260150in}}%
\pgfpathcurveto{\pgfqpoint{4.255554in}{1.267964in}}{\pgfqpoint{4.244955in}{1.272354in}}{\pgfqpoint{4.233905in}{1.272354in}}%
\pgfpathcurveto{\pgfqpoint{4.222855in}{1.272354in}}{\pgfqpoint{4.212256in}{1.267964in}}{\pgfqpoint{4.204442in}{1.260150in}}%
\pgfpathcurveto{\pgfqpoint{4.196629in}{1.252336in}}{\pgfqpoint{4.192239in}{1.241737in}}{\pgfqpoint{4.192239in}{1.230687in}}%
\pgfpathcurveto{\pgfqpoint{4.192239in}{1.219637in}}{\pgfqpoint{4.196629in}{1.209038in}}{\pgfqpoint{4.204442in}{1.201225in}}%
\pgfpathcurveto{\pgfqpoint{4.212256in}{1.193411in}}{\pgfqpoint{4.222855in}{1.189021in}}{\pgfqpoint{4.233905in}{1.189021in}}%
\pgfpathclose%
\pgfusepath{stroke,fill}%
\end{pgfscope}%
\begin{pgfscope}%
\pgfpathrectangle{\pgfqpoint{0.907100in}{0.499691in}}{\pgfqpoint{3.875000in}{2.695000in}}%
\pgfusepath{clip}%
\pgfsetbuttcap%
\pgfsetroundjoin%
\definecolor{currentfill}{rgb}{0.529412,0.807843,0.921569}%
\pgfsetfillcolor{currentfill}%
\pgfsetlinewidth{1.003750pt}%
\definecolor{currentstroke}{rgb}{0.529412,0.807843,0.921569}%
\pgfsetstrokecolor{currentstroke}%
\pgfsetdash{}{0pt}%
\pgfpathmoveto{\pgfqpoint{4.280232in}{1.167283in}}%
\pgfpathcurveto{\pgfqpoint{4.291282in}{1.167283in}}{\pgfqpoint{4.301881in}{1.171673in}}{\pgfqpoint{4.309695in}{1.179487in}}%
\pgfpathcurveto{\pgfqpoint{4.317508in}{1.187300in}}{\pgfqpoint{4.321899in}{1.197899in}}{\pgfqpoint{4.321899in}{1.208949in}}%
\pgfpathcurveto{\pgfqpoint{4.321899in}{1.220000in}}{\pgfqpoint{4.317508in}{1.230599in}}{\pgfqpoint{4.309695in}{1.238412in}}%
\pgfpathcurveto{\pgfqpoint{4.301881in}{1.246226in}}{\pgfqpoint{4.291282in}{1.250616in}}{\pgfqpoint{4.280232in}{1.250616in}}%
\pgfpathcurveto{\pgfqpoint{4.269182in}{1.250616in}}{\pgfqpoint{4.258583in}{1.246226in}}{\pgfqpoint{4.250769in}{1.238412in}}%
\pgfpathcurveto{\pgfqpoint{4.242955in}{1.230599in}}{\pgfqpoint{4.238565in}{1.220000in}}{\pgfqpoint{4.238565in}{1.208949in}}%
\pgfpathcurveto{\pgfqpoint{4.238565in}{1.197899in}}{\pgfqpoint{4.242955in}{1.187300in}}{\pgfqpoint{4.250769in}{1.179487in}}%
\pgfpathcurveto{\pgfqpoint{4.258583in}{1.171673in}}{\pgfqpoint{4.269182in}{1.167283in}}{\pgfqpoint{4.280232in}{1.167283in}}%
\pgfpathclose%
\pgfusepath{stroke,fill}%
\end{pgfscope}%
\begin{pgfscope}%
\pgfpathrectangle{\pgfqpoint{0.907100in}{0.499691in}}{\pgfqpoint{3.875000in}{2.695000in}}%
\pgfusepath{clip}%
\pgfsetbuttcap%
\pgfsetroundjoin%
\definecolor{currentfill}{rgb}{0.529412,0.807843,0.921569}%
\pgfsetfillcolor{currentfill}%
\pgfsetlinewidth{1.003750pt}%
\definecolor{currentstroke}{rgb}{0.529412,0.807843,0.921569}%
\pgfsetstrokecolor{currentstroke}%
\pgfsetdash{}{0pt}%
\pgfpathmoveto{\pgfqpoint{4.324788in}{1.146110in}}%
\pgfpathcurveto{\pgfqpoint{4.335838in}{1.146110in}}{\pgfqpoint{4.346437in}{1.150500in}}{\pgfqpoint{4.354250in}{1.158313in}}%
\pgfpathcurveto{\pgfqpoint{4.362064in}{1.166127in}}{\pgfqpoint{4.366454in}{1.176726in}}{\pgfqpoint{4.366454in}{1.187776in}}%
\pgfpathcurveto{\pgfqpoint{4.366454in}{1.198826in}}{\pgfqpoint{4.362064in}{1.209425in}}{\pgfqpoint{4.354250in}{1.217239in}}%
\pgfpathcurveto{\pgfqpoint{4.346437in}{1.225053in}}{\pgfqpoint{4.335838in}{1.229443in}}{\pgfqpoint{4.324788in}{1.229443in}}%
\pgfpathcurveto{\pgfqpoint{4.313738in}{1.229443in}}{\pgfqpoint{4.303138in}{1.225053in}}{\pgfqpoint{4.295325in}{1.217239in}}%
\pgfpathcurveto{\pgfqpoint{4.287511in}{1.209425in}}{\pgfqpoint{4.283121in}{1.198826in}}{\pgfqpoint{4.283121in}{1.187776in}}%
\pgfpathcurveto{\pgfqpoint{4.283121in}{1.176726in}}{\pgfqpoint{4.287511in}{1.166127in}}{\pgfqpoint{4.295325in}{1.158313in}}%
\pgfpathcurveto{\pgfqpoint{4.303138in}{1.150500in}}{\pgfqpoint{4.313738in}{1.146110in}}{\pgfqpoint{4.324788in}{1.146110in}}%
\pgfpathclose%
\pgfusepath{stroke,fill}%
\end{pgfscope}%
\begin{pgfscope}%
\pgfpathrectangle{\pgfqpoint{0.907100in}{0.499691in}}{\pgfqpoint{3.875000in}{2.695000in}}%
\pgfusepath{clip}%
\pgfsetbuttcap%
\pgfsetroundjoin%
\definecolor{currentfill}{rgb}{0.529412,0.807843,0.921569}%
\pgfsetfillcolor{currentfill}%
\pgfsetlinewidth{1.003750pt}%
\definecolor{currentstroke}{rgb}{0.529412,0.807843,0.921569}%
\pgfsetstrokecolor{currentstroke}%
\pgfsetdash{}{0pt}%
\pgfpathmoveto{\pgfqpoint{4.367703in}{1.146110in}}%
\pgfpathcurveto{\pgfqpoint{4.378753in}{1.146110in}}{\pgfqpoint{4.389352in}{1.150500in}}{\pgfqpoint{4.397166in}{1.158313in}}%
\pgfpathcurveto{\pgfqpoint{4.404979in}{1.166127in}}{\pgfqpoint{4.409370in}{1.176726in}}{\pgfqpoint{4.409370in}{1.187776in}}%
\pgfpathcurveto{\pgfqpoint{4.409370in}{1.198826in}}{\pgfqpoint{4.404979in}{1.209425in}}{\pgfqpoint{4.397166in}{1.217239in}}%
\pgfpathcurveto{\pgfqpoint{4.389352in}{1.225053in}}{\pgfqpoint{4.378753in}{1.229443in}}{\pgfqpoint{4.367703in}{1.229443in}}%
\pgfpathcurveto{\pgfqpoint{4.356653in}{1.229443in}}{\pgfqpoint{4.346054in}{1.225053in}}{\pgfqpoint{4.338240in}{1.217239in}}%
\pgfpathcurveto{\pgfqpoint{4.330427in}{1.209425in}}{\pgfqpoint{4.326036in}{1.198826in}}{\pgfqpoint{4.326036in}{1.187776in}}%
\pgfpathcurveto{\pgfqpoint{4.326036in}{1.176726in}}{\pgfqpoint{4.330427in}{1.166127in}}{\pgfqpoint{4.338240in}{1.158313in}}%
\pgfpathcurveto{\pgfqpoint{4.346054in}{1.150500in}}{\pgfqpoint{4.356653in}{1.146110in}}{\pgfqpoint{4.367703in}{1.146110in}}%
\pgfpathclose%
\pgfusepath{stroke,fill}%
\end{pgfscope}%
\begin{pgfscope}%
\pgfpathrectangle{\pgfqpoint{0.907100in}{0.499691in}}{\pgfqpoint{3.875000in}{2.695000in}}%
\pgfusepath{clip}%
\pgfsetbuttcap%
\pgfsetroundjoin%
\definecolor{currentfill}{rgb}{0.529412,0.807843,0.921569}%
\pgfsetfillcolor{currentfill}%
\pgfsetlinewidth{1.003750pt}%
\definecolor{currentstroke}{rgb}{0.529412,0.807843,0.921569}%
\pgfsetstrokecolor{currentstroke}%
\pgfsetdash{}{0pt}%
\pgfpathmoveto{\pgfqpoint{4.409094in}{1.125472in}}%
\pgfpathcurveto{\pgfqpoint{4.420144in}{1.125472in}}{\pgfqpoint{4.430743in}{1.129863in}}{\pgfqpoint{4.438557in}{1.137676in}}%
\pgfpathcurveto{\pgfqpoint{4.446371in}{1.145490in}}{\pgfqpoint{4.450761in}{1.156089in}}{\pgfqpoint{4.450761in}{1.167139in}}%
\pgfpathcurveto{\pgfqpoint{4.450761in}{1.178189in}}{\pgfqpoint{4.446371in}{1.188788in}}{\pgfqpoint{4.438557in}{1.196602in}}%
\pgfpathcurveto{\pgfqpoint{4.430743in}{1.204415in}}{\pgfqpoint{4.420144in}{1.208806in}}{\pgfqpoint{4.409094in}{1.208806in}}%
\pgfpathcurveto{\pgfqpoint{4.398044in}{1.208806in}}{\pgfqpoint{4.387445in}{1.204415in}}{\pgfqpoint{4.379632in}{1.196602in}}%
\pgfpathcurveto{\pgfqpoint{4.371818in}{1.188788in}}{\pgfqpoint{4.367428in}{1.178189in}}{\pgfqpoint{4.367428in}{1.167139in}}%
\pgfpathcurveto{\pgfqpoint{4.367428in}{1.156089in}}{\pgfqpoint{4.371818in}{1.145490in}}{\pgfqpoint{4.379632in}{1.137676in}}%
\pgfpathcurveto{\pgfqpoint{4.387445in}{1.129863in}}{\pgfqpoint{4.398044in}{1.125472in}}{\pgfqpoint{4.409094in}{1.125472in}}%
\pgfpathclose%
\pgfusepath{stroke,fill}%
\end{pgfscope}%
\begin{pgfscope}%
\pgfpathrectangle{\pgfqpoint{0.907100in}{0.499691in}}{\pgfqpoint{3.875000in}{2.695000in}}%
\pgfusepath{clip}%
\pgfsetbuttcap%
\pgfsetroundjoin%
\definecolor{currentfill}{rgb}{0.529412,0.807843,0.921569}%
\pgfsetfillcolor{currentfill}%
\pgfsetlinewidth{1.003750pt}%
\definecolor{currentstroke}{rgb}{0.529412,0.807843,0.921569}%
\pgfsetstrokecolor{currentstroke}%
\pgfsetdash{}{0pt}%
\pgfpathmoveto{\pgfqpoint{4.449066in}{1.125472in}}%
\pgfpathcurveto{\pgfqpoint{4.460116in}{1.125472in}}{\pgfqpoint{4.470715in}{1.129863in}}{\pgfqpoint{4.478529in}{1.137676in}}%
\pgfpathcurveto{\pgfqpoint{4.486343in}{1.145490in}}{\pgfqpoint{4.490733in}{1.156089in}}{\pgfqpoint{4.490733in}{1.167139in}}%
\pgfpathcurveto{\pgfqpoint{4.490733in}{1.178189in}}{\pgfqpoint{4.486343in}{1.188788in}}{\pgfqpoint{4.478529in}{1.196602in}}%
\pgfpathcurveto{\pgfqpoint{4.470715in}{1.204415in}}{\pgfqpoint{4.460116in}{1.208806in}}{\pgfqpoint{4.449066in}{1.208806in}}%
\pgfpathcurveto{\pgfqpoint{4.438016in}{1.208806in}}{\pgfqpoint{4.427417in}{1.204415in}}{\pgfqpoint{4.419603in}{1.196602in}}%
\pgfpathcurveto{\pgfqpoint{4.411790in}{1.188788in}}{\pgfqpoint{4.407400in}{1.178189in}}{\pgfqpoint{4.407400in}{1.167139in}}%
\pgfpathcurveto{\pgfqpoint{4.407400in}{1.156089in}}{\pgfqpoint{4.411790in}{1.145490in}}{\pgfqpoint{4.419603in}{1.137676in}}%
\pgfpathcurveto{\pgfqpoint{4.427417in}{1.129863in}}{\pgfqpoint{4.438016in}{1.125472in}}{\pgfqpoint{4.449066in}{1.125472in}}%
\pgfpathclose%
\pgfusepath{stroke,fill}%
\end{pgfscope}%
\begin{pgfscope}%
\pgfpathrectangle{\pgfqpoint{0.907100in}{0.499691in}}{\pgfqpoint{3.875000in}{2.695000in}}%
\pgfusepath{clip}%
\pgfsetbuttcap%
\pgfsetroundjoin%
\definecolor{currentfill}{rgb}{0.529412,0.807843,0.921569}%
\pgfsetfillcolor{currentfill}%
\pgfsetlinewidth{1.003750pt}%
\definecolor{currentstroke}{rgb}{0.529412,0.807843,0.921569}%
\pgfsetstrokecolor{currentstroke}%
\pgfsetdash{}{0pt}%
\pgfpathmoveto{\pgfqpoint{4.487713in}{1.105345in}}%
\pgfpathcurveto{\pgfqpoint{4.498763in}{1.105345in}}{\pgfqpoint{4.509362in}{1.109735in}}{\pgfqpoint{4.517176in}{1.117549in}}%
\pgfpathcurveto{\pgfqpoint{4.524989in}{1.125362in}}{\pgfqpoint{4.529380in}{1.135961in}}{\pgfqpoint{4.529380in}{1.147012in}}%
\pgfpathcurveto{\pgfqpoint{4.529380in}{1.158062in}}{\pgfqpoint{4.524989in}{1.168661in}}{\pgfqpoint{4.517176in}{1.176474in}}%
\pgfpathcurveto{\pgfqpoint{4.509362in}{1.184288in}}{\pgfqpoint{4.498763in}{1.188678in}}{\pgfqpoint{4.487713in}{1.188678in}}%
\pgfpathcurveto{\pgfqpoint{4.476663in}{1.188678in}}{\pgfqpoint{4.466064in}{1.184288in}}{\pgfqpoint{4.458250in}{1.176474in}}%
\pgfpathcurveto{\pgfqpoint{4.450436in}{1.168661in}}{\pgfqpoint{4.446046in}{1.158062in}}{\pgfqpoint{4.446046in}{1.147012in}}%
\pgfpathcurveto{\pgfqpoint{4.446046in}{1.135961in}}{\pgfqpoint{4.450436in}{1.125362in}}{\pgfqpoint{4.458250in}{1.117549in}}%
\pgfpathcurveto{\pgfqpoint{4.466064in}{1.109735in}}{\pgfqpoint{4.476663in}{1.105345in}}{\pgfqpoint{4.487713in}{1.105345in}}%
\pgfpathclose%
\pgfusepath{stroke,fill}%
\end{pgfscope}%
\begin{pgfscope}%
\pgfpathrectangle{\pgfqpoint{0.907100in}{0.499691in}}{\pgfqpoint{3.875000in}{2.695000in}}%
\pgfusepath{clip}%
\pgfsetbuttcap%
\pgfsetroundjoin%
\definecolor{currentfill}{rgb}{0.529412,0.807843,0.921569}%
\pgfsetfillcolor{currentfill}%
\pgfsetlinewidth{1.003750pt}%
\definecolor{currentstroke}{rgb}{0.529412,0.807843,0.921569}%
\pgfsetstrokecolor{currentstroke}%
\pgfsetdash{}{0pt}%
\pgfpathmoveto{\pgfqpoint{4.525119in}{1.105345in}}%
\pgfpathcurveto{\pgfqpoint{4.536169in}{1.105345in}}{\pgfqpoint{4.546768in}{1.109735in}}{\pgfqpoint{4.554582in}{1.117549in}}%
\pgfpathcurveto{\pgfqpoint{4.562396in}{1.125362in}}{\pgfqpoint{4.566786in}{1.135961in}}{\pgfqpoint{4.566786in}{1.147012in}}%
\pgfpathcurveto{\pgfqpoint{4.566786in}{1.158062in}}{\pgfqpoint{4.562396in}{1.168661in}}{\pgfqpoint{4.554582in}{1.176474in}}%
\pgfpathcurveto{\pgfqpoint{4.546768in}{1.184288in}}{\pgfqpoint{4.536169in}{1.188678in}}{\pgfqpoint{4.525119in}{1.188678in}}%
\pgfpathcurveto{\pgfqpoint{4.514069in}{1.188678in}}{\pgfqpoint{4.503470in}{1.184288in}}{\pgfqpoint{4.495657in}{1.176474in}}%
\pgfpathcurveto{\pgfqpoint{4.487843in}{1.168661in}}{\pgfqpoint{4.483453in}{1.158062in}}{\pgfqpoint{4.483453in}{1.147012in}}%
\pgfpathcurveto{\pgfqpoint{4.483453in}{1.135961in}}{\pgfqpoint{4.487843in}{1.125362in}}{\pgfqpoint{4.495657in}{1.117549in}}%
\pgfpathcurveto{\pgfqpoint{4.503470in}{1.109735in}}{\pgfqpoint{4.514069in}{1.105345in}}{\pgfqpoint{4.525119in}{1.105345in}}%
\pgfpathclose%
\pgfusepath{stroke,fill}%
\end{pgfscope}%
\begin{pgfscope}%
\pgfpathrectangle{\pgfqpoint{0.907100in}{0.499691in}}{\pgfqpoint{3.875000in}{2.695000in}}%
\pgfusepath{clip}%
\pgfsetbuttcap%
\pgfsetroundjoin%
\definecolor{currentfill}{rgb}{0.529412,0.807843,0.921569}%
\pgfsetfillcolor{currentfill}%
\pgfsetlinewidth{1.003750pt}%
\definecolor{currentstroke}{rgb}{0.529412,0.807843,0.921569}%
\pgfsetstrokecolor{currentstroke}%
\pgfsetdash{}{0pt}%
\pgfpathmoveto{\pgfqpoint{4.561363in}{1.095465in}}%
\pgfpathcurveto{\pgfqpoint{4.572413in}{1.095465in}}{\pgfqpoint{4.583012in}{1.099855in}}{\pgfqpoint{4.590825in}{1.107668in}}%
\pgfpathcurveto{\pgfqpoint{4.598639in}{1.115482in}}{\pgfqpoint{4.603029in}{1.126081in}}{\pgfqpoint{4.603029in}{1.137131in}}%
\pgfpathcurveto{\pgfqpoint{4.603029in}{1.148181in}}{\pgfqpoint{4.598639in}{1.158780in}}{\pgfqpoint{4.590825in}{1.166594in}}%
\pgfpathcurveto{\pgfqpoint{4.583012in}{1.174408in}}{\pgfqpoint{4.572413in}{1.178798in}}{\pgfqpoint{4.561363in}{1.178798in}}%
\pgfpathcurveto{\pgfqpoint{4.550313in}{1.178798in}}{\pgfqpoint{4.539714in}{1.174408in}}{\pgfqpoint{4.531900in}{1.166594in}}%
\pgfpathcurveto{\pgfqpoint{4.524086in}{1.158780in}}{\pgfqpoint{4.519696in}{1.148181in}}{\pgfqpoint{4.519696in}{1.137131in}}%
\pgfpathcurveto{\pgfqpoint{4.519696in}{1.126081in}}{\pgfqpoint{4.524086in}{1.115482in}}{\pgfqpoint{4.531900in}{1.107668in}}%
\pgfpathcurveto{\pgfqpoint{4.539714in}{1.099855in}}{\pgfqpoint{4.550313in}{1.095465in}}{\pgfqpoint{4.561363in}{1.095465in}}%
\pgfpathclose%
\pgfusepath{stroke,fill}%
\end{pgfscope}%
\begin{pgfscope}%
\pgfpathrectangle{\pgfqpoint{0.907100in}{0.499691in}}{\pgfqpoint{3.875000in}{2.695000in}}%
\pgfusepath{clip}%
\pgfsetbuttcap%
\pgfsetroundjoin%
\definecolor{currentfill}{rgb}{0.529412,0.807843,0.921569}%
\pgfsetfillcolor{currentfill}%
\pgfsetlinewidth{1.003750pt}%
\definecolor{currentstroke}{rgb}{0.529412,0.807843,0.921569}%
\pgfsetstrokecolor{currentstroke}%
\pgfsetdash{}{0pt}%
\pgfpathmoveto{\pgfqpoint{4.596513in}{1.085702in}}%
\pgfpathcurveto{\pgfqpoint{4.607563in}{1.085702in}}{\pgfqpoint{4.618162in}{1.090093in}}{\pgfqpoint{4.625976in}{1.097906in}}%
\pgfpathcurveto{\pgfqpoint{4.633790in}{1.105720in}}{\pgfqpoint{4.638180in}{1.116319in}}{\pgfqpoint{4.638180in}{1.127369in}}%
\pgfpathcurveto{\pgfqpoint{4.638180in}{1.138419in}}{\pgfqpoint{4.633790in}{1.149018in}}{\pgfqpoint{4.625976in}{1.156832in}}%
\pgfpathcurveto{\pgfqpoint{4.618162in}{1.164646in}}{\pgfqpoint{4.607563in}{1.169036in}}{\pgfqpoint{4.596513in}{1.169036in}}%
\pgfpathcurveto{\pgfqpoint{4.585463in}{1.169036in}}{\pgfqpoint{4.574864in}{1.164646in}}{\pgfqpoint{4.567050in}{1.156832in}}%
\pgfpathcurveto{\pgfqpoint{4.559237in}{1.149018in}}{\pgfqpoint{4.554846in}{1.138419in}}{\pgfqpoint{4.554846in}{1.127369in}}%
\pgfpathcurveto{\pgfqpoint{4.554846in}{1.116319in}}{\pgfqpoint{4.559237in}{1.105720in}}{\pgfqpoint{4.567050in}{1.097906in}}%
\pgfpathcurveto{\pgfqpoint{4.574864in}{1.090093in}}{\pgfqpoint{4.585463in}{1.085702in}}{\pgfqpoint{4.596513in}{1.085702in}}%
\pgfpathclose%
\pgfusepath{stroke,fill}%
\end{pgfscope}%
\begin{pgfscope}%
\pgfsetbuttcap%
\pgfsetroundjoin%
\definecolor{currentfill}{rgb}{0.000000,0.000000,0.000000}%
\pgfsetfillcolor{currentfill}%
\pgfsetlinewidth{0.803000pt}%
\definecolor{currentstroke}{rgb}{0.000000,0.000000,0.000000}%
\pgfsetstrokecolor{currentstroke}%
\pgfsetdash{}{0pt}%
\pgfsys@defobject{currentmarker}{\pgfqpoint{0.000000in}{-0.048611in}}{\pgfqpoint{0.000000in}{0.000000in}}{%
\pgfpathmoveto{\pgfqpoint{0.000000in}{0.000000in}}%
\pgfpathlineto{\pgfqpoint{0.000000in}{-0.048611in}}%
\pgfusepath{stroke,fill}%
}%
\begin{pgfscope}%
\pgfsys@transformshift{1.092687in}{0.499691in}%
\pgfsys@useobject{currentmarker}{}%
\end{pgfscope}%
\end{pgfscope}%
\begin{pgfscope}%
\definecolor{textcolor}{rgb}{0.000000,0.000000,0.000000}%
\pgfsetstrokecolor{textcolor}%
\pgfsetfillcolor{textcolor}%
\pgftext[x=1.092687in,y=0.402469in,,top]{\color{textcolor}\rmfamily\fontsize{10.000000}{12.000000}\selectfont \(\displaystyle 2.0\)}%
\end{pgfscope}%
\begin{pgfscope}%
\pgfsetbuttcap%
\pgfsetroundjoin%
\definecolor{currentfill}{rgb}{0.000000,0.000000,0.000000}%
\pgfsetfillcolor{currentfill}%
\pgfsetlinewidth{0.803000pt}%
\definecolor{currentstroke}{rgb}{0.000000,0.000000,0.000000}%
\pgfsetstrokecolor{currentstroke}%
\pgfsetdash{}{0pt}%
\pgfsys@defobject{currentmarker}{\pgfqpoint{0.000000in}{-0.048611in}}{\pgfqpoint{0.000000in}{0.000000in}}{%
\pgfpathmoveto{\pgfqpoint{0.000000in}{0.000000in}}%
\pgfpathlineto{\pgfqpoint{0.000000in}{-0.048611in}}%
\pgfusepath{stroke,fill}%
}%
\begin{pgfscope}%
\pgfsys@transformshift{1.629527in}{0.499691in}%
\pgfsys@useobject{currentmarker}{}%
\end{pgfscope}%
\end{pgfscope}%
\begin{pgfscope}%
\definecolor{textcolor}{rgb}{0.000000,0.000000,0.000000}%
\pgfsetstrokecolor{textcolor}%
\pgfsetfillcolor{textcolor}%
\pgftext[x=1.629527in,y=0.402469in,,top]{\color{textcolor}\rmfamily\fontsize{10.000000}{12.000000}\selectfont \(\displaystyle 2.2\)}%
\end{pgfscope}%
\begin{pgfscope}%
\pgfsetbuttcap%
\pgfsetroundjoin%
\definecolor{currentfill}{rgb}{0.000000,0.000000,0.000000}%
\pgfsetfillcolor{currentfill}%
\pgfsetlinewidth{0.803000pt}%
\definecolor{currentstroke}{rgb}{0.000000,0.000000,0.000000}%
\pgfsetstrokecolor{currentstroke}%
\pgfsetdash{}{0pt}%
\pgfsys@defobject{currentmarker}{\pgfqpoint{0.000000in}{-0.048611in}}{\pgfqpoint{0.000000in}{0.000000in}}{%
\pgfpathmoveto{\pgfqpoint{0.000000in}{0.000000in}}%
\pgfpathlineto{\pgfqpoint{0.000000in}{-0.048611in}}%
\pgfusepath{stroke,fill}%
}%
\begin{pgfscope}%
\pgfsys@transformshift{2.166367in}{0.499691in}%
\pgfsys@useobject{currentmarker}{}%
\end{pgfscope}%
\end{pgfscope}%
\begin{pgfscope}%
\definecolor{textcolor}{rgb}{0.000000,0.000000,0.000000}%
\pgfsetstrokecolor{textcolor}%
\pgfsetfillcolor{textcolor}%
\pgftext[x=2.166367in,y=0.402469in,,top]{\color{textcolor}\rmfamily\fontsize{10.000000}{12.000000}\selectfont \(\displaystyle 2.4\)}%
\end{pgfscope}%
\begin{pgfscope}%
\pgfsetbuttcap%
\pgfsetroundjoin%
\definecolor{currentfill}{rgb}{0.000000,0.000000,0.000000}%
\pgfsetfillcolor{currentfill}%
\pgfsetlinewidth{0.803000pt}%
\definecolor{currentstroke}{rgb}{0.000000,0.000000,0.000000}%
\pgfsetstrokecolor{currentstroke}%
\pgfsetdash{}{0pt}%
\pgfsys@defobject{currentmarker}{\pgfqpoint{0.000000in}{-0.048611in}}{\pgfqpoint{0.000000in}{0.000000in}}{%
\pgfpathmoveto{\pgfqpoint{0.000000in}{0.000000in}}%
\pgfpathlineto{\pgfqpoint{0.000000in}{-0.048611in}}%
\pgfusepath{stroke,fill}%
}%
\begin{pgfscope}%
\pgfsys@transformshift{2.703208in}{0.499691in}%
\pgfsys@useobject{currentmarker}{}%
\end{pgfscope}%
\end{pgfscope}%
\begin{pgfscope}%
\definecolor{textcolor}{rgb}{0.000000,0.000000,0.000000}%
\pgfsetstrokecolor{textcolor}%
\pgfsetfillcolor{textcolor}%
\pgftext[x=2.703208in,y=0.402469in,,top]{\color{textcolor}\rmfamily\fontsize{10.000000}{12.000000}\selectfont \(\displaystyle 2.6\)}%
\end{pgfscope}%
\begin{pgfscope}%
\pgfsetbuttcap%
\pgfsetroundjoin%
\definecolor{currentfill}{rgb}{0.000000,0.000000,0.000000}%
\pgfsetfillcolor{currentfill}%
\pgfsetlinewidth{0.803000pt}%
\definecolor{currentstroke}{rgb}{0.000000,0.000000,0.000000}%
\pgfsetstrokecolor{currentstroke}%
\pgfsetdash{}{0pt}%
\pgfsys@defobject{currentmarker}{\pgfqpoint{0.000000in}{-0.048611in}}{\pgfqpoint{0.000000in}{0.000000in}}{%
\pgfpathmoveto{\pgfqpoint{0.000000in}{0.000000in}}%
\pgfpathlineto{\pgfqpoint{0.000000in}{-0.048611in}}%
\pgfusepath{stroke,fill}%
}%
\begin{pgfscope}%
\pgfsys@transformshift{3.240048in}{0.499691in}%
\pgfsys@useobject{currentmarker}{}%
\end{pgfscope}%
\end{pgfscope}%
\begin{pgfscope}%
\definecolor{textcolor}{rgb}{0.000000,0.000000,0.000000}%
\pgfsetstrokecolor{textcolor}%
\pgfsetfillcolor{textcolor}%
\pgftext[x=3.240048in,y=0.402469in,,top]{\color{textcolor}\rmfamily\fontsize{10.000000}{12.000000}\selectfont \(\displaystyle 2.8\)}%
\end{pgfscope}%
\begin{pgfscope}%
\pgfsetbuttcap%
\pgfsetroundjoin%
\definecolor{currentfill}{rgb}{0.000000,0.000000,0.000000}%
\pgfsetfillcolor{currentfill}%
\pgfsetlinewidth{0.803000pt}%
\definecolor{currentstroke}{rgb}{0.000000,0.000000,0.000000}%
\pgfsetstrokecolor{currentstroke}%
\pgfsetdash{}{0pt}%
\pgfsys@defobject{currentmarker}{\pgfqpoint{0.000000in}{-0.048611in}}{\pgfqpoint{0.000000in}{0.000000in}}{%
\pgfpathmoveto{\pgfqpoint{0.000000in}{0.000000in}}%
\pgfpathlineto{\pgfqpoint{0.000000in}{-0.048611in}}%
\pgfusepath{stroke,fill}%
}%
\begin{pgfscope}%
\pgfsys@transformshift{3.776888in}{0.499691in}%
\pgfsys@useobject{currentmarker}{}%
\end{pgfscope}%
\end{pgfscope}%
\begin{pgfscope}%
\definecolor{textcolor}{rgb}{0.000000,0.000000,0.000000}%
\pgfsetstrokecolor{textcolor}%
\pgfsetfillcolor{textcolor}%
\pgftext[x=3.776888in,y=0.402469in,,top]{\color{textcolor}\rmfamily\fontsize{10.000000}{12.000000}\selectfont \(\displaystyle 3.0\)}%
\end{pgfscope}%
\begin{pgfscope}%
\pgfsetbuttcap%
\pgfsetroundjoin%
\definecolor{currentfill}{rgb}{0.000000,0.000000,0.000000}%
\pgfsetfillcolor{currentfill}%
\pgfsetlinewidth{0.803000pt}%
\definecolor{currentstroke}{rgb}{0.000000,0.000000,0.000000}%
\pgfsetstrokecolor{currentstroke}%
\pgfsetdash{}{0pt}%
\pgfsys@defobject{currentmarker}{\pgfqpoint{0.000000in}{-0.048611in}}{\pgfqpoint{0.000000in}{0.000000in}}{%
\pgfpathmoveto{\pgfqpoint{0.000000in}{0.000000in}}%
\pgfpathlineto{\pgfqpoint{0.000000in}{-0.048611in}}%
\pgfusepath{stroke,fill}%
}%
\begin{pgfscope}%
\pgfsys@transformshift{4.313729in}{0.499691in}%
\pgfsys@useobject{currentmarker}{}%
\end{pgfscope}%
\end{pgfscope}%
\begin{pgfscope}%
\definecolor{textcolor}{rgb}{0.000000,0.000000,0.000000}%
\pgfsetstrokecolor{textcolor}%
\pgfsetfillcolor{textcolor}%
\pgftext[x=4.313729in,y=0.402469in,,top]{\color{textcolor}\rmfamily\fontsize{10.000000}{12.000000}\selectfont \(\displaystyle 3.2\)}%
\end{pgfscope}%
\begin{pgfscope}%
\definecolor{textcolor}{rgb}{0.000000,0.000000,0.000000}%
\pgfsetstrokecolor{textcolor}%
\pgfsetfillcolor{textcolor}%
\pgftext[x=2.844600in,y=0.223457in,,top]{\color{textcolor}\rmfamily\fontsize{10.000000}{12.000000}\selectfont log f}%
\end{pgfscope}%
\begin{pgfscope}%
\pgfsetbuttcap%
\pgfsetroundjoin%
\definecolor{currentfill}{rgb}{0.000000,0.000000,0.000000}%
\pgfsetfillcolor{currentfill}%
\pgfsetlinewidth{0.803000pt}%
\definecolor{currentstroke}{rgb}{0.000000,0.000000,0.000000}%
\pgfsetstrokecolor{currentstroke}%
\pgfsetdash{}{0pt}%
\pgfsys@defobject{currentmarker}{\pgfqpoint{-0.048611in}{0.000000in}}{\pgfqpoint{0.000000in}{0.000000in}}{%
\pgfpathmoveto{\pgfqpoint{0.000000in}{0.000000in}}%
\pgfpathlineto{\pgfqpoint{-0.048611in}{0.000000in}}%
\pgfusepath{stroke,fill}%
}%
\begin{pgfscope}%
\pgfsys@transformshift{0.907100in}{0.965122in}%
\pgfsys@useobject{currentmarker}{}%
\end{pgfscope}%
\end{pgfscope}%
\begin{pgfscope}%
\definecolor{textcolor}{rgb}{0.000000,0.000000,0.000000}%
\pgfsetstrokecolor{textcolor}%
\pgfsetfillcolor{textcolor}%
\pgftext[x=0.670988in,y=0.916897in,left,base]{\color{textcolor}\rmfamily\fontsize{10.000000}{12.000000}\selectfont \(\displaystyle 80\)}%
\end{pgfscope}%
\begin{pgfscope}%
\pgfsetbuttcap%
\pgfsetroundjoin%
\definecolor{currentfill}{rgb}{0.000000,0.000000,0.000000}%
\pgfsetfillcolor{currentfill}%
\pgfsetlinewidth{0.803000pt}%
\definecolor{currentstroke}{rgb}{0.000000,0.000000,0.000000}%
\pgfsetstrokecolor{currentstroke}%
\pgfsetdash{}{0pt}%
\pgfsys@defobject{currentmarker}{\pgfqpoint{-0.048611in}{0.000000in}}{\pgfqpoint{0.000000in}{0.000000in}}{%
\pgfpathmoveto{\pgfqpoint{0.000000in}{0.000000in}}%
\pgfpathlineto{\pgfqpoint{-0.048611in}{0.000000in}}%
\pgfusepath{stroke,fill}%
}%
\begin{pgfscope}%
\pgfsys@transformshift{0.907100in}{1.434345in}%
\pgfsys@useobject{currentmarker}{}%
\end{pgfscope}%
\end{pgfscope}%
\begin{pgfscope}%
\definecolor{textcolor}{rgb}{0.000000,0.000000,0.000000}%
\pgfsetstrokecolor{textcolor}%
\pgfsetfillcolor{textcolor}%
\pgftext[x=0.670988in,y=1.386119in,left,base]{\color{textcolor}\rmfamily\fontsize{10.000000}{12.000000}\selectfont \(\displaystyle 85\)}%
\end{pgfscope}%
\begin{pgfscope}%
\pgfsetbuttcap%
\pgfsetroundjoin%
\definecolor{currentfill}{rgb}{0.000000,0.000000,0.000000}%
\pgfsetfillcolor{currentfill}%
\pgfsetlinewidth{0.803000pt}%
\definecolor{currentstroke}{rgb}{0.000000,0.000000,0.000000}%
\pgfsetstrokecolor{currentstroke}%
\pgfsetdash{}{0pt}%
\pgfsys@defobject{currentmarker}{\pgfqpoint{-0.048611in}{0.000000in}}{\pgfqpoint{0.000000in}{0.000000in}}{%
\pgfpathmoveto{\pgfqpoint{0.000000in}{0.000000in}}%
\pgfpathlineto{\pgfqpoint{-0.048611in}{0.000000in}}%
\pgfusepath{stroke,fill}%
}%
\begin{pgfscope}%
\pgfsys@transformshift{0.907100in}{1.903567in}%
\pgfsys@useobject{currentmarker}{}%
\end{pgfscope}%
\end{pgfscope}%
\begin{pgfscope}%
\definecolor{textcolor}{rgb}{0.000000,0.000000,0.000000}%
\pgfsetstrokecolor{textcolor}%
\pgfsetfillcolor{textcolor}%
\pgftext[x=0.670988in,y=1.855342in,left,base]{\color{textcolor}\rmfamily\fontsize{10.000000}{12.000000}\selectfont \(\displaystyle 90\)}%
\end{pgfscope}%
\begin{pgfscope}%
\pgfsetbuttcap%
\pgfsetroundjoin%
\definecolor{currentfill}{rgb}{0.000000,0.000000,0.000000}%
\pgfsetfillcolor{currentfill}%
\pgfsetlinewidth{0.803000pt}%
\definecolor{currentstroke}{rgb}{0.000000,0.000000,0.000000}%
\pgfsetstrokecolor{currentstroke}%
\pgfsetdash{}{0pt}%
\pgfsys@defobject{currentmarker}{\pgfqpoint{-0.048611in}{0.000000in}}{\pgfqpoint{0.000000in}{0.000000in}}{%
\pgfpathmoveto{\pgfqpoint{0.000000in}{0.000000in}}%
\pgfpathlineto{\pgfqpoint{-0.048611in}{0.000000in}}%
\pgfusepath{stroke,fill}%
}%
\begin{pgfscope}%
\pgfsys@transformshift{0.907100in}{2.372789in}%
\pgfsys@useobject{currentmarker}{}%
\end{pgfscope}%
\end{pgfscope}%
\begin{pgfscope}%
\definecolor{textcolor}{rgb}{0.000000,0.000000,0.000000}%
\pgfsetstrokecolor{textcolor}%
\pgfsetfillcolor{textcolor}%
\pgftext[x=0.670988in,y=2.324564in,left,base]{\color{textcolor}\rmfamily\fontsize{10.000000}{12.000000}\selectfont \(\displaystyle 95\)}%
\end{pgfscope}%
\begin{pgfscope}%
\pgfsetbuttcap%
\pgfsetroundjoin%
\definecolor{currentfill}{rgb}{0.000000,0.000000,0.000000}%
\pgfsetfillcolor{currentfill}%
\pgfsetlinewidth{0.803000pt}%
\definecolor{currentstroke}{rgb}{0.000000,0.000000,0.000000}%
\pgfsetstrokecolor{currentstroke}%
\pgfsetdash{}{0pt}%
\pgfsys@defobject{currentmarker}{\pgfqpoint{-0.048611in}{0.000000in}}{\pgfqpoint{0.000000in}{0.000000in}}{%
\pgfpathmoveto{\pgfqpoint{0.000000in}{0.000000in}}%
\pgfpathlineto{\pgfqpoint{-0.048611in}{0.000000in}}%
\pgfusepath{stroke,fill}%
}%
\begin{pgfscope}%
\pgfsys@transformshift{0.907100in}{2.842012in}%
\pgfsys@useobject{currentmarker}{}%
\end{pgfscope}%
\end{pgfscope}%
\begin{pgfscope}%
\definecolor{textcolor}{rgb}{0.000000,0.000000,0.000000}%
\pgfsetstrokecolor{textcolor}%
\pgfsetfillcolor{textcolor}%
\pgftext[x=0.601544in,y=2.793786in,left,base]{\color{textcolor}\rmfamily\fontsize{10.000000}{12.000000}\selectfont \(\displaystyle 100\)}%
\end{pgfscope}%
\begin{pgfscope}%
\definecolor{textcolor}{rgb}{0.000000,0.000000,0.000000}%
\pgfsetstrokecolor{textcolor}%
\pgfsetfillcolor{textcolor}%
\pgftext[x=0.323766in,y=1.847191in,,bottom]{\color{textcolor}\rmfamily\fontsize{10.000000}{12.000000}\selectfont 20log Z}%
\end{pgfscope}%
\begin{pgfscope}%
\pgfpathrectangle{\pgfqpoint{0.907100in}{0.499691in}}{\pgfqpoint{3.875000in}{2.695000in}}%
\pgfusepath{clip}%
\pgfsetbuttcap%
\pgfsetroundjoin%
\definecolor{currentfill}{rgb}{0.121569,0.466667,0.705882}%
\pgfsetfillcolor{currentfill}%
\pgfsetlinewidth{1.003750pt}%
\definecolor{currentstroke}{rgb}{0.121569,0.466667,0.705882}%
\pgfsetstrokecolor{currentstroke}%
\pgfsetdash{}{0pt}%
\pgfpathmoveto{\pgfqpoint{3.456968in}{1.492441in}}%
\pgfpathcurveto{\pgfqpoint{3.468018in}{1.492441in}}{\pgfqpoint{3.478617in}{1.496832in}}{\pgfqpoint{3.486431in}{1.504645in}}%
\pgfpathcurveto{\pgfqpoint{3.494244in}{1.512459in}}{\pgfqpoint{3.498635in}{1.523058in}}{\pgfqpoint{3.498635in}{1.534108in}}%
\pgfpathcurveto{\pgfqpoint{3.498635in}{1.545158in}}{\pgfqpoint{3.494244in}{1.555757in}}{\pgfqpoint{3.486431in}{1.563571in}}%
\pgfpathcurveto{\pgfqpoint{3.478617in}{1.571385in}}{\pgfqpoint{3.468018in}{1.575775in}}{\pgfqpoint{3.456968in}{1.575775in}}%
\pgfpathcurveto{\pgfqpoint{3.445918in}{1.575775in}}{\pgfqpoint{3.435319in}{1.571385in}}{\pgfqpoint{3.427505in}{1.563571in}}%
\pgfpathcurveto{\pgfqpoint{3.419692in}{1.555757in}}{\pgfqpoint{3.415301in}{1.545158in}}{\pgfqpoint{3.415301in}{1.534108in}}%
\pgfpathcurveto{\pgfqpoint{3.415301in}{1.523058in}}{\pgfqpoint{3.419692in}{1.512459in}}{\pgfqpoint{3.427505in}{1.504645in}}%
\pgfpathcurveto{\pgfqpoint{3.435319in}{1.496832in}}{\pgfqpoint{3.445918in}{1.492441in}}{\pgfqpoint{3.456968in}{1.492441in}}%
\pgfpathclose%
\pgfusepath{stroke,fill}%
\end{pgfscope}%
\begin{pgfscope}%
\pgfpathrectangle{\pgfqpoint{0.907100in}{0.499691in}}{\pgfqpoint{3.875000in}{2.695000in}}%
\pgfusepath{clip}%
\pgfsetbuttcap%
\pgfsetroundjoin%
\definecolor{currentfill}{rgb}{0.121569,0.466667,0.705882}%
\pgfsetfillcolor{currentfill}%
\pgfsetlinewidth{1.003750pt}%
\definecolor{currentstroke}{rgb}{0.121569,0.466667,0.705882}%
\pgfsetstrokecolor{currentstroke}%
\pgfsetdash{}{0pt}%
\pgfpathmoveto{\pgfqpoint{3.545547in}{1.445850in}}%
\pgfpathcurveto{\pgfqpoint{3.556598in}{1.445850in}}{\pgfqpoint{3.567197in}{1.450241in}}{\pgfqpoint{3.575010in}{1.458054in}}%
\pgfpathcurveto{\pgfqpoint{3.582824in}{1.465868in}}{\pgfqpoint{3.587214in}{1.476467in}}{\pgfqpoint{3.587214in}{1.487517in}}%
\pgfpathcurveto{\pgfqpoint{3.587214in}{1.498567in}}{\pgfqpoint{3.582824in}{1.509166in}}{\pgfqpoint{3.575010in}{1.516980in}}%
\pgfpathcurveto{\pgfqpoint{3.567197in}{1.524793in}}{\pgfqpoint{3.556598in}{1.529184in}}{\pgfqpoint{3.545547in}{1.529184in}}%
\pgfpathcurveto{\pgfqpoint{3.534497in}{1.529184in}}{\pgfqpoint{3.523898in}{1.524793in}}{\pgfqpoint{3.516085in}{1.516980in}}%
\pgfpathcurveto{\pgfqpoint{3.508271in}{1.509166in}}{\pgfqpoint{3.503881in}{1.498567in}}{\pgfqpoint{3.503881in}{1.487517in}}%
\pgfpathcurveto{\pgfqpoint{3.503881in}{1.476467in}}{\pgfqpoint{3.508271in}{1.465868in}}{\pgfqpoint{3.516085in}{1.458054in}}%
\pgfpathcurveto{\pgfqpoint{3.523898in}{1.450241in}}{\pgfqpoint{3.534497in}{1.445850in}}{\pgfqpoint{3.545547in}{1.445850in}}%
\pgfpathclose%
\pgfusepath{stroke,fill}%
\end{pgfscope}%
\begin{pgfscope}%
\pgfpathrectangle{\pgfqpoint{0.907100in}{0.499691in}}{\pgfqpoint{3.875000in}{2.695000in}}%
\pgfusepath{clip}%
\pgfsetbuttcap%
\pgfsetroundjoin%
\definecolor{currentfill}{rgb}{0.121569,0.466667,0.705882}%
\pgfsetfillcolor{currentfill}%
\pgfsetlinewidth{1.003750pt}%
\definecolor{currentstroke}{rgb}{0.121569,0.466667,0.705882}%
\pgfsetstrokecolor{currentstroke}%
\pgfsetdash{}{0pt}%
\pgfpathmoveto{\pgfqpoint{3.627869in}{1.416206in}}%
\pgfpathcurveto{\pgfqpoint{3.638919in}{1.416206in}}{\pgfqpoint{3.649518in}{1.420597in}}{\pgfqpoint{3.657332in}{1.428410in}}%
\pgfpathcurveto{\pgfqpoint{3.665145in}{1.436224in}}{\pgfqpoint{3.669535in}{1.446823in}}{\pgfqpoint{3.669535in}{1.457873in}}%
\pgfpathcurveto{\pgfqpoint{3.669535in}{1.468923in}}{\pgfqpoint{3.665145in}{1.479522in}}{\pgfqpoint{3.657332in}{1.487336in}}%
\pgfpathcurveto{\pgfqpoint{3.649518in}{1.495149in}}{\pgfqpoint{3.638919in}{1.499540in}}{\pgfqpoint{3.627869in}{1.499540in}}%
\pgfpathcurveto{\pgfqpoint{3.616819in}{1.499540in}}{\pgfqpoint{3.606220in}{1.495149in}}{\pgfqpoint{3.598406in}{1.487336in}}%
\pgfpathcurveto{\pgfqpoint{3.590592in}{1.479522in}}{\pgfqpoint{3.586202in}{1.468923in}}{\pgfqpoint{3.586202in}{1.457873in}}%
\pgfpathcurveto{\pgfqpoint{3.586202in}{1.446823in}}{\pgfqpoint{3.590592in}{1.436224in}}{\pgfqpoint{3.598406in}{1.428410in}}%
\pgfpathcurveto{\pgfqpoint{3.606220in}{1.420597in}}{\pgfqpoint{3.616819in}{1.416206in}}{\pgfqpoint{3.627869in}{1.416206in}}%
\pgfpathclose%
\pgfusepath{stroke,fill}%
\end{pgfscope}%
\begin{pgfscope}%
\pgfpathrectangle{\pgfqpoint{0.907100in}{0.499691in}}{\pgfqpoint{3.875000in}{2.695000in}}%
\pgfusepath{clip}%
\pgfsetbuttcap%
\pgfsetroundjoin%
\definecolor{currentfill}{rgb}{0.121569,0.466667,0.705882}%
\pgfsetfillcolor{currentfill}%
\pgfsetlinewidth{1.003750pt}%
\definecolor{currentstroke}{rgb}{0.121569,0.466667,0.705882}%
\pgfsetstrokecolor{currentstroke}%
\pgfsetdash{}{0pt}%
\pgfpathmoveto{\pgfqpoint{3.704758in}{1.373668in}}%
\pgfpathcurveto{\pgfqpoint{3.715808in}{1.373668in}}{\pgfqpoint{3.726407in}{1.378059in}}{\pgfqpoint{3.734221in}{1.385872in}}%
\pgfpathcurveto{\pgfqpoint{3.742035in}{1.393686in}}{\pgfqpoint{3.746425in}{1.404285in}}{\pgfqpoint{3.746425in}{1.415335in}}%
\pgfpathcurveto{\pgfqpoint{3.746425in}{1.426385in}}{\pgfqpoint{3.742035in}{1.436984in}}{\pgfqpoint{3.734221in}{1.444798in}}%
\pgfpathcurveto{\pgfqpoint{3.726407in}{1.452612in}}{\pgfqpoint{3.715808in}{1.457002in}}{\pgfqpoint{3.704758in}{1.457002in}}%
\pgfpathcurveto{\pgfqpoint{3.693708in}{1.457002in}}{\pgfqpoint{3.683109in}{1.452612in}}{\pgfqpoint{3.675295in}{1.444798in}}%
\pgfpathcurveto{\pgfqpoint{3.667482in}{1.436984in}}{\pgfqpoint{3.663092in}{1.426385in}}{\pgfqpoint{3.663092in}{1.415335in}}%
\pgfpathcurveto{\pgfqpoint{3.663092in}{1.404285in}}{\pgfqpoint{3.667482in}{1.393686in}}{\pgfqpoint{3.675295in}{1.385872in}}%
\pgfpathcurveto{\pgfqpoint{3.683109in}{1.378059in}}{\pgfqpoint{3.693708in}{1.373668in}}{\pgfqpoint{3.704758in}{1.373668in}}%
\pgfpathclose%
\pgfusepath{stroke,fill}%
\end{pgfscope}%
\begin{pgfscope}%
\pgfpathrectangle{\pgfqpoint{0.907100in}{0.499691in}}{\pgfqpoint{3.875000in}{2.695000in}}%
\pgfusepath{clip}%
\pgfsetbuttcap%
\pgfsetroundjoin%
\definecolor{currentfill}{rgb}{0.121569,0.466667,0.705882}%
\pgfsetfillcolor{currentfill}%
\pgfsetlinewidth{1.003750pt}%
\definecolor{currentstroke}{rgb}{0.121569,0.466667,0.705882}%
\pgfsetstrokecolor{currentstroke}%
\pgfsetdash{}{0pt}%
\pgfpathmoveto{\pgfqpoint{3.776888in}{1.346495in}}%
\pgfpathcurveto{\pgfqpoint{3.787939in}{1.346495in}}{\pgfqpoint{3.798538in}{1.350885in}}{\pgfqpoint{3.806351in}{1.358699in}}%
\pgfpathcurveto{\pgfqpoint{3.814165in}{1.366513in}}{\pgfqpoint{3.818555in}{1.377112in}}{\pgfqpoint{3.818555in}{1.388162in}}%
\pgfpathcurveto{\pgfqpoint{3.818555in}{1.399212in}}{\pgfqpoint{3.814165in}{1.409811in}}{\pgfqpoint{3.806351in}{1.417625in}}%
\pgfpathcurveto{\pgfqpoint{3.798538in}{1.425438in}}{\pgfqpoint{3.787939in}{1.429829in}}{\pgfqpoint{3.776888in}{1.429829in}}%
\pgfpathcurveto{\pgfqpoint{3.765838in}{1.429829in}}{\pgfqpoint{3.755239in}{1.425438in}}{\pgfqpoint{3.747426in}{1.417625in}}%
\pgfpathcurveto{\pgfqpoint{3.739612in}{1.409811in}}{\pgfqpoint{3.735222in}{1.399212in}}{\pgfqpoint{3.735222in}{1.388162in}}%
\pgfpathcurveto{\pgfqpoint{3.735222in}{1.377112in}}{\pgfqpoint{3.739612in}{1.366513in}}{\pgfqpoint{3.747426in}{1.358699in}}%
\pgfpathcurveto{\pgfqpoint{3.755239in}{1.350885in}}{\pgfqpoint{3.765838in}{1.346495in}}{\pgfqpoint{3.776888in}{1.346495in}}%
\pgfpathclose%
\pgfusepath{stroke,fill}%
\end{pgfscope}%
\begin{pgfscope}%
\pgfpathrectangle{\pgfqpoint{0.907100in}{0.499691in}}{\pgfqpoint{3.875000in}{2.695000in}}%
\pgfusepath{clip}%
\pgfsetbuttcap%
\pgfsetroundjoin%
\definecolor{currentfill}{rgb}{0.121569,0.466667,0.705882}%
\pgfsetfillcolor{currentfill}%
\pgfsetlinewidth{1.003750pt}%
\definecolor{currentstroke}{rgb}{0.121569,0.466667,0.705882}%
\pgfsetstrokecolor{currentstroke}%
\pgfsetdash{}{0pt}%
\pgfpathmoveto{\pgfqpoint{3.844814in}{1.307362in}}%
\pgfpathcurveto{\pgfqpoint{3.855865in}{1.307362in}}{\pgfqpoint{3.866464in}{1.311752in}}{\pgfqpoint{3.874277in}{1.319566in}}%
\pgfpathcurveto{\pgfqpoint{3.882091in}{1.327379in}}{\pgfqpoint{3.886481in}{1.337978in}}{\pgfqpoint{3.886481in}{1.349028in}}%
\pgfpathcurveto{\pgfqpoint{3.886481in}{1.360079in}}{\pgfqpoint{3.882091in}{1.370678in}}{\pgfqpoint{3.874277in}{1.378491in}}%
\pgfpathcurveto{\pgfqpoint{3.866464in}{1.386305in}}{\pgfqpoint{3.855865in}{1.390695in}}{\pgfqpoint{3.844814in}{1.390695in}}%
\pgfpathcurveto{\pgfqpoint{3.833764in}{1.390695in}}{\pgfqpoint{3.823165in}{1.386305in}}{\pgfqpoint{3.815352in}{1.378491in}}%
\pgfpathcurveto{\pgfqpoint{3.807538in}{1.370678in}}{\pgfqpoint{3.803148in}{1.360079in}}{\pgfqpoint{3.803148in}{1.349028in}}%
\pgfpathcurveto{\pgfqpoint{3.803148in}{1.337978in}}{\pgfqpoint{3.807538in}{1.327379in}}{\pgfqpoint{3.815352in}{1.319566in}}%
\pgfpathcurveto{\pgfqpoint{3.823165in}{1.311752in}}{\pgfqpoint{3.833764in}{1.307362in}}{\pgfqpoint{3.844814in}{1.307362in}}%
\pgfpathclose%
\pgfusepath{stroke,fill}%
\end{pgfscope}%
\begin{pgfscope}%
\pgfpathrectangle{\pgfqpoint{0.907100in}{0.499691in}}{\pgfqpoint{3.875000in}{2.695000in}}%
\pgfusepath{clip}%
\pgfsetbuttcap%
\pgfsetroundjoin%
\definecolor{currentfill}{rgb}{0.121569,0.466667,0.705882}%
\pgfsetfillcolor{currentfill}%
\pgfsetlinewidth{1.003750pt}%
\definecolor{currentstroke}{rgb}{0.121569,0.466667,0.705882}%
\pgfsetstrokecolor{currentstroke}%
\pgfsetdash{}{0pt}%
\pgfpathmoveto{\pgfqpoint{3.909000in}{1.282279in}}%
\pgfpathcurveto{\pgfqpoint{3.920050in}{1.282279in}}{\pgfqpoint{3.930649in}{1.286669in}}{\pgfqpoint{3.938462in}{1.294483in}}%
\pgfpathcurveto{\pgfqpoint{3.946276in}{1.302297in}}{\pgfqpoint{3.950666in}{1.312896in}}{\pgfqpoint{3.950666in}{1.323946in}}%
\pgfpathcurveto{\pgfqpoint{3.950666in}{1.334996in}}{\pgfqpoint{3.946276in}{1.345595in}}{\pgfqpoint{3.938462in}{1.353409in}}%
\pgfpathcurveto{\pgfqpoint{3.930649in}{1.361222in}}{\pgfqpoint{3.920050in}{1.365612in}}{\pgfqpoint{3.909000in}{1.365612in}}%
\pgfpathcurveto{\pgfqpoint{3.897949in}{1.365612in}}{\pgfqpoint{3.887350in}{1.361222in}}{\pgfqpoint{3.879537in}{1.353409in}}%
\pgfpathcurveto{\pgfqpoint{3.871723in}{1.345595in}}{\pgfqpoint{3.867333in}{1.334996in}}{\pgfqpoint{3.867333in}{1.323946in}}%
\pgfpathcurveto{\pgfqpoint{3.867333in}{1.312896in}}{\pgfqpoint{3.871723in}{1.302297in}}{\pgfqpoint{3.879537in}{1.294483in}}%
\pgfpathcurveto{\pgfqpoint{3.887350in}{1.286669in}}{\pgfqpoint{3.897949in}{1.282279in}}{\pgfqpoint{3.909000in}{1.282279in}}%
\pgfpathclose%
\pgfusepath{stroke,fill}%
\end{pgfscope}%
\begin{pgfscope}%
\pgfpathrectangle{\pgfqpoint{0.907100in}{0.499691in}}{\pgfqpoint{3.875000in}{2.695000in}}%
\pgfusepath{clip}%
\pgfsetbuttcap%
\pgfsetroundjoin%
\definecolor{currentfill}{rgb}{0.121569,0.466667,0.705882}%
\pgfsetfillcolor{currentfill}%
\pgfsetlinewidth{1.003750pt}%
\definecolor{currentstroke}{rgb}{0.121569,0.466667,0.705882}%
\pgfsetstrokecolor{currentstroke}%
\pgfsetdash{}{0pt}%
\pgfpathmoveto{\pgfqpoint{3.969834in}{1.263961in}}%
\pgfpathcurveto{\pgfqpoint{3.980884in}{1.263961in}}{\pgfqpoint{3.991483in}{1.268351in}}{\pgfqpoint{3.999297in}{1.276165in}}%
\pgfpathcurveto{\pgfqpoint{4.007111in}{1.283978in}}{\pgfqpoint{4.011501in}{1.294578in}}{\pgfqpoint{4.011501in}{1.305628in}}%
\pgfpathcurveto{\pgfqpoint{4.011501in}{1.316678in}}{\pgfqpoint{4.007111in}{1.327277in}}{\pgfqpoint{3.999297in}{1.335090in}}%
\pgfpathcurveto{\pgfqpoint{3.991483in}{1.342904in}}{\pgfqpoint{3.980884in}{1.347294in}}{\pgfqpoint{3.969834in}{1.347294in}}%
\pgfpathcurveto{\pgfqpoint{3.958784in}{1.347294in}}{\pgfqpoint{3.948185in}{1.342904in}}{\pgfqpoint{3.940371in}{1.335090in}}%
\pgfpathcurveto{\pgfqpoint{3.932558in}{1.327277in}}{\pgfqpoint{3.928168in}{1.316678in}}{\pgfqpoint{3.928168in}{1.305628in}}%
\pgfpathcurveto{\pgfqpoint{3.928168in}{1.294578in}}{\pgfqpoint{3.932558in}{1.283978in}}{\pgfqpoint{3.940371in}{1.276165in}}%
\pgfpathcurveto{\pgfqpoint{3.948185in}{1.268351in}}{\pgfqpoint{3.958784in}{1.263961in}}{\pgfqpoint{3.969834in}{1.263961in}}%
\pgfpathclose%
\pgfusepath{stroke,fill}%
\end{pgfscope}%
\begin{pgfscope}%
\pgfpathrectangle{\pgfqpoint{0.907100in}{0.499691in}}{\pgfqpoint{3.875000in}{2.695000in}}%
\pgfusepath{clip}%
\pgfsetbuttcap%
\pgfsetroundjoin%
\definecolor{currentfill}{rgb}{0.121569,0.466667,0.705882}%
\pgfsetfillcolor{currentfill}%
\pgfsetlinewidth{1.003750pt}%
\definecolor{currentstroke}{rgb}{0.121569,0.466667,0.705882}%
\pgfsetstrokecolor{currentstroke}%
\pgfsetdash{}{0pt}%
\pgfpathmoveto{\pgfqpoint{4.027651in}{1.240160in}}%
\pgfpathcurveto{\pgfqpoint{4.038701in}{1.240160in}}{\pgfqpoint{4.049300in}{1.244550in}}{\pgfqpoint{4.057114in}{1.252364in}}%
\pgfpathcurveto{\pgfqpoint{4.064927in}{1.260178in}}{\pgfqpoint{4.069318in}{1.270777in}}{\pgfqpoint{4.069318in}{1.281827in}}%
\pgfpathcurveto{\pgfqpoint{4.069318in}{1.292877in}}{\pgfqpoint{4.064927in}{1.303476in}}{\pgfqpoint{4.057114in}{1.311290in}}%
\pgfpathcurveto{\pgfqpoint{4.049300in}{1.319103in}}{\pgfqpoint{4.038701in}{1.323493in}}{\pgfqpoint{4.027651in}{1.323493in}}%
\pgfpathcurveto{\pgfqpoint{4.016601in}{1.323493in}}{\pgfqpoint{4.006002in}{1.319103in}}{\pgfqpoint{3.998188in}{1.311290in}}%
\pgfpathcurveto{\pgfqpoint{3.990375in}{1.303476in}}{\pgfqpoint{3.985984in}{1.292877in}}{\pgfqpoint{3.985984in}{1.281827in}}%
\pgfpathcurveto{\pgfqpoint{3.985984in}{1.270777in}}{\pgfqpoint{3.990375in}{1.260178in}}{\pgfqpoint{3.998188in}{1.252364in}}%
\pgfpathcurveto{\pgfqpoint{4.006002in}{1.244550in}}{\pgfqpoint{4.016601in}{1.240160in}}{\pgfqpoint{4.027651in}{1.240160in}}%
\pgfpathclose%
\pgfusepath{stroke,fill}%
\end{pgfscope}%
\begin{pgfscope}%
\pgfpathrectangle{\pgfqpoint{0.907100in}{0.499691in}}{\pgfqpoint{3.875000in}{2.695000in}}%
\pgfusepath{clip}%
\pgfsetbuttcap%
\pgfsetroundjoin%
\definecolor{currentfill}{rgb}{0.121569,0.466667,0.705882}%
\pgfsetfillcolor{currentfill}%
\pgfsetlinewidth{1.003750pt}%
\definecolor{currentstroke}{rgb}{0.121569,0.466667,0.705882}%
\pgfsetstrokecolor{currentstroke}%
\pgfsetdash{}{0pt}%
\pgfpathmoveto{\pgfqpoint{4.082735in}{1.234317in}}%
\pgfpathcurveto{\pgfqpoint{4.093786in}{1.234317in}}{\pgfqpoint{4.104385in}{1.238707in}}{\pgfqpoint{4.112198in}{1.246521in}}%
\pgfpathcurveto{\pgfqpoint{4.120012in}{1.254334in}}{\pgfqpoint{4.124402in}{1.264933in}}{\pgfqpoint{4.124402in}{1.275984in}}%
\pgfpathcurveto{\pgfqpoint{4.124402in}{1.287034in}}{\pgfqpoint{4.120012in}{1.297633in}}{\pgfqpoint{4.112198in}{1.305446in}}%
\pgfpathcurveto{\pgfqpoint{4.104385in}{1.313260in}}{\pgfqpoint{4.093786in}{1.317650in}}{\pgfqpoint{4.082735in}{1.317650in}}%
\pgfpathcurveto{\pgfqpoint{4.071685in}{1.317650in}}{\pgfqpoint{4.061086in}{1.313260in}}{\pgfqpoint{4.053273in}{1.305446in}}%
\pgfpathcurveto{\pgfqpoint{4.045459in}{1.297633in}}{\pgfqpoint{4.041069in}{1.287034in}}{\pgfqpoint{4.041069in}{1.275984in}}%
\pgfpathcurveto{\pgfqpoint{4.041069in}{1.264933in}}{\pgfqpoint{4.045459in}{1.254334in}}{\pgfqpoint{4.053273in}{1.246521in}}%
\pgfpathcurveto{\pgfqpoint{4.061086in}{1.238707in}}{\pgfqpoint{4.071685in}{1.234317in}}{\pgfqpoint{4.082735in}{1.234317in}}%
\pgfpathclose%
\pgfusepath{stroke,fill}%
\end{pgfscope}%
\begin{pgfscope}%
\pgfpathrectangle{\pgfqpoint{0.907100in}{0.499691in}}{\pgfqpoint{3.875000in}{2.695000in}}%
\pgfusepath{clip}%
\pgfsetbuttcap%
\pgfsetroundjoin%
\definecolor{currentfill}{rgb}{0.121569,0.466667,0.705882}%
\pgfsetfillcolor{currentfill}%
\pgfsetlinewidth{1.003750pt}%
\definecolor{currentstroke}{rgb}{0.121569,0.466667,0.705882}%
\pgfsetstrokecolor{currentstroke}%
\pgfsetdash{}{0pt}%
\pgfpathmoveto{\pgfqpoint{4.135334in}{1.211354in}}%
\pgfpathcurveto{\pgfqpoint{4.146384in}{1.211354in}}{\pgfqpoint{4.156983in}{1.215744in}}{\pgfqpoint{4.164797in}{1.223558in}}%
\pgfpathcurveto{\pgfqpoint{4.172610in}{1.231372in}}{\pgfqpoint{4.177000in}{1.241971in}}{\pgfqpoint{4.177000in}{1.253021in}}%
\pgfpathcurveto{\pgfqpoint{4.177000in}{1.264071in}}{\pgfqpoint{4.172610in}{1.274670in}}{\pgfqpoint{4.164797in}{1.282484in}}%
\pgfpathcurveto{\pgfqpoint{4.156983in}{1.290297in}}{\pgfqpoint{4.146384in}{1.294687in}}{\pgfqpoint{4.135334in}{1.294687in}}%
\pgfpathcurveto{\pgfqpoint{4.124284in}{1.294687in}}{\pgfqpoint{4.113685in}{1.290297in}}{\pgfqpoint{4.105871in}{1.282484in}}%
\pgfpathcurveto{\pgfqpoint{4.098057in}{1.274670in}}{\pgfqpoint{4.093667in}{1.264071in}}{\pgfqpoint{4.093667in}{1.253021in}}%
\pgfpathcurveto{\pgfqpoint{4.093667in}{1.241971in}}{\pgfqpoint{4.098057in}{1.231372in}}{\pgfqpoint{4.105871in}{1.223558in}}%
\pgfpathcurveto{\pgfqpoint{4.113685in}{1.215744in}}{\pgfqpoint{4.124284in}{1.211354in}}{\pgfqpoint{4.135334in}{1.211354in}}%
\pgfpathclose%
\pgfusepath{stroke,fill}%
\end{pgfscope}%
\begin{pgfscope}%
\pgfpathrectangle{\pgfqpoint{0.907100in}{0.499691in}}{\pgfqpoint{3.875000in}{2.695000in}}%
\pgfusepath{clip}%
\pgfsetbuttcap%
\pgfsetroundjoin%
\definecolor{currentfill}{rgb}{0.121569,0.466667,0.705882}%
\pgfsetfillcolor{currentfill}%
\pgfsetlinewidth{1.003750pt}%
\definecolor{currentstroke}{rgb}{0.121569,0.466667,0.705882}%
\pgfsetstrokecolor{currentstroke}%
\pgfsetdash{}{0pt}%
\pgfpathmoveto{\pgfqpoint{4.185661in}{1.200111in}}%
\pgfpathcurveto{\pgfqpoint{4.196711in}{1.200111in}}{\pgfqpoint{4.207310in}{1.204501in}}{\pgfqpoint{4.215124in}{1.212315in}}%
\pgfpathcurveto{\pgfqpoint{4.222937in}{1.220128in}}{\pgfqpoint{4.227328in}{1.230727in}}{\pgfqpoint{4.227328in}{1.241778in}}%
\pgfpathcurveto{\pgfqpoint{4.227328in}{1.252828in}}{\pgfqpoint{4.222937in}{1.263427in}}{\pgfqpoint{4.215124in}{1.271240in}}%
\pgfpathcurveto{\pgfqpoint{4.207310in}{1.279054in}}{\pgfqpoint{4.196711in}{1.283444in}}{\pgfqpoint{4.185661in}{1.283444in}}%
\pgfpathcurveto{\pgfqpoint{4.174611in}{1.283444in}}{\pgfqpoint{4.164012in}{1.279054in}}{\pgfqpoint{4.156198in}{1.271240in}}%
\pgfpathcurveto{\pgfqpoint{4.148385in}{1.263427in}}{\pgfqpoint{4.143994in}{1.252828in}}{\pgfqpoint{4.143994in}{1.241778in}}%
\pgfpathcurveto{\pgfqpoint{4.143994in}{1.230727in}}{\pgfqpoint{4.148385in}{1.220128in}}{\pgfqpoint{4.156198in}{1.212315in}}%
\pgfpathcurveto{\pgfqpoint{4.164012in}{1.204501in}}{\pgfqpoint{4.174611in}{1.200111in}}{\pgfqpoint{4.185661in}{1.200111in}}%
\pgfpathclose%
\pgfusepath{stroke,fill}%
\end{pgfscope}%
\begin{pgfscope}%
\pgfpathrectangle{\pgfqpoint{0.907100in}{0.499691in}}{\pgfqpoint{3.875000in}{2.695000in}}%
\pgfusepath{clip}%
\pgfsetbuttcap%
\pgfsetroundjoin%
\definecolor{currentfill}{rgb}{0.121569,0.466667,0.705882}%
\pgfsetfillcolor{currentfill}%
\pgfsetlinewidth{1.003750pt}%
\definecolor{currentstroke}{rgb}{0.121569,0.466667,0.705882}%
\pgfsetstrokecolor{currentstroke}%
\pgfsetdash{}{0pt}%
\pgfpathmoveto{\pgfqpoint{4.233905in}{1.189021in}}%
\pgfpathcurveto{\pgfqpoint{4.244955in}{1.189021in}}{\pgfqpoint{4.255554in}{1.193411in}}{\pgfqpoint{4.263368in}{1.201225in}}%
\pgfpathcurveto{\pgfqpoint{4.271182in}{1.209038in}}{\pgfqpoint{4.275572in}{1.219637in}}{\pgfqpoint{4.275572in}{1.230687in}}%
\pgfpathcurveto{\pgfqpoint{4.275572in}{1.241737in}}{\pgfqpoint{4.271182in}{1.252336in}}{\pgfqpoint{4.263368in}{1.260150in}}%
\pgfpathcurveto{\pgfqpoint{4.255554in}{1.267964in}}{\pgfqpoint{4.244955in}{1.272354in}}{\pgfqpoint{4.233905in}{1.272354in}}%
\pgfpathcurveto{\pgfqpoint{4.222855in}{1.272354in}}{\pgfqpoint{4.212256in}{1.267964in}}{\pgfqpoint{4.204442in}{1.260150in}}%
\pgfpathcurveto{\pgfqpoint{4.196629in}{1.252336in}}{\pgfqpoint{4.192239in}{1.241737in}}{\pgfqpoint{4.192239in}{1.230687in}}%
\pgfpathcurveto{\pgfqpoint{4.192239in}{1.219637in}}{\pgfqpoint{4.196629in}{1.209038in}}{\pgfqpoint{4.204442in}{1.201225in}}%
\pgfpathcurveto{\pgfqpoint{4.212256in}{1.193411in}}{\pgfqpoint{4.222855in}{1.189021in}}{\pgfqpoint{4.233905in}{1.189021in}}%
\pgfpathclose%
\pgfusepath{stroke,fill}%
\end{pgfscope}%
\begin{pgfscope}%
\pgfpathrectangle{\pgfqpoint{0.907100in}{0.499691in}}{\pgfqpoint{3.875000in}{2.695000in}}%
\pgfusepath{clip}%
\pgfsetbuttcap%
\pgfsetroundjoin%
\definecolor{currentfill}{rgb}{0.121569,0.466667,0.705882}%
\pgfsetfillcolor{currentfill}%
\pgfsetlinewidth{1.003750pt}%
\definecolor{currentstroke}{rgb}{0.121569,0.466667,0.705882}%
\pgfsetstrokecolor{currentstroke}%
\pgfsetdash{}{0pt}%
\pgfpathmoveto{\pgfqpoint{4.280232in}{1.167283in}}%
\pgfpathcurveto{\pgfqpoint{4.291282in}{1.167283in}}{\pgfqpoint{4.301881in}{1.171673in}}{\pgfqpoint{4.309695in}{1.179487in}}%
\pgfpathcurveto{\pgfqpoint{4.317508in}{1.187300in}}{\pgfqpoint{4.321899in}{1.197899in}}{\pgfqpoint{4.321899in}{1.208949in}}%
\pgfpathcurveto{\pgfqpoint{4.321899in}{1.220000in}}{\pgfqpoint{4.317508in}{1.230599in}}{\pgfqpoint{4.309695in}{1.238412in}}%
\pgfpathcurveto{\pgfqpoint{4.301881in}{1.246226in}}{\pgfqpoint{4.291282in}{1.250616in}}{\pgfqpoint{4.280232in}{1.250616in}}%
\pgfpathcurveto{\pgfqpoint{4.269182in}{1.250616in}}{\pgfqpoint{4.258583in}{1.246226in}}{\pgfqpoint{4.250769in}{1.238412in}}%
\pgfpathcurveto{\pgfqpoint{4.242955in}{1.230599in}}{\pgfqpoint{4.238565in}{1.220000in}}{\pgfqpoint{4.238565in}{1.208949in}}%
\pgfpathcurveto{\pgfqpoint{4.238565in}{1.197899in}}{\pgfqpoint{4.242955in}{1.187300in}}{\pgfqpoint{4.250769in}{1.179487in}}%
\pgfpathcurveto{\pgfqpoint{4.258583in}{1.171673in}}{\pgfqpoint{4.269182in}{1.167283in}}{\pgfqpoint{4.280232in}{1.167283in}}%
\pgfpathclose%
\pgfusepath{stroke,fill}%
\end{pgfscope}%
\begin{pgfscope}%
\pgfpathrectangle{\pgfqpoint{0.907100in}{0.499691in}}{\pgfqpoint{3.875000in}{2.695000in}}%
\pgfusepath{clip}%
\pgfsetbuttcap%
\pgfsetroundjoin%
\definecolor{currentfill}{rgb}{0.121569,0.466667,0.705882}%
\pgfsetfillcolor{currentfill}%
\pgfsetlinewidth{1.003750pt}%
\definecolor{currentstroke}{rgb}{0.121569,0.466667,0.705882}%
\pgfsetstrokecolor{currentstroke}%
\pgfsetdash{}{0pt}%
\pgfpathmoveto{\pgfqpoint{4.324788in}{1.146110in}}%
\pgfpathcurveto{\pgfqpoint{4.335838in}{1.146110in}}{\pgfqpoint{4.346437in}{1.150500in}}{\pgfqpoint{4.354250in}{1.158313in}}%
\pgfpathcurveto{\pgfqpoint{4.362064in}{1.166127in}}{\pgfqpoint{4.366454in}{1.176726in}}{\pgfqpoint{4.366454in}{1.187776in}}%
\pgfpathcurveto{\pgfqpoint{4.366454in}{1.198826in}}{\pgfqpoint{4.362064in}{1.209425in}}{\pgfqpoint{4.354250in}{1.217239in}}%
\pgfpathcurveto{\pgfqpoint{4.346437in}{1.225053in}}{\pgfqpoint{4.335838in}{1.229443in}}{\pgfqpoint{4.324788in}{1.229443in}}%
\pgfpathcurveto{\pgfqpoint{4.313738in}{1.229443in}}{\pgfqpoint{4.303138in}{1.225053in}}{\pgfqpoint{4.295325in}{1.217239in}}%
\pgfpathcurveto{\pgfqpoint{4.287511in}{1.209425in}}{\pgfqpoint{4.283121in}{1.198826in}}{\pgfqpoint{4.283121in}{1.187776in}}%
\pgfpathcurveto{\pgfqpoint{4.283121in}{1.176726in}}{\pgfqpoint{4.287511in}{1.166127in}}{\pgfqpoint{4.295325in}{1.158313in}}%
\pgfpathcurveto{\pgfqpoint{4.303138in}{1.150500in}}{\pgfqpoint{4.313738in}{1.146110in}}{\pgfqpoint{4.324788in}{1.146110in}}%
\pgfpathclose%
\pgfusepath{stroke,fill}%
\end{pgfscope}%
\begin{pgfscope}%
\pgfpathrectangle{\pgfqpoint{0.907100in}{0.499691in}}{\pgfqpoint{3.875000in}{2.695000in}}%
\pgfusepath{clip}%
\pgfsetbuttcap%
\pgfsetroundjoin%
\definecolor{currentfill}{rgb}{0.121569,0.466667,0.705882}%
\pgfsetfillcolor{currentfill}%
\pgfsetlinewidth{1.003750pt}%
\definecolor{currentstroke}{rgb}{0.121569,0.466667,0.705882}%
\pgfsetstrokecolor{currentstroke}%
\pgfsetdash{}{0pt}%
\pgfpathmoveto{\pgfqpoint{4.367703in}{1.146110in}}%
\pgfpathcurveto{\pgfqpoint{4.378753in}{1.146110in}}{\pgfqpoint{4.389352in}{1.150500in}}{\pgfqpoint{4.397166in}{1.158313in}}%
\pgfpathcurveto{\pgfqpoint{4.404979in}{1.166127in}}{\pgfqpoint{4.409370in}{1.176726in}}{\pgfqpoint{4.409370in}{1.187776in}}%
\pgfpathcurveto{\pgfqpoint{4.409370in}{1.198826in}}{\pgfqpoint{4.404979in}{1.209425in}}{\pgfqpoint{4.397166in}{1.217239in}}%
\pgfpathcurveto{\pgfqpoint{4.389352in}{1.225053in}}{\pgfqpoint{4.378753in}{1.229443in}}{\pgfqpoint{4.367703in}{1.229443in}}%
\pgfpathcurveto{\pgfqpoint{4.356653in}{1.229443in}}{\pgfqpoint{4.346054in}{1.225053in}}{\pgfqpoint{4.338240in}{1.217239in}}%
\pgfpathcurveto{\pgfqpoint{4.330427in}{1.209425in}}{\pgfqpoint{4.326036in}{1.198826in}}{\pgfqpoint{4.326036in}{1.187776in}}%
\pgfpathcurveto{\pgfqpoint{4.326036in}{1.176726in}}{\pgfqpoint{4.330427in}{1.166127in}}{\pgfqpoint{4.338240in}{1.158313in}}%
\pgfpathcurveto{\pgfqpoint{4.346054in}{1.150500in}}{\pgfqpoint{4.356653in}{1.146110in}}{\pgfqpoint{4.367703in}{1.146110in}}%
\pgfpathclose%
\pgfusepath{stroke,fill}%
\end{pgfscope}%
\begin{pgfscope}%
\pgfpathrectangle{\pgfqpoint{0.907100in}{0.499691in}}{\pgfqpoint{3.875000in}{2.695000in}}%
\pgfusepath{clip}%
\pgfsetbuttcap%
\pgfsetroundjoin%
\definecolor{currentfill}{rgb}{0.121569,0.466667,0.705882}%
\pgfsetfillcolor{currentfill}%
\pgfsetlinewidth{1.003750pt}%
\definecolor{currentstroke}{rgb}{0.121569,0.466667,0.705882}%
\pgfsetstrokecolor{currentstroke}%
\pgfsetdash{}{0pt}%
\pgfpathmoveto{\pgfqpoint{4.409094in}{1.125472in}}%
\pgfpathcurveto{\pgfqpoint{4.420144in}{1.125472in}}{\pgfqpoint{4.430743in}{1.129863in}}{\pgfqpoint{4.438557in}{1.137676in}}%
\pgfpathcurveto{\pgfqpoint{4.446371in}{1.145490in}}{\pgfqpoint{4.450761in}{1.156089in}}{\pgfqpoint{4.450761in}{1.167139in}}%
\pgfpathcurveto{\pgfqpoint{4.450761in}{1.178189in}}{\pgfqpoint{4.446371in}{1.188788in}}{\pgfqpoint{4.438557in}{1.196602in}}%
\pgfpathcurveto{\pgfqpoint{4.430743in}{1.204415in}}{\pgfqpoint{4.420144in}{1.208806in}}{\pgfqpoint{4.409094in}{1.208806in}}%
\pgfpathcurveto{\pgfqpoint{4.398044in}{1.208806in}}{\pgfqpoint{4.387445in}{1.204415in}}{\pgfqpoint{4.379632in}{1.196602in}}%
\pgfpathcurveto{\pgfqpoint{4.371818in}{1.188788in}}{\pgfqpoint{4.367428in}{1.178189in}}{\pgfqpoint{4.367428in}{1.167139in}}%
\pgfpathcurveto{\pgfqpoint{4.367428in}{1.156089in}}{\pgfqpoint{4.371818in}{1.145490in}}{\pgfqpoint{4.379632in}{1.137676in}}%
\pgfpathcurveto{\pgfqpoint{4.387445in}{1.129863in}}{\pgfqpoint{4.398044in}{1.125472in}}{\pgfqpoint{4.409094in}{1.125472in}}%
\pgfpathclose%
\pgfusepath{stroke,fill}%
\end{pgfscope}%
\begin{pgfscope}%
\pgfpathrectangle{\pgfqpoint{0.907100in}{0.499691in}}{\pgfqpoint{3.875000in}{2.695000in}}%
\pgfusepath{clip}%
\pgfsetbuttcap%
\pgfsetroundjoin%
\definecolor{currentfill}{rgb}{0.121569,0.466667,0.705882}%
\pgfsetfillcolor{currentfill}%
\pgfsetlinewidth{1.003750pt}%
\definecolor{currentstroke}{rgb}{0.121569,0.466667,0.705882}%
\pgfsetstrokecolor{currentstroke}%
\pgfsetdash{}{0pt}%
\pgfpathmoveto{\pgfqpoint{4.449066in}{1.125472in}}%
\pgfpathcurveto{\pgfqpoint{4.460116in}{1.125472in}}{\pgfqpoint{4.470715in}{1.129863in}}{\pgfqpoint{4.478529in}{1.137676in}}%
\pgfpathcurveto{\pgfqpoint{4.486343in}{1.145490in}}{\pgfqpoint{4.490733in}{1.156089in}}{\pgfqpoint{4.490733in}{1.167139in}}%
\pgfpathcurveto{\pgfqpoint{4.490733in}{1.178189in}}{\pgfqpoint{4.486343in}{1.188788in}}{\pgfqpoint{4.478529in}{1.196602in}}%
\pgfpathcurveto{\pgfqpoint{4.470715in}{1.204415in}}{\pgfqpoint{4.460116in}{1.208806in}}{\pgfqpoint{4.449066in}{1.208806in}}%
\pgfpathcurveto{\pgfqpoint{4.438016in}{1.208806in}}{\pgfqpoint{4.427417in}{1.204415in}}{\pgfqpoint{4.419603in}{1.196602in}}%
\pgfpathcurveto{\pgfqpoint{4.411790in}{1.188788in}}{\pgfqpoint{4.407400in}{1.178189in}}{\pgfqpoint{4.407400in}{1.167139in}}%
\pgfpathcurveto{\pgfqpoint{4.407400in}{1.156089in}}{\pgfqpoint{4.411790in}{1.145490in}}{\pgfqpoint{4.419603in}{1.137676in}}%
\pgfpathcurveto{\pgfqpoint{4.427417in}{1.129863in}}{\pgfqpoint{4.438016in}{1.125472in}}{\pgfqpoint{4.449066in}{1.125472in}}%
\pgfpathclose%
\pgfusepath{stroke,fill}%
\end{pgfscope}%
\begin{pgfscope}%
\pgfpathrectangle{\pgfqpoint{0.907100in}{0.499691in}}{\pgfqpoint{3.875000in}{2.695000in}}%
\pgfusepath{clip}%
\pgfsetbuttcap%
\pgfsetroundjoin%
\definecolor{currentfill}{rgb}{0.121569,0.466667,0.705882}%
\pgfsetfillcolor{currentfill}%
\pgfsetlinewidth{1.003750pt}%
\definecolor{currentstroke}{rgb}{0.121569,0.466667,0.705882}%
\pgfsetstrokecolor{currentstroke}%
\pgfsetdash{}{0pt}%
\pgfpathmoveto{\pgfqpoint{4.487713in}{1.105345in}}%
\pgfpathcurveto{\pgfqpoint{4.498763in}{1.105345in}}{\pgfqpoint{4.509362in}{1.109735in}}{\pgfqpoint{4.517176in}{1.117549in}}%
\pgfpathcurveto{\pgfqpoint{4.524989in}{1.125362in}}{\pgfqpoint{4.529380in}{1.135961in}}{\pgfqpoint{4.529380in}{1.147012in}}%
\pgfpathcurveto{\pgfqpoint{4.529380in}{1.158062in}}{\pgfqpoint{4.524989in}{1.168661in}}{\pgfqpoint{4.517176in}{1.176474in}}%
\pgfpathcurveto{\pgfqpoint{4.509362in}{1.184288in}}{\pgfqpoint{4.498763in}{1.188678in}}{\pgfqpoint{4.487713in}{1.188678in}}%
\pgfpathcurveto{\pgfqpoint{4.476663in}{1.188678in}}{\pgfqpoint{4.466064in}{1.184288in}}{\pgfqpoint{4.458250in}{1.176474in}}%
\pgfpathcurveto{\pgfqpoint{4.450436in}{1.168661in}}{\pgfqpoint{4.446046in}{1.158062in}}{\pgfqpoint{4.446046in}{1.147012in}}%
\pgfpathcurveto{\pgfqpoint{4.446046in}{1.135961in}}{\pgfqpoint{4.450436in}{1.125362in}}{\pgfqpoint{4.458250in}{1.117549in}}%
\pgfpathcurveto{\pgfqpoint{4.466064in}{1.109735in}}{\pgfqpoint{4.476663in}{1.105345in}}{\pgfqpoint{4.487713in}{1.105345in}}%
\pgfpathclose%
\pgfusepath{stroke,fill}%
\end{pgfscope}%
\begin{pgfscope}%
\pgfpathrectangle{\pgfqpoint{0.907100in}{0.499691in}}{\pgfqpoint{3.875000in}{2.695000in}}%
\pgfusepath{clip}%
\pgfsetbuttcap%
\pgfsetroundjoin%
\definecolor{currentfill}{rgb}{0.121569,0.466667,0.705882}%
\pgfsetfillcolor{currentfill}%
\pgfsetlinewidth{1.003750pt}%
\definecolor{currentstroke}{rgb}{0.121569,0.466667,0.705882}%
\pgfsetstrokecolor{currentstroke}%
\pgfsetdash{}{0pt}%
\pgfpathmoveto{\pgfqpoint{4.525119in}{1.105345in}}%
\pgfpathcurveto{\pgfqpoint{4.536169in}{1.105345in}}{\pgfqpoint{4.546768in}{1.109735in}}{\pgfqpoint{4.554582in}{1.117549in}}%
\pgfpathcurveto{\pgfqpoint{4.562396in}{1.125362in}}{\pgfqpoint{4.566786in}{1.135961in}}{\pgfqpoint{4.566786in}{1.147012in}}%
\pgfpathcurveto{\pgfqpoint{4.566786in}{1.158062in}}{\pgfqpoint{4.562396in}{1.168661in}}{\pgfqpoint{4.554582in}{1.176474in}}%
\pgfpathcurveto{\pgfqpoint{4.546768in}{1.184288in}}{\pgfqpoint{4.536169in}{1.188678in}}{\pgfqpoint{4.525119in}{1.188678in}}%
\pgfpathcurveto{\pgfqpoint{4.514069in}{1.188678in}}{\pgfqpoint{4.503470in}{1.184288in}}{\pgfqpoint{4.495657in}{1.176474in}}%
\pgfpathcurveto{\pgfqpoint{4.487843in}{1.168661in}}{\pgfqpoint{4.483453in}{1.158062in}}{\pgfqpoint{4.483453in}{1.147012in}}%
\pgfpathcurveto{\pgfqpoint{4.483453in}{1.135961in}}{\pgfqpoint{4.487843in}{1.125362in}}{\pgfqpoint{4.495657in}{1.117549in}}%
\pgfpathcurveto{\pgfqpoint{4.503470in}{1.109735in}}{\pgfqpoint{4.514069in}{1.105345in}}{\pgfqpoint{4.525119in}{1.105345in}}%
\pgfpathclose%
\pgfusepath{stroke,fill}%
\end{pgfscope}%
\begin{pgfscope}%
\pgfsetrectcap%
\pgfsetmiterjoin%
\pgfsetlinewidth{0.803000pt}%
\definecolor{currentstroke}{rgb}{0.000000,0.000000,0.000000}%
\pgfsetstrokecolor{currentstroke}%
\pgfsetdash{}{0pt}%
\pgfpathmoveto{\pgfqpoint{0.907100in}{0.499691in}}%
\pgfpathlineto{\pgfqpoint{0.907100in}{3.194691in}}%
\pgfusepath{stroke}%
\end{pgfscope}%
\begin{pgfscope}%
\pgfsetrectcap%
\pgfsetmiterjoin%
\pgfsetlinewidth{0.803000pt}%
\definecolor{currentstroke}{rgb}{0.000000,0.000000,0.000000}%
\pgfsetstrokecolor{currentstroke}%
\pgfsetdash{}{0pt}%
\pgfpathmoveto{\pgfqpoint{4.782100in}{0.499691in}}%
\pgfpathlineto{\pgfqpoint{4.782100in}{3.194691in}}%
\pgfusepath{stroke}%
\end{pgfscope}%
\begin{pgfscope}%
\pgfsetrectcap%
\pgfsetmiterjoin%
\pgfsetlinewidth{0.803000pt}%
\definecolor{currentstroke}{rgb}{0.000000,0.000000,0.000000}%
\pgfsetstrokecolor{currentstroke}%
\pgfsetdash{}{0pt}%
\pgfpathmoveto{\pgfqpoint{0.907100in}{0.499691in}}%
\pgfpathlineto{\pgfqpoint{4.782100in}{0.499691in}}%
\pgfusepath{stroke}%
\end{pgfscope}%
\begin{pgfscope}%
\pgfsetrectcap%
\pgfsetmiterjoin%
\pgfsetlinewidth{0.803000pt}%
\definecolor{currentstroke}{rgb}{0.000000,0.000000,0.000000}%
\pgfsetstrokecolor{currentstroke}%
\pgfsetdash{}{0pt}%
\pgfpathmoveto{\pgfqpoint{0.907100in}{3.194691in}}%
\pgfpathlineto{\pgfqpoint{4.782100in}{3.194691in}}%
\pgfusepath{stroke}%
\end{pgfscope}%
\begin{pgfscope}%
\pgfpathrectangle{\pgfqpoint{0.907100in}{0.499691in}}{\pgfqpoint{3.875000in}{2.695000in}}%
\pgfusepath{clip}%
\pgfsetbuttcap%
\pgfsetroundjoin%
\definecolor{currentfill}{rgb}{0.117647,0.564706,1.000000}%
\pgfsetfillcolor{currentfill}%
\pgfsetlinewidth{1.003750pt}%
\definecolor{currentstroke}{rgb}{0.117647,0.564706,1.000000}%
\pgfsetstrokecolor{currentstroke}%
\pgfsetdash{}{0pt}%
\pgfpathmoveto{\pgfqpoint{4.106076in}{0.923456in}}%
\pgfpathcurveto{\pgfqpoint{4.117126in}{0.923456in}}{\pgfqpoint{4.127725in}{0.927846in}}{\pgfqpoint{4.135539in}{0.935659in}}%
\pgfpathcurveto{\pgfqpoint{4.143352in}{0.943473in}}{\pgfqpoint{4.147743in}{0.954072in}}{\pgfqpoint{4.147743in}{0.965122in}}%
\pgfpathcurveto{\pgfqpoint{4.147743in}{0.976172in}}{\pgfqpoint{4.143352in}{0.986771in}}{\pgfqpoint{4.135539in}{0.994585in}}%
\pgfpathcurveto{\pgfqpoint{4.127725in}{1.002399in}}{\pgfqpoint{4.117126in}{1.006789in}}{\pgfqpoint{4.106076in}{1.006789in}}%
\pgfpathcurveto{\pgfqpoint{4.095026in}{1.006789in}}{\pgfqpoint{4.084427in}{1.002399in}}{\pgfqpoint{4.076613in}{0.994585in}}%
\pgfpathcurveto{\pgfqpoint{4.068800in}{0.986771in}}{\pgfqpoint{4.064409in}{0.976172in}}{\pgfqpoint{4.064409in}{0.965122in}}%
\pgfpathcurveto{\pgfqpoint{4.064409in}{0.954072in}}{\pgfqpoint{4.068800in}{0.943473in}}{\pgfqpoint{4.076613in}{0.935659in}}%
\pgfpathcurveto{\pgfqpoint{4.084427in}{0.927846in}}{\pgfqpoint{4.095026in}{0.923456in}}{\pgfqpoint{4.106076in}{0.923456in}}%
\pgfpathclose%
\pgfusepath{stroke,fill}%
\end{pgfscope}%
\end{pgfpicture}%
\makeatother%
\endgroup%
}
    \caption{Frecuencia (en Hz, escala logarítmica) frente al módulo de la impedancia (En Ohmios, escala logarímica escalada por 20), con rectas \textcolor{Red}{R} y \textcolor{Golden}{C}}
  \end{figure}

  Como podemos observar, en ámbas gráficas (la que cuenta con 20 puntos de datos y la extendida) el punto de intersección de las dos rectas \textcolor{Red}{R} y \textcolor{Golden}{C} se encuentra en el punto $(3,1226388, 80)$, calculado con \code{python}. Si tomamos el inverso del logaritmo en base 10, obtenemos $f = 10^{3,1226388} = 1326,29 Hz$, la frecuencia de corte calculada teóricamente.

  \subsection{Representación gráfica de $\frac{V_{mR}}{V_{mC}}$ frente a f}
  \label{sec:vmf}

  También podemos crear otra gráfica, en esta ocasión representando $\frac{V_{mR}}{V_{mC}}$ frente a la frecuencia (\textit{f}). Sabemos que en aquella frecuencia para la que $\frac{V_{mR}}{V_{mC}} = 1$ será la frecuencia de corte.

  \begin{python}
    import matplotlib.pyplot as plt
    import pandas as pd
    import numpy as np

    d = pd.read_csv(name + ".csv", sep=';', decimal=',')
    x = d["f"]
    vmr = d["VmR"], vmc = d["VmC"]
    y = vmr / vmc

    plt.scatter(x,y, zorder=2)
  \end{python}

  \begin{figure}[H]
    %\centering
    \hspace{2.5em} %% Creator: Matplotlib, PGF backend
%%
%% To include the figure in your LaTeX document, write
%%   \input{<filename>.pgf}
%%
%% Make sure the required packages are loaded in your preamble
%%   \usepackage{pgf}
%%
%% Figures using additional raster images can only be included by \input if
%% they are in the same directory as the main LaTeX file. For loading figures
%% from other directories you can use the `import` package
%%   \usepackage{import}
%% and then include the figures with
%%   \import{<path to file>}{<filename>.pgf}
%%
%% Matplotlib used the following preamble
%%
\begingroup%
\makeatletter%
\begin{pgfpicture}%
\pgfpathrectangle{\pgfpointorigin}{\pgfqpoint{5.052818in}{3.310123in}}%
\pgfusepath{use as bounding box, clip}%
\begin{pgfscope}%
\pgfsetbuttcap%
\pgfsetmiterjoin%
\definecolor{currentfill}{rgb}{1.000000,1.000000,1.000000}%
\pgfsetfillcolor{currentfill}%
\pgfsetlinewidth{0.000000pt}%
\definecolor{currentstroke}{rgb}{1.000000,1.000000,1.000000}%
\pgfsetstrokecolor{currentstroke}%
\pgfsetdash{}{0pt}%
\pgfpathmoveto{\pgfqpoint{-0.000000in}{0.000000in}}%
\pgfpathlineto{\pgfqpoint{5.052818in}{0.000000in}}%
\pgfpathlineto{\pgfqpoint{5.052818in}{3.310123in}}%
\pgfpathlineto{\pgfqpoint{-0.000000in}{3.310123in}}%
\pgfpathclose%
\pgfusepath{fill}%
\end{pgfscope}%
\begin{pgfscope}%
\pgfsetbuttcap%
\pgfsetmiterjoin%
\definecolor{currentfill}{rgb}{1.000000,1.000000,1.000000}%
\pgfsetfillcolor{currentfill}%
\pgfsetlinewidth{0.000000pt}%
\definecolor{currentstroke}{rgb}{0.000000,0.000000,0.000000}%
\pgfsetstrokecolor{currentstroke}%
\pgfsetstrokeopacity{0.000000}%
\pgfsetdash{}{0pt}%
\pgfpathmoveto{\pgfqpoint{1.077818in}{0.515123in}}%
\pgfpathlineto{\pgfqpoint{4.952818in}{0.515123in}}%
\pgfpathlineto{\pgfqpoint{4.952818in}{3.210123in}}%
\pgfpathlineto{\pgfqpoint{1.077818in}{3.210123in}}%
\pgfpathclose%
\pgfusepath{fill}%
\end{pgfscope}%
\begin{pgfscope}%
\pgfpathrectangle{\pgfqpoint{1.077818in}{0.515123in}}{\pgfqpoint{3.875000in}{2.695000in}}%
\pgfusepath{clip}%
\pgfsetrectcap%
\pgfsetroundjoin%
\pgfsetlinewidth{1.505625pt}%
\definecolor{currentstroke}{rgb}{0.529412,0.807843,0.921569}%
\pgfsetstrokecolor{currentstroke}%
\pgfsetdash{}{0pt}%
\pgfpathmoveto{\pgfqpoint{1.253978in}{0.672506in}}%
\pgfpathlineto{\pgfqpoint{1.439382in}{0.798264in}}%
\pgfpathlineto{\pgfqpoint{1.624786in}{0.924022in}}%
\pgfpathlineto{\pgfqpoint{1.810190in}{1.049780in}}%
\pgfpathlineto{\pgfqpoint{1.995595in}{1.175539in}}%
\pgfpathlineto{\pgfqpoint{2.180999in}{1.301297in}}%
\pgfpathlineto{\pgfqpoint{2.366403in}{1.427055in}}%
\pgfpathlineto{\pgfqpoint{2.551807in}{1.552813in}}%
\pgfpathlineto{\pgfqpoint{2.737211in}{1.678571in}}%
\pgfpathlineto{\pgfqpoint{2.922616in}{1.804330in}}%
\pgfpathlineto{\pgfqpoint{3.108020in}{1.930088in}}%
\pgfpathlineto{\pgfqpoint{3.293424in}{2.055846in}}%
\pgfpathlineto{\pgfqpoint{3.478828in}{2.181604in}}%
\pgfpathlineto{\pgfqpoint{3.664232in}{2.307362in}}%
\pgfpathlineto{\pgfqpoint{3.849636in}{2.433121in}}%
\pgfpathlineto{\pgfqpoint{4.035041in}{2.558879in}}%
\pgfpathlineto{\pgfqpoint{4.220445in}{2.684637in}}%
\pgfpathlineto{\pgfqpoint{4.405849in}{2.810395in}}%
\pgfpathlineto{\pgfqpoint{4.591253in}{2.936153in}}%
\pgfpathlineto{\pgfqpoint{4.776657in}{3.061912in}}%
\pgfusepath{stroke}%
\end{pgfscope}%
\begin{pgfscope}%
\pgfsetbuttcap%
\pgfsetroundjoin%
\definecolor{currentfill}{rgb}{0.000000,0.000000,0.000000}%
\pgfsetfillcolor{currentfill}%
\pgfsetlinewidth{0.803000pt}%
\definecolor{currentstroke}{rgb}{0.000000,0.000000,0.000000}%
\pgfsetstrokecolor{currentstroke}%
\pgfsetdash{}{0pt}%
\pgfsys@defobject{currentmarker}{\pgfqpoint{0.000000in}{-0.048611in}}{\pgfqpoint{0.000000in}{0.000000in}}{%
\pgfpathmoveto{\pgfqpoint{0.000000in}{0.000000in}}%
\pgfpathlineto{\pgfqpoint{0.000000in}{-0.048611in}}%
\pgfusepath{stroke,fill}%
}%
\begin{pgfscope}%
\pgfsys@transformshift{1.377581in}{0.515123in}%
\pgfsys@useobject{currentmarker}{}%
\end{pgfscope}%
\end{pgfscope}%
\begin{pgfscope}%
\definecolor{textcolor}{rgb}{0.000000,0.000000,0.000000}%
\pgfsetstrokecolor{textcolor}%
\pgfsetfillcolor{textcolor}%
\pgftext[x=1.377581in,y=0.417901in,,top]{\color{textcolor}\rmfamily\fontsize{10.000000}{12.000000}\selectfont \(\displaystyle 800\)}%
\end{pgfscope}%
\begin{pgfscope}%
\pgfsetbuttcap%
\pgfsetroundjoin%
\definecolor{currentfill}{rgb}{0.000000,0.000000,0.000000}%
\pgfsetfillcolor{currentfill}%
\pgfsetlinewidth{0.803000pt}%
\definecolor{currentstroke}{rgb}{0.000000,0.000000,0.000000}%
\pgfsetstrokecolor{currentstroke}%
\pgfsetdash{}{0pt}%
\pgfsys@defobject{currentmarker}{\pgfqpoint{0.000000in}{-0.048611in}}{\pgfqpoint{0.000000in}{0.000000in}}{%
\pgfpathmoveto{\pgfqpoint{0.000000in}{0.000000in}}%
\pgfpathlineto{\pgfqpoint{0.000000in}{-0.048611in}}%
\pgfusepath{stroke,fill}%
}%
\begin{pgfscope}%
\pgfsys@transformshift{1.995595in}{0.515123in}%
\pgfsys@useobject{currentmarker}{}%
\end{pgfscope}%
\end{pgfscope}%
\begin{pgfscope}%
\definecolor{textcolor}{rgb}{0.000000,0.000000,0.000000}%
\pgfsetstrokecolor{textcolor}%
\pgfsetfillcolor{textcolor}%
\pgftext[x=1.995595in,y=0.417901in,,top]{\color{textcolor}\rmfamily\fontsize{10.000000}{12.000000}\selectfont \(\displaystyle 1000\)}%
\end{pgfscope}%
\begin{pgfscope}%
\pgfsetbuttcap%
\pgfsetroundjoin%
\definecolor{currentfill}{rgb}{0.000000,0.000000,0.000000}%
\pgfsetfillcolor{currentfill}%
\pgfsetlinewidth{0.803000pt}%
\definecolor{currentstroke}{rgb}{0.000000,0.000000,0.000000}%
\pgfsetstrokecolor{currentstroke}%
\pgfsetdash{}{0pt}%
\pgfsys@defobject{currentmarker}{\pgfqpoint{0.000000in}{-0.048611in}}{\pgfqpoint{0.000000in}{0.000000in}}{%
\pgfpathmoveto{\pgfqpoint{0.000000in}{0.000000in}}%
\pgfpathlineto{\pgfqpoint{0.000000in}{-0.048611in}}%
\pgfusepath{stroke,fill}%
}%
\begin{pgfscope}%
\pgfsys@transformshift{2.613609in}{0.515123in}%
\pgfsys@useobject{currentmarker}{}%
\end{pgfscope}%
\end{pgfscope}%
\begin{pgfscope}%
\definecolor{textcolor}{rgb}{0.000000,0.000000,0.000000}%
\pgfsetstrokecolor{textcolor}%
\pgfsetfillcolor{textcolor}%
\pgftext[x=2.613609in,y=0.417901in,,top]{\color{textcolor}\rmfamily\fontsize{10.000000}{12.000000}\selectfont \(\displaystyle 1200\)}%
\end{pgfscope}%
\begin{pgfscope}%
\pgfsetbuttcap%
\pgfsetroundjoin%
\definecolor{currentfill}{rgb}{0.000000,0.000000,0.000000}%
\pgfsetfillcolor{currentfill}%
\pgfsetlinewidth{0.803000pt}%
\definecolor{currentstroke}{rgb}{0.000000,0.000000,0.000000}%
\pgfsetstrokecolor{currentstroke}%
\pgfsetdash{}{0pt}%
\pgfsys@defobject{currentmarker}{\pgfqpoint{0.000000in}{-0.048611in}}{\pgfqpoint{0.000000in}{0.000000in}}{%
\pgfpathmoveto{\pgfqpoint{0.000000in}{0.000000in}}%
\pgfpathlineto{\pgfqpoint{0.000000in}{-0.048611in}}%
\pgfusepath{stroke,fill}%
}%
\begin{pgfscope}%
\pgfsys@transformshift{3.231623in}{0.515123in}%
\pgfsys@useobject{currentmarker}{}%
\end{pgfscope}%
\end{pgfscope}%
\begin{pgfscope}%
\definecolor{textcolor}{rgb}{0.000000,0.000000,0.000000}%
\pgfsetstrokecolor{textcolor}%
\pgfsetfillcolor{textcolor}%
\pgftext[x=3.231623in,y=0.417901in,,top]{\color{textcolor}\rmfamily\fontsize{10.000000}{12.000000}\selectfont \(\displaystyle 1400\)}%
\end{pgfscope}%
\begin{pgfscope}%
\pgfsetbuttcap%
\pgfsetroundjoin%
\definecolor{currentfill}{rgb}{0.000000,0.000000,0.000000}%
\pgfsetfillcolor{currentfill}%
\pgfsetlinewidth{0.803000pt}%
\definecolor{currentstroke}{rgb}{0.000000,0.000000,0.000000}%
\pgfsetstrokecolor{currentstroke}%
\pgfsetdash{}{0pt}%
\pgfsys@defobject{currentmarker}{\pgfqpoint{0.000000in}{-0.048611in}}{\pgfqpoint{0.000000in}{0.000000in}}{%
\pgfpathmoveto{\pgfqpoint{0.000000in}{0.000000in}}%
\pgfpathlineto{\pgfqpoint{0.000000in}{-0.048611in}}%
\pgfusepath{stroke,fill}%
}%
\begin{pgfscope}%
\pgfsys@transformshift{3.849636in}{0.515123in}%
\pgfsys@useobject{currentmarker}{}%
\end{pgfscope}%
\end{pgfscope}%
\begin{pgfscope}%
\definecolor{textcolor}{rgb}{0.000000,0.000000,0.000000}%
\pgfsetstrokecolor{textcolor}%
\pgfsetfillcolor{textcolor}%
\pgftext[x=3.849636in,y=0.417901in,,top]{\color{textcolor}\rmfamily\fontsize{10.000000}{12.000000}\selectfont \(\displaystyle 1600\)}%
\end{pgfscope}%
\begin{pgfscope}%
\pgfsetbuttcap%
\pgfsetroundjoin%
\definecolor{currentfill}{rgb}{0.000000,0.000000,0.000000}%
\pgfsetfillcolor{currentfill}%
\pgfsetlinewidth{0.803000pt}%
\definecolor{currentstroke}{rgb}{0.000000,0.000000,0.000000}%
\pgfsetstrokecolor{currentstroke}%
\pgfsetdash{}{0pt}%
\pgfsys@defobject{currentmarker}{\pgfqpoint{0.000000in}{-0.048611in}}{\pgfqpoint{0.000000in}{0.000000in}}{%
\pgfpathmoveto{\pgfqpoint{0.000000in}{0.000000in}}%
\pgfpathlineto{\pgfqpoint{0.000000in}{-0.048611in}}%
\pgfusepath{stroke,fill}%
}%
\begin{pgfscope}%
\pgfsys@transformshift{4.467650in}{0.515123in}%
\pgfsys@useobject{currentmarker}{}%
\end{pgfscope}%
\end{pgfscope}%
\begin{pgfscope}%
\definecolor{textcolor}{rgb}{0.000000,0.000000,0.000000}%
\pgfsetstrokecolor{textcolor}%
\pgfsetfillcolor{textcolor}%
\pgftext[x=4.467650in,y=0.417901in,,top]{\color{textcolor}\rmfamily\fontsize{10.000000}{12.000000}\selectfont \(\displaystyle 1800\)}%
\end{pgfscope}%
\begin{pgfscope}%
\definecolor{textcolor}{rgb}{0.000000,0.000000,0.000000}%
\pgfsetstrokecolor{textcolor}%
\pgfsetfillcolor{textcolor}%
\pgftext[x=3.015318in,y=0.238889in,,top]{\color{textcolor}\rmfamily\fontsize{10.000000}{12.000000}\selectfont f (Hz)}%
\end{pgfscope}%
\begin{pgfscope}%
\pgfsetbuttcap%
\pgfsetroundjoin%
\definecolor{currentfill}{rgb}{0.000000,0.000000,0.000000}%
\pgfsetfillcolor{currentfill}%
\pgfsetlinewidth{0.803000pt}%
\definecolor{currentstroke}{rgb}{0.000000,0.000000,0.000000}%
\pgfsetstrokecolor{currentstroke}%
\pgfsetdash{}{0pt}%
\pgfsys@defobject{currentmarker}{\pgfqpoint{-0.048611in}{0.000000in}}{\pgfqpoint{0.000000in}{0.000000in}}{%
\pgfpathmoveto{\pgfqpoint{0.000000in}{0.000000in}}%
\pgfpathlineto{\pgfqpoint{-0.048611in}{0.000000in}}%
\pgfusepath{stroke,fill}%
}%
\begin{pgfscope}%
\pgfsys@transformshift{1.077818in}{0.698184in}%
\pgfsys@useobject{currentmarker}{}%
\end{pgfscope}%
\end{pgfscope}%
\begin{pgfscope}%
\definecolor{textcolor}{rgb}{0.000000,0.000000,0.000000}%
\pgfsetstrokecolor{textcolor}%
\pgfsetfillcolor{textcolor}%
\pgftext[x=0.803126in,y=0.649958in,left,base]{\color{textcolor}\rmfamily\fontsize{10.000000}{12.000000}\selectfont \(\displaystyle 0.6\)}%
\end{pgfscope}%
\begin{pgfscope}%
\pgfsetbuttcap%
\pgfsetroundjoin%
\definecolor{currentfill}{rgb}{0.000000,0.000000,0.000000}%
\pgfsetfillcolor{currentfill}%
\pgfsetlinewidth{0.803000pt}%
\definecolor{currentstroke}{rgb}{0.000000,0.000000,0.000000}%
\pgfsetstrokecolor{currentstroke}%
\pgfsetdash{}{0pt}%
\pgfsys@defobject{currentmarker}{\pgfqpoint{-0.048611in}{0.000000in}}{\pgfqpoint{0.000000in}{0.000000in}}{%
\pgfpathmoveto{\pgfqpoint{0.000000in}{0.000000in}}%
\pgfpathlineto{\pgfqpoint{-0.048611in}{0.000000in}}%
\pgfusepath{stroke,fill}%
}%
\begin{pgfscope}%
\pgfsys@transformshift{1.077818in}{1.260585in}%
\pgfsys@useobject{currentmarker}{}%
\end{pgfscope}%
\end{pgfscope}%
\begin{pgfscope}%
\definecolor{textcolor}{rgb}{0.000000,0.000000,0.000000}%
\pgfsetstrokecolor{textcolor}%
\pgfsetfillcolor{textcolor}%
\pgftext[x=0.803126in,y=1.212359in,left,base]{\color{textcolor}\rmfamily\fontsize{10.000000}{12.000000}\selectfont \(\displaystyle 0.8\)}%
\end{pgfscope}%
\begin{pgfscope}%
\pgfsetbuttcap%
\pgfsetroundjoin%
\definecolor{currentfill}{rgb}{0.000000,0.000000,0.000000}%
\pgfsetfillcolor{currentfill}%
\pgfsetlinewidth{0.803000pt}%
\definecolor{currentstroke}{rgb}{0.000000,0.000000,0.000000}%
\pgfsetstrokecolor{currentstroke}%
\pgfsetdash{}{0pt}%
\pgfsys@defobject{currentmarker}{\pgfqpoint{-0.048611in}{0.000000in}}{\pgfqpoint{0.000000in}{0.000000in}}{%
\pgfpathmoveto{\pgfqpoint{0.000000in}{0.000000in}}%
\pgfpathlineto{\pgfqpoint{-0.048611in}{0.000000in}}%
\pgfusepath{stroke,fill}%
}%
\begin{pgfscope}%
\pgfsys@transformshift{1.077818in}{1.822986in}%
\pgfsys@useobject{currentmarker}{}%
\end{pgfscope}%
\end{pgfscope}%
\begin{pgfscope}%
\definecolor{textcolor}{rgb}{0.000000,0.000000,0.000000}%
\pgfsetstrokecolor{textcolor}%
\pgfsetfillcolor{textcolor}%
\pgftext[x=0.803126in,y=1.774760in,left,base]{\color{textcolor}\rmfamily\fontsize{10.000000}{12.000000}\selectfont \(\displaystyle 1.0\)}%
\end{pgfscope}%
\begin{pgfscope}%
\pgfsetbuttcap%
\pgfsetroundjoin%
\definecolor{currentfill}{rgb}{0.000000,0.000000,0.000000}%
\pgfsetfillcolor{currentfill}%
\pgfsetlinewidth{0.803000pt}%
\definecolor{currentstroke}{rgb}{0.000000,0.000000,0.000000}%
\pgfsetstrokecolor{currentstroke}%
\pgfsetdash{}{0pt}%
\pgfsys@defobject{currentmarker}{\pgfqpoint{-0.048611in}{0.000000in}}{\pgfqpoint{0.000000in}{0.000000in}}{%
\pgfpathmoveto{\pgfqpoint{0.000000in}{0.000000in}}%
\pgfpathlineto{\pgfqpoint{-0.048611in}{0.000000in}}%
\pgfusepath{stroke,fill}%
}%
\begin{pgfscope}%
\pgfsys@transformshift{1.077818in}{2.385386in}%
\pgfsys@useobject{currentmarker}{}%
\end{pgfscope}%
\end{pgfscope}%
\begin{pgfscope}%
\definecolor{textcolor}{rgb}{0.000000,0.000000,0.000000}%
\pgfsetstrokecolor{textcolor}%
\pgfsetfillcolor{textcolor}%
\pgftext[x=0.803126in,y=2.337161in,left,base]{\color{textcolor}\rmfamily\fontsize{10.000000}{12.000000}\selectfont \(\displaystyle 1.2\)}%
\end{pgfscope}%
\begin{pgfscope}%
\pgfsetbuttcap%
\pgfsetroundjoin%
\definecolor{currentfill}{rgb}{0.000000,0.000000,0.000000}%
\pgfsetfillcolor{currentfill}%
\pgfsetlinewidth{0.803000pt}%
\definecolor{currentstroke}{rgb}{0.000000,0.000000,0.000000}%
\pgfsetstrokecolor{currentstroke}%
\pgfsetdash{}{0pt}%
\pgfsys@defobject{currentmarker}{\pgfqpoint{-0.048611in}{0.000000in}}{\pgfqpoint{0.000000in}{0.000000in}}{%
\pgfpathmoveto{\pgfqpoint{0.000000in}{0.000000in}}%
\pgfpathlineto{\pgfqpoint{-0.048611in}{0.000000in}}%
\pgfusepath{stroke,fill}%
}%
\begin{pgfscope}%
\pgfsys@transformshift{1.077818in}{2.947787in}%
\pgfsys@useobject{currentmarker}{}%
\end{pgfscope}%
\end{pgfscope}%
\begin{pgfscope}%
\definecolor{textcolor}{rgb}{0.000000,0.000000,0.000000}%
\pgfsetstrokecolor{textcolor}%
\pgfsetfillcolor{textcolor}%
\pgftext[x=0.803126in,y=2.899562in,left,base]{\color{textcolor}\rmfamily\fontsize{10.000000}{12.000000}\selectfont \(\displaystyle 1.4\)}%
\end{pgfscope}%
\begin{pgfscope}%
\definecolor{textcolor}{rgb}{0.000000,0.000000,0.000000}%
\pgfsetstrokecolor{textcolor}%
\pgfsetfillcolor{textcolor}%
\pgftext[x=0.455903in,y=1.862623in,,bottom]{\color{textcolor}\rmfamily\fontsize{10.000000}{12.000000}\selectfont VmR/VmC}%
\end{pgfscope}%
\begin{pgfscope}%
\pgfpathrectangle{\pgfqpoint{1.077818in}{0.515123in}}{\pgfqpoint{3.875000in}{2.695000in}}%
\pgfusepath{clip}%
\pgfsetbuttcap%
\pgfsetroundjoin%
\definecolor{currentfill}{rgb}{0.121569,0.466667,0.705882}%
\pgfsetfillcolor{currentfill}%
\pgfsetlinewidth{1.003750pt}%
\definecolor{currentstroke}{rgb}{0.121569,0.466667,0.705882}%
\pgfsetstrokecolor{currentstroke}%
\pgfsetdash{}{0pt}%
\pgfpathmoveto{\pgfqpoint{1.253978in}{0.627259in}}%
\pgfpathcurveto{\pgfqpoint{1.265028in}{0.627259in}}{\pgfqpoint{1.275627in}{0.631649in}}{\pgfqpoint{1.283441in}{0.639463in}}%
\pgfpathcurveto{\pgfqpoint{1.291254in}{0.647277in}}{\pgfqpoint{1.295645in}{0.657876in}}{\pgfqpoint{1.295645in}{0.668926in}}%
\pgfpathcurveto{\pgfqpoint{1.295645in}{0.679976in}}{\pgfqpoint{1.291254in}{0.690575in}}{\pgfqpoint{1.283441in}{0.698389in}}%
\pgfpathcurveto{\pgfqpoint{1.275627in}{0.706202in}}{\pgfqpoint{1.265028in}{0.710592in}}{\pgfqpoint{1.253978in}{0.710592in}}%
\pgfpathcurveto{\pgfqpoint{1.242928in}{0.710592in}}{\pgfqpoint{1.232329in}{0.706202in}}{\pgfqpoint{1.224515in}{0.698389in}}%
\pgfpathcurveto{\pgfqpoint{1.216702in}{0.690575in}}{\pgfqpoint{1.212311in}{0.679976in}}{\pgfqpoint{1.212311in}{0.668926in}}%
\pgfpathcurveto{\pgfqpoint{1.212311in}{0.657876in}}{\pgfqpoint{1.216702in}{0.647277in}}{\pgfqpoint{1.224515in}{0.639463in}}%
\pgfpathcurveto{\pgfqpoint{1.232329in}{0.631649in}}{\pgfqpoint{1.242928in}{0.627259in}}{\pgfqpoint{1.253978in}{0.627259in}}%
\pgfpathclose%
\pgfusepath{stroke,fill}%
\end{pgfscope}%
\begin{pgfscope}%
\pgfpathrectangle{\pgfqpoint{1.077818in}{0.515123in}}{\pgfqpoint{3.875000in}{2.695000in}}%
\pgfusepath{clip}%
\pgfsetbuttcap%
\pgfsetroundjoin%
\definecolor{currentfill}{rgb}{0.121569,0.466667,0.705882}%
\pgfsetfillcolor{currentfill}%
\pgfsetlinewidth{1.003750pt}%
\definecolor{currentstroke}{rgb}{0.121569,0.466667,0.705882}%
\pgfsetstrokecolor{currentstroke}%
\pgfsetdash{}{0pt}%
\pgfpathmoveto{\pgfqpoint{1.439382in}{0.755764in}}%
\pgfpathcurveto{\pgfqpoint{1.450432in}{0.755764in}}{\pgfqpoint{1.461031in}{0.760154in}}{\pgfqpoint{1.468845in}{0.767968in}}%
\pgfpathcurveto{\pgfqpoint{1.476659in}{0.775782in}}{\pgfqpoint{1.481049in}{0.786381in}}{\pgfqpoint{1.481049in}{0.797431in}}%
\pgfpathcurveto{\pgfqpoint{1.481049in}{0.808481in}}{\pgfqpoint{1.476659in}{0.819080in}}{\pgfqpoint{1.468845in}{0.826894in}}%
\pgfpathcurveto{\pgfqpoint{1.461031in}{0.834707in}}{\pgfqpoint{1.450432in}{0.839098in}}{\pgfqpoint{1.439382in}{0.839098in}}%
\pgfpathcurveto{\pgfqpoint{1.428332in}{0.839098in}}{\pgfqpoint{1.417733in}{0.834707in}}{\pgfqpoint{1.409919in}{0.826894in}}%
\pgfpathcurveto{\pgfqpoint{1.402106in}{0.819080in}}{\pgfqpoint{1.397715in}{0.808481in}}{\pgfqpoint{1.397715in}{0.797431in}}%
\pgfpathcurveto{\pgfqpoint{1.397715in}{0.786381in}}{\pgfqpoint{1.402106in}{0.775782in}}{\pgfqpoint{1.409919in}{0.767968in}}%
\pgfpathcurveto{\pgfqpoint{1.417733in}{0.760154in}}{\pgfqpoint{1.428332in}{0.755764in}}{\pgfqpoint{1.439382in}{0.755764in}}%
\pgfpathclose%
\pgfusepath{stroke,fill}%
\end{pgfscope}%
\begin{pgfscope}%
\pgfpathrectangle{\pgfqpoint{1.077818in}{0.515123in}}{\pgfqpoint{3.875000in}{2.695000in}}%
\pgfusepath{clip}%
\pgfsetbuttcap%
\pgfsetroundjoin%
\definecolor{currentfill}{rgb}{0.121569,0.466667,0.705882}%
\pgfsetfillcolor{currentfill}%
\pgfsetlinewidth{1.003750pt}%
\definecolor{currentstroke}{rgb}{0.121569,0.466667,0.705882}%
\pgfsetstrokecolor{currentstroke}%
\pgfsetdash{}{0pt}%
\pgfpathmoveto{\pgfqpoint{1.624786in}{0.866570in}}%
\pgfpathcurveto{\pgfqpoint{1.635836in}{0.866570in}}{\pgfqpoint{1.646435in}{0.870961in}}{\pgfqpoint{1.654249in}{0.878774in}}%
\pgfpathcurveto{\pgfqpoint{1.662063in}{0.886588in}}{\pgfqpoint{1.666453in}{0.897187in}}{\pgfqpoint{1.666453in}{0.908237in}}%
\pgfpathcurveto{\pgfqpoint{1.666453in}{0.919287in}}{\pgfqpoint{1.662063in}{0.929886in}}{\pgfqpoint{1.654249in}{0.937700in}}%
\pgfpathcurveto{\pgfqpoint{1.646435in}{0.945513in}}{\pgfqpoint{1.635836in}{0.949904in}}{\pgfqpoint{1.624786in}{0.949904in}}%
\pgfpathcurveto{\pgfqpoint{1.613736in}{0.949904in}}{\pgfqpoint{1.603137in}{0.945513in}}{\pgfqpoint{1.595324in}{0.937700in}}%
\pgfpathcurveto{\pgfqpoint{1.587510in}{0.929886in}}{\pgfqpoint{1.583120in}{0.919287in}}{\pgfqpoint{1.583120in}{0.908237in}}%
\pgfpathcurveto{\pgfqpoint{1.583120in}{0.897187in}}{\pgfqpoint{1.587510in}{0.886588in}}{\pgfqpoint{1.595324in}{0.878774in}}%
\pgfpathcurveto{\pgfqpoint{1.603137in}{0.870961in}}{\pgfqpoint{1.613736in}{0.866570in}}{\pgfqpoint{1.624786in}{0.866570in}}%
\pgfpathclose%
\pgfusepath{stroke,fill}%
\end{pgfscope}%
\begin{pgfscope}%
\pgfpathrectangle{\pgfqpoint{1.077818in}{0.515123in}}{\pgfqpoint{3.875000in}{2.695000in}}%
\pgfusepath{clip}%
\pgfsetbuttcap%
\pgfsetroundjoin%
\definecolor{currentfill}{rgb}{0.121569,0.466667,0.705882}%
\pgfsetfillcolor{currentfill}%
\pgfsetlinewidth{1.003750pt}%
\definecolor{currentstroke}{rgb}{0.121569,0.466667,0.705882}%
\pgfsetstrokecolor{currentstroke}%
\pgfsetdash{}{0pt}%
\pgfpathmoveto{\pgfqpoint{1.810190in}{0.992586in}}%
\pgfpathcurveto{\pgfqpoint{1.821241in}{0.992586in}}{\pgfqpoint{1.831840in}{0.996976in}}{\pgfqpoint{1.839653in}{1.004790in}}%
\pgfpathcurveto{\pgfqpoint{1.847467in}{1.012603in}}{\pgfqpoint{1.851857in}{1.023202in}}{\pgfqpoint{1.851857in}{1.034252in}}%
\pgfpathcurveto{\pgfqpoint{1.851857in}{1.045303in}}{\pgfqpoint{1.847467in}{1.055902in}}{\pgfqpoint{1.839653in}{1.063715in}}%
\pgfpathcurveto{\pgfqpoint{1.831840in}{1.071529in}}{\pgfqpoint{1.821241in}{1.075919in}}{\pgfqpoint{1.810190in}{1.075919in}}%
\pgfpathcurveto{\pgfqpoint{1.799140in}{1.075919in}}{\pgfqpoint{1.788541in}{1.071529in}}{\pgfqpoint{1.780728in}{1.063715in}}%
\pgfpathcurveto{\pgfqpoint{1.772914in}{1.055902in}}{\pgfqpoint{1.768524in}{1.045303in}}{\pgfqpoint{1.768524in}{1.034252in}}%
\pgfpathcurveto{\pgfqpoint{1.768524in}{1.023202in}}{\pgfqpoint{1.772914in}{1.012603in}}{\pgfqpoint{1.780728in}{1.004790in}}%
\pgfpathcurveto{\pgfqpoint{1.788541in}{0.996976in}}{\pgfqpoint{1.799140in}{0.992586in}}{\pgfqpoint{1.810190in}{0.992586in}}%
\pgfpathclose%
\pgfusepath{stroke,fill}%
\end{pgfscope}%
\begin{pgfscope}%
\pgfpathrectangle{\pgfqpoint{1.077818in}{0.515123in}}{\pgfqpoint{3.875000in}{2.695000in}}%
\pgfusepath{clip}%
\pgfsetbuttcap%
\pgfsetroundjoin%
\definecolor{currentfill}{rgb}{0.121569,0.466667,0.705882}%
\pgfsetfillcolor{currentfill}%
\pgfsetlinewidth{1.003750pt}%
\definecolor{currentstroke}{rgb}{0.121569,0.466667,0.705882}%
\pgfsetstrokecolor{currentstroke}%
\pgfsetdash{}{0pt}%
\pgfpathmoveto{\pgfqpoint{1.995595in}{1.113468in}}%
\pgfpathcurveto{\pgfqpoint{2.006645in}{1.113468in}}{\pgfqpoint{2.017244in}{1.117858in}}{\pgfqpoint{2.025057in}{1.125672in}}%
\pgfpathcurveto{\pgfqpoint{2.032871in}{1.133485in}}{\pgfqpoint{2.037261in}{1.144084in}}{\pgfqpoint{2.037261in}{1.155134in}}%
\pgfpathcurveto{\pgfqpoint{2.037261in}{1.166185in}}{\pgfqpoint{2.032871in}{1.176784in}}{\pgfqpoint{2.025057in}{1.184597in}}%
\pgfpathcurveto{\pgfqpoint{2.017244in}{1.192411in}}{\pgfqpoint{2.006645in}{1.196801in}}{\pgfqpoint{1.995595in}{1.196801in}}%
\pgfpathcurveto{\pgfqpoint{1.984545in}{1.196801in}}{\pgfqpoint{1.973945in}{1.192411in}}{\pgfqpoint{1.966132in}{1.184597in}}%
\pgfpathcurveto{\pgfqpoint{1.958318in}{1.176784in}}{\pgfqpoint{1.953928in}{1.166185in}}{\pgfqpoint{1.953928in}{1.155134in}}%
\pgfpathcurveto{\pgfqpoint{1.953928in}{1.144084in}}{\pgfqpoint{1.958318in}{1.133485in}}{\pgfqpoint{1.966132in}{1.125672in}}%
\pgfpathcurveto{\pgfqpoint{1.973945in}{1.117858in}}{\pgfqpoint{1.984545in}{1.113468in}}{\pgfqpoint{1.995595in}{1.113468in}}%
\pgfpathclose%
\pgfusepath{stroke,fill}%
\end{pgfscope}%
\begin{pgfscope}%
\pgfpathrectangle{\pgfqpoint{1.077818in}{0.515123in}}{\pgfqpoint{3.875000in}{2.695000in}}%
\pgfusepath{clip}%
\pgfsetbuttcap%
\pgfsetroundjoin%
\definecolor{currentfill}{rgb}{0.121569,0.466667,0.705882}%
\pgfsetfillcolor{currentfill}%
\pgfsetlinewidth{1.003750pt}%
\definecolor{currentstroke}{rgb}{0.121569,0.466667,0.705882}%
\pgfsetstrokecolor{currentstroke}%
\pgfsetdash{}{0pt}%
\pgfpathmoveto{\pgfqpoint{2.180999in}{1.276600in}}%
\pgfpathcurveto{\pgfqpoint{2.192049in}{1.276600in}}{\pgfqpoint{2.202648in}{1.280990in}}{\pgfqpoint{2.210462in}{1.288804in}}%
\pgfpathcurveto{\pgfqpoint{2.218275in}{1.296618in}}{\pgfqpoint{2.222665in}{1.307217in}}{\pgfqpoint{2.222665in}{1.318267in}}%
\pgfpathcurveto{\pgfqpoint{2.222665in}{1.329317in}}{\pgfqpoint{2.218275in}{1.339916in}}{\pgfqpoint{2.210462in}{1.347730in}}%
\pgfpathcurveto{\pgfqpoint{2.202648in}{1.355543in}}{\pgfqpoint{2.192049in}{1.359933in}}{\pgfqpoint{2.180999in}{1.359933in}}%
\pgfpathcurveto{\pgfqpoint{2.169949in}{1.359933in}}{\pgfqpoint{2.159350in}{1.355543in}}{\pgfqpoint{2.151536in}{1.347730in}}%
\pgfpathcurveto{\pgfqpoint{2.143722in}{1.339916in}}{\pgfqpoint{2.139332in}{1.329317in}}{\pgfqpoint{2.139332in}{1.318267in}}%
\pgfpathcurveto{\pgfqpoint{2.139332in}{1.307217in}}{\pgfqpoint{2.143722in}{1.296618in}}{\pgfqpoint{2.151536in}{1.288804in}}%
\pgfpathcurveto{\pgfqpoint{2.159350in}{1.280990in}}{\pgfqpoint{2.169949in}{1.276600in}}{\pgfqpoint{2.180999in}{1.276600in}}%
\pgfpathclose%
\pgfusepath{stroke,fill}%
\end{pgfscope}%
\begin{pgfscope}%
\pgfpathrectangle{\pgfqpoint{1.077818in}{0.515123in}}{\pgfqpoint{3.875000in}{2.695000in}}%
\pgfusepath{clip}%
\pgfsetbuttcap%
\pgfsetroundjoin%
\definecolor{currentfill}{rgb}{0.121569,0.466667,0.705882}%
\pgfsetfillcolor{currentfill}%
\pgfsetlinewidth{1.003750pt}%
\definecolor{currentstroke}{rgb}{0.121569,0.466667,0.705882}%
\pgfsetstrokecolor{currentstroke}%
\pgfsetdash{}{0pt}%
\pgfpathmoveto{\pgfqpoint{2.366403in}{1.411318in}}%
\pgfpathcurveto{\pgfqpoint{2.377453in}{1.411318in}}{\pgfqpoint{2.388052in}{1.415708in}}{\pgfqpoint{2.395866in}{1.423522in}}%
\pgfpathcurveto{\pgfqpoint{2.403679in}{1.431336in}}{\pgfqpoint{2.408070in}{1.441935in}}{\pgfqpoint{2.408070in}{1.452985in}}%
\pgfpathcurveto{\pgfqpoint{2.408070in}{1.464035in}}{\pgfqpoint{2.403679in}{1.474634in}}{\pgfqpoint{2.395866in}{1.482448in}}%
\pgfpathcurveto{\pgfqpoint{2.388052in}{1.490261in}}{\pgfqpoint{2.377453in}{1.494652in}}{\pgfqpoint{2.366403in}{1.494652in}}%
\pgfpathcurveto{\pgfqpoint{2.355353in}{1.494652in}}{\pgfqpoint{2.344754in}{1.490261in}}{\pgfqpoint{2.336940in}{1.482448in}}%
\pgfpathcurveto{\pgfqpoint{2.329127in}{1.474634in}}{\pgfqpoint{2.324736in}{1.464035in}}{\pgfqpoint{2.324736in}{1.452985in}}%
\pgfpathcurveto{\pgfqpoint{2.324736in}{1.441935in}}{\pgfqpoint{2.329127in}{1.431336in}}{\pgfqpoint{2.336940in}{1.423522in}}%
\pgfpathcurveto{\pgfqpoint{2.344754in}{1.415708in}}{\pgfqpoint{2.355353in}{1.411318in}}{\pgfqpoint{2.366403in}{1.411318in}}%
\pgfpathclose%
\pgfusepath{stroke,fill}%
\end{pgfscope}%
\begin{pgfscope}%
\pgfpathrectangle{\pgfqpoint{1.077818in}{0.515123in}}{\pgfqpoint{3.875000in}{2.695000in}}%
\pgfusepath{clip}%
\pgfsetbuttcap%
\pgfsetroundjoin%
\definecolor{currentfill}{rgb}{0.121569,0.466667,0.705882}%
\pgfsetfillcolor{currentfill}%
\pgfsetlinewidth{1.003750pt}%
\definecolor{currentstroke}{rgb}{0.121569,0.466667,0.705882}%
\pgfsetstrokecolor{currentstroke}%
\pgfsetdash{}{0pt}%
\pgfpathmoveto{\pgfqpoint{2.551807in}{1.534318in}}%
\pgfpathcurveto{\pgfqpoint{2.562857in}{1.534318in}}{\pgfqpoint{2.573456in}{1.538709in}}{\pgfqpoint{2.581270in}{1.546522in}}%
\pgfpathcurveto{\pgfqpoint{2.589084in}{1.554336in}}{\pgfqpoint{2.593474in}{1.564935in}}{\pgfqpoint{2.593474in}{1.575985in}}%
\pgfpathcurveto{\pgfqpoint{2.593474in}{1.587035in}}{\pgfqpoint{2.589084in}{1.597634in}}{\pgfqpoint{2.581270in}{1.605448in}}%
\pgfpathcurveto{\pgfqpoint{2.573456in}{1.613261in}}{\pgfqpoint{2.562857in}{1.617652in}}{\pgfqpoint{2.551807in}{1.617652in}}%
\pgfpathcurveto{\pgfqpoint{2.540757in}{1.617652in}}{\pgfqpoint{2.530158in}{1.613261in}}{\pgfqpoint{2.522344in}{1.605448in}}%
\pgfpathcurveto{\pgfqpoint{2.514531in}{1.597634in}}{\pgfqpoint{2.510141in}{1.587035in}}{\pgfqpoint{2.510141in}{1.575985in}}%
\pgfpathcurveto{\pgfqpoint{2.510141in}{1.564935in}}{\pgfqpoint{2.514531in}{1.554336in}}{\pgfqpoint{2.522344in}{1.546522in}}%
\pgfpathcurveto{\pgfqpoint{2.530158in}{1.538709in}}{\pgfqpoint{2.540757in}{1.534318in}}{\pgfqpoint{2.551807in}{1.534318in}}%
\pgfpathclose%
\pgfusepath{stroke,fill}%
\end{pgfscope}%
\begin{pgfscope}%
\pgfpathrectangle{\pgfqpoint{1.077818in}{0.515123in}}{\pgfqpoint{3.875000in}{2.695000in}}%
\pgfusepath{clip}%
\pgfsetbuttcap%
\pgfsetroundjoin%
\definecolor{currentfill}{rgb}{0.121569,0.466667,0.705882}%
\pgfsetfillcolor{currentfill}%
\pgfsetlinewidth{1.003750pt}%
\definecolor{currentstroke}{rgb}{0.121569,0.466667,0.705882}%
\pgfsetstrokecolor{currentstroke}%
\pgfsetdash{}{0pt}%
\pgfpathmoveto{\pgfqpoint{2.737211in}{1.683680in}}%
\pgfpathcurveto{\pgfqpoint{2.748261in}{1.683680in}}{\pgfqpoint{2.758861in}{1.688070in}}{\pgfqpoint{2.766674in}{1.695884in}}%
\pgfpathcurveto{\pgfqpoint{2.774488in}{1.703697in}}{\pgfqpoint{2.778878in}{1.714296in}}{\pgfqpoint{2.778878in}{1.725346in}}%
\pgfpathcurveto{\pgfqpoint{2.778878in}{1.736397in}}{\pgfqpoint{2.774488in}{1.746996in}}{\pgfqpoint{2.766674in}{1.754809in}}%
\pgfpathcurveto{\pgfqpoint{2.758861in}{1.762623in}}{\pgfqpoint{2.748261in}{1.767013in}}{\pgfqpoint{2.737211in}{1.767013in}}%
\pgfpathcurveto{\pgfqpoint{2.726161in}{1.767013in}}{\pgfqpoint{2.715562in}{1.762623in}}{\pgfqpoint{2.707749in}{1.754809in}}%
\pgfpathcurveto{\pgfqpoint{2.699935in}{1.746996in}}{\pgfqpoint{2.695545in}{1.736397in}}{\pgfqpoint{2.695545in}{1.725346in}}%
\pgfpathcurveto{\pgfqpoint{2.695545in}{1.714296in}}{\pgfqpoint{2.699935in}{1.703697in}}{\pgfqpoint{2.707749in}{1.695884in}}%
\pgfpathcurveto{\pgfqpoint{2.715562in}{1.688070in}}{\pgfqpoint{2.726161in}{1.683680in}}{\pgfqpoint{2.737211in}{1.683680in}}%
\pgfpathclose%
\pgfusepath{stroke,fill}%
\end{pgfscope}%
\begin{pgfscope}%
\pgfpathrectangle{\pgfqpoint{1.077818in}{0.515123in}}{\pgfqpoint{3.875000in}{2.695000in}}%
\pgfusepath{clip}%
\pgfsetbuttcap%
\pgfsetroundjoin%
\definecolor{currentfill}{rgb}{0.121569,0.466667,0.705882}%
\pgfsetfillcolor{currentfill}%
\pgfsetlinewidth{1.003750pt}%
\definecolor{currentstroke}{rgb}{0.121569,0.466667,0.705882}%
\pgfsetstrokecolor{currentstroke}%
\pgfsetdash{}{0pt}%
\pgfpathmoveto{\pgfqpoint{2.922616in}{1.781319in}}%
\pgfpathcurveto{\pgfqpoint{2.933666in}{1.781319in}}{\pgfqpoint{2.944265in}{1.785709in}}{\pgfqpoint{2.952078in}{1.793523in}}%
\pgfpathcurveto{\pgfqpoint{2.959892in}{1.801336in}}{\pgfqpoint{2.964282in}{1.811935in}}{\pgfqpoint{2.964282in}{1.822986in}}%
\pgfpathcurveto{\pgfqpoint{2.964282in}{1.834036in}}{\pgfqpoint{2.959892in}{1.844635in}}{\pgfqpoint{2.952078in}{1.852448in}}%
\pgfpathcurveto{\pgfqpoint{2.944265in}{1.860262in}}{\pgfqpoint{2.933666in}{1.864652in}}{\pgfqpoint{2.922616in}{1.864652in}}%
\pgfpathcurveto{\pgfqpoint{2.911565in}{1.864652in}}{\pgfqpoint{2.900966in}{1.860262in}}{\pgfqpoint{2.893153in}{1.852448in}}%
\pgfpathcurveto{\pgfqpoint{2.885339in}{1.844635in}}{\pgfqpoint{2.880949in}{1.834036in}}{\pgfqpoint{2.880949in}{1.822986in}}%
\pgfpathcurveto{\pgfqpoint{2.880949in}{1.811935in}}{\pgfqpoint{2.885339in}{1.801336in}}{\pgfqpoint{2.893153in}{1.793523in}}%
\pgfpathcurveto{\pgfqpoint{2.900966in}{1.785709in}}{\pgfqpoint{2.911565in}{1.781319in}}{\pgfqpoint{2.922616in}{1.781319in}}%
\pgfpathclose%
\pgfusepath{stroke,fill}%
\end{pgfscope}%
\begin{pgfscope}%
\pgfpathrectangle{\pgfqpoint{1.077818in}{0.515123in}}{\pgfqpoint{3.875000in}{2.695000in}}%
\pgfusepath{clip}%
\pgfsetbuttcap%
\pgfsetroundjoin%
\definecolor{currentfill}{rgb}{0.121569,0.466667,0.705882}%
\pgfsetfillcolor{currentfill}%
\pgfsetlinewidth{1.003750pt}%
\definecolor{currentstroke}{rgb}{0.121569,0.466667,0.705882}%
\pgfsetstrokecolor{currentstroke}%
\pgfsetdash{}{0pt}%
\pgfpathmoveto{\pgfqpoint{3.108020in}{1.861662in}}%
\pgfpathcurveto{\pgfqpoint{3.119070in}{1.861662in}}{\pgfqpoint{3.129669in}{1.866052in}}{\pgfqpoint{3.137483in}{1.873866in}}%
\pgfpathcurveto{\pgfqpoint{3.145296in}{1.881679in}}{\pgfqpoint{3.149686in}{1.892278in}}{\pgfqpoint{3.149686in}{1.903328in}}%
\pgfpathcurveto{\pgfqpoint{3.149686in}{1.914379in}}{\pgfqpoint{3.145296in}{1.924978in}}{\pgfqpoint{3.137483in}{1.932791in}}%
\pgfpathcurveto{\pgfqpoint{3.129669in}{1.940605in}}{\pgfqpoint{3.119070in}{1.944995in}}{\pgfqpoint{3.108020in}{1.944995in}}%
\pgfpathcurveto{\pgfqpoint{3.096970in}{1.944995in}}{\pgfqpoint{3.086371in}{1.940605in}}{\pgfqpoint{3.078557in}{1.932791in}}%
\pgfpathcurveto{\pgfqpoint{3.070743in}{1.924978in}}{\pgfqpoint{3.066353in}{1.914379in}}{\pgfqpoint{3.066353in}{1.903328in}}%
\pgfpathcurveto{\pgfqpoint{3.066353in}{1.892278in}}{\pgfqpoint{3.070743in}{1.881679in}}{\pgfqpoint{3.078557in}{1.873866in}}%
\pgfpathcurveto{\pgfqpoint{3.086371in}{1.866052in}}{\pgfqpoint{3.096970in}{1.861662in}}{\pgfqpoint{3.108020in}{1.861662in}}%
\pgfpathclose%
\pgfusepath{stroke,fill}%
\end{pgfscope}%
\begin{pgfscope}%
\pgfpathrectangle{\pgfqpoint{1.077818in}{0.515123in}}{\pgfqpoint{3.875000in}{2.695000in}}%
\pgfusepath{clip}%
\pgfsetbuttcap%
\pgfsetroundjoin%
\definecolor{currentfill}{rgb}{0.121569,0.466667,0.705882}%
\pgfsetfillcolor{currentfill}%
\pgfsetlinewidth{1.003750pt}%
\definecolor{currentstroke}{rgb}{0.121569,0.466667,0.705882}%
\pgfsetstrokecolor{currentstroke}%
\pgfsetdash{}{0pt}%
\pgfpathmoveto{\pgfqpoint{3.293424in}{1.988084in}}%
\pgfpathcurveto{\pgfqpoint{3.304474in}{1.988084in}}{\pgfqpoint{3.315073in}{1.992474in}}{\pgfqpoint{3.322887in}{2.000288in}}%
\pgfpathcurveto{\pgfqpoint{3.330700in}{2.008101in}}{\pgfqpoint{3.335091in}{2.018700in}}{\pgfqpoint{3.335091in}{2.029751in}}%
\pgfpathcurveto{\pgfqpoint{3.335091in}{2.040801in}}{\pgfqpoint{3.330700in}{2.051400in}}{\pgfqpoint{3.322887in}{2.059213in}}%
\pgfpathcurveto{\pgfqpoint{3.315073in}{2.067027in}}{\pgfqpoint{3.304474in}{2.071417in}}{\pgfqpoint{3.293424in}{2.071417in}}%
\pgfpathcurveto{\pgfqpoint{3.282374in}{2.071417in}}{\pgfqpoint{3.271775in}{2.067027in}}{\pgfqpoint{3.263961in}{2.059213in}}%
\pgfpathcurveto{\pgfqpoint{3.256148in}{2.051400in}}{\pgfqpoint{3.251757in}{2.040801in}}{\pgfqpoint{3.251757in}{2.029751in}}%
\pgfpathcurveto{\pgfqpoint{3.251757in}{2.018700in}}{\pgfqpoint{3.256148in}{2.008101in}}{\pgfqpoint{3.263961in}{2.000288in}}%
\pgfpathcurveto{\pgfqpoint{3.271775in}{1.992474in}}{\pgfqpoint{3.282374in}{1.988084in}}{\pgfqpoint{3.293424in}{1.988084in}}%
\pgfpathclose%
\pgfusepath{stroke,fill}%
\end{pgfscope}%
\begin{pgfscope}%
\pgfpathrectangle{\pgfqpoint{1.077818in}{0.515123in}}{\pgfqpoint{3.875000in}{2.695000in}}%
\pgfusepath{clip}%
\pgfsetbuttcap%
\pgfsetroundjoin%
\definecolor{currentfill}{rgb}{0.121569,0.466667,0.705882}%
\pgfsetfillcolor{currentfill}%
\pgfsetlinewidth{1.003750pt}%
\definecolor{currentstroke}{rgb}{0.121569,0.466667,0.705882}%
\pgfsetstrokecolor{currentstroke}%
\pgfsetdash{}{0pt}%
\pgfpathmoveto{\pgfqpoint{3.478828in}{2.122168in}}%
\pgfpathcurveto{\pgfqpoint{3.489878in}{2.122168in}}{\pgfqpoint{3.500477in}{2.126558in}}{\pgfqpoint{3.508291in}{2.134372in}}%
\pgfpathcurveto{\pgfqpoint{3.516104in}{2.142185in}}{\pgfqpoint{3.520495in}{2.152784in}}{\pgfqpoint{3.520495in}{2.163835in}}%
\pgfpathcurveto{\pgfqpoint{3.520495in}{2.174885in}}{\pgfqpoint{3.516104in}{2.185484in}}{\pgfqpoint{3.508291in}{2.193297in}}%
\pgfpathcurveto{\pgfqpoint{3.500477in}{2.201111in}}{\pgfqpoint{3.489878in}{2.205501in}}{\pgfqpoint{3.478828in}{2.205501in}}%
\pgfpathcurveto{\pgfqpoint{3.467778in}{2.205501in}}{\pgfqpoint{3.457179in}{2.201111in}}{\pgfqpoint{3.449365in}{2.193297in}}%
\pgfpathcurveto{\pgfqpoint{3.441552in}{2.185484in}}{\pgfqpoint{3.437161in}{2.174885in}}{\pgfqpoint{3.437161in}{2.163835in}}%
\pgfpathcurveto{\pgfqpoint{3.437161in}{2.152784in}}{\pgfqpoint{3.441552in}{2.142185in}}{\pgfqpoint{3.449365in}{2.134372in}}%
\pgfpathcurveto{\pgfqpoint{3.457179in}{2.126558in}}{\pgfqpoint{3.467778in}{2.122168in}}{\pgfqpoint{3.478828in}{2.122168in}}%
\pgfpathclose%
\pgfusepath{stroke,fill}%
\end{pgfscope}%
\begin{pgfscope}%
\pgfpathrectangle{\pgfqpoint{1.077818in}{0.515123in}}{\pgfqpoint{3.875000in}{2.695000in}}%
\pgfusepath{clip}%
\pgfsetbuttcap%
\pgfsetroundjoin%
\definecolor{currentfill}{rgb}{0.121569,0.466667,0.705882}%
\pgfsetfillcolor{currentfill}%
\pgfsetlinewidth{1.003750pt}%
\definecolor{currentstroke}{rgb}{0.121569,0.466667,0.705882}%
\pgfsetstrokecolor{currentstroke}%
\pgfsetdash{}{0pt}%
\pgfpathmoveto{\pgfqpoint{3.664232in}{2.257197in}}%
\pgfpathcurveto{\pgfqpoint{3.675282in}{2.257197in}}{\pgfqpoint{3.685881in}{2.261587in}}{\pgfqpoint{3.693695in}{2.269400in}}%
\pgfpathcurveto{\pgfqpoint{3.701509in}{2.277214in}}{\pgfqpoint{3.705899in}{2.287813in}}{\pgfqpoint{3.705899in}{2.298863in}}%
\pgfpathcurveto{\pgfqpoint{3.705899in}{2.309913in}}{\pgfqpoint{3.701509in}{2.320512in}}{\pgfqpoint{3.693695in}{2.328326in}}%
\pgfpathcurveto{\pgfqpoint{3.685881in}{2.336140in}}{\pgfqpoint{3.675282in}{2.340530in}}{\pgfqpoint{3.664232in}{2.340530in}}%
\pgfpathcurveto{\pgfqpoint{3.653182in}{2.340530in}}{\pgfqpoint{3.642583in}{2.336140in}}{\pgfqpoint{3.634769in}{2.328326in}}%
\pgfpathcurveto{\pgfqpoint{3.626956in}{2.320512in}}{\pgfqpoint{3.622566in}{2.309913in}}{\pgfqpoint{3.622566in}{2.298863in}}%
\pgfpathcurveto{\pgfqpoint{3.622566in}{2.287813in}}{\pgfqpoint{3.626956in}{2.277214in}}{\pgfqpoint{3.634769in}{2.269400in}}%
\pgfpathcurveto{\pgfqpoint{3.642583in}{2.261587in}}{\pgfqpoint{3.653182in}{2.257197in}}{\pgfqpoint{3.664232in}{2.257197in}}%
\pgfpathclose%
\pgfusepath{stroke,fill}%
\end{pgfscope}%
\begin{pgfscope}%
\pgfpathrectangle{\pgfqpoint{1.077818in}{0.515123in}}{\pgfqpoint{3.875000in}{2.695000in}}%
\pgfusepath{clip}%
\pgfsetbuttcap%
\pgfsetroundjoin%
\definecolor{currentfill}{rgb}{0.121569,0.466667,0.705882}%
\pgfsetfillcolor{currentfill}%
\pgfsetlinewidth{1.003750pt}%
\definecolor{currentstroke}{rgb}{0.121569,0.466667,0.705882}%
\pgfsetstrokecolor{currentstroke}%
\pgfsetdash{}{0pt}%
\pgfpathmoveto{\pgfqpoint{3.849636in}{2.396445in}}%
\pgfpathcurveto{\pgfqpoint{3.860687in}{2.396445in}}{\pgfqpoint{3.871286in}{2.400835in}}{\pgfqpoint{3.879099in}{2.408649in}}%
\pgfpathcurveto{\pgfqpoint{3.886913in}{2.416462in}}{\pgfqpoint{3.891303in}{2.427061in}}{\pgfqpoint{3.891303in}{2.438112in}}%
\pgfpathcurveto{\pgfqpoint{3.891303in}{2.449162in}}{\pgfqpoint{3.886913in}{2.459761in}}{\pgfqpoint{3.879099in}{2.467574in}}%
\pgfpathcurveto{\pgfqpoint{3.871286in}{2.475388in}}{\pgfqpoint{3.860687in}{2.479778in}}{\pgfqpoint{3.849636in}{2.479778in}}%
\pgfpathcurveto{\pgfqpoint{3.838586in}{2.479778in}}{\pgfqpoint{3.827987in}{2.475388in}}{\pgfqpoint{3.820174in}{2.467574in}}%
\pgfpathcurveto{\pgfqpoint{3.812360in}{2.459761in}}{\pgfqpoint{3.807970in}{2.449162in}}{\pgfqpoint{3.807970in}{2.438112in}}%
\pgfpathcurveto{\pgfqpoint{3.807970in}{2.427061in}}{\pgfqpoint{3.812360in}{2.416462in}}{\pgfqpoint{3.820174in}{2.408649in}}%
\pgfpathcurveto{\pgfqpoint{3.827987in}{2.400835in}}{\pgfqpoint{3.838586in}{2.396445in}}{\pgfqpoint{3.849636in}{2.396445in}}%
\pgfpathclose%
\pgfusepath{stroke,fill}%
\end{pgfscope}%
\begin{pgfscope}%
\pgfpathrectangle{\pgfqpoint{1.077818in}{0.515123in}}{\pgfqpoint{3.875000in}{2.695000in}}%
\pgfusepath{clip}%
\pgfsetbuttcap%
\pgfsetroundjoin%
\definecolor{currentfill}{rgb}{0.121569,0.466667,0.705882}%
\pgfsetfillcolor{currentfill}%
\pgfsetlinewidth{1.003750pt}%
\definecolor{currentstroke}{rgb}{0.121569,0.466667,0.705882}%
\pgfsetstrokecolor{currentstroke}%
\pgfsetdash{}{0pt}%
\pgfpathmoveto{\pgfqpoint{4.035041in}{2.506997in}}%
\pgfpathcurveto{\pgfqpoint{4.046091in}{2.506997in}}{\pgfqpoint{4.056690in}{2.511388in}}{\pgfqpoint{4.064503in}{2.519201in}}%
\pgfpathcurveto{\pgfqpoint{4.072317in}{2.527015in}}{\pgfqpoint{4.076707in}{2.537614in}}{\pgfqpoint{4.076707in}{2.548664in}}%
\pgfpathcurveto{\pgfqpoint{4.076707in}{2.559714in}}{\pgfqpoint{4.072317in}{2.570313in}}{\pgfqpoint{4.064503in}{2.578127in}}%
\pgfpathcurveto{\pgfqpoint{4.056690in}{2.585941in}}{\pgfqpoint{4.046091in}{2.590331in}}{\pgfqpoint{4.035041in}{2.590331in}}%
\pgfpathcurveto{\pgfqpoint{4.023991in}{2.590331in}}{\pgfqpoint{4.013391in}{2.585941in}}{\pgfqpoint{4.005578in}{2.578127in}}%
\pgfpathcurveto{\pgfqpoint{3.997764in}{2.570313in}}{\pgfqpoint{3.993374in}{2.559714in}}{\pgfqpoint{3.993374in}{2.548664in}}%
\pgfpathcurveto{\pgfqpoint{3.993374in}{2.537614in}}{\pgfqpoint{3.997764in}{2.527015in}}{\pgfqpoint{4.005578in}{2.519201in}}%
\pgfpathcurveto{\pgfqpoint{4.013391in}{2.511388in}}{\pgfqpoint{4.023991in}{2.506997in}}{\pgfqpoint{4.035041in}{2.506997in}}%
\pgfpathclose%
\pgfusepath{stroke,fill}%
\end{pgfscope}%
\begin{pgfscope}%
\pgfpathrectangle{\pgfqpoint{1.077818in}{0.515123in}}{\pgfqpoint{3.875000in}{2.695000in}}%
\pgfusepath{clip}%
\pgfsetbuttcap%
\pgfsetroundjoin%
\definecolor{currentfill}{rgb}{0.121569,0.466667,0.705882}%
\pgfsetfillcolor{currentfill}%
\pgfsetlinewidth{1.003750pt}%
\definecolor{currentstroke}{rgb}{0.121569,0.466667,0.705882}%
\pgfsetstrokecolor{currentstroke}%
\pgfsetdash{}{0pt}%
\pgfpathmoveto{\pgfqpoint{4.220445in}{2.657189in}}%
\pgfpathcurveto{\pgfqpoint{4.231495in}{2.657189in}}{\pgfqpoint{4.242094in}{2.661579in}}{\pgfqpoint{4.249908in}{2.669393in}}%
\pgfpathcurveto{\pgfqpoint{4.257721in}{2.677207in}}{\pgfqpoint{4.262111in}{2.687806in}}{\pgfqpoint{4.262111in}{2.698856in}}%
\pgfpathcurveto{\pgfqpoint{4.262111in}{2.709906in}}{\pgfqpoint{4.257721in}{2.720505in}}{\pgfqpoint{4.249908in}{2.728319in}}%
\pgfpathcurveto{\pgfqpoint{4.242094in}{2.736132in}}{\pgfqpoint{4.231495in}{2.740522in}}{\pgfqpoint{4.220445in}{2.740522in}}%
\pgfpathcurveto{\pgfqpoint{4.209395in}{2.740522in}}{\pgfqpoint{4.198796in}{2.736132in}}{\pgfqpoint{4.190982in}{2.728319in}}%
\pgfpathcurveto{\pgfqpoint{4.183168in}{2.720505in}}{\pgfqpoint{4.178778in}{2.709906in}}{\pgfqpoint{4.178778in}{2.698856in}}%
\pgfpathcurveto{\pgfqpoint{4.178778in}{2.687806in}}{\pgfqpoint{4.183168in}{2.677207in}}{\pgfqpoint{4.190982in}{2.669393in}}%
\pgfpathcurveto{\pgfqpoint{4.198796in}{2.661579in}}{\pgfqpoint{4.209395in}{2.657189in}}{\pgfqpoint{4.220445in}{2.657189in}}%
\pgfpathclose%
\pgfusepath{stroke,fill}%
\end{pgfscope}%
\begin{pgfscope}%
\pgfpathrectangle{\pgfqpoint{1.077818in}{0.515123in}}{\pgfqpoint{3.875000in}{2.695000in}}%
\pgfusepath{clip}%
\pgfsetbuttcap%
\pgfsetroundjoin%
\definecolor{currentfill}{rgb}{0.121569,0.466667,0.705882}%
\pgfsetfillcolor{currentfill}%
\pgfsetlinewidth{1.003750pt}%
\definecolor{currentstroke}{rgb}{0.121569,0.466667,0.705882}%
\pgfsetstrokecolor{currentstroke}%
\pgfsetdash{}{0pt}%
\pgfpathmoveto{\pgfqpoint{4.405849in}{2.718654in}}%
\pgfpathcurveto{\pgfqpoint{4.416899in}{2.718654in}}{\pgfqpoint{4.427498in}{2.723044in}}{\pgfqpoint{4.435312in}{2.730858in}}%
\pgfpathcurveto{\pgfqpoint{4.443125in}{2.738671in}}{\pgfqpoint{4.447516in}{2.749270in}}{\pgfqpoint{4.447516in}{2.760320in}}%
\pgfpathcurveto{\pgfqpoint{4.447516in}{2.771371in}}{\pgfqpoint{4.443125in}{2.781970in}}{\pgfqpoint{4.435312in}{2.789783in}}%
\pgfpathcurveto{\pgfqpoint{4.427498in}{2.797597in}}{\pgfqpoint{4.416899in}{2.801987in}}{\pgfqpoint{4.405849in}{2.801987in}}%
\pgfpathcurveto{\pgfqpoint{4.394799in}{2.801987in}}{\pgfqpoint{4.384200in}{2.797597in}}{\pgfqpoint{4.376386in}{2.789783in}}%
\pgfpathcurveto{\pgfqpoint{4.368573in}{2.781970in}}{\pgfqpoint{4.364182in}{2.771371in}}{\pgfqpoint{4.364182in}{2.760320in}}%
\pgfpathcurveto{\pgfqpoint{4.364182in}{2.749270in}}{\pgfqpoint{4.368573in}{2.738671in}}{\pgfqpoint{4.376386in}{2.730858in}}%
\pgfpathcurveto{\pgfqpoint{4.384200in}{2.723044in}}{\pgfqpoint{4.394799in}{2.718654in}}{\pgfqpoint{4.405849in}{2.718654in}}%
\pgfpathclose%
\pgfusepath{stroke,fill}%
\end{pgfscope}%
\begin{pgfscope}%
\pgfpathrectangle{\pgfqpoint{1.077818in}{0.515123in}}{\pgfqpoint{3.875000in}{2.695000in}}%
\pgfusepath{clip}%
\pgfsetbuttcap%
\pgfsetroundjoin%
\definecolor{currentfill}{rgb}{0.121569,0.466667,0.705882}%
\pgfsetfillcolor{currentfill}%
\pgfsetlinewidth{1.003750pt}%
\definecolor{currentstroke}{rgb}{0.121569,0.466667,0.705882}%
\pgfsetstrokecolor{currentstroke}%
\pgfsetdash{}{0pt}%
\pgfpathmoveto{\pgfqpoint{4.591253in}{2.944907in}}%
\pgfpathcurveto{\pgfqpoint{4.602303in}{2.944907in}}{\pgfqpoint{4.612902in}{2.949297in}}{\pgfqpoint{4.620716in}{2.957111in}}%
\pgfpathcurveto{\pgfqpoint{4.628530in}{2.964924in}}{\pgfqpoint{4.632920in}{2.975523in}}{\pgfqpoint{4.632920in}{2.986574in}}%
\pgfpathcurveto{\pgfqpoint{4.632920in}{2.997624in}}{\pgfqpoint{4.628530in}{3.008223in}}{\pgfqpoint{4.620716in}{3.016036in}}%
\pgfpathcurveto{\pgfqpoint{4.612902in}{3.023850in}}{\pgfqpoint{4.602303in}{3.028240in}}{\pgfqpoint{4.591253in}{3.028240in}}%
\pgfpathcurveto{\pgfqpoint{4.580203in}{3.028240in}}{\pgfqpoint{4.569604in}{3.023850in}}{\pgfqpoint{4.561790in}{3.016036in}}%
\pgfpathcurveto{\pgfqpoint{4.553977in}{3.008223in}}{\pgfqpoint{4.549587in}{2.997624in}}{\pgfqpoint{4.549587in}{2.986574in}}%
\pgfpathcurveto{\pgfqpoint{4.549587in}{2.975523in}}{\pgfqpoint{4.553977in}{2.964924in}}{\pgfqpoint{4.561790in}{2.957111in}}%
\pgfpathcurveto{\pgfqpoint{4.569604in}{2.949297in}}{\pgfqpoint{4.580203in}{2.944907in}}{\pgfqpoint{4.591253in}{2.944907in}}%
\pgfpathclose%
\pgfusepath{stroke,fill}%
\end{pgfscope}%
\begin{pgfscope}%
\pgfpathrectangle{\pgfqpoint{1.077818in}{0.515123in}}{\pgfqpoint{3.875000in}{2.695000in}}%
\pgfusepath{clip}%
\pgfsetbuttcap%
\pgfsetroundjoin%
\definecolor{currentfill}{rgb}{0.121569,0.466667,0.705882}%
\pgfsetfillcolor{currentfill}%
\pgfsetlinewidth{1.003750pt}%
\definecolor{currentstroke}{rgb}{0.121569,0.466667,0.705882}%
\pgfsetstrokecolor{currentstroke}%
\pgfsetdash{}{0pt}%
\pgfpathmoveto{\pgfqpoint{4.776657in}{3.014654in}}%
\pgfpathcurveto{\pgfqpoint{4.787707in}{3.014654in}}{\pgfqpoint{4.798307in}{3.019044in}}{\pgfqpoint{4.806120in}{3.026858in}}%
\pgfpathcurveto{\pgfqpoint{4.813934in}{3.034672in}}{\pgfqpoint{4.818324in}{3.045271in}}{\pgfqpoint{4.818324in}{3.056321in}}%
\pgfpathcurveto{\pgfqpoint{4.818324in}{3.067371in}}{\pgfqpoint{4.813934in}{3.077970in}}{\pgfqpoint{4.806120in}{3.085784in}}%
\pgfpathcurveto{\pgfqpoint{4.798307in}{3.093597in}}{\pgfqpoint{4.787707in}{3.097988in}}{\pgfqpoint{4.776657in}{3.097988in}}%
\pgfpathcurveto{\pgfqpoint{4.765607in}{3.097988in}}{\pgfqpoint{4.755008in}{3.093597in}}{\pgfqpoint{4.747195in}{3.085784in}}%
\pgfpathcurveto{\pgfqpoint{4.739381in}{3.077970in}}{\pgfqpoint{4.734991in}{3.067371in}}{\pgfqpoint{4.734991in}{3.056321in}}%
\pgfpathcurveto{\pgfqpoint{4.734991in}{3.045271in}}{\pgfqpoint{4.739381in}{3.034672in}}{\pgfqpoint{4.747195in}{3.026858in}}%
\pgfpathcurveto{\pgfqpoint{4.755008in}{3.019044in}}{\pgfqpoint{4.765607in}{3.014654in}}{\pgfqpoint{4.776657in}{3.014654in}}%
\pgfpathclose%
\pgfusepath{stroke,fill}%
\end{pgfscope}%
\begin{pgfscope}%
\pgfsetrectcap%
\pgfsetmiterjoin%
\pgfsetlinewidth{0.803000pt}%
\definecolor{currentstroke}{rgb}{0.000000,0.000000,0.000000}%
\pgfsetstrokecolor{currentstroke}%
\pgfsetdash{}{0pt}%
\pgfpathmoveto{\pgfqpoint{1.077818in}{0.515123in}}%
\pgfpathlineto{\pgfqpoint{1.077818in}{3.210123in}}%
\pgfusepath{stroke}%
\end{pgfscope}%
\begin{pgfscope}%
\pgfsetrectcap%
\pgfsetmiterjoin%
\pgfsetlinewidth{0.803000pt}%
\definecolor{currentstroke}{rgb}{0.000000,0.000000,0.000000}%
\pgfsetstrokecolor{currentstroke}%
\pgfsetdash{}{0pt}%
\pgfpathmoveto{\pgfqpoint{4.952818in}{0.515123in}}%
\pgfpathlineto{\pgfqpoint{4.952818in}{3.210123in}}%
\pgfusepath{stroke}%
\end{pgfscope}%
\begin{pgfscope}%
\pgfsetrectcap%
\pgfsetmiterjoin%
\pgfsetlinewidth{0.803000pt}%
\definecolor{currentstroke}{rgb}{0.000000,0.000000,0.000000}%
\pgfsetstrokecolor{currentstroke}%
\pgfsetdash{}{0pt}%
\pgfpathmoveto{\pgfqpoint{1.077818in}{0.515123in}}%
\pgfpathlineto{\pgfqpoint{4.952818in}{0.515123in}}%
\pgfusepath{stroke}%
\end{pgfscope}%
\begin{pgfscope}%
\pgfsetrectcap%
\pgfsetmiterjoin%
\pgfsetlinewidth{0.803000pt}%
\definecolor{currentstroke}{rgb}{0.000000,0.000000,0.000000}%
\pgfsetstrokecolor{currentstroke}%
\pgfsetdash{}{0pt}%
\pgfpathmoveto{\pgfqpoint{1.077818in}{3.210123in}}%
\pgfpathlineto{\pgfqpoint{4.952818in}{3.210123in}}%
\pgfusepath{stroke}%
\end{pgfscope}%
\begin{pgfscope}%
\pgfpathrectangle{\pgfqpoint{1.077818in}{0.515123in}}{\pgfqpoint{3.875000in}{2.695000in}}%
\pgfusepath{clip}%
\pgfsetbuttcap%
\pgfsetroundjoin%
\definecolor{currentfill}{rgb}{1.000000,0.388235,0.278431}%
\pgfsetfillcolor{currentfill}%
\pgfsetlinewidth{1.003750pt}%
\definecolor{currentstroke}{rgb}{1.000000,0.388235,0.278431}%
\pgfsetstrokecolor{currentstroke}%
\pgfsetdash{}{0pt}%
\pgfpathmoveto{\pgfqpoint{2.950120in}{1.781319in}}%
\pgfpathcurveto{\pgfqpoint{2.961170in}{1.781319in}}{\pgfqpoint{2.971769in}{1.785709in}}{\pgfqpoint{2.979583in}{1.793523in}}%
\pgfpathcurveto{\pgfqpoint{2.987396in}{1.801336in}}{\pgfqpoint{2.991787in}{1.811935in}}{\pgfqpoint{2.991787in}{1.822986in}}%
\pgfpathcurveto{\pgfqpoint{2.991787in}{1.834036in}}{\pgfqpoint{2.987396in}{1.844635in}}{\pgfqpoint{2.979583in}{1.852448in}}%
\pgfpathcurveto{\pgfqpoint{2.971769in}{1.860262in}}{\pgfqpoint{2.961170in}{1.864652in}}{\pgfqpoint{2.950120in}{1.864652in}}%
\pgfpathcurveto{\pgfqpoint{2.939070in}{1.864652in}}{\pgfqpoint{2.928471in}{1.860262in}}{\pgfqpoint{2.920657in}{1.852448in}}%
\pgfpathcurveto{\pgfqpoint{2.912843in}{1.844635in}}{\pgfqpoint{2.908453in}{1.834036in}}{\pgfqpoint{2.908453in}{1.822986in}}%
\pgfpathcurveto{\pgfqpoint{2.908453in}{1.811935in}}{\pgfqpoint{2.912843in}{1.801336in}}{\pgfqpoint{2.920657in}{1.793523in}}%
\pgfpathcurveto{\pgfqpoint{2.928471in}{1.785709in}}{\pgfqpoint{2.939070in}{1.781319in}}{\pgfqpoint{2.950120in}{1.781319in}}%
\pgfpathclose%
\pgfusepath{stroke,fill}%
\end{pgfscope}%
\end{pgfpicture}%
\makeatother%
\endgroup%

    \caption{$\frac{V_{mR}}{V_{mC}}$ frente a la frecuencia ($f$). Representación ajustada y con la frecuencia de corte marcada.}
  \end{figure}

  Aquí la frecuencia de corte está representada por el punto rojo, y se calcula añadiendo este pequeño código de \code{python} que ajusta la recta y averigua el $x$ que corresponde a $y = 1$. En este caso obtenemos una frecuencia de corte de 1308.90 Hz, que si bien no es exactamente igual a la teórica, se aproxima razonablemente.

  \begin{python}
    m, b = np.polyfit(x, y, 1)
    plt.plot(x, m*x + b, zorder=1, color="skyblue")
    x2 = (1 - b)/m
    plt.scatter(x2, 1, zorder=3, color="tomato")
    print("Frecuencia corte:", x2)
  \end{python}



  \newpage
  \section{Desfase entre señales}

  Ahora utilizaremos el mismo circuito que en el apartado anterior, configurado para medir el voltaje total ($V_m$, en este caso llamado $V_G$) y el voltaje en bornes de la resistencia ($V_{mR}$, en este caso llamado $V_R$).

  \begin{figure}[H]
    \centering
    \begin{circuitikz}[european]
      \draw (-2,1.5) node[oscopeshape](os){CH1};
      \draw (os.in 1) to[short, *-] ++(0,-1.2) node[tlground]{};
      \draw (os.in 2) to[short, *-] ++(0,-0.6) -- (0,0.6) node[]{};
      \draw (3,3) node[oscopeshape](os2){CH2};
      \draw (0,0) to[sV, l=$V_{in}$] (0,3)
      to[capacitor, l=C] (os2.left);
      \draw (os2.south) to[R, l=R] (3,0);
      \draw (0,0) node[tlground]{};
      \draw (3,0) node[tlground]{};
      \draw (1.5,0) node[]{$V_{out}$};
    \end{circuitikz}
    \caption{Circuito RC preparado para mediciones con el osciloscopio}
  \end{figure}

  \subsection{Procedimiento de medición}

  Vamos a configurar el osciloscopio para una observación dual, colocando las dos señales (CH1 y CH2) en la pantalla. Utilizamos los diales para ajustar la posición y ampliación de las curvas y cambiamos el modo a "Tiempo". Esto nos da acceso a dos cursores, \textit{t1} y \textit{t2}, que podremos mover en cada curva.

  \begin{figure}[H]
    \centering
    %% Creator: Matplotlib, PGF backend
%%
%% To include the figure in your LaTeX document, write
%%   \input{<filename>.pgf}
%%
%% Make sure the required packages are loaded in your preamble
%%   \usepackage{pgf}
%%
%% Figures using additional raster images can only be included by \input if
%% they are in the same directory as the main LaTeX file. For loading figures
%% from other directories you can use the `import` package
%%   \usepackage{import}
%% and then include the figures with
%%   \import{<path to file>}{<filename>.pgf}
%%
%% Matplotlib used the following preamble
%%
\begingroup%
\makeatletter%
\begin{pgfpicture}%
\pgfpathrectangle{\pgfpointorigin}{\pgfqpoint{4.075000in}{2.125000in}}%
\pgfusepath{use as bounding box, clip}%
\begin{pgfscope}%
\pgfpathrectangle{\pgfqpoint{0.100000in}{0.100000in}}{\pgfqpoint{3.875000in}{1.925000in}}%
\pgfusepath{clip}%
\pgfsetrectcap%
\pgfsetroundjoin%
\pgfsetlinewidth{0.301125pt}%
\definecolor{currentstroke}{rgb}{0.000000,0.000000,0.000000}%
\pgfsetstrokecolor{currentstroke}%
\pgfsetdash{}{0pt}%
\pgfpathmoveto{\pgfqpoint{0.276136in}{0.187500in}}%
\pgfpathlineto{\pgfqpoint{0.276136in}{0.279605in}}%
\pgfpathlineto{\pgfqpoint{0.276136in}{0.371711in}}%
\pgfpathlineto{\pgfqpoint{0.276136in}{0.463816in}}%
\pgfpathlineto{\pgfqpoint{0.276136in}{0.555921in}}%
\pgfpathlineto{\pgfqpoint{0.276136in}{0.648026in}}%
\pgfpathlineto{\pgfqpoint{0.276136in}{0.740132in}}%
\pgfpathlineto{\pgfqpoint{0.276136in}{0.832237in}}%
\pgfpathlineto{\pgfqpoint{0.276136in}{0.924342in}}%
\pgfpathlineto{\pgfqpoint{0.276136in}{1.016447in}}%
\pgfpathlineto{\pgfqpoint{0.276136in}{1.108553in}}%
\pgfpathlineto{\pgfqpoint{0.276136in}{1.200658in}}%
\pgfpathlineto{\pgfqpoint{0.276136in}{1.292763in}}%
\pgfpathlineto{\pgfqpoint{0.276136in}{1.384868in}}%
\pgfpathlineto{\pgfqpoint{0.276136in}{1.476974in}}%
\pgfpathlineto{\pgfqpoint{0.276136in}{1.569079in}}%
\pgfpathlineto{\pgfqpoint{0.276136in}{1.661184in}}%
\pgfpathlineto{\pgfqpoint{0.276136in}{1.753289in}}%
\pgfpathlineto{\pgfqpoint{0.276136in}{1.845395in}}%
\pgfpathlineto{\pgfqpoint{0.276136in}{1.937500in}}%
\pgfusepath{stroke}%
\end{pgfscope}%
\begin{pgfscope}%
\pgfpathrectangle{\pgfqpoint{0.100000in}{0.100000in}}{\pgfqpoint{3.875000in}{1.925000in}}%
\pgfusepath{clip}%
\pgfsetrectcap%
\pgfsetroundjoin%
\pgfsetlinewidth{0.301125pt}%
\definecolor{currentstroke}{rgb}{0.000000,0.000000,0.000000}%
\pgfsetstrokecolor{currentstroke}%
\pgfsetdash{}{0pt}%
\pgfpathmoveto{\pgfqpoint{0.276136in}{1.108553in}}%
\pgfpathlineto{\pgfqpoint{0.389773in}{1.108553in}}%
\pgfpathlineto{\pgfqpoint{0.503409in}{1.108553in}}%
\pgfpathlineto{\pgfqpoint{0.617045in}{1.108553in}}%
\pgfpathlineto{\pgfqpoint{0.730682in}{1.108553in}}%
\pgfpathlineto{\pgfqpoint{0.844318in}{1.108553in}}%
\pgfpathlineto{\pgfqpoint{0.957955in}{1.108553in}}%
\pgfpathlineto{\pgfqpoint{1.071591in}{1.108553in}}%
\pgfpathlineto{\pgfqpoint{1.185227in}{1.108553in}}%
\pgfpathlineto{\pgfqpoint{1.298864in}{1.108553in}}%
\pgfpathlineto{\pgfqpoint{1.412500in}{1.108553in}}%
\pgfpathlineto{\pgfqpoint{1.526136in}{1.108553in}}%
\pgfpathlineto{\pgfqpoint{1.639773in}{1.108553in}}%
\pgfpathlineto{\pgfqpoint{1.753409in}{1.108553in}}%
\pgfpathlineto{\pgfqpoint{1.867045in}{1.108553in}}%
\pgfpathlineto{\pgfqpoint{1.980682in}{1.108553in}}%
\pgfpathlineto{\pgfqpoint{2.094318in}{1.108553in}}%
\pgfpathlineto{\pgfqpoint{2.207955in}{1.108553in}}%
\pgfpathlineto{\pgfqpoint{2.321591in}{1.108553in}}%
\pgfpathlineto{\pgfqpoint{2.435227in}{1.108553in}}%
\pgfpathlineto{\pgfqpoint{2.548864in}{1.108553in}}%
\pgfpathlineto{\pgfqpoint{2.662500in}{1.108553in}}%
\pgfpathlineto{\pgfqpoint{2.776136in}{1.108553in}}%
\pgfpathlineto{\pgfqpoint{2.889773in}{1.108553in}}%
\pgfpathlineto{\pgfqpoint{3.003409in}{1.108553in}}%
\pgfpathlineto{\pgfqpoint{3.117045in}{1.108553in}}%
\pgfpathlineto{\pgfqpoint{3.230682in}{1.108553in}}%
\pgfpathlineto{\pgfqpoint{3.344318in}{1.108553in}}%
\pgfpathlineto{\pgfqpoint{3.457955in}{1.108553in}}%
\pgfpathlineto{\pgfqpoint{3.571591in}{1.108553in}}%
\pgfpathlineto{\pgfqpoint{3.685227in}{1.108553in}}%
\pgfpathlineto{\pgfqpoint{3.798864in}{1.108553in}}%
\pgfusepath{stroke}%
\end{pgfscope}%
\begin{pgfscope}%
\pgfpathrectangle{\pgfqpoint{0.100000in}{0.100000in}}{\pgfqpoint{3.875000in}{1.925000in}}%
\pgfusepath{clip}%
\pgfsetrectcap%
\pgfsetroundjoin%
\pgfsetlinewidth{1.505625pt}%
\definecolor{currentstroke}{rgb}{0.117647,0.564706,1.000000}%
\pgfsetstrokecolor{currentstroke}%
\pgfsetdash{}{0pt}%
\pgfpathmoveto{\pgfqpoint{0.276136in}{1.707237in}}%
\pgfpathlineto{\pgfqpoint{0.332955in}{1.704246in}}%
\pgfpathlineto{\pgfqpoint{0.389773in}{1.695303in}}%
\pgfpathlineto{\pgfqpoint{0.446591in}{1.680498in}}%
\pgfpathlineto{\pgfqpoint{0.503409in}{1.659977in}}%
\pgfpathlineto{\pgfqpoint{0.560227in}{1.633947in}}%
\pgfpathlineto{\pgfqpoint{0.617045in}{1.602668in}}%
\pgfpathlineto{\pgfqpoint{0.673864in}{1.566452in}}%
\pgfpathlineto{\pgfqpoint{0.730682in}{1.525660in}}%
\pgfpathlineto{\pgfqpoint{0.787500in}{1.480701in}}%
\pgfpathlineto{\pgfqpoint{0.844318in}{1.432023in}}%
\pgfpathlineto{\pgfqpoint{0.901136in}{1.380113in}}%
\pgfpathlineto{\pgfqpoint{0.957955in}{1.325490in}}%
\pgfpathlineto{\pgfqpoint{1.014773in}{1.268700in}}%
\pgfpathlineto{\pgfqpoint{1.071591in}{1.210309in}}%
\pgfpathlineto{\pgfqpoint{1.128409in}{1.150902in}}%
\pgfpathlineto{\pgfqpoint{1.185227in}{1.091071in}}%
\pgfpathlineto{\pgfqpoint{1.242045in}{1.031415in}}%
\pgfpathlineto{\pgfqpoint{1.298864in}{0.972530in}}%
\pgfpathlineto{\pgfqpoint{1.355682in}{0.915004in}}%
\pgfpathlineto{\pgfqpoint{1.412500in}{0.859412in}}%
\pgfpathlineto{\pgfqpoint{1.469318in}{0.806309in}}%
\pgfpathlineto{\pgfqpoint{1.526136in}{0.756226in}}%
\pgfpathlineto{\pgfqpoint{1.582955in}{0.709664in}}%
\pgfpathlineto{\pgfqpoint{1.639773in}{0.667087in}}%
\pgfpathlineto{\pgfqpoint{1.696591in}{0.628921in}}%
\pgfpathlineto{\pgfqpoint{1.753409in}{0.595547in}}%
\pgfpathlineto{\pgfqpoint{1.810227in}{0.567299in}}%
\pgfpathlineto{\pgfqpoint{1.867045in}{0.544459in}}%
\pgfpathlineto{\pgfqpoint{1.923864in}{0.527255in}}%
\pgfpathlineto{\pgfqpoint{1.980682in}{0.515860in}}%
\pgfpathlineto{\pgfqpoint{2.037500in}{0.510386in}}%
\pgfpathlineto{\pgfqpoint{2.094318in}{0.510889in}}%
\pgfpathlineto{\pgfqpoint{2.151136in}{0.517364in}}%
\pgfpathlineto{\pgfqpoint{2.207955in}{0.529746in}}%
\pgfpathlineto{\pgfqpoint{2.264773in}{0.547911in}}%
\pgfpathlineto{\pgfqpoint{2.321591in}{0.571678in}}%
\pgfpathlineto{\pgfqpoint{2.378409in}{0.600809in}}%
\pgfpathlineto{\pgfqpoint{2.435227in}{0.635013in}}%
\pgfpathlineto{\pgfqpoint{2.492045in}{0.673948in}}%
\pgfpathlineto{\pgfqpoint{2.548864in}{0.717227in}}%
\pgfpathlineto{\pgfqpoint{2.605682in}{0.764415in}}%
\pgfpathlineto{\pgfqpoint{2.662500in}{0.815041in}}%
\pgfpathlineto{\pgfqpoint{2.719318in}{0.868600in}}%
\pgfpathlineto{\pgfqpoint{2.776136in}{0.924557in}}%
\pgfpathlineto{\pgfqpoint{2.832955in}{0.982353in}}%
\pgfpathlineto{\pgfqpoint{2.889773in}{1.041409in}}%
\pgfpathlineto{\pgfqpoint{2.946591in}{1.101136in}}%
\pgfpathlineto{\pgfqpoint{3.003409in}{1.160937in}}%
\pgfpathlineto{\pgfqpoint{3.060227in}{1.220215in}}%
\pgfpathlineto{\pgfqpoint{3.117045in}{1.278377in}}%
\pgfpathlineto{\pgfqpoint{3.173864in}{1.334842in}}%
\pgfpathlineto{\pgfqpoint{3.230682in}{1.389046in}}%
\pgfpathlineto{\pgfqpoint{3.287500in}{1.440448in}}%
\pgfpathlineto{\pgfqpoint{3.344318in}{1.488533in}}%
\pgfpathlineto{\pgfqpoint{3.401136in}{1.532822in}}%
\pgfpathlineto{\pgfqpoint{3.457955in}{1.572872in}}%
\pgfpathlineto{\pgfqpoint{3.514773in}{1.608282in}}%
\pgfpathlineto{\pgfqpoint{3.571591in}{1.638699in}}%
\pgfpathlineto{\pgfqpoint{3.628409in}{1.663819in}}%
\pgfpathlineto{\pgfqpoint{3.685227in}{1.683391in}}%
\pgfpathlineto{\pgfqpoint{3.742045in}{1.697220in}}%
\pgfpathlineto{\pgfqpoint{3.798864in}{1.705167in}}%
\pgfusepath{stroke}%
\end{pgfscope}%
\begin{pgfscope}%
\pgfpathrectangle{\pgfqpoint{0.100000in}{0.100000in}}{\pgfqpoint{3.875000in}{1.925000in}}%
\pgfusepath{clip}%
\pgfsetrectcap%
\pgfsetroundjoin%
\pgfsetlinewidth{1.505625pt}%
\definecolor{currentstroke}{rgb}{1.000000,0.388235,0.278431}%
\pgfsetstrokecolor{currentstroke}%
\pgfsetdash{}{0pt}%
\pgfpathmoveto{\pgfqpoint{0.276136in}{1.627028in}}%
\pgfpathlineto{\pgfqpoint{0.332955in}{1.594554in}}%
\pgfpathlineto{\pgfqpoint{0.389773in}{1.557223in}}%
\pgfpathlineto{\pgfqpoint{0.446591in}{1.515410in}}%
\pgfpathlineto{\pgfqpoint{0.503409in}{1.469531in}}%
\pgfpathlineto{\pgfqpoint{0.560227in}{1.420046in}}%
\pgfpathlineto{\pgfqpoint{0.617045in}{1.367448in}}%
\pgfpathlineto{\pgfqpoint{0.673864in}{1.312263in}}%
\pgfpathlineto{\pgfqpoint{0.730682in}{1.255043in}}%
\pgfpathlineto{\pgfqpoint{0.787500in}{1.196360in}}%
\pgfpathlineto{\pgfqpoint{0.844318in}{1.136799in}}%
\pgfpathlineto{\pgfqpoint{0.901136in}{1.076955in}}%
\pgfpathlineto{\pgfqpoint{0.957955in}{1.017428in}}%
\pgfpathlineto{\pgfqpoint{1.014773in}{0.958811in}}%
\pgfpathlineto{\pgfqpoint{1.071591in}{0.901690in}}%
\pgfpathlineto{\pgfqpoint{1.128409in}{0.846636in}}%
\pgfpathlineto{\pgfqpoint{1.185227in}{0.794199in}}%
\pgfpathlineto{\pgfqpoint{1.242045in}{0.744903in}}%
\pgfpathlineto{\pgfqpoint{1.298864in}{0.699240in}}%
\pgfpathlineto{\pgfqpoint{1.355682in}{0.657667in}}%
\pgfpathlineto{\pgfqpoint{1.412500in}{0.620600in}}%
\pgfpathlineto{\pgfqpoint{1.469318in}{0.588407in}}%
\pgfpathlineto{\pgfqpoint{1.526136in}{0.561412in}}%
\pgfpathlineto{\pgfqpoint{1.582955in}{0.539884in}}%
\pgfpathlineto{\pgfqpoint{1.639773in}{0.524037in}}%
\pgfpathlineto{\pgfqpoint{1.696591in}{0.514031in}}%
\pgfpathlineto{\pgfqpoint{1.753409in}{0.509965in}}%
\pgfpathlineto{\pgfqpoint{1.810227in}{0.511880in}}%
\pgfpathlineto{\pgfqpoint{1.867045in}{0.519757in}}%
\pgfpathlineto{\pgfqpoint{1.923864in}{0.533517in}}%
\pgfpathlineto{\pgfqpoint{1.980682in}{0.553022in}}%
\pgfpathlineto{\pgfqpoint{2.037500in}{0.578078in}}%
\pgfpathlineto{\pgfqpoint{2.094318in}{0.608435in}}%
\pgfpathlineto{\pgfqpoint{2.151136in}{0.643788in}}%
\pgfpathlineto{\pgfqpoint{2.207955in}{0.683785in}}%
\pgfpathlineto{\pgfqpoint{2.264773in}{0.728027in}}%
\pgfpathlineto{\pgfqpoint{2.321591in}{0.776070in}}%
\pgfpathlineto{\pgfqpoint{2.378409in}{0.827436in}}%
\pgfpathlineto{\pgfqpoint{2.435227in}{0.881610in}}%
\pgfpathlineto{\pgfqpoint{2.492045in}{0.938052in}}%
\pgfpathlineto{\pgfqpoint{2.548864in}{0.996197in}}%
\pgfpathlineto{\pgfqpoint{2.605682in}{1.055465in}}%
\pgfpathlineto{\pgfqpoint{2.662500in}{1.115264in}}%
\pgfpathlineto{\pgfqpoint{2.719318in}{1.174995in}}%
\pgfpathlineto{\pgfqpoint{2.776136in}{1.234063in}}%
\pgfpathlineto{\pgfqpoint{2.832955in}{1.291876in}}%
\pgfpathlineto{\pgfqpoint{2.889773in}{1.347858in}}%
\pgfpathlineto{\pgfqpoint{2.946591in}{1.401449in}}%
\pgfpathlineto{\pgfqpoint{3.003409in}{1.452113in}}%
\pgfpathlineto{\pgfqpoint{3.060227in}{1.499344in}}%
\pgfpathlineto{\pgfqpoint{3.117045in}{1.542671in}}%
\pgfpathlineto{\pgfqpoint{3.173864in}{1.581660in}}%
\pgfpathlineto{\pgfqpoint{3.230682in}{1.615922in}}%
\pgfpathlineto{\pgfqpoint{3.287500in}{1.645115in}}%
\pgfpathlineto{\pgfqpoint{3.344318in}{1.668946in}}%
\pgfpathlineto{\pgfqpoint{3.401136in}{1.687179in}}%
\pgfpathlineto{\pgfqpoint{3.457955in}{1.699629in}}%
\pgfpathlineto{\pgfqpoint{3.514773in}{1.706174in}}%
\pgfpathlineto{\pgfqpoint{3.571591in}{1.706748in}}%
\pgfpathlineto{\pgfqpoint{3.628409in}{1.701345in}}%
\pgfpathlineto{\pgfqpoint{3.685227in}{1.690018in}}%
\pgfpathlineto{\pgfqpoint{3.742045in}{1.672882in}}%
\pgfpathlineto{\pgfqpoint{3.798864in}{1.650108in}}%
\pgfusepath{stroke}%
\end{pgfscope}%
\begin{pgfscope}%
\pgfpathrectangle{\pgfqpoint{0.100000in}{0.100000in}}{\pgfqpoint{3.875000in}{1.925000in}}%
\pgfusepath{clip}%
\pgfsetrectcap%
\pgfsetroundjoin%
\pgfsetlinewidth{1.505625pt}%
\definecolor{currentstroke}{rgb}{0.529412,0.807843,0.921569}%
\pgfsetstrokecolor{currentstroke}%
\pgfsetdash{}{0pt}%
\pgfpathmoveto{\pgfqpoint{2.953630in}{0.325658in}}%
\pgfpathlineto{\pgfqpoint{2.953630in}{0.417763in}}%
\pgfpathlineto{\pgfqpoint{2.953630in}{0.509868in}}%
\pgfpathlineto{\pgfqpoint{2.953630in}{0.601974in}}%
\pgfpathlineto{\pgfqpoint{2.953630in}{0.694079in}}%
\pgfpathlineto{\pgfqpoint{2.953630in}{0.786184in}}%
\pgfpathlineto{\pgfqpoint{2.953630in}{0.878289in}}%
\pgfpathlineto{\pgfqpoint{2.953630in}{0.970395in}}%
\pgfpathlineto{\pgfqpoint{2.953630in}{1.062500in}}%
\pgfpathlineto{\pgfqpoint{2.953630in}{1.154605in}}%
\pgfpathlineto{\pgfqpoint{2.953630in}{1.246711in}}%
\pgfpathlineto{\pgfqpoint{2.953630in}{1.338816in}}%
\pgfpathlineto{\pgfqpoint{2.953630in}{1.430921in}}%
\pgfpathlineto{\pgfqpoint{2.953630in}{1.523026in}}%
\pgfpathlineto{\pgfqpoint{2.953630in}{1.615132in}}%
\pgfpathlineto{\pgfqpoint{2.953630in}{1.707237in}}%
\pgfpathlineto{\pgfqpoint{2.953630in}{1.799342in}}%
\pgfusepath{stroke}%
\end{pgfscope}%
\begin{pgfscope}%
\pgfpathrectangle{\pgfqpoint{0.100000in}{0.100000in}}{\pgfqpoint{3.875000in}{1.925000in}}%
\pgfusepath{clip}%
\pgfsetrectcap%
\pgfsetroundjoin%
\pgfsetlinewidth{1.505625pt}%
\definecolor{currentstroke}{rgb}{1.000000,0.854902,0.725490}%
\pgfsetstrokecolor{currentstroke}%
\pgfsetdash{}{0pt}%
\pgfpathmoveto{\pgfqpoint{2.656131in}{0.325658in}}%
\pgfpathlineto{\pgfqpoint{2.656131in}{0.417763in}}%
\pgfpathlineto{\pgfqpoint{2.656131in}{0.509868in}}%
\pgfpathlineto{\pgfqpoint{2.656131in}{0.601974in}}%
\pgfpathlineto{\pgfqpoint{2.656131in}{0.694079in}}%
\pgfpathlineto{\pgfqpoint{2.656131in}{0.786184in}}%
\pgfpathlineto{\pgfqpoint{2.656131in}{0.878289in}}%
\pgfpathlineto{\pgfqpoint{2.656131in}{0.970395in}}%
\pgfpathlineto{\pgfqpoint{2.656131in}{1.062500in}}%
\pgfpathlineto{\pgfqpoint{2.656131in}{1.154605in}}%
\pgfpathlineto{\pgfqpoint{2.656131in}{1.246711in}}%
\pgfpathlineto{\pgfqpoint{2.656131in}{1.338816in}}%
\pgfpathlineto{\pgfqpoint{2.656131in}{1.430921in}}%
\pgfpathlineto{\pgfqpoint{2.656131in}{1.523026in}}%
\pgfpathlineto{\pgfqpoint{2.656131in}{1.615132in}}%
\pgfpathlineto{\pgfqpoint{2.656131in}{1.707237in}}%
\pgfpathlineto{\pgfqpoint{2.656131in}{1.799342in}}%
\pgfusepath{stroke}%
\end{pgfscope}%
\begin{pgfscope}%
\definecolor{textcolor}{rgb}{0.121569,0.466667,0.705882}%
\pgfsetstrokecolor{textcolor}%
\pgfsetfillcolor{textcolor}%
\pgftext[x=1.054998in,y=1.523026in,left,base]{\color{textcolor}\rmfamily\fontsize{12.000000}{14.400000}\selectfont \(\displaystyle V_G\)}%
\end{pgfscope}%
\begin{pgfscope}%
\definecolor{textcolor}{rgb}{1.000000,0.388235,0.278431}%
\pgfsetstrokecolor{textcolor}%
\pgfsetfillcolor{textcolor}%
\pgftext[x=0.884543in,y=0.509868in,left,base]{\color{textcolor}\rmfamily\fontsize{12.000000}{14.400000}\selectfont \(\displaystyle V_R\)}%
\end{pgfscope}%
\begin{pgfscope}%
\definecolor{textcolor}{rgb}{0.121569,0.466667,0.705882}%
\pgfsetstrokecolor{textcolor}%
\pgfsetfillcolor{textcolor}%
\pgftext[x=3.067266in,y=0.371711in,left,base]{\color{textcolor}\rmfamily\fontsize{12.000000}{14.400000}\selectfont t1}%
\end{pgfscope}%
\begin{pgfscope}%
\definecolor{textcolor}{rgb}{1.000000,0.388235,0.278431}%
\pgfsetstrokecolor{textcolor}%
\pgfsetfillcolor{textcolor}%
\pgftext[x=2.372040in,y=0.371711in,left,base]{\color{textcolor}\rmfamily\fontsize{12.000000}{14.400000}\selectfont t2}%
\end{pgfscope}%
\begin{pgfscope}%
\definecolor{textcolor}{rgb}{0.000000,0.000000,0.000000}%
\pgfsetstrokecolor{textcolor}%
\pgfsetfillcolor{textcolor}%
\pgftext[x=3.756878in,y=0.924342in,left,base]{\color{textcolor}\rmfamily\fontsize{12.000000}{14.400000}\selectfont t}%
\end{pgfscope}%
\begin{pgfscope}%
\pgfpathrectangle{\pgfqpoint{0.100000in}{0.100000in}}{\pgfqpoint{3.875000in}{1.925000in}}%
\pgfusepath{clip}%
\pgfsetbuttcap%
\pgfsetroundjoin%
\definecolor{currentfill}{rgb}{0.117647,0.564706,1.000000}%
\pgfsetfillcolor{currentfill}%
\pgfsetlinewidth{1.003750pt}%
\definecolor{currentstroke}{rgb}{0.117647,0.564706,1.000000}%
\pgfsetstrokecolor{currentstroke}%
\pgfsetdash{}{0pt}%
\pgfpathmoveto{\pgfqpoint{2.953630in}{1.066886in}}%
\pgfpathcurveto{\pgfqpoint{2.964680in}{1.066886in}}{\pgfqpoint{2.975279in}{1.071276in}}{\pgfqpoint{2.983093in}{1.079090in}}%
\pgfpathcurveto{\pgfqpoint{2.990907in}{1.086903in}}{\pgfqpoint{2.995297in}{1.097503in}}{\pgfqpoint{2.995297in}{1.108553in}}%
\pgfpathcurveto{\pgfqpoint{2.995297in}{1.119603in}}{\pgfqpoint{2.990907in}{1.130202in}}{\pgfqpoint{2.983093in}{1.138015in}}%
\pgfpathcurveto{\pgfqpoint{2.975279in}{1.145829in}}{\pgfqpoint{2.964680in}{1.150219in}}{\pgfqpoint{2.953630in}{1.150219in}}%
\pgfpathcurveto{\pgfqpoint{2.942580in}{1.150219in}}{\pgfqpoint{2.931981in}{1.145829in}}{\pgfqpoint{2.924167in}{1.138015in}}%
\pgfpathcurveto{\pgfqpoint{2.916354in}{1.130202in}}{\pgfqpoint{2.911963in}{1.119603in}}{\pgfqpoint{2.911963in}{1.108553in}}%
\pgfpathcurveto{\pgfqpoint{2.911963in}{1.097503in}}{\pgfqpoint{2.916354in}{1.086903in}}{\pgfqpoint{2.924167in}{1.079090in}}%
\pgfpathcurveto{\pgfqpoint{2.931981in}{1.071276in}}{\pgfqpoint{2.942580in}{1.066886in}}{\pgfqpoint{2.953630in}{1.066886in}}%
\pgfpathclose%
\pgfusepath{stroke,fill}%
\end{pgfscope}%
\begin{pgfscope}%
\pgfpathrectangle{\pgfqpoint{0.100000in}{0.100000in}}{\pgfqpoint{3.875000in}{1.925000in}}%
\pgfusepath{clip}%
\pgfsetbuttcap%
\pgfsetroundjoin%
\definecolor{currentfill}{rgb}{1.000000,0.627451,0.478431}%
\pgfsetfillcolor{currentfill}%
\pgfsetlinewidth{1.003750pt}%
\definecolor{currentstroke}{rgb}{1.000000,0.627451,0.478431}%
\pgfsetstrokecolor{currentstroke}%
\pgfsetdash{}{0pt}%
\pgfpathmoveto{\pgfqpoint{2.656131in}{1.066886in}}%
\pgfpathcurveto{\pgfqpoint{2.667181in}{1.066886in}}{\pgfqpoint{2.677780in}{1.071276in}}{\pgfqpoint{2.685594in}{1.079090in}}%
\pgfpathcurveto{\pgfqpoint{2.693407in}{1.086903in}}{\pgfqpoint{2.697797in}{1.097503in}}{\pgfqpoint{2.697797in}{1.108553in}}%
\pgfpathcurveto{\pgfqpoint{2.697797in}{1.119603in}}{\pgfqpoint{2.693407in}{1.130202in}}{\pgfqpoint{2.685594in}{1.138015in}}%
\pgfpathcurveto{\pgfqpoint{2.677780in}{1.145829in}}{\pgfqpoint{2.667181in}{1.150219in}}{\pgfqpoint{2.656131in}{1.150219in}}%
\pgfpathcurveto{\pgfqpoint{2.645081in}{1.150219in}}{\pgfqpoint{2.634482in}{1.145829in}}{\pgfqpoint{2.626668in}{1.138015in}}%
\pgfpathcurveto{\pgfqpoint{2.618854in}{1.130202in}}{\pgfqpoint{2.614464in}{1.119603in}}{\pgfqpoint{2.614464in}{1.108553in}}%
\pgfpathcurveto{\pgfqpoint{2.614464in}{1.097503in}}{\pgfqpoint{2.618854in}{1.086903in}}{\pgfqpoint{2.626668in}{1.079090in}}%
\pgfpathcurveto{\pgfqpoint{2.634482in}{1.071276in}}{\pgfqpoint{2.645081in}{1.066886in}}{\pgfqpoint{2.656131in}{1.066886in}}%
\pgfpathclose%
\pgfusepath{stroke,fill}%
\end{pgfscope}%
\end{pgfpicture}%
\makeatother%
\endgroup%

    \caption{Visualización de $V_G$ y $V_R$ en el osciloscopio}
  \end{figure}

  Podemos observar en la figura que las curvas $V_G$ y $V_R$ intersecan al eje horizontal en diversos puntos. Tomaremos dos puntos contiguos en los que la distancia entre ambas curvas sea la mínima, y nombraremos los cursores \textit{t1} para $V_G$ y \textit{t2} para $V_R$. Utilizando estos dos datos, podemos calcular la diferencia de fase entre ambas señales, $\varphi$.

  \begin{equation}
    \varphi = -2\pi f (t_2 - t_1) = -2\pi f \Delta t
    \label{ec:desfase}
  \end{equation}

  De aquí en adelante llamaremos $\Delta t = (t_2 - t_1)$ a la diferencia entre ambos tiempos. Con esta información procederemos a las mediciones.

  \subsection{Medición experimental}

  Procederemos de la siguiente manera. Variaremos la frecuencia en torno a la frecuencia de corte y mediremos $\Delta t$ utilizando el método descrito en la sección anterior. Completaremos una tabla de la siguiente manera:

  \begin{table}[H]
  \centering
  \csvreader[
    tabular=|c|c|c|,
    table head=\hline Medida & $f (Hz)$ & $\Delta t (s)$ \\ \hline,
    late after last line=\\\hline,
    filter test=\ifnumless{\thecsvrow}{2},
    separator=semicolon
    ]{CA5.csv}
    {f=\f, deltat=\dt}
    {\thecsvrow & \f & \dt}
  \caption{Ejemplo de las mediciones de $\Delta t$ según la frecuencia}
  \end{table}

  Antes de completar todos los resultados añadiremos tres columnas adicionales a la tabla. Estas serán el $\log f$ y el desfase ($\varphi$), tanto en radianes como en grados sexagesimales. Así podremos representar la gráfica de la próxima sección. Ahora si, presentamos los resultados obtenidos:

  \begin{table}[H]
  \centering
  \csvreader[
    tabular=|c|c|c|c|c|c|,
    table head=\hline Medida & $f (Hz)$ & $\log f$ & $\Delta t (s)$ & $\varphi (rad)$ & $\varphi$ \textit{(º)} \\ \hline,
    late after last line=\\\hline,
    separator=semicolon
    ]{CA5.csv}
    {f=\f, logf=\logf, deltat=\dt, phirad=\phir, phideg=\phid}
    {\thecsvrow & \f & \logf & \dt & \phir & \phid}
  \caption{Mediciones de $\Delta t$ según la frecuencia}
  \end{table}

  Cómo en el apartado anterior, tuvimos tiempo extra en esta práctica, y tras tratar de revisar un problema (expuesto en las conclusiones) decidimos emplearlo en tomar más medidas, para frecuencias más pequeñas y más grandes. En el caso de frecuencias pequeñas, el "ruído" de las ondas impedía hacer una estimación apropiada, puesto que añadía una incertidbumbre considerable. Para frecuencias grandes, el valor de $\Delta t$ se hacía tan pequeño que también comenzaba a ser complicado calcularlo con certeza. Por eso le daremos menor importancia a estos resultados extremos, teniendo en cuenta el error que pueden acarrear. Están expuestas en el siguiente anexo (\ref{extra2}).

  \begin{figure}[H]
    \centering
    %% Creator: Matplotlib, PGF backend
%%
%% To include the figure in your LaTeX document, write
%%   \input{<filename>.pgf}
%%
%% Make sure the required packages are loaded in your preamble
%%   \usepackage{pgf}
%%
%% Figures using additional raster images can only be included by \input if
%% they are in the same directory as the main LaTeX file. For loading figures
%% from other directories you can use the `import` package
%%   \usepackage{import}
%% and then include the figures with
%%   \import{<path to file>}{<filename>.pgf}
%%
%% Matplotlib used the following preamble
%%
\begingroup%
\makeatletter%
\begin{pgfpicture}%
\pgfpathrectangle{\pgfpointorigin}{\pgfqpoint{4.553420in}{2.510802in}}%
\pgfusepath{use as bounding box, clip}%
\begin{pgfscope}%
\pgfsetbuttcap%
\pgfsetmiterjoin%
\definecolor{currentfill}{rgb}{1.000000,1.000000,1.000000}%
\pgfsetfillcolor{currentfill}%
\pgfsetlinewidth{0.000000pt}%
\definecolor{currentstroke}{rgb}{1.000000,1.000000,1.000000}%
\pgfsetstrokecolor{currentstroke}%
\pgfsetdash{}{0pt}%
\pgfpathmoveto{\pgfqpoint{0.000000in}{0.000000in}}%
\pgfpathlineto{\pgfqpoint{4.553420in}{0.000000in}}%
\pgfpathlineto{\pgfqpoint{4.553420in}{2.510802in}}%
\pgfpathlineto{\pgfqpoint{0.000000in}{2.510802in}}%
\pgfpathclose%
\pgfusepath{fill}%
\end{pgfscope}%
\begin{pgfscope}%
\pgfsetbuttcap%
\pgfsetmiterjoin%
\definecolor{currentfill}{rgb}{1.000000,1.000000,1.000000}%
\pgfsetfillcolor{currentfill}%
\pgfsetlinewidth{0.000000pt}%
\definecolor{currentstroke}{rgb}{0.000000,0.000000,0.000000}%
\pgfsetstrokecolor{currentstroke}%
\pgfsetstrokeopacity{0.000000}%
\pgfsetdash{}{0pt}%
\pgfpathmoveto{\pgfqpoint{0.482717in}{0.485802in}}%
\pgfpathlineto{\pgfqpoint{4.357717in}{0.485802in}}%
\pgfpathlineto{\pgfqpoint{4.357717in}{2.410802in}}%
\pgfpathlineto{\pgfqpoint{0.482717in}{2.410802in}}%
\pgfpathclose%
\pgfusepath{fill}%
\end{pgfscope}%
\begin{pgfscope}%
\pgfpathrectangle{\pgfqpoint{0.482717in}{0.485802in}}{\pgfqpoint{3.875000in}{1.925000in}}%
\pgfusepath{clip}%
\pgfsetrectcap%
\pgfsetroundjoin%
\pgfsetlinewidth{0.301125pt}%
\definecolor{currentstroke}{rgb}{0.000000,0.000000,0.000000}%
\pgfsetstrokecolor{currentstroke}%
\pgfsetdash{}{0pt}%
\pgfpathmoveto{\pgfqpoint{0.658853in}{0.573302in}}%
\pgfpathlineto{\pgfqpoint{0.658853in}{0.665407in}}%
\pgfpathlineto{\pgfqpoint{0.658853in}{0.757513in}}%
\pgfpathlineto{\pgfqpoint{0.658853in}{0.849618in}}%
\pgfpathlineto{\pgfqpoint{0.658853in}{0.941723in}}%
\pgfpathlineto{\pgfqpoint{0.658853in}{1.033829in}}%
\pgfpathlineto{\pgfqpoint{0.658853in}{1.125934in}}%
\pgfpathlineto{\pgfqpoint{0.658853in}{1.218039in}}%
\pgfpathlineto{\pgfqpoint{0.658853in}{1.310144in}}%
\pgfpathlineto{\pgfqpoint{0.658853in}{1.402250in}}%
\pgfpathlineto{\pgfqpoint{0.658853in}{1.494355in}}%
\pgfpathlineto{\pgfqpoint{0.658853in}{1.586460in}}%
\pgfpathlineto{\pgfqpoint{0.658853in}{1.678565in}}%
\pgfpathlineto{\pgfqpoint{0.658853in}{1.770671in}}%
\pgfpathlineto{\pgfqpoint{0.658853in}{1.862776in}}%
\pgfpathlineto{\pgfqpoint{0.658853in}{1.954881in}}%
\pgfpathlineto{\pgfqpoint{0.658853in}{2.046986in}}%
\pgfpathlineto{\pgfqpoint{0.658853in}{2.139092in}}%
\pgfpathlineto{\pgfqpoint{0.658853in}{2.231197in}}%
\pgfpathlineto{\pgfqpoint{0.658853in}{2.323302in}}%
\pgfusepath{stroke}%
\end{pgfscope}%
\begin{pgfscope}%
\pgfpathrectangle{\pgfqpoint{0.482717in}{0.485802in}}{\pgfqpoint{3.875000in}{1.925000in}}%
\pgfusepath{clip}%
\pgfsetrectcap%
\pgfsetroundjoin%
\pgfsetlinewidth{0.301125pt}%
\definecolor{currentstroke}{rgb}{0.000000,0.000000,0.000000}%
\pgfsetstrokecolor{currentstroke}%
\pgfsetdash{}{0pt}%
\pgfpathmoveto{\pgfqpoint{0.658853in}{1.494355in}}%
\pgfpathlineto{\pgfqpoint{0.690907in}{1.494355in}}%
\pgfpathlineto{\pgfqpoint{0.722961in}{1.494355in}}%
\pgfpathlineto{\pgfqpoint{0.755015in}{1.494355in}}%
\pgfpathlineto{\pgfqpoint{0.787069in}{1.494355in}}%
\pgfpathlineto{\pgfqpoint{0.819123in}{1.494355in}}%
\pgfpathlineto{\pgfqpoint{0.851177in}{1.494355in}}%
\pgfpathlineto{\pgfqpoint{0.883231in}{1.494355in}}%
\pgfpathlineto{\pgfqpoint{0.915285in}{1.494355in}}%
\pgfpathlineto{\pgfqpoint{0.947339in}{1.494355in}}%
\pgfpathlineto{\pgfqpoint{0.979393in}{1.494355in}}%
\pgfpathlineto{\pgfqpoint{1.011447in}{1.494355in}}%
\pgfpathlineto{\pgfqpoint{1.043501in}{1.494355in}}%
\pgfpathlineto{\pgfqpoint{1.075554in}{1.494355in}}%
\pgfpathlineto{\pgfqpoint{1.107608in}{1.494355in}}%
\pgfpathlineto{\pgfqpoint{1.139662in}{1.494355in}}%
\pgfpathlineto{\pgfqpoint{1.171716in}{1.494355in}}%
\pgfpathlineto{\pgfqpoint{1.203770in}{1.494355in}}%
\pgfpathlineto{\pgfqpoint{1.235824in}{1.494355in}}%
\pgfpathlineto{\pgfqpoint{1.267878in}{1.494355in}}%
\pgfpathlineto{\pgfqpoint{1.299932in}{1.494355in}}%
\pgfpathlineto{\pgfqpoint{1.331986in}{1.494355in}}%
\pgfpathlineto{\pgfqpoint{1.364040in}{1.494355in}}%
\pgfpathlineto{\pgfqpoint{1.396094in}{1.494355in}}%
\pgfpathlineto{\pgfqpoint{1.428148in}{1.494355in}}%
\pgfpathlineto{\pgfqpoint{1.460202in}{1.494355in}}%
\pgfpathlineto{\pgfqpoint{1.492256in}{1.494355in}}%
\pgfpathlineto{\pgfqpoint{1.524310in}{1.494355in}}%
\pgfpathlineto{\pgfqpoint{1.556363in}{1.494355in}}%
\pgfpathlineto{\pgfqpoint{1.588417in}{1.494355in}}%
\pgfpathlineto{\pgfqpoint{1.620471in}{1.494355in}}%
\pgfpathlineto{\pgfqpoint{1.652525in}{1.494355in}}%
\pgfpathlineto{\pgfqpoint{1.684579in}{1.494355in}}%
\pgfpathlineto{\pgfqpoint{1.716633in}{1.494355in}}%
\pgfpathlineto{\pgfqpoint{1.748687in}{1.494355in}}%
\pgfpathlineto{\pgfqpoint{1.780741in}{1.494355in}}%
\pgfpathlineto{\pgfqpoint{1.812795in}{1.494355in}}%
\pgfpathlineto{\pgfqpoint{1.844849in}{1.494355in}}%
\pgfpathlineto{\pgfqpoint{1.876903in}{1.494355in}}%
\pgfpathlineto{\pgfqpoint{1.908957in}{1.494355in}}%
\pgfpathlineto{\pgfqpoint{1.941011in}{1.494355in}}%
\pgfpathlineto{\pgfqpoint{1.973065in}{1.494355in}}%
\pgfpathlineto{\pgfqpoint{2.005119in}{1.494355in}}%
\pgfpathlineto{\pgfqpoint{2.037172in}{1.494355in}}%
\pgfpathlineto{\pgfqpoint{2.069226in}{1.494355in}}%
\pgfpathlineto{\pgfqpoint{2.101280in}{1.494355in}}%
\pgfpathlineto{\pgfqpoint{2.133334in}{1.494355in}}%
\pgfpathlineto{\pgfqpoint{2.165388in}{1.494355in}}%
\pgfpathlineto{\pgfqpoint{2.197442in}{1.494355in}}%
\pgfpathlineto{\pgfqpoint{2.229496in}{1.494355in}}%
\pgfpathlineto{\pgfqpoint{2.261550in}{1.494355in}}%
\pgfpathlineto{\pgfqpoint{2.293604in}{1.494355in}}%
\pgfpathlineto{\pgfqpoint{2.325658in}{1.494355in}}%
\pgfpathlineto{\pgfqpoint{2.357712in}{1.494355in}}%
\pgfpathlineto{\pgfqpoint{2.389766in}{1.494355in}}%
\pgfpathlineto{\pgfqpoint{2.421820in}{1.494355in}}%
\pgfpathlineto{\pgfqpoint{2.453874in}{1.494355in}}%
\pgfpathlineto{\pgfqpoint{2.485928in}{1.494355in}}%
\pgfpathlineto{\pgfqpoint{2.517981in}{1.494355in}}%
\pgfpathlineto{\pgfqpoint{2.550035in}{1.494355in}}%
\pgfpathlineto{\pgfqpoint{2.582089in}{1.494355in}}%
\pgfpathlineto{\pgfqpoint{2.614143in}{1.494355in}}%
\pgfpathlineto{\pgfqpoint{2.646197in}{1.494355in}}%
\pgfpathlineto{\pgfqpoint{2.678251in}{1.494355in}}%
\pgfpathlineto{\pgfqpoint{2.710305in}{1.494355in}}%
\pgfpathlineto{\pgfqpoint{2.742359in}{1.494355in}}%
\pgfpathlineto{\pgfqpoint{2.774413in}{1.494355in}}%
\pgfpathlineto{\pgfqpoint{2.806467in}{1.494355in}}%
\pgfpathlineto{\pgfqpoint{2.838521in}{1.494355in}}%
\pgfpathlineto{\pgfqpoint{2.870575in}{1.494355in}}%
\pgfpathlineto{\pgfqpoint{2.902629in}{1.494355in}}%
\pgfpathlineto{\pgfqpoint{2.934683in}{1.494355in}}%
\pgfpathlineto{\pgfqpoint{2.966737in}{1.494355in}}%
\pgfpathlineto{\pgfqpoint{2.998790in}{1.494355in}}%
\pgfpathlineto{\pgfqpoint{3.030844in}{1.494355in}}%
\pgfpathlineto{\pgfqpoint{3.062898in}{1.494355in}}%
\pgfpathlineto{\pgfqpoint{3.094952in}{1.494355in}}%
\pgfpathlineto{\pgfqpoint{3.127006in}{1.494355in}}%
\pgfpathlineto{\pgfqpoint{3.159060in}{1.494355in}}%
\pgfpathlineto{\pgfqpoint{3.191114in}{1.494355in}}%
\pgfpathlineto{\pgfqpoint{3.223168in}{1.494355in}}%
\pgfpathlineto{\pgfqpoint{3.255222in}{1.494355in}}%
\pgfpathlineto{\pgfqpoint{3.287276in}{1.494355in}}%
\pgfpathlineto{\pgfqpoint{3.319330in}{1.494355in}}%
\pgfpathlineto{\pgfqpoint{3.351384in}{1.494355in}}%
\pgfpathlineto{\pgfqpoint{3.383438in}{1.494355in}}%
\pgfpathlineto{\pgfqpoint{3.415492in}{1.494355in}}%
\pgfpathlineto{\pgfqpoint{3.447546in}{1.494355in}}%
\pgfpathlineto{\pgfqpoint{3.479599in}{1.494355in}}%
\pgfpathlineto{\pgfqpoint{3.511653in}{1.494355in}}%
\pgfpathlineto{\pgfqpoint{3.543707in}{1.494355in}}%
\pgfpathlineto{\pgfqpoint{3.575761in}{1.494355in}}%
\pgfpathlineto{\pgfqpoint{3.607815in}{1.494355in}}%
\pgfpathlineto{\pgfqpoint{3.639869in}{1.494355in}}%
\pgfpathlineto{\pgfqpoint{3.671923in}{1.494355in}}%
\pgfpathlineto{\pgfqpoint{3.703977in}{1.494355in}}%
\pgfpathlineto{\pgfqpoint{3.736031in}{1.494355in}}%
\pgfpathlineto{\pgfqpoint{3.768085in}{1.494355in}}%
\pgfpathlineto{\pgfqpoint{3.800139in}{1.494355in}}%
\pgfpathlineto{\pgfqpoint{3.832193in}{1.494355in}}%
\pgfpathlineto{\pgfqpoint{3.864247in}{1.494355in}}%
\pgfpathlineto{\pgfqpoint{3.896301in}{1.494355in}}%
\pgfpathlineto{\pgfqpoint{3.928355in}{1.494355in}}%
\pgfpathlineto{\pgfqpoint{3.960408in}{1.494355in}}%
\pgfpathlineto{\pgfqpoint{3.992462in}{1.494355in}}%
\pgfpathlineto{\pgfqpoint{4.024516in}{1.494355in}}%
\pgfpathlineto{\pgfqpoint{4.056570in}{1.494355in}}%
\pgfpathlineto{\pgfqpoint{4.088624in}{1.494355in}}%
\pgfpathlineto{\pgfqpoint{4.120678in}{1.494355in}}%
\pgfpathlineto{\pgfqpoint{4.152732in}{1.494355in}}%
\pgfusepath{stroke}%
\end{pgfscope}%
\begin{pgfscope}%
\pgfsetbuttcap%
\pgfsetroundjoin%
\definecolor{currentfill}{rgb}{0.000000,0.000000,0.000000}%
\pgfsetfillcolor{currentfill}%
\pgfsetlinewidth{0.803000pt}%
\definecolor{currentstroke}{rgb}{0.000000,0.000000,0.000000}%
\pgfsetstrokecolor{currentstroke}%
\pgfsetdash{}{0pt}%
\pgfsys@defobject{currentmarker}{\pgfqpoint{0.000000in}{-0.048611in}}{\pgfqpoint{0.000000in}{0.000000in}}{%
\pgfpathmoveto{\pgfqpoint{0.000000in}{0.000000in}}%
\pgfpathlineto{\pgfqpoint{0.000000in}{-0.048611in}}%
\pgfusepath{stroke,fill}%
}%
\begin{pgfscope}%
\pgfsys@transformshift{0.985680in}{0.485802in}%
\pgfsys@useobject{currentmarker}{}%
\end{pgfscope}%
\end{pgfscope}%
\begin{pgfscope}%
\definecolor{textcolor}{rgb}{0.000000,0.000000,0.000000}%
\pgfsetstrokecolor{textcolor}%
\pgfsetfillcolor{textcolor}%
\pgftext[x=0.985680in,y=0.388580in,,top]{\color{textcolor}\rmfamily\fontsize{10.000000}{12.000000}\selectfont \(\displaystyle 0.000080\)}%
\end{pgfscope}%
\begin{pgfscope}%
\pgfsetbuttcap%
\pgfsetroundjoin%
\definecolor{currentfill}{rgb}{0.000000,0.000000,0.000000}%
\pgfsetfillcolor{currentfill}%
\pgfsetlinewidth{0.803000pt}%
\definecolor{currentstroke}{rgb}{0.000000,0.000000,0.000000}%
\pgfsetstrokecolor{currentstroke}%
\pgfsetdash{}{0pt}%
\pgfsys@defobject{currentmarker}{\pgfqpoint{0.000000in}{-0.048611in}}{\pgfqpoint{0.000000in}{0.000000in}}{%
\pgfpathmoveto{\pgfqpoint{0.000000in}{0.000000in}}%
\pgfpathlineto{\pgfqpoint{0.000000in}{-0.048611in}}%
\pgfusepath{stroke,fill}%
}%
\begin{pgfscope}%
\pgfsys@transformshift{1.626759in}{0.485802in}%
\pgfsys@useobject{currentmarker}{}%
\end{pgfscope}%
\end{pgfscope}%
\begin{pgfscope}%
\definecolor{textcolor}{rgb}{0.000000,0.000000,0.000000}%
\pgfsetstrokecolor{textcolor}%
\pgfsetfillcolor{textcolor}%
\pgftext[x=1.626759in,y=0.388580in,,top]{\color{textcolor}\rmfamily\fontsize{10.000000}{12.000000}\selectfont \(\displaystyle 0.000082\)}%
\end{pgfscope}%
\begin{pgfscope}%
\pgfsetbuttcap%
\pgfsetroundjoin%
\definecolor{currentfill}{rgb}{0.000000,0.000000,0.000000}%
\pgfsetfillcolor{currentfill}%
\pgfsetlinewidth{0.803000pt}%
\definecolor{currentstroke}{rgb}{0.000000,0.000000,0.000000}%
\pgfsetstrokecolor{currentstroke}%
\pgfsetdash{}{0pt}%
\pgfsys@defobject{currentmarker}{\pgfqpoint{0.000000in}{-0.048611in}}{\pgfqpoint{0.000000in}{0.000000in}}{%
\pgfpathmoveto{\pgfqpoint{0.000000in}{0.000000in}}%
\pgfpathlineto{\pgfqpoint{0.000000in}{-0.048611in}}%
\pgfusepath{stroke,fill}%
}%
\begin{pgfscope}%
\pgfsys@transformshift{2.267837in}{0.485802in}%
\pgfsys@useobject{currentmarker}{}%
\end{pgfscope}%
\end{pgfscope}%
\begin{pgfscope}%
\definecolor{textcolor}{rgb}{0.000000,0.000000,0.000000}%
\pgfsetstrokecolor{textcolor}%
\pgfsetfillcolor{textcolor}%
\pgftext[x=2.267837in,y=0.388580in,,top]{\color{textcolor}\rmfamily\fontsize{10.000000}{12.000000}\selectfont \(\displaystyle 0.000084\)}%
\end{pgfscope}%
\begin{pgfscope}%
\pgfsetbuttcap%
\pgfsetroundjoin%
\definecolor{currentfill}{rgb}{0.000000,0.000000,0.000000}%
\pgfsetfillcolor{currentfill}%
\pgfsetlinewidth{0.803000pt}%
\definecolor{currentstroke}{rgb}{0.000000,0.000000,0.000000}%
\pgfsetstrokecolor{currentstroke}%
\pgfsetdash{}{0pt}%
\pgfsys@defobject{currentmarker}{\pgfqpoint{0.000000in}{-0.048611in}}{\pgfqpoint{0.000000in}{0.000000in}}{%
\pgfpathmoveto{\pgfqpoint{0.000000in}{0.000000in}}%
\pgfpathlineto{\pgfqpoint{0.000000in}{-0.048611in}}%
\pgfusepath{stroke,fill}%
}%
\begin{pgfscope}%
\pgfsys@transformshift{2.908916in}{0.485802in}%
\pgfsys@useobject{currentmarker}{}%
\end{pgfscope}%
\end{pgfscope}%
\begin{pgfscope}%
\definecolor{textcolor}{rgb}{0.000000,0.000000,0.000000}%
\pgfsetstrokecolor{textcolor}%
\pgfsetfillcolor{textcolor}%
\pgftext[x=2.908916in,y=0.388580in,,top]{\color{textcolor}\rmfamily\fontsize{10.000000}{12.000000}\selectfont \(\displaystyle 0.000086\)}%
\end{pgfscope}%
\begin{pgfscope}%
\pgfsetbuttcap%
\pgfsetroundjoin%
\definecolor{currentfill}{rgb}{0.000000,0.000000,0.000000}%
\pgfsetfillcolor{currentfill}%
\pgfsetlinewidth{0.803000pt}%
\definecolor{currentstroke}{rgb}{0.000000,0.000000,0.000000}%
\pgfsetstrokecolor{currentstroke}%
\pgfsetdash{}{0pt}%
\pgfsys@defobject{currentmarker}{\pgfqpoint{0.000000in}{-0.048611in}}{\pgfqpoint{0.000000in}{0.000000in}}{%
\pgfpathmoveto{\pgfqpoint{0.000000in}{0.000000in}}%
\pgfpathlineto{\pgfqpoint{0.000000in}{-0.048611in}}%
\pgfusepath{stroke,fill}%
}%
\begin{pgfscope}%
\pgfsys@transformshift{3.549995in}{0.485802in}%
\pgfsys@useobject{currentmarker}{}%
\end{pgfscope}%
\end{pgfscope}%
\begin{pgfscope}%
\definecolor{textcolor}{rgb}{0.000000,0.000000,0.000000}%
\pgfsetstrokecolor{textcolor}%
\pgfsetfillcolor{textcolor}%
\pgftext[x=3.549995in,y=0.388580in,,top]{\color{textcolor}\rmfamily\fontsize{10.000000}{12.000000}\selectfont \(\displaystyle 0.000088\)}%
\end{pgfscope}%
\begin{pgfscope}%
\pgfsetbuttcap%
\pgfsetroundjoin%
\definecolor{currentfill}{rgb}{0.000000,0.000000,0.000000}%
\pgfsetfillcolor{currentfill}%
\pgfsetlinewidth{0.803000pt}%
\definecolor{currentstroke}{rgb}{0.000000,0.000000,0.000000}%
\pgfsetstrokecolor{currentstroke}%
\pgfsetdash{}{0pt}%
\pgfsys@defobject{currentmarker}{\pgfqpoint{0.000000in}{-0.048611in}}{\pgfqpoint{0.000000in}{0.000000in}}{%
\pgfpathmoveto{\pgfqpoint{0.000000in}{0.000000in}}%
\pgfpathlineto{\pgfqpoint{0.000000in}{-0.048611in}}%
\pgfusepath{stroke,fill}%
}%
\begin{pgfscope}%
\pgfsys@transformshift{4.191073in}{0.485802in}%
\pgfsys@useobject{currentmarker}{}%
\end{pgfscope}%
\end{pgfscope}%
\begin{pgfscope}%
\definecolor{textcolor}{rgb}{0.000000,0.000000,0.000000}%
\pgfsetstrokecolor{textcolor}%
\pgfsetfillcolor{textcolor}%
\pgftext[x=4.191073in,y=0.388580in,,top]{\color{textcolor}\rmfamily\fontsize{10.000000}{12.000000}\selectfont \(\displaystyle 0.000090\)}%
\end{pgfscope}%
\begin{pgfscope}%
\definecolor{textcolor}{rgb}{0.000000,0.000000,0.000000}%
\pgfsetstrokecolor{textcolor}%
\pgfsetfillcolor{textcolor}%
\pgftext[x=4.357717in,y=0.223457in,right,top]{\color{textcolor}\rmfamily\fontsize{10.000000}{12.000000}\selectfont \(\displaystyle +4.7123\)}%
\end{pgfscope}%
\begin{pgfscope}%
\pgfsetbuttcap%
\pgfsetroundjoin%
\definecolor{currentfill}{rgb}{0.000000,0.000000,0.000000}%
\pgfsetfillcolor{currentfill}%
\pgfsetlinewidth{0.803000pt}%
\definecolor{currentstroke}{rgb}{0.000000,0.000000,0.000000}%
\pgfsetstrokecolor{currentstroke}%
\pgfsetdash{}{0pt}%
\pgfsys@defobject{currentmarker}{\pgfqpoint{-0.048611in}{0.000000in}}{\pgfqpoint{0.000000in}{0.000000in}}{%
\pgfpathmoveto{\pgfqpoint{0.000000in}{0.000000in}}%
\pgfpathlineto{\pgfqpoint{-0.048611in}{0.000000in}}%
\pgfusepath{stroke,fill}%
}%
\begin{pgfscope}%
\pgfsys@transformshift{0.482717in}{0.573302in}%
\pgfsys@useobject{currentmarker}{}%
\end{pgfscope}%
\end{pgfscope}%
\begin{pgfscope}%
\definecolor{textcolor}{rgb}{0.000000,0.000000,0.000000}%
\pgfsetstrokecolor{textcolor}%
\pgfsetfillcolor{textcolor}%
\pgftext[x=0.100000in,y=0.525077in,left,base]{\color{textcolor}\rmfamily\fontsize{10.000000}{12.000000}\selectfont \(\displaystyle -1.0\)}%
\end{pgfscope}%
\begin{pgfscope}%
\pgfsetbuttcap%
\pgfsetroundjoin%
\definecolor{currentfill}{rgb}{0.000000,0.000000,0.000000}%
\pgfsetfillcolor{currentfill}%
\pgfsetlinewidth{0.803000pt}%
\definecolor{currentstroke}{rgb}{0.000000,0.000000,0.000000}%
\pgfsetstrokecolor{currentstroke}%
\pgfsetdash{}{0pt}%
\pgfsys@defobject{currentmarker}{\pgfqpoint{-0.048611in}{0.000000in}}{\pgfqpoint{0.000000in}{0.000000in}}{%
\pgfpathmoveto{\pgfqpoint{0.000000in}{0.000000in}}%
\pgfpathlineto{\pgfqpoint{-0.048611in}{0.000000in}}%
\pgfusepath{stroke,fill}%
}%
\begin{pgfscope}%
\pgfsys@transformshift{0.482717in}{1.033829in}%
\pgfsys@useobject{currentmarker}{}%
\end{pgfscope}%
\end{pgfscope}%
\begin{pgfscope}%
\definecolor{textcolor}{rgb}{0.000000,0.000000,0.000000}%
\pgfsetstrokecolor{textcolor}%
\pgfsetfillcolor{textcolor}%
\pgftext[x=0.100000in,y=0.985603in,left,base]{\color{textcolor}\rmfamily\fontsize{10.000000}{12.000000}\selectfont \(\displaystyle -0.5\)}%
\end{pgfscope}%
\begin{pgfscope}%
\pgfsetbuttcap%
\pgfsetroundjoin%
\definecolor{currentfill}{rgb}{0.000000,0.000000,0.000000}%
\pgfsetfillcolor{currentfill}%
\pgfsetlinewidth{0.803000pt}%
\definecolor{currentstroke}{rgb}{0.000000,0.000000,0.000000}%
\pgfsetstrokecolor{currentstroke}%
\pgfsetdash{}{0pt}%
\pgfsys@defobject{currentmarker}{\pgfqpoint{-0.048611in}{0.000000in}}{\pgfqpoint{0.000000in}{0.000000in}}{%
\pgfpathmoveto{\pgfqpoint{0.000000in}{0.000000in}}%
\pgfpathlineto{\pgfqpoint{-0.048611in}{0.000000in}}%
\pgfusepath{stroke,fill}%
}%
\begin{pgfscope}%
\pgfsys@transformshift{0.482717in}{1.494355in}%
\pgfsys@useobject{currentmarker}{}%
\end{pgfscope}%
\end{pgfscope}%
\begin{pgfscope}%
\definecolor{textcolor}{rgb}{0.000000,0.000000,0.000000}%
\pgfsetstrokecolor{textcolor}%
\pgfsetfillcolor{textcolor}%
\pgftext[x=0.208025in,y=1.446130in,left,base]{\color{textcolor}\rmfamily\fontsize{10.000000}{12.000000}\selectfont \(\displaystyle 0.0\)}%
\end{pgfscope}%
\begin{pgfscope}%
\pgfsetbuttcap%
\pgfsetroundjoin%
\definecolor{currentfill}{rgb}{0.000000,0.000000,0.000000}%
\pgfsetfillcolor{currentfill}%
\pgfsetlinewidth{0.803000pt}%
\definecolor{currentstroke}{rgb}{0.000000,0.000000,0.000000}%
\pgfsetstrokecolor{currentstroke}%
\pgfsetdash{}{0pt}%
\pgfsys@defobject{currentmarker}{\pgfqpoint{-0.048611in}{0.000000in}}{\pgfqpoint{0.000000in}{0.000000in}}{%
\pgfpathmoveto{\pgfqpoint{0.000000in}{0.000000in}}%
\pgfpathlineto{\pgfqpoint{-0.048611in}{0.000000in}}%
\pgfusepath{stroke,fill}%
}%
\begin{pgfscope}%
\pgfsys@transformshift{0.482717in}{1.954881in}%
\pgfsys@useobject{currentmarker}{}%
\end{pgfscope}%
\end{pgfscope}%
\begin{pgfscope}%
\definecolor{textcolor}{rgb}{0.000000,0.000000,0.000000}%
\pgfsetstrokecolor{textcolor}%
\pgfsetfillcolor{textcolor}%
\pgftext[x=0.208025in,y=1.906656in,left,base]{\color{textcolor}\rmfamily\fontsize{10.000000}{12.000000}\selectfont \(\displaystyle 0.5\)}%
\end{pgfscope}%
\begin{pgfscope}%
\pgfpathrectangle{\pgfqpoint{0.482717in}{0.485802in}}{\pgfqpoint{3.875000in}{1.925000in}}%
\pgfusepath{clip}%
\pgfsetrectcap%
\pgfsetroundjoin%
\pgfsetlinewidth{1.505625pt}%
\definecolor{currentstroke}{rgb}{0.117647,0.564706,1.000000}%
\pgfsetstrokecolor{currentstroke}%
\pgfsetdash{}{0pt}%
\pgfpathmoveto{\pgfqpoint{0.658853in}{1.494349in}}%
\pgfpathlineto{\pgfqpoint{4.181581in}{1.494355in}}%
\pgfpathlineto{\pgfqpoint{4.181581in}{1.494355in}}%
\pgfusepath{stroke}%
\end{pgfscope}%
\begin{pgfscope}%
\pgfpathrectangle{\pgfqpoint{0.482717in}{0.485802in}}{\pgfqpoint{3.875000in}{1.925000in}}%
\pgfusepath{clip}%
\pgfsetrectcap%
\pgfsetroundjoin%
\pgfsetlinewidth{1.505625pt}%
\definecolor{currentstroke}{rgb}{1.000000,0.388235,0.278431}%
\pgfsetstrokecolor{currentstroke}%
\pgfsetdash{}{0pt}%
\pgfpathmoveto{\pgfqpoint{0.658853in}{1.494352in}}%
\pgfpathlineto{\pgfqpoint{4.181581in}{1.494359in}}%
\pgfpathlineto{\pgfqpoint{4.181581in}{1.494359in}}%
\pgfusepath{stroke}%
\end{pgfscope}%
\begin{pgfscope}%
\pgfsetrectcap%
\pgfsetmiterjoin%
\pgfsetlinewidth{0.803000pt}%
\definecolor{currentstroke}{rgb}{0.000000,0.000000,0.000000}%
\pgfsetstrokecolor{currentstroke}%
\pgfsetdash{}{0pt}%
\pgfpathmoveto{\pgfqpoint{0.482717in}{0.485802in}}%
\pgfpathlineto{\pgfqpoint{0.482717in}{2.410802in}}%
\pgfusepath{stroke}%
\end{pgfscope}%
\begin{pgfscope}%
\pgfsetrectcap%
\pgfsetmiterjoin%
\pgfsetlinewidth{0.803000pt}%
\definecolor{currentstroke}{rgb}{0.000000,0.000000,0.000000}%
\pgfsetstrokecolor{currentstroke}%
\pgfsetdash{}{0pt}%
\pgfpathmoveto{\pgfqpoint{4.357717in}{0.485802in}}%
\pgfpathlineto{\pgfqpoint{4.357717in}{2.410802in}}%
\pgfusepath{stroke}%
\end{pgfscope}%
\begin{pgfscope}%
\pgfsetrectcap%
\pgfsetmiterjoin%
\pgfsetlinewidth{0.803000pt}%
\definecolor{currentstroke}{rgb}{0.000000,0.000000,0.000000}%
\pgfsetstrokecolor{currentstroke}%
\pgfsetdash{}{0pt}%
\pgfpathmoveto{\pgfqpoint{0.482717in}{0.485802in}}%
\pgfpathlineto{\pgfqpoint{4.357717in}{0.485802in}}%
\pgfusepath{stroke}%
\end{pgfscope}%
\begin{pgfscope}%
\pgfsetrectcap%
\pgfsetmiterjoin%
\pgfsetlinewidth{0.803000pt}%
\definecolor{currentstroke}{rgb}{0.000000,0.000000,0.000000}%
\pgfsetstrokecolor{currentstroke}%
\pgfsetdash{}{0pt}%
\pgfpathmoveto{\pgfqpoint{0.482717in}{2.410802in}}%
\pgfpathlineto{\pgfqpoint{4.357717in}{2.410802in}}%
\pgfusepath{stroke}%
\end{pgfscope}%
\begin{pgfscope}%
\pgfpathrectangle{\pgfqpoint{0.482717in}{0.485802in}}{\pgfqpoint{3.875000in}{1.925000in}}%
\pgfusepath{clip}%
\pgfsetrectcap%
\pgfsetroundjoin%
\pgfsetlinewidth{1.505625pt}%
\definecolor{currentstroke}{rgb}{0.529412,0.807843,0.921569}%
\pgfsetstrokecolor{currentstroke}%
\pgfsetdash{}{0pt}%
\pgfpathmoveto{\pgfqpoint{3.864247in}{0.711460in}}%
\pgfpathlineto{\pgfqpoint{3.864247in}{0.803565in}}%
\pgfpathlineto{\pgfqpoint{3.864247in}{0.895671in}}%
\pgfpathlineto{\pgfqpoint{3.864247in}{0.987776in}}%
\pgfpathlineto{\pgfqpoint{3.864247in}{1.079881in}}%
\pgfpathlineto{\pgfqpoint{3.864247in}{1.171986in}}%
\pgfpathlineto{\pgfqpoint{3.864247in}{1.264092in}}%
\pgfpathlineto{\pgfqpoint{3.864247in}{1.356197in}}%
\pgfpathlineto{\pgfqpoint{3.864247in}{1.448302in}}%
\pgfpathlineto{\pgfqpoint{3.864247in}{1.540407in}}%
\pgfpathlineto{\pgfqpoint{3.864247in}{1.632513in}}%
\pgfpathlineto{\pgfqpoint{3.864247in}{1.724618in}}%
\pgfpathlineto{\pgfqpoint{3.864247in}{1.816723in}}%
\pgfpathlineto{\pgfqpoint{3.864247in}{1.908829in}}%
\pgfpathlineto{\pgfqpoint{3.864247in}{2.000934in}}%
\pgfpathlineto{\pgfqpoint{3.864247in}{2.093039in}}%
\pgfpathlineto{\pgfqpoint{3.864247in}{2.185144in}}%
\pgfusepath{stroke}%
\end{pgfscope}%
\begin{pgfscope}%
\pgfpathrectangle{\pgfqpoint{0.482717in}{0.485802in}}{\pgfqpoint{3.875000in}{1.925000in}}%
\pgfusepath{clip}%
\pgfsetrectcap%
\pgfsetroundjoin%
\pgfsetlinewidth{1.505625pt}%
\definecolor{currentstroke}{rgb}{1.000000,0.854902,0.725490}%
\pgfsetstrokecolor{currentstroke}%
\pgfsetdash{}{0pt}%
\pgfpathmoveto{\pgfqpoint{1.941011in}{0.711460in}}%
\pgfpathlineto{\pgfqpoint{1.941011in}{0.803565in}}%
\pgfpathlineto{\pgfqpoint{1.941011in}{0.895671in}}%
\pgfpathlineto{\pgfqpoint{1.941011in}{0.987776in}}%
\pgfpathlineto{\pgfqpoint{1.941011in}{1.079881in}}%
\pgfpathlineto{\pgfqpoint{1.941011in}{1.171986in}}%
\pgfpathlineto{\pgfqpoint{1.941011in}{1.264092in}}%
\pgfpathlineto{\pgfqpoint{1.941011in}{1.356197in}}%
\pgfpathlineto{\pgfqpoint{1.941011in}{1.448302in}}%
\pgfpathlineto{\pgfqpoint{1.941011in}{1.540407in}}%
\pgfpathlineto{\pgfqpoint{1.941011in}{1.632513in}}%
\pgfpathlineto{\pgfqpoint{1.941011in}{1.724618in}}%
\pgfpathlineto{\pgfqpoint{1.941011in}{1.816723in}}%
\pgfpathlineto{\pgfqpoint{1.941011in}{1.908829in}}%
\pgfpathlineto{\pgfqpoint{1.941011in}{2.000934in}}%
\pgfpathlineto{\pgfqpoint{1.941011in}{2.093039in}}%
\pgfpathlineto{\pgfqpoint{1.941011in}{2.185144in}}%
\pgfusepath{stroke}%
\end{pgfscope}%
\begin{pgfscope}%
\pgfpathrectangle{\pgfqpoint{0.482717in}{0.485802in}}{\pgfqpoint{3.875000in}{1.925000in}}%
\pgfusepath{clip}%
\pgfsetbuttcap%
\pgfsetroundjoin%
\definecolor{currentfill}{rgb}{0.117647,0.564706,1.000000}%
\pgfsetfillcolor{currentfill}%
\pgfsetlinewidth{1.003750pt}%
\definecolor{currentstroke}{rgb}{0.117647,0.564706,1.000000}%
\pgfsetstrokecolor{currentstroke}%
\pgfsetdash{}{0pt}%
\pgfpathmoveto{\pgfqpoint{3.864247in}{1.452688in}}%
\pgfpathcurveto{\pgfqpoint{3.875297in}{1.452688in}}{\pgfqpoint{3.885896in}{1.457078in}}{\pgfqpoint{3.893709in}{1.464892in}}%
\pgfpathcurveto{\pgfqpoint{3.901523in}{1.472706in}}{\pgfqpoint{3.905913in}{1.483305in}}{\pgfqpoint{3.905913in}{1.494355in}}%
\pgfpathcurveto{\pgfqpoint{3.905913in}{1.505405in}}{\pgfqpoint{3.901523in}{1.516004in}}{\pgfqpoint{3.893709in}{1.523818in}}%
\pgfpathcurveto{\pgfqpoint{3.885896in}{1.531631in}}{\pgfqpoint{3.875297in}{1.536022in}}{\pgfqpoint{3.864247in}{1.536022in}}%
\pgfpathcurveto{\pgfqpoint{3.853197in}{1.536022in}}{\pgfqpoint{3.842597in}{1.531631in}}{\pgfqpoint{3.834784in}{1.523818in}}%
\pgfpathcurveto{\pgfqpoint{3.826970in}{1.516004in}}{\pgfqpoint{3.822580in}{1.505405in}}{\pgfqpoint{3.822580in}{1.494355in}}%
\pgfpathcurveto{\pgfqpoint{3.822580in}{1.483305in}}{\pgfqpoint{3.826970in}{1.472706in}}{\pgfqpoint{3.834784in}{1.464892in}}%
\pgfpathcurveto{\pgfqpoint{3.842597in}{1.457078in}}{\pgfqpoint{3.853197in}{1.452688in}}{\pgfqpoint{3.864247in}{1.452688in}}%
\pgfpathclose%
\pgfusepath{stroke,fill}%
\end{pgfscope}%
\begin{pgfscope}%
\pgfpathrectangle{\pgfqpoint{0.482717in}{0.485802in}}{\pgfqpoint{3.875000in}{1.925000in}}%
\pgfusepath{clip}%
\pgfsetbuttcap%
\pgfsetroundjoin%
\definecolor{currentfill}{rgb}{1.000000,0.627451,0.478431}%
\pgfsetfillcolor{currentfill}%
\pgfsetlinewidth{1.003750pt}%
\definecolor{currentstroke}{rgb}{1.000000,0.627451,0.478431}%
\pgfsetstrokecolor{currentstroke}%
\pgfsetdash{}{0pt}%
\pgfpathmoveto{\pgfqpoint{1.941011in}{1.452688in}}%
\pgfpathcurveto{\pgfqpoint{1.952061in}{1.452688in}}{\pgfqpoint{1.962660in}{1.457078in}}{\pgfqpoint{1.970473in}{1.464892in}}%
\pgfpathcurveto{\pgfqpoint{1.978287in}{1.472706in}}{\pgfqpoint{1.982677in}{1.483305in}}{\pgfqpoint{1.982677in}{1.494355in}}%
\pgfpathcurveto{\pgfqpoint{1.982677in}{1.505405in}}{\pgfqpoint{1.978287in}{1.516004in}}{\pgfqpoint{1.970473in}{1.523818in}}%
\pgfpathcurveto{\pgfqpoint{1.962660in}{1.531631in}}{\pgfqpoint{1.952061in}{1.536022in}}{\pgfqpoint{1.941011in}{1.536022in}}%
\pgfpathcurveto{\pgfqpoint{1.929961in}{1.536022in}}{\pgfqpoint{1.919361in}{1.531631in}}{\pgfqpoint{1.911548in}{1.523818in}}%
\pgfpathcurveto{\pgfqpoint{1.903734in}{1.516004in}}{\pgfqpoint{1.899344in}{1.505405in}}{\pgfqpoint{1.899344in}{1.494355in}}%
\pgfpathcurveto{\pgfqpoint{1.899344in}{1.483305in}}{\pgfqpoint{1.903734in}{1.472706in}}{\pgfqpoint{1.911548in}{1.464892in}}%
\pgfpathcurveto{\pgfqpoint{1.919361in}{1.457078in}}{\pgfqpoint{1.929961in}{1.452688in}}{\pgfqpoint{1.941011in}{1.452688in}}%
\pgfpathclose%
\pgfusepath{stroke,fill}%
\end{pgfscope}%
\end{pgfpicture}%
\makeatother%
\endgroup%

    %% Creator: Matplotlib, PGF backend
%%
%% To include the figure in your LaTeX document, write
%%   \input{<filename>.pgf}
%%
%% Make sure the required packages are loaded in your preamble
%%   \usepackage{pgf}
%%
%% Figures using additional raster images can only be included by \input if
%% they are in the same directory as the main LaTeX file. For loading figures
%% from other directories you can use the `import` package
%%   \usepackage{import}
%% and then include the figures with
%%   \import{<path to file>}{<filename>.pgf}
%%
%% Matplotlib used the following preamble
%%
\begingroup%
\makeatletter%
\begin{pgfpicture}%
\pgfpathrectangle{\pgfpointorigin}{\pgfqpoint{4.527162in}{2.345679in}}%
\pgfusepath{use as bounding box, clip}%
\begin{pgfscope}%
\pgfsetbuttcap%
\pgfsetmiterjoin%
\definecolor{currentfill}{rgb}{1.000000,1.000000,1.000000}%
\pgfsetfillcolor{currentfill}%
\pgfsetlinewidth{0.000000pt}%
\definecolor{currentstroke}{rgb}{1.000000,1.000000,1.000000}%
\pgfsetstrokecolor{currentstroke}%
\pgfsetdash{}{0pt}%
\pgfpathmoveto{\pgfqpoint{0.000000in}{0.000000in}}%
\pgfpathlineto{\pgfqpoint{4.527162in}{0.000000in}}%
\pgfpathlineto{\pgfqpoint{4.527162in}{2.345679in}}%
\pgfpathlineto{\pgfqpoint{0.000000in}{2.345679in}}%
\pgfpathclose%
\pgfusepath{fill}%
\end{pgfscope}%
\begin{pgfscope}%
\pgfsetbuttcap%
\pgfsetmiterjoin%
\definecolor{currentfill}{rgb}{1.000000,1.000000,1.000000}%
\pgfsetfillcolor{currentfill}%
\pgfsetlinewidth{0.000000pt}%
\definecolor{currentstroke}{rgb}{0.000000,0.000000,0.000000}%
\pgfsetstrokecolor{currentstroke}%
\pgfsetstrokeopacity{0.000000}%
\pgfsetdash{}{0pt}%
\pgfpathmoveto{\pgfqpoint{0.552162in}{0.320679in}}%
\pgfpathlineto{\pgfqpoint{4.427162in}{0.320679in}}%
\pgfpathlineto{\pgfqpoint{4.427162in}{2.245679in}}%
\pgfpathlineto{\pgfqpoint{0.552162in}{2.245679in}}%
\pgfpathclose%
\pgfusepath{fill}%
\end{pgfscope}%
\begin{pgfscope}%
\pgfpathrectangle{\pgfqpoint{0.552162in}{0.320679in}}{\pgfqpoint{3.875000in}{1.925000in}}%
\pgfusepath{clip}%
\pgfsetrectcap%
\pgfsetroundjoin%
\pgfsetlinewidth{0.301125pt}%
\definecolor{currentstroke}{rgb}{0.000000,0.000000,0.000000}%
\pgfsetstrokecolor{currentstroke}%
\pgfsetdash{}{0pt}%
\pgfpathmoveto{\pgfqpoint{0.728298in}{1.313356in}}%
\pgfpathlineto{\pgfqpoint{1.608980in}{1.313356in}}%
\pgfpathlineto{\pgfqpoint{2.489662in}{1.313356in}}%
\pgfpathlineto{\pgfqpoint{3.370343in}{1.313356in}}%
\pgfpathlineto{\pgfqpoint{4.251025in}{1.313356in}}%
\pgfusepath{stroke}%
\end{pgfscope}%
\begin{pgfscope}%
\pgfsetbuttcap%
\pgfsetroundjoin%
\definecolor{currentfill}{rgb}{0.000000,0.000000,0.000000}%
\pgfsetfillcolor{currentfill}%
\pgfsetlinewidth{0.803000pt}%
\definecolor{currentstroke}{rgb}{0.000000,0.000000,0.000000}%
\pgfsetstrokecolor{currentstroke}%
\pgfsetdash{}{0pt}%
\pgfsys@defobject{currentmarker}{\pgfqpoint{0.000000in}{-0.048611in}}{\pgfqpoint{0.000000in}{0.000000in}}{%
\pgfpathmoveto{\pgfqpoint{0.000000in}{0.000000in}}%
\pgfpathlineto{\pgfqpoint{0.000000in}{-0.048611in}}%
\pgfusepath{stroke,fill}%
}%
\begin{pgfscope}%
\pgfsys@transformshift{0.728298in}{0.320679in}%
\pgfsys@useobject{currentmarker}{}%
\end{pgfscope}%
\end{pgfscope}%
\begin{pgfscope}%
\definecolor{textcolor}{rgb}{0.000000,0.000000,0.000000}%
\pgfsetstrokecolor{textcolor}%
\pgfsetfillcolor{textcolor}%
\pgftext[x=0.728298in,y=0.223457in,,top]{\color{textcolor}\rmfamily\fontsize{10.000000}{12.000000}\selectfont \(\displaystyle 4.600\)}%
\end{pgfscope}%
\begin{pgfscope}%
\pgfsetbuttcap%
\pgfsetroundjoin%
\definecolor{currentfill}{rgb}{0.000000,0.000000,0.000000}%
\pgfsetfillcolor{currentfill}%
\pgfsetlinewidth{0.803000pt}%
\definecolor{currentstroke}{rgb}{0.000000,0.000000,0.000000}%
\pgfsetstrokecolor{currentstroke}%
\pgfsetdash{}{0pt}%
\pgfsys@defobject{currentmarker}{\pgfqpoint{0.000000in}{-0.048611in}}{\pgfqpoint{0.000000in}{0.000000in}}{%
\pgfpathmoveto{\pgfqpoint{0.000000in}{0.000000in}}%
\pgfpathlineto{\pgfqpoint{0.000000in}{-0.048611in}}%
\pgfusepath{stroke,fill}%
}%
\begin{pgfscope}%
\pgfsys@transformshift{1.168639in}{0.320679in}%
\pgfsys@useobject{currentmarker}{}%
\end{pgfscope}%
\end{pgfscope}%
\begin{pgfscope}%
\definecolor{textcolor}{rgb}{0.000000,0.000000,0.000000}%
\pgfsetstrokecolor{textcolor}%
\pgfsetfillcolor{textcolor}%
\pgftext[x=1.168639in,y=0.223457in,,top]{\color{textcolor}\rmfamily\fontsize{10.000000}{12.000000}\selectfont \(\displaystyle 4.625\)}%
\end{pgfscope}%
\begin{pgfscope}%
\pgfsetbuttcap%
\pgfsetroundjoin%
\definecolor{currentfill}{rgb}{0.000000,0.000000,0.000000}%
\pgfsetfillcolor{currentfill}%
\pgfsetlinewidth{0.803000pt}%
\definecolor{currentstroke}{rgb}{0.000000,0.000000,0.000000}%
\pgfsetstrokecolor{currentstroke}%
\pgfsetdash{}{0pt}%
\pgfsys@defobject{currentmarker}{\pgfqpoint{0.000000in}{-0.048611in}}{\pgfqpoint{0.000000in}{0.000000in}}{%
\pgfpathmoveto{\pgfqpoint{0.000000in}{0.000000in}}%
\pgfpathlineto{\pgfqpoint{0.000000in}{-0.048611in}}%
\pgfusepath{stroke,fill}%
}%
\begin{pgfscope}%
\pgfsys@transformshift{1.608980in}{0.320679in}%
\pgfsys@useobject{currentmarker}{}%
\end{pgfscope}%
\end{pgfscope}%
\begin{pgfscope}%
\definecolor{textcolor}{rgb}{0.000000,0.000000,0.000000}%
\pgfsetstrokecolor{textcolor}%
\pgfsetfillcolor{textcolor}%
\pgftext[x=1.608980in,y=0.223457in,,top]{\color{textcolor}\rmfamily\fontsize{10.000000}{12.000000}\selectfont \(\displaystyle 4.650\)}%
\end{pgfscope}%
\begin{pgfscope}%
\pgfsetbuttcap%
\pgfsetroundjoin%
\definecolor{currentfill}{rgb}{0.000000,0.000000,0.000000}%
\pgfsetfillcolor{currentfill}%
\pgfsetlinewidth{0.803000pt}%
\definecolor{currentstroke}{rgb}{0.000000,0.000000,0.000000}%
\pgfsetstrokecolor{currentstroke}%
\pgfsetdash{}{0pt}%
\pgfsys@defobject{currentmarker}{\pgfqpoint{0.000000in}{-0.048611in}}{\pgfqpoint{0.000000in}{0.000000in}}{%
\pgfpathmoveto{\pgfqpoint{0.000000in}{0.000000in}}%
\pgfpathlineto{\pgfqpoint{0.000000in}{-0.048611in}}%
\pgfusepath{stroke,fill}%
}%
\begin{pgfscope}%
\pgfsys@transformshift{2.049321in}{0.320679in}%
\pgfsys@useobject{currentmarker}{}%
\end{pgfscope}%
\end{pgfscope}%
\begin{pgfscope}%
\definecolor{textcolor}{rgb}{0.000000,0.000000,0.000000}%
\pgfsetstrokecolor{textcolor}%
\pgfsetfillcolor{textcolor}%
\pgftext[x=2.049321in,y=0.223457in,,top]{\color{textcolor}\rmfamily\fontsize{10.000000}{12.000000}\selectfont \(\displaystyle 4.675\)}%
\end{pgfscope}%
\begin{pgfscope}%
\pgfsetbuttcap%
\pgfsetroundjoin%
\definecolor{currentfill}{rgb}{0.000000,0.000000,0.000000}%
\pgfsetfillcolor{currentfill}%
\pgfsetlinewidth{0.803000pt}%
\definecolor{currentstroke}{rgb}{0.000000,0.000000,0.000000}%
\pgfsetstrokecolor{currentstroke}%
\pgfsetdash{}{0pt}%
\pgfsys@defobject{currentmarker}{\pgfqpoint{0.000000in}{-0.048611in}}{\pgfqpoint{0.000000in}{0.000000in}}{%
\pgfpathmoveto{\pgfqpoint{0.000000in}{0.000000in}}%
\pgfpathlineto{\pgfqpoint{0.000000in}{-0.048611in}}%
\pgfusepath{stroke,fill}%
}%
\begin{pgfscope}%
\pgfsys@transformshift{2.489662in}{0.320679in}%
\pgfsys@useobject{currentmarker}{}%
\end{pgfscope}%
\end{pgfscope}%
\begin{pgfscope}%
\definecolor{textcolor}{rgb}{0.000000,0.000000,0.000000}%
\pgfsetstrokecolor{textcolor}%
\pgfsetfillcolor{textcolor}%
\pgftext[x=2.489662in,y=0.223457in,,top]{\color{textcolor}\rmfamily\fontsize{10.000000}{12.000000}\selectfont \(\displaystyle 4.700\)}%
\end{pgfscope}%
\begin{pgfscope}%
\pgfsetbuttcap%
\pgfsetroundjoin%
\definecolor{currentfill}{rgb}{0.000000,0.000000,0.000000}%
\pgfsetfillcolor{currentfill}%
\pgfsetlinewidth{0.803000pt}%
\definecolor{currentstroke}{rgb}{0.000000,0.000000,0.000000}%
\pgfsetstrokecolor{currentstroke}%
\pgfsetdash{}{0pt}%
\pgfsys@defobject{currentmarker}{\pgfqpoint{0.000000in}{-0.048611in}}{\pgfqpoint{0.000000in}{0.000000in}}{%
\pgfpathmoveto{\pgfqpoint{0.000000in}{0.000000in}}%
\pgfpathlineto{\pgfqpoint{0.000000in}{-0.048611in}}%
\pgfusepath{stroke,fill}%
}%
\begin{pgfscope}%
\pgfsys@transformshift{2.930002in}{0.320679in}%
\pgfsys@useobject{currentmarker}{}%
\end{pgfscope}%
\end{pgfscope}%
\begin{pgfscope}%
\definecolor{textcolor}{rgb}{0.000000,0.000000,0.000000}%
\pgfsetstrokecolor{textcolor}%
\pgfsetfillcolor{textcolor}%
\pgftext[x=2.930002in,y=0.223457in,,top]{\color{textcolor}\rmfamily\fontsize{10.000000}{12.000000}\selectfont \(\displaystyle 4.725\)}%
\end{pgfscope}%
\begin{pgfscope}%
\pgfsetbuttcap%
\pgfsetroundjoin%
\definecolor{currentfill}{rgb}{0.000000,0.000000,0.000000}%
\pgfsetfillcolor{currentfill}%
\pgfsetlinewidth{0.803000pt}%
\definecolor{currentstroke}{rgb}{0.000000,0.000000,0.000000}%
\pgfsetstrokecolor{currentstroke}%
\pgfsetdash{}{0pt}%
\pgfsys@defobject{currentmarker}{\pgfqpoint{0.000000in}{-0.048611in}}{\pgfqpoint{0.000000in}{0.000000in}}{%
\pgfpathmoveto{\pgfqpoint{0.000000in}{0.000000in}}%
\pgfpathlineto{\pgfqpoint{0.000000in}{-0.048611in}}%
\pgfusepath{stroke,fill}%
}%
\begin{pgfscope}%
\pgfsys@transformshift{3.370343in}{0.320679in}%
\pgfsys@useobject{currentmarker}{}%
\end{pgfscope}%
\end{pgfscope}%
\begin{pgfscope}%
\definecolor{textcolor}{rgb}{0.000000,0.000000,0.000000}%
\pgfsetstrokecolor{textcolor}%
\pgfsetfillcolor{textcolor}%
\pgftext[x=3.370343in,y=0.223457in,,top]{\color{textcolor}\rmfamily\fontsize{10.000000}{12.000000}\selectfont \(\displaystyle 4.750\)}%
\end{pgfscope}%
\begin{pgfscope}%
\pgfsetbuttcap%
\pgfsetroundjoin%
\definecolor{currentfill}{rgb}{0.000000,0.000000,0.000000}%
\pgfsetfillcolor{currentfill}%
\pgfsetlinewidth{0.803000pt}%
\definecolor{currentstroke}{rgb}{0.000000,0.000000,0.000000}%
\pgfsetstrokecolor{currentstroke}%
\pgfsetdash{}{0pt}%
\pgfsys@defobject{currentmarker}{\pgfqpoint{0.000000in}{-0.048611in}}{\pgfqpoint{0.000000in}{0.000000in}}{%
\pgfpathmoveto{\pgfqpoint{0.000000in}{0.000000in}}%
\pgfpathlineto{\pgfqpoint{0.000000in}{-0.048611in}}%
\pgfusepath{stroke,fill}%
}%
\begin{pgfscope}%
\pgfsys@transformshift{3.810684in}{0.320679in}%
\pgfsys@useobject{currentmarker}{}%
\end{pgfscope}%
\end{pgfscope}%
\begin{pgfscope}%
\definecolor{textcolor}{rgb}{0.000000,0.000000,0.000000}%
\pgfsetstrokecolor{textcolor}%
\pgfsetfillcolor{textcolor}%
\pgftext[x=3.810684in,y=0.223457in,,top]{\color{textcolor}\rmfamily\fontsize{10.000000}{12.000000}\selectfont \(\displaystyle 4.775\)}%
\end{pgfscope}%
\begin{pgfscope}%
\pgfsetbuttcap%
\pgfsetroundjoin%
\definecolor{currentfill}{rgb}{0.000000,0.000000,0.000000}%
\pgfsetfillcolor{currentfill}%
\pgfsetlinewidth{0.803000pt}%
\definecolor{currentstroke}{rgb}{0.000000,0.000000,0.000000}%
\pgfsetstrokecolor{currentstroke}%
\pgfsetdash{}{0pt}%
\pgfsys@defobject{currentmarker}{\pgfqpoint{0.000000in}{-0.048611in}}{\pgfqpoint{0.000000in}{0.000000in}}{%
\pgfpathmoveto{\pgfqpoint{0.000000in}{0.000000in}}%
\pgfpathlineto{\pgfqpoint{0.000000in}{-0.048611in}}%
\pgfusepath{stroke,fill}%
}%
\begin{pgfscope}%
\pgfsys@transformshift{4.251025in}{0.320679in}%
\pgfsys@useobject{currentmarker}{}%
\end{pgfscope}%
\end{pgfscope}%
\begin{pgfscope}%
\definecolor{textcolor}{rgb}{0.000000,0.000000,0.000000}%
\pgfsetstrokecolor{textcolor}%
\pgfsetfillcolor{textcolor}%
\pgftext[x=4.251025in,y=0.223457in,,top]{\color{textcolor}\rmfamily\fontsize{10.000000}{12.000000}\selectfont \(\displaystyle 4.800\)}%
\end{pgfscope}%
\begin{pgfscope}%
\pgfsetbuttcap%
\pgfsetroundjoin%
\definecolor{currentfill}{rgb}{0.000000,0.000000,0.000000}%
\pgfsetfillcolor{currentfill}%
\pgfsetlinewidth{0.803000pt}%
\definecolor{currentstroke}{rgb}{0.000000,0.000000,0.000000}%
\pgfsetstrokecolor{currentstroke}%
\pgfsetdash{}{0pt}%
\pgfsys@defobject{currentmarker}{\pgfqpoint{-0.048611in}{0.000000in}}{\pgfqpoint{0.000000in}{0.000000in}}{%
\pgfpathmoveto{\pgfqpoint{0.000000in}{0.000000in}}%
\pgfpathlineto{\pgfqpoint{-0.048611in}{0.000000in}}%
\pgfusepath{stroke,fill}%
}%
\begin{pgfscope}%
\pgfsys@transformshift{0.552162in}{0.324038in}%
\pgfsys@useobject{currentmarker}{}%
\end{pgfscope}%
\end{pgfscope}%
\begin{pgfscope}%
\definecolor{textcolor}{rgb}{0.000000,0.000000,0.000000}%
\pgfsetstrokecolor{textcolor}%
\pgfsetfillcolor{textcolor}%
\pgftext[x=0.100000in,y=0.275813in,left,base]{\color{textcolor}\rmfamily\fontsize{10.000000}{12.000000}\selectfont \(\displaystyle -0.10\)}%
\end{pgfscope}%
\begin{pgfscope}%
\pgfsetbuttcap%
\pgfsetroundjoin%
\definecolor{currentfill}{rgb}{0.000000,0.000000,0.000000}%
\pgfsetfillcolor{currentfill}%
\pgfsetlinewidth{0.803000pt}%
\definecolor{currentstroke}{rgb}{0.000000,0.000000,0.000000}%
\pgfsetstrokecolor{currentstroke}%
\pgfsetdash{}{0pt}%
\pgfsys@defobject{currentmarker}{\pgfqpoint{-0.048611in}{0.000000in}}{\pgfqpoint{0.000000in}{0.000000in}}{%
\pgfpathmoveto{\pgfqpoint{0.000000in}{0.000000in}}%
\pgfpathlineto{\pgfqpoint{-0.048611in}{0.000000in}}%
\pgfusepath{stroke,fill}%
}%
\begin{pgfscope}%
\pgfsys@transformshift{0.552162in}{0.818697in}%
\pgfsys@useobject{currentmarker}{}%
\end{pgfscope}%
\end{pgfscope}%
\begin{pgfscope}%
\definecolor{textcolor}{rgb}{0.000000,0.000000,0.000000}%
\pgfsetstrokecolor{textcolor}%
\pgfsetfillcolor{textcolor}%
\pgftext[x=0.100000in,y=0.770472in,left,base]{\color{textcolor}\rmfamily\fontsize{10.000000}{12.000000}\selectfont \(\displaystyle -0.05\)}%
\end{pgfscope}%
\begin{pgfscope}%
\pgfsetbuttcap%
\pgfsetroundjoin%
\definecolor{currentfill}{rgb}{0.000000,0.000000,0.000000}%
\pgfsetfillcolor{currentfill}%
\pgfsetlinewidth{0.803000pt}%
\definecolor{currentstroke}{rgb}{0.000000,0.000000,0.000000}%
\pgfsetstrokecolor{currentstroke}%
\pgfsetdash{}{0pt}%
\pgfsys@defobject{currentmarker}{\pgfqpoint{-0.048611in}{0.000000in}}{\pgfqpoint{0.000000in}{0.000000in}}{%
\pgfpathmoveto{\pgfqpoint{0.000000in}{0.000000in}}%
\pgfpathlineto{\pgfqpoint{-0.048611in}{0.000000in}}%
\pgfusepath{stroke,fill}%
}%
\begin{pgfscope}%
\pgfsys@transformshift{0.552162in}{1.313356in}%
\pgfsys@useobject{currentmarker}{}%
\end{pgfscope}%
\end{pgfscope}%
\begin{pgfscope}%
\definecolor{textcolor}{rgb}{0.000000,0.000000,0.000000}%
\pgfsetstrokecolor{textcolor}%
\pgfsetfillcolor{textcolor}%
\pgftext[x=0.208025in,y=1.265131in,left,base]{\color{textcolor}\rmfamily\fontsize{10.000000}{12.000000}\selectfont \(\displaystyle 0.00\)}%
\end{pgfscope}%
\begin{pgfscope}%
\pgfsetbuttcap%
\pgfsetroundjoin%
\definecolor{currentfill}{rgb}{0.000000,0.000000,0.000000}%
\pgfsetfillcolor{currentfill}%
\pgfsetlinewidth{0.803000pt}%
\definecolor{currentstroke}{rgb}{0.000000,0.000000,0.000000}%
\pgfsetstrokecolor{currentstroke}%
\pgfsetdash{}{0pt}%
\pgfsys@defobject{currentmarker}{\pgfqpoint{-0.048611in}{0.000000in}}{\pgfqpoint{0.000000in}{0.000000in}}{%
\pgfpathmoveto{\pgfqpoint{0.000000in}{0.000000in}}%
\pgfpathlineto{\pgfqpoint{-0.048611in}{0.000000in}}%
\pgfusepath{stroke,fill}%
}%
\begin{pgfscope}%
\pgfsys@transformshift{0.552162in}{1.808015in}%
\pgfsys@useobject{currentmarker}{}%
\end{pgfscope}%
\end{pgfscope}%
\begin{pgfscope}%
\definecolor{textcolor}{rgb}{0.000000,0.000000,0.000000}%
\pgfsetstrokecolor{textcolor}%
\pgfsetfillcolor{textcolor}%
\pgftext[x=0.208025in,y=1.759790in,left,base]{\color{textcolor}\rmfamily\fontsize{10.000000}{12.000000}\selectfont \(\displaystyle 0.05\)}%
\end{pgfscope}%
\begin{pgfscope}%
\pgfpathrectangle{\pgfqpoint{0.552162in}{0.320679in}}{\pgfqpoint{3.875000in}{1.925000in}}%
\pgfusepath{clip}%
\pgfsetrectcap%
\pgfsetroundjoin%
\pgfsetlinewidth{1.505625pt}%
\definecolor{currentstroke}{rgb}{0.117647,0.564706,1.000000}%
\pgfsetstrokecolor{currentstroke}%
\pgfsetdash{}{0pt}%
\pgfpathmoveto{\pgfqpoint{0.728298in}{0.603611in}}%
\pgfpathlineto{\pgfqpoint{0.737105in}{0.662288in}}%
\pgfpathlineto{\pgfqpoint{0.745912in}{0.606949in}}%
\pgfpathlineto{\pgfqpoint{0.754718in}{0.644304in}}%
\pgfpathlineto{\pgfqpoint{0.763525in}{0.585891in}}%
\pgfpathlineto{\pgfqpoint{0.772332in}{0.688988in}}%
\pgfpathlineto{\pgfqpoint{0.781139in}{0.448816in}}%
\pgfpathlineto{\pgfqpoint{0.789946in}{0.690950in}}%
\pgfpathlineto{\pgfqpoint{0.798752in}{0.478231in}}%
\pgfpathlineto{\pgfqpoint{0.807559in}{0.683113in}}%
\pgfpathlineto{\pgfqpoint{0.816366in}{0.589392in}}%
\pgfpathlineto{\pgfqpoint{0.825173in}{0.733186in}}%
\pgfpathlineto{\pgfqpoint{0.833980in}{0.590164in}}%
\pgfpathlineto{\pgfqpoint{0.842787in}{0.647338in}}%
\pgfpathlineto{\pgfqpoint{0.851593in}{0.539135in}}%
\pgfpathlineto{\pgfqpoint{0.860400in}{0.527595in}}%
\pgfpathlineto{\pgfqpoint{0.869207in}{0.708121in}}%
\pgfpathlineto{\pgfqpoint{0.878014in}{0.504018in}}%
\pgfpathlineto{\pgfqpoint{0.886821in}{0.661577in}}%
\pgfpathlineto{\pgfqpoint{0.895627in}{0.674232in}}%
\pgfpathlineto{\pgfqpoint{0.904434in}{0.620480in}}%
\pgfpathlineto{\pgfqpoint{0.913241in}{0.717534in}}%
\pgfpathlineto{\pgfqpoint{0.922048in}{0.615054in}}%
\pgfpathlineto{\pgfqpoint{0.930855in}{0.708798in}}%
\pgfpathlineto{\pgfqpoint{0.939662in}{0.408179in}}%
\pgfpathlineto{\pgfqpoint{0.948468in}{0.637278in}}%
\pgfpathlineto{\pgfqpoint{0.957275in}{0.690815in}}%
\pgfpathlineto{\pgfqpoint{0.966082in}{0.871871in}}%
\pgfpathlineto{\pgfqpoint{0.974889in}{0.666235in}}%
\pgfpathlineto{\pgfqpoint{0.983696in}{0.620533in}}%
\pgfpathlineto{\pgfqpoint{0.992502in}{0.939170in}}%
\pgfpathlineto{\pgfqpoint{1.001309in}{0.700450in}}%
\pgfpathlineto{\pgfqpoint{1.010116in}{0.825356in}}%
\pgfpathlineto{\pgfqpoint{1.018923in}{0.598431in}}%
\pgfpathlineto{\pgfqpoint{1.027730in}{0.591474in}}%
\pgfpathlineto{\pgfqpoint{1.036537in}{0.623102in}}%
\pgfpathlineto{\pgfqpoint{1.045343in}{0.747232in}}%
\pgfpathlineto{\pgfqpoint{1.054150in}{0.591576in}}%
\pgfpathlineto{\pgfqpoint{1.062957in}{0.632301in}}%
\pgfpathlineto{\pgfqpoint{1.071764in}{0.597429in}}%
\pgfpathlineto{\pgfqpoint{1.080571in}{0.755809in}}%
\pgfpathlineto{\pgfqpoint{1.089377in}{0.759977in}}%
\pgfpathlineto{\pgfqpoint{1.098184in}{0.848406in}}%
\pgfpathlineto{\pgfqpoint{1.106991in}{0.814127in}}%
\pgfpathlineto{\pgfqpoint{1.115798in}{0.700741in}}%
\pgfpathlineto{\pgfqpoint{1.124605in}{0.620107in}}%
\pgfpathlineto{\pgfqpoint{1.133412in}{0.667908in}}%
\pgfpathlineto{\pgfqpoint{1.142218in}{0.834455in}}%
\pgfpathlineto{\pgfqpoint{1.151025in}{0.739883in}}%
\pgfpathlineto{\pgfqpoint{1.159832in}{0.718757in}}%
\pgfpathlineto{\pgfqpoint{1.168639in}{0.677170in}}%
\pgfpathlineto{\pgfqpoint{1.177446in}{0.876073in}}%
\pgfpathlineto{\pgfqpoint{1.186252in}{0.802594in}}%
\pgfpathlineto{\pgfqpoint{1.195059in}{0.751560in}}%
\pgfpathlineto{\pgfqpoint{1.203866in}{0.656339in}}%
\pgfpathlineto{\pgfqpoint{1.212673in}{0.706627in}}%
\pgfpathlineto{\pgfqpoint{1.221480in}{0.619192in}}%
\pgfpathlineto{\pgfqpoint{1.230287in}{0.833930in}}%
\pgfpathlineto{\pgfqpoint{1.239093in}{0.769651in}}%
\pgfpathlineto{\pgfqpoint{1.247900in}{0.784568in}}%
\pgfpathlineto{\pgfqpoint{1.256707in}{0.745420in}}%
\pgfpathlineto{\pgfqpoint{1.265514in}{0.739332in}}%
\pgfpathlineto{\pgfqpoint{1.274321in}{0.679675in}}%
\pgfpathlineto{\pgfqpoint{1.283127in}{0.760759in}}%
\pgfpathlineto{\pgfqpoint{1.291934in}{0.763794in}}%
\pgfpathlineto{\pgfqpoint{1.300741in}{0.757411in}}%
\pgfpathlineto{\pgfqpoint{1.309548in}{0.730864in}}%
\pgfpathlineto{\pgfqpoint{1.318355in}{0.736748in}}%
\pgfpathlineto{\pgfqpoint{1.327162in}{0.872072in}}%
\pgfpathlineto{\pgfqpoint{1.335968in}{0.924633in}}%
\pgfpathlineto{\pgfqpoint{1.344775in}{0.886783in}}%
\pgfpathlineto{\pgfqpoint{1.353582in}{0.822014in}}%
\pgfpathlineto{\pgfqpoint{1.362389in}{0.881034in}}%
\pgfpathlineto{\pgfqpoint{1.371196in}{0.845683in}}%
\pgfpathlineto{\pgfqpoint{1.380002in}{0.911430in}}%
\pgfpathlineto{\pgfqpoint{1.397616in}{0.800660in}}%
\pgfpathlineto{\pgfqpoint{1.406423in}{0.561779in}}%
\pgfpathlineto{\pgfqpoint{1.415230in}{0.713172in}}%
\pgfpathlineto{\pgfqpoint{1.424037in}{0.768838in}}%
\pgfpathlineto{\pgfqpoint{1.432843in}{0.913647in}}%
\pgfpathlineto{\pgfqpoint{1.441650in}{0.888668in}}%
\pgfpathlineto{\pgfqpoint{1.450457in}{0.947106in}}%
\pgfpathlineto{\pgfqpoint{1.459264in}{0.739947in}}%
\pgfpathlineto{\pgfqpoint{1.468071in}{0.964611in}}%
\pgfpathlineto{\pgfqpoint{1.476877in}{0.842095in}}%
\pgfpathlineto{\pgfqpoint{1.485684in}{0.918439in}}%
\pgfpathlineto{\pgfqpoint{1.494491in}{0.869461in}}%
\pgfpathlineto{\pgfqpoint{1.503298in}{0.870905in}}%
\pgfpathlineto{\pgfqpoint{1.512105in}{0.830422in}}%
\pgfpathlineto{\pgfqpoint{1.520912in}{0.827306in}}%
\pgfpathlineto{\pgfqpoint{1.529718in}{0.517507in}}%
\pgfpathlineto{\pgfqpoint{1.538525in}{0.817848in}}%
\pgfpathlineto{\pgfqpoint{1.547332in}{0.746006in}}%
\pgfpathlineto{\pgfqpoint{1.556139in}{0.917252in}}%
\pgfpathlineto{\pgfqpoint{1.564946in}{0.830605in}}%
\pgfpathlineto{\pgfqpoint{1.573752in}{0.826458in}}%
\pgfpathlineto{\pgfqpoint{1.582559in}{0.894298in}}%
\pgfpathlineto{\pgfqpoint{1.591366in}{0.767592in}}%
\pgfpathlineto{\pgfqpoint{1.600173in}{1.040307in}}%
\pgfpathlineto{\pgfqpoint{1.608980in}{0.883246in}}%
\pgfpathlineto{\pgfqpoint{1.617787in}{0.919531in}}%
\pgfpathlineto{\pgfqpoint{1.626593in}{0.867229in}}%
\pgfpathlineto{\pgfqpoint{1.635400in}{0.994966in}}%
\pgfpathlineto{\pgfqpoint{1.644207in}{1.013983in}}%
\pgfpathlineto{\pgfqpoint{1.653014in}{0.980414in}}%
\pgfpathlineto{\pgfqpoint{1.661821in}{0.997709in}}%
\pgfpathlineto{\pgfqpoint{1.670627in}{0.865987in}}%
\pgfpathlineto{\pgfqpoint{1.679434in}{0.803618in}}%
\pgfpathlineto{\pgfqpoint{1.688241in}{0.971467in}}%
\pgfpathlineto{\pgfqpoint{1.697048in}{1.003807in}}%
\pgfpathlineto{\pgfqpoint{1.705855in}{0.954638in}}%
\pgfpathlineto{\pgfqpoint{1.714662in}{0.867853in}}%
\pgfpathlineto{\pgfqpoint{1.723468in}{0.998704in}}%
\pgfpathlineto{\pgfqpoint{1.732275in}{1.017908in}}%
\pgfpathlineto{\pgfqpoint{1.741082in}{0.874434in}}%
\pgfpathlineto{\pgfqpoint{1.749889in}{0.961441in}}%
\pgfpathlineto{\pgfqpoint{1.758696in}{0.867480in}}%
\pgfpathlineto{\pgfqpoint{1.767502in}{1.088848in}}%
\pgfpathlineto{\pgfqpoint{1.776309in}{1.047426in}}%
\pgfpathlineto{\pgfqpoint{1.785116in}{0.941542in}}%
\pgfpathlineto{\pgfqpoint{1.793923in}{1.046770in}}%
\pgfpathlineto{\pgfqpoint{1.802730in}{1.006071in}}%
\pgfpathlineto{\pgfqpoint{1.811537in}{0.937677in}}%
\pgfpathlineto{\pgfqpoint{1.820343in}{1.133231in}}%
\pgfpathlineto{\pgfqpoint{1.829150in}{1.052845in}}%
\pgfpathlineto{\pgfqpoint{1.837957in}{0.858342in}}%
\pgfpathlineto{\pgfqpoint{1.846764in}{0.888802in}}%
\pgfpathlineto{\pgfqpoint{1.855571in}{1.090847in}}%
\pgfpathlineto{\pgfqpoint{1.864377in}{1.044614in}}%
\pgfpathlineto{\pgfqpoint{1.873184in}{1.085673in}}%
\pgfpathlineto{\pgfqpoint{1.881991in}{0.850642in}}%
\pgfpathlineto{\pgfqpoint{1.890798in}{1.070241in}}%
\pgfpathlineto{\pgfqpoint{1.899605in}{1.196485in}}%
\pgfpathlineto{\pgfqpoint{1.917218in}{0.864289in}}%
\pgfpathlineto{\pgfqpoint{1.926025in}{1.046678in}}%
\pgfpathlineto{\pgfqpoint{1.934832in}{1.004344in}}%
\pgfpathlineto{\pgfqpoint{1.943639in}{1.089112in}}%
\pgfpathlineto{\pgfqpoint{1.952446in}{0.931364in}}%
\pgfpathlineto{\pgfqpoint{1.961252in}{1.041866in}}%
\pgfpathlineto{\pgfqpoint{1.970059in}{1.113684in}}%
\pgfpathlineto{\pgfqpoint{1.978866in}{1.064870in}}%
\pgfpathlineto{\pgfqpoint{1.987673in}{1.119216in}}%
\pgfpathlineto{\pgfqpoint{1.996480in}{0.956340in}}%
\pgfpathlineto{\pgfqpoint{2.005287in}{1.153501in}}%
\pgfpathlineto{\pgfqpoint{2.014093in}{1.037082in}}%
\pgfpathlineto{\pgfqpoint{2.022900in}{0.900663in}}%
\pgfpathlineto{\pgfqpoint{2.031707in}{1.282312in}}%
\pgfpathlineto{\pgfqpoint{2.040514in}{0.849125in}}%
\pgfpathlineto{\pgfqpoint{2.049321in}{1.175625in}}%
\pgfpathlineto{\pgfqpoint{2.058127in}{1.185500in}}%
\pgfpathlineto{\pgfqpoint{2.066934in}{1.139107in}}%
\pgfpathlineto{\pgfqpoint{2.075741in}{1.099298in}}%
\pgfpathlineto{\pgfqpoint{2.084548in}{1.210363in}}%
\pgfpathlineto{\pgfqpoint{2.093355in}{1.145361in}}%
\pgfpathlineto{\pgfqpoint{2.102162in}{1.164940in}}%
\pgfpathlineto{\pgfqpoint{2.110968in}{1.091411in}}%
\pgfpathlineto{\pgfqpoint{2.119775in}{1.140158in}}%
\pgfpathlineto{\pgfqpoint{2.128582in}{1.140374in}}%
\pgfpathlineto{\pgfqpoint{2.137389in}{0.942739in}}%
\pgfpathlineto{\pgfqpoint{2.146196in}{1.209514in}}%
\pgfpathlineto{\pgfqpoint{2.155002in}{1.089074in}}%
\pgfpathlineto{\pgfqpoint{2.163809in}{1.197436in}}%
\pgfpathlineto{\pgfqpoint{2.172616in}{1.067858in}}%
\pgfpathlineto{\pgfqpoint{2.181423in}{1.049725in}}%
\pgfpathlineto{\pgfqpoint{2.190230in}{1.165748in}}%
\pgfpathlineto{\pgfqpoint{2.199037in}{1.165512in}}%
\pgfpathlineto{\pgfqpoint{2.207843in}{1.171682in}}%
\pgfpathlineto{\pgfqpoint{2.216650in}{1.119013in}}%
\pgfpathlineto{\pgfqpoint{2.225457in}{1.003267in}}%
\pgfpathlineto{\pgfqpoint{2.234264in}{1.190647in}}%
\pgfpathlineto{\pgfqpoint{2.243071in}{1.254650in}}%
\pgfpathlineto{\pgfqpoint{2.251877in}{1.031296in}}%
\pgfpathlineto{\pgfqpoint{2.260684in}{1.297262in}}%
\pgfpathlineto{\pgfqpoint{2.269491in}{1.093953in}}%
\pgfpathlineto{\pgfqpoint{2.278298in}{1.186911in}}%
\pgfpathlineto{\pgfqpoint{2.287105in}{0.977911in}}%
\pgfpathlineto{\pgfqpoint{2.295912in}{1.238065in}}%
\pgfpathlineto{\pgfqpoint{2.304718in}{1.262486in}}%
\pgfpathlineto{\pgfqpoint{2.313525in}{1.243311in}}%
\pgfpathlineto{\pgfqpoint{2.322332in}{1.278513in}}%
\pgfpathlineto{\pgfqpoint{2.331139in}{1.109622in}}%
\pgfpathlineto{\pgfqpoint{2.339946in}{1.172670in}}%
\pgfpathlineto{\pgfqpoint{2.348752in}{0.885453in}}%
\pgfpathlineto{\pgfqpoint{2.357559in}{1.019106in}}%
\pgfpathlineto{\pgfqpoint{2.366366in}{1.049715in}}%
\pgfpathlineto{\pgfqpoint{2.375173in}{1.207246in}}%
\pgfpathlineto{\pgfqpoint{2.383980in}{1.269155in}}%
\pgfpathlineto{\pgfqpoint{2.392787in}{1.185860in}}%
\pgfpathlineto{\pgfqpoint{2.401593in}{1.276353in}}%
\pgfpathlineto{\pgfqpoint{2.410400in}{1.229350in}}%
\pgfpathlineto{\pgfqpoint{2.419207in}{1.151502in}}%
\pgfpathlineto{\pgfqpoint{2.436821in}{1.317749in}}%
\pgfpathlineto{\pgfqpoint{2.445627in}{1.113374in}}%
\pgfpathlineto{\pgfqpoint{2.454434in}{1.272050in}}%
\pgfpathlineto{\pgfqpoint{2.463241in}{1.181512in}}%
\pgfpathlineto{\pgfqpoint{2.472048in}{1.133610in}}%
\pgfpathlineto{\pgfqpoint{2.480855in}{1.485400in}}%
\pgfpathlineto{\pgfqpoint{2.489662in}{1.332646in}}%
\pgfpathlineto{\pgfqpoint{2.498468in}{1.314993in}}%
\pgfpathlineto{\pgfqpoint{2.507275in}{1.346876in}}%
\pgfpathlineto{\pgfqpoint{2.516082in}{1.384745in}}%
\pgfpathlineto{\pgfqpoint{2.524889in}{1.171676in}}%
\pgfpathlineto{\pgfqpoint{2.533696in}{1.193075in}}%
\pgfpathlineto{\pgfqpoint{2.542502in}{1.448836in}}%
\pgfpathlineto{\pgfqpoint{2.551309in}{1.044117in}}%
\pgfpathlineto{\pgfqpoint{2.560116in}{1.130780in}}%
\pgfpathlineto{\pgfqpoint{2.568923in}{1.116531in}}%
\pgfpathlineto{\pgfqpoint{2.577730in}{1.265972in}}%
\pgfpathlineto{\pgfqpoint{2.586537in}{1.286951in}}%
\pgfpathlineto{\pgfqpoint{2.595343in}{1.123929in}}%
\pgfpathlineto{\pgfqpoint{2.604150in}{1.239902in}}%
\pgfpathlineto{\pgfqpoint{2.612957in}{1.326978in}}%
\pgfpathlineto{\pgfqpoint{2.621764in}{1.212807in}}%
\pgfpathlineto{\pgfqpoint{2.630571in}{1.294524in}}%
\pgfpathlineto{\pgfqpoint{2.639377in}{1.286377in}}%
\pgfpathlineto{\pgfqpoint{2.648184in}{1.319033in}}%
\pgfpathlineto{\pgfqpoint{2.656991in}{1.150941in}}%
\pgfpathlineto{\pgfqpoint{2.665798in}{1.248888in}}%
\pgfpathlineto{\pgfqpoint{2.674605in}{1.300304in}}%
\pgfpathlineto{\pgfqpoint{2.683412in}{1.423982in}}%
\pgfpathlineto{\pgfqpoint{2.692218in}{1.328551in}}%
\pgfpathlineto{\pgfqpoint{2.701025in}{1.217116in}}%
\pgfpathlineto{\pgfqpoint{2.709832in}{1.463150in}}%
\pgfpathlineto{\pgfqpoint{2.718639in}{1.345102in}}%
\pgfpathlineto{\pgfqpoint{2.727446in}{1.336009in}}%
\pgfpathlineto{\pgfqpoint{2.736252in}{1.193389in}}%
\pgfpathlineto{\pgfqpoint{2.745059in}{1.382476in}}%
\pgfpathlineto{\pgfqpoint{2.753866in}{1.332344in}}%
\pgfpathlineto{\pgfqpoint{2.762673in}{1.417373in}}%
\pgfpathlineto{\pgfqpoint{2.771480in}{1.366564in}}%
\pgfpathlineto{\pgfqpoint{2.780287in}{1.293514in}}%
\pgfpathlineto{\pgfqpoint{2.789093in}{1.335687in}}%
\pgfpathlineto{\pgfqpoint{2.797900in}{1.235126in}}%
\pgfpathlineto{\pgfqpoint{2.806707in}{1.425426in}}%
\pgfpathlineto{\pgfqpoint{2.815514in}{1.269921in}}%
\pgfpathlineto{\pgfqpoint{2.824321in}{1.392544in}}%
\pgfpathlineto{\pgfqpoint{2.833127in}{1.299108in}}%
\pgfpathlineto{\pgfqpoint{2.841934in}{1.314706in}}%
\pgfpathlineto{\pgfqpoint{2.850741in}{1.359689in}}%
\pgfpathlineto{\pgfqpoint{2.859548in}{1.431793in}}%
\pgfpathlineto{\pgfqpoint{2.868355in}{1.260571in}}%
\pgfpathlineto{\pgfqpoint{2.877162in}{1.272154in}}%
\pgfpathlineto{\pgfqpoint{2.885968in}{1.272295in}}%
\pgfpathlineto{\pgfqpoint{2.903582in}{1.380342in}}%
\pgfpathlineto{\pgfqpoint{2.912389in}{1.397101in}}%
\pgfpathlineto{\pgfqpoint{2.921196in}{1.304046in}}%
\pgfpathlineto{\pgfqpoint{2.930002in}{1.249601in}}%
\pgfpathlineto{\pgfqpoint{2.938809in}{1.385108in}}%
\pgfpathlineto{\pgfqpoint{2.947616in}{1.363849in}}%
\pgfpathlineto{\pgfqpoint{2.956423in}{1.494299in}}%
\pgfpathlineto{\pgfqpoint{2.965230in}{1.510753in}}%
\pgfpathlineto{\pgfqpoint{2.974037in}{1.266931in}}%
\pgfpathlineto{\pgfqpoint{2.982843in}{1.367609in}}%
\pgfpathlineto{\pgfqpoint{2.991650in}{1.436225in}}%
\pgfpathlineto{\pgfqpoint{3.000457in}{1.350492in}}%
\pgfpathlineto{\pgfqpoint{3.009264in}{1.473919in}}%
\pgfpathlineto{\pgfqpoint{3.018071in}{1.444444in}}%
\pgfpathlineto{\pgfqpoint{3.026877in}{1.427865in}}%
\pgfpathlineto{\pgfqpoint{3.035684in}{1.267519in}}%
\pgfpathlineto{\pgfqpoint{3.044491in}{1.431517in}}%
\pgfpathlineto{\pgfqpoint{3.053298in}{1.405345in}}%
\pgfpathlineto{\pgfqpoint{3.062105in}{1.462252in}}%
\pgfpathlineto{\pgfqpoint{3.070912in}{1.374993in}}%
\pgfpathlineto{\pgfqpoint{3.079718in}{1.440127in}}%
\pgfpathlineto{\pgfqpoint{3.088525in}{1.456797in}}%
\pgfpathlineto{\pgfqpoint{3.097332in}{1.373076in}}%
\pgfpathlineto{\pgfqpoint{3.106139in}{1.333418in}}%
\pgfpathlineto{\pgfqpoint{3.114946in}{1.367132in}}%
\pgfpathlineto{\pgfqpoint{3.123752in}{1.338918in}}%
\pgfpathlineto{\pgfqpoint{3.132559in}{1.527284in}}%
\pgfpathlineto{\pgfqpoint{3.141366in}{1.410471in}}%
\pgfpathlineto{\pgfqpoint{3.150173in}{1.420108in}}%
\pgfpathlineto{\pgfqpoint{3.158980in}{1.479705in}}%
\pgfpathlineto{\pgfqpoint{3.167787in}{1.573943in}}%
\pgfpathlineto{\pgfqpoint{3.176593in}{1.426942in}}%
\pgfpathlineto{\pgfqpoint{3.194207in}{1.382112in}}%
\pgfpathlineto{\pgfqpoint{3.203014in}{1.406831in}}%
\pgfpathlineto{\pgfqpoint{3.211821in}{1.645988in}}%
\pgfpathlineto{\pgfqpoint{3.229434in}{1.445262in}}%
\pgfpathlineto{\pgfqpoint{3.238241in}{1.377907in}}%
\pgfpathlineto{\pgfqpoint{3.247048in}{1.559006in}}%
\pgfpathlineto{\pgfqpoint{3.255855in}{1.644302in}}%
\pgfpathlineto{\pgfqpoint{3.264662in}{1.456072in}}%
\pgfpathlineto{\pgfqpoint{3.273468in}{1.576580in}}%
\pgfpathlineto{\pgfqpoint{3.282275in}{1.551882in}}%
\pgfpathlineto{\pgfqpoint{3.291082in}{1.535901in}}%
\pgfpathlineto{\pgfqpoint{3.299889in}{1.514794in}}%
\pgfpathlineto{\pgfqpoint{3.308696in}{1.482358in}}%
\pgfpathlineto{\pgfqpoint{3.317502in}{1.599924in}}%
\pgfpathlineto{\pgfqpoint{3.326309in}{1.519504in}}%
\pgfpathlineto{\pgfqpoint{3.335116in}{1.533496in}}%
\pgfpathlineto{\pgfqpoint{3.343923in}{1.507624in}}%
\pgfpathlineto{\pgfqpoint{3.352730in}{1.565853in}}%
\pgfpathlineto{\pgfqpoint{3.361537in}{1.533139in}}%
\pgfpathlineto{\pgfqpoint{3.370343in}{1.449147in}}%
\pgfpathlineto{\pgfqpoint{3.379150in}{1.505367in}}%
\pgfpathlineto{\pgfqpoint{3.387957in}{1.782587in}}%
\pgfpathlineto{\pgfqpoint{3.396764in}{1.560750in}}%
\pgfpathlineto{\pgfqpoint{3.405571in}{1.508063in}}%
\pgfpathlineto{\pgfqpoint{3.414377in}{1.675748in}}%
\pgfpathlineto{\pgfqpoint{3.423184in}{1.538653in}}%
\pgfpathlineto{\pgfqpoint{3.431991in}{1.622500in}}%
\pgfpathlineto{\pgfqpoint{3.440798in}{1.555585in}}%
\pgfpathlineto{\pgfqpoint{3.449605in}{1.574392in}}%
\pgfpathlineto{\pgfqpoint{3.458412in}{1.493149in}}%
\pgfpathlineto{\pgfqpoint{3.467218in}{1.605725in}}%
\pgfpathlineto{\pgfqpoint{3.476025in}{1.748211in}}%
\pgfpathlineto{\pgfqpoint{3.484832in}{1.658288in}}%
\pgfpathlineto{\pgfqpoint{3.493639in}{1.609759in}}%
\pgfpathlineto{\pgfqpoint{3.502446in}{1.621740in}}%
\pgfpathlineto{\pgfqpoint{3.511252in}{1.661334in}}%
\pgfpathlineto{\pgfqpoint{3.520059in}{1.648961in}}%
\pgfpathlineto{\pgfqpoint{3.528866in}{1.596978in}}%
\pgfpathlineto{\pgfqpoint{3.537673in}{1.819568in}}%
\pgfpathlineto{\pgfqpoint{3.546480in}{1.587024in}}%
\pgfpathlineto{\pgfqpoint{3.555287in}{1.742371in}}%
\pgfpathlineto{\pgfqpoint{3.564093in}{1.761367in}}%
\pgfpathlineto{\pgfqpoint{3.572900in}{1.680584in}}%
\pgfpathlineto{\pgfqpoint{3.581707in}{1.520716in}}%
\pgfpathlineto{\pgfqpoint{3.590514in}{1.675031in}}%
\pgfpathlineto{\pgfqpoint{3.599321in}{1.553429in}}%
\pgfpathlineto{\pgfqpoint{3.608127in}{1.627741in}}%
\pgfpathlineto{\pgfqpoint{3.616934in}{1.565782in}}%
\pgfpathlineto{\pgfqpoint{3.634548in}{1.667112in}}%
\pgfpathlineto{\pgfqpoint{3.643355in}{1.529025in}}%
\pgfpathlineto{\pgfqpoint{3.652162in}{1.517283in}}%
\pgfpathlineto{\pgfqpoint{3.660968in}{1.638920in}}%
\pgfpathlineto{\pgfqpoint{3.669775in}{1.655540in}}%
\pgfpathlineto{\pgfqpoint{3.678582in}{1.736055in}}%
\pgfpathlineto{\pgfqpoint{3.687389in}{1.662235in}}%
\pgfpathlineto{\pgfqpoint{3.696196in}{1.657130in}}%
\pgfpathlineto{\pgfqpoint{3.705002in}{1.648141in}}%
\pgfpathlineto{\pgfqpoint{3.713809in}{1.735907in}}%
\pgfpathlineto{\pgfqpoint{3.722616in}{1.753182in}}%
\pgfpathlineto{\pgfqpoint{3.731423in}{1.683679in}}%
\pgfpathlineto{\pgfqpoint{3.740230in}{1.539842in}}%
\pgfpathlineto{\pgfqpoint{3.749037in}{1.621501in}}%
\pgfpathlineto{\pgfqpoint{3.757843in}{1.569401in}}%
\pgfpathlineto{\pgfqpoint{3.766650in}{1.618571in}}%
\pgfpathlineto{\pgfqpoint{3.775457in}{1.801326in}}%
\pgfpathlineto{\pgfqpoint{3.784264in}{1.654774in}}%
\pgfpathlineto{\pgfqpoint{3.793071in}{1.621296in}}%
\pgfpathlineto{\pgfqpoint{3.801877in}{1.911929in}}%
\pgfpathlineto{\pgfqpoint{3.810684in}{1.509563in}}%
\pgfpathlineto{\pgfqpoint{3.819491in}{1.783919in}}%
\pgfpathlineto{\pgfqpoint{3.828298in}{1.737656in}}%
\pgfpathlineto{\pgfqpoint{3.837105in}{1.617542in}}%
\pgfpathlineto{\pgfqpoint{3.845912in}{1.713641in}}%
\pgfpathlineto{\pgfqpoint{3.854718in}{1.744414in}}%
\pgfpathlineto{\pgfqpoint{3.872332in}{1.535429in}}%
\pgfpathlineto{\pgfqpoint{3.881139in}{1.864283in}}%
\pgfpathlineto{\pgfqpoint{3.898752in}{1.781458in}}%
\pgfpathlineto{\pgfqpoint{3.907559in}{1.898624in}}%
\pgfpathlineto{\pgfqpoint{3.916366in}{1.731733in}}%
\pgfpathlineto{\pgfqpoint{3.925173in}{1.678978in}}%
\pgfpathlineto{\pgfqpoint{3.933980in}{1.644728in}}%
\pgfpathlineto{\pgfqpoint{3.942787in}{1.818273in}}%
\pgfpathlineto{\pgfqpoint{3.951593in}{1.778990in}}%
\pgfpathlineto{\pgfqpoint{3.960400in}{1.665818in}}%
\pgfpathlineto{\pgfqpoint{3.969207in}{1.867013in}}%
\pgfpathlineto{\pgfqpoint{3.978014in}{1.677049in}}%
\pgfpathlineto{\pgfqpoint{3.986821in}{1.794417in}}%
\pgfpathlineto{\pgfqpoint{3.995627in}{1.822089in}}%
\pgfpathlineto{\pgfqpoint{4.004434in}{1.880235in}}%
\pgfpathlineto{\pgfqpoint{4.013241in}{1.782040in}}%
\pgfpathlineto{\pgfqpoint{4.022048in}{1.878328in}}%
\pgfpathlineto{\pgfqpoint{4.030855in}{1.866023in}}%
\pgfpathlineto{\pgfqpoint{4.039662in}{1.894079in}}%
\pgfpathlineto{\pgfqpoint{4.048468in}{1.656293in}}%
\pgfpathlineto{\pgfqpoint{4.057275in}{1.893772in}}%
\pgfpathlineto{\pgfqpoint{4.066082in}{1.833309in}}%
\pgfpathlineto{\pgfqpoint{4.074889in}{1.742461in}}%
\pgfpathlineto{\pgfqpoint{4.083696in}{1.782895in}}%
\pgfpathlineto{\pgfqpoint{4.092502in}{1.797573in}}%
\pgfpathlineto{\pgfqpoint{4.110116in}{1.916491in}}%
\pgfpathlineto{\pgfqpoint{4.118923in}{1.989089in}}%
\pgfpathlineto{\pgfqpoint{4.127730in}{1.599348in}}%
\pgfpathlineto{\pgfqpoint{4.136537in}{1.781068in}}%
\pgfpathlineto{\pgfqpoint{4.145343in}{1.832690in}}%
\pgfpathlineto{\pgfqpoint{4.154150in}{1.794498in}}%
\pgfpathlineto{\pgfqpoint{4.162957in}{1.981382in}}%
\pgfpathlineto{\pgfqpoint{4.171764in}{1.916506in}}%
\pgfpathlineto{\pgfqpoint{4.180571in}{1.723779in}}%
\pgfpathlineto{\pgfqpoint{4.189377in}{1.744207in}}%
\pgfpathlineto{\pgfqpoint{4.198184in}{1.879997in}}%
\pgfpathlineto{\pgfqpoint{4.206991in}{1.880056in}}%
\pgfpathlineto{\pgfqpoint{4.215798in}{1.890858in}}%
\pgfpathlineto{\pgfqpoint{4.224605in}{1.675025in}}%
\pgfpathlineto{\pgfqpoint{4.233412in}{1.796871in}}%
\pgfpathlineto{\pgfqpoint{4.242218in}{1.801169in}}%
\pgfpathlineto{\pgfqpoint{4.251025in}{1.964675in}}%
\pgfpathlineto{\pgfqpoint{4.251025in}{1.964675in}}%
\pgfusepath{stroke}%
\end{pgfscope}%
\begin{pgfscope}%
\pgfpathrectangle{\pgfqpoint{0.552162in}{0.320679in}}{\pgfqpoint{3.875000in}{1.925000in}}%
\pgfusepath{clip}%
\pgfsetrectcap%
\pgfsetroundjoin%
\pgfsetlinewidth{1.505625pt}%
\definecolor{currentstroke}{rgb}{1.000000,0.388235,0.278431}%
\pgfsetstrokecolor{currentstroke}%
\pgfsetdash{}{0pt}%
\pgfpathmoveto{\pgfqpoint{0.728298in}{0.666015in}}%
\pgfpathlineto{\pgfqpoint{0.737105in}{0.866836in}}%
\pgfpathlineto{\pgfqpoint{0.745912in}{0.737253in}}%
\pgfpathlineto{\pgfqpoint{0.754718in}{0.845810in}}%
\pgfpathlineto{\pgfqpoint{0.763525in}{0.667119in}}%
\pgfpathlineto{\pgfqpoint{0.772332in}{0.878836in}}%
\pgfpathlineto{\pgfqpoint{0.781139in}{0.908022in}}%
\pgfpathlineto{\pgfqpoint{0.798752in}{0.773782in}}%
\pgfpathlineto{\pgfqpoint{0.807559in}{0.770714in}}%
\pgfpathlineto{\pgfqpoint{0.816366in}{0.848803in}}%
\pgfpathlineto{\pgfqpoint{0.825173in}{0.769578in}}%
\pgfpathlineto{\pgfqpoint{0.833980in}{0.816786in}}%
\pgfpathlineto{\pgfqpoint{0.842787in}{0.805727in}}%
\pgfpathlineto{\pgfqpoint{0.851593in}{0.637910in}}%
\pgfpathlineto{\pgfqpoint{0.860400in}{0.751278in}}%
\pgfpathlineto{\pgfqpoint{0.869207in}{0.884718in}}%
\pgfpathlineto{\pgfqpoint{0.878014in}{0.702643in}}%
\pgfpathlineto{\pgfqpoint{0.886821in}{0.773503in}}%
\pgfpathlineto{\pgfqpoint{0.895627in}{0.833903in}}%
\pgfpathlineto{\pgfqpoint{0.904434in}{0.664719in}}%
\pgfpathlineto{\pgfqpoint{0.913241in}{0.843970in}}%
\pgfpathlineto{\pgfqpoint{0.922048in}{0.764916in}}%
\pgfpathlineto{\pgfqpoint{0.930855in}{0.658971in}}%
\pgfpathlineto{\pgfqpoint{0.939662in}{0.773938in}}%
\pgfpathlineto{\pgfqpoint{0.948468in}{0.909499in}}%
\pgfpathlineto{\pgfqpoint{0.957275in}{0.958353in}}%
\pgfpathlineto{\pgfqpoint{0.966082in}{0.742655in}}%
\pgfpathlineto{\pgfqpoint{0.974889in}{0.805352in}}%
\pgfpathlineto{\pgfqpoint{0.983696in}{0.766276in}}%
\pgfpathlineto{\pgfqpoint{0.992502in}{0.777832in}}%
\pgfpathlineto{\pgfqpoint{1.001309in}{0.757510in}}%
\pgfpathlineto{\pgfqpoint{1.010116in}{0.886863in}}%
\pgfpathlineto{\pgfqpoint{1.018923in}{0.966264in}}%
\pgfpathlineto{\pgfqpoint{1.027730in}{0.826737in}}%
\pgfpathlineto{\pgfqpoint{1.036537in}{0.924365in}}%
\pgfpathlineto{\pgfqpoint{1.045343in}{0.831130in}}%
\pgfpathlineto{\pgfqpoint{1.054150in}{0.848924in}}%
\pgfpathlineto{\pgfqpoint{1.071764in}{0.938784in}}%
\pgfpathlineto{\pgfqpoint{1.080571in}{0.790942in}}%
\pgfpathlineto{\pgfqpoint{1.089377in}{0.906903in}}%
\pgfpathlineto{\pgfqpoint{1.098184in}{0.840499in}}%
\pgfpathlineto{\pgfqpoint{1.106991in}{1.021025in}}%
\pgfpathlineto{\pgfqpoint{1.115798in}{0.753182in}}%
\pgfpathlineto{\pgfqpoint{1.124605in}{0.983411in}}%
\pgfpathlineto{\pgfqpoint{1.142218in}{0.766575in}}%
\pgfpathlineto{\pgfqpoint{1.151025in}{0.837289in}}%
\pgfpathlineto{\pgfqpoint{1.159832in}{0.966336in}}%
\pgfpathlineto{\pgfqpoint{1.168639in}{0.945562in}}%
\pgfpathlineto{\pgfqpoint{1.177446in}{0.853497in}}%
\pgfpathlineto{\pgfqpoint{1.186252in}{0.919137in}}%
\pgfpathlineto{\pgfqpoint{1.195059in}{0.966106in}}%
\pgfpathlineto{\pgfqpoint{1.203866in}{1.095270in}}%
\pgfpathlineto{\pgfqpoint{1.212673in}{0.760523in}}%
\pgfpathlineto{\pgfqpoint{1.221480in}{0.897945in}}%
\pgfpathlineto{\pgfqpoint{1.230287in}{0.878036in}}%
\pgfpathlineto{\pgfqpoint{1.239093in}{0.885659in}}%
\pgfpathlineto{\pgfqpoint{1.247900in}{1.024565in}}%
\pgfpathlineto{\pgfqpoint{1.256707in}{0.941174in}}%
\pgfpathlineto{\pgfqpoint{1.265514in}{0.986630in}}%
\pgfpathlineto{\pgfqpoint{1.274321in}{0.967321in}}%
\pgfpathlineto{\pgfqpoint{1.283127in}{0.868214in}}%
\pgfpathlineto{\pgfqpoint{1.291934in}{0.911926in}}%
\pgfpathlineto{\pgfqpoint{1.300741in}{1.024434in}}%
\pgfpathlineto{\pgfqpoint{1.309548in}{0.927058in}}%
\pgfpathlineto{\pgfqpoint{1.318355in}{0.850359in}}%
\pgfpathlineto{\pgfqpoint{1.327162in}{1.013251in}}%
\pgfpathlineto{\pgfqpoint{1.335968in}{0.971947in}}%
\pgfpathlineto{\pgfqpoint{1.344775in}{1.023083in}}%
\pgfpathlineto{\pgfqpoint{1.353582in}{0.888384in}}%
\pgfpathlineto{\pgfqpoint{1.362389in}{1.089542in}}%
\pgfpathlineto{\pgfqpoint{1.371196in}{1.036778in}}%
\pgfpathlineto{\pgfqpoint{1.388809in}{0.882671in}}%
\pgfpathlineto{\pgfqpoint{1.397616in}{0.850410in}}%
\pgfpathlineto{\pgfqpoint{1.406423in}{1.064088in}}%
\pgfpathlineto{\pgfqpoint{1.415230in}{1.018371in}}%
\pgfpathlineto{\pgfqpoint{1.424037in}{0.955573in}}%
\pgfpathlineto{\pgfqpoint{1.432843in}{0.981967in}}%
\pgfpathlineto{\pgfqpoint{1.441650in}{1.163509in}}%
\pgfpathlineto{\pgfqpoint{1.450457in}{0.945942in}}%
\pgfpathlineto{\pgfqpoint{1.459264in}{1.043039in}}%
\pgfpathlineto{\pgfqpoint{1.468071in}{1.048059in}}%
\pgfpathlineto{\pgfqpoint{1.476877in}{0.815040in}}%
\pgfpathlineto{\pgfqpoint{1.485684in}{0.997289in}}%
\pgfpathlineto{\pgfqpoint{1.494491in}{0.923653in}}%
\pgfpathlineto{\pgfqpoint{1.503298in}{1.181762in}}%
\pgfpathlineto{\pgfqpoint{1.512105in}{1.017305in}}%
\pgfpathlineto{\pgfqpoint{1.520912in}{1.105100in}}%
\pgfpathlineto{\pgfqpoint{1.529718in}{1.146857in}}%
\pgfpathlineto{\pgfqpoint{1.538525in}{1.068396in}}%
\pgfpathlineto{\pgfqpoint{1.547332in}{1.036249in}}%
\pgfpathlineto{\pgfqpoint{1.556139in}{0.885195in}}%
\pgfpathlineto{\pgfqpoint{1.564946in}{1.008658in}}%
\pgfpathlineto{\pgfqpoint{1.573752in}{0.983453in}}%
\pgfpathlineto{\pgfqpoint{1.582559in}{1.090489in}}%
\pgfpathlineto{\pgfqpoint{1.591366in}{1.036878in}}%
\pgfpathlineto{\pgfqpoint{1.600173in}{1.211066in}}%
\pgfpathlineto{\pgfqpoint{1.608980in}{1.087133in}}%
\pgfpathlineto{\pgfqpoint{1.617787in}{0.836141in}}%
\pgfpathlineto{\pgfqpoint{1.626593in}{0.976041in}}%
\pgfpathlineto{\pgfqpoint{1.635400in}{1.256929in}}%
\pgfpathlineto{\pgfqpoint{1.644207in}{1.075725in}}%
\pgfpathlineto{\pgfqpoint{1.653014in}{1.171516in}}%
\pgfpathlineto{\pgfqpoint{1.661821in}{0.887608in}}%
\pgfpathlineto{\pgfqpoint{1.670627in}{1.217839in}}%
\pgfpathlineto{\pgfqpoint{1.688241in}{1.023936in}}%
\pgfpathlineto{\pgfqpoint{1.697048in}{1.076473in}}%
\pgfpathlineto{\pgfqpoint{1.705855in}{1.183421in}}%
\pgfpathlineto{\pgfqpoint{1.714662in}{1.078266in}}%
\pgfpathlineto{\pgfqpoint{1.723468in}{1.111978in}}%
\pgfpathlineto{\pgfqpoint{1.732275in}{1.099426in}}%
\pgfpathlineto{\pgfqpoint{1.741082in}{1.286005in}}%
\pgfpathlineto{\pgfqpoint{1.749889in}{1.237390in}}%
\pgfpathlineto{\pgfqpoint{1.758696in}{1.051971in}}%
\pgfpathlineto{\pgfqpoint{1.767502in}{1.147911in}}%
\pgfpathlineto{\pgfqpoint{1.776309in}{1.161110in}}%
\pgfpathlineto{\pgfqpoint{1.785116in}{1.166064in}}%
\pgfpathlineto{\pgfqpoint{1.793923in}{1.116658in}}%
\pgfpathlineto{\pgfqpoint{1.802730in}{1.250390in}}%
\pgfpathlineto{\pgfqpoint{1.811537in}{1.170878in}}%
\pgfpathlineto{\pgfqpoint{1.820343in}{1.183683in}}%
\pgfpathlineto{\pgfqpoint{1.829150in}{1.057853in}}%
\pgfpathlineto{\pgfqpoint{1.837957in}{1.242866in}}%
\pgfpathlineto{\pgfqpoint{1.846764in}{1.330367in}}%
\pgfpathlineto{\pgfqpoint{1.855571in}{1.351168in}}%
\pgfpathlineto{\pgfqpoint{1.864377in}{1.148761in}}%
\pgfpathlineto{\pgfqpoint{1.873184in}{1.102424in}}%
\pgfpathlineto{\pgfqpoint{1.881991in}{1.251200in}}%
\pgfpathlineto{\pgfqpoint{1.890798in}{1.285623in}}%
\pgfpathlineto{\pgfqpoint{1.899605in}{1.244486in}}%
\pgfpathlineto{\pgfqpoint{1.908412in}{1.062048in}}%
\pgfpathlineto{\pgfqpoint{1.917218in}{1.139170in}}%
\pgfpathlineto{\pgfqpoint{1.926025in}{1.205495in}}%
\pgfpathlineto{\pgfqpoint{1.934832in}{1.217656in}}%
\pgfpathlineto{\pgfqpoint{1.943639in}{1.268007in}}%
\pgfpathlineto{\pgfqpoint{1.952446in}{1.326330in}}%
\pgfpathlineto{\pgfqpoint{1.961252in}{1.244080in}}%
\pgfpathlineto{\pgfqpoint{1.970059in}{1.117971in}}%
\pgfpathlineto{\pgfqpoint{1.978866in}{1.127893in}}%
\pgfpathlineto{\pgfqpoint{1.987673in}{1.156367in}}%
\pgfpathlineto{\pgfqpoint{1.996480in}{1.194591in}}%
\pgfpathlineto{\pgfqpoint{2.005287in}{1.177754in}}%
\pgfpathlineto{\pgfqpoint{2.014093in}{1.299086in}}%
\pgfpathlineto{\pgfqpoint{2.022900in}{1.010828in}}%
\pgfpathlineto{\pgfqpoint{2.031707in}{1.206601in}}%
\pgfpathlineto{\pgfqpoint{2.040514in}{1.202201in}}%
\pgfpathlineto{\pgfqpoint{2.049321in}{1.390136in}}%
\pgfpathlineto{\pgfqpoint{2.058127in}{1.220990in}}%
\pgfpathlineto{\pgfqpoint{2.066934in}{1.239805in}}%
\pgfpathlineto{\pgfqpoint{2.075741in}{1.337691in}}%
\pgfpathlineto{\pgfqpoint{2.084548in}{1.229899in}}%
\pgfpathlineto{\pgfqpoint{2.093355in}{1.402705in}}%
\pgfpathlineto{\pgfqpoint{2.102162in}{1.290349in}}%
\pgfpathlineto{\pgfqpoint{2.110968in}{1.096128in}}%
\pgfpathlineto{\pgfqpoint{2.119775in}{1.226540in}}%
\pgfpathlineto{\pgfqpoint{2.128582in}{1.316769in}}%
\pgfpathlineto{\pgfqpoint{2.137389in}{1.212228in}}%
\pgfpathlineto{\pgfqpoint{2.146196in}{1.390153in}}%
\pgfpathlineto{\pgfqpoint{2.155002in}{1.351217in}}%
\pgfpathlineto{\pgfqpoint{2.163809in}{1.392301in}}%
\pgfpathlineto{\pgfqpoint{2.172616in}{1.284315in}}%
\pgfpathlineto{\pgfqpoint{2.181423in}{1.301491in}}%
\pgfpathlineto{\pgfqpoint{2.190230in}{1.341722in}}%
\pgfpathlineto{\pgfqpoint{2.199037in}{1.254979in}}%
\pgfpathlineto{\pgfqpoint{2.207843in}{1.353077in}}%
\pgfpathlineto{\pgfqpoint{2.216650in}{1.315615in}}%
\pgfpathlineto{\pgfqpoint{2.225457in}{1.331105in}}%
\pgfpathlineto{\pgfqpoint{2.234264in}{1.265862in}}%
\pgfpathlineto{\pgfqpoint{2.243071in}{1.449337in}}%
\pgfpathlineto{\pgfqpoint{2.260684in}{1.216059in}}%
\pgfpathlineto{\pgfqpoint{2.269491in}{1.263705in}}%
\pgfpathlineto{\pgfqpoint{2.278298in}{1.235350in}}%
\pgfpathlineto{\pgfqpoint{2.287105in}{1.320014in}}%
\pgfpathlineto{\pgfqpoint{2.295912in}{1.317628in}}%
\pgfpathlineto{\pgfqpoint{2.304718in}{1.343361in}}%
\pgfpathlineto{\pgfqpoint{2.313525in}{1.271430in}}%
\pgfpathlineto{\pgfqpoint{2.322332in}{1.486385in}}%
\pgfpathlineto{\pgfqpoint{2.331139in}{1.354842in}}%
\pgfpathlineto{\pgfqpoint{2.339946in}{1.309885in}}%
\pgfpathlineto{\pgfqpoint{2.348752in}{1.478864in}}%
\pgfpathlineto{\pgfqpoint{2.357559in}{1.406546in}}%
\pgfpathlineto{\pgfqpoint{2.366366in}{1.413293in}}%
\pgfpathlineto{\pgfqpoint{2.375173in}{1.451914in}}%
\pgfpathlineto{\pgfqpoint{2.383980in}{1.255629in}}%
\pgfpathlineto{\pgfqpoint{2.392787in}{1.266748in}}%
\pgfpathlineto{\pgfqpoint{2.401593in}{1.238285in}}%
\pgfpathlineto{\pgfqpoint{2.410400in}{1.262052in}}%
\pgfpathlineto{\pgfqpoint{2.428014in}{1.498624in}}%
\pgfpathlineto{\pgfqpoint{2.436821in}{1.402699in}}%
\pgfpathlineto{\pgfqpoint{2.445627in}{1.321004in}}%
\pgfpathlineto{\pgfqpoint{2.454434in}{1.490477in}}%
\pgfpathlineto{\pgfqpoint{2.463241in}{1.313432in}}%
\pgfpathlineto{\pgfqpoint{2.472048in}{1.497411in}}%
\pgfpathlineto{\pgfqpoint{2.480855in}{1.369706in}}%
\pgfpathlineto{\pgfqpoint{2.498468in}{1.456845in}}%
\pgfpathlineto{\pgfqpoint{2.507275in}{1.446564in}}%
\pgfpathlineto{\pgfqpoint{2.516082in}{1.305487in}}%
\pgfpathlineto{\pgfqpoint{2.524889in}{1.525271in}}%
\pgfpathlineto{\pgfqpoint{2.533696in}{1.459787in}}%
\pgfpathlineto{\pgfqpoint{2.542502in}{1.213592in}}%
\pgfpathlineto{\pgfqpoint{2.551309in}{1.416719in}}%
\pgfpathlineto{\pgfqpoint{2.560116in}{1.412458in}}%
\pgfpathlineto{\pgfqpoint{2.568923in}{1.591079in}}%
\pgfpathlineto{\pgfqpoint{2.577730in}{1.651924in}}%
\pgfpathlineto{\pgfqpoint{2.586537in}{1.234908in}}%
\pgfpathlineto{\pgfqpoint{2.595343in}{1.486208in}}%
\pgfpathlineto{\pgfqpoint{2.604150in}{1.586802in}}%
\pgfpathlineto{\pgfqpoint{2.612957in}{1.649367in}}%
\pgfpathlineto{\pgfqpoint{2.621764in}{1.536738in}}%
\pgfpathlineto{\pgfqpoint{2.630571in}{1.466912in}}%
\pgfpathlineto{\pgfqpoint{2.639377in}{1.303020in}}%
\pgfpathlineto{\pgfqpoint{2.648184in}{1.655564in}}%
\pgfpathlineto{\pgfqpoint{2.656991in}{1.349862in}}%
\pgfpathlineto{\pgfqpoint{2.665798in}{1.569626in}}%
\pgfpathlineto{\pgfqpoint{2.674605in}{1.499003in}}%
\pgfpathlineto{\pgfqpoint{2.683412in}{1.403738in}}%
\pgfpathlineto{\pgfqpoint{2.692218in}{1.381620in}}%
\pgfpathlineto{\pgfqpoint{2.701025in}{1.532722in}}%
\pgfpathlineto{\pgfqpoint{2.709832in}{1.479611in}}%
\pgfpathlineto{\pgfqpoint{2.718639in}{1.511761in}}%
\pgfpathlineto{\pgfqpoint{2.727446in}{1.353488in}}%
\pgfpathlineto{\pgfqpoint{2.736252in}{1.551505in}}%
\pgfpathlineto{\pgfqpoint{2.745059in}{1.491758in}}%
\pgfpathlineto{\pgfqpoint{2.753866in}{1.497470in}}%
\pgfpathlineto{\pgfqpoint{2.762673in}{1.625272in}}%
\pgfpathlineto{\pgfqpoint{2.771480in}{1.644700in}}%
\pgfpathlineto{\pgfqpoint{2.780287in}{1.392051in}}%
\pgfpathlineto{\pgfqpoint{2.789093in}{1.609145in}}%
\pgfpathlineto{\pgfqpoint{2.797900in}{1.613365in}}%
\pgfpathlineto{\pgfqpoint{2.815514in}{1.362201in}}%
\pgfpathlineto{\pgfqpoint{2.824321in}{1.612280in}}%
\pgfpathlineto{\pgfqpoint{2.833127in}{1.637455in}}%
\pgfpathlineto{\pgfqpoint{2.841934in}{1.686594in}}%
\pgfpathlineto{\pgfqpoint{2.850741in}{1.584458in}}%
\pgfpathlineto{\pgfqpoint{2.859548in}{1.604457in}}%
\pgfpathlineto{\pgfqpoint{2.868355in}{1.405492in}}%
\pgfpathlineto{\pgfqpoint{2.877162in}{1.697566in}}%
\pgfpathlineto{\pgfqpoint{2.885968in}{1.672421in}}%
\pgfpathlineto{\pgfqpoint{2.894775in}{1.545540in}}%
\pgfpathlineto{\pgfqpoint{2.903582in}{1.596634in}}%
\pgfpathlineto{\pgfqpoint{2.912389in}{1.352279in}}%
\pgfpathlineto{\pgfqpoint{2.921196in}{1.603308in}}%
\pgfpathlineto{\pgfqpoint{2.930002in}{1.578831in}}%
\pgfpathlineto{\pgfqpoint{2.938809in}{1.612412in}}%
\pgfpathlineto{\pgfqpoint{2.947616in}{1.486971in}}%
\pgfpathlineto{\pgfqpoint{2.956423in}{1.503876in}}%
\pgfpathlineto{\pgfqpoint{2.965230in}{1.555589in}}%
\pgfpathlineto{\pgfqpoint{2.974037in}{1.546228in}}%
\pgfpathlineto{\pgfqpoint{2.982843in}{1.491502in}}%
\pgfpathlineto{\pgfqpoint{2.991650in}{1.703798in}}%
\pgfpathlineto{\pgfqpoint{3.009264in}{1.468538in}}%
\pgfpathlineto{\pgfqpoint{3.018071in}{1.717845in}}%
\pgfpathlineto{\pgfqpoint{3.026877in}{1.710169in}}%
\pgfpathlineto{\pgfqpoint{3.035684in}{1.649243in}}%
\pgfpathlineto{\pgfqpoint{3.044491in}{1.516637in}}%
\pgfpathlineto{\pgfqpoint{3.053298in}{1.662467in}}%
\pgfpathlineto{\pgfqpoint{3.062105in}{1.551679in}}%
\pgfpathlineto{\pgfqpoint{3.070912in}{1.598848in}}%
\pgfpathlineto{\pgfqpoint{3.079718in}{1.619692in}}%
\pgfpathlineto{\pgfqpoint{3.088525in}{1.829223in}}%
\pgfpathlineto{\pgfqpoint{3.097332in}{1.624290in}}%
\pgfpathlineto{\pgfqpoint{3.106139in}{1.560184in}}%
\pgfpathlineto{\pgfqpoint{3.114946in}{1.739165in}}%
\pgfpathlineto{\pgfqpoint{3.123752in}{1.844241in}}%
\pgfpathlineto{\pgfqpoint{3.132559in}{1.611150in}}%
\pgfpathlineto{\pgfqpoint{3.141366in}{1.506826in}}%
\pgfpathlineto{\pgfqpoint{3.150173in}{1.747551in}}%
\pgfpathlineto{\pgfqpoint{3.158980in}{1.569807in}}%
\pgfpathlineto{\pgfqpoint{3.167787in}{1.678454in}}%
\pgfpathlineto{\pgfqpoint{3.176593in}{1.693518in}}%
\pgfpathlineto{\pgfqpoint{3.185400in}{1.809719in}}%
\pgfpathlineto{\pgfqpoint{3.194207in}{1.795519in}}%
\pgfpathlineto{\pgfqpoint{3.203014in}{1.737566in}}%
\pgfpathlineto{\pgfqpoint{3.211821in}{1.807754in}}%
\pgfpathlineto{\pgfqpoint{3.220627in}{1.738743in}}%
\pgfpathlineto{\pgfqpoint{3.229434in}{1.684513in}}%
\pgfpathlineto{\pgfqpoint{3.238241in}{1.703084in}}%
\pgfpathlineto{\pgfqpoint{3.247048in}{1.796991in}}%
\pgfpathlineto{\pgfqpoint{3.255855in}{1.646728in}}%
\pgfpathlineto{\pgfqpoint{3.264662in}{1.804619in}}%
\pgfpathlineto{\pgfqpoint{3.273468in}{1.794504in}}%
\pgfpathlineto{\pgfqpoint{3.282275in}{1.773077in}}%
\pgfpathlineto{\pgfqpoint{3.291082in}{1.884126in}}%
\pgfpathlineto{\pgfqpoint{3.299889in}{1.660262in}}%
\pgfpathlineto{\pgfqpoint{3.308696in}{1.860429in}}%
\pgfpathlineto{\pgfqpoint{3.317502in}{1.801900in}}%
\pgfpathlineto{\pgfqpoint{3.326309in}{1.645941in}}%
\pgfpathlineto{\pgfqpoint{3.335116in}{1.640188in}}%
\pgfpathlineto{\pgfqpoint{3.343923in}{1.696663in}}%
\pgfpathlineto{\pgfqpoint{3.352730in}{1.911065in}}%
\pgfpathlineto{\pgfqpoint{3.361537in}{1.915017in}}%
\pgfpathlineto{\pgfqpoint{3.370343in}{1.778226in}}%
\pgfpathlineto{\pgfqpoint{3.379150in}{1.750262in}}%
\pgfpathlineto{\pgfqpoint{3.387957in}{1.735301in}}%
\pgfpathlineto{\pgfqpoint{3.396764in}{1.825465in}}%
\pgfpathlineto{\pgfqpoint{3.405571in}{1.404097in}}%
\pgfpathlineto{\pgfqpoint{3.423184in}{1.949106in}}%
\pgfpathlineto{\pgfqpoint{3.431991in}{1.748077in}}%
\pgfpathlineto{\pgfqpoint{3.440798in}{1.608666in}}%
\pgfpathlineto{\pgfqpoint{3.449605in}{1.775863in}}%
\pgfpathlineto{\pgfqpoint{3.458412in}{1.643664in}}%
\pgfpathlineto{\pgfqpoint{3.467218in}{1.858857in}}%
\pgfpathlineto{\pgfqpoint{3.476025in}{1.816402in}}%
\pgfpathlineto{\pgfqpoint{3.484832in}{1.706115in}}%
\pgfpathlineto{\pgfqpoint{3.493639in}{1.779833in}}%
\pgfpathlineto{\pgfqpoint{3.502446in}{1.776608in}}%
\pgfpathlineto{\pgfqpoint{3.511252in}{1.833374in}}%
\pgfpathlineto{\pgfqpoint{3.520059in}{1.665041in}}%
\pgfpathlineto{\pgfqpoint{3.528866in}{1.985436in}}%
\pgfpathlineto{\pgfqpoint{3.537673in}{1.790097in}}%
\pgfpathlineto{\pgfqpoint{3.546480in}{1.869225in}}%
\pgfpathlineto{\pgfqpoint{3.555287in}{1.711400in}}%
\pgfpathlineto{\pgfqpoint{3.564093in}{1.821936in}}%
\pgfpathlineto{\pgfqpoint{3.572900in}{1.712961in}}%
\pgfpathlineto{\pgfqpoint{3.581707in}{1.837136in}}%
\pgfpathlineto{\pgfqpoint{3.590514in}{1.815731in}}%
\pgfpathlineto{\pgfqpoint{3.599321in}{1.682846in}}%
\pgfpathlineto{\pgfqpoint{3.608127in}{1.721893in}}%
\pgfpathlineto{\pgfqpoint{3.616934in}{1.854031in}}%
\pgfpathlineto{\pgfqpoint{3.625741in}{2.032822in}}%
\pgfpathlineto{\pgfqpoint{3.634548in}{1.726987in}}%
\pgfpathlineto{\pgfqpoint{3.643355in}{1.660438in}}%
\pgfpathlineto{\pgfqpoint{3.652162in}{1.812323in}}%
\pgfpathlineto{\pgfqpoint{3.660968in}{1.762082in}}%
\pgfpathlineto{\pgfqpoint{3.669775in}{1.880985in}}%
\pgfpathlineto{\pgfqpoint{3.678582in}{1.689471in}}%
\pgfpathlineto{\pgfqpoint{3.687389in}{1.899588in}}%
\pgfpathlineto{\pgfqpoint{3.696196in}{2.000071in}}%
\pgfpathlineto{\pgfqpoint{3.705002in}{1.852376in}}%
\pgfpathlineto{\pgfqpoint{3.713809in}{1.854825in}}%
\pgfpathlineto{\pgfqpoint{3.722616in}{1.908390in}}%
\pgfpathlineto{\pgfqpoint{3.731423in}{1.812400in}}%
\pgfpathlineto{\pgfqpoint{3.740230in}{1.978697in}}%
\pgfpathlineto{\pgfqpoint{3.749037in}{1.956380in}}%
\pgfpathlineto{\pgfqpoint{3.757843in}{1.943288in}}%
\pgfpathlineto{\pgfqpoint{3.766650in}{1.834616in}}%
\pgfpathlineto{\pgfqpoint{3.775457in}{1.825536in}}%
\pgfpathlineto{\pgfqpoint{3.784264in}{1.983318in}}%
\pgfpathlineto{\pgfqpoint{3.793071in}{1.677912in}}%
\pgfpathlineto{\pgfqpoint{3.810684in}{1.899449in}}%
\pgfpathlineto{\pgfqpoint{3.819491in}{1.987067in}}%
\pgfpathlineto{\pgfqpoint{3.828298in}{1.797416in}}%
\pgfpathlineto{\pgfqpoint{3.837105in}{2.048113in}}%
\pgfpathlineto{\pgfqpoint{3.845912in}{2.102143in}}%
\pgfpathlineto{\pgfqpoint{3.854718in}{1.897857in}}%
\pgfpathlineto{\pgfqpoint{3.863525in}{2.050367in}}%
\pgfpathlineto{\pgfqpoint{3.872332in}{1.997856in}}%
\pgfpathlineto{\pgfqpoint{3.881139in}{2.003322in}}%
\pgfpathlineto{\pgfqpoint{3.889946in}{1.853880in}}%
\pgfpathlineto{\pgfqpoint{3.898752in}{1.825631in}}%
\pgfpathlineto{\pgfqpoint{3.907559in}{2.088991in}}%
\pgfpathlineto{\pgfqpoint{3.916366in}{1.848534in}}%
\pgfpathlineto{\pgfqpoint{3.925173in}{1.991298in}}%
\pgfpathlineto{\pgfqpoint{3.933980in}{2.011707in}}%
\pgfpathlineto{\pgfqpoint{3.942787in}{1.843836in}}%
\pgfpathlineto{\pgfqpoint{3.951593in}{1.993778in}}%
\pgfpathlineto{\pgfqpoint{3.960400in}{1.736207in}}%
\pgfpathlineto{\pgfqpoint{3.969207in}{2.001044in}}%
\pgfpathlineto{\pgfqpoint{3.978014in}{1.934187in}}%
\pgfpathlineto{\pgfqpoint{3.986821in}{2.093499in}}%
\pgfpathlineto{\pgfqpoint{3.995627in}{2.059294in}}%
\pgfpathlineto{\pgfqpoint{4.004434in}{1.978966in}}%
\pgfpathlineto{\pgfqpoint{4.013241in}{2.133052in}}%
\pgfpathlineto{\pgfqpoint{4.022048in}{1.948484in}}%
\pgfpathlineto{\pgfqpoint{4.030855in}{2.093237in}}%
\pgfpathlineto{\pgfqpoint{4.039662in}{1.962702in}}%
\pgfpathlineto{\pgfqpoint{4.048468in}{2.158179in}}%
\pgfpathlineto{\pgfqpoint{4.057275in}{2.110011in}}%
\pgfpathlineto{\pgfqpoint{4.066082in}{1.951934in}}%
\pgfpathlineto{\pgfqpoint{4.074889in}{1.944628in}}%
\pgfpathlineto{\pgfqpoint{4.083696in}{1.911391in}}%
\pgfpathlineto{\pgfqpoint{4.092502in}{2.087101in}}%
\pgfpathlineto{\pgfqpoint{4.101309in}{2.011141in}}%
\pgfpathlineto{\pgfqpoint{4.110116in}{2.126584in}}%
\pgfpathlineto{\pgfqpoint{4.118923in}{2.117771in}}%
\pgfpathlineto{\pgfqpoint{4.127730in}{1.903846in}}%
\pgfpathlineto{\pgfqpoint{4.136537in}{2.034089in}}%
\pgfpathlineto{\pgfqpoint{4.145343in}{2.052734in}}%
\pgfpathlineto{\pgfqpoint{4.154150in}{2.049236in}}%
\pgfpathlineto{\pgfqpoint{4.162957in}{1.948846in}}%
\pgfpathlineto{\pgfqpoint{4.171764in}{2.096478in}}%
\pgfpathlineto{\pgfqpoint{4.180571in}{1.877202in}}%
\pgfpathlineto{\pgfqpoint{4.189377in}{1.933386in}}%
\pgfpathlineto{\pgfqpoint{4.198184in}{2.002575in}}%
\pgfpathlineto{\pgfqpoint{4.215798in}{1.991479in}}%
\pgfpathlineto{\pgfqpoint{4.224605in}{1.919557in}}%
\pgfpathlineto{\pgfqpoint{4.233412in}{2.063546in}}%
\pgfpathlineto{\pgfqpoint{4.242218in}{2.050733in}}%
\pgfpathlineto{\pgfqpoint{4.251025in}{1.897295in}}%
\pgfpathlineto{\pgfqpoint{4.251025in}{1.897295in}}%
\pgfusepath{stroke}%
\end{pgfscope}%
\begin{pgfscope}%
\pgfsetrectcap%
\pgfsetmiterjoin%
\pgfsetlinewidth{0.803000pt}%
\definecolor{currentstroke}{rgb}{0.000000,0.000000,0.000000}%
\pgfsetstrokecolor{currentstroke}%
\pgfsetdash{}{0pt}%
\pgfpathmoveto{\pgfqpoint{0.552162in}{0.320679in}}%
\pgfpathlineto{\pgfqpoint{0.552162in}{2.245679in}}%
\pgfusepath{stroke}%
\end{pgfscope}%
\begin{pgfscope}%
\pgfsetrectcap%
\pgfsetmiterjoin%
\pgfsetlinewidth{0.803000pt}%
\definecolor{currentstroke}{rgb}{0.000000,0.000000,0.000000}%
\pgfsetstrokecolor{currentstroke}%
\pgfsetdash{}{0pt}%
\pgfpathmoveto{\pgfqpoint{4.427162in}{0.320679in}}%
\pgfpathlineto{\pgfqpoint{4.427162in}{2.245679in}}%
\pgfusepath{stroke}%
\end{pgfscope}%
\begin{pgfscope}%
\pgfsetrectcap%
\pgfsetmiterjoin%
\pgfsetlinewidth{0.803000pt}%
\definecolor{currentstroke}{rgb}{0.000000,0.000000,0.000000}%
\pgfsetstrokecolor{currentstroke}%
\pgfsetdash{}{0pt}%
\pgfpathmoveto{\pgfqpoint{0.552162in}{0.320679in}}%
\pgfpathlineto{\pgfqpoint{4.427162in}{0.320679in}}%
\pgfusepath{stroke}%
\end{pgfscope}%
\begin{pgfscope}%
\pgfsetrectcap%
\pgfsetmiterjoin%
\pgfsetlinewidth{0.803000pt}%
\definecolor{currentstroke}{rgb}{0.000000,0.000000,0.000000}%
\pgfsetstrokecolor{currentstroke}%
\pgfsetdash{}{0pt}%
\pgfpathmoveto{\pgfqpoint{0.552162in}{2.245679in}}%
\pgfpathlineto{\pgfqpoint{4.427162in}{2.245679in}}%
\pgfusepath{stroke}%
\end{pgfscope}%
\end{pgfpicture}%
\makeatother%
\endgroup%

    \caption{Problemas en mediciones de frecuencias muy altas (superior) y muy bajas (inferior)}
  \end{figure}

  \subsection{Representación gráfica de $\varphi$ frente a f}

  En la siguiente gráfica veremos la relación entre los valores del desfase $\varphi$ representado frente a la escala logarítmica de la frecuencia. Utilizaremos un programa muy similar al descrito en el apartado \ref{sec:vmf}. Si simplemente tomamos los 20 valores de la tabla superior, y tratamos de justar a una recta, podemos ver que el ajuste no es demasiado perfecto.

  \begin{figure}[H]
    %\centering
    \hspace{2.5em} %% Creator: Matplotlib, PGF backend
%%
%% To include the figure in your LaTeX document, write
%%   \input{<filename>.pgf}
%%
%% Make sure the required packages are loaded in your preamble
%%   \usepackage{pgf}
%%
%% Figures using additional raster images can only be included by \input if
%% they are in the same directory as the main LaTeX file. For loading figures
%% from other directories you can use the `import` package
%%   \usepackage{import}
%% and then include the figures with
%%   \import{<path to file>}{<filename>.pgf}
%%
%% Matplotlib used the following preamble
%%
\begingroup%
\makeatletter%
\begin{pgfpicture}%
\pgfpathrectangle{\pgfpointorigin}{\pgfqpoint{4.938706in}{3.294691in}}%
\pgfusepath{use as bounding box, clip}%
\begin{pgfscope}%
\pgfsetbuttcap%
\pgfsetmiterjoin%
\definecolor{currentfill}{rgb}{1.000000,1.000000,1.000000}%
\pgfsetfillcolor{currentfill}%
\pgfsetlinewidth{0.000000pt}%
\definecolor{currentstroke}{rgb}{1.000000,1.000000,1.000000}%
\pgfsetstrokecolor{currentstroke}%
\pgfsetdash{}{0pt}%
\pgfpathmoveto{\pgfqpoint{0.000000in}{0.000000in}}%
\pgfpathlineto{\pgfqpoint{4.938706in}{0.000000in}}%
\pgfpathlineto{\pgfqpoint{4.938706in}{3.294691in}}%
\pgfpathlineto{\pgfqpoint{0.000000in}{3.294691in}}%
\pgfpathclose%
\pgfusepath{fill}%
\end{pgfscope}%
\begin{pgfscope}%
\pgfsetbuttcap%
\pgfsetmiterjoin%
\definecolor{currentfill}{rgb}{1.000000,1.000000,1.000000}%
\pgfsetfillcolor{currentfill}%
\pgfsetlinewidth{0.000000pt}%
\definecolor{currentstroke}{rgb}{0.000000,0.000000,0.000000}%
\pgfsetstrokecolor{currentstroke}%
\pgfsetstrokeopacity{0.000000}%
\pgfsetdash{}{0pt}%
\pgfpathmoveto{\pgfqpoint{0.947176in}{0.499691in}}%
\pgfpathlineto{\pgfqpoint{4.822176in}{0.499691in}}%
\pgfpathlineto{\pgfqpoint{4.822176in}{3.194691in}}%
\pgfpathlineto{\pgfqpoint{0.947176in}{3.194691in}}%
\pgfpathclose%
\pgfusepath{fill}%
\end{pgfscope}%
\begin{pgfscope}%
\pgfpathrectangle{\pgfqpoint{0.947176in}{0.499691in}}{\pgfqpoint{3.875000in}{2.695000in}}%
\pgfusepath{clip}%
\pgfsetrectcap%
\pgfsetroundjoin%
\pgfsetlinewidth{1.505625pt}%
\definecolor{currentstroke}{rgb}{0.529412,0.807843,0.921569}%
\pgfsetstrokecolor{currentstroke}%
\pgfsetdash{}{0pt}%
\pgfpathmoveto{\pgfqpoint{1.188953in}{0.622191in}}%
\pgfpathlineto{\pgfqpoint{1.443312in}{0.805941in}}%
\pgfpathlineto{\pgfqpoint{1.697670in}{0.989691in}}%
\pgfpathlineto{\pgfqpoint{1.952029in}{1.173441in}}%
\pgfpathlineto{\pgfqpoint{2.206387in}{1.357191in}}%
\pgfpathlineto{\pgfqpoint{2.460745in}{1.540941in}}%
\pgfpathlineto{\pgfqpoint{2.630318in}{1.663441in}}%
\pgfpathlineto{\pgfqpoint{2.799890in}{1.785941in}}%
\pgfpathlineto{\pgfqpoint{2.969462in}{1.908441in}}%
\pgfpathlineto{\pgfqpoint{3.139035in}{2.030941in}}%
\pgfpathlineto{\pgfqpoint{3.308607in}{2.153441in}}%
\pgfpathlineto{\pgfqpoint{3.478179in}{2.275941in}}%
\pgfpathlineto{\pgfqpoint{3.647751in}{2.398441in}}%
\pgfpathlineto{\pgfqpoint{3.817324in}{2.520941in}}%
\pgfpathlineto{\pgfqpoint{3.902110in}{2.582191in}}%
\pgfpathlineto{\pgfqpoint{4.071682in}{2.704691in}}%
\pgfpathlineto{\pgfqpoint{4.241254in}{2.827191in}}%
\pgfpathlineto{\pgfqpoint{4.326040in}{2.888441in}}%
\pgfpathlineto{\pgfqpoint{4.410827in}{2.949691in}}%
\pgfpathlineto{\pgfqpoint{4.580399in}{3.072191in}}%
\pgfusepath{stroke}%
\end{pgfscope}%
\begin{pgfscope}%
\pgfsetbuttcap%
\pgfsetroundjoin%
\definecolor{currentfill}{rgb}{0.000000,0.000000,0.000000}%
\pgfsetfillcolor{currentfill}%
\pgfsetlinewidth{0.803000pt}%
\definecolor{currentstroke}{rgb}{0.000000,0.000000,0.000000}%
\pgfsetstrokecolor{currentstroke}%
\pgfsetdash{}{0pt}%
\pgfsys@defobject{currentmarker}{\pgfqpoint{0.000000in}{-0.048611in}}{\pgfqpoint{0.000000in}{0.000000in}}{%
\pgfpathmoveto{\pgfqpoint{0.000000in}{0.000000in}}%
\pgfpathlineto{\pgfqpoint{0.000000in}{-0.048611in}}%
\pgfusepath{stroke,fill}%
}%
\begin{pgfscope}%
\pgfsys@transformshift{1.358526in}{0.499691in}%
\pgfsys@useobject{currentmarker}{}%
\end{pgfscope}%
\end{pgfscope}%
\begin{pgfscope}%
\definecolor{textcolor}{rgb}{0.000000,0.000000,0.000000}%
\pgfsetstrokecolor{textcolor}%
\pgfsetfillcolor{textcolor}%
\pgftext[x=1.358526in,y=0.402469in,,top]{\color{textcolor}\rmfamily\fontsize{10.000000}{12.000000}\selectfont \(\displaystyle 2.9\)}%
\end{pgfscope}%
\begin{pgfscope}%
\pgfsetbuttcap%
\pgfsetroundjoin%
\definecolor{currentfill}{rgb}{0.000000,0.000000,0.000000}%
\pgfsetfillcolor{currentfill}%
\pgfsetlinewidth{0.803000pt}%
\definecolor{currentstroke}{rgb}{0.000000,0.000000,0.000000}%
\pgfsetstrokecolor{currentstroke}%
\pgfsetdash{}{0pt}%
\pgfsys@defobject{currentmarker}{\pgfqpoint{0.000000in}{-0.048611in}}{\pgfqpoint{0.000000in}{0.000000in}}{%
\pgfpathmoveto{\pgfqpoint{0.000000in}{0.000000in}}%
\pgfpathlineto{\pgfqpoint{0.000000in}{-0.048611in}}%
\pgfusepath{stroke,fill}%
}%
\begin{pgfscope}%
\pgfsys@transformshift{2.206387in}{0.499691in}%
\pgfsys@useobject{currentmarker}{}%
\end{pgfscope}%
\end{pgfscope}%
\begin{pgfscope}%
\definecolor{textcolor}{rgb}{0.000000,0.000000,0.000000}%
\pgfsetstrokecolor{textcolor}%
\pgfsetfillcolor{textcolor}%
\pgftext[x=2.206387in,y=0.402469in,,top]{\color{textcolor}\rmfamily\fontsize{10.000000}{12.000000}\selectfont \(\displaystyle 3.0\)}%
\end{pgfscope}%
\begin{pgfscope}%
\pgfsetbuttcap%
\pgfsetroundjoin%
\definecolor{currentfill}{rgb}{0.000000,0.000000,0.000000}%
\pgfsetfillcolor{currentfill}%
\pgfsetlinewidth{0.803000pt}%
\definecolor{currentstroke}{rgb}{0.000000,0.000000,0.000000}%
\pgfsetstrokecolor{currentstroke}%
\pgfsetdash{}{0pt}%
\pgfsys@defobject{currentmarker}{\pgfqpoint{0.000000in}{-0.048611in}}{\pgfqpoint{0.000000in}{0.000000in}}{%
\pgfpathmoveto{\pgfqpoint{0.000000in}{0.000000in}}%
\pgfpathlineto{\pgfqpoint{0.000000in}{-0.048611in}}%
\pgfusepath{stroke,fill}%
}%
\begin{pgfscope}%
\pgfsys@transformshift{3.054248in}{0.499691in}%
\pgfsys@useobject{currentmarker}{}%
\end{pgfscope}%
\end{pgfscope}%
\begin{pgfscope}%
\definecolor{textcolor}{rgb}{0.000000,0.000000,0.000000}%
\pgfsetstrokecolor{textcolor}%
\pgfsetfillcolor{textcolor}%
\pgftext[x=3.054248in,y=0.402469in,,top]{\color{textcolor}\rmfamily\fontsize{10.000000}{12.000000}\selectfont \(\displaystyle 3.1\)}%
\end{pgfscope}%
\begin{pgfscope}%
\pgfsetbuttcap%
\pgfsetroundjoin%
\definecolor{currentfill}{rgb}{0.000000,0.000000,0.000000}%
\pgfsetfillcolor{currentfill}%
\pgfsetlinewidth{0.803000pt}%
\definecolor{currentstroke}{rgb}{0.000000,0.000000,0.000000}%
\pgfsetstrokecolor{currentstroke}%
\pgfsetdash{}{0pt}%
\pgfsys@defobject{currentmarker}{\pgfqpoint{0.000000in}{-0.048611in}}{\pgfqpoint{0.000000in}{0.000000in}}{%
\pgfpathmoveto{\pgfqpoint{0.000000in}{0.000000in}}%
\pgfpathlineto{\pgfqpoint{0.000000in}{-0.048611in}}%
\pgfusepath{stroke,fill}%
}%
\begin{pgfscope}%
\pgfsys@transformshift{3.902110in}{0.499691in}%
\pgfsys@useobject{currentmarker}{}%
\end{pgfscope}%
\end{pgfscope}%
\begin{pgfscope}%
\definecolor{textcolor}{rgb}{0.000000,0.000000,0.000000}%
\pgfsetstrokecolor{textcolor}%
\pgfsetfillcolor{textcolor}%
\pgftext[x=3.902110in,y=0.402469in,,top]{\color{textcolor}\rmfamily\fontsize{10.000000}{12.000000}\selectfont \(\displaystyle 3.2\)}%
\end{pgfscope}%
\begin{pgfscope}%
\pgfsetbuttcap%
\pgfsetroundjoin%
\definecolor{currentfill}{rgb}{0.000000,0.000000,0.000000}%
\pgfsetfillcolor{currentfill}%
\pgfsetlinewidth{0.803000pt}%
\definecolor{currentstroke}{rgb}{0.000000,0.000000,0.000000}%
\pgfsetstrokecolor{currentstroke}%
\pgfsetdash{}{0pt}%
\pgfsys@defobject{currentmarker}{\pgfqpoint{0.000000in}{-0.048611in}}{\pgfqpoint{0.000000in}{0.000000in}}{%
\pgfpathmoveto{\pgfqpoint{0.000000in}{0.000000in}}%
\pgfpathlineto{\pgfqpoint{0.000000in}{-0.048611in}}%
\pgfusepath{stroke,fill}%
}%
\begin{pgfscope}%
\pgfsys@transformshift{4.749971in}{0.499691in}%
\pgfsys@useobject{currentmarker}{}%
\end{pgfscope}%
\end{pgfscope}%
\begin{pgfscope}%
\definecolor{textcolor}{rgb}{0.000000,0.000000,0.000000}%
\pgfsetstrokecolor{textcolor}%
\pgfsetfillcolor{textcolor}%
\pgftext[x=4.749971in,y=0.402469in,,top]{\color{textcolor}\rmfamily\fontsize{10.000000}{12.000000}\selectfont \(\displaystyle 3.3\)}%
\end{pgfscope}%
\begin{pgfscope}%
\definecolor{textcolor}{rgb}{0.000000,0.000000,0.000000}%
\pgfsetstrokecolor{textcolor}%
\pgfsetfillcolor{textcolor}%
\pgftext[x=2.884676in,y=0.223457in,,top]{\color{textcolor}\rmfamily\fontsize{10.000000}{12.000000}\selectfont log f}%
\end{pgfscope}%
\begin{pgfscope}%
\pgfsetbuttcap%
\pgfsetroundjoin%
\definecolor{currentfill}{rgb}{0.000000,0.000000,0.000000}%
\pgfsetfillcolor{currentfill}%
\pgfsetlinewidth{0.803000pt}%
\definecolor{currentstroke}{rgb}{0.000000,0.000000,0.000000}%
\pgfsetstrokecolor{currentstroke}%
\pgfsetdash{}{0pt}%
\pgfsys@defobject{currentmarker}{\pgfqpoint{-0.048611in}{0.000000in}}{\pgfqpoint{0.000000in}{0.000000in}}{%
\pgfpathmoveto{\pgfqpoint{0.000000in}{0.000000in}}%
\pgfpathlineto{\pgfqpoint{-0.048611in}{0.000000in}}%
\pgfusepath{stroke,fill}%
}%
\begin{pgfscope}%
\pgfsys@transformshift{0.947176in}{0.702877in}%
\pgfsys@useobject{currentmarker}{}%
\end{pgfscope}%
\end{pgfscope}%
\begin{pgfscope}%
\definecolor{textcolor}{rgb}{0.000000,0.000000,0.000000}%
\pgfsetstrokecolor{textcolor}%
\pgfsetfillcolor{textcolor}%
\pgftext[x=0.603040in,y=0.654652in,left,base]{\color{textcolor}\rmfamily\fontsize{10.000000}{12.000000}\selectfont \(\displaystyle -65\)}%
\end{pgfscope}%
\begin{pgfscope}%
\pgfsetbuttcap%
\pgfsetroundjoin%
\definecolor{currentfill}{rgb}{0.000000,0.000000,0.000000}%
\pgfsetfillcolor{currentfill}%
\pgfsetlinewidth{0.803000pt}%
\definecolor{currentstroke}{rgb}{0.000000,0.000000,0.000000}%
\pgfsetstrokecolor{currentstroke}%
\pgfsetdash{}{0pt}%
\pgfsys@defobject{currentmarker}{\pgfqpoint{-0.048611in}{0.000000in}}{\pgfqpoint{0.000000in}{0.000000in}}{%
\pgfpathmoveto{\pgfqpoint{0.000000in}{0.000000in}}%
\pgfpathlineto{\pgfqpoint{-0.048611in}{0.000000in}}%
\pgfusepath{stroke,fill}%
}%
\begin{pgfscope}%
\pgfsys@transformshift{0.947176in}{1.134548in}%
\pgfsys@useobject{currentmarker}{}%
\end{pgfscope}%
\end{pgfscope}%
\begin{pgfscope}%
\definecolor{textcolor}{rgb}{0.000000,0.000000,0.000000}%
\pgfsetstrokecolor{textcolor}%
\pgfsetfillcolor{textcolor}%
\pgftext[x=0.603040in,y=1.086323in,left,base]{\color{textcolor}\rmfamily\fontsize{10.000000}{12.000000}\selectfont \(\displaystyle -60\)}%
\end{pgfscope}%
\begin{pgfscope}%
\pgfsetbuttcap%
\pgfsetroundjoin%
\definecolor{currentfill}{rgb}{0.000000,0.000000,0.000000}%
\pgfsetfillcolor{currentfill}%
\pgfsetlinewidth{0.803000pt}%
\definecolor{currentstroke}{rgb}{0.000000,0.000000,0.000000}%
\pgfsetstrokecolor{currentstroke}%
\pgfsetdash{}{0pt}%
\pgfsys@defobject{currentmarker}{\pgfqpoint{-0.048611in}{0.000000in}}{\pgfqpoint{0.000000in}{0.000000in}}{%
\pgfpathmoveto{\pgfqpoint{0.000000in}{0.000000in}}%
\pgfpathlineto{\pgfqpoint{-0.048611in}{0.000000in}}%
\pgfusepath{stroke,fill}%
}%
\begin{pgfscope}%
\pgfsys@transformshift{0.947176in}{1.566219in}%
\pgfsys@useobject{currentmarker}{}%
\end{pgfscope}%
\end{pgfscope}%
\begin{pgfscope}%
\definecolor{textcolor}{rgb}{0.000000,0.000000,0.000000}%
\pgfsetstrokecolor{textcolor}%
\pgfsetfillcolor{textcolor}%
\pgftext[x=0.603040in,y=1.517993in,left,base]{\color{textcolor}\rmfamily\fontsize{10.000000}{12.000000}\selectfont \(\displaystyle -55\)}%
\end{pgfscope}%
\begin{pgfscope}%
\pgfsetbuttcap%
\pgfsetroundjoin%
\definecolor{currentfill}{rgb}{0.000000,0.000000,0.000000}%
\pgfsetfillcolor{currentfill}%
\pgfsetlinewidth{0.803000pt}%
\definecolor{currentstroke}{rgb}{0.000000,0.000000,0.000000}%
\pgfsetstrokecolor{currentstroke}%
\pgfsetdash{}{0pt}%
\pgfsys@defobject{currentmarker}{\pgfqpoint{-0.048611in}{0.000000in}}{\pgfqpoint{0.000000in}{0.000000in}}{%
\pgfpathmoveto{\pgfqpoint{0.000000in}{0.000000in}}%
\pgfpathlineto{\pgfqpoint{-0.048611in}{0.000000in}}%
\pgfusepath{stroke,fill}%
}%
\begin{pgfscope}%
\pgfsys@transformshift{0.947176in}{1.997889in}%
\pgfsys@useobject{currentmarker}{}%
\end{pgfscope}%
\end{pgfscope}%
\begin{pgfscope}%
\definecolor{textcolor}{rgb}{0.000000,0.000000,0.000000}%
\pgfsetstrokecolor{textcolor}%
\pgfsetfillcolor{textcolor}%
\pgftext[x=0.603040in,y=1.949664in,left,base]{\color{textcolor}\rmfamily\fontsize{10.000000}{12.000000}\selectfont \(\displaystyle -50\)}%
\end{pgfscope}%
\begin{pgfscope}%
\pgfsetbuttcap%
\pgfsetroundjoin%
\definecolor{currentfill}{rgb}{0.000000,0.000000,0.000000}%
\pgfsetfillcolor{currentfill}%
\pgfsetlinewidth{0.803000pt}%
\definecolor{currentstroke}{rgb}{0.000000,0.000000,0.000000}%
\pgfsetstrokecolor{currentstroke}%
\pgfsetdash{}{0pt}%
\pgfsys@defobject{currentmarker}{\pgfqpoint{-0.048611in}{0.000000in}}{\pgfqpoint{0.000000in}{0.000000in}}{%
\pgfpathmoveto{\pgfqpoint{0.000000in}{0.000000in}}%
\pgfpathlineto{\pgfqpoint{-0.048611in}{0.000000in}}%
\pgfusepath{stroke,fill}%
}%
\begin{pgfscope}%
\pgfsys@transformshift{0.947176in}{2.429560in}%
\pgfsys@useobject{currentmarker}{}%
\end{pgfscope}%
\end{pgfscope}%
\begin{pgfscope}%
\definecolor{textcolor}{rgb}{0.000000,0.000000,0.000000}%
\pgfsetstrokecolor{textcolor}%
\pgfsetfillcolor{textcolor}%
\pgftext[x=0.603040in,y=2.381335in,left,base]{\color{textcolor}\rmfamily\fontsize{10.000000}{12.000000}\selectfont \(\displaystyle -45\)}%
\end{pgfscope}%
\begin{pgfscope}%
\pgfsetbuttcap%
\pgfsetroundjoin%
\definecolor{currentfill}{rgb}{0.000000,0.000000,0.000000}%
\pgfsetfillcolor{currentfill}%
\pgfsetlinewidth{0.803000pt}%
\definecolor{currentstroke}{rgb}{0.000000,0.000000,0.000000}%
\pgfsetstrokecolor{currentstroke}%
\pgfsetdash{}{0pt}%
\pgfsys@defobject{currentmarker}{\pgfqpoint{-0.048611in}{0.000000in}}{\pgfqpoint{0.000000in}{0.000000in}}{%
\pgfpathmoveto{\pgfqpoint{0.000000in}{0.000000in}}%
\pgfpathlineto{\pgfqpoint{-0.048611in}{0.000000in}}%
\pgfusepath{stroke,fill}%
}%
\begin{pgfscope}%
\pgfsys@transformshift{0.947176in}{2.861231in}%
\pgfsys@useobject{currentmarker}{}%
\end{pgfscope}%
\end{pgfscope}%
\begin{pgfscope}%
\definecolor{textcolor}{rgb}{0.000000,0.000000,0.000000}%
\pgfsetstrokecolor{textcolor}%
\pgfsetfillcolor{textcolor}%
\pgftext[x=0.603040in,y=2.813006in,left,base]{\color{textcolor}\rmfamily\fontsize{10.000000}{12.000000}\selectfont \(\displaystyle -40\)}%
\end{pgfscope}%
\begin{pgfscope}%
\definecolor{textcolor}{rgb}{0.000000,0.000000,0.000000}%
\pgfsetstrokecolor{textcolor}%
\pgfsetfillcolor{textcolor}%
\pgftext[x=0.255817in,y=1.847191in,,bottom]{\color{textcolor}\rmfamily\fontsize{10.000000}{12.000000}\selectfont \(\displaystyle \phi\) \textit{(º)}}%
\end{pgfscope}%
\begin{pgfscope}%
\pgfpathrectangle{\pgfqpoint{0.947176in}{0.499691in}}{\pgfqpoint{3.875000in}{2.695000in}}%
\pgfusepath{clip}%
\pgfsetbuttcap%
\pgfsetroundjoin%
\definecolor{currentfill}{rgb}{0.121569,0.466667,0.705882}%
\pgfsetfillcolor{currentfill}%
\pgfsetlinewidth{1.003750pt}%
\definecolor{currentstroke}{rgb}{0.121569,0.466667,0.705882}%
\pgfsetstrokecolor{currentstroke}%
\pgfsetdash{}{0pt}%
\pgfpathmoveto{\pgfqpoint{1.188953in}{0.790712in}}%
\pgfpathcurveto{\pgfqpoint{1.200003in}{0.790712in}}{\pgfqpoint{1.210603in}{0.795102in}}{\pgfqpoint{1.218416in}{0.802915in}}%
\pgfpathcurveto{\pgfqpoint{1.226230in}{0.810729in}}{\pgfqpoint{1.230620in}{0.821328in}}{\pgfqpoint{1.230620in}{0.832378in}}%
\pgfpathcurveto{\pgfqpoint{1.230620in}{0.843428in}}{\pgfqpoint{1.226230in}{0.854027in}}{\pgfqpoint{1.218416in}{0.861841in}}%
\pgfpathcurveto{\pgfqpoint{1.210603in}{0.869655in}}{\pgfqpoint{1.200003in}{0.874045in}}{\pgfqpoint{1.188953in}{0.874045in}}%
\pgfpathcurveto{\pgfqpoint{1.177903in}{0.874045in}}{\pgfqpoint{1.167304in}{0.869655in}}{\pgfqpoint{1.159491in}{0.861841in}}%
\pgfpathcurveto{\pgfqpoint{1.151677in}{0.854027in}}{\pgfqpoint{1.147287in}{0.843428in}}{\pgfqpoint{1.147287in}{0.832378in}}%
\pgfpathcurveto{\pgfqpoint{1.147287in}{0.821328in}}{\pgfqpoint{1.151677in}{0.810729in}}{\pgfqpoint{1.159491in}{0.802915in}}%
\pgfpathcurveto{\pgfqpoint{1.167304in}{0.795102in}}{\pgfqpoint{1.177903in}{0.790712in}}{\pgfqpoint{1.188953in}{0.790712in}}%
\pgfpathclose%
\pgfusepath{stroke,fill}%
\end{pgfscope}%
\begin{pgfscope}%
\pgfpathrectangle{\pgfqpoint{0.947176in}{0.499691in}}{\pgfqpoint{3.875000in}{2.695000in}}%
\pgfusepath{clip}%
\pgfsetbuttcap%
\pgfsetroundjoin%
\definecolor{currentfill}{rgb}{0.121569,0.466667,0.705882}%
\pgfsetfillcolor{currentfill}%
\pgfsetlinewidth{1.003750pt}%
\definecolor{currentstroke}{rgb}{0.121569,0.466667,0.705882}%
\pgfsetstrokecolor{currentstroke}%
\pgfsetdash{}{0pt}%
\pgfpathmoveto{\pgfqpoint{1.443312in}{0.669844in}}%
\pgfpathcurveto{\pgfqpoint{1.454362in}{0.669844in}}{\pgfqpoint{1.464961in}{0.674234in}}{\pgfqpoint{1.472775in}{0.682048in}}%
\pgfpathcurveto{\pgfqpoint{1.480588in}{0.689861in}}{\pgfqpoint{1.484978in}{0.700460in}}{\pgfqpoint{1.484978in}{0.711510in}}%
\pgfpathcurveto{\pgfqpoint{1.484978in}{0.722561in}}{\pgfqpoint{1.480588in}{0.733160in}}{\pgfqpoint{1.472775in}{0.740973in}}%
\pgfpathcurveto{\pgfqpoint{1.464961in}{0.748787in}}{\pgfqpoint{1.454362in}{0.753177in}}{\pgfqpoint{1.443312in}{0.753177in}}%
\pgfpathcurveto{\pgfqpoint{1.432262in}{0.753177in}}{\pgfqpoint{1.421663in}{0.748787in}}{\pgfqpoint{1.413849in}{0.740973in}}%
\pgfpathcurveto{\pgfqpoint{1.406035in}{0.733160in}}{\pgfqpoint{1.401645in}{0.722561in}}{\pgfqpoint{1.401645in}{0.711510in}}%
\pgfpathcurveto{\pgfqpoint{1.401645in}{0.700460in}}{\pgfqpoint{1.406035in}{0.689861in}}{\pgfqpoint{1.413849in}{0.682048in}}%
\pgfpathcurveto{\pgfqpoint{1.421663in}{0.674234in}}{\pgfqpoint{1.432262in}{0.669844in}}{\pgfqpoint{1.443312in}{0.669844in}}%
\pgfpathclose%
\pgfusepath{stroke,fill}%
\end{pgfscope}%
\begin{pgfscope}%
\pgfpathrectangle{\pgfqpoint{0.947176in}{0.499691in}}{\pgfqpoint{3.875000in}{2.695000in}}%
\pgfusepath{clip}%
\pgfsetbuttcap%
\pgfsetroundjoin%
\definecolor{currentfill}{rgb}{0.121569,0.466667,0.705882}%
\pgfsetfillcolor{currentfill}%
\pgfsetlinewidth{1.003750pt}%
\definecolor{currentstroke}{rgb}{0.121569,0.466667,0.705882}%
\pgfsetstrokecolor{currentstroke}%
\pgfsetdash{}{0pt}%
\pgfpathmoveto{\pgfqpoint{1.697670in}{0.583510in}}%
\pgfpathcurveto{\pgfqpoint{1.708720in}{0.583510in}}{\pgfqpoint{1.719319in}{0.587900in}}{\pgfqpoint{1.727133in}{0.595714in}}%
\pgfpathcurveto{\pgfqpoint{1.734947in}{0.603527in}}{\pgfqpoint{1.739337in}{0.614126in}}{\pgfqpoint{1.739337in}{0.625176in}}%
\pgfpathcurveto{\pgfqpoint{1.739337in}{0.636226in}}{\pgfqpoint{1.734947in}{0.646825in}}{\pgfqpoint{1.727133in}{0.654639in}}%
\pgfpathcurveto{\pgfqpoint{1.719319in}{0.662453in}}{\pgfqpoint{1.708720in}{0.666843in}}{\pgfqpoint{1.697670in}{0.666843in}}%
\pgfpathcurveto{\pgfqpoint{1.686620in}{0.666843in}}{\pgfqpoint{1.676021in}{0.662453in}}{\pgfqpoint{1.668207in}{0.654639in}}%
\pgfpathcurveto{\pgfqpoint{1.660394in}{0.646825in}}{\pgfqpoint{1.656004in}{0.636226in}}{\pgfqpoint{1.656004in}{0.625176in}}%
\pgfpathcurveto{\pgfqpoint{1.656004in}{0.614126in}}{\pgfqpoint{1.660394in}{0.603527in}}{\pgfqpoint{1.668207in}{0.595714in}}%
\pgfpathcurveto{\pgfqpoint{1.676021in}{0.587900in}}{\pgfqpoint{1.686620in}{0.583510in}}{\pgfqpoint{1.697670in}{0.583510in}}%
\pgfpathclose%
\pgfusepath{stroke,fill}%
\end{pgfscope}%
\begin{pgfscope}%
\pgfpathrectangle{\pgfqpoint{0.947176in}{0.499691in}}{\pgfqpoint{3.875000in}{2.695000in}}%
\pgfusepath{clip}%
\pgfsetbuttcap%
\pgfsetroundjoin%
\definecolor{currentfill}{rgb}{0.121569,0.466667,0.705882}%
\pgfsetfillcolor{currentfill}%
\pgfsetlinewidth{1.003750pt}%
\definecolor{currentstroke}{rgb}{0.121569,0.466667,0.705882}%
\pgfsetstrokecolor{currentstroke}%
\pgfsetdash{}{0pt}%
\pgfpathmoveto{\pgfqpoint{1.952029in}{1.015180in}}%
\pgfpathcurveto{\pgfqpoint{1.963079in}{1.015180in}}{\pgfqpoint{1.973678in}{1.019571in}}{\pgfqpoint{1.981491in}{1.027384in}}%
\pgfpathcurveto{\pgfqpoint{1.989305in}{1.035198in}}{\pgfqpoint{1.993695in}{1.045797in}}{\pgfqpoint{1.993695in}{1.056847in}}%
\pgfpathcurveto{\pgfqpoint{1.993695in}{1.067897in}}{\pgfqpoint{1.989305in}{1.078496in}}{\pgfqpoint{1.981491in}{1.086310in}}%
\pgfpathcurveto{\pgfqpoint{1.973678in}{1.094123in}}{\pgfqpoint{1.963079in}{1.098514in}}{\pgfqpoint{1.952029in}{1.098514in}}%
\pgfpathcurveto{\pgfqpoint{1.940978in}{1.098514in}}{\pgfqpoint{1.930379in}{1.094123in}}{\pgfqpoint{1.922566in}{1.086310in}}%
\pgfpathcurveto{\pgfqpoint{1.914752in}{1.078496in}}{\pgfqpoint{1.910362in}{1.067897in}}{\pgfqpoint{1.910362in}{1.056847in}}%
\pgfpathcurveto{\pgfqpoint{1.910362in}{1.045797in}}{\pgfqpoint{1.914752in}{1.035198in}}{\pgfqpoint{1.922566in}{1.027384in}}%
\pgfpathcurveto{\pgfqpoint{1.930379in}{1.019571in}}{\pgfqpoint{1.940978in}{1.015180in}}{\pgfqpoint{1.952029in}{1.015180in}}%
\pgfpathclose%
\pgfusepath{stroke,fill}%
\end{pgfscope}%
\begin{pgfscope}%
\pgfpathrectangle{\pgfqpoint{0.947176in}{0.499691in}}{\pgfqpoint{3.875000in}{2.695000in}}%
\pgfusepath{clip}%
\pgfsetbuttcap%
\pgfsetroundjoin%
\definecolor{currentfill}{rgb}{0.121569,0.466667,0.705882}%
\pgfsetfillcolor{currentfill}%
\pgfsetlinewidth{1.003750pt}%
\definecolor{currentstroke}{rgb}{0.121569,0.466667,0.705882}%
\pgfsetstrokecolor{currentstroke}%
\pgfsetdash{}{0pt}%
\pgfpathmoveto{\pgfqpoint{2.206387in}{1.300083in}}%
\pgfpathcurveto{\pgfqpoint{2.217437in}{1.300083in}}{\pgfqpoint{2.228036in}{1.304473in}}{\pgfqpoint{2.235850in}{1.312287in}}%
\pgfpathcurveto{\pgfqpoint{2.243663in}{1.320101in}}{\pgfqpoint{2.248054in}{1.330700in}}{\pgfqpoint{2.248054in}{1.341750in}}%
\pgfpathcurveto{\pgfqpoint{2.248054in}{1.352800in}}{\pgfqpoint{2.243663in}{1.363399in}}{\pgfqpoint{2.235850in}{1.371213in}}%
\pgfpathcurveto{\pgfqpoint{2.228036in}{1.379026in}}{\pgfqpoint{2.217437in}{1.383416in}}{\pgfqpoint{2.206387in}{1.383416in}}%
\pgfpathcurveto{\pgfqpoint{2.195337in}{1.383416in}}{\pgfqpoint{2.184738in}{1.379026in}}{\pgfqpoint{2.176924in}{1.371213in}}%
\pgfpathcurveto{\pgfqpoint{2.169111in}{1.363399in}}{\pgfqpoint{2.164720in}{1.352800in}}{\pgfqpoint{2.164720in}{1.341750in}}%
\pgfpathcurveto{\pgfqpoint{2.164720in}{1.330700in}}{\pgfqpoint{2.169111in}{1.320101in}}{\pgfqpoint{2.176924in}{1.312287in}}%
\pgfpathcurveto{\pgfqpoint{2.184738in}{1.304473in}}{\pgfqpoint{2.195337in}{1.300083in}}{\pgfqpoint{2.206387in}{1.300083in}}%
\pgfpathclose%
\pgfusepath{stroke,fill}%
\end{pgfscope}%
\begin{pgfscope}%
\pgfpathrectangle{\pgfqpoint{0.947176in}{0.499691in}}{\pgfqpoint{3.875000in}{2.695000in}}%
\pgfusepath{clip}%
\pgfsetbuttcap%
\pgfsetroundjoin%
\definecolor{currentfill}{rgb}{0.121569,0.466667,0.705882}%
\pgfsetfillcolor{currentfill}%
\pgfsetlinewidth{1.003750pt}%
\definecolor{currentstroke}{rgb}{0.121569,0.466667,0.705882}%
\pgfsetstrokecolor{currentstroke}%
\pgfsetdash{}{0pt}%
\pgfpathmoveto{\pgfqpoint{2.460745in}{1.792188in}}%
\pgfpathcurveto{\pgfqpoint{2.471796in}{1.792188in}}{\pgfqpoint{2.482395in}{1.796578in}}{\pgfqpoint{2.490208in}{1.804392in}}%
\pgfpathcurveto{\pgfqpoint{2.498022in}{1.812205in}}{\pgfqpoint{2.502412in}{1.822804in}}{\pgfqpoint{2.502412in}{1.833854in}}%
\pgfpathcurveto{\pgfqpoint{2.502412in}{1.844905in}}{\pgfqpoint{2.498022in}{1.855504in}}{\pgfqpoint{2.490208in}{1.863317in}}%
\pgfpathcurveto{\pgfqpoint{2.482395in}{1.871131in}}{\pgfqpoint{2.471796in}{1.875521in}}{\pgfqpoint{2.460745in}{1.875521in}}%
\pgfpathcurveto{\pgfqpoint{2.449695in}{1.875521in}}{\pgfqpoint{2.439096in}{1.871131in}}{\pgfqpoint{2.431283in}{1.863317in}}%
\pgfpathcurveto{\pgfqpoint{2.423469in}{1.855504in}}{\pgfqpoint{2.419079in}{1.844905in}}{\pgfqpoint{2.419079in}{1.833854in}}%
\pgfpathcurveto{\pgfqpoint{2.419079in}{1.822804in}}{\pgfqpoint{2.423469in}{1.812205in}}{\pgfqpoint{2.431283in}{1.804392in}}%
\pgfpathcurveto{\pgfqpoint{2.439096in}{1.796578in}}{\pgfqpoint{2.449695in}{1.792188in}}{\pgfqpoint{2.460745in}{1.792188in}}%
\pgfpathclose%
\pgfusepath{stroke,fill}%
\end{pgfscope}%
\begin{pgfscope}%
\pgfpathrectangle{\pgfqpoint{0.947176in}{0.499691in}}{\pgfqpoint{3.875000in}{2.695000in}}%
\pgfusepath{clip}%
\pgfsetbuttcap%
\pgfsetroundjoin%
\definecolor{currentfill}{rgb}{0.121569,0.466667,0.705882}%
\pgfsetfillcolor{currentfill}%
\pgfsetlinewidth{1.003750pt}%
\definecolor{currentstroke}{rgb}{0.121569,0.466667,0.705882}%
\pgfsetstrokecolor{currentstroke}%
\pgfsetdash{}{0pt}%
\pgfpathmoveto{\pgfqpoint{2.630318in}{1.679953in}}%
\pgfpathcurveto{\pgfqpoint{2.641368in}{1.679953in}}{\pgfqpoint{2.651967in}{1.684344in}}{\pgfqpoint{2.659780in}{1.692157in}}%
\pgfpathcurveto{\pgfqpoint{2.667594in}{1.699971in}}{\pgfqpoint{2.671984in}{1.710570in}}{\pgfqpoint{2.671984in}{1.721620in}}%
\pgfpathcurveto{\pgfqpoint{2.671984in}{1.732670in}}{\pgfqpoint{2.667594in}{1.743269in}}{\pgfqpoint{2.659780in}{1.751083in}}%
\pgfpathcurveto{\pgfqpoint{2.651967in}{1.758896in}}{\pgfqpoint{2.641368in}{1.763287in}}{\pgfqpoint{2.630318in}{1.763287in}}%
\pgfpathcurveto{\pgfqpoint{2.619268in}{1.763287in}}{\pgfqpoint{2.608669in}{1.758896in}}{\pgfqpoint{2.600855in}{1.751083in}}%
\pgfpathcurveto{\pgfqpoint{2.593041in}{1.743269in}}{\pgfqpoint{2.588651in}{1.732670in}}{\pgfqpoint{2.588651in}{1.721620in}}%
\pgfpathcurveto{\pgfqpoint{2.588651in}{1.710570in}}{\pgfqpoint{2.593041in}{1.699971in}}{\pgfqpoint{2.600855in}{1.692157in}}%
\pgfpathcurveto{\pgfqpoint{2.608669in}{1.684344in}}{\pgfqpoint{2.619268in}{1.679953in}}{\pgfqpoint{2.630318in}{1.679953in}}%
\pgfpathclose%
\pgfusepath{stroke,fill}%
\end{pgfscope}%
\begin{pgfscope}%
\pgfpathrectangle{\pgfqpoint{0.947176in}{0.499691in}}{\pgfqpoint{3.875000in}{2.695000in}}%
\pgfusepath{clip}%
\pgfsetbuttcap%
\pgfsetroundjoin%
\definecolor{currentfill}{rgb}{0.121569,0.466667,0.705882}%
\pgfsetfillcolor{currentfill}%
\pgfsetlinewidth{1.003750pt}%
\definecolor{currentstroke}{rgb}{0.121569,0.466667,0.705882}%
\pgfsetstrokecolor{currentstroke}%
\pgfsetdash{}{0pt}%
\pgfpathmoveto{\pgfqpoint{2.799890in}{1.723120in}}%
\pgfpathcurveto{\pgfqpoint{2.810940in}{1.723120in}}{\pgfqpoint{2.821539in}{1.727511in}}{\pgfqpoint{2.829353in}{1.735324in}}%
\pgfpathcurveto{\pgfqpoint{2.837166in}{1.743138in}}{\pgfqpoint{2.841557in}{1.753737in}}{\pgfqpoint{2.841557in}{1.764787in}}%
\pgfpathcurveto{\pgfqpoint{2.841557in}{1.775837in}}{\pgfqpoint{2.837166in}{1.786436in}}{\pgfqpoint{2.829353in}{1.794250in}}%
\pgfpathcurveto{\pgfqpoint{2.821539in}{1.802064in}}{\pgfqpoint{2.810940in}{1.806454in}}{\pgfqpoint{2.799890in}{1.806454in}}%
\pgfpathcurveto{\pgfqpoint{2.788840in}{1.806454in}}{\pgfqpoint{2.778241in}{1.802064in}}{\pgfqpoint{2.770427in}{1.794250in}}%
\pgfpathcurveto{\pgfqpoint{2.762614in}{1.786436in}}{\pgfqpoint{2.758223in}{1.775837in}}{\pgfqpoint{2.758223in}{1.764787in}}%
\pgfpathcurveto{\pgfqpoint{2.758223in}{1.753737in}}{\pgfqpoint{2.762614in}{1.743138in}}{\pgfqpoint{2.770427in}{1.735324in}}%
\pgfpathcurveto{\pgfqpoint{2.778241in}{1.727511in}}{\pgfqpoint{2.788840in}{1.723120in}}{\pgfqpoint{2.799890in}{1.723120in}}%
\pgfpathclose%
\pgfusepath{stroke,fill}%
\end{pgfscope}%
\begin{pgfscope}%
\pgfpathrectangle{\pgfqpoint{0.947176in}{0.499691in}}{\pgfqpoint{3.875000in}{2.695000in}}%
\pgfusepath{clip}%
\pgfsetbuttcap%
\pgfsetroundjoin%
\definecolor{currentfill}{rgb}{0.121569,0.466667,0.705882}%
\pgfsetfillcolor{currentfill}%
\pgfsetlinewidth{1.003750pt}%
\definecolor{currentstroke}{rgb}{0.121569,0.466667,0.705882}%
\pgfsetstrokecolor{currentstroke}%
\pgfsetdash{}{0pt}%
\pgfpathmoveto{\pgfqpoint{2.969462in}{1.800821in}}%
\pgfpathcurveto{\pgfqpoint{2.980512in}{1.800821in}}{\pgfqpoint{2.991111in}{1.805211in}}{\pgfqpoint{2.998925in}{1.813025in}}%
\pgfpathcurveto{\pgfqpoint{3.006739in}{1.820839in}}{\pgfqpoint{3.011129in}{1.831438in}}{\pgfqpoint{3.011129in}{1.842488in}}%
\pgfpathcurveto{\pgfqpoint{3.011129in}{1.853538in}}{\pgfqpoint{3.006739in}{1.864137in}}{\pgfqpoint{2.998925in}{1.871951in}}%
\pgfpathcurveto{\pgfqpoint{2.991111in}{1.879764in}}{\pgfqpoint{2.980512in}{1.884154in}}{\pgfqpoint{2.969462in}{1.884154in}}%
\pgfpathcurveto{\pgfqpoint{2.958412in}{1.884154in}}{\pgfqpoint{2.947813in}{1.879764in}}{\pgfqpoint{2.939999in}{1.871951in}}%
\pgfpathcurveto{\pgfqpoint{2.932186in}{1.864137in}}{\pgfqpoint{2.927796in}{1.853538in}}{\pgfqpoint{2.927796in}{1.842488in}}%
\pgfpathcurveto{\pgfqpoint{2.927796in}{1.831438in}}{\pgfqpoint{2.932186in}{1.820839in}}{\pgfqpoint{2.939999in}{1.813025in}}%
\pgfpathcurveto{\pgfqpoint{2.947813in}{1.805211in}}{\pgfqpoint{2.958412in}{1.800821in}}{\pgfqpoint{2.969462in}{1.800821in}}%
\pgfpathclose%
\pgfusepath{stroke,fill}%
\end{pgfscope}%
\begin{pgfscope}%
\pgfpathrectangle{\pgfqpoint{0.947176in}{0.499691in}}{\pgfqpoint{3.875000in}{2.695000in}}%
\pgfusepath{clip}%
\pgfsetbuttcap%
\pgfsetroundjoin%
\definecolor{currentfill}{rgb}{0.121569,0.466667,0.705882}%
\pgfsetfillcolor{currentfill}%
\pgfsetlinewidth{1.003750pt}%
\definecolor{currentstroke}{rgb}{0.121569,0.466667,0.705882}%
\pgfsetstrokecolor{currentstroke}%
\pgfsetdash{}{0pt}%
\pgfpathmoveto{\pgfqpoint{3.139035in}{2.068457in}}%
\pgfpathcurveto{\pgfqpoint{3.150085in}{2.068457in}}{\pgfqpoint{3.160684in}{2.072847in}}{\pgfqpoint{3.168497in}{2.080661in}}%
\pgfpathcurveto{\pgfqpoint{3.176311in}{2.088475in}}{\pgfqpoint{3.180701in}{2.099074in}}{\pgfqpoint{3.180701in}{2.110124in}}%
\pgfpathcurveto{\pgfqpoint{3.180701in}{2.121174in}}{\pgfqpoint{3.176311in}{2.131773in}}{\pgfqpoint{3.168497in}{2.139586in}}%
\pgfpathcurveto{\pgfqpoint{3.160684in}{2.147400in}}{\pgfqpoint{3.150085in}{2.151790in}}{\pgfqpoint{3.139035in}{2.151790in}}%
\pgfpathcurveto{\pgfqpoint{3.127984in}{2.151790in}}{\pgfqpoint{3.117385in}{2.147400in}}{\pgfqpoint{3.109572in}{2.139586in}}%
\pgfpathcurveto{\pgfqpoint{3.101758in}{2.131773in}}{\pgfqpoint{3.097368in}{2.121174in}}{\pgfqpoint{3.097368in}{2.110124in}}%
\pgfpathcurveto{\pgfqpoint{3.097368in}{2.099074in}}{\pgfqpoint{3.101758in}{2.088475in}}{\pgfqpoint{3.109572in}{2.080661in}}%
\pgfpathcurveto{\pgfqpoint{3.117385in}{2.072847in}}{\pgfqpoint{3.127984in}{2.068457in}}{\pgfqpoint{3.139035in}{2.068457in}}%
\pgfpathclose%
\pgfusepath{stroke,fill}%
\end{pgfscope}%
\begin{pgfscope}%
\pgfpathrectangle{\pgfqpoint{0.947176in}{0.499691in}}{\pgfqpoint{3.875000in}{2.695000in}}%
\pgfusepath{clip}%
\pgfsetbuttcap%
\pgfsetroundjoin%
\definecolor{currentfill}{rgb}{0.121569,0.466667,0.705882}%
\pgfsetfillcolor{currentfill}%
\pgfsetlinewidth{1.003750pt}%
\definecolor{currentstroke}{rgb}{0.121569,0.466667,0.705882}%
\pgfsetstrokecolor{currentstroke}%
\pgfsetdash{}{0pt}%
\pgfpathmoveto{\pgfqpoint{3.308607in}{2.215225in}}%
\pgfpathcurveto{\pgfqpoint{3.319657in}{2.215225in}}{\pgfqpoint{3.330256in}{2.219615in}}{\pgfqpoint{3.338070in}{2.227429in}}%
\pgfpathcurveto{\pgfqpoint{3.345883in}{2.235243in}}{\pgfqpoint{3.350273in}{2.245842in}}{\pgfqpoint{3.350273in}{2.256892in}}%
\pgfpathcurveto{\pgfqpoint{3.350273in}{2.267942in}}{\pgfqpoint{3.345883in}{2.278541in}}{\pgfqpoint{3.338070in}{2.286355in}}%
\pgfpathcurveto{\pgfqpoint{3.330256in}{2.294168in}}{\pgfqpoint{3.319657in}{2.298558in}}{\pgfqpoint{3.308607in}{2.298558in}}%
\pgfpathcurveto{\pgfqpoint{3.297557in}{2.298558in}}{\pgfqpoint{3.286958in}{2.294168in}}{\pgfqpoint{3.279144in}{2.286355in}}%
\pgfpathcurveto{\pgfqpoint{3.271330in}{2.278541in}}{\pgfqpoint{3.266940in}{2.267942in}}{\pgfqpoint{3.266940in}{2.256892in}}%
\pgfpathcurveto{\pgfqpoint{3.266940in}{2.245842in}}{\pgfqpoint{3.271330in}{2.235243in}}{\pgfqpoint{3.279144in}{2.227429in}}%
\pgfpathcurveto{\pgfqpoint{3.286958in}{2.219615in}}{\pgfqpoint{3.297557in}{2.215225in}}{\pgfqpoint{3.308607in}{2.215225in}}%
\pgfpathclose%
\pgfusepath{stroke,fill}%
\end{pgfscope}%
\begin{pgfscope}%
\pgfpathrectangle{\pgfqpoint{0.947176in}{0.499691in}}{\pgfqpoint{3.875000in}{2.695000in}}%
\pgfusepath{clip}%
\pgfsetbuttcap%
\pgfsetroundjoin%
\definecolor{currentfill}{rgb}{0.121569,0.466667,0.705882}%
\pgfsetfillcolor{currentfill}%
\pgfsetlinewidth{1.003750pt}%
\definecolor{currentstroke}{rgb}{0.121569,0.466667,0.705882}%
\pgfsetstrokecolor{currentstroke}%
\pgfsetdash{}{0pt}%
\pgfpathmoveto{\pgfqpoint{3.478179in}{2.387893in}}%
\pgfpathcurveto{\pgfqpoint{3.489229in}{2.387893in}}{\pgfqpoint{3.499828in}{2.392284in}}{\pgfqpoint{3.507642in}{2.400097in}}%
\pgfpathcurveto{\pgfqpoint{3.515455in}{2.407911in}}{\pgfqpoint{3.519846in}{2.418510in}}{\pgfqpoint{3.519846in}{2.429560in}}%
\pgfpathcurveto{\pgfqpoint{3.519846in}{2.440610in}}{\pgfqpoint{3.515455in}{2.451209in}}{\pgfqpoint{3.507642in}{2.459023in}}%
\pgfpathcurveto{\pgfqpoint{3.499828in}{2.466836in}}{\pgfqpoint{3.489229in}{2.471227in}}{\pgfqpoint{3.478179in}{2.471227in}}%
\pgfpathcurveto{\pgfqpoint{3.467129in}{2.471227in}}{\pgfqpoint{3.456530in}{2.466836in}}{\pgfqpoint{3.448716in}{2.459023in}}%
\pgfpathcurveto{\pgfqpoint{3.440903in}{2.451209in}}{\pgfqpoint{3.436512in}{2.440610in}}{\pgfqpoint{3.436512in}{2.429560in}}%
\pgfpathcurveto{\pgfqpoint{3.436512in}{2.418510in}}{\pgfqpoint{3.440903in}{2.407911in}}{\pgfqpoint{3.448716in}{2.400097in}}%
\pgfpathcurveto{\pgfqpoint{3.456530in}{2.392284in}}{\pgfqpoint{3.467129in}{2.387893in}}{\pgfqpoint{3.478179in}{2.387893in}}%
\pgfpathclose%
\pgfusepath{stroke,fill}%
\end{pgfscope}%
\begin{pgfscope}%
\pgfpathrectangle{\pgfqpoint{0.947176in}{0.499691in}}{\pgfqpoint{3.875000in}{2.695000in}}%
\pgfusepath{clip}%
\pgfsetbuttcap%
\pgfsetroundjoin%
\definecolor{currentfill}{rgb}{0.121569,0.466667,0.705882}%
\pgfsetfillcolor{currentfill}%
\pgfsetlinewidth{1.003750pt}%
\definecolor{currentstroke}{rgb}{0.121569,0.466667,0.705882}%
\pgfsetstrokecolor{currentstroke}%
\pgfsetdash{}{0pt}%
\pgfpathmoveto{\pgfqpoint{3.647751in}{2.405160in}}%
\pgfpathcurveto{\pgfqpoint{3.658801in}{2.405160in}}{\pgfqpoint{3.669401in}{2.409550in}}{\pgfqpoint{3.677214in}{2.417364in}}%
\pgfpathcurveto{\pgfqpoint{3.685028in}{2.425178in}}{\pgfqpoint{3.689418in}{2.435777in}}{\pgfqpoint{3.689418in}{2.446827in}}%
\pgfpathcurveto{\pgfqpoint{3.689418in}{2.457877in}}{\pgfqpoint{3.685028in}{2.468476in}}{\pgfqpoint{3.677214in}{2.476290in}}%
\pgfpathcurveto{\pgfqpoint{3.669401in}{2.484103in}}{\pgfqpoint{3.658801in}{2.488494in}}{\pgfqpoint{3.647751in}{2.488494in}}%
\pgfpathcurveto{\pgfqpoint{3.636701in}{2.488494in}}{\pgfqpoint{3.626102in}{2.484103in}}{\pgfqpoint{3.618289in}{2.476290in}}%
\pgfpathcurveto{\pgfqpoint{3.610475in}{2.468476in}}{\pgfqpoint{3.606085in}{2.457877in}}{\pgfqpoint{3.606085in}{2.446827in}}%
\pgfpathcurveto{\pgfqpoint{3.606085in}{2.435777in}}{\pgfqpoint{3.610475in}{2.425178in}}{\pgfqpoint{3.618289in}{2.417364in}}%
\pgfpathcurveto{\pgfqpoint{3.626102in}{2.409550in}}{\pgfqpoint{3.636701in}{2.405160in}}{\pgfqpoint{3.647751in}{2.405160in}}%
\pgfpathclose%
\pgfusepath{stroke,fill}%
\end{pgfscope}%
\begin{pgfscope}%
\pgfpathrectangle{\pgfqpoint{0.947176in}{0.499691in}}{\pgfqpoint{3.875000in}{2.695000in}}%
\pgfusepath{clip}%
\pgfsetbuttcap%
\pgfsetroundjoin%
\definecolor{currentfill}{rgb}{0.121569,0.466667,0.705882}%
\pgfsetfillcolor{currentfill}%
\pgfsetlinewidth{1.003750pt}%
\definecolor{currentstroke}{rgb}{0.121569,0.466667,0.705882}%
\pgfsetstrokecolor{currentstroke}%
\pgfsetdash{}{0pt}%
\pgfpathmoveto{\pgfqpoint{3.817324in}{2.491494in}}%
\pgfpathcurveto{\pgfqpoint{3.828374in}{2.491494in}}{\pgfqpoint{3.838973in}{2.495885in}}{\pgfqpoint{3.846786in}{2.503698in}}%
\pgfpathcurveto{\pgfqpoint{3.854600in}{2.511512in}}{\pgfqpoint{3.858990in}{2.522111in}}{\pgfqpoint{3.858990in}{2.533161in}}%
\pgfpathcurveto{\pgfqpoint{3.858990in}{2.544211in}}{\pgfqpoint{3.854600in}{2.554810in}}{\pgfqpoint{3.846786in}{2.562624in}}%
\pgfpathcurveto{\pgfqpoint{3.838973in}{2.570437in}}{\pgfqpoint{3.828374in}{2.574828in}}{\pgfqpoint{3.817324in}{2.574828in}}%
\pgfpathcurveto{\pgfqpoint{3.806273in}{2.574828in}}{\pgfqpoint{3.795674in}{2.570437in}}{\pgfqpoint{3.787861in}{2.562624in}}%
\pgfpathcurveto{\pgfqpoint{3.780047in}{2.554810in}}{\pgfqpoint{3.775657in}{2.544211in}}{\pgfqpoint{3.775657in}{2.533161in}}%
\pgfpathcurveto{\pgfqpoint{3.775657in}{2.522111in}}{\pgfqpoint{3.780047in}{2.511512in}}{\pgfqpoint{3.787861in}{2.503698in}}%
\pgfpathcurveto{\pgfqpoint{3.795674in}{2.495885in}}{\pgfqpoint{3.806273in}{2.491494in}}{\pgfqpoint{3.817324in}{2.491494in}}%
\pgfpathclose%
\pgfusepath{stroke,fill}%
\end{pgfscope}%
\begin{pgfscope}%
\pgfpathrectangle{\pgfqpoint{0.947176in}{0.499691in}}{\pgfqpoint{3.875000in}{2.695000in}}%
\pgfusepath{clip}%
\pgfsetbuttcap%
\pgfsetroundjoin%
\definecolor{currentfill}{rgb}{0.121569,0.466667,0.705882}%
\pgfsetfillcolor{currentfill}%
\pgfsetlinewidth{1.003750pt}%
\definecolor{currentstroke}{rgb}{0.121569,0.466667,0.705882}%
\pgfsetstrokecolor{currentstroke}%
\pgfsetdash{}{0pt}%
\pgfpathmoveto{\pgfqpoint{3.902110in}{2.543295in}}%
\pgfpathcurveto{\pgfqpoint{3.913160in}{2.543295in}}{\pgfqpoint{3.923759in}{2.547685in}}{\pgfqpoint{3.931573in}{2.555499in}}%
\pgfpathcurveto{\pgfqpoint{3.939386in}{2.563312in}}{\pgfqpoint{3.943776in}{2.573911in}}{\pgfqpoint{3.943776in}{2.584962in}}%
\pgfpathcurveto{\pgfqpoint{3.943776in}{2.596012in}}{\pgfqpoint{3.939386in}{2.606611in}}{\pgfqpoint{3.931573in}{2.614424in}}%
\pgfpathcurveto{\pgfqpoint{3.923759in}{2.622238in}}{\pgfqpoint{3.913160in}{2.626628in}}{\pgfqpoint{3.902110in}{2.626628in}}%
\pgfpathcurveto{\pgfqpoint{3.891060in}{2.626628in}}{\pgfqpoint{3.880461in}{2.622238in}}{\pgfqpoint{3.872647in}{2.614424in}}%
\pgfpathcurveto{\pgfqpoint{3.864833in}{2.606611in}}{\pgfqpoint{3.860443in}{2.596012in}}{\pgfqpoint{3.860443in}{2.584962in}}%
\pgfpathcurveto{\pgfqpoint{3.860443in}{2.573911in}}{\pgfqpoint{3.864833in}{2.563312in}}{\pgfqpoint{3.872647in}{2.555499in}}%
\pgfpathcurveto{\pgfqpoint{3.880461in}{2.547685in}}{\pgfqpoint{3.891060in}{2.543295in}}{\pgfqpoint{3.902110in}{2.543295in}}%
\pgfpathclose%
\pgfusepath{stroke,fill}%
\end{pgfscope}%
\begin{pgfscope}%
\pgfpathrectangle{\pgfqpoint{0.947176in}{0.499691in}}{\pgfqpoint{3.875000in}{2.695000in}}%
\pgfusepath{clip}%
\pgfsetbuttcap%
\pgfsetroundjoin%
\definecolor{currentfill}{rgb}{0.121569,0.466667,0.705882}%
\pgfsetfillcolor{currentfill}%
\pgfsetlinewidth{1.003750pt}%
\definecolor{currentstroke}{rgb}{0.121569,0.466667,0.705882}%
\pgfsetstrokecolor{currentstroke}%
\pgfsetdash{}{0pt}%
\pgfpathmoveto{\pgfqpoint{4.071682in}{2.560562in}}%
\pgfpathcurveto{\pgfqpoint{4.082732in}{2.560562in}}{\pgfqpoint{4.093331in}{2.564952in}}{\pgfqpoint{4.101145in}{2.572766in}}%
\pgfpathcurveto{\pgfqpoint{4.108958in}{2.580579in}}{\pgfqpoint{4.113349in}{2.591178in}}{\pgfqpoint{4.113349in}{2.602228in}}%
\pgfpathcurveto{\pgfqpoint{4.113349in}{2.613278in}}{\pgfqpoint{4.108958in}{2.623878in}}{\pgfqpoint{4.101145in}{2.631691in}}%
\pgfpathcurveto{\pgfqpoint{4.093331in}{2.639505in}}{\pgfqpoint{4.082732in}{2.643895in}}{\pgfqpoint{4.071682in}{2.643895in}}%
\pgfpathcurveto{\pgfqpoint{4.060632in}{2.643895in}}{\pgfqpoint{4.050033in}{2.639505in}}{\pgfqpoint{4.042219in}{2.631691in}}%
\pgfpathcurveto{\pgfqpoint{4.034406in}{2.623878in}}{\pgfqpoint{4.030015in}{2.613278in}}{\pgfqpoint{4.030015in}{2.602228in}}%
\pgfpathcurveto{\pgfqpoint{4.030015in}{2.591178in}}{\pgfqpoint{4.034406in}{2.580579in}}{\pgfqpoint{4.042219in}{2.572766in}}%
\pgfpathcurveto{\pgfqpoint{4.050033in}{2.564952in}}{\pgfqpoint{4.060632in}{2.560562in}}{\pgfqpoint{4.071682in}{2.560562in}}%
\pgfpathclose%
\pgfusepath{stroke,fill}%
\end{pgfscope}%
\begin{pgfscope}%
\pgfpathrectangle{\pgfqpoint{0.947176in}{0.499691in}}{\pgfqpoint{3.875000in}{2.695000in}}%
\pgfusepath{clip}%
\pgfsetbuttcap%
\pgfsetroundjoin%
\definecolor{currentfill}{rgb}{0.121569,0.466667,0.705882}%
\pgfsetfillcolor{currentfill}%
\pgfsetlinewidth{1.003750pt}%
\definecolor{currentstroke}{rgb}{0.121569,0.466667,0.705882}%
\pgfsetstrokecolor{currentstroke}%
\pgfsetdash{}{0pt}%
\pgfpathmoveto{\pgfqpoint{4.241254in}{2.741863in}}%
\pgfpathcurveto{\pgfqpoint{4.252304in}{2.741863in}}{\pgfqpoint{4.262903in}{2.746254in}}{\pgfqpoint{4.270717in}{2.754067in}}%
\pgfpathcurveto{\pgfqpoint{4.278531in}{2.761881in}}{\pgfqpoint{4.282921in}{2.772480in}}{\pgfqpoint{4.282921in}{2.783530in}}%
\pgfpathcurveto{\pgfqpoint{4.282921in}{2.794580in}}{\pgfqpoint{4.278531in}{2.805179in}}{\pgfqpoint{4.270717in}{2.812993in}}%
\pgfpathcurveto{\pgfqpoint{4.262903in}{2.820806in}}{\pgfqpoint{4.252304in}{2.825197in}}{\pgfqpoint{4.241254in}{2.825197in}}%
\pgfpathcurveto{\pgfqpoint{4.230204in}{2.825197in}}{\pgfqpoint{4.219605in}{2.820806in}}{\pgfqpoint{4.211792in}{2.812993in}}%
\pgfpathcurveto{\pgfqpoint{4.203978in}{2.805179in}}{\pgfqpoint{4.199588in}{2.794580in}}{\pgfqpoint{4.199588in}{2.783530in}}%
\pgfpathcurveto{\pgfqpoint{4.199588in}{2.772480in}}{\pgfqpoint{4.203978in}{2.761881in}}{\pgfqpoint{4.211792in}{2.754067in}}%
\pgfpathcurveto{\pgfqpoint{4.219605in}{2.746254in}}{\pgfqpoint{4.230204in}{2.741863in}}{\pgfqpoint{4.241254in}{2.741863in}}%
\pgfpathclose%
\pgfusepath{stroke,fill}%
\end{pgfscope}%
\begin{pgfscope}%
\pgfpathrectangle{\pgfqpoint{0.947176in}{0.499691in}}{\pgfqpoint{3.875000in}{2.695000in}}%
\pgfusepath{clip}%
\pgfsetbuttcap%
\pgfsetroundjoin%
\definecolor{currentfill}{rgb}{0.121569,0.466667,0.705882}%
\pgfsetfillcolor{currentfill}%
\pgfsetlinewidth{1.003750pt}%
\definecolor{currentstroke}{rgb}{0.121569,0.466667,0.705882}%
\pgfsetstrokecolor{currentstroke}%
\pgfsetdash{}{0pt}%
\pgfpathmoveto{\pgfqpoint{4.326040in}{2.785030in}}%
\pgfpathcurveto{\pgfqpoint{4.337091in}{2.785030in}}{\pgfqpoint{4.347690in}{2.789421in}}{\pgfqpoint{4.355503in}{2.797234in}}%
\pgfpathcurveto{\pgfqpoint{4.363317in}{2.805048in}}{\pgfqpoint{4.367707in}{2.815647in}}{\pgfqpoint{4.367707in}{2.826697in}}%
\pgfpathcurveto{\pgfqpoint{4.367707in}{2.837747in}}{\pgfqpoint{4.363317in}{2.848346in}}{\pgfqpoint{4.355503in}{2.856160in}}%
\pgfpathcurveto{\pgfqpoint{4.347690in}{2.863974in}}{\pgfqpoint{4.337091in}{2.868364in}}{\pgfqpoint{4.326040in}{2.868364in}}%
\pgfpathcurveto{\pgfqpoint{4.314990in}{2.868364in}}{\pgfqpoint{4.304391in}{2.863974in}}{\pgfqpoint{4.296578in}{2.856160in}}%
\pgfpathcurveto{\pgfqpoint{4.288764in}{2.848346in}}{\pgfqpoint{4.284374in}{2.837747in}}{\pgfqpoint{4.284374in}{2.826697in}}%
\pgfpathcurveto{\pgfqpoint{4.284374in}{2.815647in}}{\pgfqpoint{4.288764in}{2.805048in}}{\pgfqpoint{4.296578in}{2.797234in}}%
\pgfpathcurveto{\pgfqpoint{4.304391in}{2.789421in}}{\pgfqpoint{4.314990in}{2.785030in}}{\pgfqpoint{4.326040in}{2.785030in}}%
\pgfpathclose%
\pgfusepath{stroke,fill}%
\end{pgfscope}%
\begin{pgfscope}%
\pgfpathrectangle{\pgfqpoint{0.947176in}{0.499691in}}{\pgfqpoint{3.875000in}{2.695000in}}%
\pgfusepath{clip}%
\pgfsetbuttcap%
\pgfsetroundjoin%
\definecolor{currentfill}{rgb}{0.121569,0.466667,0.705882}%
\pgfsetfillcolor{currentfill}%
\pgfsetlinewidth{1.003750pt}%
\definecolor{currentstroke}{rgb}{0.121569,0.466667,0.705882}%
\pgfsetstrokecolor{currentstroke}%
\pgfsetdash{}{0pt}%
\pgfpathmoveto{\pgfqpoint{4.410827in}{2.897265in}}%
\pgfpathcurveto{\pgfqpoint{4.421877in}{2.897265in}}{\pgfqpoint{4.432476in}{2.901655in}}{\pgfqpoint{4.440289in}{2.909469in}}%
\pgfpathcurveto{\pgfqpoint{4.448103in}{2.917282in}}{\pgfqpoint{4.452493in}{2.927881in}}{\pgfqpoint{4.452493in}{2.938932in}}%
\pgfpathcurveto{\pgfqpoint{4.452493in}{2.949982in}}{\pgfqpoint{4.448103in}{2.960581in}}{\pgfqpoint{4.440289in}{2.968394in}}%
\pgfpathcurveto{\pgfqpoint{4.432476in}{2.976208in}}{\pgfqpoint{4.421877in}{2.980598in}}{\pgfqpoint{4.410827in}{2.980598in}}%
\pgfpathcurveto{\pgfqpoint{4.399776in}{2.980598in}}{\pgfqpoint{4.389177in}{2.976208in}}{\pgfqpoint{4.381364in}{2.968394in}}%
\pgfpathcurveto{\pgfqpoint{4.373550in}{2.960581in}}{\pgfqpoint{4.369160in}{2.949982in}}{\pgfqpoint{4.369160in}{2.938932in}}%
\pgfpathcurveto{\pgfqpoint{4.369160in}{2.927881in}}{\pgfqpoint{4.373550in}{2.917282in}}{\pgfqpoint{4.381364in}{2.909469in}}%
\pgfpathcurveto{\pgfqpoint{4.389177in}{2.901655in}}{\pgfqpoint{4.399776in}{2.897265in}}{\pgfqpoint{4.410827in}{2.897265in}}%
\pgfpathclose%
\pgfusepath{stroke,fill}%
\end{pgfscope}%
\begin{pgfscope}%
\pgfpathrectangle{\pgfqpoint{0.947176in}{0.499691in}}{\pgfqpoint{3.875000in}{2.695000in}}%
\pgfusepath{clip}%
\pgfsetbuttcap%
\pgfsetroundjoin%
\definecolor{currentfill}{rgb}{0.121569,0.466667,0.705882}%
\pgfsetfillcolor{currentfill}%
\pgfsetlinewidth{1.003750pt}%
\definecolor{currentstroke}{rgb}{0.121569,0.466667,0.705882}%
\pgfsetstrokecolor{currentstroke}%
\pgfsetdash{}{0pt}%
\pgfpathmoveto{\pgfqpoint{4.580399in}{2.966332in}}%
\pgfpathcurveto{\pgfqpoint{4.591449in}{2.966332in}}{\pgfqpoint{4.602048in}{2.970722in}}{\pgfqpoint{4.609862in}{2.978536in}}%
\pgfpathcurveto{\pgfqpoint{4.617675in}{2.986350in}}{\pgfqpoint{4.622066in}{2.996949in}}{\pgfqpoint{4.622066in}{3.007999in}}%
\pgfpathcurveto{\pgfqpoint{4.622066in}{3.019049in}}{\pgfqpoint{4.617675in}{3.029648in}}{\pgfqpoint{4.609862in}{3.037462in}}%
\pgfpathcurveto{\pgfqpoint{4.602048in}{3.045275in}}{\pgfqpoint{4.591449in}{3.049666in}}{\pgfqpoint{4.580399in}{3.049666in}}%
\pgfpathcurveto{\pgfqpoint{4.569349in}{3.049666in}}{\pgfqpoint{4.558750in}{3.045275in}}{\pgfqpoint{4.550936in}{3.037462in}}%
\pgfpathcurveto{\pgfqpoint{4.543122in}{3.029648in}}{\pgfqpoint{4.538732in}{3.019049in}}{\pgfqpoint{4.538732in}{3.007999in}}%
\pgfpathcurveto{\pgfqpoint{4.538732in}{2.996949in}}{\pgfqpoint{4.543122in}{2.986350in}}{\pgfqpoint{4.550936in}{2.978536in}}%
\pgfpathcurveto{\pgfqpoint{4.558750in}{2.970722in}}{\pgfqpoint{4.569349in}{2.966332in}}{\pgfqpoint{4.580399in}{2.966332in}}%
\pgfpathclose%
\pgfusepath{stroke,fill}%
\end{pgfscope}%
\begin{pgfscope}%
\pgfsetrectcap%
\pgfsetmiterjoin%
\pgfsetlinewidth{0.803000pt}%
\definecolor{currentstroke}{rgb}{0.000000,0.000000,0.000000}%
\pgfsetstrokecolor{currentstroke}%
\pgfsetdash{}{0pt}%
\pgfpathmoveto{\pgfqpoint{0.947176in}{0.499691in}}%
\pgfpathlineto{\pgfqpoint{0.947176in}{3.194691in}}%
\pgfusepath{stroke}%
\end{pgfscope}%
\begin{pgfscope}%
\pgfsetrectcap%
\pgfsetmiterjoin%
\pgfsetlinewidth{0.803000pt}%
\definecolor{currentstroke}{rgb}{0.000000,0.000000,0.000000}%
\pgfsetstrokecolor{currentstroke}%
\pgfsetdash{}{0pt}%
\pgfpathmoveto{\pgfqpoint{4.822176in}{0.499691in}}%
\pgfpathlineto{\pgfqpoint{4.822176in}{3.194691in}}%
\pgfusepath{stroke}%
\end{pgfscope}%
\begin{pgfscope}%
\pgfsetrectcap%
\pgfsetmiterjoin%
\pgfsetlinewidth{0.803000pt}%
\definecolor{currentstroke}{rgb}{0.000000,0.000000,0.000000}%
\pgfsetstrokecolor{currentstroke}%
\pgfsetdash{}{0pt}%
\pgfpathmoveto{\pgfqpoint{0.947176in}{0.499691in}}%
\pgfpathlineto{\pgfqpoint{4.822176in}{0.499691in}}%
\pgfusepath{stroke}%
\end{pgfscope}%
\begin{pgfscope}%
\pgfsetrectcap%
\pgfsetmiterjoin%
\pgfsetlinewidth{0.803000pt}%
\definecolor{currentstroke}{rgb}{0.000000,0.000000,0.000000}%
\pgfsetstrokecolor{currentstroke}%
\pgfsetdash{}{0pt}%
\pgfpathmoveto{\pgfqpoint{0.947176in}{3.194691in}}%
\pgfpathlineto{\pgfqpoint{4.822176in}{3.194691in}}%
\pgfusepath{stroke}%
\end{pgfscope}%
\begin{pgfscope}%
\pgfpathrectangle{\pgfqpoint{0.947176in}{0.499691in}}{\pgfqpoint{3.875000in}{2.695000in}}%
\pgfusepath{clip}%
\pgfsetbuttcap%
\pgfsetroundjoin%
\definecolor{currentfill}{rgb}{1.000000,0.388235,0.278431}%
\pgfsetfillcolor{currentfill}%
\pgfsetlinewidth{1.003750pt}%
\definecolor{currentstroke}{rgb}{1.000000,0.388235,0.278431}%
\pgfsetstrokecolor{currentstroke}%
\pgfsetdash{}{0pt}%
\pgfpathmoveto{\pgfqpoint{3.690828in}{2.387893in}}%
\pgfpathcurveto{\pgfqpoint{3.701878in}{2.387893in}}{\pgfqpoint{3.712477in}{2.392284in}}{\pgfqpoint{3.720291in}{2.400097in}}%
\pgfpathcurveto{\pgfqpoint{3.728105in}{2.407911in}}{\pgfqpoint{3.732495in}{2.418510in}}{\pgfqpoint{3.732495in}{2.429560in}}%
\pgfpathcurveto{\pgfqpoint{3.732495in}{2.440610in}}{\pgfqpoint{3.728105in}{2.451209in}}{\pgfqpoint{3.720291in}{2.459023in}}%
\pgfpathcurveto{\pgfqpoint{3.712477in}{2.466836in}}{\pgfqpoint{3.701878in}{2.471227in}}{\pgfqpoint{3.690828in}{2.471227in}}%
\pgfpathcurveto{\pgfqpoint{3.679778in}{2.471227in}}{\pgfqpoint{3.669179in}{2.466836in}}{\pgfqpoint{3.661365in}{2.459023in}}%
\pgfpathcurveto{\pgfqpoint{3.653552in}{2.451209in}}{\pgfqpoint{3.649162in}{2.440610in}}{\pgfqpoint{3.649162in}{2.429560in}}%
\pgfpathcurveto{\pgfqpoint{3.649162in}{2.418510in}}{\pgfqpoint{3.653552in}{2.407911in}}{\pgfqpoint{3.661365in}{2.400097in}}%
\pgfpathcurveto{\pgfqpoint{3.669179in}{2.392284in}}{\pgfqpoint{3.679778in}{2.387893in}}{\pgfqpoint{3.690828in}{2.387893in}}%
\pgfpathclose%
\pgfusepath{stroke,fill}%
\end{pgfscope}%
\end{pgfpicture}%
\makeatother%
\endgroup%

    \caption{Primer intento de representación de $\varphi$ frente a f (escala logarítmica)}
  \end{figure}

  Probemos ahora a representar todos los puntos obtenidos, incluyendo aquellos que se aproximan más a los extremos.

  \begin{figure}[H]
    %\centering
    \hspace{2.5em} %% Creator: Matplotlib, PGF backend
%%
%% To include the figure in your LaTeX document, write
%%   \input{<filename>.pgf}
%%
%% Make sure the required packages are loaded in your preamble
%%   \usepackage{pgf}
%%
%% Figures using additional raster images can only be included by \input if
%% they are in the same directory as the main LaTeX file. For loading figures
%% from other directories you can use the `import` package
%%   \usepackage{import}
%% and then include the figures with
%%   \import{<path to file>}{<filename>.pgf}
%%
%% Matplotlib used the following preamble
%%
\begingroup%
\makeatletter%
\begin{pgfpicture}%
\pgfpathrectangle{\pgfpointorigin}{\pgfqpoint{4.991621in}{3.294691in}}%
\pgfusepath{use as bounding box, clip}%
\begin{pgfscope}%
\pgfsetbuttcap%
\pgfsetmiterjoin%
\definecolor{currentfill}{rgb}{1.000000,1.000000,1.000000}%
\pgfsetfillcolor{currentfill}%
\pgfsetlinewidth{0.000000pt}%
\definecolor{currentstroke}{rgb}{1.000000,1.000000,1.000000}%
\pgfsetstrokecolor{currentstroke}%
\pgfsetdash{}{0pt}%
\pgfpathmoveto{\pgfqpoint{0.000000in}{0.000000in}}%
\pgfpathlineto{\pgfqpoint{4.991621in}{0.000000in}}%
\pgfpathlineto{\pgfqpoint{4.991621in}{3.294691in}}%
\pgfpathlineto{\pgfqpoint{0.000000in}{3.294691in}}%
\pgfpathclose%
\pgfusepath{fill}%
\end{pgfscope}%
\begin{pgfscope}%
\pgfsetbuttcap%
\pgfsetmiterjoin%
\definecolor{currentfill}{rgb}{1.000000,1.000000,1.000000}%
\pgfsetfillcolor{currentfill}%
\pgfsetlinewidth{0.000000pt}%
\definecolor{currentstroke}{rgb}{0.000000,0.000000,0.000000}%
\pgfsetstrokecolor{currentstroke}%
\pgfsetstrokeopacity{0.000000}%
\pgfsetdash{}{0pt}%
\pgfpathmoveto{\pgfqpoint{1.016621in}{0.499691in}}%
\pgfpathlineto{\pgfqpoint{4.891621in}{0.499691in}}%
\pgfpathlineto{\pgfqpoint{4.891621in}{3.194691in}}%
\pgfpathlineto{\pgfqpoint{1.016621in}{3.194691in}}%
\pgfpathclose%
\pgfusepath{fill}%
\end{pgfscope}%
\begin{pgfscope}%
\pgfpathrectangle{\pgfqpoint{1.016621in}{0.499691in}}{\pgfqpoint{3.875000in}{2.695000in}}%
\pgfusepath{clip}%
\pgfsetbuttcap%
\pgfsetroundjoin%
\definecolor{currentfill}{rgb}{0.117647,0.564706,1.000000}%
\pgfsetfillcolor{currentfill}%
\pgfsetlinewidth{1.003750pt}%
\definecolor{currentstroke}{rgb}{0.117647,0.564706,1.000000}%
\pgfsetstrokecolor{currentstroke}%
\pgfsetdash{}{0pt}%
\pgfpathmoveto{\pgfqpoint{1.197151in}{0.590219in}}%
\pgfpathcurveto{\pgfqpoint{1.208201in}{0.590219in}}{\pgfqpoint{1.218800in}{0.594609in}}{\pgfqpoint{1.226613in}{0.602423in}}%
\pgfpathcurveto{\pgfqpoint{1.234427in}{0.610237in}}{\pgfqpoint{1.238817in}{0.620836in}}{\pgfqpoint{1.238817in}{0.631886in}}%
\pgfpathcurveto{\pgfqpoint{1.238817in}{0.642936in}}{\pgfqpoint{1.234427in}{0.653535in}}{\pgfqpoint{1.226613in}{0.661349in}}%
\pgfpathcurveto{\pgfqpoint{1.218800in}{0.669162in}}{\pgfqpoint{1.208201in}{0.673552in}}{\pgfqpoint{1.197151in}{0.673552in}}%
\pgfpathcurveto{\pgfqpoint{1.186101in}{0.673552in}}{\pgfqpoint{1.175501in}{0.669162in}}{\pgfqpoint{1.167688in}{0.661349in}}%
\pgfpathcurveto{\pgfqpoint{1.159874in}{0.653535in}}{\pgfqpoint{1.155484in}{0.642936in}}{\pgfqpoint{1.155484in}{0.631886in}}%
\pgfpathcurveto{\pgfqpoint{1.155484in}{0.620836in}}{\pgfqpoint{1.159874in}{0.610237in}}{\pgfqpoint{1.167688in}{0.602423in}}%
\pgfpathcurveto{\pgfqpoint{1.175501in}{0.594609in}}{\pgfqpoint{1.186101in}{0.590219in}}{\pgfqpoint{1.197151in}{0.590219in}}%
\pgfpathclose%
\pgfusepath{stroke,fill}%
\end{pgfscope}%
\begin{pgfscope}%
\pgfpathrectangle{\pgfqpoint{1.016621in}{0.499691in}}{\pgfqpoint{3.875000in}{2.695000in}}%
\pgfusepath{clip}%
\pgfsetbuttcap%
\pgfsetroundjoin%
\definecolor{currentfill}{rgb}{0.117647,0.564706,1.000000}%
\pgfsetfillcolor{currentfill}%
\pgfsetlinewidth{1.003750pt}%
\definecolor{currentstroke}{rgb}{0.117647,0.564706,1.000000}%
\pgfsetstrokecolor{currentstroke}%
\pgfsetdash{}{0pt}%
\pgfpathmoveto{\pgfqpoint{2.165657in}{0.716336in}}%
\pgfpathcurveto{\pgfqpoint{2.176707in}{0.716336in}}{\pgfqpoint{2.187306in}{0.720726in}}{\pgfqpoint{2.195120in}{0.728540in}}%
\pgfpathcurveto{\pgfqpoint{2.202934in}{0.736353in}}{\pgfqpoint{2.207324in}{0.746952in}}{\pgfqpoint{2.207324in}{0.758002in}}%
\pgfpathcurveto{\pgfqpoint{2.207324in}{0.769053in}}{\pgfqpoint{2.202934in}{0.779652in}}{\pgfqpoint{2.195120in}{0.787465in}}%
\pgfpathcurveto{\pgfqpoint{2.187306in}{0.795279in}}{\pgfqpoint{2.176707in}{0.799669in}}{\pgfqpoint{2.165657in}{0.799669in}}%
\pgfpathcurveto{\pgfqpoint{2.154607in}{0.799669in}}{\pgfqpoint{2.144008in}{0.795279in}}{\pgfqpoint{2.136194in}{0.787465in}}%
\pgfpathcurveto{\pgfqpoint{2.128381in}{0.779652in}}{\pgfqpoint{2.123991in}{0.769053in}}{\pgfqpoint{2.123991in}{0.758002in}}%
\pgfpathcurveto{\pgfqpoint{2.123991in}{0.746952in}}{\pgfqpoint{2.128381in}{0.736353in}}{\pgfqpoint{2.136194in}{0.728540in}}%
\pgfpathcurveto{\pgfqpoint{2.144008in}{0.720726in}}{\pgfqpoint{2.154607in}{0.716336in}}{\pgfqpoint{2.165657in}{0.716336in}}%
\pgfpathclose%
\pgfusepath{stroke,fill}%
\end{pgfscope}%
\begin{pgfscope}%
\pgfpathrectangle{\pgfqpoint{1.016621in}{0.499691in}}{\pgfqpoint{3.875000in}{2.695000in}}%
\pgfusepath{clip}%
\pgfsetbuttcap%
\pgfsetroundjoin%
\definecolor{currentfill}{rgb}{0.117647,0.564706,1.000000}%
\pgfsetfillcolor{currentfill}%
\pgfsetlinewidth{1.003750pt}%
\definecolor{currentstroke}{rgb}{0.117647,0.564706,1.000000}%
\pgfsetstrokecolor{currentstroke}%
\pgfsetdash{}{0pt}%
\pgfpathmoveto{\pgfqpoint{2.538160in}{0.962836in}}%
\pgfpathcurveto{\pgfqpoint{2.549210in}{0.962836in}}{\pgfqpoint{2.559809in}{0.967227in}}{\pgfqpoint{2.567622in}{0.975040in}}%
\pgfpathcurveto{\pgfqpoint{2.575436in}{0.982854in}}{\pgfqpoint{2.579826in}{0.993453in}}{\pgfqpoint{2.579826in}{1.004503in}}%
\pgfpathcurveto{\pgfqpoint{2.579826in}{1.015553in}}{\pgfqpoint{2.575436in}{1.026152in}}{\pgfqpoint{2.567622in}{1.033966in}}%
\pgfpathcurveto{\pgfqpoint{2.559809in}{1.041779in}}{\pgfqpoint{2.549210in}{1.046170in}}{\pgfqpoint{2.538160in}{1.046170in}}%
\pgfpathcurveto{\pgfqpoint{2.527110in}{1.046170in}}{\pgfqpoint{2.516511in}{1.041779in}}{\pgfqpoint{2.508697in}{1.033966in}}%
\pgfpathcurveto{\pgfqpoint{2.500883in}{1.026152in}}{\pgfqpoint{2.496493in}{1.015553in}}{\pgfqpoint{2.496493in}{1.004503in}}%
\pgfpathcurveto{\pgfqpoint{2.496493in}{0.993453in}}{\pgfqpoint{2.500883in}{0.982854in}}{\pgfqpoint{2.508697in}{0.975040in}}%
\pgfpathcurveto{\pgfqpoint{2.516511in}{0.967227in}}{\pgfqpoint{2.527110in}{0.962836in}}{\pgfqpoint{2.538160in}{0.962836in}}%
\pgfpathclose%
\pgfusepath{stroke,fill}%
\end{pgfscope}%
\begin{pgfscope}%
\pgfpathrectangle{\pgfqpoint{1.016621in}{0.499691in}}{\pgfqpoint{3.875000in}{2.695000in}}%
\pgfusepath{clip}%
\pgfsetbuttcap%
\pgfsetroundjoin%
\definecolor{currentfill}{rgb}{0.117647,0.564706,1.000000}%
\pgfsetfillcolor{currentfill}%
\pgfsetlinewidth{1.003750pt}%
\definecolor{currentstroke}{rgb}{0.117647,0.564706,1.000000}%
\pgfsetstrokecolor{currentstroke}%
\pgfsetdash{}{0pt}%
\pgfpathmoveto{\pgfqpoint{2.997579in}{1.011563in}}%
\pgfpathcurveto{\pgfqpoint{3.008630in}{1.011563in}}{\pgfqpoint{3.019229in}{1.015953in}}{\pgfqpoint{3.027042in}{1.023767in}}%
\pgfpathcurveto{\pgfqpoint{3.034856in}{1.031581in}}{\pgfqpoint{3.039246in}{1.042180in}}{\pgfqpoint{3.039246in}{1.053230in}}%
\pgfpathcurveto{\pgfqpoint{3.039246in}{1.064280in}}{\pgfqpoint{3.034856in}{1.074879in}}{\pgfqpoint{3.027042in}{1.082693in}}%
\pgfpathcurveto{\pgfqpoint{3.019229in}{1.090506in}}{\pgfqpoint{3.008630in}{1.094897in}}{\pgfqpoint{2.997579in}{1.094897in}}%
\pgfpathcurveto{\pgfqpoint{2.986529in}{1.094897in}}{\pgfqpoint{2.975930in}{1.090506in}}{\pgfqpoint{2.968117in}{1.082693in}}%
\pgfpathcurveto{\pgfqpoint{2.960303in}{1.074879in}}{\pgfqpoint{2.955913in}{1.064280in}}{\pgfqpoint{2.955913in}{1.053230in}}%
\pgfpathcurveto{\pgfqpoint{2.955913in}{1.042180in}}{\pgfqpoint{2.960303in}{1.031581in}}{\pgfqpoint{2.968117in}{1.023767in}}%
\pgfpathcurveto{\pgfqpoint{2.975930in}{1.015953in}}{\pgfqpoint{2.986529in}{1.011563in}}{\pgfqpoint{2.997579in}{1.011563in}}%
\pgfpathclose%
\pgfusepath{stroke,fill}%
\end{pgfscope}%
\begin{pgfscope}%
\pgfpathrectangle{\pgfqpoint{1.016621in}{0.499691in}}{\pgfqpoint{3.875000in}{2.695000in}}%
\pgfusepath{clip}%
\pgfsetbuttcap%
\pgfsetroundjoin%
\definecolor{currentfill}{rgb}{0.117647,0.564706,1.000000}%
\pgfsetfillcolor{currentfill}%
\pgfsetlinewidth{1.003750pt}%
\definecolor{currentstroke}{rgb}{0.117647,0.564706,1.000000}%
\pgfsetstrokecolor{currentstroke}%
\pgfsetdash{}{0pt}%
\pgfpathmoveto{\pgfqpoint{3.183831in}{1.226535in}}%
\pgfpathcurveto{\pgfqpoint{3.194881in}{1.226535in}}{\pgfqpoint{3.205480in}{1.230925in}}{\pgfqpoint{3.213294in}{1.238739in}}%
\pgfpathcurveto{\pgfqpoint{3.221107in}{1.246552in}}{\pgfqpoint{3.225497in}{1.257151in}}{\pgfqpoint{3.225497in}{1.268201in}}%
\pgfpathcurveto{\pgfqpoint{3.225497in}{1.279251in}}{\pgfqpoint{3.221107in}{1.289850in}}{\pgfqpoint{3.213294in}{1.297664in}}%
\pgfpathcurveto{\pgfqpoint{3.205480in}{1.305478in}}{\pgfqpoint{3.194881in}{1.309868in}}{\pgfqpoint{3.183831in}{1.309868in}}%
\pgfpathcurveto{\pgfqpoint{3.172781in}{1.309868in}}{\pgfqpoint{3.162182in}{1.305478in}}{\pgfqpoint{3.154368in}{1.297664in}}%
\pgfpathcurveto{\pgfqpoint{3.146554in}{1.289850in}}{\pgfqpoint{3.142164in}{1.279251in}}{\pgfqpoint{3.142164in}{1.268201in}}%
\pgfpathcurveto{\pgfqpoint{3.142164in}{1.257151in}}{\pgfqpoint{3.146554in}{1.246552in}}{\pgfqpoint{3.154368in}{1.238739in}}%
\pgfpathcurveto{\pgfqpoint{3.162182in}{1.230925in}}{\pgfqpoint{3.172781in}{1.226535in}}{\pgfqpoint{3.183831in}{1.226535in}}%
\pgfpathclose%
\pgfusepath{stroke,fill}%
\end{pgfscope}%
\begin{pgfscope}%
\pgfpathrectangle{\pgfqpoint{1.016621in}{0.499691in}}{\pgfqpoint{3.875000in}{2.695000in}}%
\pgfusepath{clip}%
\pgfsetbuttcap%
\pgfsetroundjoin%
\definecolor{currentfill}{rgb}{0.117647,0.564706,1.000000}%
\pgfsetfillcolor{currentfill}%
\pgfsetlinewidth{1.003750pt}%
\definecolor{currentstroke}{rgb}{0.117647,0.564706,1.000000}%
\pgfsetstrokecolor{currentstroke}%
\pgfsetdash{}{0pt}%
\pgfpathmoveto{\pgfqpoint{3.332832in}{1.315390in}}%
\pgfpathcurveto{\pgfqpoint{3.343882in}{1.315390in}}{\pgfqpoint{3.354481in}{1.319780in}}{\pgfqpoint{3.362295in}{1.327593in}}%
\pgfpathcurveto{\pgfqpoint{3.370108in}{1.335407in}}{\pgfqpoint{3.374498in}{1.346006in}}{\pgfqpoint{3.374498in}{1.357056in}}%
\pgfpathcurveto{\pgfqpoint{3.374498in}{1.368106in}}{\pgfqpoint{3.370108in}{1.378705in}}{\pgfqpoint{3.362295in}{1.386519in}}%
\pgfpathcurveto{\pgfqpoint{3.354481in}{1.394333in}}{\pgfqpoint{3.343882in}{1.398723in}}{\pgfqpoint{3.332832in}{1.398723in}}%
\pgfpathcurveto{\pgfqpoint{3.321782in}{1.398723in}}{\pgfqpoint{3.311183in}{1.394333in}}{\pgfqpoint{3.303369in}{1.386519in}}%
\pgfpathcurveto{\pgfqpoint{3.295555in}{1.378705in}}{\pgfqpoint{3.291165in}{1.368106in}}{\pgfqpoint{3.291165in}{1.357056in}}%
\pgfpathcurveto{\pgfqpoint{3.291165in}{1.346006in}}{\pgfqpoint{3.295555in}{1.335407in}}{\pgfqpoint{3.303369in}{1.327593in}}%
\pgfpathcurveto{\pgfqpoint{3.311183in}{1.319780in}}{\pgfqpoint{3.321782in}{1.315390in}}{\pgfqpoint{3.332832in}{1.315390in}}%
\pgfpathclose%
\pgfusepath{stroke,fill}%
\end{pgfscope}%
\begin{pgfscope}%
\pgfpathrectangle{\pgfqpoint{1.016621in}{0.499691in}}{\pgfqpoint{3.875000in}{2.695000in}}%
\pgfusepath{clip}%
\pgfsetbuttcap%
\pgfsetroundjoin%
\definecolor{currentfill}{rgb}{0.117647,0.564706,1.000000}%
\pgfsetfillcolor{currentfill}%
\pgfsetlinewidth{1.003750pt}%
\definecolor{currentstroke}{rgb}{0.117647,0.564706,1.000000}%
\pgfsetstrokecolor{currentstroke}%
\pgfsetdash{}{0pt}%
\pgfpathmoveto{\pgfqpoint{3.382499in}{1.404244in}}%
\pgfpathcurveto{\pgfqpoint{3.393549in}{1.404244in}}{\pgfqpoint{3.404148in}{1.408635in}}{\pgfqpoint{3.411962in}{1.416448in}}%
\pgfpathcurveto{\pgfqpoint{3.419775in}{1.424262in}}{\pgfqpoint{3.424165in}{1.434861in}}{\pgfqpoint{3.424165in}{1.445911in}}%
\pgfpathcurveto{\pgfqpoint{3.424165in}{1.456961in}}{\pgfqpoint{3.419775in}{1.467560in}}{\pgfqpoint{3.411962in}{1.475374in}}%
\pgfpathcurveto{\pgfqpoint{3.404148in}{1.483187in}}{\pgfqpoint{3.393549in}{1.487578in}}{\pgfqpoint{3.382499in}{1.487578in}}%
\pgfpathcurveto{\pgfqpoint{3.371449in}{1.487578in}}{\pgfqpoint{3.360850in}{1.483187in}}{\pgfqpoint{3.353036in}{1.475374in}}%
\pgfpathcurveto{\pgfqpoint{3.345222in}{1.467560in}}{\pgfqpoint{3.340832in}{1.456961in}}{\pgfqpoint{3.340832in}{1.445911in}}%
\pgfpathcurveto{\pgfqpoint{3.340832in}{1.434861in}}{\pgfqpoint{3.345222in}{1.424262in}}{\pgfqpoint{3.353036in}{1.416448in}}%
\pgfpathcurveto{\pgfqpoint{3.360850in}{1.408635in}}{\pgfqpoint{3.371449in}{1.404244in}}{\pgfqpoint{3.382499in}{1.404244in}}%
\pgfpathclose%
\pgfusepath{stroke,fill}%
\end{pgfscope}%
\begin{pgfscope}%
\pgfpathrectangle{\pgfqpoint{1.016621in}{0.499691in}}{\pgfqpoint{3.875000in}{2.695000in}}%
\pgfusepath{clip}%
\pgfsetbuttcap%
\pgfsetroundjoin%
\definecolor{currentfill}{rgb}{0.117647,0.564706,1.000000}%
\pgfsetfillcolor{currentfill}%
\pgfsetlinewidth{1.003750pt}%
\definecolor{currentstroke}{rgb}{0.117647,0.564706,1.000000}%
\pgfsetstrokecolor{currentstroke}%
\pgfsetdash{}{0pt}%
\pgfpathmoveto{\pgfqpoint{3.444582in}{1.484500in}}%
\pgfpathcurveto{\pgfqpoint{3.455633in}{1.484500in}}{\pgfqpoint{3.466232in}{1.488891in}}{\pgfqpoint{3.474045in}{1.496704in}}%
\pgfpathcurveto{\pgfqpoint{3.481859in}{1.504518in}}{\pgfqpoint{3.486249in}{1.515117in}}{\pgfqpoint{3.486249in}{1.526167in}}%
\pgfpathcurveto{\pgfqpoint{3.486249in}{1.537217in}}{\pgfqpoint{3.481859in}{1.547816in}}{\pgfqpoint{3.474045in}{1.555630in}}%
\pgfpathcurveto{\pgfqpoint{3.466232in}{1.563443in}}{\pgfqpoint{3.455633in}{1.567834in}}{\pgfqpoint{3.444582in}{1.567834in}}%
\pgfpathcurveto{\pgfqpoint{3.433532in}{1.567834in}}{\pgfqpoint{3.422933in}{1.563443in}}{\pgfqpoint{3.415120in}{1.555630in}}%
\pgfpathcurveto{\pgfqpoint{3.407306in}{1.547816in}}{\pgfqpoint{3.402916in}{1.537217in}}{\pgfqpoint{3.402916in}{1.526167in}}%
\pgfpathcurveto{\pgfqpoint{3.402916in}{1.515117in}}{\pgfqpoint{3.407306in}{1.504518in}}{\pgfqpoint{3.415120in}{1.496704in}}%
\pgfpathcurveto{\pgfqpoint{3.422933in}{1.488891in}}{\pgfqpoint{3.433532in}{1.484500in}}{\pgfqpoint{3.444582in}{1.484500in}}%
\pgfpathclose%
\pgfusepath{stroke,fill}%
\end{pgfscope}%
\begin{pgfscope}%
\pgfpathrectangle{\pgfqpoint{1.016621in}{0.499691in}}{\pgfqpoint{3.875000in}{2.695000in}}%
\pgfusepath{clip}%
\pgfsetbuttcap%
\pgfsetroundjoin%
\definecolor{currentfill}{rgb}{0.117647,0.564706,1.000000}%
\pgfsetfillcolor{currentfill}%
\pgfsetlinewidth{1.003750pt}%
\definecolor{currentstroke}{rgb}{0.117647,0.564706,1.000000}%
\pgfsetstrokecolor{currentstroke}%
\pgfsetdash{}{0pt}%
\pgfpathmoveto{\pgfqpoint{3.494249in}{1.676542in}}%
\pgfpathcurveto{\pgfqpoint{3.505300in}{1.676542in}}{\pgfqpoint{3.515899in}{1.680932in}}{\pgfqpoint{3.523712in}{1.688745in}}%
\pgfpathcurveto{\pgfqpoint{3.531526in}{1.696559in}}{\pgfqpoint{3.535916in}{1.707158in}}{\pgfqpoint{3.535916in}{1.718208in}}%
\pgfpathcurveto{\pgfqpoint{3.535916in}{1.729258in}}{\pgfqpoint{3.531526in}{1.739857in}}{\pgfqpoint{3.523712in}{1.747671in}}%
\pgfpathcurveto{\pgfqpoint{3.515899in}{1.755485in}}{\pgfqpoint{3.505300in}{1.759875in}}{\pgfqpoint{3.494249in}{1.759875in}}%
\pgfpathcurveto{\pgfqpoint{3.483199in}{1.759875in}}{\pgfqpoint{3.472600in}{1.755485in}}{\pgfqpoint{3.464787in}{1.747671in}}%
\pgfpathcurveto{\pgfqpoint{3.456973in}{1.739857in}}{\pgfqpoint{3.452583in}{1.729258in}}{\pgfqpoint{3.452583in}{1.718208in}}%
\pgfpathcurveto{\pgfqpoint{3.452583in}{1.707158in}}{\pgfqpoint{3.456973in}{1.696559in}}{\pgfqpoint{3.464787in}{1.688745in}}%
\pgfpathcurveto{\pgfqpoint{3.472600in}{1.680932in}}{\pgfqpoint{3.483199in}{1.676542in}}{\pgfqpoint{3.494249in}{1.676542in}}%
\pgfpathclose%
\pgfusepath{stroke,fill}%
\end{pgfscope}%
\begin{pgfscope}%
\pgfpathrectangle{\pgfqpoint{1.016621in}{0.499691in}}{\pgfqpoint{3.875000in}{2.695000in}}%
\pgfusepath{clip}%
\pgfsetbuttcap%
\pgfsetroundjoin%
\definecolor{currentfill}{rgb}{0.117647,0.564706,1.000000}%
\pgfsetfillcolor{currentfill}%
\pgfsetlinewidth{1.003750pt}%
\definecolor{currentstroke}{rgb}{0.117647,0.564706,1.000000}%
\pgfsetstrokecolor{currentstroke}%
\pgfsetdash{}{0pt}%
\pgfpathmoveto{\pgfqpoint{3.531500in}{1.619216in}}%
\pgfpathcurveto{\pgfqpoint{3.542550in}{1.619216in}}{\pgfqpoint{3.553149in}{1.623606in}}{\pgfqpoint{3.560963in}{1.631420in}}%
\pgfpathcurveto{\pgfqpoint{3.568776in}{1.639233in}}{\pgfqpoint{3.573166in}{1.649832in}}{\pgfqpoint{3.573166in}{1.660883in}}%
\pgfpathcurveto{\pgfqpoint{3.573166in}{1.671933in}}{\pgfqpoint{3.568776in}{1.682532in}}{\pgfqpoint{3.560963in}{1.690345in}}%
\pgfpathcurveto{\pgfqpoint{3.553149in}{1.698159in}}{\pgfqpoint{3.542550in}{1.702549in}}{\pgfqpoint{3.531500in}{1.702549in}}%
\pgfpathcurveto{\pgfqpoint{3.520450in}{1.702549in}}{\pgfqpoint{3.509851in}{1.698159in}}{\pgfqpoint{3.502037in}{1.690345in}}%
\pgfpathcurveto{\pgfqpoint{3.494223in}{1.682532in}}{\pgfqpoint{3.489833in}{1.671933in}}{\pgfqpoint{3.489833in}{1.660883in}}%
\pgfpathcurveto{\pgfqpoint{3.489833in}{1.649832in}}{\pgfqpoint{3.494223in}{1.639233in}}{\pgfqpoint{3.502037in}{1.631420in}}%
\pgfpathcurveto{\pgfqpoint{3.509851in}{1.623606in}}{\pgfqpoint{3.520450in}{1.619216in}}{\pgfqpoint{3.531500in}{1.619216in}}%
\pgfpathclose%
\pgfusepath{stroke,fill}%
\end{pgfscope}%
\begin{pgfscope}%
\pgfpathrectangle{\pgfqpoint{1.016621in}{0.499691in}}{\pgfqpoint{3.875000in}{2.695000in}}%
\pgfusepath{clip}%
\pgfsetbuttcap%
\pgfsetroundjoin%
\definecolor{currentfill}{rgb}{0.117647,0.564706,1.000000}%
\pgfsetfillcolor{currentfill}%
\pgfsetlinewidth{1.003750pt}%
\definecolor{currentstroke}{rgb}{0.117647,0.564706,1.000000}%
\pgfsetstrokecolor{currentstroke}%
\pgfsetdash{}{0pt}%
\pgfpathmoveto{\pgfqpoint{3.568750in}{1.579088in}}%
\pgfpathcurveto{\pgfqpoint{3.579800in}{1.579088in}}{\pgfqpoint{3.590399in}{1.583478in}}{\pgfqpoint{3.598213in}{1.591292in}}%
\pgfpathcurveto{\pgfqpoint{3.606026in}{1.599105in}}{\pgfqpoint{3.610417in}{1.609704in}}{\pgfqpoint{3.610417in}{1.620755in}}%
\pgfpathcurveto{\pgfqpoint{3.610417in}{1.631805in}}{\pgfqpoint{3.606026in}{1.642404in}}{\pgfqpoint{3.598213in}{1.650217in}}%
\pgfpathcurveto{\pgfqpoint{3.590399in}{1.658031in}}{\pgfqpoint{3.579800in}{1.662421in}}{\pgfqpoint{3.568750in}{1.662421in}}%
\pgfpathcurveto{\pgfqpoint{3.557700in}{1.662421in}}{\pgfqpoint{3.547101in}{1.658031in}}{\pgfqpoint{3.539287in}{1.650217in}}%
\pgfpathcurveto{\pgfqpoint{3.531474in}{1.642404in}}{\pgfqpoint{3.527083in}{1.631805in}}{\pgfqpoint{3.527083in}{1.620755in}}%
\pgfpathcurveto{\pgfqpoint{3.527083in}{1.609704in}}{\pgfqpoint{3.531474in}{1.599105in}}{\pgfqpoint{3.539287in}{1.591292in}}%
\pgfpathcurveto{\pgfqpoint{3.547101in}{1.583478in}}{\pgfqpoint{3.557700in}{1.579088in}}{\pgfqpoint{3.568750in}{1.579088in}}%
\pgfpathclose%
\pgfusepath{stroke,fill}%
\end{pgfscope}%
\begin{pgfscope}%
\pgfpathrectangle{\pgfqpoint{1.016621in}{0.499691in}}{\pgfqpoint{3.875000in}{2.695000in}}%
\pgfusepath{clip}%
\pgfsetbuttcap%
\pgfsetroundjoin%
\definecolor{currentfill}{rgb}{0.117647,0.564706,1.000000}%
\pgfsetfillcolor{currentfill}%
\pgfsetlinewidth{1.003750pt}%
\definecolor{currentstroke}{rgb}{0.117647,0.564706,1.000000}%
\pgfsetstrokecolor{currentstroke}%
\pgfsetdash{}{0pt}%
\pgfpathmoveto{\pgfqpoint{3.606000in}{1.550425in}}%
\pgfpathcurveto{\pgfqpoint{3.617050in}{1.550425in}}{\pgfqpoint{3.627649in}{1.554815in}}{\pgfqpoint{3.635463in}{1.562629in}}%
\pgfpathcurveto{\pgfqpoint{3.643277in}{1.570442in}}{\pgfqpoint{3.647667in}{1.581042in}}{\pgfqpoint{3.647667in}{1.592092in}}%
\pgfpathcurveto{\pgfqpoint{3.647667in}{1.603142in}}{\pgfqpoint{3.643277in}{1.613741in}}{\pgfqpoint{3.635463in}{1.621554in}}%
\pgfpathcurveto{\pgfqpoint{3.627649in}{1.629368in}}{\pgfqpoint{3.617050in}{1.633758in}}{\pgfqpoint{3.606000in}{1.633758in}}%
\pgfpathcurveto{\pgfqpoint{3.594950in}{1.633758in}}{\pgfqpoint{3.584351in}{1.629368in}}{\pgfqpoint{3.576537in}{1.621554in}}%
\pgfpathcurveto{\pgfqpoint{3.568724in}{1.613741in}}{\pgfqpoint{3.564334in}{1.603142in}}{\pgfqpoint{3.564334in}{1.592092in}}%
\pgfpathcurveto{\pgfqpoint{3.564334in}{1.581042in}}{\pgfqpoint{3.568724in}{1.570442in}}{\pgfqpoint{3.576537in}{1.562629in}}%
\pgfpathcurveto{\pgfqpoint{3.584351in}{1.554815in}}{\pgfqpoint{3.594950in}{1.550425in}}{\pgfqpoint{3.606000in}{1.550425in}}%
\pgfpathclose%
\pgfusepath{stroke,fill}%
\end{pgfscope}%
\begin{pgfscope}%
\pgfpathrectangle{\pgfqpoint{1.016621in}{0.499691in}}{\pgfqpoint{3.875000in}{2.695000in}}%
\pgfusepath{clip}%
\pgfsetbuttcap%
\pgfsetroundjoin%
\definecolor{currentfill}{rgb}{0.117647,0.564706,1.000000}%
\pgfsetfillcolor{currentfill}%
\pgfsetlinewidth{1.003750pt}%
\definecolor{currentstroke}{rgb}{0.117647,0.564706,1.000000}%
\pgfsetstrokecolor{currentstroke}%
\pgfsetdash{}{0pt}%
\pgfpathmoveto{\pgfqpoint{3.643250in}{1.693739in}}%
\pgfpathcurveto{\pgfqpoint{3.654301in}{1.693739in}}{\pgfqpoint{3.664900in}{1.698130in}}{\pgfqpoint{3.672713in}{1.705943in}}%
\pgfpathcurveto{\pgfqpoint{3.680527in}{1.713757in}}{\pgfqpoint{3.684917in}{1.724356in}}{\pgfqpoint{3.684917in}{1.735406in}}%
\pgfpathcurveto{\pgfqpoint{3.684917in}{1.746456in}}{\pgfqpoint{3.680527in}{1.757055in}}{\pgfqpoint{3.672713in}{1.764869in}}%
\pgfpathcurveto{\pgfqpoint{3.664900in}{1.772682in}}{\pgfqpoint{3.654301in}{1.777073in}}{\pgfqpoint{3.643250in}{1.777073in}}%
\pgfpathcurveto{\pgfqpoint{3.632200in}{1.777073in}}{\pgfqpoint{3.621601in}{1.772682in}}{\pgfqpoint{3.613788in}{1.764869in}}%
\pgfpathcurveto{\pgfqpoint{3.605974in}{1.757055in}}{\pgfqpoint{3.601584in}{1.746456in}}{\pgfqpoint{3.601584in}{1.735406in}}%
\pgfpathcurveto{\pgfqpoint{3.601584in}{1.724356in}}{\pgfqpoint{3.605974in}{1.713757in}}{\pgfqpoint{3.613788in}{1.705943in}}%
\pgfpathcurveto{\pgfqpoint{3.621601in}{1.698130in}}{\pgfqpoint{3.632200in}{1.693739in}}{\pgfqpoint{3.643250in}{1.693739in}}%
\pgfpathclose%
\pgfusepath{stroke,fill}%
\end{pgfscope}%
\begin{pgfscope}%
\pgfpathrectangle{\pgfqpoint{1.016621in}{0.499691in}}{\pgfqpoint{3.875000in}{2.695000in}}%
\pgfusepath{clip}%
\pgfsetbuttcap%
\pgfsetroundjoin%
\definecolor{currentfill}{rgb}{0.117647,0.564706,1.000000}%
\pgfsetfillcolor{currentfill}%
\pgfsetlinewidth{1.003750pt}%
\definecolor{currentstroke}{rgb}{0.117647,0.564706,1.000000}%
\pgfsetstrokecolor{currentstroke}%
\pgfsetdash{}{0pt}%
\pgfpathmoveto{\pgfqpoint{3.680501in}{1.788327in}}%
\pgfpathcurveto{\pgfqpoint{3.691551in}{1.788327in}}{\pgfqpoint{3.702150in}{1.792717in}}{\pgfqpoint{3.709964in}{1.800531in}}%
\pgfpathcurveto{\pgfqpoint{3.717777in}{1.808344in}}{\pgfqpoint{3.722167in}{1.818943in}}{\pgfqpoint{3.722167in}{1.829993in}}%
\pgfpathcurveto{\pgfqpoint{3.722167in}{1.841044in}}{\pgfqpoint{3.717777in}{1.851643in}}{\pgfqpoint{3.709964in}{1.859456in}}%
\pgfpathcurveto{\pgfqpoint{3.702150in}{1.867270in}}{\pgfqpoint{3.691551in}{1.871660in}}{\pgfqpoint{3.680501in}{1.871660in}}%
\pgfpathcurveto{\pgfqpoint{3.669451in}{1.871660in}}{\pgfqpoint{3.658852in}{1.867270in}}{\pgfqpoint{3.651038in}{1.859456in}}%
\pgfpathcurveto{\pgfqpoint{3.643224in}{1.851643in}}{\pgfqpoint{3.638834in}{1.841044in}}{\pgfqpoint{3.638834in}{1.829993in}}%
\pgfpathcurveto{\pgfqpoint{3.638834in}{1.818943in}}{\pgfqpoint{3.643224in}{1.808344in}}{\pgfqpoint{3.651038in}{1.800531in}}%
\pgfpathcurveto{\pgfqpoint{3.658852in}{1.792717in}}{\pgfqpoint{3.669451in}{1.788327in}}{\pgfqpoint{3.680501in}{1.788327in}}%
\pgfpathclose%
\pgfusepath{stroke,fill}%
\end{pgfscope}%
\begin{pgfscope}%
\pgfpathrectangle{\pgfqpoint{1.016621in}{0.499691in}}{\pgfqpoint{3.875000in}{2.695000in}}%
\pgfusepath{clip}%
\pgfsetbuttcap%
\pgfsetroundjoin%
\definecolor{currentfill}{rgb}{0.117647,0.564706,1.000000}%
\pgfsetfillcolor{currentfill}%
\pgfsetlinewidth{1.003750pt}%
\definecolor{currentstroke}{rgb}{0.117647,0.564706,1.000000}%
\pgfsetstrokecolor{currentstroke}%
\pgfsetdash{}{0pt}%
\pgfpathmoveto{\pgfqpoint{3.717751in}{1.951705in}}%
\pgfpathcurveto{\pgfqpoint{3.728801in}{1.951705in}}{\pgfqpoint{3.739400in}{1.956095in}}{\pgfqpoint{3.747214in}{1.963909in}}%
\pgfpathcurveto{\pgfqpoint{3.755027in}{1.971723in}}{\pgfqpoint{3.759418in}{1.982322in}}{\pgfqpoint{3.759418in}{1.993372in}}%
\pgfpathcurveto{\pgfqpoint{3.759418in}{2.004422in}}{\pgfqpoint{3.755027in}{2.015021in}}{\pgfqpoint{3.747214in}{2.022834in}}%
\pgfpathcurveto{\pgfqpoint{3.739400in}{2.030648in}}{\pgfqpoint{3.728801in}{2.035038in}}{\pgfqpoint{3.717751in}{2.035038in}}%
\pgfpathcurveto{\pgfqpoint{3.706701in}{2.035038in}}{\pgfqpoint{3.696102in}{2.030648in}}{\pgfqpoint{3.688288in}{2.022834in}}%
\pgfpathcurveto{\pgfqpoint{3.680475in}{2.015021in}}{\pgfqpoint{3.676084in}{2.004422in}}{\pgfqpoint{3.676084in}{1.993372in}}%
\pgfpathcurveto{\pgfqpoint{3.676084in}{1.982322in}}{\pgfqpoint{3.680475in}{1.971723in}}{\pgfqpoint{3.688288in}{1.963909in}}%
\pgfpathcurveto{\pgfqpoint{3.696102in}{1.956095in}}{\pgfqpoint{3.706701in}{1.951705in}}{\pgfqpoint{3.717751in}{1.951705in}}%
\pgfpathclose%
\pgfusepath{stroke,fill}%
\end{pgfscope}%
\begin{pgfscope}%
\pgfpathrectangle{\pgfqpoint{1.016621in}{0.499691in}}{\pgfqpoint{3.875000in}{2.695000in}}%
\pgfusepath{clip}%
\pgfsetbuttcap%
\pgfsetroundjoin%
\definecolor{currentfill}{rgb}{0.117647,0.564706,1.000000}%
\pgfsetfillcolor{currentfill}%
\pgfsetlinewidth{1.003750pt}%
\definecolor{currentstroke}{rgb}{0.117647,0.564706,1.000000}%
\pgfsetstrokecolor{currentstroke}%
\pgfsetdash{}{0pt}%
\pgfpathmoveto{\pgfqpoint{3.742584in}{1.914443in}}%
\pgfpathcurveto{\pgfqpoint{3.753635in}{1.914443in}}{\pgfqpoint{3.764234in}{1.918834in}}{\pgfqpoint{3.772047in}{1.926647in}}%
\pgfpathcurveto{\pgfqpoint{3.779861in}{1.934461in}}{\pgfqpoint{3.784251in}{1.945060in}}{\pgfqpoint{3.784251in}{1.956110in}}%
\pgfpathcurveto{\pgfqpoint{3.784251in}{1.967160in}}{\pgfqpoint{3.779861in}{1.977759in}}{\pgfqpoint{3.772047in}{1.985573in}}%
\pgfpathcurveto{\pgfqpoint{3.764234in}{1.993386in}}{\pgfqpoint{3.753635in}{1.997777in}}{\pgfqpoint{3.742584in}{1.997777in}}%
\pgfpathcurveto{\pgfqpoint{3.731534in}{1.997777in}}{\pgfqpoint{3.720935in}{1.993386in}}{\pgfqpoint{3.713122in}{1.985573in}}%
\pgfpathcurveto{\pgfqpoint{3.705308in}{1.977759in}}{\pgfqpoint{3.700918in}{1.967160in}}{\pgfqpoint{3.700918in}{1.956110in}}%
\pgfpathcurveto{\pgfqpoint{3.700918in}{1.945060in}}{\pgfqpoint{3.705308in}{1.934461in}}{\pgfqpoint{3.713122in}{1.926647in}}%
\pgfpathcurveto{\pgfqpoint{3.720935in}{1.918834in}}{\pgfqpoint{3.731534in}{1.914443in}}{\pgfqpoint{3.742584in}{1.914443in}}%
\pgfpathclose%
\pgfusepath{stroke,fill}%
\end{pgfscope}%
\begin{pgfscope}%
\pgfpathrectangle{\pgfqpoint{1.016621in}{0.499691in}}{\pgfqpoint{3.875000in}{2.695000in}}%
\pgfusepath{clip}%
\pgfsetbuttcap%
\pgfsetroundjoin%
\definecolor{currentfill}{rgb}{0.117647,0.564706,1.000000}%
\pgfsetfillcolor{currentfill}%
\pgfsetlinewidth{1.003750pt}%
\definecolor{currentstroke}{rgb}{0.117647,0.564706,1.000000}%
\pgfsetstrokecolor{currentstroke}%
\pgfsetdash{}{0pt}%
\pgfpathmoveto{\pgfqpoint{3.767418in}{1.928775in}}%
\pgfpathcurveto{\pgfqpoint{3.778468in}{1.928775in}}{\pgfqpoint{3.789067in}{1.933165in}}{\pgfqpoint{3.796881in}{1.940979in}}%
\pgfpathcurveto{\pgfqpoint{3.804694in}{1.948792in}}{\pgfqpoint{3.809085in}{1.959391in}}{\pgfqpoint{3.809085in}{1.970441in}}%
\pgfpathcurveto{\pgfqpoint{3.809085in}{1.981492in}}{\pgfqpoint{3.804694in}{1.992091in}}{\pgfqpoint{3.796881in}{1.999904in}}%
\pgfpathcurveto{\pgfqpoint{3.789067in}{2.007718in}}{\pgfqpoint{3.778468in}{2.012108in}}{\pgfqpoint{3.767418in}{2.012108in}}%
\pgfpathcurveto{\pgfqpoint{3.756368in}{2.012108in}}{\pgfqpoint{3.745769in}{2.007718in}}{\pgfqpoint{3.737955in}{1.999904in}}%
\pgfpathcurveto{\pgfqpoint{3.730142in}{1.992091in}}{\pgfqpoint{3.725751in}{1.981492in}}{\pgfqpoint{3.725751in}{1.970441in}}%
\pgfpathcurveto{\pgfqpoint{3.725751in}{1.959391in}}{\pgfqpoint{3.730142in}{1.948792in}}{\pgfqpoint{3.737955in}{1.940979in}}%
\pgfpathcurveto{\pgfqpoint{3.745769in}{1.933165in}}{\pgfqpoint{3.756368in}{1.928775in}}{\pgfqpoint{3.767418in}{1.928775in}}%
\pgfpathclose%
\pgfusepath{stroke,fill}%
\end{pgfscope}%
\begin{pgfscope}%
\pgfpathrectangle{\pgfqpoint{1.016621in}{0.499691in}}{\pgfqpoint{3.875000in}{2.695000in}}%
\pgfusepath{clip}%
\pgfsetbuttcap%
\pgfsetroundjoin%
\definecolor{currentfill}{rgb}{0.117647,0.564706,1.000000}%
\pgfsetfillcolor{currentfill}%
\pgfsetlinewidth{1.003750pt}%
\definecolor{currentstroke}{rgb}{0.117647,0.564706,1.000000}%
\pgfsetstrokecolor{currentstroke}%
\pgfsetdash{}{0pt}%
\pgfpathmoveto{\pgfqpoint{3.792251in}{1.954571in}}%
\pgfpathcurveto{\pgfqpoint{3.803302in}{1.954571in}}{\pgfqpoint{3.813901in}{1.958962in}}{\pgfqpoint{3.821714in}{1.966775in}}%
\pgfpathcurveto{\pgfqpoint{3.829528in}{1.974589in}}{\pgfqpoint{3.833918in}{1.985188in}}{\pgfqpoint{3.833918in}{1.996238in}}%
\pgfpathcurveto{\pgfqpoint{3.833918in}{2.007288in}}{\pgfqpoint{3.829528in}{2.017887in}}{\pgfqpoint{3.821714in}{2.025701in}}%
\pgfpathcurveto{\pgfqpoint{3.813901in}{2.033514in}}{\pgfqpoint{3.803302in}{2.037905in}}{\pgfqpoint{3.792251in}{2.037905in}}%
\pgfpathcurveto{\pgfqpoint{3.781201in}{2.037905in}}{\pgfqpoint{3.770602in}{2.033514in}}{\pgfqpoint{3.762789in}{2.025701in}}%
\pgfpathcurveto{\pgfqpoint{3.754975in}{2.017887in}}{\pgfqpoint{3.750585in}{2.007288in}}{\pgfqpoint{3.750585in}{1.996238in}}%
\pgfpathcurveto{\pgfqpoint{3.750585in}{1.985188in}}{\pgfqpoint{3.754975in}{1.974589in}}{\pgfqpoint{3.762789in}{1.966775in}}%
\pgfpathcurveto{\pgfqpoint{3.770602in}{1.958962in}}{\pgfqpoint{3.781201in}{1.954571in}}{\pgfqpoint{3.792251in}{1.954571in}}%
\pgfpathclose%
\pgfusepath{stroke,fill}%
\end{pgfscope}%
\begin{pgfscope}%
\pgfpathrectangle{\pgfqpoint{1.016621in}{0.499691in}}{\pgfqpoint{3.875000in}{2.695000in}}%
\pgfusepath{clip}%
\pgfsetbuttcap%
\pgfsetroundjoin%
\definecolor{currentfill}{rgb}{0.117647,0.564706,1.000000}%
\pgfsetfillcolor{currentfill}%
\pgfsetlinewidth{1.003750pt}%
\definecolor{currentstroke}{rgb}{0.117647,0.564706,1.000000}%
\pgfsetstrokecolor{currentstroke}%
\pgfsetdash{}{0pt}%
\pgfpathmoveto{\pgfqpoint{3.817085in}{2.043426in}}%
\pgfpathcurveto{\pgfqpoint{3.828135in}{2.043426in}}{\pgfqpoint{3.838734in}{2.047816in}}{\pgfqpoint{3.846548in}{2.055630in}}%
\pgfpathcurveto{\pgfqpoint{3.854361in}{2.063444in}}{\pgfqpoint{3.858752in}{2.074043in}}{\pgfqpoint{3.858752in}{2.085093in}}%
\pgfpathcurveto{\pgfqpoint{3.858752in}{2.096143in}}{\pgfqpoint{3.854361in}{2.106742in}}{\pgfqpoint{3.846548in}{2.114556in}}%
\pgfpathcurveto{\pgfqpoint{3.838734in}{2.122369in}}{\pgfqpoint{3.828135in}{2.126760in}}{\pgfqpoint{3.817085in}{2.126760in}}%
\pgfpathcurveto{\pgfqpoint{3.806035in}{2.126760in}}{\pgfqpoint{3.795436in}{2.122369in}}{\pgfqpoint{3.787622in}{2.114556in}}%
\pgfpathcurveto{\pgfqpoint{3.779809in}{2.106742in}}{\pgfqpoint{3.775418in}{2.096143in}}{\pgfqpoint{3.775418in}{2.085093in}}%
\pgfpathcurveto{\pgfqpoint{3.775418in}{2.074043in}}{\pgfqpoint{3.779809in}{2.063444in}}{\pgfqpoint{3.787622in}{2.055630in}}%
\pgfpathcurveto{\pgfqpoint{3.795436in}{2.047816in}}{\pgfqpoint{3.806035in}{2.043426in}}{\pgfqpoint{3.817085in}{2.043426in}}%
\pgfpathclose%
\pgfusepath{stroke,fill}%
\end{pgfscope}%
\begin{pgfscope}%
\pgfpathrectangle{\pgfqpoint{1.016621in}{0.499691in}}{\pgfqpoint{3.875000in}{2.695000in}}%
\pgfusepath{clip}%
\pgfsetbuttcap%
\pgfsetroundjoin%
\definecolor{currentfill}{rgb}{0.117647,0.564706,1.000000}%
\pgfsetfillcolor{currentfill}%
\pgfsetlinewidth{1.003750pt}%
\definecolor{currentstroke}{rgb}{0.117647,0.564706,1.000000}%
\pgfsetstrokecolor{currentstroke}%
\pgfsetdash{}{0pt}%
\pgfpathmoveto{\pgfqpoint{3.841918in}{2.092153in}}%
\pgfpathcurveto{\pgfqpoint{3.852969in}{2.092153in}}{\pgfqpoint{3.863568in}{2.096543in}}{\pgfqpoint{3.871381in}{2.104357in}}%
\pgfpathcurveto{\pgfqpoint{3.879195in}{2.112171in}}{\pgfqpoint{3.883585in}{2.122770in}}{\pgfqpoint{3.883585in}{2.133820in}}%
\pgfpathcurveto{\pgfqpoint{3.883585in}{2.144870in}}{\pgfqpoint{3.879195in}{2.155469in}}{\pgfqpoint{3.871381in}{2.163283in}}%
\pgfpathcurveto{\pgfqpoint{3.863568in}{2.171096in}}{\pgfqpoint{3.852969in}{2.175486in}}{\pgfqpoint{3.841918in}{2.175486in}}%
\pgfpathcurveto{\pgfqpoint{3.830868in}{2.175486in}}{\pgfqpoint{3.820269in}{2.171096in}}{\pgfqpoint{3.812456in}{2.163283in}}%
\pgfpathcurveto{\pgfqpoint{3.804642in}{2.155469in}}{\pgfqpoint{3.800252in}{2.144870in}}{\pgfqpoint{3.800252in}{2.133820in}}%
\pgfpathcurveto{\pgfqpoint{3.800252in}{2.122770in}}{\pgfqpoint{3.804642in}{2.112171in}}{\pgfqpoint{3.812456in}{2.104357in}}%
\pgfpathcurveto{\pgfqpoint{3.820269in}{2.096543in}}{\pgfqpoint{3.830868in}{2.092153in}}{\pgfqpoint{3.841918in}{2.092153in}}%
\pgfpathclose%
\pgfusepath{stroke,fill}%
\end{pgfscope}%
\begin{pgfscope}%
\pgfpathrectangle{\pgfqpoint{1.016621in}{0.499691in}}{\pgfqpoint{3.875000in}{2.695000in}}%
\pgfusepath{clip}%
\pgfsetbuttcap%
\pgfsetroundjoin%
\definecolor{currentfill}{rgb}{0.117647,0.564706,1.000000}%
\pgfsetfillcolor{currentfill}%
\pgfsetlinewidth{1.003750pt}%
\definecolor{currentstroke}{rgb}{0.117647,0.564706,1.000000}%
\pgfsetstrokecolor{currentstroke}%
\pgfsetdash{}{0pt}%
\pgfpathmoveto{\pgfqpoint{3.866752in}{2.149479in}}%
\pgfpathcurveto{\pgfqpoint{3.877802in}{2.149479in}}{\pgfqpoint{3.888401in}{2.153869in}}{\pgfqpoint{3.896215in}{2.161683in}}%
\pgfpathcurveto{\pgfqpoint{3.904028in}{2.169496in}}{\pgfqpoint{3.908419in}{2.180095in}}{\pgfqpoint{3.908419in}{2.191145in}}%
\pgfpathcurveto{\pgfqpoint{3.908419in}{2.202196in}}{\pgfqpoint{3.904028in}{2.212795in}}{\pgfqpoint{3.896215in}{2.220608in}}%
\pgfpathcurveto{\pgfqpoint{3.888401in}{2.228422in}}{\pgfqpoint{3.877802in}{2.232812in}}{\pgfqpoint{3.866752in}{2.232812in}}%
\pgfpathcurveto{\pgfqpoint{3.855702in}{2.232812in}}{\pgfqpoint{3.845103in}{2.228422in}}{\pgfqpoint{3.837289in}{2.220608in}}%
\pgfpathcurveto{\pgfqpoint{3.829476in}{2.212795in}}{\pgfqpoint{3.825085in}{2.202196in}}{\pgfqpoint{3.825085in}{2.191145in}}%
\pgfpathcurveto{\pgfqpoint{3.825085in}{2.180095in}}{\pgfqpoint{3.829476in}{2.169496in}}{\pgfqpoint{3.837289in}{2.161683in}}%
\pgfpathcurveto{\pgfqpoint{3.845103in}{2.153869in}}{\pgfqpoint{3.855702in}{2.149479in}}{\pgfqpoint{3.866752in}{2.149479in}}%
\pgfpathclose%
\pgfusepath{stroke,fill}%
\end{pgfscope}%
\begin{pgfscope}%
\pgfpathrectangle{\pgfqpoint{1.016621in}{0.499691in}}{\pgfqpoint{3.875000in}{2.695000in}}%
\pgfusepath{clip}%
\pgfsetbuttcap%
\pgfsetroundjoin%
\definecolor{currentfill}{rgb}{0.117647,0.564706,1.000000}%
\pgfsetfillcolor{currentfill}%
\pgfsetlinewidth{1.003750pt}%
\definecolor{currentstroke}{rgb}{0.117647,0.564706,1.000000}%
\pgfsetstrokecolor{currentstroke}%
\pgfsetdash{}{0pt}%
\pgfpathmoveto{\pgfqpoint{3.891585in}{2.155211in}}%
\pgfpathcurveto{\pgfqpoint{3.902636in}{2.155211in}}{\pgfqpoint{3.913235in}{2.159602in}}{\pgfqpoint{3.921048in}{2.167415in}}%
\pgfpathcurveto{\pgfqpoint{3.928862in}{2.175229in}}{\pgfqpoint{3.933252in}{2.185828in}}{\pgfqpoint{3.933252in}{2.196878in}}%
\pgfpathcurveto{\pgfqpoint{3.933252in}{2.207928in}}{\pgfqpoint{3.928862in}{2.218527in}}{\pgfqpoint{3.921048in}{2.226341in}}%
\pgfpathcurveto{\pgfqpoint{3.913235in}{2.234154in}}{\pgfqpoint{3.902636in}{2.238545in}}{\pgfqpoint{3.891585in}{2.238545in}}%
\pgfpathcurveto{\pgfqpoint{3.880535in}{2.238545in}}{\pgfqpoint{3.869936in}{2.234154in}}{\pgfqpoint{3.862123in}{2.226341in}}%
\pgfpathcurveto{\pgfqpoint{3.854309in}{2.218527in}}{\pgfqpoint{3.849919in}{2.207928in}}{\pgfqpoint{3.849919in}{2.196878in}}%
\pgfpathcurveto{\pgfqpoint{3.849919in}{2.185828in}}{\pgfqpoint{3.854309in}{2.175229in}}{\pgfqpoint{3.862123in}{2.167415in}}%
\pgfpathcurveto{\pgfqpoint{3.869936in}{2.159602in}}{\pgfqpoint{3.880535in}{2.155211in}}{\pgfqpoint{3.891585in}{2.155211in}}%
\pgfpathclose%
\pgfusepath{stroke,fill}%
\end{pgfscope}%
\begin{pgfscope}%
\pgfpathrectangle{\pgfqpoint{1.016621in}{0.499691in}}{\pgfqpoint{3.875000in}{2.695000in}}%
\pgfusepath{clip}%
\pgfsetbuttcap%
\pgfsetroundjoin%
\definecolor{currentfill}{rgb}{0.117647,0.564706,1.000000}%
\pgfsetfillcolor{currentfill}%
\pgfsetlinewidth{1.003750pt}%
\definecolor{currentstroke}{rgb}{0.117647,0.564706,1.000000}%
\pgfsetstrokecolor{currentstroke}%
\pgfsetdash{}{0pt}%
\pgfpathmoveto{\pgfqpoint{3.916419in}{2.183874in}}%
\pgfpathcurveto{\pgfqpoint{3.927469in}{2.183874in}}{\pgfqpoint{3.938068in}{2.188264in}}{\pgfqpoint{3.945882in}{2.196078in}}%
\pgfpathcurveto{\pgfqpoint{3.953695in}{2.203892in}}{\pgfqpoint{3.958086in}{2.214491in}}{\pgfqpoint{3.958086in}{2.225541in}}%
\pgfpathcurveto{\pgfqpoint{3.958086in}{2.236591in}}{\pgfqpoint{3.953695in}{2.247190in}}{\pgfqpoint{3.945882in}{2.255004in}}%
\pgfpathcurveto{\pgfqpoint{3.938068in}{2.262817in}}{\pgfqpoint{3.927469in}{2.267208in}}{\pgfqpoint{3.916419in}{2.267208in}}%
\pgfpathcurveto{\pgfqpoint{3.905369in}{2.267208in}}{\pgfqpoint{3.894770in}{2.262817in}}{\pgfqpoint{3.886956in}{2.255004in}}%
\pgfpathcurveto{\pgfqpoint{3.879143in}{2.247190in}}{\pgfqpoint{3.874752in}{2.236591in}}{\pgfqpoint{3.874752in}{2.225541in}}%
\pgfpathcurveto{\pgfqpoint{3.874752in}{2.214491in}}{\pgfqpoint{3.879143in}{2.203892in}}{\pgfqpoint{3.886956in}{2.196078in}}%
\pgfpathcurveto{\pgfqpoint{3.894770in}{2.188264in}}{\pgfqpoint{3.905369in}{2.183874in}}{\pgfqpoint{3.916419in}{2.183874in}}%
\pgfpathclose%
\pgfusepath{stroke,fill}%
\end{pgfscope}%
\begin{pgfscope}%
\pgfpathrectangle{\pgfqpoint{1.016621in}{0.499691in}}{\pgfqpoint{3.875000in}{2.695000in}}%
\pgfusepath{clip}%
\pgfsetbuttcap%
\pgfsetroundjoin%
\definecolor{currentfill}{rgb}{0.117647,0.564706,1.000000}%
\pgfsetfillcolor{currentfill}%
\pgfsetlinewidth{1.003750pt}%
\definecolor{currentstroke}{rgb}{0.117647,0.564706,1.000000}%
\pgfsetstrokecolor{currentstroke}%
\pgfsetdash{}{0pt}%
\pgfpathmoveto{\pgfqpoint{3.928836in}{2.201072in}}%
\pgfpathcurveto{\pgfqpoint{3.939886in}{2.201072in}}{\pgfqpoint{3.950485in}{2.205462in}}{\pgfqpoint{3.958299in}{2.213276in}}%
\pgfpathcurveto{\pgfqpoint{3.966112in}{2.221089in}}{\pgfqpoint{3.970502in}{2.231688in}}{\pgfqpoint{3.970502in}{2.242739in}}%
\pgfpathcurveto{\pgfqpoint{3.970502in}{2.253789in}}{\pgfqpoint{3.966112in}{2.264388in}}{\pgfqpoint{3.958299in}{2.272201in}}%
\pgfpathcurveto{\pgfqpoint{3.950485in}{2.280015in}}{\pgfqpoint{3.939886in}{2.284405in}}{\pgfqpoint{3.928836in}{2.284405in}}%
\pgfpathcurveto{\pgfqpoint{3.917786in}{2.284405in}}{\pgfqpoint{3.907187in}{2.280015in}}{\pgfqpoint{3.899373in}{2.272201in}}%
\pgfpathcurveto{\pgfqpoint{3.891559in}{2.264388in}}{\pgfqpoint{3.887169in}{2.253789in}}{\pgfqpoint{3.887169in}{2.242739in}}%
\pgfpathcurveto{\pgfqpoint{3.887169in}{2.231688in}}{\pgfqpoint{3.891559in}{2.221089in}}{\pgfqpoint{3.899373in}{2.213276in}}%
\pgfpathcurveto{\pgfqpoint{3.907187in}{2.205462in}}{\pgfqpoint{3.917786in}{2.201072in}}{\pgfqpoint{3.928836in}{2.201072in}}%
\pgfpathclose%
\pgfusepath{stroke,fill}%
\end{pgfscope}%
\begin{pgfscope}%
\pgfpathrectangle{\pgfqpoint{1.016621in}{0.499691in}}{\pgfqpoint{3.875000in}{2.695000in}}%
\pgfusepath{clip}%
\pgfsetbuttcap%
\pgfsetroundjoin%
\definecolor{currentfill}{rgb}{0.117647,0.564706,1.000000}%
\pgfsetfillcolor{currentfill}%
\pgfsetlinewidth{1.003750pt}%
\definecolor{currentstroke}{rgb}{0.117647,0.564706,1.000000}%
\pgfsetstrokecolor{currentstroke}%
\pgfsetdash{}{0pt}%
\pgfpathmoveto{\pgfqpoint{3.953669in}{2.206804in}}%
\pgfpathcurveto{\pgfqpoint{3.964719in}{2.206804in}}{\pgfqpoint{3.975318in}{2.211195in}}{\pgfqpoint{3.983132in}{2.219008in}}%
\pgfpathcurveto{\pgfqpoint{3.990946in}{2.226822in}}{\pgfqpoint{3.995336in}{2.237421in}}{\pgfqpoint{3.995336in}{2.248471in}}%
\pgfpathcurveto{\pgfqpoint{3.995336in}{2.259521in}}{\pgfqpoint{3.990946in}{2.270120in}}{\pgfqpoint{3.983132in}{2.277934in}}%
\pgfpathcurveto{\pgfqpoint{3.975318in}{2.285748in}}{\pgfqpoint{3.964719in}{2.290138in}}{\pgfqpoint{3.953669in}{2.290138in}}%
\pgfpathcurveto{\pgfqpoint{3.942619in}{2.290138in}}{\pgfqpoint{3.932020in}{2.285748in}}{\pgfqpoint{3.924206in}{2.277934in}}%
\pgfpathcurveto{\pgfqpoint{3.916393in}{2.270120in}}{\pgfqpoint{3.912003in}{2.259521in}}{\pgfqpoint{3.912003in}{2.248471in}}%
\pgfpathcurveto{\pgfqpoint{3.912003in}{2.237421in}}{\pgfqpoint{3.916393in}{2.226822in}}{\pgfqpoint{3.924206in}{2.219008in}}%
\pgfpathcurveto{\pgfqpoint{3.932020in}{2.211195in}}{\pgfqpoint{3.942619in}{2.206804in}}{\pgfqpoint{3.953669in}{2.206804in}}%
\pgfpathclose%
\pgfusepath{stroke,fill}%
\end{pgfscope}%
\begin{pgfscope}%
\pgfpathrectangle{\pgfqpoint{1.016621in}{0.499691in}}{\pgfqpoint{3.875000in}{2.695000in}}%
\pgfusepath{clip}%
\pgfsetbuttcap%
\pgfsetroundjoin%
\definecolor{currentfill}{rgb}{0.117647,0.564706,1.000000}%
\pgfsetfillcolor{currentfill}%
\pgfsetlinewidth{1.003750pt}%
\definecolor{currentstroke}{rgb}{0.117647,0.564706,1.000000}%
\pgfsetstrokecolor{currentstroke}%
\pgfsetdash{}{0pt}%
\pgfpathmoveto{\pgfqpoint{3.978503in}{2.266997in}}%
\pgfpathcurveto{\pgfqpoint{3.989553in}{2.266997in}}{\pgfqpoint{4.000152in}{2.271387in}}{\pgfqpoint{4.007966in}{2.279200in}}%
\pgfpathcurveto{\pgfqpoint{4.015779in}{2.287014in}}{\pgfqpoint{4.020169in}{2.297613in}}{\pgfqpoint{4.020169in}{2.308663in}}%
\pgfpathcurveto{\pgfqpoint{4.020169in}{2.319713in}}{\pgfqpoint{4.015779in}{2.330312in}}{\pgfqpoint{4.007966in}{2.338126in}}%
\pgfpathcurveto{\pgfqpoint{4.000152in}{2.345940in}}{\pgfqpoint{3.989553in}{2.350330in}}{\pgfqpoint{3.978503in}{2.350330in}}%
\pgfpathcurveto{\pgfqpoint{3.967453in}{2.350330in}}{\pgfqpoint{3.956854in}{2.345940in}}{\pgfqpoint{3.949040in}{2.338126in}}%
\pgfpathcurveto{\pgfqpoint{3.941226in}{2.330312in}}{\pgfqpoint{3.936836in}{2.319713in}}{\pgfqpoint{3.936836in}{2.308663in}}%
\pgfpathcurveto{\pgfqpoint{3.936836in}{2.297613in}}{\pgfqpoint{3.941226in}{2.287014in}}{\pgfqpoint{3.949040in}{2.279200in}}%
\pgfpathcurveto{\pgfqpoint{3.956854in}{2.271387in}}{\pgfqpoint{3.967453in}{2.266997in}}{\pgfqpoint{3.978503in}{2.266997in}}%
\pgfpathclose%
\pgfusepath{stroke,fill}%
\end{pgfscope}%
\begin{pgfscope}%
\pgfpathrectangle{\pgfqpoint{1.016621in}{0.499691in}}{\pgfqpoint{3.875000in}{2.695000in}}%
\pgfusepath{clip}%
\pgfsetbuttcap%
\pgfsetroundjoin%
\definecolor{currentfill}{rgb}{0.117647,0.564706,1.000000}%
\pgfsetfillcolor{currentfill}%
\pgfsetlinewidth{1.003750pt}%
\definecolor{currentstroke}{rgb}{0.117647,0.564706,1.000000}%
\pgfsetstrokecolor{currentstroke}%
\pgfsetdash{}{0pt}%
\pgfpathmoveto{\pgfqpoint{3.990920in}{2.281328in}}%
\pgfpathcurveto{\pgfqpoint{4.001970in}{2.281328in}}{\pgfqpoint{4.012569in}{2.285718in}}{\pgfqpoint{4.020382in}{2.293532in}}%
\pgfpathcurveto{\pgfqpoint{4.028196in}{2.301345in}}{\pgfqpoint{4.032586in}{2.311944in}}{\pgfqpoint{4.032586in}{2.322995in}}%
\pgfpathcurveto{\pgfqpoint{4.032586in}{2.334045in}}{\pgfqpoint{4.028196in}{2.344644in}}{\pgfqpoint{4.020382in}{2.352457in}}%
\pgfpathcurveto{\pgfqpoint{4.012569in}{2.360271in}}{\pgfqpoint{4.001970in}{2.364661in}}{\pgfqpoint{3.990920in}{2.364661in}}%
\pgfpathcurveto{\pgfqpoint{3.979869in}{2.364661in}}{\pgfqpoint{3.969270in}{2.360271in}}{\pgfqpoint{3.961457in}{2.352457in}}%
\pgfpathcurveto{\pgfqpoint{3.953643in}{2.344644in}}{\pgfqpoint{3.949253in}{2.334045in}}{\pgfqpoint{3.949253in}{2.322995in}}%
\pgfpathcurveto{\pgfqpoint{3.949253in}{2.311944in}}{\pgfqpoint{3.953643in}{2.301345in}}{\pgfqpoint{3.961457in}{2.293532in}}%
\pgfpathcurveto{\pgfqpoint{3.969270in}{2.285718in}}{\pgfqpoint{3.979869in}{2.281328in}}{\pgfqpoint{3.990920in}{2.281328in}}%
\pgfpathclose%
\pgfusepath{stroke,fill}%
\end{pgfscope}%
\begin{pgfscope}%
\pgfpathrectangle{\pgfqpoint{1.016621in}{0.499691in}}{\pgfqpoint{3.875000in}{2.695000in}}%
\pgfusepath{clip}%
\pgfsetbuttcap%
\pgfsetroundjoin%
\definecolor{currentfill}{rgb}{0.117647,0.564706,1.000000}%
\pgfsetfillcolor{currentfill}%
\pgfsetlinewidth{1.003750pt}%
\definecolor{currentstroke}{rgb}{0.117647,0.564706,1.000000}%
\pgfsetstrokecolor{currentstroke}%
\pgfsetdash{}{0pt}%
\pgfpathmoveto{\pgfqpoint{4.003336in}{2.318590in}}%
\pgfpathcurveto{\pgfqpoint{4.014386in}{2.318590in}}{\pgfqpoint{4.024985in}{2.322980in}}{\pgfqpoint{4.032799in}{2.330794in}}%
\pgfpathcurveto{\pgfqpoint{4.040613in}{2.338607in}}{\pgfqpoint{4.045003in}{2.349206in}}{\pgfqpoint{4.045003in}{2.360256in}}%
\pgfpathcurveto{\pgfqpoint{4.045003in}{2.371306in}}{\pgfqpoint{4.040613in}{2.381905in}}{\pgfqpoint{4.032799in}{2.389719in}}%
\pgfpathcurveto{\pgfqpoint{4.024985in}{2.397533in}}{\pgfqpoint{4.014386in}{2.401923in}}{\pgfqpoint{4.003336in}{2.401923in}}%
\pgfpathcurveto{\pgfqpoint{3.992286in}{2.401923in}}{\pgfqpoint{3.981687in}{2.397533in}}{\pgfqpoint{3.973873in}{2.389719in}}%
\pgfpathcurveto{\pgfqpoint{3.966060in}{2.381905in}}{\pgfqpoint{3.961670in}{2.371306in}}{\pgfqpoint{3.961670in}{2.360256in}}%
\pgfpathcurveto{\pgfqpoint{3.961670in}{2.349206in}}{\pgfqpoint{3.966060in}{2.338607in}}{\pgfqpoint{3.973873in}{2.330794in}}%
\pgfpathcurveto{\pgfqpoint{3.981687in}{2.322980in}}{\pgfqpoint{3.992286in}{2.318590in}}{\pgfqpoint{4.003336in}{2.318590in}}%
\pgfpathclose%
\pgfusepath{stroke,fill}%
\end{pgfscope}%
\begin{pgfscope}%
\pgfpathrectangle{\pgfqpoint{1.016621in}{0.499691in}}{\pgfqpoint{3.875000in}{2.695000in}}%
\pgfusepath{clip}%
\pgfsetbuttcap%
\pgfsetroundjoin%
\definecolor{currentfill}{rgb}{0.117647,0.564706,1.000000}%
\pgfsetfillcolor{currentfill}%
\pgfsetlinewidth{1.003750pt}%
\definecolor{currentstroke}{rgb}{0.117647,0.564706,1.000000}%
\pgfsetstrokecolor{currentstroke}%
\pgfsetdash{}{0pt}%
\pgfpathmoveto{\pgfqpoint{4.028170in}{2.341520in}}%
\pgfpathcurveto{\pgfqpoint{4.039220in}{2.341520in}}{\pgfqpoint{4.049819in}{2.345910in}}{\pgfqpoint{4.057633in}{2.353724in}}%
\pgfpathcurveto{\pgfqpoint{4.065446in}{2.361537in}}{\pgfqpoint{4.069836in}{2.372136in}}{\pgfqpoint{4.069836in}{2.383187in}}%
\pgfpathcurveto{\pgfqpoint{4.069836in}{2.394237in}}{\pgfqpoint{4.065446in}{2.404836in}}{\pgfqpoint{4.057633in}{2.412649in}}%
\pgfpathcurveto{\pgfqpoint{4.049819in}{2.420463in}}{\pgfqpoint{4.039220in}{2.424853in}}{\pgfqpoint{4.028170in}{2.424853in}}%
\pgfpathcurveto{\pgfqpoint{4.017120in}{2.424853in}}{\pgfqpoint{4.006521in}{2.420463in}}{\pgfqpoint{3.998707in}{2.412649in}}%
\pgfpathcurveto{\pgfqpoint{3.990893in}{2.404836in}}{\pgfqpoint{3.986503in}{2.394237in}}{\pgfqpoint{3.986503in}{2.383187in}}%
\pgfpathcurveto{\pgfqpoint{3.986503in}{2.372136in}}{\pgfqpoint{3.990893in}{2.361537in}}{\pgfqpoint{3.998707in}{2.353724in}}%
\pgfpathcurveto{\pgfqpoint{4.006521in}{2.345910in}}{\pgfqpoint{4.017120in}{2.341520in}}{\pgfqpoint{4.028170in}{2.341520in}}%
\pgfpathclose%
\pgfusepath{stroke,fill}%
\end{pgfscope}%
\begin{pgfscope}%
\pgfpathrectangle{\pgfqpoint{1.016621in}{0.499691in}}{\pgfqpoint{3.875000in}{2.695000in}}%
\pgfusepath{clip}%
\pgfsetbuttcap%
\pgfsetroundjoin%
\definecolor{currentfill}{rgb}{0.117647,0.564706,1.000000}%
\pgfsetfillcolor{currentfill}%
\pgfsetlinewidth{1.003750pt}%
\definecolor{currentstroke}{rgb}{0.117647,0.564706,1.000000}%
\pgfsetstrokecolor{currentstroke}%
\pgfsetdash{}{0pt}%
\pgfpathmoveto{\pgfqpoint{4.040587in}{2.367317in}}%
\pgfpathcurveto{\pgfqpoint{4.051637in}{2.367317in}}{\pgfqpoint{4.062236in}{2.371707in}}{\pgfqpoint{4.070049in}{2.379520in}}%
\pgfpathcurveto{\pgfqpoint{4.077863in}{2.387334in}}{\pgfqpoint{4.082253in}{2.397933in}}{\pgfqpoint{4.082253in}{2.408983in}}%
\pgfpathcurveto{\pgfqpoint{4.082253in}{2.420033in}}{\pgfqpoint{4.077863in}{2.430632in}}{\pgfqpoint{4.070049in}{2.438446in}}%
\pgfpathcurveto{\pgfqpoint{4.062236in}{2.446260in}}{\pgfqpoint{4.051637in}{2.450650in}}{\pgfqpoint{4.040587in}{2.450650in}}%
\pgfpathcurveto{\pgfqpoint{4.029536in}{2.450650in}}{\pgfqpoint{4.018937in}{2.446260in}}{\pgfqpoint{4.011124in}{2.438446in}}%
\pgfpathcurveto{\pgfqpoint{4.003310in}{2.430632in}}{\pgfqpoint{3.998920in}{2.420033in}}{\pgfqpoint{3.998920in}{2.408983in}}%
\pgfpathcurveto{\pgfqpoint{3.998920in}{2.397933in}}{\pgfqpoint{4.003310in}{2.387334in}}{\pgfqpoint{4.011124in}{2.379520in}}%
\pgfpathcurveto{\pgfqpoint{4.018937in}{2.371707in}}{\pgfqpoint{4.029536in}{2.367317in}}{\pgfqpoint{4.040587in}{2.367317in}}%
\pgfpathclose%
\pgfusepath{stroke,fill}%
\end{pgfscope}%
\begin{pgfscope}%
\pgfpathrectangle{\pgfqpoint{1.016621in}{0.499691in}}{\pgfqpoint{3.875000in}{2.695000in}}%
\pgfusepath{clip}%
\pgfsetbuttcap%
\pgfsetroundjoin%
\definecolor{currentfill}{rgb}{0.117647,0.564706,1.000000}%
\pgfsetfillcolor{currentfill}%
\pgfsetlinewidth{1.003750pt}%
\definecolor{currentstroke}{rgb}{0.117647,0.564706,1.000000}%
\pgfsetstrokecolor{currentstroke}%
\pgfsetdash{}{0pt}%
\pgfpathmoveto{\pgfqpoint{4.065420in}{2.418910in}}%
\pgfpathcurveto{\pgfqpoint{4.076470in}{2.418910in}}{\pgfqpoint{4.087069in}{2.423300in}}{\pgfqpoint{4.094883in}{2.431114in}}%
\pgfpathcurveto{\pgfqpoint{4.102696in}{2.438927in}}{\pgfqpoint{4.107087in}{2.449526in}}{\pgfqpoint{4.107087in}{2.460576in}}%
\pgfpathcurveto{\pgfqpoint{4.107087in}{2.471626in}}{\pgfqpoint{4.102696in}{2.482225in}}{\pgfqpoint{4.094883in}{2.490039in}}%
\pgfpathcurveto{\pgfqpoint{4.087069in}{2.497853in}}{\pgfqpoint{4.076470in}{2.502243in}}{\pgfqpoint{4.065420in}{2.502243in}}%
\pgfpathcurveto{\pgfqpoint{4.054370in}{2.502243in}}{\pgfqpoint{4.043771in}{2.497853in}}{\pgfqpoint{4.035957in}{2.490039in}}%
\pgfpathcurveto{\pgfqpoint{4.028144in}{2.482225in}}{\pgfqpoint{4.023753in}{2.471626in}}{\pgfqpoint{4.023753in}{2.460576in}}%
\pgfpathcurveto{\pgfqpoint{4.023753in}{2.449526in}}{\pgfqpoint{4.028144in}{2.438927in}}{\pgfqpoint{4.035957in}{2.431114in}}%
\pgfpathcurveto{\pgfqpoint{4.043771in}{2.423300in}}{\pgfqpoint{4.054370in}{2.418910in}}{\pgfqpoint{4.065420in}{2.418910in}}%
\pgfpathclose%
\pgfusepath{stroke,fill}%
\end{pgfscope}%
\begin{pgfscope}%
\pgfpathrectangle{\pgfqpoint{1.016621in}{0.499691in}}{\pgfqpoint{3.875000in}{2.695000in}}%
\pgfusepath{clip}%
\pgfsetbuttcap%
\pgfsetroundjoin%
\definecolor{currentfill}{rgb}{0.117647,0.564706,1.000000}%
\pgfsetfillcolor{currentfill}%
\pgfsetlinewidth{1.003750pt}%
\definecolor{currentstroke}{rgb}{0.117647,0.564706,1.000000}%
\pgfsetstrokecolor{currentstroke}%
\pgfsetdash{}{0pt}%
\pgfpathmoveto{\pgfqpoint{4.090254in}{2.510631in}}%
\pgfpathcurveto{\pgfqpoint{4.101304in}{2.510631in}}{\pgfqpoint{4.111903in}{2.515021in}}{\pgfqpoint{4.119716in}{2.522835in}}%
\pgfpathcurveto{\pgfqpoint{4.127530in}{2.530648in}}{\pgfqpoint{4.131920in}{2.541247in}}{\pgfqpoint{4.131920in}{2.552297in}}%
\pgfpathcurveto{\pgfqpoint{4.131920in}{2.563348in}}{\pgfqpoint{4.127530in}{2.573947in}}{\pgfqpoint{4.119716in}{2.581760in}}%
\pgfpathcurveto{\pgfqpoint{4.111903in}{2.589574in}}{\pgfqpoint{4.101304in}{2.593964in}}{\pgfqpoint{4.090254in}{2.593964in}}%
\pgfpathcurveto{\pgfqpoint{4.079203in}{2.593964in}}{\pgfqpoint{4.068604in}{2.589574in}}{\pgfqpoint{4.060791in}{2.581760in}}%
\pgfpathcurveto{\pgfqpoint{4.052977in}{2.573947in}}{\pgfqpoint{4.048587in}{2.563348in}}{\pgfqpoint{4.048587in}{2.552297in}}%
\pgfpathcurveto{\pgfqpoint{4.048587in}{2.541247in}}{\pgfqpoint{4.052977in}{2.530648in}}{\pgfqpoint{4.060791in}{2.522835in}}%
\pgfpathcurveto{\pgfqpoint{4.068604in}{2.515021in}}{\pgfqpoint{4.079203in}{2.510631in}}{\pgfqpoint{4.090254in}{2.510631in}}%
\pgfpathclose%
\pgfusepath{stroke,fill}%
\end{pgfscope}%
\begin{pgfscope}%
\pgfpathrectangle{\pgfqpoint{1.016621in}{0.499691in}}{\pgfqpoint{3.875000in}{2.695000in}}%
\pgfusepath{clip}%
\pgfsetbuttcap%
\pgfsetroundjoin%
\definecolor{currentfill}{rgb}{0.117647,0.564706,1.000000}%
\pgfsetfillcolor{currentfill}%
\pgfsetlinewidth{1.003750pt}%
\definecolor{currentstroke}{rgb}{0.117647,0.564706,1.000000}%
\pgfsetstrokecolor{currentstroke}%
\pgfsetdash{}{0pt}%
\pgfpathmoveto{\pgfqpoint{4.115087in}{2.530695in}}%
\pgfpathcurveto{\pgfqpoint{4.126137in}{2.530695in}}{\pgfqpoint{4.136736in}{2.535085in}}{\pgfqpoint{4.144550in}{2.542899in}}%
\pgfpathcurveto{\pgfqpoint{4.152363in}{2.550712in}}{\pgfqpoint{4.156754in}{2.561311in}}{\pgfqpoint{4.156754in}{2.572361in}}%
\pgfpathcurveto{\pgfqpoint{4.156754in}{2.583412in}}{\pgfqpoint{4.152363in}{2.594011in}}{\pgfqpoint{4.144550in}{2.601824in}}%
\pgfpathcurveto{\pgfqpoint{4.136736in}{2.609638in}}{\pgfqpoint{4.126137in}{2.614028in}}{\pgfqpoint{4.115087in}{2.614028in}}%
\pgfpathcurveto{\pgfqpoint{4.104037in}{2.614028in}}{\pgfqpoint{4.093438in}{2.609638in}}{\pgfqpoint{4.085624in}{2.601824in}}%
\pgfpathcurveto{\pgfqpoint{4.077811in}{2.594011in}}{\pgfqpoint{4.073420in}{2.583412in}}{\pgfqpoint{4.073420in}{2.572361in}}%
\pgfpathcurveto{\pgfqpoint{4.073420in}{2.561311in}}{\pgfqpoint{4.077811in}{2.550712in}}{\pgfqpoint{4.085624in}{2.542899in}}%
\pgfpathcurveto{\pgfqpoint{4.093438in}{2.535085in}}{\pgfqpoint{4.104037in}{2.530695in}}{\pgfqpoint{4.115087in}{2.530695in}}%
\pgfpathclose%
\pgfusepath{stroke,fill}%
\end{pgfscope}%
\begin{pgfscope}%
\pgfpathrectangle{\pgfqpoint{1.016621in}{0.499691in}}{\pgfqpoint{3.875000in}{2.695000in}}%
\pgfusepath{clip}%
\pgfsetbuttcap%
\pgfsetroundjoin%
\definecolor{currentfill}{rgb}{0.117647,0.564706,1.000000}%
\pgfsetfillcolor{currentfill}%
\pgfsetlinewidth{1.003750pt}%
\definecolor{currentstroke}{rgb}{0.117647,0.564706,1.000000}%
\pgfsetstrokecolor{currentstroke}%
\pgfsetdash{}{0pt}%
\pgfpathmoveto{\pgfqpoint{4.177171in}{2.562224in}}%
\pgfpathcurveto{\pgfqpoint{4.188221in}{2.562224in}}{\pgfqpoint{4.198820in}{2.566614in}}{\pgfqpoint{4.206634in}{2.574428in}}%
\pgfpathcurveto{\pgfqpoint{4.214447in}{2.582241in}}{\pgfqpoint{4.218837in}{2.592841in}}{\pgfqpoint{4.218837in}{2.603891in}}%
\pgfpathcurveto{\pgfqpoint{4.218837in}{2.614941in}}{\pgfqpoint{4.214447in}{2.625540in}}{\pgfqpoint{4.206634in}{2.633353in}}%
\pgfpathcurveto{\pgfqpoint{4.198820in}{2.641167in}}{\pgfqpoint{4.188221in}{2.645557in}}{\pgfqpoint{4.177171in}{2.645557in}}%
\pgfpathcurveto{\pgfqpoint{4.166121in}{2.645557in}}{\pgfqpoint{4.155522in}{2.641167in}}{\pgfqpoint{4.147708in}{2.633353in}}%
\pgfpathcurveto{\pgfqpoint{4.139894in}{2.625540in}}{\pgfqpoint{4.135504in}{2.614941in}}{\pgfqpoint{4.135504in}{2.603891in}}%
\pgfpathcurveto{\pgfqpoint{4.135504in}{2.592841in}}{\pgfqpoint{4.139894in}{2.582241in}}{\pgfqpoint{4.147708in}{2.574428in}}%
\pgfpathcurveto{\pgfqpoint{4.155522in}{2.566614in}}{\pgfqpoint{4.166121in}{2.562224in}}{\pgfqpoint{4.177171in}{2.562224in}}%
\pgfpathclose%
\pgfusepath{stroke,fill}%
\end{pgfscope}%
\begin{pgfscope}%
\pgfpathrectangle{\pgfqpoint{1.016621in}{0.499691in}}{\pgfqpoint{3.875000in}{2.695000in}}%
\pgfusepath{clip}%
\pgfsetbuttcap%
\pgfsetroundjoin%
\definecolor{currentfill}{rgb}{0.117647,0.564706,1.000000}%
\pgfsetfillcolor{currentfill}%
\pgfsetlinewidth{1.003750pt}%
\definecolor{currentstroke}{rgb}{0.117647,0.564706,1.000000}%
\pgfsetstrokecolor{currentstroke}%
\pgfsetdash{}{0pt}%
\pgfpathmoveto{\pgfqpoint{4.226838in}{2.676875in}}%
\pgfpathcurveto{\pgfqpoint{4.237888in}{2.676875in}}{\pgfqpoint{4.248487in}{2.681266in}}{\pgfqpoint{4.256301in}{2.689079in}}%
\pgfpathcurveto{\pgfqpoint{4.264114in}{2.696893in}}{\pgfqpoint{4.268504in}{2.707492in}}{\pgfqpoint{4.268504in}{2.718542in}}%
\pgfpathcurveto{\pgfqpoint{4.268504in}{2.729592in}}{\pgfqpoint{4.264114in}{2.740191in}}{\pgfqpoint{4.256301in}{2.748005in}}%
\pgfpathcurveto{\pgfqpoint{4.248487in}{2.755818in}}{\pgfqpoint{4.237888in}{2.760209in}}{\pgfqpoint{4.226838in}{2.760209in}}%
\pgfpathcurveto{\pgfqpoint{4.215788in}{2.760209in}}{\pgfqpoint{4.205189in}{2.755818in}}{\pgfqpoint{4.197375in}{2.748005in}}%
\pgfpathcurveto{\pgfqpoint{4.189561in}{2.740191in}}{\pgfqpoint{4.185171in}{2.729592in}}{\pgfqpoint{4.185171in}{2.718542in}}%
\pgfpathcurveto{\pgfqpoint{4.185171in}{2.707492in}}{\pgfqpoint{4.189561in}{2.696893in}}{\pgfqpoint{4.197375in}{2.689079in}}%
\pgfpathcurveto{\pgfqpoint{4.205189in}{2.681266in}}{\pgfqpoint{4.215788in}{2.676875in}}{\pgfqpoint{4.226838in}{2.676875in}}%
\pgfpathclose%
\pgfusepath{stroke,fill}%
\end{pgfscope}%
\begin{pgfscope}%
\pgfpathrectangle{\pgfqpoint{1.016621in}{0.499691in}}{\pgfqpoint{3.875000in}{2.695000in}}%
\pgfusepath{clip}%
\pgfsetbuttcap%
\pgfsetroundjoin%
\definecolor{currentfill}{rgb}{0.117647,0.564706,1.000000}%
\pgfsetfillcolor{currentfill}%
\pgfsetlinewidth{1.003750pt}%
\definecolor{currentstroke}{rgb}{0.117647,0.564706,1.000000}%
\pgfsetstrokecolor{currentstroke}%
\pgfsetdash{}{0pt}%
\pgfpathmoveto{\pgfqpoint{4.400672in}{2.848853in}}%
\pgfpathcurveto{\pgfqpoint{4.411722in}{2.848853in}}{\pgfqpoint{4.422321in}{2.853243in}}{\pgfqpoint{4.430135in}{2.861056in}}%
\pgfpathcurveto{\pgfqpoint{4.437949in}{2.868870in}}{\pgfqpoint{4.442339in}{2.879469in}}{\pgfqpoint{4.442339in}{2.890519in}}%
\pgfpathcurveto{\pgfqpoint{4.442339in}{2.901569in}}{\pgfqpoint{4.437949in}{2.912168in}}{\pgfqpoint{4.430135in}{2.919982in}}%
\pgfpathcurveto{\pgfqpoint{4.422321in}{2.927796in}}{\pgfqpoint{4.411722in}{2.932186in}}{\pgfqpoint{4.400672in}{2.932186in}}%
\pgfpathcurveto{\pgfqpoint{4.389622in}{2.932186in}}{\pgfqpoint{4.379023in}{2.927796in}}{\pgfqpoint{4.371209in}{2.919982in}}%
\pgfpathcurveto{\pgfqpoint{4.363396in}{2.912168in}}{\pgfqpoint{4.359006in}{2.901569in}}{\pgfqpoint{4.359006in}{2.890519in}}%
\pgfpathcurveto{\pgfqpoint{4.359006in}{2.879469in}}{\pgfqpoint{4.363396in}{2.868870in}}{\pgfqpoint{4.371209in}{2.861056in}}%
\pgfpathcurveto{\pgfqpoint{4.379023in}{2.853243in}}{\pgfqpoint{4.389622in}{2.848853in}}{\pgfqpoint{4.400672in}{2.848853in}}%
\pgfpathclose%
\pgfusepath{stroke,fill}%
\end{pgfscope}%
\begin{pgfscope}%
\pgfpathrectangle{\pgfqpoint{1.016621in}{0.499691in}}{\pgfqpoint{3.875000in}{2.695000in}}%
\pgfusepath{clip}%
\pgfsetbuttcap%
\pgfsetroundjoin%
\definecolor{currentfill}{rgb}{0.117647,0.564706,1.000000}%
\pgfsetfillcolor{currentfill}%
\pgfsetlinewidth{1.003750pt}%
\definecolor{currentstroke}{rgb}{0.117647,0.564706,1.000000}%
\pgfsetstrokecolor{currentstroke}%
\pgfsetdash{}{0pt}%
\pgfpathmoveto{\pgfqpoint{4.711091in}{3.020830in}}%
\pgfpathcurveto{\pgfqpoint{4.722141in}{3.020830in}}{\pgfqpoint{4.732740in}{3.025220in}}{\pgfqpoint{4.740554in}{3.033034in}}%
\pgfpathcurveto{\pgfqpoint{4.748367in}{3.040847in}}{\pgfqpoint{4.752758in}{3.051446in}}{\pgfqpoint{4.752758in}{3.062496in}}%
\pgfpathcurveto{\pgfqpoint{4.752758in}{3.073547in}}{\pgfqpoint{4.748367in}{3.084146in}}{\pgfqpoint{4.740554in}{3.091959in}}%
\pgfpathcurveto{\pgfqpoint{4.732740in}{3.099773in}}{\pgfqpoint{4.722141in}{3.104163in}}{\pgfqpoint{4.711091in}{3.104163in}}%
\pgfpathcurveto{\pgfqpoint{4.700041in}{3.104163in}}{\pgfqpoint{4.689442in}{3.099773in}}{\pgfqpoint{4.681628in}{3.091959in}}%
\pgfpathcurveto{\pgfqpoint{4.673815in}{3.084146in}}{\pgfqpoint{4.669424in}{3.073547in}}{\pgfqpoint{4.669424in}{3.062496in}}%
\pgfpathcurveto{\pgfqpoint{4.669424in}{3.051446in}}{\pgfqpoint{4.673815in}{3.040847in}}{\pgfqpoint{4.681628in}{3.033034in}}%
\pgfpathcurveto{\pgfqpoint{4.689442in}{3.025220in}}{\pgfqpoint{4.700041in}{3.020830in}}{\pgfqpoint{4.711091in}{3.020830in}}%
\pgfpathclose%
\pgfusepath{stroke,fill}%
\end{pgfscope}%
\begin{pgfscope}%
\pgfsetbuttcap%
\pgfsetroundjoin%
\definecolor{currentfill}{rgb}{0.000000,0.000000,0.000000}%
\pgfsetfillcolor{currentfill}%
\pgfsetlinewidth{0.803000pt}%
\definecolor{currentstroke}{rgb}{0.000000,0.000000,0.000000}%
\pgfsetstrokecolor{currentstroke}%
\pgfsetdash{}{0pt}%
\pgfsys@defobject{currentmarker}{\pgfqpoint{0.000000in}{-0.048611in}}{\pgfqpoint{0.000000in}{0.000000in}}{%
\pgfpathmoveto{\pgfqpoint{0.000000in}{0.000000in}}%
\pgfpathlineto{\pgfqpoint{0.000000in}{-0.048611in}}%
\pgfusepath{stroke,fill}%
}%
\begin{pgfscope}%
\pgfsys@transformshift{1.197151in}{0.499691in}%
\pgfsys@useobject{currentmarker}{}%
\end{pgfscope}%
\end{pgfscope}%
\begin{pgfscope}%
\definecolor{textcolor}{rgb}{0.000000,0.000000,0.000000}%
\pgfsetstrokecolor{textcolor}%
\pgfsetfillcolor{textcolor}%
\pgftext[x=1.197151in,y=0.402469in,,top]{\color{textcolor}\rmfamily\fontsize{10.000000}{12.000000}\selectfont \(\displaystyle 1.0\)}%
\end{pgfscope}%
\begin{pgfscope}%
\pgfsetbuttcap%
\pgfsetroundjoin%
\definecolor{currentfill}{rgb}{0.000000,0.000000,0.000000}%
\pgfsetfillcolor{currentfill}%
\pgfsetlinewidth{0.803000pt}%
\definecolor{currentstroke}{rgb}{0.000000,0.000000,0.000000}%
\pgfsetstrokecolor{currentstroke}%
\pgfsetdash{}{0pt}%
\pgfsys@defobject{currentmarker}{\pgfqpoint{0.000000in}{-0.048611in}}{\pgfqpoint{0.000000in}{0.000000in}}{%
\pgfpathmoveto{\pgfqpoint{0.000000in}{0.000000in}}%
\pgfpathlineto{\pgfqpoint{0.000000in}{-0.048611in}}%
\pgfusepath{stroke,fill}%
}%
\begin{pgfscope}%
\pgfsys@transformshift{1.817988in}{0.499691in}%
\pgfsys@useobject{currentmarker}{}%
\end{pgfscope}%
\end{pgfscope}%
\begin{pgfscope}%
\definecolor{textcolor}{rgb}{0.000000,0.000000,0.000000}%
\pgfsetstrokecolor{textcolor}%
\pgfsetfillcolor{textcolor}%
\pgftext[x=1.817988in,y=0.402469in,,top]{\color{textcolor}\rmfamily\fontsize{10.000000}{12.000000}\selectfont \(\displaystyle 1.5\)}%
\end{pgfscope}%
\begin{pgfscope}%
\pgfsetbuttcap%
\pgfsetroundjoin%
\definecolor{currentfill}{rgb}{0.000000,0.000000,0.000000}%
\pgfsetfillcolor{currentfill}%
\pgfsetlinewidth{0.803000pt}%
\definecolor{currentstroke}{rgb}{0.000000,0.000000,0.000000}%
\pgfsetstrokecolor{currentstroke}%
\pgfsetdash{}{0pt}%
\pgfsys@defobject{currentmarker}{\pgfqpoint{0.000000in}{-0.048611in}}{\pgfqpoint{0.000000in}{0.000000in}}{%
\pgfpathmoveto{\pgfqpoint{0.000000in}{0.000000in}}%
\pgfpathlineto{\pgfqpoint{0.000000in}{-0.048611in}}%
\pgfusepath{stroke,fill}%
}%
\begin{pgfscope}%
\pgfsys@transformshift{2.438826in}{0.499691in}%
\pgfsys@useobject{currentmarker}{}%
\end{pgfscope}%
\end{pgfscope}%
\begin{pgfscope}%
\definecolor{textcolor}{rgb}{0.000000,0.000000,0.000000}%
\pgfsetstrokecolor{textcolor}%
\pgfsetfillcolor{textcolor}%
\pgftext[x=2.438826in,y=0.402469in,,top]{\color{textcolor}\rmfamily\fontsize{10.000000}{12.000000}\selectfont \(\displaystyle 2.0\)}%
\end{pgfscope}%
\begin{pgfscope}%
\pgfsetbuttcap%
\pgfsetroundjoin%
\definecolor{currentfill}{rgb}{0.000000,0.000000,0.000000}%
\pgfsetfillcolor{currentfill}%
\pgfsetlinewidth{0.803000pt}%
\definecolor{currentstroke}{rgb}{0.000000,0.000000,0.000000}%
\pgfsetstrokecolor{currentstroke}%
\pgfsetdash{}{0pt}%
\pgfsys@defobject{currentmarker}{\pgfqpoint{0.000000in}{-0.048611in}}{\pgfqpoint{0.000000in}{0.000000in}}{%
\pgfpathmoveto{\pgfqpoint{0.000000in}{0.000000in}}%
\pgfpathlineto{\pgfqpoint{0.000000in}{-0.048611in}}%
\pgfusepath{stroke,fill}%
}%
\begin{pgfscope}%
\pgfsys@transformshift{3.059663in}{0.499691in}%
\pgfsys@useobject{currentmarker}{}%
\end{pgfscope}%
\end{pgfscope}%
\begin{pgfscope}%
\definecolor{textcolor}{rgb}{0.000000,0.000000,0.000000}%
\pgfsetstrokecolor{textcolor}%
\pgfsetfillcolor{textcolor}%
\pgftext[x=3.059663in,y=0.402469in,,top]{\color{textcolor}\rmfamily\fontsize{10.000000}{12.000000}\selectfont \(\displaystyle 2.5\)}%
\end{pgfscope}%
\begin{pgfscope}%
\pgfsetbuttcap%
\pgfsetroundjoin%
\definecolor{currentfill}{rgb}{0.000000,0.000000,0.000000}%
\pgfsetfillcolor{currentfill}%
\pgfsetlinewidth{0.803000pt}%
\definecolor{currentstroke}{rgb}{0.000000,0.000000,0.000000}%
\pgfsetstrokecolor{currentstroke}%
\pgfsetdash{}{0pt}%
\pgfsys@defobject{currentmarker}{\pgfqpoint{0.000000in}{-0.048611in}}{\pgfqpoint{0.000000in}{0.000000in}}{%
\pgfpathmoveto{\pgfqpoint{0.000000in}{0.000000in}}%
\pgfpathlineto{\pgfqpoint{0.000000in}{-0.048611in}}%
\pgfusepath{stroke,fill}%
}%
\begin{pgfscope}%
\pgfsys@transformshift{3.680501in}{0.499691in}%
\pgfsys@useobject{currentmarker}{}%
\end{pgfscope}%
\end{pgfscope}%
\begin{pgfscope}%
\definecolor{textcolor}{rgb}{0.000000,0.000000,0.000000}%
\pgfsetstrokecolor{textcolor}%
\pgfsetfillcolor{textcolor}%
\pgftext[x=3.680501in,y=0.402469in,,top]{\color{textcolor}\rmfamily\fontsize{10.000000}{12.000000}\selectfont \(\displaystyle 3.0\)}%
\end{pgfscope}%
\begin{pgfscope}%
\pgfsetbuttcap%
\pgfsetroundjoin%
\definecolor{currentfill}{rgb}{0.000000,0.000000,0.000000}%
\pgfsetfillcolor{currentfill}%
\pgfsetlinewidth{0.803000pt}%
\definecolor{currentstroke}{rgb}{0.000000,0.000000,0.000000}%
\pgfsetstrokecolor{currentstroke}%
\pgfsetdash{}{0pt}%
\pgfsys@defobject{currentmarker}{\pgfqpoint{0.000000in}{-0.048611in}}{\pgfqpoint{0.000000in}{0.000000in}}{%
\pgfpathmoveto{\pgfqpoint{0.000000in}{0.000000in}}%
\pgfpathlineto{\pgfqpoint{0.000000in}{-0.048611in}}%
\pgfusepath{stroke,fill}%
}%
\begin{pgfscope}%
\pgfsys@transformshift{4.301338in}{0.499691in}%
\pgfsys@useobject{currentmarker}{}%
\end{pgfscope}%
\end{pgfscope}%
\begin{pgfscope}%
\definecolor{textcolor}{rgb}{0.000000,0.000000,0.000000}%
\pgfsetstrokecolor{textcolor}%
\pgfsetfillcolor{textcolor}%
\pgftext[x=4.301338in,y=0.402469in,,top]{\color{textcolor}\rmfamily\fontsize{10.000000}{12.000000}\selectfont \(\displaystyle 3.5\)}%
\end{pgfscope}%
\begin{pgfscope}%
\definecolor{textcolor}{rgb}{0.000000,0.000000,0.000000}%
\pgfsetstrokecolor{textcolor}%
\pgfsetfillcolor{textcolor}%
\pgftext[x=2.954121in,y=0.223457in,,top]{\color{textcolor}\rmfamily\fontsize{10.000000}{12.000000}\selectfont log f}%
\end{pgfscope}%
\begin{pgfscope}%
\pgfsetbuttcap%
\pgfsetroundjoin%
\definecolor{currentfill}{rgb}{0.000000,0.000000,0.000000}%
\pgfsetfillcolor{currentfill}%
\pgfsetlinewidth{0.803000pt}%
\definecolor{currentstroke}{rgb}{0.000000,0.000000,0.000000}%
\pgfsetstrokecolor{currentstroke}%
\pgfsetdash{}{0pt}%
\pgfsys@defobject{currentmarker}{\pgfqpoint{-0.048611in}{0.000000in}}{\pgfqpoint{0.000000in}{0.000000in}}{%
\pgfpathmoveto{\pgfqpoint{0.000000in}{0.000000in}}%
\pgfpathlineto{\pgfqpoint{-0.048611in}{0.000000in}}%
\pgfusepath{stroke,fill}%
}%
\begin{pgfscope}%
\pgfsys@transformshift{1.016621in}{0.614688in}%
\pgfsys@useobject{currentmarker}{}%
\end{pgfscope}%
\end{pgfscope}%
\begin{pgfscope}%
\definecolor{textcolor}{rgb}{0.000000,0.000000,0.000000}%
\pgfsetstrokecolor{textcolor}%
\pgfsetfillcolor{textcolor}%
\pgftext[x=0.603040in,y=0.566463in,left,base]{\color{textcolor}\rmfamily\fontsize{10.000000}{12.000000}\selectfont \(\displaystyle -100\)}%
\end{pgfscope}%
\begin{pgfscope}%
\pgfsetbuttcap%
\pgfsetroundjoin%
\definecolor{currentfill}{rgb}{0.000000,0.000000,0.000000}%
\pgfsetfillcolor{currentfill}%
\pgfsetlinewidth{0.803000pt}%
\definecolor{currentstroke}{rgb}{0.000000,0.000000,0.000000}%
\pgfsetstrokecolor{currentstroke}%
\pgfsetdash{}{0pt}%
\pgfsys@defobject{currentmarker}{\pgfqpoint{-0.048611in}{0.000000in}}{\pgfqpoint{0.000000in}{0.000000in}}{%
\pgfpathmoveto{\pgfqpoint{0.000000in}{0.000000in}}%
\pgfpathlineto{\pgfqpoint{-0.048611in}{0.000000in}}%
\pgfusepath{stroke,fill}%
}%
\begin{pgfscope}%
\pgfsys@transformshift{1.016621in}{1.187945in}%
\pgfsys@useobject{currentmarker}{}%
\end{pgfscope}%
\end{pgfscope}%
\begin{pgfscope}%
\definecolor{textcolor}{rgb}{0.000000,0.000000,0.000000}%
\pgfsetstrokecolor{textcolor}%
\pgfsetfillcolor{textcolor}%
\pgftext[x=0.672484in,y=1.139720in,left,base]{\color{textcolor}\rmfamily\fontsize{10.000000}{12.000000}\selectfont \(\displaystyle -80\)}%
\end{pgfscope}%
\begin{pgfscope}%
\pgfsetbuttcap%
\pgfsetroundjoin%
\definecolor{currentfill}{rgb}{0.000000,0.000000,0.000000}%
\pgfsetfillcolor{currentfill}%
\pgfsetlinewidth{0.803000pt}%
\definecolor{currentstroke}{rgb}{0.000000,0.000000,0.000000}%
\pgfsetstrokecolor{currentstroke}%
\pgfsetdash{}{0pt}%
\pgfsys@defobject{currentmarker}{\pgfqpoint{-0.048611in}{0.000000in}}{\pgfqpoint{0.000000in}{0.000000in}}{%
\pgfpathmoveto{\pgfqpoint{0.000000in}{0.000000in}}%
\pgfpathlineto{\pgfqpoint{-0.048611in}{0.000000in}}%
\pgfusepath{stroke,fill}%
}%
\begin{pgfscope}%
\pgfsys@transformshift{1.016621in}{1.761203in}%
\pgfsys@useobject{currentmarker}{}%
\end{pgfscope}%
\end{pgfscope}%
\begin{pgfscope}%
\definecolor{textcolor}{rgb}{0.000000,0.000000,0.000000}%
\pgfsetstrokecolor{textcolor}%
\pgfsetfillcolor{textcolor}%
\pgftext[x=0.672484in,y=1.712977in,left,base]{\color{textcolor}\rmfamily\fontsize{10.000000}{12.000000}\selectfont \(\displaystyle -60\)}%
\end{pgfscope}%
\begin{pgfscope}%
\pgfsetbuttcap%
\pgfsetroundjoin%
\definecolor{currentfill}{rgb}{0.000000,0.000000,0.000000}%
\pgfsetfillcolor{currentfill}%
\pgfsetlinewidth{0.803000pt}%
\definecolor{currentstroke}{rgb}{0.000000,0.000000,0.000000}%
\pgfsetstrokecolor{currentstroke}%
\pgfsetdash{}{0pt}%
\pgfsys@defobject{currentmarker}{\pgfqpoint{-0.048611in}{0.000000in}}{\pgfqpoint{0.000000in}{0.000000in}}{%
\pgfpathmoveto{\pgfqpoint{0.000000in}{0.000000in}}%
\pgfpathlineto{\pgfqpoint{-0.048611in}{0.000000in}}%
\pgfusepath{stroke,fill}%
}%
\begin{pgfscope}%
\pgfsys@transformshift{1.016621in}{2.334460in}%
\pgfsys@useobject{currentmarker}{}%
\end{pgfscope}%
\end{pgfscope}%
\begin{pgfscope}%
\definecolor{textcolor}{rgb}{0.000000,0.000000,0.000000}%
\pgfsetstrokecolor{textcolor}%
\pgfsetfillcolor{textcolor}%
\pgftext[x=0.672484in,y=2.286234in,left,base]{\color{textcolor}\rmfamily\fontsize{10.000000}{12.000000}\selectfont \(\displaystyle -40\)}%
\end{pgfscope}%
\begin{pgfscope}%
\pgfsetbuttcap%
\pgfsetroundjoin%
\definecolor{currentfill}{rgb}{0.000000,0.000000,0.000000}%
\pgfsetfillcolor{currentfill}%
\pgfsetlinewidth{0.803000pt}%
\definecolor{currentstroke}{rgb}{0.000000,0.000000,0.000000}%
\pgfsetstrokecolor{currentstroke}%
\pgfsetdash{}{0pt}%
\pgfsys@defobject{currentmarker}{\pgfqpoint{-0.048611in}{0.000000in}}{\pgfqpoint{0.000000in}{0.000000in}}{%
\pgfpathmoveto{\pgfqpoint{0.000000in}{0.000000in}}%
\pgfpathlineto{\pgfqpoint{-0.048611in}{0.000000in}}%
\pgfusepath{stroke,fill}%
}%
\begin{pgfscope}%
\pgfsys@transformshift{1.016621in}{2.907717in}%
\pgfsys@useobject{currentmarker}{}%
\end{pgfscope}%
\end{pgfscope}%
\begin{pgfscope}%
\definecolor{textcolor}{rgb}{0.000000,0.000000,0.000000}%
\pgfsetstrokecolor{textcolor}%
\pgfsetfillcolor{textcolor}%
\pgftext[x=0.672484in,y=2.859492in,left,base]{\color{textcolor}\rmfamily\fontsize{10.000000}{12.000000}\selectfont \(\displaystyle -20\)}%
\end{pgfscope}%
\begin{pgfscope}%
\definecolor{textcolor}{rgb}{0.000000,0.000000,0.000000}%
\pgfsetstrokecolor{textcolor}%
\pgfsetfillcolor{textcolor}%
\pgftext[x=0.255817in,y=1.847191in,,bottom]{\color{textcolor}\rmfamily\fontsize{10.000000}{12.000000}\selectfont \(\displaystyle \phi\) \textit{(º)}}%
\end{pgfscope}%
\begin{pgfscope}%
\pgfpathrectangle{\pgfqpoint{1.016621in}{0.499691in}}{\pgfqpoint{3.875000in}{2.695000in}}%
\pgfusepath{clip}%
\pgfsetbuttcap%
\pgfsetroundjoin%
\definecolor{currentfill}{rgb}{0.121569,0.466667,0.705882}%
\pgfsetfillcolor{currentfill}%
\pgfsetlinewidth{1.003750pt}%
\definecolor{currentstroke}{rgb}{0.121569,0.466667,0.705882}%
\pgfsetstrokecolor{currentstroke}%
\pgfsetdash{}{0pt}%
\pgfpathmoveto{\pgfqpoint{3.531500in}{1.619216in}}%
\pgfpathcurveto{\pgfqpoint{3.542550in}{1.619216in}}{\pgfqpoint{3.553149in}{1.623606in}}{\pgfqpoint{3.560963in}{1.631420in}}%
\pgfpathcurveto{\pgfqpoint{3.568776in}{1.639233in}}{\pgfqpoint{3.573166in}{1.649832in}}{\pgfqpoint{3.573166in}{1.660883in}}%
\pgfpathcurveto{\pgfqpoint{3.573166in}{1.671933in}}{\pgfqpoint{3.568776in}{1.682532in}}{\pgfqpoint{3.560963in}{1.690345in}}%
\pgfpathcurveto{\pgfqpoint{3.553149in}{1.698159in}}{\pgfqpoint{3.542550in}{1.702549in}}{\pgfqpoint{3.531500in}{1.702549in}}%
\pgfpathcurveto{\pgfqpoint{3.520450in}{1.702549in}}{\pgfqpoint{3.509851in}{1.698159in}}{\pgfqpoint{3.502037in}{1.690345in}}%
\pgfpathcurveto{\pgfqpoint{3.494223in}{1.682532in}}{\pgfqpoint{3.489833in}{1.671933in}}{\pgfqpoint{3.489833in}{1.660883in}}%
\pgfpathcurveto{\pgfqpoint{3.489833in}{1.649832in}}{\pgfqpoint{3.494223in}{1.639233in}}{\pgfqpoint{3.502037in}{1.631420in}}%
\pgfpathcurveto{\pgfqpoint{3.509851in}{1.623606in}}{\pgfqpoint{3.520450in}{1.619216in}}{\pgfqpoint{3.531500in}{1.619216in}}%
\pgfpathclose%
\pgfusepath{stroke,fill}%
\end{pgfscope}%
\begin{pgfscope}%
\pgfpathrectangle{\pgfqpoint{1.016621in}{0.499691in}}{\pgfqpoint{3.875000in}{2.695000in}}%
\pgfusepath{clip}%
\pgfsetbuttcap%
\pgfsetroundjoin%
\definecolor{currentfill}{rgb}{0.121569,0.466667,0.705882}%
\pgfsetfillcolor{currentfill}%
\pgfsetlinewidth{1.003750pt}%
\definecolor{currentstroke}{rgb}{0.121569,0.466667,0.705882}%
\pgfsetstrokecolor{currentstroke}%
\pgfsetdash{}{0pt}%
\pgfpathmoveto{\pgfqpoint{3.568750in}{1.579088in}}%
\pgfpathcurveto{\pgfqpoint{3.579800in}{1.579088in}}{\pgfqpoint{3.590399in}{1.583478in}}{\pgfqpoint{3.598213in}{1.591292in}}%
\pgfpathcurveto{\pgfqpoint{3.606026in}{1.599105in}}{\pgfqpoint{3.610417in}{1.609704in}}{\pgfqpoint{3.610417in}{1.620755in}}%
\pgfpathcurveto{\pgfqpoint{3.610417in}{1.631805in}}{\pgfqpoint{3.606026in}{1.642404in}}{\pgfqpoint{3.598213in}{1.650217in}}%
\pgfpathcurveto{\pgfqpoint{3.590399in}{1.658031in}}{\pgfqpoint{3.579800in}{1.662421in}}{\pgfqpoint{3.568750in}{1.662421in}}%
\pgfpathcurveto{\pgfqpoint{3.557700in}{1.662421in}}{\pgfqpoint{3.547101in}{1.658031in}}{\pgfqpoint{3.539287in}{1.650217in}}%
\pgfpathcurveto{\pgfqpoint{3.531474in}{1.642404in}}{\pgfqpoint{3.527083in}{1.631805in}}{\pgfqpoint{3.527083in}{1.620755in}}%
\pgfpathcurveto{\pgfqpoint{3.527083in}{1.609704in}}{\pgfqpoint{3.531474in}{1.599105in}}{\pgfqpoint{3.539287in}{1.591292in}}%
\pgfpathcurveto{\pgfqpoint{3.547101in}{1.583478in}}{\pgfqpoint{3.557700in}{1.579088in}}{\pgfqpoint{3.568750in}{1.579088in}}%
\pgfpathclose%
\pgfusepath{stroke,fill}%
\end{pgfscope}%
\begin{pgfscope}%
\pgfpathrectangle{\pgfqpoint{1.016621in}{0.499691in}}{\pgfqpoint{3.875000in}{2.695000in}}%
\pgfusepath{clip}%
\pgfsetbuttcap%
\pgfsetroundjoin%
\definecolor{currentfill}{rgb}{0.121569,0.466667,0.705882}%
\pgfsetfillcolor{currentfill}%
\pgfsetlinewidth{1.003750pt}%
\definecolor{currentstroke}{rgb}{0.121569,0.466667,0.705882}%
\pgfsetstrokecolor{currentstroke}%
\pgfsetdash{}{0pt}%
\pgfpathmoveto{\pgfqpoint{3.606000in}{1.550425in}}%
\pgfpathcurveto{\pgfqpoint{3.617050in}{1.550425in}}{\pgfqpoint{3.627649in}{1.554815in}}{\pgfqpoint{3.635463in}{1.562629in}}%
\pgfpathcurveto{\pgfqpoint{3.643277in}{1.570442in}}{\pgfqpoint{3.647667in}{1.581042in}}{\pgfqpoint{3.647667in}{1.592092in}}%
\pgfpathcurveto{\pgfqpoint{3.647667in}{1.603142in}}{\pgfqpoint{3.643277in}{1.613741in}}{\pgfqpoint{3.635463in}{1.621554in}}%
\pgfpathcurveto{\pgfqpoint{3.627649in}{1.629368in}}{\pgfqpoint{3.617050in}{1.633758in}}{\pgfqpoint{3.606000in}{1.633758in}}%
\pgfpathcurveto{\pgfqpoint{3.594950in}{1.633758in}}{\pgfqpoint{3.584351in}{1.629368in}}{\pgfqpoint{3.576537in}{1.621554in}}%
\pgfpathcurveto{\pgfqpoint{3.568724in}{1.613741in}}{\pgfqpoint{3.564334in}{1.603142in}}{\pgfqpoint{3.564334in}{1.592092in}}%
\pgfpathcurveto{\pgfqpoint{3.564334in}{1.581042in}}{\pgfqpoint{3.568724in}{1.570442in}}{\pgfqpoint{3.576537in}{1.562629in}}%
\pgfpathcurveto{\pgfqpoint{3.584351in}{1.554815in}}{\pgfqpoint{3.594950in}{1.550425in}}{\pgfqpoint{3.606000in}{1.550425in}}%
\pgfpathclose%
\pgfusepath{stroke,fill}%
\end{pgfscope}%
\begin{pgfscope}%
\pgfpathrectangle{\pgfqpoint{1.016621in}{0.499691in}}{\pgfqpoint{3.875000in}{2.695000in}}%
\pgfusepath{clip}%
\pgfsetbuttcap%
\pgfsetroundjoin%
\definecolor{currentfill}{rgb}{0.121569,0.466667,0.705882}%
\pgfsetfillcolor{currentfill}%
\pgfsetlinewidth{1.003750pt}%
\definecolor{currentstroke}{rgb}{0.121569,0.466667,0.705882}%
\pgfsetstrokecolor{currentstroke}%
\pgfsetdash{}{0pt}%
\pgfpathmoveto{\pgfqpoint{3.643250in}{1.693739in}}%
\pgfpathcurveto{\pgfqpoint{3.654301in}{1.693739in}}{\pgfqpoint{3.664900in}{1.698130in}}{\pgfqpoint{3.672713in}{1.705943in}}%
\pgfpathcurveto{\pgfqpoint{3.680527in}{1.713757in}}{\pgfqpoint{3.684917in}{1.724356in}}{\pgfqpoint{3.684917in}{1.735406in}}%
\pgfpathcurveto{\pgfqpoint{3.684917in}{1.746456in}}{\pgfqpoint{3.680527in}{1.757055in}}{\pgfqpoint{3.672713in}{1.764869in}}%
\pgfpathcurveto{\pgfqpoint{3.664900in}{1.772682in}}{\pgfqpoint{3.654301in}{1.777073in}}{\pgfqpoint{3.643250in}{1.777073in}}%
\pgfpathcurveto{\pgfqpoint{3.632200in}{1.777073in}}{\pgfqpoint{3.621601in}{1.772682in}}{\pgfqpoint{3.613788in}{1.764869in}}%
\pgfpathcurveto{\pgfqpoint{3.605974in}{1.757055in}}{\pgfqpoint{3.601584in}{1.746456in}}{\pgfqpoint{3.601584in}{1.735406in}}%
\pgfpathcurveto{\pgfqpoint{3.601584in}{1.724356in}}{\pgfqpoint{3.605974in}{1.713757in}}{\pgfqpoint{3.613788in}{1.705943in}}%
\pgfpathcurveto{\pgfqpoint{3.621601in}{1.698130in}}{\pgfqpoint{3.632200in}{1.693739in}}{\pgfqpoint{3.643250in}{1.693739in}}%
\pgfpathclose%
\pgfusepath{stroke,fill}%
\end{pgfscope}%
\begin{pgfscope}%
\pgfpathrectangle{\pgfqpoint{1.016621in}{0.499691in}}{\pgfqpoint{3.875000in}{2.695000in}}%
\pgfusepath{clip}%
\pgfsetbuttcap%
\pgfsetroundjoin%
\definecolor{currentfill}{rgb}{0.121569,0.466667,0.705882}%
\pgfsetfillcolor{currentfill}%
\pgfsetlinewidth{1.003750pt}%
\definecolor{currentstroke}{rgb}{0.121569,0.466667,0.705882}%
\pgfsetstrokecolor{currentstroke}%
\pgfsetdash{}{0pt}%
\pgfpathmoveto{\pgfqpoint{3.680501in}{1.788327in}}%
\pgfpathcurveto{\pgfqpoint{3.691551in}{1.788327in}}{\pgfqpoint{3.702150in}{1.792717in}}{\pgfqpoint{3.709964in}{1.800531in}}%
\pgfpathcurveto{\pgfqpoint{3.717777in}{1.808344in}}{\pgfqpoint{3.722167in}{1.818943in}}{\pgfqpoint{3.722167in}{1.829993in}}%
\pgfpathcurveto{\pgfqpoint{3.722167in}{1.841044in}}{\pgfqpoint{3.717777in}{1.851643in}}{\pgfqpoint{3.709964in}{1.859456in}}%
\pgfpathcurveto{\pgfqpoint{3.702150in}{1.867270in}}{\pgfqpoint{3.691551in}{1.871660in}}{\pgfqpoint{3.680501in}{1.871660in}}%
\pgfpathcurveto{\pgfqpoint{3.669451in}{1.871660in}}{\pgfqpoint{3.658852in}{1.867270in}}{\pgfqpoint{3.651038in}{1.859456in}}%
\pgfpathcurveto{\pgfqpoint{3.643224in}{1.851643in}}{\pgfqpoint{3.638834in}{1.841044in}}{\pgfqpoint{3.638834in}{1.829993in}}%
\pgfpathcurveto{\pgfqpoint{3.638834in}{1.818943in}}{\pgfqpoint{3.643224in}{1.808344in}}{\pgfqpoint{3.651038in}{1.800531in}}%
\pgfpathcurveto{\pgfqpoint{3.658852in}{1.792717in}}{\pgfqpoint{3.669451in}{1.788327in}}{\pgfqpoint{3.680501in}{1.788327in}}%
\pgfpathclose%
\pgfusepath{stroke,fill}%
\end{pgfscope}%
\begin{pgfscope}%
\pgfpathrectangle{\pgfqpoint{1.016621in}{0.499691in}}{\pgfqpoint{3.875000in}{2.695000in}}%
\pgfusepath{clip}%
\pgfsetbuttcap%
\pgfsetroundjoin%
\definecolor{currentfill}{rgb}{0.121569,0.466667,0.705882}%
\pgfsetfillcolor{currentfill}%
\pgfsetlinewidth{1.003750pt}%
\definecolor{currentstroke}{rgb}{0.121569,0.466667,0.705882}%
\pgfsetstrokecolor{currentstroke}%
\pgfsetdash{}{0pt}%
\pgfpathmoveto{\pgfqpoint{3.717751in}{1.951705in}}%
\pgfpathcurveto{\pgfqpoint{3.728801in}{1.951705in}}{\pgfqpoint{3.739400in}{1.956095in}}{\pgfqpoint{3.747214in}{1.963909in}}%
\pgfpathcurveto{\pgfqpoint{3.755027in}{1.971723in}}{\pgfqpoint{3.759418in}{1.982322in}}{\pgfqpoint{3.759418in}{1.993372in}}%
\pgfpathcurveto{\pgfqpoint{3.759418in}{2.004422in}}{\pgfqpoint{3.755027in}{2.015021in}}{\pgfqpoint{3.747214in}{2.022834in}}%
\pgfpathcurveto{\pgfqpoint{3.739400in}{2.030648in}}{\pgfqpoint{3.728801in}{2.035038in}}{\pgfqpoint{3.717751in}{2.035038in}}%
\pgfpathcurveto{\pgfqpoint{3.706701in}{2.035038in}}{\pgfqpoint{3.696102in}{2.030648in}}{\pgfqpoint{3.688288in}{2.022834in}}%
\pgfpathcurveto{\pgfqpoint{3.680475in}{2.015021in}}{\pgfqpoint{3.676084in}{2.004422in}}{\pgfqpoint{3.676084in}{1.993372in}}%
\pgfpathcurveto{\pgfqpoint{3.676084in}{1.982322in}}{\pgfqpoint{3.680475in}{1.971723in}}{\pgfqpoint{3.688288in}{1.963909in}}%
\pgfpathcurveto{\pgfqpoint{3.696102in}{1.956095in}}{\pgfqpoint{3.706701in}{1.951705in}}{\pgfqpoint{3.717751in}{1.951705in}}%
\pgfpathclose%
\pgfusepath{stroke,fill}%
\end{pgfscope}%
\begin{pgfscope}%
\pgfpathrectangle{\pgfqpoint{1.016621in}{0.499691in}}{\pgfqpoint{3.875000in}{2.695000in}}%
\pgfusepath{clip}%
\pgfsetbuttcap%
\pgfsetroundjoin%
\definecolor{currentfill}{rgb}{0.121569,0.466667,0.705882}%
\pgfsetfillcolor{currentfill}%
\pgfsetlinewidth{1.003750pt}%
\definecolor{currentstroke}{rgb}{0.121569,0.466667,0.705882}%
\pgfsetstrokecolor{currentstroke}%
\pgfsetdash{}{0pt}%
\pgfpathmoveto{\pgfqpoint{3.742584in}{1.914443in}}%
\pgfpathcurveto{\pgfqpoint{3.753635in}{1.914443in}}{\pgfqpoint{3.764234in}{1.918834in}}{\pgfqpoint{3.772047in}{1.926647in}}%
\pgfpathcurveto{\pgfqpoint{3.779861in}{1.934461in}}{\pgfqpoint{3.784251in}{1.945060in}}{\pgfqpoint{3.784251in}{1.956110in}}%
\pgfpathcurveto{\pgfqpoint{3.784251in}{1.967160in}}{\pgfqpoint{3.779861in}{1.977759in}}{\pgfqpoint{3.772047in}{1.985573in}}%
\pgfpathcurveto{\pgfqpoint{3.764234in}{1.993386in}}{\pgfqpoint{3.753635in}{1.997777in}}{\pgfqpoint{3.742584in}{1.997777in}}%
\pgfpathcurveto{\pgfqpoint{3.731534in}{1.997777in}}{\pgfqpoint{3.720935in}{1.993386in}}{\pgfqpoint{3.713122in}{1.985573in}}%
\pgfpathcurveto{\pgfqpoint{3.705308in}{1.977759in}}{\pgfqpoint{3.700918in}{1.967160in}}{\pgfqpoint{3.700918in}{1.956110in}}%
\pgfpathcurveto{\pgfqpoint{3.700918in}{1.945060in}}{\pgfqpoint{3.705308in}{1.934461in}}{\pgfqpoint{3.713122in}{1.926647in}}%
\pgfpathcurveto{\pgfqpoint{3.720935in}{1.918834in}}{\pgfqpoint{3.731534in}{1.914443in}}{\pgfqpoint{3.742584in}{1.914443in}}%
\pgfpathclose%
\pgfusepath{stroke,fill}%
\end{pgfscope}%
\begin{pgfscope}%
\pgfpathrectangle{\pgfqpoint{1.016621in}{0.499691in}}{\pgfqpoint{3.875000in}{2.695000in}}%
\pgfusepath{clip}%
\pgfsetbuttcap%
\pgfsetroundjoin%
\definecolor{currentfill}{rgb}{0.121569,0.466667,0.705882}%
\pgfsetfillcolor{currentfill}%
\pgfsetlinewidth{1.003750pt}%
\definecolor{currentstroke}{rgb}{0.121569,0.466667,0.705882}%
\pgfsetstrokecolor{currentstroke}%
\pgfsetdash{}{0pt}%
\pgfpathmoveto{\pgfqpoint{3.767418in}{1.928775in}}%
\pgfpathcurveto{\pgfqpoint{3.778468in}{1.928775in}}{\pgfqpoint{3.789067in}{1.933165in}}{\pgfqpoint{3.796881in}{1.940979in}}%
\pgfpathcurveto{\pgfqpoint{3.804694in}{1.948792in}}{\pgfqpoint{3.809085in}{1.959391in}}{\pgfqpoint{3.809085in}{1.970441in}}%
\pgfpathcurveto{\pgfqpoint{3.809085in}{1.981492in}}{\pgfqpoint{3.804694in}{1.992091in}}{\pgfqpoint{3.796881in}{1.999904in}}%
\pgfpathcurveto{\pgfqpoint{3.789067in}{2.007718in}}{\pgfqpoint{3.778468in}{2.012108in}}{\pgfqpoint{3.767418in}{2.012108in}}%
\pgfpathcurveto{\pgfqpoint{3.756368in}{2.012108in}}{\pgfqpoint{3.745769in}{2.007718in}}{\pgfqpoint{3.737955in}{1.999904in}}%
\pgfpathcurveto{\pgfqpoint{3.730142in}{1.992091in}}{\pgfqpoint{3.725751in}{1.981492in}}{\pgfqpoint{3.725751in}{1.970441in}}%
\pgfpathcurveto{\pgfqpoint{3.725751in}{1.959391in}}{\pgfqpoint{3.730142in}{1.948792in}}{\pgfqpoint{3.737955in}{1.940979in}}%
\pgfpathcurveto{\pgfqpoint{3.745769in}{1.933165in}}{\pgfqpoint{3.756368in}{1.928775in}}{\pgfqpoint{3.767418in}{1.928775in}}%
\pgfpathclose%
\pgfusepath{stroke,fill}%
\end{pgfscope}%
\begin{pgfscope}%
\pgfpathrectangle{\pgfqpoint{1.016621in}{0.499691in}}{\pgfqpoint{3.875000in}{2.695000in}}%
\pgfusepath{clip}%
\pgfsetbuttcap%
\pgfsetroundjoin%
\definecolor{currentfill}{rgb}{0.121569,0.466667,0.705882}%
\pgfsetfillcolor{currentfill}%
\pgfsetlinewidth{1.003750pt}%
\definecolor{currentstroke}{rgb}{0.121569,0.466667,0.705882}%
\pgfsetstrokecolor{currentstroke}%
\pgfsetdash{}{0pt}%
\pgfpathmoveto{\pgfqpoint{3.792251in}{1.954571in}}%
\pgfpathcurveto{\pgfqpoint{3.803302in}{1.954571in}}{\pgfqpoint{3.813901in}{1.958962in}}{\pgfqpoint{3.821714in}{1.966775in}}%
\pgfpathcurveto{\pgfqpoint{3.829528in}{1.974589in}}{\pgfqpoint{3.833918in}{1.985188in}}{\pgfqpoint{3.833918in}{1.996238in}}%
\pgfpathcurveto{\pgfqpoint{3.833918in}{2.007288in}}{\pgfqpoint{3.829528in}{2.017887in}}{\pgfqpoint{3.821714in}{2.025701in}}%
\pgfpathcurveto{\pgfqpoint{3.813901in}{2.033514in}}{\pgfqpoint{3.803302in}{2.037905in}}{\pgfqpoint{3.792251in}{2.037905in}}%
\pgfpathcurveto{\pgfqpoint{3.781201in}{2.037905in}}{\pgfqpoint{3.770602in}{2.033514in}}{\pgfqpoint{3.762789in}{2.025701in}}%
\pgfpathcurveto{\pgfqpoint{3.754975in}{2.017887in}}{\pgfqpoint{3.750585in}{2.007288in}}{\pgfqpoint{3.750585in}{1.996238in}}%
\pgfpathcurveto{\pgfqpoint{3.750585in}{1.985188in}}{\pgfqpoint{3.754975in}{1.974589in}}{\pgfqpoint{3.762789in}{1.966775in}}%
\pgfpathcurveto{\pgfqpoint{3.770602in}{1.958962in}}{\pgfqpoint{3.781201in}{1.954571in}}{\pgfqpoint{3.792251in}{1.954571in}}%
\pgfpathclose%
\pgfusepath{stroke,fill}%
\end{pgfscope}%
\begin{pgfscope}%
\pgfpathrectangle{\pgfqpoint{1.016621in}{0.499691in}}{\pgfqpoint{3.875000in}{2.695000in}}%
\pgfusepath{clip}%
\pgfsetbuttcap%
\pgfsetroundjoin%
\definecolor{currentfill}{rgb}{0.121569,0.466667,0.705882}%
\pgfsetfillcolor{currentfill}%
\pgfsetlinewidth{1.003750pt}%
\definecolor{currentstroke}{rgb}{0.121569,0.466667,0.705882}%
\pgfsetstrokecolor{currentstroke}%
\pgfsetdash{}{0pt}%
\pgfpathmoveto{\pgfqpoint{3.817085in}{2.043426in}}%
\pgfpathcurveto{\pgfqpoint{3.828135in}{2.043426in}}{\pgfqpoint{3.838734in}{2.047816in}}{\pgfqpoint{3.846548in}{2.055630in}}%
\pgfpathcurveto{\pgfqpoint{3.854361in}{2.063444in}}{\pgfqpoint{3.858752in}{2.074043in}}{\pgfqpoint{3.858752in}{2.085093in}}%
\pgfpathcurveto{\pgfqpoint{3.858752in}{2.096143in}}{\pgfqpoint{3.854361in}{2.106742in}}{\pgfqpoint{3.846548in}{2.114556in}}%
\pgfpathcurveto{\pgfqpoint{3.838734in}{2.122369in}}{\pgfqpoint{3.828135in}{2.126760in}}{\pgfqpoint{3.817085in}{2.126760in}}%
\pgfpathcurveto{\pgfqpoint{3.806035in}{2.126760in}}{\pgfqpoint{3.795436in}{2.122369in}}{\pgfqpoint{3.787622in}{2.114556in}}%
\pgfpathcurveto{\pgfqpoint{3.779809in}{2.106742in}}{\pgfqpoint{3.775418in}{2.096143in}}{\pgfqpoint{3.775418in}{2.085093in}}%
\pgfpathcurveto{\pgfqpoint{3.775418in}{2.074043in}}{\pgfqpoint{3.779809in}{2.063444in}}{\pgfqpoint{3.787622in}{2.055630in}}%
\pgfpathcurveto{\pgfqpoint{3.795436in}{2.047816in}}{\pgfqpoint{3.806035in}{2.043426in}}{\pgfqpoint{3.817085in}{2.043426in}}%
\pgfpathclose%
\pgfusepath{stroke,fill}%
\end{pgfscope}%
\begin{pgfscope}%
\pgfpathrectangle{\pgfqpoint{1.016621in}{0.499691in}}{\pgfqpoint{3.875000in}{2.695000in}}%
\pgfusepath{clip}%
\pgfsetbuttcap%
\pgfsetroundjoin%
\definecolor{currentfill}{rgb}{0.121569,0.466667,0.705882}%
\pgfsetfillcolor{currentfill}%
\pgfsetlinewidth{1.003750pt}%
\definecolor{currentstroke}{rgb}{0.121569,0.466667,0.705882}%
\pgfsetstrokecolor{currentstroke}%
\pgfsetdash{}{0pt}%
\pgfpathmoveto{\pgfqpoint{3.841918in}{2.092153in}}%
\pgfpathcurveto{\pgfqpoint{3.852969in}{2.092153in}}{\pgfqpoint{3.863568in}{2.096543in}}{\pgfqpoint{3.871381in}{2.104357in}}%
\pgfpathcurveto{\pgfqpoint{3.879195in}{2.112171in}}{\pgfqpoint{3.883585in}{2.122770in}}{\pgfqpoint{3.883585in}{2.133820in}}%
\pgfpathcurveto{\pgfqpoint{3.883585in}{2.144870in}}{\pgfqpoint{3.879195in}{2.155469in}}{\pgfqpoint{3.871381in}{2.163283in}}%
\pgfpathcurveto{\pgfqpoint{3.863568in}{2.171096in}}{\pgfqpoint{3.852969in}{2.175486in}}{\pgfqpoint{3.841918in}{2.175486in}}%
\pgfpathcurveto{\pgfqpoint{3.830868in}{2.175486in}}{\pgfqpoint{3.820269in}{2.171096in}}{\pgfqpoint{3.812456in}{2.163283in}}%
\pgfpathcurveto{\pgfqpoint{3.804642in}{2.155469in}}{\pgfqpoint{3.800252in}{2.144870in}}{\pgfqpoint{3.800252in}{2.133820in}}%
\pgfpathcurveto{\pgfqpoint{3.800252in}{2.122770in}}{\pgfqpoint{3.804642in}{2.112171in}}{\pgfqpoint{3.812456in}{2.104357in}}%
\pgfpathcurveto{\pgfqpoint{3.820269in}{2.096543in}}{\pgfqpoint{3.830868in}{2.092153in}}{\pgfqpoint{3.841918in}{2.092153in}}%
\pgfpathclose%
\pgfusepath{stroke,fill}%
\end{pgfscope}%
\begin{pgfscope}%
\pgfpathrectangle{\pgfqpoint{1.016621in}{0.499691in}}{\pgfqpoint{3.875000in}{2.695000in}}%
\pgfusepath{clip}%
\pgfsetbuttcap%
\pgfsetroundjoin%
\definecolor{currentfill}{rgb}{0.121569,0.466667,0.705882}%
\pgfsetfillcolor{currentfill}%
\pgfsetlinewidth{1.003750pt}%
\definecolor{currentstroke}{rgb}{0.121569,0.466667,0.705882}%
\pgfsetstrokecolor{currentstroke}%
\pgfsetdash{}{0pt}%
\pgfpathmoveto{\pgfqpoint{3.866752in}{2.149479in}}%
\pgfpathcurveto{\pgfqpoint{3.877802in}{2.149479in}}{\pgfqpoint{3.888401in}{2.153869in}}{\pgfqpoint{3.896215in}{2.161683in}}%
\pgfpathcurveto{\pgfqpoint{3.904028in}{2.169496in}}{\pgfqpoint{3.908419in}{2.180095in}}{\pgfqpoint{3.908419in}{2.191145in}}%
\pgfpathcurveto{\pgfqpoint{3.908419in}{2.202196in}}{\pgfqpoint{3.904028in}{2.212795in}}{\pgfqpoint{3.896215in}{2.220608in}}%
\pgfpathcurveto{\pgfqpoint{3.888401in}{2.228422in}}{\pgfqpoint{3.877802in}{2.232812in}}{\pgfqpoint{3.866752in}{2.232812in}}%
\pgfpathcurveto{\pgfqpoint{3.855702in}{2.232812in}}{\pgfqpoint{3.845103in}{2.228422in}}{\pgfqpoint{3.837289in}{2.220608in}}%
\pgfpathcurveto{\pgfqpoint{3.829476in}{2.212795in}}{\pgfqpoint{3.825085in}{2.202196in}}{\pgfqpoint{3.825085in}{2.191145in}}%
\pgfpathcurveto{\pgfqpoint{3.825085in}{2.180095in}}{\pgfqpoint{3.829476in}{2.169496in}}{\pgfqpoint{3.837289in}{2.161683in}}%
\pgfpathcurveto{\pgfqpoint{3.845103in}{2.153869in}}{\pgfqpoint{3.855702in}{2.149479in}}{\pgfqpoint{3.866752in}{2.149479in}}%
\pgfpathclose%
\pgfusepath{stroke,fill}%
\end{pgfscope}%
\begin{pgfscope}%
\pgfpathrectangle{\pgfqpoint{1.016621in}{0.499691in}}{\pgfqpoint{3.875000in}{2.695000in}}%
\pgfusepath{clip}%
\pgfsetbuttcap%
\pgfsetroundjoin%
\definecolor{currentfill}{rgb}{0.121569,0.466667,0.705882}%
\pgfsetfillcolor{currentfill}%
\pgfsetlinewidth{1.003750pt}%
\definecolor{currentstroke}{rgb}{0.121569,0.466667,0.705882}%
\pgfsetstrokecolor{currentstroke}%
\pgfsetdash{}{0pt}%
\pgfpathmoveto{\pgfqpoint{3.891585in}{2.155211in}}%
\pgfpathcurveto{\pgfqpoint{3.902636in}{2.155211in}}{\pgfqpoint{3.913235in}{2.159602in}}{\pgfqpoint{3.921048in}{2.167415in}}%
\pgfpathcurveto{\pgfqpoint{3.928862in}{2.175229in}}{\pgfqpoint{3.933252in}{2.185828in}}{\pgfqpoint{3.933252in}{2.196878in}}%
\pgfpathcurveto{\pgfqpoint{3.933252in}{2.207928in}}{\pgfqpoint{3.928862in}{2.218527in}}{\pgfqpoint{3.921048in}{2.226341in}}%
\pgfpathcurveto{\pgfqpoint{3.913235in}{2.234154in}}{\pgfqpoint{3.902636in}{2.238545in}}{\pgfqpoint{3.891585in}{2.238545in}}%
\pgfpathcurveto{\pgfqpoint{3.880535in}{2.238545in}}{\pgfqpoint{3.869936in}{2.234154in}}{\pgfqpoint{3.862123in}{2.226341in}}%
\pgfpathcurveto{\pgfqpoint{3.854309in}{2.218527in}}{\pgfqpoint{3.849919in}{2.207928in}}{\pgfqpoint{3.849919in}{2.196878in}}%
\pgfpathcurveto{\pgfqpoint{3.849919in}{2.185828in}}{\pgfqpoint{3.854309in}{2.175229in}}{\pgfqpoint{3.862123in}{2.167415in}}%
\pgfpathcurveto{\pgfqpoint{3.869936in}{2.159602in}}{\pgfqpoint{3.880535in}{2.155211in}}{\pgfqpoint{3.891585in}{2.155211in}}%
\pgfpathclose%
\pgfusepath{stroke,fill}%
\end{pgfscope}%
\begin{pgfscope}%
\pgfpathrectangle{\pgfqpoint{1.016621in}{0.499691in}}{\pgfqpoint{3.875000in}{2.695000in}}%
\pgfusepath{clip}%
\pgfsetbuttcap%
\pgfsetroundjoin%
\definecolor{currentfill}{rgb}{0.121569,0.466667,0.705882}%
\pgfsetfillcolor{currentfill}%
\pgfsetlinewidth{1.003750pt}%
\definecolor{currentstroke}{rgb}{0.121569,0.466667,0.705882}%
\pgfsetstrokecolor{currentstroke}%
\pgfsetdash{}{0pt}%
\pgfpathmoveto{\pgfqpoint{3.916419in}{2.183874in}}%
\pgfpathcurveto{\pgfqpoint{3.927469in}{2.183874in}}{\pgfqpoint{3.938068in}{2.188264in}}{\pgfqpoint{3.945882in}{2.196078in}}%
\pgfpathcurveto{\pgfqpoint{3.953695in}{2.203892in}}{\pgfqpoint{3.958086in}{2.214491in}}{\pgfqpoint{3.958086in}{2.225541in}}%
\pgfpathcurveto{\pgfqpoint{3.958086in}{2.236591in}}{\pgfqpoint{3.953695in}{2.247190in}}{\pgfqpoint{3.945882in}{2.255004in}}%
\pgfpathcurveto{\pgfqpoint{3.938068in}{2.262817in}}{\pgfqpoint{3.927469in}{2.267208in}}{\pgfqpoint{3.916419in}{2.267208in}}%
\pgfpathcurveto{\pgfqpoint{3.905369in}{2.267208in}}{\pgfqpoint{3.894770in}{2.262817in}}{\pgfqpoint{3.886956in}{2.255004in}}%
\pgfpathcurveto{\pgfqpoint{3.879143in}{2.247190in}}{\pgfqpoint{3.874752in}{2.236591in}}{\pgfqpoint{3.874752in}{2.225541in}}%
\pgfpathcurveto{\pgfqpoint{3.874752in}{2.214491in}}{\pgfqpoint{3.879143in}{2.203892in}}{\pgfqpoint{3.886956in}{2.196078in}}%
\pgfpathcurveto{\pgfqpoint{3.894770in}{2.188264in}}{\pgfqpoint{3.905369in}{2.183874in}}{\pgfqpoint{3.916419in}{2.183874in}}%
\pgfpathclose%
\pgfusepath{stroke,fill}%
\end{pgfscope}%
\begin{pgfscope}%
\pgfpathrectangle{\pgfqpoint{1.016621in}{0.499691in}}{\pgfqpoint{3.875000in}{2.695000in}}%
\pgfusepath{clip}%
\pgfsetbuttcap%
\pgfsetroundjoin%
\definecolor{currentfill}{rgb}{0.121569,0.466667,0.705882}%
\pgfsetfillcolor{currentfill}%
\pgfsetlinewidth{1.003750pt}%
\definecolor{currentstroke}{rgb}{0.121569,0.466667,0.705882}%
\pgfsetstrokecolor{currentstroke}%
\pgfsetdash{}{0pt}%
\pgfpathmoveto{\pgfqpoint{3.928836in}{2.201072in}}%
\pgfpathcurveto{\pgfqpoint{3.939886in}{2.201072in}}{\pgfqpoint{3.950485in}{2.205462in}}{\pgfqpoint{3.958299in}{2.213276in}}%
\pgfpathcurveto{\pgfqpoint{3.966112in}{2.221089in}}{\pgfqpoint{3.970502in}{2.231688in}}{\pgfqpoint{3.970502in}{2.242739in}}%
\pgfpathcurveto{\pgfqpoint{3.970502in}{2.253789in}}{\pgfqpoint{3.966112in}{2.264388in}}{\pgfqpoint{3.958299in}{2.272201in}}%
\pgfpathcurveto{\pgfqpoint{3.950485in}{2.280015in}}{\pgfqpoint{3.939886in}{2.284405in}}{\pgfqpoint{3.928836in}{2.284405in}}%
\pgfpathcurveto{\pgfqpoint{3.917786in}{2.284405in}}{\pgfqpoint{3.907187in}{2.280015in}}{\pgfqpoint{3.899373in}{2.272201in}}%
\pgfpathcurveto{\pgfqpoint{3.891559in}{2.264388in}}{\pgfqpoint{3.887169in}{2.253789in}}{\pgfqpoint{3.887169in}{2.242739in}}%
\pgfpathcurveto{\pgfqpoint{3.887169in}{2.231688in}}{\pgfqpoint{3.891559in}{2.221089in}}{\pgfqpoint{3.899373in}{2.213276in}}%
\pgfpathcurveto{\pgfqpoint{3.907187in}{2.205462in}}{\pgfqpoint{3.917786in}{2.201072in}}{\pgfqpoint{3.928836in}{2.201072in}}%
\pgfpathclose%
\pgfusepath{stroke,fill}%
\end{pgfscope}%
\begin{pgfscope}%
\pgfpathrectangle{\pgfqpoint{1.016621in}{0.499691in}}{\pgfqpoint{3.875000in}{2.695000in}}%
\pgfusepath{clip}%
\pgfsetbuttcap%
\pgfsetroundjoin%
\definecolor{currentfill}{rgb}{0.121569,0.466667,0.705882}%
\pgfsetfillcolor{currentfill}%
\pgfsetlinewidth{1.003750pt}%
\definecolor{currentstroke}{rgb}{0.121569,0.466667,0.705882}%
\pgfsetstrokecolor{currentstroke}%
\pgfsetdash{}{0pt}%
\pgfpathmoveto{\pgfqpoint{3.953669in}{2.206804in}}%
\pgfpathcurveto{\pgfqpoint{3.964719in}{2.206804in}}{\pgfqpoint{3.975318in}{2.211195in}}{\pgfqpoint{3.983132in}{2.219008in}}%
\pgfpathcurveto{\pgfqpoint{3.990946in}{2.226822in}}{\pgfqpoint{3.995336in}{2.237421in}}{\pgfqpoint{3.995336in}{2.248471in}}%
\pgfpathcurveto{\pgfqpoint{3.995336in}{2.259521in}}{\pgfqpoint{3.990946in}{2.270120in}}{\pgfqpoint{3.983132in}{2.277934in}}%
\pgfpathcurveto{\pgfqpoint{3.975318in}{2.285748in}}{\pgfqpoint{3.964719in}{2.290138in}}{\pgfqpoint{3.953669in}{2.290138in}}%
\pgfpathcurveto{\pgfqpoint{3.942619in}{2.290138in}}{\pgfqpoint{3.932020in}{2.285748in}}{\pgfqpoint{3.924206in}{2.277934in}}%
\pgfpathcurveto{\pgfqpoint{3.916393in}{2.270120in}}{\pgfqpoint{3.912003in}{2.259521in}}{\pgfqpoint{3.912003in}{2.248471in}}%
\pgfpathcurveto{\pgfqpoint{3.912003in}{2.237421in}}{\pgfqpoint{3.916393in}{2.226822in}}{\pgfqpoint{3.924206in}{2.219008in}}%
\pgfpathcurveto{\pgfqpoint{3.932020in}{2.211195in}}{\pgfqpoint{3.942619in}{2.206804in}}{\pgfqpoint{3.953669in}{2.206804in}}%
\pgfpathclose%
\pgfusepath{stroke,fill}%
\end{pgfscope}%
\begin{pgfscope}%
\pgfpathrectangle{\pgfqpoint{1.016621in}{0.499691in}}{\pgfqpoint{3.875000in}{2.695000in}}%
\pgfusepath{clip}%
\pgfsetbuttcap%
\pgfsetroundjoin%
\definecolor{currentfill}{rgb}{0.121569,0.466667,0.705882}%
\pgfsetfillcolor{currentfill}%
\pgfsetlinewidth{1.003750pt}%
\definecolor{currentstroke}{rgb}{0.121569,0.466667,0.705882}%
\pgfsetstrokecolor{currentstroke}%
\pgfsetdash{}{0pt}%
\pgfpathmoveto{\pgfqpoint{3.978503in}{2.266997in}}%
\pgfpathcurveto{\pgfqpoint{3.989553in}{2.266997in}}{\pgfqpoint{4.000152in}{2.271387in}}{\pgfqpoint{4.007966in}{2.279200in}}%
\pgfpathcurveto{\pgfqpoint{4.015779in}{2.287014in}}{\pgfqpoint{4.020169in}{2.297613in}}{\pgfqpoint{4.020169in}{2.308663in}}%
\pgfpathcurveto{\pgfqpoint{4.020169in}{2.319713in}}{\pgfqpoint{4.015779in}{2.330312in}}{\pgfqpoint{4.007966in}{2.338126in}}%
\pgfpathcurveto{\pgfqpoint{4.000152in}{2.345940in}}{\pgfqpoint{3.989553in}{2.350330in}}{\pgfqpoint{3.978503in}{2.350330in}}%
\pgfpathcurveto{\pgfqpoint{3.967453in}{2.350330in}}{\pgfqpoint{3.956854in}{2.345940in}}{\pgfqpoint{3.949040in}{2.338126in}}%
\pgfpathcurveto{\pgfqpoint{3.941226in}{2.330312in}}{\pgfqpoint{3.936836in}{2.319713in}}{\pgfqpoint{3.936836in}{2.308663in}}%
\pgfpathcurveto{\pgfqpoint{3.936836in}{2.297613in}}{\pgfqpoint{3.941226in}{2.287014in}}{\pgfqpoint{3.949040in}{2.279200in}}%
\pgfpathcurveto{\pgfqpoint{3.956854in}{2.271387in}}{\pgfqpoint{3.967453in}{2.266997in}}{\pgfqpoint{3.978503in}{2.266997in}}%
\pgfpathclose%
\pgfusepath{stroke,fill}%
\end{pgfscope}%
\begin{pgfscope}%
\pgfpathrectangle{\pgfqpoint{1.016621in}{0.499691in}}{\pgfqpoint{3.875000in}{2.695000in}}%
\pgfusepath{clip}%
\pgfsetbuttcap%
\pgfsetroundjoin%
\definecolor{currentfill}{rgb}{0.121569,0.466667,0.705882}%
\pgfsetfillcolor{currentfill}%
\pgfsetlinewidth{1.003750pt}%
\definecolor{currentstroke}{rgb}{0.121569,0.466667,0.705882}%
\pgfsetstrokecolor{currentstroke}%
\pgfsetdash{}{0pt}%
\pgfpathmoveto{\pgfqpoint{3.990920in}{2.281328in}}%
\pgfpathcurveto{\pgfqpoint{4.001970in}{2.281328in}}{\pgfqpoint{4.012569in}{2.285718in}}{\pgfqpoint{4.020382in}{2.293532in}}%
\pgfpathcurveto{\pgfqpoint{4.028196in}{2.301345in}}{\pgfqpoint{4.032586in}{2.311944in}}{\pgfqpoint{4.032586in}{2.322995in}}%
\pgfpathcurveto{\pgfqpoint{4.032586in}{2.334045in}}{\pgfqpoint{4.028196in}{2.344644in}}{\pgfqpoint{4.020382in}{2.352457in}}%
\pgfpathcurveto{\pgfqpoint{4.012569in}{2.360271in}}{\pgfqpoint{4.001970in}{2.364661in}}{\pgfqpoint{3.990920in}{2.364661in}}%
\pgfpathcurveto{\pgfqpoint{3.979869in}{2.364661in}}{\pgfqpoint{3.969270in}{2.360271in}}{\pgfqpoint{3.961457in}{2.352457in}}%
\pgfpathcurveto{\pgfqpoint{3.953643in}{2.344644in}}{\pgfqpoint{3.949253in}{2.334045in}}{\pgfqpoint{3.949253in}{2.322995in}}%
\pgfpathcurveto{\pgfqpoint{3.949253in}{2.311944in}}{\pgfqpoint{3.953643in}{2.301345in}}{\pgfqpoint{3.961457in}{2.293532in}}%
\pgfpathcurveto{\pgfqpoint{3.969270in}{2.285718in}}{\pgfqpoint{3.979869in}{2.281328in}}{\pgfqpoint{3.990920in}{2.281328in}}%
\pgfpathclose%
\pgfusepath{stroke,fill}%
\end{pgfscope}%
\begin{pgfscope}%
\pgfpathrectangle{\pgfqpoint{1.016621in}{0.499691in}}{\pgfqpoint{3.875000in}{2.695000in}}%
\pgfusepath{clip}%
\pgfsetbuttcap%
\pgfsetroundjoin%
\definecolor{currentfill}{rgb}{0.121569,0.466667,0.705882}%
\pgfsetfillcolor{currentfill}%
\pgfsetlinewidth{1.003750pt}%
\definecolor{currentstroke}{rgb}{0.121569,0.466667,0.705882}%
\pgfsetstrokecolor{currentstroke}%
\pgfsetdash{}{0pt}%
\pgfpathmoveto{\pgfqpoint{4.003336in}{2.318590in}}%
\pgfpathcurveto{\pgfqpoint{4.014386in}{2.318590in}}{\pgfqpoint{4.024985in}{2.322980in}}{\pgfqpoint{4.032799in}{2.330794in}}%
\pgfpathcurveto{\pgfqpoint{4.040613in}{2.338607in}}{\pgfqpoint{4.045003in}{2.349206in}}{\pgfqpoint{4.045003in}{2.360256in}}%
\pgfpathcurveto{\pgfqpoint{4.045003in}{2.371306in}}{\pgfqpoint{4.040613in}{2.381905in}}{\pgfqpoint{4.032799in}{2.389719in}}%
\pgfpathcurveto{\pgfqpoint{4.024985in}{2.397533in}}{\pgfqpoint{4.014386in}{2.401923in}}{\pgfqpoint{4.003336in}{2.401923in}}%
\pgfpathcurveto{\pgfqpoint{3.992286in}{2.401923in}}{\pgfqpoint{3.981687in}{2.397533in}}{\pgfqpoint{3.973873in}{2.389719in}}%
\pgfpathcurveto{\pgfqpoint{3.966060in}{2.381905in}}{\pgfqpoint{3.961670in}{2.371306in}}{\pgfqpoint{3.961670in}{2.360256in}}%
\pgfpathcurveto{\pgfqpoint{3.961670in}{2.349206in}}{\pgfqpoint{3.966060in}{2.338607in}}{\pgfqpoint{3.973873in}{2.330794in}}%
\pgfpathcurveto{\pgfqpoint{3.981687in}{2.322980in}}{\pgfqpoint{3.992286in}{2.318590in}}{\pgfqpoint{4.003336in}{2.318590in}}%
\pgfpathclose%
\pgfusepath{stroke,fill}%
\end{pgfscope}%
\begin{pgfscope}%
\pgfpathrectangle{\pgfqpoint{1.016621in}{0.499691in}}{\pgfqpoint{3.875000in}{2.695000in}}%
\pgfusepath{clip}%
\pgfsetbuttcap%
\pgfsetroundjoin%
\definecolor{currentfill}{rgb}{0.121569,0.466667,0.705882}%
\pgfsetfillcolor{currentfill}%
\pgfsetlinewidth{1.003750pt}%
\definecolor{currentstroke}{rgb}{0.121569,0.466667,0.705882}%
\pgfsetstrokecolor{currentstroke}%
\pgfsetdash{}{0pt}%
\pgfpathmoveto{\pgfqpoint{4.028170in}{2.341520in}}%
\pgfpathcurveto{\pgfqpoint{4.039220in}{2.341520in}}{\pgfqpoint{4.049819in}{2.345910in}}{\pgfqpoint{4.057633in}{2.353724in}}%
\pgfpathcurveto{\pgfqpoint{4.065446in}{2.361537in}}{\pgfqpoint{4.069836in}{2.372136in}}{\pgfqpoint{4.069836in}{2.383187in}}%
\pgfpathcurveto{\pgfqpoint{4.069836in}{2.394237in}}{\pgfqpoint{4.065446in}{2.404836in}}{\pgfqpoint{4.057633in}{2.412649in}}%
\pgfpathcurveto{\pgfqpoint{4.049819in}{2.420463in}}{\pgfqpoint{4.039220in}{2.424853in}}{\pgfqpoint{4.028170in}{2.424853in}}%
\pgfpathcurveto{\pgfqpoint{4.017120in}{2.424853in}}{\pgfqpoint{4.006521in}{2.420463in}}{\pgfqpoint{3.998707in}{2.412649in}}%
\pgfpathcurveto{\pgfqpoint{3.990893in}{2.404836in}}{\pgfqpoint{3.986503in}{2.394237in}}{\pgfqpoint{3.986503in}{2.383187in}}%
\pgfpathcurveto{\pgfqpoint{3.986503in}{2.372136in}}{\pgfqpoint{3.990893in}{2.361537in}}{\pgfqpoint{3.998707in}{2.353724in}}%
\pgfpathcurveto{\pgfqpoint{4.006521in}{2.345910in}}{\pgfqpoint{4.017120in}{2.341520in}}{\pgfqpoint{4.028170in}{2.341520in}}%
\pgfpathclose%
\pgfusepath{stroke,fill}%
\end{pgfscope}%
\begin{pgfscope}%
\pgfsetrectcap%
\pgfsetmiterjoin%
\pgfsetlinewidth{0.803000pt}%
\definecolor{currentstroke}{rgb}{0.000000,0.000000,0.000000}%
\pgfsetstrokecolor{currentstroke}%
\pgfsetdash{}{0pt}%
\pgfpathmoveto{\pgfqpoint{1.016621in}{0.499691in}}%
\pgfpathlineto{\pgfqpoint{1.016621in}{3.194691in}}%
\pgfusepath{stroke}%
\end{pgfscope}%
\begin{pgfscope}%
\pgfsetrectcap%
\pgfsetmiterjoin%
\pgfsetlinewidth{0.803000pt}%
\definecolor{currentstroke}{rgb}{0.000000,0.000000,0.000000}%
\pgfsetstrokecolor{currentstroke}%
\pgfsetdash{}{0pt}%
\pgfpathmoveto{\pgfqpoint{4.891621in}{0.499691in}}%
\pgfpathlineto{\pgfqpoint{4.891621in}{3.194691in}}%
\pgfusepath{stroke}%
\end{pgfscope}%
\begin{pgfscope}%
\pgfsetrectcap%
\pgfsetmiterjoin%
\pgfsetlinewidth{0.803000pt}%
\definecolor{currentstroke}{rgb}{0.000000,0.000000,0.000000}%
\pgfsetstrokecolor{currentstroke}%
\pgfsetdash{}{0pt}%
\pgfpathmoveto{\pgfqpoint{1.016621in}{0.499691in}}%
\pgfpathlineto{\pgfqpoint{4.891621in}{0.499691in}}%
\pgfusepath{stroke}%
\end{pgfscope}%
\begin{pgfscope}%
\pgfsetrectcap%
\pgfsetmiterjoin%
\pgfsetlinewidth{0.803000pt}%
\definecolor{currentstroke}{rgb}{0.000000,0.000000,0.000000}%
\pgfsetstrokecolor{currentstroke}%
\pgfsetdash{}{0pt}%
\pgfpathmoveto{\pgfqpoint{1.016621in}{3.194691in}}%
\pgfpathlineto{\pgfqpoint{4.891621in}{3.194691in}}%
\pgfusepath{stroke}%
\end{pgfscope}%
\end{pgfpicture}%
\makeatother%
\endgroup%

    \caption{Segundo intento de representación de $\varphi$ frente a f (escala logarítmica), ampliando los puntos seleccionados hacia los extremos}
  \end{figure}

  ACTUALIZACIÓN: (Saltar a sección \ref{sec:arctg} para el ajuste final, evitando este razonamiento que resultó ser erróneo.)

  Podemos observar que hacia los lados la gráfica ya no se comporta cómo una recta. También vemos que parece tener un punto de inflexión alrededor del 3, y que parece (mirando desde arriba) cóncava hasta ese valor y convexa después. Podría tratarse de una gráfica polinómica, aunque no parece ajustarse a ninguna de sus definiciones. Probamos a ajustar a diversos polinomios para comprobar que efectivamente ninguno termina de cuadrar.

  \begin{python}
    p2 = np.polynomial.Polynomial.fit(x2, y2, 2)
    plt.plot(*p2.linspace(), color="tomato")
    ...
  \end{python}

  \begin{figure}[H]
    %\centering
    \hspace{2.5em} %% Creator: Matplotlib, PGF backend
%%
%% To include the figure in your LaTeX document, write
%%   \input{<filename>.pgf}
%%
%% Make sure the required packages are loaded in your preamble
%%   \usepackage{pgf}
%%
%% Figures using additional raster images can only be included by \input if
%% they are in the same directory as the main LaTeX file. For loading figures
%% from other directories you can use the `import` package
%%   \usepackage{import}
%% and then include the figures with
%%   \import{<path to file>}{<filename>.pgf}
%%
%% Matplotlib used the following preamble
%%
\begingroup%
\makeatletter%
\begin{pgfpicture}%
\pgfpathrectangle{\pgfpointorigin}{\pgfqpoint{4.991621in}{3.294691in}}%
\pgfusepath{use as bounding box, clip}%
\begin{pgfscope}%
\pgfsetbuttcap%
\pgfsetmiterjoin%
\definecolor{currentfill}{rgb}{1.000000,1.000000,1.000000}%
\pgfsetfillcolor{currentfill}%
\pgfsetlinewidth{0.000000pt}%
\definecolor{currentstroke}{rgb}{1.000000,1.000000,1.000000}%
\pgfsetstrokecolor{currentstroke}%
\pgfsetdash{}{0pt}%
\pgfpathmoveto{\pgfqpoint{0.000000in}{0.000000in}}%
\pgfpathlineto{\pgfqpoint{4.991621in}{0.000000in}}%
\pgfpathlineto{\pgfqpoint{4.991621in}{3.294691in}}%
\pgfpathlineto{\pgfqpoint{0.000000in}{3.294691in}}%
\pgfpathclose%
\pgfusepath{fill}%
\end{pgfscope}%
\begin{pgfscope}%
\pgfsetbuttcap%
\pgfsetmiterjoin%
\definecolor{currentfill}{rgb}{1.000000,1.000000,1.000000}%
\pgfsetfillcolor{currentfill}%
\pgfsetlinewidth{0.000000pt}%
\definecolor{currentstroke}{rgb}{0.000000,0.000000,0.000000}%
\pgfsetstrokecolor{currentstroke}%
\pgfsetstrokeopacity{0.000000}%
\pgfsetdash{}{0pt}%
\pgfpathmoveto{\pgfqpoint{1.016621in}{0.499691in}}%
\pgfpathlineto{\pgfqpoint{4.891621in}{0.499691in}}%
\pgfpathlineto{\pgfqpoint{4.891621in}{3.194691in}}%
\pgfpathlineto{\pgfqpoint{1.016621in}{3.194691in}}%
\pgfpathclose%
\pgfusepath{fill}%
\end{pgfscope}%
\begin{pgfscope}%
\pgfpathrectangle{\pgfqpoint{1.016621in}{0.499691in}}{\pgfqpoint{3.875000in}{2.695000in}}%
\pgfusepath{clip}%
\pgfsetbuttcap%
\pgfsetroundjoin%
\definecolor{currentfill}{rgb}{0.117647,0.564706,1.000000}%
\pgfsetfillcolor{currentfill}%
\pgfsetlinewidth{1.003750pt}%
\definecolor{currentstroke}{rgb}{0.117647,0.564706,1.000000}%
\pgfsetstrokecolor{currentstroke}%
\pgfsetdash{}{0pt}%
\pgfpathmoveto{\pgfqpoint{1.197151in}{0.657741in}}%
\pgfpathcurveto{\pgfqpoint{1.208201in}{0.657741in}}{\pgfqpoint{1.218800in}{0.662131in}}{\pgfqpoint{1.226613in}{0.669945in}}%
\pgfpathcurveto{\pgfqpoint{1.234427in}{0.677758in}}{\pgfqpoint{1.238817in}{0.688357in}}{\pgfqpoint{1.238817in}{0.699408in}}%
\pgfpathcurveto{\pgfqpoint{1.238817in}{0.710458in}}{\pgfqpoint{1.234427in}{0.721057in}}{\pgfqpoint{1.226613in}{0.728870in}}%
\pgfpathcurveto{\pgfqpoint{1.218800in}{0.736684in}}{\pgfqpoint{1.208201in}{0.741074in}}{\pgfqpoint{1.197151in}{0.741074in}}%
\pgfpathcurveto{\pgfqpoint{1.186101in}{0.741074in}}{\pgfqpoint{1.175501in}{0.736684in}}{\pgfqpoint{1.167688in}{0.728870in}}%
\pgfpathcurveto{\pgfqpoint{1.159874in}{0.721057in}}{\pgfqpoint{1.155484in}{0.710458in}}{\pgfqpoint{1.155484in}{0.699408in}}%
\pgfpathcurveto{\pgfqpoint{1.155484in}{0.688357in}}{\pgfqpoint{1.159874in}{0.677758in}}{\pgfqpoint{1.167688in}{0.669945in}}%
\pgfpathcurveto{\pgfqpoint{1.175501in}{0.662131in}}{\pgfqpoint{1.186101in}{0.657741in}}{\pgfqpoint{1.197151in}{0.657741in}}%
\pgfpathclose%
\pgfusepath{stroke,fill}%
\end{pgfscope}%
\begin{pgfscope}%
\pgfpathrectangle{\pgfqpoint{1.016621in}{0.499691in}}{\pgfqpoint{3.875000in}{2.695000in}}%
\pgfusepath{clip}%
\pgfsetbuttcap%
\pgfsetroundjoin%
\definecolor{currentfill}{rgb}{0.117647,0.564706,1.000000}%
\pgfsetfillcolor{currentfill}%
\pgfsetlinewidth{1.003750pt}%
\definecolor{currentstroke}{rgb}{0.117647,0.564706,1.000000}%
\pgfsetstrokecolor{currentstroke}%
\pgfsetdash{}{0pt}%
\pgfpathmoveto{\pgfqpoint{2.165657in}{0.765068in}}%
\pgfpathcurveto{\pgfqpoint{2.176707in}{0.765068in}}{\pgfqpoint{2.187306in}{0.769459in}}{\pgfqpoint{2.195120in}{0.777272in}}%
\pgfpathcurveto{\pgfqpoint{2.202934in}{0.785086in}}{\pgfqpoint{2.207324in}{0.795685in}}{\pgfqpoint{2.207324in}{0.806735in}}%
\pgfpathcurveto{\pgfqpoint{2.207324in}{0.817785in}}{\pgfqpoint{2.202934in}{0.828384in}}{\pgfqpoint{2.195120in}{0.836198in}}%
\pgfpathcurveto{\pgfqpoint{2.187306in}{0.844011in}}{\pgfqpoint{2.176707in}{0.848402in}}{\pgfqpoint{2.165657in}{0.848402in}}%
\pgfpathcurveto{\pgfqpoint{2.154607in}{0.848402in}}{\pgfqpoint{2.144008in}{0.844011in}}{\pgfqpoint{2.136194in}{0.836198in}}%
\pgfpathcurveto{\pgfqpoint{2.128381in}{0.828384in}}{\pgfqpoint{2.123991in}{0.817785in}}{\pgfqpoint{2.123991in}{0.806735in}}%
\pgfpathcurveto{\pgfqpoint{2.123991in}{0.795685in}}{\pgfqpoint{2.128381in}{0.785086in}}{\pgfqpoint{2.136194in}{0.777272in}}%
\pgfpathcurveto{\pgfqpoint{2.144008in}{0.769459in}}{\pgfqpoint{2.154607in}{0.765068in}}{\pgfqpoint{2.165657in}{0.765068in}}%
\pgfpathclose%
\pgfusepath{stroke,fill}%
\end{pgfscope}%
\begin{pgfscope}%
\pgfpathrectangle{\pgfqpoint{1.016621in}{0.499691in}}{\pgfqpoint{3.875000in}{2.695000in}}%
\pgfusepath{clip}%
\pgfsetbuttcap%
\pgfsetroundjoin%
\definecolor{currentfill}{rgb}{0.117647,0.564706,1.000000}%
\pgfsetfillcolor{currentfill}%
\pgfsetlinewidth{1.003750pt}%
\definecolor{currentstroke}{rgb}{0.117647,0.564706,1.000000}%
\pgfsetstrokecolor{currentstroke}%
\pgfsetdash{}{0pt}%
\pgfpathmoveto{\pgfqpoint{2.538160in}{0.974845in}}%
\pgfpathcurveto{\pgfqpoint{2.549210in}{0.974845in}}{\pgfqpoint{2.559809in}{0.979235in}}{\pgfqpoint{2.567622in}{0.987049in}}%
\pgfpathcurveto{\pgfqpoint{2.575436in}{0.994862in}}{\pgfqpoint{2.579826in}{1.005461in}}{\pgfqpoint{2.579826in}{1.016511in}}%
\pgfpathcurveto{\pgfqpoint{2.579826in}{1.027561in}}{\pgfqpoint{2.575436in}{1.038160in}}{\pgfqpoint{2.567622in}{1.045974in}}%
\pgfpathcurveto{\pgfqpoint{2.559809in}{1.053788in}}{\pgfqpoint{2.549210in}{1.058178in}}{\pgfqpoint{2.538160in}{1.058178in}}%
\pgfpathcurveto{\pgfqpoint{2.527110in}{1.058178in}}{\pgfqpoint{2.516511in}{1.053788in}}{\pgfqpoint{2.508697in}{1.045974in}}%
\pgfpathcurveto{\pgfqpoint{2.500883in}{1.038160in}}{\pgfqpoint{2.496493in}{1.027561in}}{\pgfqpoint{2.496493in}{1.016511in}}%
\pgfpathcurveto{\pgfqpoint{2.496493in}{1.005461in}}{\pgfqpoint{2.500883in}{0.994862in}}{\pgfqpoint{2.508697in}{0.987049in}}%
\pgfpathcurveto{\pgfqpoint{2.516511in}{0.979235in}}{\pgfqpoint{2.527110in}{0.974845in}}{\pgfqpoint{2.538160in}{0.974845in}}%
\pgfpathclose%
\pgfusepath{stroke,fill}%
\end{pgfscope}%
\begin{pgfscope}%
\pgfpathrectangle{\pgfqpoint{1.016621in}{0.499691in}}{\pgfqpoint{3.875000in}{2.695000in}}%
\pgfusepath{clip}%
\pgfsetbuttcap%
\pgfsetroundjoin%
\definecolor{currentfill}{rgb}{0.117647,0.564706,1.000000}%
\pgfsetfillcolor{currentfill}%
\pgfsetlinewidth{1.003750pt}%
\definecolor{currentstroke}{rgb}{0.117647,0.564706,1.000000}%
\pgfsetstrokecolor{currentstroke}%
\pgfsetdash{}{0pt}%
\pgfpathmoveto{\pgfqpoint{2.997579in}{1.016312in}}%
\pgfpathcurveto{\pgfqpoint{3.008630in}{1.016312in}}{\pgfqpoint{3.019229in}{1.020702in}}{\pgfqpoint{3.027042in}{1.028516in}}%
\pgfpathcurveto{\pgfqpoint{3.034856in}{1.036330in}}{\pgfqpoint{3.039246in}{1.046929in}}{\pgfqpoint{3.039246in}{1.057979in}}%
\pgfpathcurveto{\pgfqpoint{3.039246in}{1.069029in}}{\pgfqpoint{3.034856in}{1.079628in}}{\pgfqpoint{3.027042in}{1.087442in}}%
\pgfpathcurveto{\pgfqpoint{3.019229in}{1.095255in}}{\pgfqpoint{3.008630in}{1.099645in}}{\pgfqpoint{2.997579in}{1.099645in}}%
\pgfpathcurveto{\pgfqpoint{2.986529in}{1.099645in}}{\pgfqpoint{2.975930in}{1.095255in}}{\pgfqpoint{2.968117in}{1.087442in}}%
\pgfpathcurveto{\pgfqpoint{2.960303in}{1.079628in}}{\pgfqpoint{2.955913in}{1.069029in}}{\pgfqpoint{2.955913in}{1.057979in}}%
\pgfpathcurveto{\pgfqpoint{2.955913in}{1.046929in}}{\pgfqpoint{2.960303in}{1.036330in}}{\pgfqpoint{2.968117in}{1.028516in}}%
\pgfpathcurveto{\pgfqpoint{2.975930in}{1.020702in}}{\pgfqpoint{2.986529in}{1.016312in}}{\pgfqpoint{2.997579in}{1.016312in}}%
\pgfpathclose%
\pgfusepath{stroke,fill}%
\end{pgfscope}%
\begin{pgfscope}%
\pgfpathrectangle{\pgfqpoint{1.016621in}{0.499691in}}{\pgfqpoint{3.875000in}{2.695000in}}%
\pgfusepath{clip}%
\pgfsetbuttcap%
\pgfsetroundjoin%
\definecolor{currentfill}{rgb}{0.117647,0.564706,1.000000}%
\pgfsetfillcolor{currentfill}%
\pgfsetlinewidth{1.003750pt}%
\definecolor{currentstroke}{rgb}{0.117647,0.564706,1.000000}%
\pgfsetstrokecolor{currentstroke}%
\pgfsetdash{}{0pt}%
\pgfpathmoveto{\pgfqpoint{3.183831in}{1.199257in}}%
\pgfpathcurveto{\pgfqpoint{3.194881in}{1.199257in}}{\pgfqpoint{3.205480in}{1.203647in}}{\pgfqpoint{3.213294in}{1.211460in}}%
\pgfpathcurveto{\pgfqpoint{3.221107in}{1.219274in}}{\pgfqpoint{3.225497in}{1.229873in}}{\pgfqpoint{3.225497in}{1.240923in}}%
\pgfpathcurveto{\pgfqpoint{3.225497in}{1.251973in}}{\pgfqpoint{3.221107in}{1.262572in}}{\pgfqpoint{3.213294in}{1.270386in}}%
\pgfpathcurveto{\pgfqpoint{3.205480in}{1.278200in}}{\pgfqpoint{3.194881in}{1.282590in}}{\pgfqpoint{3.183831in}{1.282590in}}%
\pgfpathcurveto{\pgfqpoint{3.172781in}{1.282590in}}{\pgfqpoint{3.162182in}{1.278200in}}{\pgfqpoint{3.154368in}{1.270386in}}%
\pgfpathcurveto{\pgfqpoint{3.146554in}{1.262572in}}{\pgfqpoint{3.142164in}{1.251973in}}{\pgfqpoint{3.142164in}{1.240923in}}%
\pgfpathcurveto{\pgfqpoint{3.142164in}{1.229873in}}{\pgfqpoint{3.146554in}{1.219274in}}{\pgfqpoint{3.154368in}{1.211460in}}%
\pgfpathcurveto{\pgfqpoint{3.162182in}{1.203647in}}{\pgfqpoint{3.172781in}{1.199257in}}{\pgfqpoint{3.183831in}{1.199257in}}%
\pgfpathclose%
\pgfusepath{stroke,fill}%
\end{pgfscope}%
\begin{pgfscope}%
\pgfpathrectangle{\pgfqpoint{1.016621in}{0.499691in}}{\pgfqpoint{3.875000in}{2.695000in}}%
\pgfusepath{clip}%
\pgfsetbuttcap%
\pgfsetroundjoin%
\definecolor{currentfill}{rgb}{0.117647,0.564706,1.000000}%
\pgfsetfillcolor{currentfill}%
\pgfsetlinewidth{1.003750pt}%
\definecolor{currentstroke}{rgb}{0.117647,0.564706,1.000000}%
\pgfsetstrokecolor{currentstroke}%
\pgfsetdash{}{0pt}%
\pgfpathmoveto{\pgfqpoint{3.332832in}{1.274874in}}%
\pgfpathcurveto{\pgfqpoint{3.343882in}{1.274874in}}{\pgfqpoint{3.354481in}{1.279264in}}{\pgfqpoint{3.362295in}{1.287077in}}%
\pgfpathcurveto{\pgfqpoint{3.370108in}{1.294891in}}{\pgfqpoint{3.374498in}{1.305490in}}{\pgfqpoint{3.374498in}{1.316540in}}%
\pgfpathcurveto{\pgfqpoint{3.374498in}{1.327590in}}{\pgfqpoint{3.370108in}{1.338189in}}{\pgfqpoint{3.362295in}{1.346003in}}%
\pgfpathcurveto{\pgfqpoint{3.354481in}{1.353817in}}{\pgfqpoint{3.343882in}{1.358207in}}{\pgfqpoint{3.332832in}{1.358207in}}%
\pgfpathcurveto{\pgfqpoint{3.321782in}{1.358207in}}{\pgfqpoint{3.311183in}{1.353817in}}{\pgfqpoint{3.303369in}{1.346003in}}%
\pgfpathcurveto{\pgfqpoint{3.295555in}{1.338189in}}{\pgfqpoint{3.291165in}{1.327590in}}{\pgfqpoint{3.291165in}{1.316540in}}%
\pgfpathcurveto{\pgfqpoint{3.291165in}{1.305490in}}{\pgfqpoint{3.295555in}{1.294891in}}{\pgfqpoint{3.303369in}{1.287077in}}%
\pgfpathcurveto{\pgfqpoint{3.311183in}{1.279264in}}{\pgfqpoint{3.321782in}{1.274874in}}{\pgfqpoint{3.332832in}{1.274874in}}%
\pgfpathclose%
\pgfusepath{stroke,fill}%
\end{pgfscope}%
\begin{pgfscope}%
\pgfpathrectangle{\pgfqpoint{1.016621in}{0.499691in}}{\pgfqpoint{3.875000in}{2.695000in}}%
\pgfusepath{clip}%
\pgfsetbuttcap%
\pgfsetroundjoin%
\definecolor{currentfill}{rgb}{0.117647,0.564706,1.000000}%
\pgfsetfillcolor{currentfill}%
\pgfsetlinewidth{1.003750pt}%
\definecolor{currentstroke}{rgb}{0.117647,0.564706,1.000000}%
\pgfsetstrokecolor{currentstroke}%
\pgfsetdash{}{0pt}%
\pgfpathmoveto{\pgfqpoint{3.382499in}{1.350491in}}%
\pgfpathcurveto{\pgfqpoint{3.393549in}{1.350491in}}{\pgfqpoint{3.404148in}{1.354881in}}{\pgfqpoint{3.411962in}{1.362694in}}%
\pgfpathcurveto{\pgfqpoint{3.419775in}{1.370508in}}{\pgfqpoint{3.424165in}{1.381107in}}{\pgfqpoint{3.424165in}{1.392157in}}%
\pgfpathcurveto{\pgfqpoint{3.424165in}{1.403207in}}{\pgfqpoint{3.419775in}{1.413806in}}{\pgfqpoint{3.411962in}{1.421620in}}%
\pgfpathcurveto{\pgfqpoint{3.404148in}{1.429434in}}{\pgfqpoint{3.393549in}{1.433824in}}{\pgfqpoint{3.382499in}{1.433824in}}%
\pgfpathcurveto{\pgfqpoint{3.371449in}{1.433824in}}{\pgfqpoint{3.360850in}{1.429434in}}{\pgfqpoint{3.353036in}{1.421620in}}%
\pgfpathcurveto{\pgfqpoint{3.345222in}{1.413806in}}{\pgfqpoint{3.340832in}{1.403207in}}{\pgfqpoint{3.340832in}{1.392157in}}%
\pgfpathcurveto{\pgfqpoint{3.340832in}{1.381107in}}{\pgfqpoint{3.345222in}{1.370508in}}{\pgfqpoint{3.353036in}{1.362694in}}%
\pgfpathcurveto{\pgfqpoint{3.360850in}{1.354881in}}{\pgfqpoint{3.371449in}{1.350491in}}{\pgfqpoint{3.382499in}{1.350491in}}%
\pgfpathclose%
\pgfusepath{stroke,fill}%
\end{pgfscope}%
\begin{pgfscope}%
\pgfpathrectangle{\pgfqpoint{1.016621in}{0.499691in}}{\pgfqpoint{3.875000in}{2.695000in}}%
\pgfusepath{clip}%
\pgfsetbuttcap%
\pgfsetroundjoin%
\definecolor{currentfill}{rgb}{0.117647,0.564706,1.000000}%
\pgfsetfillcolor{currentfill}%
\pgfsetlinewidth{1.003750pt}%
\definecolor{currentstroke}{rgb}{0.117647,0.564706,1.000000}%
\pgfsetstrokecolor{currentstroke}%
\pgfsetdash{}{0pt}%
\pgfpathmoveto{\pgfqpoint{3.444582in}{1.418790in}}%
\pgfpathcurveto{\pgfqpoint{3.455633in}{1.418790in}}{\pgfqpoint{3.466232in}{1.423180in}}{\pgfqpoint{3.474045in}{1.430994in}}%
\pgfpathcurveto{\pgfqpoint{3.481859in}{1.438807in}}{\pgfqpoint{3.486249in}{1.449406in}}{\pgfqpoint{3.486249in}{1.460457in}}%
\pgfpathcurveto{\pgfqpoint{3.486249in}{1.471507in}}{\pgfqpoint{3.481859in}{1.482106in}}{\pgfqpoint{3.474045in}{1.489919in}}%
\pgfpathcurveto{\pgfqpoint{3.466232in}{1.497733in}}{\pgfqpoint{3.455633in}{1.502123in}}{\pgfqpoint{3.444582in}{1.502123in}}%
\pgfpathcurveto{\pgfqpoint{3.433532in}{1.502123in}}{\pgfqpoint{3.422933in}{1.497733in}}{\pgfqpoint{3.415120in}{1.489919in}}%
\pgfpathcurveto{\pgfqpoint{3.407306in}{1.482106in}}{\pgfqpoint{3.402916in}{1.471507in}}{\pgfqpoint{3.402916in}{1.460457in}}%
\pgfpathcurveto{\pgfqpoint{3.402916in}{1.449406in}}{\pgfqpoint{3.407306in}{1.438807in}}{\pgfqpoint{3.415120in}{1.430994in}}%
\pgfpathcurveto{\pgfqpoint{3.422933in}{1.423180in}}{\pgfqpoint{3.433532in}{1.418790in}}{\pgfqpoint{3.444582in}{1.418790in}}%
\pgfpathclose%
\pgfusepath{stroke,fill}%
\end{pgfscope}%
\begin{pgfscope}%
\pgfpathrectangle{\pgfqpoint{1.016621in}{0.499691in}}{\pgfqpoint{3.875000in}{2.695000in}}%
\pgfusepath{clip}%
\pgfsetbuttcap%
\pgfsetroundjoin%
\definecolor{currentfill}{rgb}{0.117647,0.564706,1.000000}%
\pgfsetfillcolor{currentfill}%
\pgfsetlinewidth{1.003750pt}%
\definecolor{currentstroke}{rgb}{0.117647,0.564706,1.000000}%
\pgfsetstrokecolor{currentstroke}%
\pgfsetdash{}{0pt}%
\pgfpathmoveto{\pgfqpoint{3.494249in}{1.582220in}}%
\pgfpathcurveto{\pgfqpoint{3.505300in}{1.582220in}}{\pgfqpoint{3.515899in}{1.586611in}}{\pgfqpoint{3.523712in}{1.594424in}}%
\pgfpathcurveto{\pgfqpoint{3.531526in}{1.602238in}}{\pgfqpoint{3.535916in}{1.612837in}}{\pgfqpoint{3.535916in}{1.623887in}}%
\pgfpathcurveto{\pgfqpoint{3.535916in}{1.634937in}}{\pgfqpoint{3.531526in}{1.645536in}}{\pgfqpoint{3.523712in}{1.653350in}}%
\pgfpathcurveto{\pgfqpoint{3.515899in}{1.661163in}}{\pgfqpoint{3.505300in}{1.665554in}}{\pgfqpoint{3.494249in}{1.665554in}}%
\pgfpathcurveto{\pgfqpoint{3.483199in}{1.665554in}}{\pgfqpoint{3.472600in}{1.661163in}}{\pgfqpoint{3.464787in}{1.653350in}}%
\pgfpathcurveto{\pgfqpoint{3.456973in}{1.645536in}}{\pgfqpoint{3.452583in}{1.634937in}}{\pgfqpoint{3.452583in}{1.623887in}}%
\pgfpathcurveto{\pgfqpoint{3.452583in}{1.612837in}}{\pgfqpoint{3.456973in}{1.602238in}}{\pgfqpoint{3.464787in}{1.594424in}}%
\pgfpathcurveto{\pgfqpoint{3.472600in}{1.586611in}}{\pgfqpoint{3.483199in}{1.582220in}}{\pgfqpoint{3.494249in}{1.582220in}}%
\pgfpathclose%
\pgfusepath{stroke,fill}%
\end{pgfscope}%
\begin{pgfscope}%
\pgfpathrectangle{\pgfqpoint{1.016621in}{0.499691in}}{\pgfqpoint{3.875000in}{2.695000in}}%
\pgfusepath{clip}%
\pgfsetbuttcap%
\pgfsetroundjoin%
\definecolor{currentfill}{rgb}{0.117647,0.564706,1.000000}%
\pgfsetfillcolor{currentfill}%
\pgfsetlinewidth{1.003750pt}%
\definecolor{currentstroke}{rgb}{0.117647,0.564706,1.000000}%
\pgfsetstrokecolor{currentstroke}%
\pgfsetdash{}{0pt}%
\pgfpathmoveto{\pgfqpoint{3.531500in}{1.533435in}}%
\pgfpathcurveto{\pgfqpoint{3.542550in}{1.533435in}}{\pgfqpoint{3.553149in}{1.537825in}}{\pgfqpoint{3.560963in}{1.545639in}}%
\pgfpathcurveto{\pgfqpoint{3.568776in}{1.553453in}}{\pgfqpoint{3.573166in}{1.564052in}}{\pgfqpoint{3.573166in}{1.575102in}}%
\pgfpathcurveto{\pgfqpoint{3.573166in}{1.586152in}}{\pgfqpoint{3.568776in}{1.596751in}}{\pgfqpoint{3.560963in}{1.604565in}}%
\pgfpathcurveto{\pgfqpoint{3.553149in}{1.612378in}}{\pgfqpoint{3.542550in}{1.616768in}}{\pgfqpoint{3.531500in}{1.616768in}}%
\pgfpathcurveto{\pgfqpoint{3.520450in}{1.616768in}}{\pgfqpoint{3.509851in}{1.612378in}}{\pgfqpoint{3.502037in}{1.604565in}}%
\pgfpathcurveto{\pgfqpoint{3.494223in}{1.596751in}}{\pgfqpoint{3.489833in}{1.586152in}}{\pgfqpoint{3.489833in}{1.575102in}}%
\pgfpathcurveto{\pgfqpoint{3.489833in}{1.564052in}}{\pgfqpoint{3.494223in}{1.553453in}}{\pgfqpoint{3.502037in}{1.545639in}}%
\pgfpathcurveto{\pgfqpoint{3.509851in}{1.537825in}}{\pgfqpoint{3.520450in}{1.533435in}}{\pgfqpoint{3.531500in}{1.533435in}}%
\pgfpathclose%
\pgfusepath{stroke,fill}%
\end{pgfscope}%
\begin{pgfscope}%
\pgfpathrectangle{\pgfqpoint{1.016621in}{0.499691in}}{\pgfqpoint{3.875000in}{2.695000in}}%
\pgfusepath{clip}%
\pgfsetbuttcap%
\pgfsetroundjoin%
\definecolor{currentfill}{rgb}{0.117647,0.564706,1.000000}%
\pgfsetfillcolor{currentfill}%
\pgfsetlinewidth{1.003750pt}%
\definecolor{currentstroke}{rgb}{0.117647,0.564706,1.000000}%
\pgfsetstrokecolor{currentstroke}%
\pgfsetdash{}{0pt}%
\pgfpathmoveto{\pgfqpoint{3.568750in}{1.499285in}}%
\pgfpathcurveto{\pgfqpoint{3.579800in}{1.499285in}}{\pgfqpoint{3.590399in}{1.503676in}}{\pgfqpoint{3.598213in}{1.511489in}}%
\pgfpathcurveto{\pgfqpoint{3.606026in}{1.519303in}}{\pgfqpoint{3.610417in}{1.529902in}}{\pgfqpoint{3.610417in}{1.540952in}}%
\pgfpathcurveto{\pgfqpoint{3.610417in}{1.552002in}}{\pgfqpoint{3.606026in}{1.562601in}}{\pgfqpoint{3.598213in}{1.570415in}}%
\pgfpathcurveto{\pgfqpoint{3.590399in}{1.578228in}}{\pgfqpoint{3.579800in}{1.582619in}}{\pgfqpoint{3.568750in}{1.582619in}}%
\pgfpathcurveto{\pgfqpoint{3.557700in}{1.582619in}}{\pgfqpoint{3.547101in}{1.578228in}}{\pgfqpoint{3.539287in}{1.570415in}}%
\pgfpathcurveto{\pgfqpoint{3.531474in}{1.562601in}}{\pgfqpoint{3.527083in}{1.552002in}}{\pgfqpoint{3.527083in}{1.540952in}}%
\pgfpathcurveto{\pgfqpoint{3.527083in}{1.529902in}}{\pgfqpoint{3.531474in}{1.519303in}}{\pgfqpoint{3.539287in}{1.511489in}}%
\pgfpathcurveto{\pgfqpoint{3.547101in}{1.503676in}}{\pgfqpoint{3.557700in}{1.499285in}}{\pgfqpoint{3.568750in}{1.499285in}}%
\pgfpathclose%
\pgfusepath{stroke,fill}%
\end{pgfscope}%
\begin{pgfscope}%
\pgfpathrectangle{\pgfqpoint{1.016621in}{0.499691in}}{\pgfqpoint{3.875000in}{2.695000in}}%
\pgfusepath{clip}%
\pgfsetbuttcap%
\pgfsetroundjoin%
\definecolor{currentfill}{rgb}{0.117647,0.564706,1.000000}%
\pgfsetfillcolor{currentfill}%
\pgfsetlinewidth{1.003750pt}%
\definecolor{currentstroke}{rgb}{0.117647,0.564706,1.000000}%
\pgfsetstrokecolor{currentstroke}%
\pgfsetdash{}{0pt}%
\pgfpathmoveto{\pgfqpoint{3.606000in}{1.474893in}}%
\pgfpathcurveto{\pgfqpoint{3.617050in}{1.474893in}}{\pgfqpoint{3.627649in}{1.479283in}}{\pgfqpoint{3.635463in}{1.487097in}}%
\pgfpathcurveto{\pgfqpoint{3.643277in}{1.494910in}}{\pgfqpoint{3.647667in}{1.505509in}}{\pgfqpoint{3.647667in}{1.516560in}}%
\pgfpathcurveto{\pgfqpoint{3.647667in}{1.527610in}}{\pgfqpoint{3.643277in}{1.538209in}}{\pgfqpoint{3.635463in}{1.546022in}}%
\pgfpathcurveto{\pgfqpoint{3.627649in}{1.553836in}}{\pgfqpoint{3.617050in}{1.558226in}}{\pgfqpoint{3.606000in}{1.558226in}}%
\pgfpathcurveto{\pgfqpoint{3.594950in}{1.558226in}}{\pgfqpoint{3.584351in}{1.553836in}}{\pgfqpoint{3.576537in}{1.546022in}}%
\pgfpathcurveto{\pgfqpoint{3.568724in}{1.538209in}}{\pgfqpoint{3.564334in}{1.527610in}}{\pgfqpoint{3.564334in}{1.516560in}}%
\pgfpathcurveto{\pgfqpoint{3.564334in}{1.505509in}}{\pgfqpoint{3.568724in}{1.494910in}}{\pgfqpoint{3.576537in}{1.487097in}}%
\pgfpathcurveto{\pgfqpoint{3.584351in}{1.479283in}}{\pgfqpoint{3.594950in}{1.474893in}}{\pgfqpoint{3.606000in}{1.474893in}}%
\pgfpathclose%
\pgfusepath{stroke,fill}%
\end{pgfscope}%
\begin{pgfscope}%
\pgfpathrectangle{\pgfqpoint{1.016621in}{0.499691in}}{\pgfqpoint{3.875000in}{2.695000in}}%
\pgfusepath{clip}%
\pgfsetbuttcap%
\pgfsetroundjoin%
\definecolor{currentfill}{rgb}{0.117647,0.564706,1.000000}%
\pgfsetfillcolor{currentfill}%
\pgfsetlinewidth{1.003750pt}%
\definecolor{currentstroke}{rgb}{0.117647,0.564706,1.000000}%
\pgfsetstrokecolor{currentstroke}%
\pgfsetdash{}{0pt}%
\pgfpathmoveto{\pgfqpoint{3.643250in}{1.596856in}}%
\pgfpathcurveto{\pgfqpoint{3.654301in}{1.596856in}}{\pgfqpoint{3.664900in}{1.601246in}}{\pgfqpoint{3.672713in}{1.609060in}}%
\pgfpathcurveto{\pgfqpoint{3.680527in}{1.616873in}}{\pgfqpoint{3.684917in}{1.627472in}}{\pgfqpoint{3.684917in}{1.638522in}}%
\pgfpathcurveto{\pgfqpoint{3.684917in}{1.649573in}}{\pgfqpoint{3.680527in}{1.660172in}}{\pgfqpoint{3.672713in}{1.667985in}}%
\pgfpathcurveto{\pgfqpoint{3.664900in}{1.675799in}}{\pgfqpoint{3.654301in}{1.680189in}}{\pgfqpoint{3.643250in}{1.680189in}}%
\pgfpathcurveto{\pgfqpoint{3.632200in}{1.680189in}}{\pgfqpoint{3.621601in}{1.675799in}}{\pgfqpoint{3.613788in}{1.667985in}}%
\pgfpathcurveto{\pgfqpoint{3.605974in}{1.660172in}}{\pgfqpoint{3.601584in}{1.649573in}}{\pgfqpoint{3.601584in}{1.638522in}}%
\pgfpathcurveto{\pgfqpoint{3.601584in}{1.627472in}}{\pgfqpoint{3.605974in}{1.616873in}}{\pgfqpoint{3.613788in}{1.609060in}}%
\pgfpathcurveto{\pgfqpoint{3.621601in}{1.601246in}}{\pgfqpoint{3.632200in}{1.596856in}}{\pgfqpoint{3.643250in}{1.596856in}}%
\pgfpathclose%
\pgfusepath{stroke,fill}%
\end{pgfscope}%
\begin{pgfscope}%
\pgfpathrectangle{\pgfqpoint{1.016621in}{0.499691in}}{\pgfqpoint{3.875000in}{2.695000in}}%
\pgfusepath{clip}%
\pgfsetbuttcap%
\pgfsetroundjoin%
\definecolor{currentfill}{rgb}{0.117647,0.564706,1.000000}%
\pgfsetfillcolor{currentfill}%
\pgfsetlinewidth{1.003750pt}%
\definecolor{currentstroke}{rgb}{0.117647,0.564706,1.000000}%
\pgfsetstrokecolor{currentstroke}%
\pgfsetdash{}{0pt}%
\pgfpathmoveto{\pgfqpoint{3.680501in}{1.677351in}}%
\pgfpathcurveto{\pgfqpoint{3.691551in}{1.677351in}}{\pgfqpoint{3.702150in}{1.681742in}}{\pgfqpoint{3.709964in}{1.689555in}}%
\pgfpathcurveto{\pgfqpoint{3.717777in}{1.697369in}}{\pgfqpoint{3.722167in}{1.707968in}}{\pgfqpoint{3.722167in}{1.719018in}}%
\pgfpathcurveto{\pgfqpoint{3.722167in}{1.730068in}}{\pgfqpoint{3.717777in}{1.740667in}}{\pgfqpoint{3.709964in}{1.748481in}}%
\pgfpathcurveto{\pgfqpoint{3.702150in}{1.756294in}}{\pgfqpoint{3.691551in}{1.760685in}}{\pgfqpoint{3.680501in}{1.760685in}}%
\pgfpathcurveto{\pgfqpoint{3.669451in}{1.760685in}}{\pgfqpoint{3.658852in}{1.756294in}}{\pgfqpoint{3.651038in}{1.748481in}}%
\pgfpathcurveto{\pgfqpoint{3.643224in}{1.740667in}}{\pgfqpoint{3.638834in}{1.730068in}}{\pgfqpoint{3.638834in}{1.719018in}}%
\pgfpathcurveto{\pgfqpoint{3.638834in}{1.707968in}}{\pgfqpoint{3.643224in}{1.697369in}}{\pgfqpoint{3.651038in}{1.689555in}}%
\pgfpathcurveto{\pgfqpoint{3.658852in}{1.681742in}}{\pgfqpoint{3.669451in}{1.677351in}}{\pgfqpoint{3.680501in}{1.677351in}}%
\pgfpathclose%
\pgfusepath{stroke,fill}%
\end{pgfscope}%
\begin{pgfscope}%
\pgfpathrectangle{\pgfqpoint{1.016621in}{0.499691in}}{\pgfqpoint{3.875000in}{2.695000in}}%
\pgfusepath{clip}%
\pgfsetbuttcap%
\pgfsetroundjoin%
\definecolor{currentfill}{rgb}{0.117647,0.564706,1.000000}%
\pgfsetfillcolor{currentfill}%
\pgfsetlinewidth{1.003750pt}%
\definecolor{currentstroke}{rgb}{0.117647,0.564706,1.000000}%
\pgfsetstrokecolor{currentstroke}%
\pgfsetdash{}{0pt}%
\pgfpathmoveto{\pgfqpoint{3.717751in}{1.816389in}}%
\pgfpathcurveto{\pgfqpoint{3.728801in}{1.816389in}}{\pgfqpoint{3.739400in}{1.820779in}}{\pgfqpoint{3.747214in}{1.828593in}}%
\pgfpathcurveto{\pgfqpoint{3.755027in}{1.836407in}}{\pgfqpoint{3.759418in}{1.847006in}}{\pgfqpoint{3.759418in}{1.858056in}}%
\pgfpathcurveto{\pgfqpoint{3.759418in}{1.869106in}}{\pgfqpoint{3.755027in}{1.879705in}}{\pgfqpoint{3.747214in}{1.887519in}}%
\pgfpathcurveto{\pgfqpoint{3.739400in}{1.895332in}}{\pgfqpoint{3.728801in}{1.899722in}}{\pgfqpoint{3.717751in}{1.899722in}}%
\pgfpathcurveto{\pgfqpoint{3.706701in}{1.899722in}}{\pgfqpoint{3.696102in}{1.895332in}}{\pgfqpoint{3.688288in}{1.887519in}}%
\pgfpathcurveto{\pgfqpoint{3.680475in}{1.879705in}}{\pgfqpoint{3.676084in}{1.869106in}}{\pgfqpoint{3.676084in}{1.858056in}}%
\pgfpathcurveto{\pgfqpoint{3.676084in}{1.847006in}}{\pgfqpoint{3.680475in}{1.836407in}}{\pgfqpoint{3.688288in}{1.828593in}}%
\pgfpathcurveto{\pgfqpoint{3.696102in}{1.820779in}}{\pgfqpoint{3.706701in}{1.816389in}}{\pgfqpoint{3.717751in}{1.816389in}}%
\pgfpathclose%
\pgfusepath{stroke,fill}%
\end{pgfscope}%
\begin{pgfscope}%
\pgfpathrectangle{\pgfqpoint{1.016621in}{0.499691in}}{\pgfqpoint{3.875000in}{2.695000in}}%
\pgfusepath{clip}%
\pgfsetbuttcap%
\pgfsetroundjoin%
\definecolor{currentfill}{rgb}{0.117647,0.564706,1.000000}%
\pgfsetfillcolor{currentfill}%
\pgfsetlinewidth{1.003750pt}%
\definecolor{currentstroke}{rgb}{0.117647,0.564706,1.000000}%
\pgfsetstrokecolor{currentstroke}%
\pgfsetdash{}{0pt}%
\pgfpathmoveto{\pgfqpoint{3.742584in}{1.784679in}}%
\pgfpathcurveto{\pgfqpoint{3.753635in}{1.784679in}}{\pgfqpoint{3.764234in}{1.789069in}}{\pgfqpoint{3.772047in}{1.796883in}}%
\pgfpathcurveto{\pgfqpoint{3.779861in}{1.804696in}}{\pgfqpoint{3.784251in}{1.815295in}}{\pgfqpoint{3.784251in}{1.826345in}}%
\pgfpathcurveto{\pgfqpoint{3.784251in}{1.837396in}}{\pgfqpoint{3.779861in}{1.847995in}}{\pgfqpoint{3.772047in}{1.855808in}}%
\pgfpathcurveto{\pgfqpoint{3.764234in}{1.863622in}}{\pgfqpoint{3.753635in}{1.868012in}}{\pgfqpoint{3.742584in}{1.868012in}}%
\pgfpathcurveto{\pgfqpoint{3.731534in}{1.868012in}}{\pgfqpoint{3.720935in}{1.863622in}}{\pgfqpoint{3.713122in}{1.855808in}}%
\pgfpathcurveto{\pgfqpoint{3.705308in}{1.847995in}}{\pgfqpoint{3.700918in}{1.837396in}}{\pgfqpoint{3.700918in}{1.826345in}}%
\pgfpathcurveto{\pgfqpoint{3.700918in}{1.815295in}}{\pgfqpoint{3.705308in}{1.804696in}}{\pgfqpoint{3.713122in}{1.796883in}}%
\pgfpathcurveto{\pgfqpoint{3.720935in}{1.789069in}}{\pgfqpoint{3.731534in}{1.784679in}}{\pgfqpoint{3.742584in}{1.784679in}}%
\pgfpathclose%
\pgfusepath{stroke,fill}%
\end{pgfscope}%
\begin{pgfscope}%
\pgfpathrectangle{\pgfqpoint{1.016621in}{0.499691in}}{\pgfqpoint{3.875000in}{2.695000in}}%
\pgfusepath{clip}%
\pgfsetbuttcap%
\pgfsetroundjoin%
\definecolor{currentfill}{rgb}{0.117647,0.564706,1.000000}%
\pgfsetfillcolor{currentfill}%
\pgfsetlinewidth{1.003750pt}%
\definecolor{currentstroke}{rgb}{0.117647,0.564706,1.000000}%
\pgfsetstrokecolor{currentstroke}%
\pgfsetdash{}{0pt}%
\pgfpathmoveto{\pgfqpoint{3.767418in}{1.796875in}}%
\pgfpathcurveto{\pgfqpoint{3.778468in}{1.796875in}}{\pgfqpoint{3.789067in}{1.801265in}}{\pgfqpoint{3.796881in}{1.809079in}}%
\pgfpathcurveto{\pgfqpoint{3.804694in}{1.816893in}}{\pgfqpoint{3.809085in}{1.827492in}}{\pgfqpoint{3.809085in}{1.838542in}}%
\pgfpathcurveto{\pgfqpoint{3.809085in}{1.849592in}}{\pgfqpoint{3.804694in}{1.860191in}}{\pgfqpoint{3.796881in}{1.868005in}}%
\pgfpathcurveto{\pgfqpoint{3.789067in}{1.875818in}}{\pgfqpoint{3.778468in}{1.880208in}}{\pgfqpoint{3.767418in}{1.880208in}}%
\pgfpathcurveto{\pgfqpoint{3.756368in}{1.880208in}}{\pgfqpoint{3.745769in}{1.875818in}}{\pgfqpoint{3.737955in}{1.868005in}}%
\pgfpathcurveto{\pgfqpoint{3.730142in}{1.860191in}}{\pgfqpoint{3.725751in}{1.849592in}}{\pgfqpoint{3.725751in}{1.838542in}}%
\pgfpathcurveto{\pgfqpoint{3.725751in}{1.827492in}}{\pgfqpoint{3.730142in}{1.816893in}}{\pgfqpoint{3.737955in}{1.809079in}}%
\pgfpathcurveto{\pgfqpoint{3.745769in}{1.801265in}}{\pgfqpoint{3.756368in}{1.796875in}}{\pgfqpoint{3.767418in}{1.796875in}}%
\pgfpathclose%
\pgfusepath{stroke,fill}%
\end{pgfscope}%
\begin{pgfscope}%
\pgfpathrectangle{\pgfqpoint{1.016621in}{0.499691in}}{\pgfqpoint{3.875000in}{2.695000in}}%
\pgfusepath{clip}%
\pgfsetbuttcap%
\pgfsetroundjoin%
\definecolor{currentfill}{rgb}{0.117647,0.564706,1.000000}%
\pgfsetfillcolor{currentfill}%
\pgfsetlinewidth{1.003750pt}%
\definecolor{currentstroke}{rgb}{0.117647,0.564706,1.000000}%
\pgfsetstrokecolor{currentstroke}%
\pgfsetdash{}{0pt}%
\pgfpathmoveto{\pgfqpoint{3.792251in}{1.818828in}}%
\pgfpathcurveto{\pgfqpoint{3.803302in}{1.818828in}}{\pgfqpoint{3.813901in}{1.823219in}}{\pgfqpoint{3.821714in}{1.831032in}}%
\pgfpathcurveto{\pgfqpoint{3.829528in}{1.838846in}}{\pgfqpoint{3.833918in}{1.849445in}}{\pgfqpoint{3.833918in}{1.860495in}}%
\pgfpathcurveto{\pgfqpoint{3.833918in}{1.871545in}}{\pgfqpoint{3.829528in}{1.882144in}}{\pgfqpoint{3.821714in}{1.889958in}}%
\pgfpathcurveto{\pgfqpoint{3.813901in}{1.897771in}}{\pgfqpoint{3.803302in}{1.902162in}}{\pgfqpoint{3.792251in}{1.902162in}}%
\pgfpathcurveto{\pgfqpoint{3.781201in}{1.902162in}}{\pgfqpoint{3.770602in}{1.897771in}}{\pgfqpoint{3.762789in}{1.889958in}}%
\pgfpathcurveto{\pgfqpoint{3.754975in}{1.882144in}}{\pgfqpoint{3.750585in}{1.871545in}}{\pgfqpoint{3.750585in}{1.860495in}}%
\pgfpathcurveto{\pgfqpoint{3.750585in}{1.849445in}}{\pgfqpoint{3.754975in}{1.838846in}}{\pgfqpoint{3.762789in}{1.831032in}}%
\pgfpathcurveto{\pgfqpoint{3.770602in}{1.823219in}}{\pgfqpoint{3.781201in}{1.818828in}}{\pgfqpoint{3.792251in}{1.818828in}}%
\pgfpathclose%
\pgfusepath{stroke,fill}%
\end{pgfscope}%
\begin{pgfscope}%
\pgfpathrectangle{\pgfqpoint{1.016621in}{0.499691in}}{\pgfqpoint{3.875000in}{2.695000in}}%
\pgfusepath{clip}%
\pgfsetbuttcap%
\pgfsetroundjoin%
\definecolor{currentfill}{rgb}{0.117647,0.564706,1.000000}%
\pgfsetfillcolor{currentfill}%
\pgfsetlinewidth{1.003750pt}%
\definecolor{currentstroke}{rgb}{0.117647,0.564706,1.000000}%
\pgfsetstrokecolor{currentstroke}%
\pgfsetdash{}{0pt}%
\pgfpathmoveto{\pgfqpoint{3.817085in}{1.894445in}}%
\pgfpathcurveto{\pgfqpoint{3.828135in}{1.894445in}}{\pgfqpoint{3.838734in}{1.898836in}}{\pgfqpoint{3.846548in}{1.906649in}}%
\pgfpathcurveto{\pgfqpoint{3.854361in}{1.914463in}}{\pgfqpoint{3.858752in}{1.925062in}}{\pgfqpoint{3.858752in}{1.936112in}}%
\pgfpathcurveto{\pgfqpoint{3.858752in}{1.947162in}}{\pgfqpoint{3.854361in}{1.957761in}}{\pgfqpoint{3.846548in}{1.965575in}}%
\pgfpathcurveto{\pgfqpoint{3.838734in}{1.973389in}}{\pgfqpoint{3.828135in}{1.977779in}}{\pgfqpoint{3.817085in}{1.977779in}}%
\pgfpathcurveto{\pgfqpoint{3.806035in}{1.977779in}}{\pgfqpoint{3.795436in}{1.973389in}}{\pgfqpoint{3.787622in}{1.965575in}}%
\pgfpathcurveto{\pgfqpoint{3.779809in}{1.957761in}}{\pgfqpoint{3.775418in}{1.947162in}}{\pgfqpoint{3.775418in}{1.936112in}}%
\pgfpathcurveto{\pgfqpoint{3.775418in}{1.925062in}}{\pgfqpoint{3.779809in}{1.914463in}}{\pgfqpoint{3.787622in}{1.906649in}}%
\pgfpathcurveto{\pgfqpoint{3.795436in}{1.898836in}}{\pgfqpoint{3.806035in}{1.894445in}}{\pgfqpoint{3.817085in}{1.894445in}}%
\pgfpathclose%
\pgfusepath{stroke,fill}%
\end{pgfscope}%
\begin{pgfscope}%
\pgfpathrectangle{\pgfqpoint{1.016621in}{0.499691in}}{\pgfqpoint{3.875000in}{2.695000in}}%
\pgfusepath{clip}%
\pgfsetbuttcap%
\pgfsetroundjoin%
\definecolor{currentfill}{rgb}{0.117647,0.564706,1.000000}%
\pgfsetfillcolor{currentfill}%
\pgfsetlinewidth{1.003750pt}%
\definecolor{currentstroke}{rgb}{0.117647,0.564706,1.000000}%
\pgfsetstrokecolor{currentstroke}%
\pgfsetdash{}{0pt}%
\pgfpathmoveto{\pgfqpoint{3.841918in}{1.935913in}}%
\pgfpathcurveto{\pgfqpoint{3.852969in}{1.935913in}}{\pgfqpoint{3.863568in}{1.940303in}}{\pgfqpoint{3.871381in}{1.948117in}}%
\pgfpathcurveto{\pgfqpoint{3.879195in}{1.955930in}}{\pgfqpoint{3.883585in}{1.966529in}}{\pgfqpoint{3.883585in}{1.977580in}}%
\pgfpathcurveto{\pgfqpoint{3.883585in}{1.988630in}}{\pgfqpoint{3.879195in}{1.999229in}}{\pgfqpoint{3.871381in}{2.007042in}}%
\pgfpathcurveto{\pgfqpoint{3.863568in}{2.014856in}}{\pgfqpoint{3.852969in}{2.019246in}}{\pgfqpoint{3.841918in}{2.019246in}}%
\pgfpathcurveto{\pgfqpoint{3.830868in}{2.019246in}}{\pgfqpoint{3.820269in}{2.014856in}}{\pgfqpoint{3.812456in}{2.007042in}}%
\pgfpathcurveto{\pgfqpoint{3.804642in}{1.999229in}}{\pgfqpoint{3.800252in}{1.988630in}}{\pgfqpoint{3.800252in}{1.977580in}}%
\pgfpathcurveto{\pgfqpoint{3.800252in}{1.966529in}}{\pgfqpoint{3.804642in}{1.955930in}}{\pgfqpoint{3.812456in}{1.948117in}}%
\pgfpathcurveto{\pgfqpoint{3.820269in}{1.940303in}}{\pgfqpoint{3.830868in}{1.935913in}}{\pgfqpoint{3.841918in}{1.935913in}}%
\pgfpathclose%
\pgfusepath{stroke,fill}%
\end{pgfscope}%
\begin{pgfscope}%
\pgfpathrectangle{\pgfqpoint{1.016621in}{0.499691in}}{\pgfqpoint{3.875000in}{2.695000in}}%
\pgfusepath{clip}%
\pgfsetbuttcap%
\pgfsetroundjoin%
\definecolor{currentfill}{rgb}{0.117647,0.564706,1.000000}%
\pgfsetfillcolor{currentfill}%
\pgfsetlinewidth{1.003750pt}%
\definecolor{currentstroke}{rgb}{0.117647,0.564706,1.000000}%
\pgfsetstrokecolor{currentstroke}%
\pgfsetdash{}{0pt}%
\pgfpathmoveto{\pgfqpoint{3.866752in}{1.984698in}}%
\pgfpathcurveto{\pgfqpoint{3.877802in}{1.984698in}}{\pgfqpoint{3.888401in}{1.989088in}}{\pgfqpoint{3.896215in}{1.996902in}}%
\pgfpathcurveto{\pgfqpoint{3.904028in}{2.004716in}}{\pgfqpoint{3.908419in}{2.015315in}}{\pgfqpoint{3.908419in}{2.026365in}}%
\pgfpathcurveto{\pgfqpoint{3.908419in}{2.037415in}}{\pgfqpoint{3.904028in}{2.048014in}}{\pgfqpoint{3.896215in}{2.055827in}}%
\pgfpathcurveto{\pgfqpoint{3.888401in}{2.063641in}}{\pgfqpoint{3.877802in}{2.068031in}}{\pgfqpoint{3.866752in}{2.068031in}}%
\pgfpathcurveto{\pgfqpoint{3.855702in}{2.068031in}}{\pgfqpoint{3.845103in}{2.063641in}}{\pgfqpoint{3.837289in}{2.055827in}}%
\pgfpathcurveto{\pgfqpoint{3.829476in}{2.048014in}}{\pgfqpoint{3.825085in}{2.037415in}}{\pgfqpoint{3.825085in}{2.026365in}}%
\pgfpathcurveto{\pgfqpoint{3.825085in}{2.015315in}}{\pgfqpoint{3.829476in}{2.004716in}}{\pgfqpoint{3.837289in}{1.996902in}}%
\pgfpathcurveto{\pgfqpoint{3.845103in}{1.989088in}}{\pgfqpoint{3.855702in}{1.984698in}}{\pgfqpoint{3.866752in}{1.984698in}}%
\pgfpathclose%
\pgfusepath{stroke,fill}%
\end{pgfscope}%
\begin{pgfscope}%
\pgfpathrectangle{\pgfqpoint{1.016621in}{0.499691in}}{\pgfqpoint{3.875000in}{2.695000in}}%
\pgfusepath{clip}%
\pgfsetbuttcap%
\pgfsetroundjoin%
\definecolor{currentfill}{rgb}{0.117647,0.564706,1.000000}%
\pgfsetfillcolor{currentfill}%
\pgfsetlinewidth{1.003750pt}%
\definecolor{currentstroke}{rgb}{0.117647,0.564706,1.000000}%
\pgfsetstrokecolor{currentstroke}%
\pgfsetdash{}{0pt}%
\pgfpathmoveto{\pgfqpoint{3.891585in}{1.989577in}}%
\pgfpathcurveto{\pgfqpoint{3.902636in}{1.989577in}}{\pgfqpoint{3.913235in}{1.993967in}}{\pgfqpoint{3.921048in}{2.001780in}}%
\pgfpathcurveto{\pgfqpoint{3.928862in}{2.009594in}}{\pgfqpoint{3.933252in}{2.020193in}}{\pgfqpoint{3.933252in}{2.031243in}}%
\pgfpathcurveto{\pgfqpoint{3.933252in}{2.042293in}}{\pgfqpoint{3.928862in}{2.052892in}}{\pgfqpoint{3.921048in}{2.060706in}}%
\pgfpathcurveto{\pgfqpoint{3.913235in}{2.068520in}}{\pgfqpoint{3.902636in}{2.072910in}}{\pgfqpoint{3.891585in}{2.072910in}}%
\pgfpathcurveto{\pgfqpoint{3.880535in}{2.072910in}}{\pgfqpoint{3.869936in}{2.068520in}}{\pgfqpoint{3.862123in}{2.060706in}}%
\pgfpathcurveto{\pgfqpoint{3.854309in}{2.052892in}}{\pgfqpoint{3.849919in}{2.042293in}}{\pgfqpoint{3.849919in}{2.031243in}}%
\pgfpathcurveto{\pgfqpoint{3.849919in}{2.020193in}}{\pgfqpoint{3.854309in}{2.009594in}}{\pgfqpoint{3.862123in}{2.001780in}}%
\pgfpathcurveto{\pgfqpoint{3.869936in}{1.993967in}}{\pgfqpoint{3.880535in}{1.989577in}}{\pgfqpoint{3.891585in}{1.989577in}}%
\pgfpathclose%
\pgfusepath{stroke,fill}%
\end{pgfscope}%
\begin{pgfscope}%
\pgfpathrectangle{\pgfqpoint{1.016621in}{0.499691in}}{\pgfqpoint{3.875000in}{2.695000in}}%
\pgfusepath{clip}%
\pgfsetbuttcap%
\pgfsetroundjoin%
\definecolor{currentfill}{rgb}{0.117647,0.564706,1.000000}%
\pgfsetfillcolor{currentfill}%
\pgfsetlinewidth{1.003750pt}%
\definecolor{currentstroke}{rgb}{0.117647,0.564706,1.000000}%
\pgfsetstrokecolor{currentstroke}%
\pgfsetdash{}{0pt}%
\pgfpathmoveto{\pgfqpoint{3.916419in}{2.013969in}}%
\pgfpathcurveto{\pgfqpoint{3.927469in}{2.013969in}}{\pgfqpoint{3.938068in}{2.018359in}}{\pgfqpoint{3.945882in}{2.026173in}}%
\pgfpathcurveto{\pgfqpoint{3.953695in}{2.033987in}}{\pgfqpoint{3.958086in}{2.044586in}}{\pgfqpoint{3.958086in}{2.055636in}}%
\pgfpathcurveto{\pgfqpoint{3.958086in}{2.066686in}}{\pgfqpoint{3.953695in}{2.077285in}}{\pgfqpoint{3.945882in}{2.085099in}}%
\pgfpathcurveto{\pgfqpoint{3.938068in}{2.092912in}}{\pgfqpoint{3.927469in}{2.097302in}}{\pgfqpoint{3.916419in}{2.097302in}}%
\pgfpathcurveto{\pgfqpoint{3.905369in}{2.097302in}}{\pgfqpoint{3.894770in}{2.092912in}}{\pgfqpoint{3.886956in}{2.085099in}}%
\pgfpathcurveto{\pgfqpoint{3.879143in}{2.077285in}}{\pgfqpoint{3.874752in}{2.066686in}}{\pgfqpoint{3.874752in}{2.055636in}}%
\pgfpathcurveto{\pgfqpoint{3.874752in}{2.044586in}}{\pgfqpoint{3.879143in}{2.033987in}}{\pgfqpoint{3.886956in}{2.026173in}}%
\pgfpathcurveto{\pgfqpoint{3.894770in}{2.018359in}}{\pgfqpoint{3.905369in}{2.013969in}}{\pgfqpoint{3.916419in}{2.013969in}}%
\pgfpathclose%
\pgfusepath{stroke,fill}%
\end{pgfscope}%
\begin{pgfscope}%
\pgfpathrectangle{\pgfqpoint{1.016621in}{0.499691in}}{\pgfqpoint{3.875000in}{2.695000in}}%
\pgfusepath{clip}%
\pgfsetbuttcap%
\pgfsetroundjoin%
\definecolor{currentfill}{rgb}{0.117647,0.564706,1.000000}%
\pgfsetfillcolor{currentfill}%
\pgfsetlinewidth{1.003750pt}%
\definecolor{currentstroke}{rgb}{0.117647,0.564706,1.000000}%
\pgfsetstrokecolor{currentstroke}%
\pgfsetdash{}{0pt}%
\pgfpathmoveto{\pgfqpoint{3.928836in}{2.028605in}}%
\pgfpathcurveto{\pgfqpoint{3.939886in}{2.028605in}}{\pgfqpoint{3.950485in}{2.032995in}}{\pgfqpoint{3.958299in}{2.040809in}}%
\pgfpathcurveto{\pgfqpoint{3.966112in}{2.048622in}}{\pgfqpoint{3.970502in}{2.059221in}}{\pgfqpoint{3.970502in}{2.070271in}}%
\pgfpathcurveto{\pgfqpoint{3.970502in}{2.081322in}}{\pgfqpoint{3.966112in}{2.091921in}}{\pgfqpoint{3.958299in}{2.099734in}}%
\pgfpathcurveto{\pgfqpoint{3.950485in}{2.107548in}}{\pgfqpoint{3.939886in}{2.111938in}}{\pgfqpoint{3.928836in}{2.111938in}}%
\pgfpathcurveto{\pgfqpoint{3.917786in}{2.111938in}}{\pgfqpoint{3.907187in}{2.107548in}}{\pgfqpoint{3.899373in}{2.099734in}}%
\pgfpathcurveto{\pgfqpoint{3.891559in}{2.091921in}}{\pgfqpoint{3.887169in}{2.081322in}}{\pgfqpoint{3.887169in}{2.070271in}}%
\pgfpathcurveto{\pgfqpoint{3.887169in}{2.059221in}}{\pgfqpoint{3.891559in}{2.048622in}}{\pgfqpoint{3.899373in}{2.040809in}}%
\pgfpathcurveto{\pgfqpoint{3.907187in}{2.032995in}}{\pgfqpoint{3.917786in}{2.028605in}}{\pgfqpoint{3.928836in}{2.028605in}}%
\pgfpathclose%
\pgfusepath{stroke,fill}%
\end{pgfscope}%
\begin{pgfscope}%
\pgfpathrectangle{\pgfqpoint{1.016621in}{0.499691in}}{\pgfqpoint{3.875000in}{2.695000in}}%
\pgfusepath{clip}%
\pgfsetbuttcap%
\pgfsetroundjoin%
\definecolor{currentfill}{rgb}{0.117647,0.564706,1.000000}%
\pgfsetfillcolor{currentfill}%
\pgfsetlinewidth{1.003750pt}%
\definecolor{currentstroke}{rgb}{0.117647,0.564706,1.000000}%
\pgfsetstrokecolor{currentstroke}%
\pgfsetdash{}{0pt}%
\pgfpathmoveto{\pgfqpoint{3.953669in}{2.033483in}}%
\pgfpathcurveto{\pgfqpoint{3.964719in}{2.033483in}}{\pgfqpoint{3.975318in}{2.037873in}}{\pgfqpoint{3.983132in}{2.045687in}}%
\pgfpathcurveto{\pgfqpoint{3.990946in}{2.053501in}}{\pgfqpoint{3.995336in}{2.064100in}}{\pgfqpoint{3.995336in}{2.075150in}}%
\pgfpathcurveto{\pgfqpoint{3.995336in}{2.086200in}}{\pgfqpoint{3.990946in}{2.096799in}}{\pgfqpoint{3.983132in}{2.104613in}}%
\pgfpathcurveto{\pgfqpoint{3.975318in}{2.112426in}}{\pgfqpoint{3.964719in}{2.116817in}}{\pgfqpoint{3.953669in}{2.116817in}}%
\pgfpathcurveto{\pgfqpoint{3.942619in}{2.116817in}}{\pgfqpoint{3.932020in}{2.112426in}}{\pgfqpoint{3.924206in}{2.104613in}}%
\pgfpathcurveto{\pgfqpoint{3.916393in}{2.096799in}}{\pgfqpoint{3.912003in}{2.086200in}}{\pgfqpoint{3.912003in}{2.075150in}}%
\pgfpathcurveto{\pgfqpoint{3.912003in}{2.064100in}}{\pgfqpoint{3.916393in}{2.053501in}}{\pgfqpoint{3.924206in}{2.045687in}}%
\pgfpathcurveto{\pgfqpoint{3.932020in}{2.037873in}}{\pgfqpoint{3.942619in}{2.033483in}}{\pgfqpoint{3.953669in}{2.033483in}}%
\pgfpathclose%
\pgfusepath{stroke,fill}%
\end{pgfscope}%
\begin{pgfscope}%
\pgfpathrectangle{\pgfqpoint{1.016621in}{0.499691in}}{\pgfqpoint{3.875000in}{2.695000in}}%
\pgfusepath{clip}%
\pgfsetbuttcap%
\pgfsetroundjoin%
\definecolor{currentfill}{rgb}{0.117647,0.564706,1.000000}%
\pgfsetfillcolor{currentfill}%
\pgfsetlinewidth{1.003750pt}%
\definecolor{currentstroke}{rgb}{0.117647,0.564706,1.000000}%
\pgfsetstrokecolor{currentstroke}%
\pgfsetdash{}{0pt}%
\pgfpathmoveto{\pgfqpoint{3.978503in}{2.084708in}}%
\pgfpathcurveto{\pgfqpoint{3.989553in}{2.084708in}}{\pgfqpoint{4.000152in}{2.089098in}}{\pgfqpoint{4.007966in}{2.096912in}}%
\pgfpathcurveto{\pgfqpoint{4.015779in}{2.104725in}}{\pgfqpoint{4.020169in}{2.115324in}}{\pgfqpoint{4.020169in}{2.126374in}}%
\pgfpathcurveto{\pgfqpoint{4.020169in}{2.137424in}}{\pgfqpoint{4.015779in}{2.148024in}}{\pgfqpoint{4.007966in}{2.155837in}}%
\pgfpathcurveto{\pgfqpoint{4.000152in}{2.163651in}}{\pgfqpoint{3.989553in}{2.168041in}}{\pgfqpoint{3.978503in}{2.168041in}}%
\pgfpathcurveto{\pgfqpoint{3.967453in}{2.168041in}}{\pgfqpoint{3.956854in}{2.163651in}}{\pgfqpoint{3.949040in}{2.155837in}}%
\pgfpathcurveto{\pgfqpoint{3.941226in}{2.148024in}}{\pgfqpoint{3.936836in}{2.137424in}}{\pgfqpoint{3.936836in}{2.126374in}}%
\pgfpathcurveto{\pgfqpoint{3.936836in}{2.115324in}}{\pgfqpoint{3.941226in}{2.104725in}}{\pgfqpoint{3.949040in}{2.096912in}}%
\pgfpathcurveto{\pgfqpoint{3.956854in}{2.089098in}}{\pgfqpoint{3.967453in}{2.084708in}}{\pgfqpoint{3.978503in}{2.084708in}}%
\pgfpathclose%
\pgfusepath{stroke,fill}%
\end{pgfscope}%
\begin{pgfscope}%
\pgfpathrectangle{\pgfqpoint{1.016621in}{0.499691in}}{\pgfqpoint{3.875000in}{2.695000in}}%
\pgfusepath{clip}%
\pgfsetbuttcap%
\pgfsetroundjoin%
\definecolor{currentfill}{rgb}{0.117647,0.564706,1.000000}%
\pgfsetfillcolor{currentfill}%
\pgfsetlinewidth{1.003750pt}%
\definecolor{currentstroke}{rgb}{0.117647,0.564706,1.000000}%
\pgfsetstrokecolor{currentstroke}%
\pgfsetdash{}{0pt}%
\pgfpathmoveto{\pgfqpoint{3.990920in}{2.096904in}}%
\pgfpathcurveto{\pgfqpoint{4.001970in}{2.096904in}}{\pgfqpoint{4.012569in}{2.101294in}}{\pgfqpoint{4.020382in}{2.109108in}}%
\pgfpathcurveto{\pgfqpoint{4.028196in}{2.116921in}}{\pgfqpoint{4.032586in}{2.127521in}}{\pgfqpoint{4.032586in}{2.138571in}}%
\pgfpathcurveto{\pgfqpoint{4.032586in}{2.149621in}}{\pgfqpoint{4.028196in}{2.160220in}}{\pgfqpoint{4.020382in}{2.168033in}}%
\pgfpathcurveto{\pgfqpoint{4.012569in}{2.175847in}}{\pgfqpoint{4.001970in}{2.180237in}}{\pgfqpoint{3.990920in}{2.180237in}}%
\pgfpathcurveto{\pgfqpoint{3.979869in}{2.180237in}}{\pgfqpoint{3.969270in}{2.175847in}}{\pgfqpoint{3.961457in}{2.168033in}}%
\pgfpathcurveto{\pgfqpoint{3.953643in}{2.160220in}}{\pgfqpoint{3.949253in}{2.149621in}}{\pgfqpoint{3.949253in}{2.138571in}}%
\pgfpathcurveto{\pgfqpoint{3.949253in}{2.127521in}}{\pgfqpoint{3.953643in}{2.116921in}}{\pgfqpoint{3.961457in}{2.109108in}}%
\pgfpathcurveto{\pgfqpoint{3.969270in}{2.101294in}}{\pgfqpoint{3.979869in}{2.096904in}}{\pgfqpoint{3.990920in}{2.096904in}}%
\pgfpathclose%
\pgfusepath{stroke,fill}%
\end{pgfscope}%
\begin{pgfscope}%
\pgfpathrectangle{\pgfqpoint{1.016621in}{0.499691in}}{\pgfqpoint{3.875000in}{2.695000in}}%
\pgfusepath{clip}%
\pgfsetbuttcap%
\pgfsetroundjoin%
\definecolor{currentfill}{rgb}{0.117647,0.564706,1.000000}%
\pgfsetfillcolor{currentfill}%
\pgfsetlinewidth{1.003750pt}%
\definecolor{currentstroke}{rgb}{0.117647,0.564706,1.000000}%
\pgfsetstrokecolor{currentstroke}%
\pgfsetdash{}{0pt}%
\pgfpathmoveto{\pgfqpoint{4.003336in}{2.128614in}}%
\pgfpathcurveto{\pgfqpoint{4.014386in}{2.128614in}}{\pgfqpoint{4.024985in}{2.133005in}}{\pgfqpoint{4.032799in}{2.140818in}}%
\pgfpathcurveto{\pgfqpoint{4.040613in}{2.148632in}}{\pgfqpoint{4.045003in}{2.159231in}}{\pgfqpoint{4.045003in}{2.170281in}}%
\pgfpathcurveto{\pgfqpoint{4.045003in}{2.181331in}}{\pgfqpoint{4.040613in}{2.191930in}}{\pgfqpoint{4.032799in}{2.199744in}}%
\pgfpathcurveto{\pgfqpoint{4.024985in}{2.207557in}}{\pgfqpoint{4.014386in}{2.211948in}}{\pgfqpoint{4.003336in}{2.211948in}}%
\pgfpathcurveto{\pgfqpoint{3.992286in}{2.211948in}}{\pgfqpoint{3.981687in}{2.207557in}}{\pgfqpoint{3.973873in}{2.199744in}}%
\pgfpathcurveto{\pgfqpoint{3.966060in}{2.191930in}}{\pgfqpoint{3.961670in}{2.181331in}}{\pgfqpoint{3.961670in}{2.170281in}}%
\pgfpathcurveto{\pgfqpoint{3.961670in}{2.159231in}}{\pgfqpoint{3.966060in}{2.148632in}}{\pgfqpoint{3.973873in}{2.140818in}}%
\pgfpathcurveto{\pgfqpoint{3.981687in}{2.133005in}}{\pgfqpoint{3.992286in}{2.128614in}}{\pgfqpoint{4.003336in}{2.128614in}}%
\pgfpathclose%
\pgfusepath{stroke,fill}%
\end{pgfscope}%
\begin{pgfscope}%
\pgfpathrectangle{\pgfqpoint{1.016621in}{0.499691in}}{\pgfqpoint{3.875000in}{2.695000in}}%
\pgfusepath{clip}%
\pgfsetbuttcap%
\pgfsetroundjoin%
\definecolor{currentfill}{rgb}{0.117647,0.564706,1.000000}%
\pgfsetfillcolor{currentfill}%
\pgfsetlinewidth{1.003750pt}%
\definecolor{currentstroke}{rgb}{0.117647,0.564706,1.000000}%
\pgfsetstrokecolor{currentstroke}%
\pgfsetdash{}{0pt}%
\pgfpathmoveto{\pgfqpoint{4.028170in}{2.148128in}}%
\pgfpathcurveto{\pgfqpoint{4.039220in}{2.148128in}}{\pgfqpoint{4.049819in}{2.152519in}}{\pgfqpoint{4.057633in}{2.160332in}}%
\pgfpathcurveto{\pgfqpoint{4.065446in}{2.168146in}}{\pgfqpoint{4.069836in}{2.178745in}}{\pgfqpoint{4.069836in}{2.189795in}}%
\pgfpathcurveto{\pgfqpoint{4.069836in}{2.200845in}}{\pgfqpoint{4.065446in}{2.211444in}}{\pgfqpoint{4.057633in}{2.219258in}}%
\pgfpathcurveto{\pgfqpoint{4.049819in}{2.227071in}}{\pgfqpoint{4.039220in}{2.231462in}}{\pgfqpoint{4.028170in}{2.231462in}}%
\pgfpathcurveto{\pgfqpoint{4.017120in}{2.231462in}}{\pgfqpoint{4.006521in}{2.227071in}}{\pgfqpoint{3.998707in}{2.219258in}}%
\pgfpathcurveto{\pgfqpoint{3.990893in}{2.211444in}}{\pgfqpoint{3.986503in}{2.200845in}}{\pgfqpoint{3.986503in}{2.189795in}}%
\pgfpathcurveto{\pgfqpoint{3.986503in}{2.178745in}}{\pgfqpoint{3.990893in}{2.168146in}}{\pgfqpoint{3.998707in}{2.160332in}}%
\pgfpathcurveto{\pgfqpoint{4.006521in}{2.152519in}}{\pgfqpoint{4.017120in}{2.148128in}}{\pgfqpoint{4.028170in}{2.148128in}}%
\pgfpathclose%
\pgfusepath{stroke,fill}%
\end{pgfscope}%
\begin{pgfscope}%
\pgfpathrectangle{\pgfqpoint{1.016621in}{0.499691in}}{\pgfqpoint{3.875000in}{2.695000in}}%
\pgfusepath{clip}%
\pgfsetbuttcap%
\pgfsetroundjoin%
\definecolor{currentfill}{rgb}{0.117647,0.564706,1.000000}%
\pgfsetfillcolor{currentfill}%
\pgfsetlinewidth{1.003750pt}%
\definecolor{currentstroke}{rgb}{0.117647,0.564706,1.000000}%
\pgfsetstrokecolor{currentstroke}%
\pgfsetdash{}{0pt}%
\pgfpathmoveto{\pgfqpoint{4.040587in}{2.170082in}}%
\pgfpathcurveto{\pgfqpoint{4.051637in}{2.170082in}}{\pgfqpoint{4.062236in}{2.174472in}}{\pgfqpoint{4.070049in}{2.182286in}}%
\pgfpathcurveto{\pgfqpoint{4.077863in}{2.190099in}}{\pgfqpoint{4.082253in}{2.200698in}}{\pgfqpoint{4.082253in}{2.211748in}}%
\pgfpathcurveto{\pgfqpoint{4.082253in}{2.222799in}}{\pgfqpoint{4.077863in}{2.233398in}}{\pgfqpoint{4.070049in}{2.241211in}}%
\pgfpathcurveto{\pgfqpoint{4.062236in}{2.249025in}}{\pgfqpoint{4.051637in}{2.253415in}}{\pgfqpoint{4.040587in}{2.253415in}}%
\pgfpathcurveto{\pgfqpoint{4.029536in}{2.253415in}}{\pgfqpoint{4.018937in}{2.249025in}}{\pgfqpoint{4.011124in}{2.241211in}}%
\pgfpathcurveto{\pgfqpoint{4.003310in}{2.233398in}}{\pgfqpoint{3.998920in}{2.222799in}}{\pgfqpoint{3.998920in}{2.211748in}}%
\pgfpathcurveto{\pgfqpoint{3.998920in}{2.200698in}}{\pgfqpoint{4.003310in}{2.190099in}}{\pgfqpoint{4.011124in}{2.182286in}}%
\pgfpathcurveto{\pgfqpoint{4.018937in}{2.174472in}}{\pgfqpoint{4.029536in}{2.170082in}}{\pgfqpoint{4.040587in}{2.170082in}}%
\pgfpathclose%
\pgfusepath{stroke,fill}%
\end{pgfscope}%
\begin{pgfscope}%
\pgfpathrectangle{\pgfqpoint{1.016621in}{0.499691in}}{\pgfqpoint{3.875000in}{2.695000in}}%
\pgfusepath{clip}%
\pgfsetbuttcap%
\pgfsetroundjoin%
\definecolor{currentfill}{rgb}{0.117647,0.564706,1.000000}%
\pgfsetfillcolor{currentfill}%
\pgfsetlinewidth{1.003750pt}%
\definecolor{currentstroke}{rgb}{0.117647,0.564706,1.000000}%
\pgfsetstrokecolor{currentstroke}%
\pgfsetdash{}{0pt}%
\pgfpathmoveto{\pgfqpoint{4.065420in}{2.213988in}}%
\pgfpathcurveto{\pgfqpoint{4.076470in}{2.213988in}}{\pgfqpoint{4.087069in}{2.218379in}}{\pgfqpoint{4.094883in}{2.226192in}}%
\pgfpathcurveto{\pgfqpoint{4.102696in}{2.234006in}}{\pgfqpoint{4.107087in}{2.244605in}}{\pgfqpoint{4.107087in}{2.255655in}}%
\pgfpathcurveto{\pgfqpoint{4.107087in}{2.266705in}}{\pgfqpoint{4.102696in}{2.277304in}}{\pgfqpoint{4.094883in}{2.285118in}}%
\pgfpathcurveto{\pgfqpoint{4.087069in}{2.292931in}}{\pgfqpoint{4.076470in}{2.297322in}}{\pgfqpoint{4.065420in}{2.297322in}}%
\pgfpathcurveto{\pgfqpoint{4.054370in}{2.297322in}}{\pgfqpoint{4.043771in}{2.292931in}}{\pgfqpoint{4.035957in}{2.285118in}}%
\pgfpathcurveto{\pgfqpoint{4.028144in}{2.277304in}}{\pgfqpoint{4.023753in}{2.266705in}}{\pgfqpoint{4.023753in}{2.255655in}}%
\pgfpathcurveto{\pgfqpoint{4.023753in}{2.244605in}}{\pgfqpoint{4.028144in}{2.234006in}}{\pgfqpoint{4.035957in}{2.226192in}}%
\pgfpathcurveto{\pgfqpoint{4.043771in}{2.218379in}}{\pgfqpoint{4.054370in}{2.213988in}}{\pgfqpoint{4.065420in}{2.213988in}}%
\pgfpathclose%
\pgfusepath{stroke,fill}%
\end{pgfscope}%
\begin{pgfscope}%
\pgfpathrectangle{\pgfqpoint{1.016621in}{0.499691in}}{\pgfqpoint{3.875000in}{2.695000in}}%
\pgfusepath{clip}%
\pgfsetbuttcap%
\pgfsetroundjoin%
\definecolor{currentfill}{rgb}{0.117647,0.564706,1.000000}%
\pgfsetfillcolor{currentfill}%
\pgfsetlinewidth{1.003750pt}%
\definecolor{currentstroke}{rgb}{0.117647,0.564706,1.000000}%
\pgfsetstrokecolor{currentstroke}%
\pgfsetdash{}{0pt}%
\pgfpathmoveto{\pgfqpoint{4.090254in}{2.292045in}}%
\pgfpathcurveto{\pgfqpoint{4.101304in}{2.292045in}}{\pgfqpoint{4.111903in}{2.296435in}}{\pgfqpoint{4.119716in}{2.304249in}}%
\pgfpathcurveto{\pgfqpoint{4.127530in}{2.312062in}}{\pgfqpoint{4.131920in}{2.322661in}}{\pgfqpoint{4.131920in}{2.333711in}}%
\pgfpathcurveto{\pgfqpoint{4.131920in}{2.344762in}}{\pgfqpoint{4.127530in}{2.355361in}}{\pgfqpoint{4.119716in}{2.363174in}}%
\pgfpathcurveto{\pgfqpoint{4.111903in}{2.370988in}}{\pgfqpoint{4.101304in}{2.375378in}}{\pgfqpoint{4.090254in}{2.375378in}}%
\pgfpathcurveto{\pgfqpoint{4.079203in}{2.375378in}}{\pgfqpoint{4.068604in}{2.370988in}}{\pgfqpoint{4.060791in}{2.363174in}}%
\pgfpathcurveto{\pgfqpoint{4.052977in}{2.355361in}}{\pgfqpoint{4.048587in}{2.344762in}}{\pgfqpoint{4.048587in}{2.333711in}}%
\pgfpathcurveto{\pgfqpoint{4.048587in}{2.322661in}}{\pgfqpoint{4.052977in}{2.312062in}}{\pgfqpoint{4.060791in}{2.304249in}}%
\pgfpathcurveto{\pgfqpoint{4.068604in}{2.296435in}}{\pgfqpoint{4.079203in}{2.292045in}}{\pgfqpoint{4.090254in}{2.292045in}}%
\pgfpathclose%
\pgfusepath{stroke,fill}%
\end{pgfscope}%
\begin{pgfscope}%
\pgfpathrectangle{\pgfqpoint{1.016621in}{0.499691in}}{\pgfqpoint{3.875000in}{2.695000in}}%
\pgfusepath{clip}%
\pgfsetbuttcap%
\pgfsetroundjoin%
\definecolor{currentfill}{rgb}{0.117647,0.564706,1.000000}%
\pgfsetfillcolor{currentfill}%
\pgfsetlinewidth{1.003750pt}%
\definecolor{currentstroke}{rgb}{0.117647,0.564706,1.000000}%
\pgfsetstrokecolor{currentstroke}%
\pgfsetdash{}{0pt}%
\pgfpathmoveto{\pgfqpoint{4.115087in}{2.309120in}}%
\pgfpathcurveto{\pgfqpoint{4.126137in}{2.309120in}}{\pgfqpoint{4.136736in}{2.313510in}}{\pgfqpoint{4.144550in}{2.321323in}}%
\pgfpathcurveto{\pgfqpoint{4.152363in}{2.329137in}}{\pgfqpoint{4.156754in}{2.339736in}}{\pgfqpoint{4.156754in}{2.350786in}}%
\pgfpathcurveto{\pgfqpoint{4.156754in}{2.361836in}}{\pgfqpoint{4.152363in}{2.372435in}}{\pgfqpoint{4.144550in}{2.380249in}}%
\pgfpathcurveto{\pgfqpoint{4.136736in}{2.388063in}}{\pgfqpoint{4.126137in}{2.392453in}}{\pgfqpoint{4.115087in}{2.392453in}}%
\pgfpathcurveto{\pgfqpoint{4.104037in}{2.392453in}}{\pgfqpoint{4.093438in}{2.388063in}}{\pgfqpoint{4.085624in}{2.380249in}}%
\pgfpathcurveto{\pgfqpoint{4.077811in}{2.372435in}}{\pgfqpoint{4.073420in}{2.361836in}}{\pgfqpoint{4.073420in}{2.350786in}}%
\pgfpathcurveto{\pgfqpoint{4.073420in}{2.339736in}}{\pgfqpoint{4.077811in}{2.329137in}}{\pgfqpoint{4.085624in}{2.321323in}}%
\pgfpathcurveto{\pgfqpoint{4.093438in}{2.313510in}}{\pgfqpoint{4.104037in}{2.309120in}}{\pgfqpoint{4.115087in}{2.309120in}}%
\pgfpathclose%
\pgfusepath{stroke,fill}%
\end{pgfscope}%
\begin{pgfscope}%
\pgfpathrectangle{\pgfqpoint{1.016621in}{0.499691in}}{\pgfqpoint{3.875000in}{2.695000in}}%
\pgfusepath{clip}%
\pgfsetbuttcap%
\pgfsetroundjoin%
\definecolor{currentfill}{rgb}{0.117647,0.564706,1.000000}%
\pgfsetfillcolor{currentfill}%
\pgfsetlinewidth{1.003750pt}%
\definecolor{currentstroke}{rgb}{0.117647,0.564706,1.000000}%
\pgfsetstrokecolor{currentstroke}%
\pgfsetdash{}{0pt}%
\pgfpathmoveto{\pgfqpoint{4.177171in}{2.335951in}}%
\pgfpathcurveto{\pgfqpoint{4.188221in}{2.335951in}}{\pgfqpoint{4.198820in}{2.340342in}}{\pgfqpoint{4.206634in}{2.348155in}}%
\pgfpathcurveto{\pgfqpoint{4.214447in}{2.355969in}}{\pgfqpoint{4.218837in}{2.366568in}}{\pgfqpoint{4.218837in}{2.377618in}}%
\pgfpathcurveto{\pgfqpoint{4.218837in}{2.388668in}}{\pgfqpoint{4.214447in}{2.399267in}}{\pgfqpoint{4.206634in}{2.407081in}}%
\pgfpathcurveto{\pgfqpoint{4.198820in}{2.414894in}}{\pgfqpoint{4.188221in}{2.419285in}}{\pgfqpoint{4.177171in}{2.419285in}}%
\pgfpathcurveto{\pgfqpoint{4.166121in}{2.419285in}}{\pgfqpoint{4.155522in}{2.414894in}}{\pgfqpoint{4.147708in}{2.407081in}}%
\pgfpathcurveto{\pgfqpoint{4.139894in}{2.399267in}}{\pgfqpoint{4.135504in}{2.388668in}}{\pgfqpoint{4.135504in}{2.377618in}}%
\pgfpathcurveto{\pgfqpoint{4.135504in}{2.366568in}}{\pgfqpoint{4.139894in}{2.355969in}}{\pgfqpoint{4.147708in}{2.348155in}}%
\pgfpathcurveto{\pgfqpoint{4.155522in}{2.340342in}}{\pgfqpoint{4.166121in}{2.335951in}}{\pgfqpoint{4.177171in}{2.335951in}}%
\pgfpathclose%
\pgfusepath{stroke,fill}%
\end{pgfscope}%
\begin{pgfscope}%
\pgfpathrectangle{\pgfqpoint{1.016621in}{0.499691in}}{\pgfqpoint{3.875000in}{2.695000in}}%
\pgfusepath{clip}%
\pgfsetbuttcap%
\pgfsetroundjoin%
\definecolor{currentfill}{rgb}{0.117647,0.564706,1.000000}%
\pgfsetfillcolor{currentfill}%
\pgfsetlinewidth{1.003750pt}%
\definecolor{currentstroke}{rgb}{0.117647,0.564706,1.000000}%
\pgfsetstrokecolor{currentstroke}%
\pgfsetdash{}{0pt}%
\pgfpathmoveto{\pgfqpoint{4.226838in}{2.433522in}}%
\pgfpathcurveto{\pgfqpoint{4.237888in}{2.433522in}}{\pgfqpoint{4.248487in}{2.437912in}}{\pgfqpoint{4.256301in}{2.445726in}}%
\pgfpathcurveto{\pgfqpoint{4.264114in}{2.453539in}}{\pgfqpoint{4.268504in}{2.464138in}}{\pgfqpoint{4.268504in}{2.475188in}}%
\pgfpathcurveto{\pgfqpoint{4.268504in}{2.486239in}}{\pgfqpoint{4.264114in}{2.496838in}}{\pgfqpoint{4.256301in}{2.504651in}}%
\pgfpathcurveto{\pgfqpoint{4.248487in}{2.512465in}}{\pgfqpoint{4.237888in}{2.516855in}}{\pgfqpoint{4.226838in}{2.516855in}}%
\pgfpathcurveto{\pgfqpoint{4.215788in}{2.516855in}}{\pgfqpoint{4.205189in}{2.512465in}}{\pgfqpoint{4.197375in}{2.504651in}}%
\pgfpathcurveto{\pgfqpoint{4.189561in}{2.496838in}}{\pgfqpoint{4.185171in}{2.486239in}}{\pgfqpoint{4.185171in}{2.475188in}}%
\pgfpathcurveto{\pgfqpoint{4.185171in}{2.464138in}}{\pgfqpoint{4.189561in}{2.453539in}}{\pgfqpoint{4.197375in}{2.445726in}}%
\pgfpathcurveto{\pgfqpoint{4.205189in}{2.437912in}}{\pgfqpoint{4.215788in}{2.433522in}}{\pgfqpoint{4.226838in}{2.433522in}}%
\pgfpathclose%
\pgfusepath{stroke,fill}%
\end{pgfscope}%
\begin{pgfscope}%
\pgfpathrectangle{\pgfqpoint{1.016621in}{0.499691in}}{\pgfqpoint{3.875000in}{2.695000in}}%
\pgfusepath{clip}%
\pgfsetbuttcap%
\pgfsetroundjoin%
\definecolor{currentfill}{rgb}{0.117647,0.564706,1.000000}%
\pgfsetfillcolor{currentfill}%
\pgfsetlinewidth{1.003750pt}%
\definecolor{currentstroke}{rgb}{0.117647,0.564706,1.000000}%
\pgfsetstrokecolor{currentstroke}%
\pgfsetdash{}{0pt}%
\pgfpathmoveto{\pgfqpoint{4.400672in}{2.579877in}}%
\pgfpathcurveto{\pgfqpoint{4.411722in}{2.579877in}}{\pgfqpoint{4.422321in}{2.584268in}}{\pgfqpoint{4.430135in}{2.592081in}}%
\pgfpathcurveto{\pgfqpoint{4.437949in}{2.599895in}}{\pgfqpoint{4.442339in}{2.610494in}}{\pgfqpoint{4.442339in}{2.621544in}}%
\pgfpathcurveto{\pgfqpoint{4.442339in}{2.632594in}}{\pgfqpoint{4.437949in}{2.643193in}}{\pgfqpoint{4.430135in}{2.651007in}}%
\pgfpathcurveto{\pgfqpoint{4.422321in}{2.658820in}}{\pgfqpoint{4.411722in}{2.663211in}}{\pgfqpoint{4.400672in}{2.663211in}}%
\pgfpathcurveto{\pgfqpoint{4.389622in}{2.663211in}}{\pgfqpoint{4.379023in}{2.658820in}}{\pgfqpoint{4.371209in}{2.651007in}}%
\pgfpathcurveto{\pgfqpoint{4.363396in}{2.643193in}}{\pgfqpoint{4.359006in}{2.632594in}}{\pgfqpoint{4.359006in}{2.621544in}}%
\pgfpathcurveto{\pgfqpoint{4.359006in}{2.610494in}}{\pgfqpoint{4.363396in}{2.599895in}}{\pgfqpoint{4.371209in}{2.592081in}}%
\pgfpathcurveto{\pgfqpoint{4.379023in}{2.584268in}}{\pgfqpoint{4.389622in}{2.579877in}}{\pgfqpoint{4.400672in}{2.579877in}}%
\pgfpathclose%
\pgfusepath{stroke,fill}%
\end{pgfscope}%
\begin{pgfscope}%
\pgfpathrectangle{\pgfqpoint{1.016621in}{0.499691in}}{\pgfqpoint{3.875000in}{2.695000in}}%
\pgfusepath{clip}%
\pgfsetbuttcap%
\pgfsetroundjoin%
\definecolor{currentfill}{rgb}{0.117647,0.564706,1.000000}%
\pgfsetfillcolor{currentfill}%
\pgfsetlinewidth{1.003750pt}%
\definecolor{currentstroke}{rgb}{0.117647,0.564706,1.000000}%
\pgfsetstrokecolor{currentstroke}%
\pgfsetdash{}{0pt}%
\pgfpathmoveto{\pgfqpoint{4.711091in}{2.726233in}}%
\pgfpathcurveto{\pgfqpoint{4.722141in}{2.726233in}}{\pgfqpoint{4.732740in}{2.730623in}}{\pgfqpoint{4.740554in}{2.738437in}}%
\pgfpathcurveto{\pgfqpoint{4.748367in}{2.746250in}}{\pgfqpoint{4.752758in}{2.756849in}}{\pgfqpoint{4.752758in}{2.767900in}}%
\pgfpathcurveto{\pgfqpoint{4.752758in}{2.778950in}}{\pgfqpoint{4.748367in}{2.789549in}}{\pgfqpoint{4.740554in}{2.797362in}}%
\pgfpathcurveto{\pgfqpoint{4.732740in}{2.805176in}}{\pgfqpoint{4.722141in}{2.809566in}}{\pgfqpoint{4.711091in}{2.809566in}}%
\pgfpathcurveto{\pgfqpoint{4.700041in}{2.809566in}}{\pgfqpoint{4.689442in}{2.805176in}}{\pgfqpoint{4.681628in}{2.797362in}}%
\pgfpathcurveto{\pgfqpoint{4.673815in}{2.789549in}}{\pgfqpoint{4.669424in}{2.778950in}}{\pgfqpoint{4.669424in}{2.767900in}}%
\pgfpathcurveto{\pgfqpoint{4.669424in}{2.756849in}}{\pgfqpoint{4.673815in}{2.746250in}}{\pgfqpoint{4.681628in}{2.738437in}}%
\pgfpathcurveto{\pgfqpoint{4.689442in}{2.730623in}}{\pgfqpoint{4.700041in}{2.726233in}}{\pgfqpoint{4.711091in}{2.726233in}}%
\pgfpathclose%
\pgfusepath{stroke,fill}%
\end{pgfscope}%
\begin{pgfscope}%
\pgfsetbuttcap%
\pgfsetroundjoin%
\definecolor{currentfill}{rgb}{0.000000,0.000000,0.000000}%
\pgfsetfillcolor{currentfill}%
\pgfsetlinewidth{0.803000pt}%
\definecolor{currentstroke}{rgb}{0.000000,0.000000,0.000000}%
\pgfsetstrokecolor{currentstroke}%
\pgfsetdash{}{0pt}%
\pgfsys@defobject{currentmarker}{\pgfqpoint{0.000000in}{-0.048611in}}{\pgfqpoint{0.000000in}{0.000000in}}{%
\pgfpathmoveto{\pgfqpoint{0.000000in}{0.000000in}}%
\pgfpathlineto{\pgfqpoint{0.000000in}{-0.048611in}}%
\pgfusepath{stroke,fill}%
}%
\begin{pgfscope}%
\pgfsys@transformshift{1.197151in}{0.499691in}%
\pgfsys@useobject{currentmarker}{}%
\end{pgfscope}%
\end{pgfscope}%
\begin{pgfscope}%
\definecolor{textcolor}{rgb}{0.000000,0.000000,0.000000}%
\pgfsetstrokecolor{textcolor}%
\pgfsetfillcolor{textcolor}%
\pgftext[x=1.197151in,y=0.402469in,,top]{\color{textcolor}\rmfamily\fontsize{10.000000}{12.000000}\selectfont \(\displaystyle 1.0\)}%
\end{pgfscope}%
\begin{pgfscope}%
\pgfsetbuttcap%
\pgfsetroundjoin%
\definecolor{currentfill}{rgb}{0.000000,0.000000,0.000000}%
\pgfsetfillcolor{currentfill}%
\pgfsetlinewidth{0.803000pt}%
\definecolor{currentstroke}{rgb}{0.000000,0.000000,0.000000}%
\pgfsetstrokecolor{currentstroke}%
\pgfsetdash{}{0pt}%
\pgfsys@defobject{currentmarker}{\pgfqpoint{0.000000in}{-0.048611in}}{\pgfqpoint{0.000000in}{0.000000in}}{%
\pgfpathmoveto{\pgfqpoint{0.000000in}{0.000000in}}%
\pgfpathlineto{\pgfqpoint{0.000000in}{-0.048611in}}%
\pgfusepath{stroke,fill}%
}%
\begin{pgfscope}%
\pgfsys@transformshift{1.817988in}{0.499691in}%
\pgfsys@useobject{currentmarker}{}%
\end{pgfscope}%
\end{pgfscope}%
\begin{pgfscope}%
\definecolor{textcolor}{rgb}{0.000000,0.000000,0.000000}%
\pgfsetstrokecolor{textcolor}%
\pgfsetfillcolor{textcolor}%
\pgftext[x=1.817988in,y=0.402469in,,top]{\color{textcolor}\rmfamily\fontsize{10.000000}{12.000000}\selectfont \(\displaystyle 1.5\)}%
\end{pgfscope}%
\begin{pgfscope}%
\pgfsetbuttcap%
\pgfsetroundjoin%
\definecolor{currentfill}{rgb}{0.000000,0.000000,0.000000}%
\pgfsetfillcolor{currentfill}%
\pgfsetlinewidth{0.803000pt}%
\definecolor{currentstroke}{rgb}{0.000000,0.000000,0.000000}%
\pgfsetstrokecolor{currentstroke}%
\pgfsetdash{}{0pt}%
\pgfsys@defobject{currentmarker}{\pgfqpoint{0.000000in}{-0.048611in}}{\pgfqpoint{0.000000in}{0.000000in}}{%
\pgfpathmoveto{\pgfqpoint{0.000000in}{0.000000in}}%
\pgfpathlineto{\pgfqpoint{0.000000in}{-0.048611in}}%
\pgfusepath{stroke,fill}%
}%
\begin{pgfscope}%
\pgfsys@transformshift{2.438826in}{0.499691in}%
\pgfsys@useobject{currentmarker}{}%
\end{pgfscope}%
\end{pgfscope}%
\begin{pgfscope}%
\definecolor{textcolor}{rgb}{0.000000,0.000000,0.000000}%
\pgfsetstrokecolor{textcolor}%
\pgfsetfillcolor{textcolor}%
\pgftext[x=2.438826in,y=0.402469in,,top]{\color{textcolor}\rmfamily\fontsize{10.000000}{12.000000}\selectfont \(\displaystyle 2.0\)}%
\end{pgfscope}%
\begin{pgfscope}%
\pgfsetbuttcap%
\pgfsetroundjoin%
\definecolor{currentfill}{rgb}{0.000000,0.000000,0.000000}%
\pgfsetfillcolor{currentfill}%
\pgfsetlinewidth{0.803000pt}%
\definecolor{currentstroke}{rgb}{0.000000,0.000000,0.000000}%
\pgfsetstrokecolor{currentstroke}%
\pgfsetdash{}{0pt}%
\pgfsys@defobject{currentmarker}{\pgfqpoint{0.000000in}{-0.048611in}}{\pgfqpoint{0.000000in}{0.000000in}}{%
\pgfpathmoveto{\pgfqpoint{0.000000in}{0.000000in}}%
\pgfpathlineto{\pgfqpoint{0.000000in}{-0.048611in}}%
\pgfusepath{stroke,fill}%
}%
\begin{pgfscope}%
\pgfsys@transformshift{3.059663in}{0.499691in}%
\pgfsys@useobject{currentmarker}{}%
\end{pgfscope}%
\end{pgfscope}%
\begin{pgfscope}%
\definecolor{textcolor}{rgb}{0.000000,0.000000,0.000000}%
\pgfsetstrokecolor{textcolor}%
\pgfsetfillcolor{textcolor}%
\pgftext[x=3.059663in,y=0.402469in,,top]{\color{textcolor}\rmfamily\fontsize{10.000000}{12.000000}\selectfont \(\displaystyle 2.5\)}%
\end{pgfscope}%
\begin{pgfscope}%
\pgfsetbuttcap%
\pgfsetroundjoin%
\definecolor{currentfill}{rgb}{0.000000,0.000000,0.000000}%
\pgfsetfillcolor{currentfill}%
\pgfsetlinewidth{0.803000pt}%
\definecolor{currentstroke}{rgb}{0.000000,0.000000,0.000000}%
\pgfsetstrokecolor{currentstroke}%
\pgfsetdash{}{0pt}%
\pgfsys@defobject{currentmarker}{\pgfqpoint{0.000000in}{-0.048611in}}{\pgfqpoint{0.000000in}{0.000000in}}{%
\pgfpathmoveto{\pgfqpoint{0.000000in}{0.000000in}}%
\pgfpathlineto{\pgfqpoint{0.000000in}{-0.048611in}}%
\pgfusepath{stroke,fill}%
}%
\begin{pgfscope}%
\pgfsys@transformshift{3.680501in}{0.499691in}%
\pgfsys@useobject{currentmarker}{}%
\end{pgfscope}%
\end{pgfscope}%
\begin{pgfscope}%
\definecolor{textcolor}{rgb}{0.000000,0.000000,0.000000}%
\pgfsetstrokecolor{textcolor}%
\pgfsetfillcolor{textcolor}%
\pgftext[x=3.680501in,y=0.402469in,,top]{\color{textcolor}\rmfamily\fontsize{10.000000}{12.000000}\selectfont \(\displaystyle 3.0\)}%
\end{pgfscope}%
\begin{pgfscope}%
\pgfsetbuttcap%
\pgfsetroundjoin%
\definecolor{currentfill}{rgb}{0.000000,0.000000,0.000000}%
\pgfsetfillcolor{currentfill}%
\pgfsetlinewidth{0.803000pt}%
\definecolor{currentstroke}{rgb}{0.000000,0.000000,0.000000}%
\pgfsetstrokecolor{currentstroke}%
\pgfsetdash{}{0pt}%
\pgfsys@defobject{currentmarker}{\pgfqpoint{0.000000in}{-0.048611in}}{\pgfqpoint{0.000000in}{0.000000in}}{%
\pgfpathmoveto{\pgfqpoint{0.000000in}{0.000000in}}%
\pgfpathlineto{\pgfqpoint{0.000000in}{-0.048611in}}%
\pgfusepath{stroke,fill}%
}%
\begin{pgfscope}%
\pgfsys@transformshift{4.301338in}{0.499691in}%
\pgfsys@useobject{currentmarker}{}%
\end{pgfscope}%
\end{pgfscope}%
\begin{pgfscope}%
\definecolor{textcolor}{rgb}{0.000000,0.000000,0.000000}%
\pgfsetstrokecolor{textcolor}%
\pgfsetfillcolor{textcolor}%
\pgftext[x=4.301338in,y=0.402469in,,top]{\color{textcolor}\rmfamily\fontsize{10.000000}{12.000000}\selectfont \(\displaystyle 3.5\)}%
\end{pgfscope}%
\begin{pgfscope}%
\definecolor{textcolor}{rgb}{0.000000,0.000000,0.000000}%
\pgfsetstrokecolor{textcolor}%
\pgfsetfillcolor{textcolor}%
\pgftext[x=2.954121in,y=0.223457in,,top]{\color{textcolor}\rmfamily\fontsize{10.000000}{12.000000}\selectfont log f}%
\end{pgfscope}%
\begin{pgfscope}%
\pgfsetbuttcap%
\pgfsetroundjoin%
\definecolor{currentfill}{rgb}{0.000000,0.000000,0.000000}%
\pgfsetfillcolor{currentfill}%
\pgfsetlinewidth{0.803000pt}%
\definecolor{currentstroke}{rgb}{0.000000,0.000000,0.000000}%
\pgfsetstrokecolor{currentstroke}%
\pgfsetdash{}{0pt}%
\pgfsys@defobject{currentmarker}{\pgfqpoint{-0.048611in}{0.000000in}}{\pgfqpoint{0.000000in}{0.000000in}}{%
\pgfpathmoveto{\pgfqpoint{0.000000in}{0.000000in}}%
\pgfpathlineto{\pgfqpoint{-0.048611in}{0.000000in}}%
\pgfusepath{stroke,fill}%
}%
\begin{pgfscope}%
\pgfsys@transformshift{1.016621in}{0.684772in}%
\pgfsys@useobject{currentmarker}{}%
\end{pgfscope}%
\end{pgfscope}%
\begin{pgfscope}%
\definecolor{textcolor}{rgb}{0.000000,0.000000,0.000000}%
\pgfsetstrokecolor{textcolor}%
\pgfsetfillcolor{textcolor}%
\pgftext[x=0.603040in,y=0.636547in,left,base]{\color{textcolor}\rmfamily\fontsize{10.000000}{12.000000}\selectfont \(\displaystyle -100\)}%
\end{pgfscope}%
\begin{pgfscope}%
\pgfsetbuttcap%
\pgfsetroundjoin%
\definecolor{currentfill}{rgb}{0.000000,0.000000,0.000000}%
\pgfsetfillcolor{currentfill}%
\pgfsetlinewidth{0.803000pt}%
\definecolor{currentstroke}{rgb}{0.000000,0.000000,0.000000}%
\pgfsetstrokecolor{currentstroke}%
\pgfsetdash{}{0pt}%
\pgfsys@defobject{currentmarker}{\pgfqpoint{-0.048611in}{0.000000in}}{\pgfqpoint{0.000000in}{0.000000in}}{%
\pgfpathmoveto{\pgfqpoint{0.000000in}{0.000000in}}%
\pgfpathlineto{\pgfqpoint{-0.048611in}{0.000000in}}%
\pgfusepath{stroke,fill}%
}%
\begin{pgfscope}%
\pgfsys@transformshift{1.016621in}{1.172624in}%
\pgfsys@useobject{currentmarker}{}%
\end{pgfscope}%
\end{pgfscope}%
\begin{pgfscope}%
\definecolor{textcolor}{rgb}{0.000000,0.000000,0.000000}%
\pgfsetstrokecolor{textcolor}%
\pgfsetfillcolor{textcolor}%
\pgftext[x=0.672484in,y=1.124399in,left,base]{\color{textcolor}\rmfamily\fontsize{10.000000}{12.000000}\selectfont \(\displaystyle -80\)}%
\end{pgfscope}%
\begin{pgfscope}%
\pgfsetbuttcap%
\pgfsetroundjoin%
\definecolor{currentfill}{rgb}{0.000000,0.000000,0.000000}%
\pgfsetfillcolor{currentfill}%
\pgfsetlinewidth{0.803000pt}%
\definecolor{currentstroke}{rgb}{0.000000,0.000000,0.000000}%
\pgfsetstrokecolor{currentstroke}%
\pgfsetdash{}{0pt}%
\pgfsys@defobject{currentmarker}{\pgfqpoint{-0.048611in}{0.000000in}}{\pgfqpoint{0.000000in}{0.000000in}}{%
\pgfpathmoveto{\pgfqpoint{0.000000in}{0.000000in}}%
\pgfpathlineto{\pgfqpoint{-0.048611in}{0.000000in}}%
\pgfusepath{stroke,fill}%
}%
\begin{pgfscope}%
\pgfsys@transformshift{1.016621in}{1.660476in}%
\pgfsys@useobject{currentmarker}{}%
\end{pgfscope}%
\end{pgfscope}%
\begin{pgfscope}%
\definecolor{textcolor}{rgb}{0.000000,0.000000,0.000000}%
\pgfsetstrokecolor{textcolor}%
\pgfsetfillcolor{textcolor}%
\pgftext[x=0.672484in,y=1.612251in,left,base]{\color{textcolor}\rmfamily\fontsize{10.000000}{12.000000}\selectfont \(\displaystyle -60\)}%
\end{pgfscope}%
\begin{pgfscope}%
\pgfsetbuttcap%
\pgfsetroundjoin%
\definecolor{currentfill}{rgb}{0.000000,0.000000,0.000000}%
\pgfsetfillcolor{currentfill}%
\pgfsetlinewidth{0.803000pt}%
\definecolor{currentstroke}{rgb}{0.000000,0.000000,0.000000}%
\pgfsetstrokecolor{currentstroke}%
\pgfsetdash{}{0pt}%
\pgfsys@defobject{currentmarker}{\pgfqpoint{-0.048611in}{0.000000in}}{\pgfqpoint{0.000000in}{0.000000in}}{%
\pgfpathmoveto{\pgfqpoint{0.000000in}{0.000000in}}%
\pgfpathlineto{\pgfqpoint{-0.048611in}{0.000000in}}%
\pgfusepath{stroke,fill}%
}%
\begin{pgfscope}%
\pgfsys@transformshift{1.016621in}{2.148328in}%
\pgfsys@useobject{currentmarker}{}%
\end{pgfscope}%
\end{pgfscope}%
\begin{pgfscope}%
\definecolor{textcolor}{rgb}{0.000000,0.000000,0.000000}%
\pgfsetstrokecolor{textcolor}%
\pgfsetfillcolor{textcolor}%
\pgftext[x=0.672484in,y=2.100102in,left,base]{\color{textcolor}\rmfamily\fontsize{10.000000}{12.000000}\selectfont \(\displaystyle -40\)}%
\end{pgfscope}%
\begin{pgfscope}%
\pgfsetbuttcap%
\pgfsetroundjoin%
\definecolor{currentfill}{rgb}{0.000000,0.000000,0.000000}%
\pgfsetfillcolor{currentfill}%
\pgfsetlinewidth{0.803000pt}%
\definecolor{currentstroke}{rgb}{0.000000,0.000000,0.000000}%
\pgfsetstrokecolor{currentstroke}%
\pgfsetdash{}{0pt}%
\pgfsys@defobject{currentmarker}{\pgfqpoint{-0.048611in}{0.000000in}}{\pgfqpoint{0.000000in}{0.000000in}}{%
\pgfpathmoveto{\pgfqpoint{0.000000in}{0.000000in}}%
\pgfpathlineto{\pgfqpoint{-0.048611in}{0.000000in}}%
\pgfusepath{stroke,fill}%
}%
\begin{pgfscope}%
\pgfsys@transformshift{1.016621in}{2.636180in}%
\pgfsys@useobject{currentmarker}{}%
\end{pgfscope}%
\end{pgfscope}%
\begin{pgfscope}%
\definecolor{textcolor}{rgb}{0.000000,0.000000,0.000000}%
\pgfsetstrokecolor{textcolor}%
\pgfsetfillcolor{textcolor}%
\pgftext[x=0.672484in,y=2.587954in,left,base]{\color{textcolor}\rmfamily\fontsize{10.000000}{12.000000}\selectfont \(\displaystyle -20\)}%
\end{pgfscope}%
\begin{pgfscope}%
\pgfsetbuttcap%
\pgfsetroundjoin%
\definecolor{currentfill}{rgb}{0.000000,0.000000,0.000000}%
\pgfsetfillcolor{currentfill}%
\pgfsetlinewidth{0.803000pt}%
\definecolor{currentstroke}{rgb}{0.000000,0.000000,0.000000}%
\pgfsetstrokecolor{currentstroke}%
\pgfsetdash{}{0pt}%
\pgfsys@defobject{currentmarker}{\pgfqpoint{-0.048611in}{0.000000in}}{\pgfqpoint{0.000000in}{0.000000in}}{%
\pgfpathmoveto{\pgfqpoint{0.000000in}{0.000000in}}%
\pgfpathlineto{\pgfqpoint{-0.048611in}{0.000000in}}%
\pgfusepath{stroke,fill}%
}%
\begin{pgfscope}%
\pgfsys@transformshift{1.016621in}{3.124031in}%
\pgfsys@useobject{currentmarker}{}%
\end{pgfscope}%
\end{pgfscope}%
\begin{pgfscope}%
\definecolor{textcolor}{rgb}{0.000000,0.000000,0.000000}%
\pgfsetstrokecolor{textcolor}%
\pgfsetfillcolor{textcolor}%
\pgftext[x=0.849954in,y=3.075806in,left,base]{\color{textcolor}\rmfamily\fontsize{10.000000}{12.000000}\selectfont \(\displaystyle 0\)}%
\end{pgfscope}%
\begin{pgfscope}%
\definecolor{textcolor}{rgb}{0.000000,0.000000,0.000000}%
\pgfsetstrokecolor{textcolor}%
\pgfsetfillcolor{textcolor}%
\pgftext[x=0.255817in,y=1.847191in,,bottom]{\color{textcolor}\rmfamily\fontsize{10.000000}{12.000000}\selectfont \(\displaystyle \phi\) \textit{(º)}}%
\end{pgfscope}%
\begin{pgfscope}%
\pgfpathrectangle{\pgfqpoint{1.016621in}{0.499691in}}{\pgfqpoint{3.875000in}{2.695000in}}%
\pgfusepath{clip}%
\pgfsetbuttcap%
\pgfsetroundjoin%
\definecolor{currentfill}{rgb}{0.121569,0.466667,0.705882}%
\pgfsetfillcolor{currentfill}%
\pgfsetlinewidth{1.003750pt}%
\definecolor{currentstroke}{rgb}{0.121569,0.466667,0.705882}%
\pgfsetstrokecolor{currentstroke}%
\pgfsetdash{}{0pt}%
\pgfpathmoveto{\pgfqpoint{3.531500in}{1.533435in}}%
\pgfpathcurveto{\pgfqpoint{3.542550in}{1.533435in}}{\pgfqpoint{3.553149in}{1.537825in}}{\pgfqpoint{3.560963in}{1.545639in}}%
\pgfpathcurveto{\pgfqpoint{3.568776in}{1.553453in}}{\pgfqpoint{3.573166in}{1.564052in}}{\pgfqpoint{3.573166in}{1.575102in}}%
\pgfpathcurveto{\pgfqpoint{3.573166in}{1.586152in}}{\pgfqpoint{3.568776in}{1.596751in}}{\pgfqpoint{3.560963in}{1.604565in}}%
\pgfpathcurveto{\pgfqpoint{3.553149in}{1.612378in}}{\pgfqpoint{3.542550in}{1.616768in}}{\pgfqpoint{3.531500in}{1.616768in}}%
\pgfpathcurveto{\pgfqpoint{3.520450in}{1.616768in}}{\pgfqpoint{3.509851in}{1.612378in}}{\pgfqpoint{3.502037in}{1.604565in}}%
\pgfpathcurveto{\pgfqpoint{3.494223in}{1.596751in}}{\pgfqpoint{3.489833in}{1.586152in}}{\pgfqpoint{3.489833in}{1.575102in}}%
\pgfpathcurveto{\pgfqpoint{3.489833in}{1.564052in}}{\pgfqpoint{3.494223in}{1.553453in}}{\pgfqpoint{3.502037in}{1.545639in}}%
\pgfpathcurveto{\pgfqpoint{3.509851in}{1.537825in}}{\pgfqpoint{3.520450in}{1.533435in}}{\pgfqpoint{3.531500in}{1.533435in}}%
\pgfpathclose%
\pgfusepath{stroke,fill}%
\end{pgfscope}%
\begin{pgfscope}%
\pgfpathrectangle{\pgfqpoint{1.016621in}{0.499691in}}{\pgfqpoint{3.875000in}{2.695000in}}%
\pgfusepath{clip}%
\pgfsetbuttcap%
\pgfsetroundjoin%
\definecolor{currentfill}{rgb}{0.121569,0.466667,0.705882}%
\pgfsetfillcolor{currentfill}%
\pgfsetlinewidth{1.003750pt}%
\definecolor{currentstroke}{rgb}{0.121569,0.466667,0.705882}%
\pgfsetstrokecolor{currentstroke}%
\pgfsetdash{}{0pt}%
\pgfpathmoveto{\pgfqpoint{3.568750in}{1.499285in}}%
\pgfpathcurveto{\pgfqpoint{3.579800in}{1.499285in}}{\pgfqpoint{3.590399in}{1.503676in}}{\pgfqpoint{3.598213in}{1.511489in}}%
\pgfpathcurveto{\pgfqpoint{3.606026in}{1.519303in}}{\pgfqpoint{3.610417in}{1.529902in}}{\pgfqpoint{3.610417in}{1.540952in}}%
\pgfpathcurveto{\pgfqpoint{3.610417in}{1.552002in}}{\pgfqpoint{3.606026in}{1.562601in}}{\pgfqpoint{3.598213in}{1.570415in}}%
\pgfpathcurveto{\pgfqpoint{3.590399in}{1.578228in}}{\pgfqpoint{3.579800in}{1.582619in}}{\pgfqpoint{3.568750in}{1.582619in}}%
\pgfpathcurveto{\pgfqpoint{3.557700in}{1.582619in}}{\pgfqpoint{3.547101in}{1.578228in}}{\pgfqpoint{3.539287in}{1.570415in}}%
\pgfpathcurveto{\pgfqpoint{3.531474in}{1.562601in}}{\pgfqpoint{3.527083in}{1.552002in}}{\pgfqpoint{3.527083in}{1.540952in}}%
\pgfpathcurveto{\pgfqpoint{3.527083in}{1.529902in}}{\pgfqpoint{3.531474in}{1.519303in}}{\pgfqpoint{3.539287in}{1.511489in}}%
\pgfpathcurveto{\pgfqpoint{3.547101in}{1.503676in}}{\pgfqpoint{3.557700in}{1.499285in}}{\pgfqpoint{3.568750in}{1.499285in}}%
\pgfpathclose%
\pgfusepath{stroke,fill}%
\end{pgfscope}%
\begin{pgfscope}%
\pgfpathrectangle{\pgfqpoint{1.016621in}{0.499691in}}{\pgfqpoint{3.875000in}{2.695000in}}%
\pgfusepath{clip}%
\pgfsetbuttcap%
\pgfsetroundjoin%
\definecolor{currentfill}{rgb}{0.121569,0.466667,0.705882}%
\pgfsetfillcolor{currentfill}%
\pgfsetlinewidth{1.003750pt}%
\definecolor{currentstroke}{rgb}{0.121569,0.466667,0.705882}%
\pgfsetstrokecolor{currentstroke}%
\pgfsetdash{}{0pt}%
\pgfpathmoveto{\pgfqpoint{3.606000in}{1.474893in}}%
\pgfpathcurveto{\pgfqpoint{3.617050in}{1.474893in}}{\pgfqpoint{3.627649in}{1.479283in}}{\pgfqpoint{3.635463in}{1.487097in}}%
\pgfpathcurveto{\pgfqpoint{3.643277in}{1.494910in}}{\pgfqpoint{3.647667in}{1.505509in}}{\pgfqpoint{3.647667in}{1.516560in}}%
\pgfpathcurveto{\pgfqpoint{3.647667in}{1.527610in}}{\pgfqpoint{3.643277in}{1.538209in}}{\pgfqpoint{3.635463in}{1.546022in}}%
\pgfpathcurveto{\pgfqpoint{3.627649in}{1.553836in}}{\pgfqpoint{3.617050in}{1.558226in}}{\pgfqpoint{3.606000in}{1.558226in}}%
\pgfpathcurveto{\pgfqpoint{3.594950in}{1.558226in}}{\pgfqpoint{3.584351in}{1.553836in}}{\pgfqpoint{3.576537in}{1.546022in}}%
\pgfpathcurveto{\pgfqpoint{3.568724in}{1.538209in}}{\pgfqpoint{3.564334in}{1.527610in}}{\pgfqpoint{3.564334in}{1.516560in}}%
\pgfpathcurveto{\pgfqpoint{3.564334in}{1.505509in}}{\pgfqpoint{3.568724in}{1.494910in}}{\pgfqpoint{3.576537in}{1.487097in}}%
\pgfpathcurveto{\pgfqpoint{3.584351in}{1.479283in}}{\pgfqpoint{3.594950in}{1.474893in}}{\pgfqpoint{3.606000in}{1.474893in}}%
\pgfpathclose%
\pgfusepath{stroke,fill}%
\end{pgfscope}%
\begin{pgfscope}%
\pgfpathrectangle{\pgfqpoint{1.016621in}{0.499691in}}{\pgfqpoint{3.875000in}{2.695000in}}%
\pgfusepath{clip}%
\pgfsetbuttcap%
\pgfsetroundjoin%
\definecolor{currentfill}{rgb}{0.121569,0.466667,0.705882}%
\pgfsetfillcolor{currentfill}%
\pgfsetlinewidth{1.003750pt}%
\definecolor{currentstroke}{rgb}{0.121569,0.466667,0.705882}%
\pgfsetstrokecolor{currentstroke}%
\pgfsetdash{}{0pt}%
\pgfpathmoveto{\pgfqpoint{3.643250in}{1.596856in}}%
\pgfpathcurveto{\pgfqpoint{3.654301in}{1.596856in}}{\pgfqpoint{3.664900in}{1.601246in}}{\pgfqpoint{3.672713in}{1.609060in}}%
\pgfpathcurveto{\pgfqpoint{3.680527in}{1.616873in}}{\pgfqpoint{3.684917in}{1.627472in}}{\pgfqpoint{3.684917in}{1.638522in}}%
\pgfpathcurveto{\pgfqpoint{3.684917in}{1.649573in}}{\pgfqpoint{3.680527in}{1.660172in}}{\pgfqpoint{3.672713in}{1.667985in}}%
\pgfpathcurveto{\pgfqpoint{3.664900in}{1.675799in}}{\pgfqpoint{3.654301in}{1.680189in}}{\pgfqpoint{3.643250in}{1.680189in}}%
\pgfpathcurveto{\pgfqpoint{3.632200in}{1.680189in}}{\pgfqpoint{3.621601in}{1.675799in}}{\pgfqpoint{3.613788in}{1.667985in}}%
\pgfpathcurveto{\pgfqpoint{3.605974in}{1.660172in}}{\pgfqpoint{3.601584in}{1.649573in}}{\pgfqpoint{3.601584in}{1.638522in}}%
\pgfpathcurveto{\pgfqpoint{3.601584in}{1.627472in}}{\pgfqpoint{3.605974in}{1.616873in}}{\pgfqpoint{3.613788in}{1.609060in}}%
\pgfpathcurveto{\pgfqpoint{3.621601in}{1.601246in}}{\pgfqpoint{3.632200in}{1.596856in}}{\pgfqpoint{3.643250in}{1.596856in}}%
\pgfpathclose%
\pgfusepath{stroke,fill}%
\end{pgfscope}%
\begin{pgfscope}%
\pgfpathrectangle{\pgfqpoint{1.016621in}{0.499691in}}{\pgfqpoint{3.875000in}{2.695000in}}%
\pgfusepath{clip}%
\pgfsetbuttcap%
\pgfsetroundjoin%
\definecolor{currentfill}{rgb}{0.121569,0.466667,0.705882}%
\pgfsetfillcolor{currentfill}%
\pgfsetlinewidth{1.003750pt}%
\definecolor{currentstroke}{rgb}{0.121569,0.466667,0.705882}%
\pgfsetstrokecolor{currentstroke}%
\pgfsetdash{}{0pt}%
\pgfpathmoveto{\pgfqpoint{3.680501in}{1.677351in}}%
\pgfpathcurveto{\pgfqpoint{3.691551in}{1.677351in}}{\pgfqpoint{3.702150in}{1.681742in}}{\pgfqpoint{3.709964in}{1.689555in}}%
\pgfpathcurveto{\pgfqpoint{3.717777in}{1.697369in}}{\pgfqpoint{3.722167in}{1.707968in}}{\pgfqpoint{3.722167in}{1.719018in}}%
\pgfpathcurveto{\pgfqpoint{3.722167in}{1.730068in}}{\pgfqpoint{3.717777in}{1.740667in}}{\pgfqpoint{3.709964in}{1.748481in}}%
\pgfpathcurveto{\pgfqpoint{3.702150in}{1.756294in}}{\pgfqpoint{3.691551in}{1.760685in}}{\pgfqpoint{3.680501in}{1.760685in}}%
\pgfpathcurveto{\pgfqpoint{3.669451in}{1.760685in}}{\pgfqpoint{3.658852in}{1.756294in}}{\pgfqpoint{3.651038in}{1.748481in}}%
\pgfpathcurveto{\pgfqpoint{3.643224in}{1.740667in}}{\pgfqpoint{3.638834in}{1.730068in}}{\pgfqpoint{3.638834in}{1.719018in}}%
\pgfpathcurveto{\pgfqpoint{3.638834in}{1.707968in}}{\pgfqpoint{3.643224in}{1.697369in}}{\pgfqpoint{3.651038in}{1.689555in}}%
\pgfpathcurveto{\pgfqpoint{3.658852in}{1.681742in}}{\pgfqpoint{3.669451in}{1.677351in}}{\pgfqpoint{3.680501in}{1.677351in}}%
\pgfpathclose%
\pgfusepath{stroke,fill}%
\end{pgfscope}%
\begin{pgfscope}%
\pgfpathrectangle{\pgfqpoint{1.016621in}{0.499691in}}{\pgfqpoint{3.875000in}{2.695000in}}%
\pgfusepath{clip}%
\pgfsetbuttcap%
\pgfsetroundjoin%
\definecolor{currentfill}{rgb}{0.121569,0.466667,0.705882}%
\pgfsetfillcolor{currentfill}%
\pgfsetlinewidth{1.003750pt}%
\definecolor{currentstroke}{rgb}{0.121569,0.466667,0.705882}%
\pgfsetstrokecolor{currentstroke}%
\pgfsetdash{}{0pt}%
\pgfpathmoveto{\pgfqpoint{3.717751in}{1.816389in}}%
\pgfpathcurveto{\pgfqpoint{3.728801in}{1.816389in}}{\pgfqpoint{3.739400in}{1.820779in}}{\pgfqpoint{3.747214in}{1.828593in}}%
\pgfpathcurveto{\pgfqpoint{3.755027in}{1.836407in}}{\pgfqpoint{3.759418in}{1.847006in}}{\pgfqpoint{3.759418in}{1.858056in}}%
\pgfpathcurveto{\pgfqpoint{3.759418in}{1.869106in}}{\pgfqpoint{3.755027in}{1.879705in}}{\pgfqpoint{3.747214in}{1.887519in}}%
\pgfpathcurveto{\pgfqpoint{3.739400in}{1.895332in}}{\pgfqpoint{3.728801in}{1.899722in}}{\pgfqpoint{3.717751in}{1.899722in}}%
\pgfpathcurveto{\pgfqpoint{3.706701in}{1.899722in}}{\pgfqpoint{3.696102in}{1.895332in}}{\pgfqpoint{3.688288in}{1.887519in}}%
\pgfpathcurveto{\pgfqpoint{3.680475in}{1.879705in}}{\pgfqpoint{3.676084in}{1.869106in}}{\pgfqpoint{3.676084in}{1.858056in}}%
\pgfpathcurveto{\pgfqpoint{3.676084in}{1.847006in}}{\pgfqpoint{3.680475in}{1.836407in}}{\pgfqpoint{3.688288in}{1.828593in}}%
\pgfpathcurveto{\pgfqpoint{3.696102in}{1.820779in}}{\pgfqpoint{3.706701in}{1.816389in}}{\pgfqpoint{3.717751in}{1.816389in}}%
\pgfpathclose%
\pgfusepath{stroke,fill}%
\end{pgfscope}%
\begin{pgfscope}%
\pgfpathrectangle{\pgfqpoint{1.016621in}{0.499691in}}{\pgfqpoint{3.875000in}{2.695000in}}%
\pgfusepath{clip}%
\pgfsetbuttcap%
\pgfsetroundjoin%
\definecolor{currentfill}{rgb}{0.121569,0.466667,0.705882}%
\pgfsetfillcolor{currentfill}%
\pgfsetlinewidth{1.003750pt}%
\definecolor{currentstroke}{rgb}{0.121569,0.466667,0.705882}%
\pgfsetstrokecolor{currentstroke}%
\pgfsetdash{}{0pt}%
\pgfpathmoveto{\pgfqpoint{3.742584in}{1.784679in}}%
\pgfpathcurveto{\pgfqpoint{3.753635in}{1.784679in}}{\pgfqpoint{3.764234in}{1.789069in}}{\pgfqpoint{3.772047in}{1.796883in}}%
\pgfpathcurveto{\pgfqpoint{3.779861in}{1.804696in}}{\pgfqpoint{3.784251in}{1.815295in}}{\pgfqpoint{3.784251in}{1.826345in}}%
\pgfpathcurveto{\pgfqpoint{3.784251in}{1.837396in}}{\pgfqpoint{3.779861in}{1.847995in}}{\pgfqpoint{3.772047in}{1.855808in}}%
\pgfpathcurveto{\pgfqpoint{3.764234in}{1.863622in}}{\pgfqpoint{3.753635in}{1.868012in}}{\pgfqpoint{3.742584in}{1.868012in}}%
\pgfpathcurveto{\pgfqpoint{3.731534in}{1.868012in}}{\pgfqpoint{3.720935in}{1.863622in}}{\pgfqpoint{3.713122in}{1.855808in}}%
\pgfpathcurveto{\pgfqpoint{3.705308in}{1.847995in}}{\pgfqpoint{3.700918in}{1.837396in}}{\pgfqpoint{3.700918in}{1.826345in}}%
\pgfpathcurveto{\pgfqpoint{3.700918in}{1.815295in}}{\pgfqpoint{3.705308in}{1.804696in}}{\pgfqpoint{3.713122in}{1.796883in}}%
\pgfpathcurveto{\pgfqpoint{3.720935in}{1.789069in}}{\pgfqpoint{3.731534in}{1.784679in}}{\pgfqpoint{3.742584in}{1.784679in}}%
\pgfpathclose%
\pgfusepath{stroke,fill}%
\end{pgfscope}%
\begin{pgfscope}%
\pgfpathrectangle{\pgfqpoint{1.016621in}{0.499691in}}{\pgfqpoint{3.875000in}{2.695000in}}%
\pgfusepath{clip}%
\pgfsetbuttcap%
\pgfsetroundjoin%
\definecolor{currentfill}{rgb}{0.121569,0.466667,0.705882}%
\pgfsetfillcolor{currentfill}%
\pgfsetlinewidth{1.003750pt}%
\definecolor{currentstroke}{rgb}{0.121569,0.466667,0.705882}%
\pgfsetstrokecolor{currentstroke}%
\pgfsetdash{}{0pt}%
\pgfpathmoveto{\pgfqpoint{3.767418in}{1.796875in}}%
\pgfpathcurveto{\pgfqpoint{3.778468in}{1.796875in}}{\pgfqpoint{3.789067in}{1.801265in}}{\pgfqpoint{3.796881in}{1.809079in}}%
\pgfpathcurveto{\pgfqpoint{3.804694in}{1.816893in}}{\pgfqpoint{3.809085in}{1.827492in}}{\pgfqpoint{3.809085in}{1.838542in}}%
\pgfpathcurveto{\pgfqpoint{3.809085in}{1.849592in}}{\pgfqpoint{3.804694in}{1.860191in}}{\pgfqpoint{3.796881in}{1.868005in}}%
\pgfpathcurveto{\pgfqpoint{3.789067in}{1.875818in}}{\pgfqpoint{3.778468in}{1.880208in}}{\pgfqpoint{3.767418in}{1.880208in}}%
\pgfpathcurveto{\pgfqpoint{3.756368in}{1.880208in}}{\pgfqpoint{3.745769in}{1.875818in}}{\pgfqpoint{3.737955in}{1.868005in}}%
\pgfpathcurveto{\pgfqpoint{3.730142in}{1.860191in}}{\pgfqpoint{3.725751in}{1.849592in}}{\pgfqpoint{3.725751in}{1.838542in}}%
\pgfpathcurveto{\pgfqpoint{3.725751in}{1.827492in}}{\pgfqpoint{3.730142in}{1.816893in}}{\pgfqpoint{3.737955in}{1.809079in}}%
\pgfpathcurveto{\pgfqpoint{3.745769in}{1.801265in}}{\pgfqpoint{3.756368in}{1.796875in}}{\pgfqpoint{3.767418in}{1.796875in}}%
\pgfpathclose%
\pgfusepath{stroke,fill}%
\end{pgfscope}%
\begin{pgfscope}%
\pgfpathrectangle{\pgfqpoint{1.016621in}{0.499691in}}{\pgfqpoint{3.875000in}{2.695000in}}%
\pgfusepath{clip}%
\pgfsetbuttcap%
\pgfsetroundjoin%
\definecolor{currentfill}{rgb}{0.121569,0.466667,0.705882}%
\pgfsetfillcolor{currentfill}%
\pgfsetlinewidth{1.003750pt}%
\definecolor{currentstroke}{rgb}{0.121569,0.466667,0.705882}%
\pgfsetstrokecolor{currentstroke}%
\pgfsetdash{}{0pt}%
\pgfpathmoveto{\pgfqpoint{3.792251in}{1.818828in}}%
\pgfpathcurveto{\pgfqpoint{3.803302in}{1.818828in}}{\pgfqpoint{3.813901in}{1.823219in}}{\pgfqpoint{3.821714in}{1.831032in}}%
\pgfpathcurveto{\pgfqpoint{3.829528in}{1.838846in}}{\pgfqpoint{3.833918in}{1.849445in}}{\pgfqpoint{3.833918in}{1.860495in}}%
\pgfpathcurveto{\pgfqpoint{3.833918in}{1.871545in}}{\pgfqpoint{3.829528in}{1.882144in}}{\pgfqpoint{3.821714in}{1.889958in}}%
\pgfpathcurveto{\pgfqpoint{3.813901in}{1.897771in}}{\pgfqpoint{3.803302in}{1.902162in}}{\pgfqpoint{3.792251in}{1.902162in}}%
\pgfpathcurveto{\pgfqpoint{3.781201in}{1.902162in}}{\pgfqpoint{3.770602in}{1.897771in}}{\pgfqpoint{3.762789in}{1.889958in}}%
\pgfpathcurveto{\pgfqpoint{3.754975in}{1.882144in}}{\pgfqpoint{3.750585in}{1.871545in}}{\pgfqpoint{3.750585in}{1.860495in}}%
\pgfpathcurveto{\pgfqpoint{3.750585in}{1.849445in}}{\pgfqpoint{3.754975in}{1.838846in}}{\pgfqpoint{3.762789in}{1.831032in}}%
\pgfpathcurveto{\pgfqpoint{3.770602in}{1.823219in}}{\pgfqpoint{3.781201in}{1.818828in}}{\pgfqpoint{3.792251in}{1.818828in}}%
\pgfpathclose%
\pgfusepath{stroke,fill}%
\end{pgfscope}%
\begin{pgfscope}%
\pgfpathrectangle{\pgfqpoint{1.016621in}{0.499691in}}{\pgfqpoint{3.875000in}{2.695000in}}%
\pgfusepath{clip}%
\pgfsetbuttcap%
\pgfsetroundjoin%
\definecolor{currentfill}{rgb}{0.121569,0.466667,0.705882}%
\pgfsetfillcolor{currentfill}%
\pgfsetlinewidth{1.003750pt}%
\definecolor{currentstroke}{rgb}{0.121569,0.466667,0.705882}%
\pgfsetstrokecolor{currentstroke}%
\pgfsetdash{}{0pt}%
\pgfpathmoveto{\pgfqpoint{3.817085in}{1.894445in}}%
\pgfpathcurveto{\pgfqpoint{3.828135in}{1.894445in}}{\pgfqpoint{3.838734in}{1.898836in}}{\pgfqpoint{3.846548in}{1.906649in}}%
\pgfpathcurveto{\pgfqpoint{3.854361in}{1.914463in}}{\pgfqpoint{3.858752in}{1.925062in}}{\pgfqpoint{3.858752in}{1.936112in}}%
\pgfpathcurveto{\pgfqpoint{3.858752in}{1.947162in}}{\pgfqpoint{3.854361in}{1.957761in}}{\pgfqpoint{3.846548in}{1.965575in}}%
\pgfpathcurveto{\pgfqpoint{3.838734in}{1.973389in}}{\pgfqpoint{3.828135in}{1.977779in}}{\pgfqpoint{3.817085in}{1.977779in}}%
\pgfpathcurveto{\pgfqpoint{3.806035in}{1.977779in}}{\pgfqpoint{3.795436in}{1.973389in}}{\pgfqpoint{3.787622in}{1.965575in}}%
\pgfpathcurveto{\pgfqpoint{3.779809in}{1.957761in}}{\pgfqpoint{3.775418in}{1.947162in}}{\pgfqpoint{3.775418in}{1.936112in}}%
\pgfpathcurveto{\pgfqpoint{3.775418in}{1.925062in}}{\pgfqpoint{3.779809in}{1.914463in}}{\pgfqpoint{3.787622in}{1.906649in}}%
\pgfpathcurveto{\pgfqpoint{3.795436in}{1.898836in}}{\pgfqpoint{3.806035in}{1.894445in}}{\pgfqpoint{3.817085in}{1.894445in}}%
\pgfpathclose%
\pgfusepath{stroke,fill}%
\end{pgfscope}%
\begin{pgfscope}%
\pgfpathrectangle{\pgfqpoint{1.016621in}{0.499691in}}{\pgfqpoint{3.875000in}{2.695000in}}%
\pgfusepath{clip}%
\pgfsetbuttcap%
\pgfsetroundjoin%
\definecolor{currentfill}{rgb}{0.121569,0.466667,0.705882}%
\pgfsetfillcolor{currentfill}%
\pgfsetlinewidth{1.003750pt}%
\definecolor{currentstroke}{rgb}{0.121569,0.466667,0.705882}%
\pgfsetstrokecolor{currentstroke}%
\pgfsetdash{}{0pt}%
\pgfpathmoveto{\pgfqpoint{3.841918in}{1.935913in}}%
\pgfpathcurveto{\pgfqpoint{3.852969in}{1.935913in}}{\pgfqpoint{3.863568in}{1.940303in}}{\pgfqpoint{3.871381in}{1.948117in}}%
\pgfpathcurveto{\pgfqpoint{3.879195in}{1.955930in}}{\pgfqpoint{3.883585in}{1.966529in}}{\pgfqpoint{3.883585in}{1.977580in}}%
\pgfpathcurveto{\pgfqpoint{3.883585in}{1.988630in}}{\pgfqpoint{3.879195in}{1.999229in}}{\pgfqpoint{3.871381in}{2.007042in}}%
\pgfpathcurveto{\pgfqpoint{3.863568in}{2.014856in}}{\pgfqpoint{3.852969in}{2.019246in}}{\pgfqpoint{3.841918in}{2.019246in}}%
\pgfpathcurveto{\pgfqpoint{3.830868in}{2.019246in}}{\pgfqpoint{3.820269in}{2.014856in}}{\pgfqpoint{3.812456in}{2.007042in}}%
\pgfpathcurveto{\pgfqpoint{3.804642in}{1.999229in}}{\pgfqpoint{3.800252in}{1.988630in}}{\pgfqpoint{3.800252in}{1.977580in}}%
\pgfpathcurveto{\pgfqpoint{3.800252in}{1.966529in}}{\pgfqpoint{3.804642in}{1.955930in}}{\pgfqpoint{3.812456in}{1.948117in}}%
\pgfpathcurveto{\pgfqpoint{3.820269in}{1.940303in}}{\pgfqpoint{3.830868in}{1.935913in}}{\pgfqpoint{3.841918in}{1.935913in}}%
\pgfpathclose%
\pgfusepath{stroke,fill}%
\end{pgfscope}%
\begin{pgfscope}%
\pgfpathrectangle{\pgfqpoint{1.016621in}{0.499691in}}{\pgfqpoint{3.875000in}{2.695000in}}%
\pgfusepath{clip}%
\pgfsetbuttcap%
\pgfsetroundjoin%
\definecolor{currentfill}{rgb}{0.121569,0.466667,0.705882}%
\pgfsetfillcolor{currentfill}%
\pgfsetlinewidth{1.003750pt}%
\definecolor{currentstroke}{rgb}{0.121569,0.466667,0.705882}%
\pgfsetstrokecolor{currentstroke}%
\pgfsetdash{}{0pt}%
\pgfpathmoveto{\pgfqpoint{3.866752in}{1.984698in}}%
\pgfpathcurveto{\pgfqpoint{3.877802in}{1.984698in}}{\pgfqpoint{3.888401in}{1.989088in}}{\pgfqpoint{3.896215in}{1.996902in}}%
\pgfpathcurveto{\pgfqpoint{3.904028in}{2.004716in}}{\pgfqpoint{3.908419in}{2.015315in}}{\pgfqpoint{3.908419in}{2.026365in}}%
\pgfpathcurveto{\pgfqpoint{3.908419in}{2.037415in}}{\pgfqpoint{3.904028in}{2.048014in}}{\pgfqpoint{3.896215in}{2.055827in}}%
\pgfpathcurveto{\pgfqpoint{3.888401in}{2.063641in}}{\pgfqpoint{3.877802in}{2.068031in}}{\pgfqpoint{3.866752in}{2.068031in}}%
\pgfpathcurveto{\pgfqpoint{3.855702in}{2.068031in}}{\pgfqpoint{3.845103in}{2.063641in}}{\pgfqpoint{3.837289in}{2.055827in}}%
\pgfpathcurveto{\pgfqpoint{3.829476in}{2.048014in}}{\pgfqpoint{3.825085in}{2.037415in}}{\pgfqpoint{3.825085in}{2.026365in}}%
\pgfpathcurveto{\pgfqpoint{3.825085in}{2.015315in}}{\pgfqpoint{3.829476in}{2.004716in}}{\pgfqpoint{3.837289in}{1.996902in}}%
\pgfpathcurveto{\pgfqpoint{3.845103in}{1.989088in}}{\pgfqpoint{3.855702in}{1.984698in}}{\pgfqpoint{3.866752in}{1.984698in}}%
\pgfpathclose%
\pgfusepath{stroke,fill}%
\end{pgfscope}%
\begin{pgfscope}%
\pgfpathrectangle{\pgfqpoint{1.016621in}{0.499691in}}{\pgfqpoint{3.875000in}{2.695000in}}%
\pgfusepath{clip}%
\pgfsetbuttcap%
\pgfsetroundjoin%
\definecolor{currentfill}{rgb}{0.121569,0.466667,0.705882}%
\pgfsetfillcolor{currentfill}%
\pgfsetlinewidth{1.003750pt}%
\definecolor{currentstroke}{rgb}{0.121569,0.466667,0.705882}%
\pgfsetstrokecolor{currentstroke}%
\pgfsetdash{}{0pt}%
\pgfpathmoveto{\pgfqpoint{3.891585in}{1.989577in}}%
\pgfpathcurveto{\pgfqpoint{3.902636in}{1.989577in}}{\pgfqpoint{3.913235in}{1.993967in}}{\pgfqpoint{3.921048in}{2.001780in}}%
\pgfpathcurveto{\pgfqpoint{3.928862in}{2.009594in}}{\pgfqpoint{3.933252in}{2.020193in}}{\pgfqpoint{3.933252in}{2.031243in}}%
\pgfpathcurveto{\pgfqpoint{3.933252in}{2.042293in}}{\pgfqpoint{3.928862in}{2.052892in}}{\pgfqpoint{3.921048in}{2.060706in}}%
\pgfpathcurveto{\pgfqpoint{3.913235in}{2.068520in}}{\pgfqpoint{3.902636in}{2.072910in}}{\pgfqpoint{3.891585in}{2.072910in}}%
\pgfpathcurveto{\pgfqpoint{3.880535in}{2.072910in}}{\pgfqpoint{3.869936in}{2.068520in}}{\pgfqpoint{3.862123in}{2.060706in}}%
\pgfpathcurveto{\pgfqpoint{3.854309in}{2.052892in}}{\pgfqpoint{3.849919in}{2.042293in}}{\pgfqpoint{3.849919in}{2.031243in}}%
\pgfpathcurveto{\pgfqpoint{3.849919in}{2.020193in}}{\pgfqpoint{3.854309in}{2.009594in}}{\pgfqpoint{3.862123in}{2.001780in}}%
\pgfpathcurveto{\pgfqpoint{3.869936in}{1.993967in}}{\pgfqpoint{3.880535in}{1.989577in}}{\pgfqpoint{3.891585in}{1.989577in}}%
\pgfpathclose%
\pgfusepath{stroke,fill}%
\end{pgfscope}%
\begin{pgfscope}%
\pgfpathrectangle{\pgfqpoint{1.016621in}{0.499691in}}{\pgfqpoint{3.875000in}{2.695000in}}%
\pgfusepath{clip}%
\pgfsetbuttcap%
\pgfsetroundjoin%
\definecolor{currentfill}{rgb}{0.121569,0.466667,0.705882}%
\pgfsetfillcolor{currentfill}%
\pgfsetlinewidth{1.003750pt}%
\definecolor{currentstroke}{rgb}{0.121569,0.466667,0.705882}%
\pgfsetstrokecolor{currentstroke}%
\pgfsetdash{}{0pt}%
\pgfpathmoveto{\pgfqpoint{3.916419in}{2.013969in}}%
\pgfpathcurveto{\pgfqpoint{3.927469in}{2.013969in}}{\pgfqpoint{3.938068in}{2.018359in}}{\pgfqpoint{3.945882in}{2.026173in}}%
\pgfpathcurveto{\pgfqpoint{3.953695in}{2.033987in}}{\pgfqpoint{3.958086in}{2.044586in}}{\pgfqpoint{3.958086in}{2.055636in}}%
\pgfpathcurveto{\pgfqpoint{3.958086in}{2.066686in}}{\pgfqpoint{3.953695in}{2.077285in}}{\pgfqpoint{3.945882in}{2.085099in}}%
\pgfpathcurveto{\pgfqpoint{3.938068in}{2.092912in}}{\pgfqpoint{3.927469in}{2.097302in}}{\pgfqpoint{3.916419in}{2.097302in}}%
\pgfpathcurveto{\pgfqpoint{3.905369in}{2.097302in}}{\pgfqpoint{3.894770in}{2.092912in}}{\pgfqpoint{3.886956in}{2.085099in}}%
\pgfpathcurveto{\pgfqpoint{3.879143in}{2.077285in}}{\pgfqpoint{3.874752in}{2.066686in}}{\pgfqpoint{3.874752in}{2.055636in}}%
\pgfpathcurveto{\pgfqpoint{3.874752in}{2.044586in}}{\pgfqpoint{3.879143in}{2.033987in}}{\pgfqpoint{3.886956in}{2.026173in}}%
\pgfpathcurveto{\pgfqpoint{3.894770in}{2.018359in}}{\pgfqpoint{3.905369in}{2.013969in}}{\pgfqpoint{3.916419in}{2.013969in}}%
\pgfpathclose%
\pgfusepath{stroke,fill}%
\end{pgfscope}%
\begin{pgfscope}%
\pgfpathrectangle{\pgfqpoint{1.016621in}{0.499691in}}{\pgfqpoint{3.875000in}{2.695000in}}%
\pgfusepath{clip}%
\pgfsetbuttcap%
\pgfsetroundjoin%
\definecolor{currentfill}{rgb}{0.121569,0.466667,0.705882}%
\pgfsetfillcolor{currentfill}%
\pgfsetlinewidth{1.003750pt}%
\definecolor{currentstroke}{rgb}{0.121569,0.466667,0.705882}%
\pgfsetstrokecolor{currentstroke}%
\pgfsetdash{}{0pt}%
\pgfpathmoveto{\pgfqpoint{3.928836in}{2.028605in}}%
\pgfpathcurveto{\pgfqpoint{3.939886in}{2.028605in}}{\pgfqpoint{3.950485in}{2.032995in}}{\pgfqpoint{3.958299in}{2.040809in}}%
\pgfpathcurveto{\pgfqpoint{3.966112in}{2.048622in}}{\pgfqpoint{3.970502in}{2.059221in}}{\pgfqpoint{3.970502in}{2.070271in}}%
\pgfpathcurveto{\pgfqpoint{3.970502in}{2.081322in}}{\pgfqpoint{3.966112in}{2.091921in}}{\pgfqpoint{3.958299in}{2.099734in}}%
\pgfpathcurveto{\pgfqpoint{3.950485in}{2.107548in}}{\pgfqpoint{3.939886in}{2.111938in}}{\pgfqpoint{3.928836in}{2.111938in}}%
\pgfpathcurveto{\pgfqpoint{3.917786in}{2.111938in}}{\pgfqpoint{3.907187in}{2.107548in}}{\pgfqpoint{3.899373in}{2.099734in}}%
\pgfpathcurveto{\pgfqpoint{3.891559in}{2.091921in}}{\pgfqpoint{3.887169in}{2.081322in}}{\pgfqpoint{3.887169in}{2.070271in}}%
\pgfpathcurveto{\pgfqpoint{3.887169in}{2.059221in}}{\pgfqpoint{3.891559in}{2.048622in}}{\pgfqpoint{3.899373in}{2.040809in}}%
\pgfpathcurveto{\pgfqpoint{3.907187in}{2.032995in}}{\pgfqpoint{3.917786in}{2.028605in}}{\pgfqpoint{3.928836in}{2.028605in}}%
\pgfpathclose%
\pgfusepath{stroke,fill}%
\end{pgfscope}%
\begin{pgfscope}%
\pgfpathrectangle{\pgfqpoint{1.016621in}{0.499691in}}{\pgfqpoint{3.875000in}{2.695000in}}%
\pgfusepath{clip}%
\pgfsetbuttcap%
\pgfsetroundjoin%
\definecolor{currentfill}{rgb}{0.121569,0.466667,0.705882}%
\pgfsetfillcolor{currentfill}%
\pgfsetlinewidth{1.003750pt}%
\definecolor{currentstroke}{rgb}{0.121569,0.466667,0.705882}%
\pgfsetstrokecolor{currentstroke}%
\pgfsetdash{}{0pt}%
\pgfpathmoveto{\pgfqpoint{3.953669in}{2.033483in}}%
\pgfpathcurveto{\pgfqpoint{3.964719in}{2.033483in}}{\pgfqpoint{3.975318in}{2.037873in}}{\pgfqpoint{3.983132in}{2.045687in}}%
\pgfpathcurveto{\pgfqpoint{3.990946in}{2.053501in}}{\pgfqpoint{3.995336in}{2.064100in}}{\pgfqpoint{3.995336in}{2.075150in}}%
\pgfpathcurveto{\pgfqpoint{3.995336in}{2.086200in}}{\pgfqpoint{3.990946in}{2.096799in}}{\pgfqpoint{3.983132in}{2.104613in}}%
\pgfpathcurveto{\pgfqpoint{3.975318in}{2.112426in}}{\pgfqpoint{3.964719in}{2.116817in}}{\pgfqpoint{3.953669in}{2.116817in}}%
\pgfpathcurveto{\pgfqpoint{3.942619in}{2.116817in}}{\pgfqpoint{3.932020in}{2.112426in}}{\pgfqpoint{3.924206in}{2.104613in}}%
\pgfpathcurveto{\pgfqpoint{3.916393in}{2.096799in}}{\pgfqpoint{3.912003in}{2.086200in}}{\pgfqpoint{3.912003in}{2.075150in}}%
\pgfpathcurveto{\pgfqpoint{3.912003in}{2.064100in}}{\pgfqpoint{3.916393in}{2.053501in}}{\pgfqpoint{3.924206in}{2.045687in}}%
\pgfpathcurveto{\pgfqpoint{3.932020in}{2.037873in}}{\pgfqpoint{3.942619in}{2.033483in}}{\pgfqpoint{3.953669in}{2.033483in}}%
\pgfpathclose%
\pgfusepath{stroke,fill}%
\end{pgfscope}%
\begin{pgfscope}%
\pgfpathrectangle{\pgfqpoint{1.016621in}{0.499691in}}{\pgfqpoint{3.875000in}{2.695000in}}%
\pgfusepath{clip}%
\pgfsetbuttcap%
\pgfsetroundjoin%
\definecolor{currentfill}{rgb}{0.121569,0.466667,0.705882}%
\pgfsetfillcolor{currentfill}%
\pgfsetlinewidth{1.003750pt}%
\definecolor{currentstroke}{rgb}{0.121569,0.466667,0.705882}%
\pgfsetstrokecolor{currentstroke}%
\pgfsetdash{}{0pt}%
\pgfpathmoveto{\pgfqpoint{3.978503in}{2.084708in}}%
\pgfpathcurveto{\pgfqpoint{3.989553in}{2.084708in}}{\pgfqpoint{4.000152in}{2.089098in}}{\pgfqpoint{4.007966in}{2.096912in}}%
\pgfpathcurveto{\pgfqpoint{4.015779in}{2.104725in}}{\pgfqpoint{4.020169in}{2.115324in}}{\pgfqpoint{4.020169in}{2.126374in}}%
\pgfpathcurveto{\pgfqpoint{4.020169in}{2.137424in}}{\pgfqpoint{4.015779in}{2.148024in}}{\pgfqpoint{4.007966in}{2.155837in}}%
\pgfpathcurveto{\pgfqpoint{4.000152in}{2.163651in}}{\pgfqpoint{3.989553in}{2.168041in}}{\pgfqpoint{3.978503in}{2.168041in}}%
\pgfpathcurveto{\pgfqpoint{3.967453in}{2.168041in}}{\pgfqpoint{3.956854in}{2.163651in}}{\pgfqpoint{3.949040in}{2.155837in}}%
\pgfpathcurveto{\pgfqpoint{3.941226in}{2.148024in}}{\pgfqpoint{3.936836in}{2.137424in}}{\pgfqpoint{3.936836in}{2.126374in}}%
\pgfpathcurveto{\pgfqpoint{3.936836in}{2.115324in}}{\pgfqpoint{3.941226in}{2.104725in}}{\pgfqpoint{3.949040in}{2.096912in}}%
\pgfpathcurveto{\pgfqpoint{3.956854in}{2.089098in}}{\pgfqpoint{3.967453in}{2.084708in}}{\pgfqpoint{3.978503in}{2.084708in}}%
\pgfpathclose%
\pgfusepath{stroke,fill}%
\end{pgfscope}%
\begin{pgfscope}%
\pgfpathrectangle{\pgfqpoint{1.016621in}{0.499691in}}{\pgfqpoint{3.875000in}{2.695000in}}%
\pgfusepath{clip}%
\pgfsetbuttcap%
\pgfsetroundjoin%
\definecolor{currentfill}{rgb}{0.121569,0.466667,0.705882}%
\pgfsetfillcolor{currentfill}%
\pgfsetlinewidth{1.003750pt}%
\definecolor{currentstroke}{rgb}{0.121569,0.466667,0.705882}%
\pgfsetstrokecolor{currentstroke}%
\pgfsetdash{}{0pt}%
\pgfpathmoveto{\pgfqpoint{3.990920in}{2.096904in}}%
\pgfpathcurveto{\pgfqpoint{4.001970in}{2.096904in}}{\pgfqpoint{4.012569in}{2.101294in}}{\pgfqpoint{4.020382in}{2.109108in}}%
\pgfpathcurveto{\pgfqpoint{4.028196in}{2.116921in}}{\pgfqpoint{4.032586in}{2.127521in}}{\pgfqpoint{4.032586in}{2.138571in}}%
\pgfpathcurveto{\pgfqpoint{4.032586in}{2.149621in}}{\pgfqpoint{4.028196in}{2.160220in}}{\pgfqpoint{4.020382in}{2.168033in}}%
\pgfpathcurveto{\pgfqpoint{4.012569in}{2.175847in}}{\pgfqpoint{4.001970in}{2.180237in}}{\pgfqpoint{3.990920in}{2.180237in}}%
\pgfpathcurveto{\pgfqpoint{3.979869in}{2.180237in}}{\pgfqpoint{3.969270in}{2.175847in}}{\pgfqpoint{3.961457in}{2.168033in}}%
\pgfpathcurveto{\pgfqpoint{3.953643in}{2.160220in}}{\pgfqpoint{3.949253in}{2.149621in}}{\pgfqpoint{3.949253in}{2.138571in}}%
\pgfpathcurveto{\pgfqpoint{3.949253in}{2.127521in}}{\pgfqpoint{3.953643in}{2.116921in}}{\pgfqpoint{3.961457in}{2.109108in}}%
\pgfpathcurveto{\pgfqpoint{3.969270in}{2.101294in}}{\pgfqpoint{3.979869in}{2.096904in}}{\pgfqpoint{3.990920in}{2.096904in}}%
\pgfpathclose%
\pgfusepath{stroke,fill}%
\end{pgfscope}%
\begin{pgfscope}%
\pgfpathrectangle{\pgfqpoint{1.016621in}{0.499691in}}{\pgfqpoint{3.875000in}{2.695000in}}%
\pgfusepath{clip}%
\pgfsetbuttcap%
\pgfsetroundjoin%
\definecolor{currentfill}{rgb}{0.121569,0.466667,0.705882}%
\pgfsetfillcolor{currentfill}%
\pgfsetlinewidth{1.003750pt}%
\definecolor{currentstroke}{rgb}{0.121569,0.466667,0.705882}%
\pgfsetstrokecolor{currentstroke}%
\pgfsetdash{}{0pt}%
\pgfpathmoveto{\pgfqpoint{4.003336in}{2.128614in}}%
\pgfpathcurveto{\pgfqpoint{4.014386in}{2.128614in}}{\pgfqpoint{4.024985in}{2.133005in}}{\pgfqpoint{4.032799in}{2.140818in}}%
\pgfpathcurveto{\pgfqpoint{4.040613in}{2.148632in}}{\pgfqpoint{4.045003in}{2.159231in}}{\pgfqpoint{4.045003in}{2.170281in}}%
\pgfpathcurveto{\pgfqpoint{4.045003in}{2.181331in}}{\pgfqpoint{4.040613in}{2.191930in}}{\pgfqpoint{4.032799in}{2.199744in}}%
\pgfpathcurveto{\pgfqpoint{4.024985in}{2.207557in}}{\pgfqpoint{4.014386in}{2.211948in}}{\pgfqpoint{4.003336in}{2.211948in}}%
\pgfpathcurveto{\pgfqpoint{3.992286in}{2.211948in}}{\pgfqpoint{3.981687in}{2.207557in}}{\pgfqpoint{3.973873in}{2.199744in}}%
\pgfpathcurveto{\pgfqpoint{3.966060in}{2.191930in}}{\pgfqpoint{3.961670in}{2.181331in}}{\pgfqpoint{3.961670in}{2.170281in}}%
\pgfpathcurveto{\pgfqpoint{3.961670in}{2.159231in}}{\pgfqpoint{3.966060in}{2.148632in}}{\pgfqpoint{3.973873in}{2.140818in}}%
\pgfpathcurveto{\pgfqpoint{3.981687in}{2.133005in}}{\pgfqpoint{3.992286in}{2.128614in}}{\pgfqpoint{4.003336in}{2.128614in}}%
\pgfpathclose%
\pgfusepath{stroke,fill}%
\end{pgfscope}%
\begin{pgfscope}%
\pgfpathrectangle{\pgfqpoint{1.016621in}{0.499691in}}{\pgfqpoint{3.875000in}{2.695000in}}%
\pgfusepath{clip}%
\pgfsetbuttcap%
\pgfsetroundjoin%
\definecolor{currentfill}{rgb}{0.121569,0.466667,0.705882}%
\pgfsetfillcolor{currentfill}%
\pgfsetlinewidth{1.003750pt}%
\definecolor{currentstroke}{rgb}{0.121569,0.466667,0.705882}%
\pgfsetstrokecolor{currentstroke}%
\pgfsetdash{}{0pt}%
\pgfpathmoveto{\pgfqpoint{4.028170in}{2.148128in}}%
\pgfpathcurveto{\pgfqpoint{4.039220in}{2.148128in}}{\pgfqpoint{4.049819in}{2.152519in}}{\pgfqpoint{4.057633in}{2.160332in}}%
\pgfpathcurveto{\pgfqpoint{4.065446in}{2.168146in}}{\pgfqpoint{4.069836in}{2.178745in}}{\pgfqpoint{4.069836in}{2.189795in}}%
\pgfpathcurveto{\pgfqpoint{4.069836in}{2.200845in}}{\pgfqpoint{4.065446in}{2.211444in}}{\pgfqpoint{4.057633in}{2.219258in}}%
\pgfpathcurveto{\pgfqpoint{4.049819in}{2.227071in}}{\pgfqpoint{4.039220in}{2.231462in}}{\pgfqpoint{4.028170in}{2.231462in}}%
\pgfpathcurveto{\pgfqpoint{4.017120in}{2.231462in}}{\pgfqpoint{4.006521in}{2.227071in}}{\pgfqpoint{3.998707in}{2.219258in}}%
\pgfpathcurveto{\pgfqpoint{3.990893in}{2.211444in}}{\pgfqpoint{3.986503in}{2.200845in}}{\pgfqpoint{3.986503in}{2.189795in}}%
\pgfpathcurveto{\pgfqpoint{3.986503in}{2.178745in}}{\pgfqpoint{3.990893in}{2.168146in}}{\pgfqpoint{3.998707in}{2.160332in}}%
\pgfpathcurveto{\pgfqpoint{4.006521in}{2.152519in}}{\pgfqpoint{4.017120in}{2.148128in}}{\pgfqpoint{4.028170in}{2.148128in}}%
\pgfpathclose%
\pgfusepath{stroke,fill}%
\end{pgfscope}%
\begin{pgfscope}%
\pgfpathrectangle{\pgfqpoint{1.016621in}{0.499691in}}{\pgfqpoint{3.875000in}{2.695000in}}%
\pgfusepath{clip}%
\pgfsetrectcap%
\pgfsetroundjoin%
\pgfsetlinewidth{1.505625pt}%
\definecolor{currentstroke}{rgb}{1.000000,0.388235,0.278431}%
\pgfsetstrokecolor{currentstroke}%
\pgfsetdash{}{0pt}%
\pgfpathmoveto{\pgfqpoint{1.197151in}{0.694248in}}%
\pgfpathlineto{\pgfqpoint{1.232645in}{0.688569in}}%
\pgfpathlineto{\pgfqpoint{1.268139in}{0.683497in}}%
\pgfpathlineto{\pgfqpoint{1.303634in}{0.679030in}}%
\pgfpathlineto{\pgfqpoint{1.339128in}{0.675170in}}%
\pgfpathlineto{\pgfqpoint{1.374622in}{0.671915in}}%
\pgfpathlineto{\pgfqpoint{1.410117in}{0.669267in}}%
\pgfpathlineto{\pgfqpoint{1.445611in}{0.667225in}}%
\pgfpathlineto{\pgfqpoint{1.481105in}{0.665789in}}%
\pgfpathlineto{\pgfqpoint{1.516600in}{0.664959in}}%
\pgfpathlineto{\pgfqpoint{1.552094in}{0.664735in}}%
\pgfpathlineto{\pgfqpoint{1.587588in}{0.665117in}}%
\pgfpathlineto{\pgfqpoint{1.623083in}{0.666105in}}%
\pgfpathlineto{\pgfqpoint{1.658577in}{0.667700in}}%
\pgfpathlineto{\pgfqpoint{1.694071in}{0.669900in}}%
\pgfpathlineto{\pgfqpoint{1.729566in}{0.672707in}}%
\pgfpathlineto{\pgfqpoint{1.765060in}{0.676119in}}%
\pgfpathlineto{\pgfqpoint{1.800555in}{0.680138in}}%
\pgfpathlineto{\pgfqpoint{1.836049in}{0.684763in}}%
\pgfpathlineto{\pgfqpoint{1.871543in}{0.689994in}}%
\pgfpathlineto{\pgfqpoint{1.907038in}{0.695831in}}%
\pgfpathlineto{\pgfqpoint{1.942532in}{0.702274in}}%
\pgfpathlineto{\pgfqpoint{1.978026in}{0.709323in}}%
\pgfpathlineto{\pgfqpoint{2.013521in}{0.716978in}}%
\pgfpathlineto{\pgfqpoint{2.049015in}{0.725239in}}%
\pgfpathlineto{\pgfqpoint{2.084509in}{0.734107in}}%
\pgfpathlineto{\pgfqpoint{2.120004in}{0.743580in}}%
\pgfpathlineto{\pgfqpoint{2.155498in}{0.753660in}}%
\pgfpathlineto{\pgfqpoint{2.190992in}{0.764346in}}%
\pgfpathlineto{\pgfqpoint{2.226487in}{0.775637in}}%
\pgfpathlineto{\pgfqpoint{2.261981in}{0.787535in}}%
\pgfpathlineto{\pgfqpoint{2.297475in}{0.800039in}}%
\pgfpathlineto{\pgfqpoint{2.332970in}{0.813149in}}%
\pgfpathlineto{\pgfqpoint{2.368464in}{0.826865in}}%
\pgfpathlineto{\pgfqpoint{2.403958in}{0.841188in}}%
\pgfpathlineto{\pgfqpoint{2.439453in}{0.856116in}}%
\pgfpathlineto{\pgfqpoint{2.474947in}{0.871650in}}%
\pgfpathlineto{\pgfqpoint{2.510441in}{0.887791in}}%
\pgfpathlineto{\pgfqpoint{2.545936in}{0.904537in}}%
\pgfpathlineto{\pgfqpoint{2.581430in}{0.921890in}}%
\pgfpathlineto{\pgfqpoint{2.616925in}{0.939849in}}%
\pgfpathlineto{\pgfqpoint{2.652419in}{0.958413in}}%
\pgfpathlineto{\pgfqpoint{2.687913in}{0.977584in}}%
\pgfpathlineto{\pgfqpoint{2.723408in}{0.997361in}}%
\pgfpathlineto{\pgfqpoint{2.758902in}{1.017745in}}%
\pgfpathlineto{\pgfqpoint{2.794396in}{1.038734in}}%
\pgfpathlineto{\pgfqpoint{2.829891in}{1.060329in}}%
\pgfpathlineto{\pgfqpoint{2.865385in}{1.082530in}}%
\pgfpathlineto{\pgfqpoint{2.900879in}{1.105338in}}%
\pgfpathlineto{\pgfqpoint{2.936374in}{1.128751in}}%
\pgfpathlineto{\pgfqpoint{2.971868in}{1.152771in}}%
\pgfpathlineto{\pgfqpoint{3.007362in}{1.177397in}}%
\pgfpathlineto{\pgfqpoint{3.042857in}{1.202628in}}%
\pgfpathlineto{\pgfqpoint{3.078351in}{1.228466in}}%
\pgfpathlineto{\pgfqpoint{3.113845in}{1.254910in}}%
\pgfpathlineto{\pgfqpoint{3.149340in}{1.281960in}}%
\pgfpathlineto{\pgfqpoint{3.184834in}{1.309617in}}%
\pgfpathlineto{\pgfqpoint{3.220328in}{1.337879in}}%
\pgfpathlineto{\pgfqpoint{3.255823in}{1.366747in}}%
\pgfpathlineto{\pgfqpoint{3.291317in}{1.396222in}}%
\pgfpathlineto{\pgfqpoint{3.326811in}{1.426302in}}%
\pgfpathlineto{\pgfqpoint{3.362306in}{1.456989in}}%
\pgfpathlineto{\pgfqpoint{3.397800in}{1.488281in}}%
\pgfpathlineto{\pgfqpoint{3.433295in}{1.520180in}}%
\pgfpathlineto{\pgfqpoint{3.468789in}{1.552685in}}%
\pgfpathlineto{\pgfqpoint{3.504283in}{1.585796in}}%
\pgfpathlineto{\pgfqpoint{3.539778in}{1.619513in}}%
\pgfpathlineto{\pgfqpoint{3.575272in}{1.653836in}}%
\pgfpathlineto{\pgfqpoint{3.610766in}{1.688765in}}%
\pgfpathlineto{\pgfqpoint{3.646261in}{1.724301in}}%
\pgfpathlineto{\pgfqpoint{3.681755in}{1.760442in}}%
\pgfpathlineto{\pgfqpoint{3.717249in}{1.797190in}}%
\pgfpathlineto{\pgfqpoint{3.752744in}{1.834543in}}%
\pgfpathlineto{\pgfqpoint{3.788238in}{1.872503in}}%
\pgfpathlineto{\pgfqpoint{3.823732in}{1.911069in}}%
\pgfpathlineto{\pgfqpoint{3.859227in}{1.950240in}}%
\pgfpathlineto{\pgfqpoint{3.894721in}{1.990018in}}%
\pgfpathlineto{\pgfqpoint{3.930215in}{2.030402in}}%
\pgfpathlineto{\pgfqpoint{3.965710in}{2.071392in}}%
\pgfpathlineto{\pgfqpoint{4.001204in}{2.112989in}}%
\pgfpathlineto{\pgfqpoint{4.036698in}{2.155191in}}%
\pgfpathlineto{\pgfqpoint{4.072193in}{2.197999in}}%
\pgfpathlineto{\pgfqpoint{4.107687in}{2.241414in}}%
\pgfpathlineto{\pgfqpoint{4.143181in}{2.285434in}}%
\pgfpathlineto{\pgfqpoint{4.178676in}{2.330061in}}%
\pgfpathlineto{\pgfqpoint{4.214170in}{2.375294in}}%
\pgfpathlineto{\pgfqpoint{4.249665in}{2.421132in}}%
\pgfpathlineto{\pgfqpoint{4.285159in}{2.467577in}}%
\pgfpathlineto{\pgfqpoint{4.320653in}{2.514628in}}%
\pgfpathlineto{\pgfqpoint{4.356148in}{2.562285in}}%
\pgfpathlineto{\pgfqpoint{4.391642in}{2.610549in}}%
\pgfpathlineto{\pgfqpoint{4.427136in}{2.659418in}}%
\pgfpathlineto{\pgfqpoint{4.462631in}{2.708893in}}%
\pgfpathlineto{\pgfqpoint{4.498125in}{2.758975in}}%
\pgfpathlineto{\pgfqpoint{4.533619in}{2.809662in}}%
\pgfpathlineto{\pgfqpoint{4.569114in}{2.860956in}}%
\pgfpathlineto{\pgfqpoint{4.604608in}{2.912855in}}%
\pgfpathlineto{\pgfqpoint{4.640102in}{2.965361in}}%
\pgfpathlineto{\pgfqpoint{4.675597in}{3.018473in}}%
\pgfpathlineto{\pgfqpoint{4.711091in}{3.072191in}}%
\pgfusepath{stroke}%
\end{pgfscope}%
\begin{pgfscope}%
\pgfpathrectangle{\pgfqpoint{1.016621in}{0.499691in}}{\pgfqpoint{3.875000in}{2.695000in}}%
\pgfusepath{clip}%
\pgfsetrectcap%
\pgfsetroundjoin%
\pgfsetlinewidth{1.505625pt}%
\definecolor{currentstroke}{rgb}{1.000000,0.843137,0.000000}%
\pgfsetstrokecolor{currentstroke}%
\pgfsetdash{}{0pt}%
\pgfpathmoveto{\pgfqpoint{1.197151in}{0.754316in}}%
\pgfpathlineto{\pgfqpoint{1.232645in}{0.737897in}}%
\pgfpathlineto{\pgfqpoint{1.268139in}{0.722615in}}%
\pgfpathlineto{\pgfqpoint{1.303634in}{0.708461in}}%
\pgfpathlineto{\pgfqpoint{1.339128in}{0.695423in}}%
\pgfpathlineto{\pgfqpoint{1.374622in}{0.683491in}}%
\pgfpathlineto{\pgfqpoint{1.410117in}{0.672654in}}%
\pgfpathlineto{\pgfqpoint{1.445611in}{0.662902in}}%
\pgfpathlineto{\pgfqpoint{1.481105in}{0.654224in}}%
\pgfpathlineto{\pgfqpoint{1.516600in}{0.646610in}}%
\pgfpathlineto{\pgfqpoint{1.552094in}{0.640048in}}%
\pgfpathlineto{\pgfqpoint{1.587588in}{0.634528in}}%
\pgfpathlineto{\pgfqpoint{1.623083in}{0.630040in}}%
\pgfpathlineto{\pgfqpoint{1.658577in}{0.626573in}}%
\pgfpathlineto{\pgfqpoint{1.694071in}{0.624116in}}%
\pgfpathlineto{\pgfqpoint{1.729566in}{0.622659in}}%
\pgfpathlineto{\pgfqpoint{1.765060in}{0.622191in}}%
\pgfpathlineto{\pgfqpoint{1.800555in}{0.622702in}}%
\pgfpathlineto{\pgfqpoint{1.836049in}{0.624180in}}%
\pgfpathlineto{\pgfqpoint{1.871543in}{0.626616in}}%
\pgfpathlineto{\pgfqpoint{1.907038in}{0.629999in}}%
\pgfpathlineto{\pgfqpoint{1.942532in}{0.634317in}}%
\pgfpathlineto{\pgfqpoint{1.978026in}{0.639561in}}%
\pgfpathlineto{\pgfqpoint{2.013521in}{0.645720in}}%
\pgfpathlineto{\pgfqpoint{2.049015in}{0.652783in}}%
\pgfpathlineto{\pgfqpoint{2.084509in}{0.660740in}}%
\pgfpathlineto{\pgfqpoint{2.120004in}{0.669580in}}%
\pgfpathlineto{\pgfqpoint{2.155498in}{0.679292in}}%
\pgfpathlineto{\pgfqpoint{2.190992in}{0.689866in}}%
\pgfpathlineto{\pgfqpoint{2.226487in}{0.701291in}}%
\pgfpathlineto{\pgfqpoint{2.261981in}{0.713557in}}%
\pgfpathlineto{\pgfqpoint{2.297475in}{0.726653in}}%
\pgfpathlineto{\pgfqpoint{2.332970in}{0.740568in}}%
\pgfpathlineto{\pgfqpoint{2.368464in}{0.755291in}}%
\pgfpathlineto{\pgfqpoint{2.403958in}{0.770813in}}%
\pgfpathlineto{\pgfqpoint{2.439453in}{0.787123in}}%
\pgfpathlineto{\pgfqpoint{2.474947in}{0.804209in}}%
\pgfpathlineto{\pgfqpoint{2.510441in}{0.822061in}}%
\pgfpathlineto{\pgfqpoint{2.545936in}{0.840670in}}%
\pgfpathlineto{\pgfqpoint{2.581430in}{0.860023in}}%
\pgfpathlineto{\pgfqpoint{2.616925in}{0.880111in}}%
\pgfpathlineto{\pgfqpoint{2.652419in}{0.900922in}}%
\pgfpathlineto{\pgfqpoint{2.687913in}{0.922447in}}%
\pgfpathlineto{\pgfqpoint{2.723408in}{0.944675in}}%
\pgfpathlineto{\pgfqpoint{2.758902in}{0.967594in}}%
\pgfpathlineto{\pgfqpoint{2.794396in}{0.991195in}}%
\pgfpathlineto{\pgfqpoint{2.829891in}{1.015466in}}%
\pgfpathlineto{\pgfqpoint{2.865385in}{1.040398in}}%
\pgfpathlineto{\pgfqpoint{2.900879in}{1.065980in}}%
\pgfpathlineto{\pgfqpoint{2.936374in}{1.092200in}}%
\pgfpathlineto{\pgfqpoint{2.971868in}{1.119048in}}%
\pgfpathlineto{\pgfqpoint{3.007362in}{1.146515in}}%
\pgfpathlineto{\pgfqpoint{3.042857in}{1.174588in}}%
\pgfpathlineto{\pgfqpoint{3.078351in}{1.203258in}}%
\pgfpathlineto{\pgfqpoint{3.113845in}{1.232514in}}%
\pgfpathlineto{\pgfqpoint{3.149340in}{1.262345in}}%
\pgfpathlineto{\pgfqpoint{3.184834in}{1.292741in}}%
\pgfpathlineto{\pgfqpoint{3.220328in}{1.323691in}}%
\pgfpathlineto{\pgfqpoint{3.255823in}{1.355184in}}%
\pgfpathlineto{\pgfqpoint{3.291317in}{1.387210in}}%
\pgfpathlineto{\pgfqpoint{3.326811in}{1.419758in}}%
\pgfpathlineto{\pgfqpoint{3.362306in}{1.452818in}}%
\pgfpathlineto{\pgfqpoint{3.397800in}{1.486379in}}%
\pgfpathlineto{\pgfqpoint{3.433295in}{1.520430in}}%
\pgfpathlineto{\pgfqpoint{3.468789in}{1.554961in}}%
\pgfpathlineto{\pgfqpoint{3.504283in}{1.589961in}}%
\pgfpathlineto{\pgfqpoint{3.539778in}{1.625420in}}%
\pgfpathlineto{\pgfqpoint{3.575272in}{1.661327in}}%
\pgfpathlineto{\pgfqpoint{3.610766in}{1.697670in}}%
\pgfpathlineto{\pgfqpoint{3.646261in}{1.734441in}}%
\pgfpathlineto{\pgfqpoint{3.681755in}{1.771628in}}%
\pgfpathlineto{\pgfqpoint{3.717249in}{1.809220in}}%
\pgfpathlineto{\pgfqpoint{3.752744in}{1.847207in}}%
\pgfpathlineto{\pgfqpoint{3.788238in}{1.885579in}}%
\pgfpathlineto{\pgfqpoint{3.823732in}{1.924324in}}%
\pgfpathlineto{\pgfqpoint{3.859227in}{1.963432in}}%
\pgfpathlineto{\pgfqpoint{3.894721in}{2.002892in}}%
\pgfpathlineto{\pgfqpoint{3.930215in}{2.042694in}}%
\pgfpathlineto{\pgfqpoint{3.965710in}{2.082828in}}%
\pgfpathlineto{\pgfqpoint{4.001204in}{2.123282in}}%
\pgfpathlineto{\pgfqpoint{4.036698in}{2.164046in}}%
\pgfpathlineto{\pgfqpoint{4.072193in}{2.205109in}}%
\pgfpathlineto{\pgfqpoint{4.107687in}{2.246462in}}%
\pgfpathlineto{\pgfqpoint{4.143181in}{2.288092in}}%
\pgfpathlineto{\pgfqpoint{4.178676in}{2.329990in}}%
\pgfpathlineto{\pgfqpoint{4.214170in}{2.372145in}}%
\pgfpathlineto{\pgfqpoint{4.249665in}{2.414546in}}%
\pgfpathlineto{\pgfqpoint{4.285159in}{2.457182in}}%
\pgfpathlineto{\pgfqpoint{4.320653in}{2.500044in}}%
\pgfpathlineto{\pgfqpoint{4.356148in}{2.543121in}}%
\pgfpathlineto{\pgfqpoint{4.391642in}{2.586401in}}%
\pgfpathlineto{\pgfqpoint{4.427136in}{2.629874in}}%
\pgfpathlineto{\pgfqpoint{4.462631in}{2.673530in}}%
\pgfpathlineto{\pgfqpoint{4.498125in}{2.717358in}}%
\pgfpathlineto{\pgfqpoint{4.533619in}{2.761348in}}%
\pgfpathlineto{\pgfqpoint{4.569114in}{2.805488in}}%
\pgfpathlineto{\pgfqpoint{4.604608in}{2.849769in}}%
\pgfpathlineto{\pgfqpoint{4.640102in}{2.894179in}}%
\pgfpathlineto{\pgfqpoint{4.675597in}{2.938708in}}%
\pgfpathlineto{\pgfqpoint{4.711091in}{2.983345in}}%
\pgfusepath{stroke}%
\end{pgfscope}%
\begin{pgfscope}%
\pgfpathrectangle{\pgfqpoint{1.016621in}{0.499691in}}{\pgfqpoint{3.875000in}{2.695000in}}%
\pgfusepath{clip}%
\pgfsetrectcap%
\pgfsetroundjoin%
\pgfsetlinewidth{1.505625pt}%
\definecolor{currentstroke}{rgb}{0.196078,0.803922,0.196078}%
\pgfsetstrokecolor{currentstroke}%
\pgfsetdash{}{0pt}%
\pgfpathmoveto{\pgfqpoint{1.197151in}{0.691730in}}%
\pgfpathlineto{\pgfqpoint{1.232645in}{0.729032in}}%
\pgfpathlineto{\pgfqpoint{1.268139in}{0.762464in}}%
\pgfpathlineto{\pgfqpoint{1.303634in}{0.792244in}}%
\pgfpathlineto{\pgfqpoint{1.339128in}{0.818587in}}%
\pgfpathlineto{\pgfqpoint{1.374622in}{0.841701in}}%
\pgfpathlineto{\pgfqpoint{1.410117in}{0.861790in}}%
\pgfpathlineto{\pgfqpoint{1.445611in}{0.879055in}}%
\pgfpathlineto{\pgfqpoint{1.481105in}{0.893691in}}%
\pgfpathlineto{\pgfqpoint{1.516600in}{0.905888in}}%
\pgfpathlineto{\pgfqpoint{1.552094in}{0.915834in}}%
\pgfpathlineto{\pgfqpoint{1.587588in}{0.923711in}}%
\pgfpathlineto{\pgfqpoint{1.623083in}{0.929695in}}%
\pgfpathlineto{\pgfqpoint{1.658577in}{0.933959in}}%
\pgfpathlineto{\pgfqpoint{1.694071in}{0.936672in}}%
\pgfpathlineto{\pgfqpoint{1.729566in}{0.937998in}}%
\pgfpathlineto{\pgfqpoint{1.765060in}{0.938095in}}%
\pgfpathlineto{\pgfqpoint{1.800555in}{0.937119in}}%
\pgfpathlineto{\pgfqpoint{1.836049in}{0.935220in}}%
\pgfpathlineto{\pgfqpoint{1.871543in}{0.932544in}}%
\pgfpathlineto{\pgfqpoint{1.907038in}{0.929231in}}%
\pgfpathlineto{\pgfqpoint{1.942532in}{0.925418in}}%
\pgfpathlineto{\pgfqpoint{1.978026in}{0.921238in}}%
\pgfpathlineto{\pgfqpoint{2.013521in}{0.916817in}}%
\pgfpathlineto{\pgfqpoint{2.049015in}{0.912280in}}%
\pgfpathlineto{\pgfqpoint{2.084509in}{0.907745in}}%
\pgfpathlineto{\pgfqpoint{2.120004in}{0.903326in}}%
\pgfpathlineto{\pgfqpoint{2.155498in}{0.899131in}}%
\pgfpathlineto{\pgfqpoint{2.190992in}{0.895267in}}%
\pgfpathlineto{\pgfqpoint{2.226487in}{0.891834in}}%
\pgfpathlineto{\pgfqpoint{2.261981in}{0.888927in}}%
\pgfpathlineto{\pgfqpoint{2.297475in}{0.886639in}}%
\pgfpathlineto{\pgfqpoint{2.332970in}{0.885055in}}%
\pgfpathlineto{\pgfqpoint{2.368464in}{0.884259in}}%
\pgfpathlineto{\pgfqpoint{2.403958in}{0.884328in}}%
\pgfpathlineto{\pgfqpoint{2.439453in}{0.885336in}}%
\pgfpathlineto{\pgfqpoint{2.474947in}{0.887351in}}%
\pgfpathlineto{\pgfqpoint{2.510441in}{0.890438in}}%
\pgfpathlineto{\pgfqpoint{2.545936in}{0.894657in}}%
\pgfpathlineto{\pgfqpoint{2.581430in}{0.900062in}}%
\pgfpathlineto{\pgfqpoint{2.616925in}{0.906704in}}%
\pgfpathlineto{\pgfqpoint{2.652419in}{0.914630in}}%
\pgfpathlineto{\pgfqpoint{2.687913in}{0.923882in}}%
\pgfpathlineto{\pgfqpoint{2.723408in}{0.934495in}}%
\pgfpathlineto{\pgfqpoint{2.758902in}{0.946504in}}%
\pgfpathlineto{\pgfqpoint{2.794396in}{0.959936in}}%
\pgfpathlineto{\pgfqpoint{2.829891in}{0.974815in}}%
\pgfpathlineto{\pgfqpoint{2.865385in}{0.991160in}}%
\pgfpathlineto{\pgfqpoint{2.900879in}{1.008985in}}%
\pgfpathlineto{\pgfqpoint{2.936374in}{1.028300in}}%
\pgfpathlineto{\pgfqpoint{2.971868in}{1.049111in}}%
\pgfpathlineto{\pgfqpoint{3.007362in}{1.071419in}}%
\pgfpathlineto{\pgfqpoint{3.042857in}{1.095221in}}%
\pgfpathlineto{\pgfqpoint{3.078351in}{1.120507in}}%
\pgfpathlineto{\pgfqpoint{3.113845in}{1.147266in}}%
\pgfpathlineto{\pgfqpoint{3.149340in}{1.175481in}}%
\pgfpathlineto{\pgfqpoint{3.184834in}{1.205129in}}%
\pgfpathlineto{\pgfqpoint{3.220328in}{1.236185in}}%
\pgfpathlineto{\pgfqpoint{3.255823in}{1.268618in}}%
\pgfpathlineto{\pgfqpoint{3.291317in}{1.302393in}}%
\pgfpathlineto{\pgfqpoint{3.326811in}{1.337470in}}%
\pgfpathlineto{\pgfqpoint{3.362306in}{1.373804in}}%
\pgfpathlineto{\pgfqpoint{3.397800in}{1.411348in}}%
\pgfpathlineto{\pgfqpoint{3.433295in}{1.450047in}}%
\pgfpathlineto{\pgfqpoint{3.468789in}{1.489844in}}%
\pgfpathlineto{\pgfqpoint{3.504283in}{1.530676in}}%
\pgfpathlineto{\pgfqpoint{3.539778in}{1.572477in}}%
\pgfpathlineto{\pgfqpoint{3.575272in}{1.615176in}}%
\pgfpathlineto{\pgfqpoint{3.610766in}{1.658695in}}%
\pgfpathlineto{\pgfqpoint{3.646261in}{1.702955in}}%
\pgfpathlineto{\pgfqpoint{3.681755in}{1.747872in}}%
\pgfpathlineto{\pgfqpoint{3.717249in}{1.793354in}}%
\pgfpathlineto{\pgfqpoint{3.752744in}{1.839309in}}%
\pgfpathlineto{\pgfqpoint{3.788238in}{1.885638in}}%
\pgfpathlineto{\pgfqpoint{3.823732in}{1.932238in}}%
\pgfpathlineto{\pgfqpoint{3.859227in}{1.979000in}}%
\pgfpathlineto{\pgfqpoint{3.894721in}{2.025814in}}%
\pgfpathlineto{\pgfqpoint{3.930215in}{2.072562in}}%
\pgfpathlineto{\pgfqpoint{3.965710in}{2.119124in}}%
\pgfpathlineto{\pgfqpoint{4.001204in}{2.165373in}}%
\pgfpathlineto{\pgfqpoint{4.036698in}{2.211180in}}%
\pgfpathlineto{\pgfqpoint{4.072193in}{2.256409in}}%
\pgfpathlineto{\pgfqpoint{4.107687in}{2.300922in}}%
\pgfpathlineto{\pgfqpoint{4.143181in}{2.344574in}}%
\pgfpathlineto{\pgfqpoint{4.178676in}{2.387218in}}%
\pgfpathlineto{\pgfqpoint{4.214170in}{2.428701in}}%
\pgfpathlineto{\pgfqpoint{4.249665in}{2.468865in}}%
\pgfpathlineto{\pgfqpoint{4.285159in}{2.507548in}}%
\pgfpathlineto{\pgfqpoint{4.320653in}{2.544584in}}%
\pgfpathlineto{\pgfqpoint{4.356148in}{2.579802in}}%
\pgfpathlineto{\pgfqpoint{4.391642in}{2.613027in}}%
\pgfpathlineto{\pgfqpoint{4.427136in}{2.644079in}}%
\pgfpathlineto{\pgfqpoint{4.462631in}{2.672773in}}%
\pgfpathlineto{\pgfqpoint{4.498125in}{2.698921in}}%
\pgfpathlineto{\pgfqpoint{4.533619in}{2.722328in}}%
\pgfpathlineto{\pgfqpoint{4.569114in}{2.742797in}}%
\pgfpathlineto{\pgfqpoint{4.604608in}{2.760125in}}%
\pgfpathlineto{\pgfqpoint{4.640102in}{2.774105in}}%
\pgfpathlineto{\pgfqpoint{4.675597in}{2.784525in}}%
\pgfpathlineto{\pgfqpoint{4.711091in}{2.791170in}}%
\pgfusepath{stroke}%
\end{pgfscope}%
\begin{pgfscope}%
\pgfpathrectangle{\pgfqpoint{1.016621in}{0.499691in}}{\pgfqpoint{3.875000in}{2.695000in}}%
\pgfusepath{clip}%
\pgfsetrectcap%
\pgfsetroundjoin%
\pgfsetlinewidth{1.505625pt}%
\definecolor{currentstroke}{rgb}{0.529412,0.807843,0.921569}%
\pgfsetstrokecolor{currentstroke}%
\pgfsetdash{}{0pt}%
\pgfpathmoveto{\pgfqpoint{1.197151in}{0.697898in}}%
\pgfpathlineto{\pgfqpoint{1.232645in}{0.684788in}}%
\pgfpathlineto{\pgfqpoint{1.268139in}{0.674841in}}%
\pgfpathlineto{\pgfqpoint{1.303634in}{0.667752in}}%
\pgfpathlineto{\pgfqpoint{1.339128in}{0.663236in}}%
\pgfpathlineto{\pgfqpoint{1.374622in}{0.661022in}}%
\pgfpathlineto{\pgfqpoint{1.410117in}{0.660859in}}%
\pgfpathlineto{\pgfqpoint{1.445611in}{0.662510in}}%
\pgfpathlineto{\pgfqpoint{1.481105in}{0.665754in}}%
\pgfpathlineto{\pgfqpoint{1.516600in}{0.670387in}}%
\pgfpathlineto{\pgfqpoint{1.552094in}{0.676218in}}%
\pgfpathlineto{\pgfqpoint{1.587588in}{0.683072in}}%
\pgfpathlineto{\pgfqpoint{1.623083in}{0.690787in}}%
\pgfpathlineto{\pgfqpoint{1.658577in}{0.699217in}}%
\pgfpathlineto{\pgfqpoint{1.694071in}{0.708226in}}%
\pgfpathlineto{\pgfqpoint{1.729566in}{0.717694in}}%
\pgfpathlineto{\pgfqpoint{1.765060in}{0.727511in}}%
\pgfpathlineto{\pgfqpoint{1.800555in}{0.737581in}}%
\pgfpathlineto{\pgfqpoint{1.836049in}{0.747818in}}%
\pgfpathlineto{\pgfqpoint{1.871543in}{0.758149in}}%
\pgfpathlineto{\pgfqpoint{1.907038in}{0.768510in}}%
\pgfpathlineto{\pgfqpoint{1.942532in}{0.778848in}}%
\pgfpathlineto{\pgfqpoint{1.978026in}{0.789121in}}%
\pgfpathlineto{\pgfqpoint{2.013521in}{0.799293in}}%
\pgfpathlineto{\pgfqpoint{2.049015in}{0.809342in}}%
\pgfpathlineto{\pgfqpoint{2.084509in}{0.819251in}}%
\pgfpathlineto{\pgfqpoint{2.120004in}{0.829012in}}%
\pgfpathlineto{\pgfqpoint{2.155498in}{0.838625in}}%
\pgfpathlineto{\pgfqpoint{2.190992in}{0.848099in}}%
\pgfpathlineto{\pgfqpoint{2.226487in}{0.857447in}}%
\pgfpathlineto{\pgfqpoint{2.261981in}{0.866690in}}%
\pgfpathlineto{\pgfqpoint{2.297475in}{0.875856in}}%
\pgfpathlineto{\pgfqpoint{2.332970in}{0.884976in}}%
\pgfpathlineto{\pgfqpoint{2.368464in}{0.894090in}}%
\pgfpathlineto{\pgfqpoint{2.403958in}{0.903238in}}%
\pgfpathlineto{\pgfqpoint{2.439453in}{0.912470in}}%
\pgfpathlineto{\pgfqpoint{2.474947in}{0.921834in}}%
\pgfpathlineto{\pgfqpoint{2.510441in}{0.931387in}}%
\pgfpathlineto{\pgfqpoint{2.545936in}{0.941186in}}%
\pgfpathlineto{\pgfqpoint{2.581430in}{0.951290in}}%
\pgfpathlineto{\pgfqpoint{2.616925in}{0.961764in}}%
\pgfpathlineto{\pgfqpoint{2.652419in}{0.972670in}}%
\pgfpathlineto{\pgfqpoint{2.687913in}{0.984075in}}%
\pgfpathlineto{\pgfqpoint{2.723408in}{0.996046in}}%
\pgfpathlineto{\pgfqpoint{2.758902in}{1.008650in}}%
\pgfpathlineto{\pgfqpoint{2.794396in}{1.021954in}}%
\pgfpathlineto{\pgfqpoint{2.829891in}{1.036027in}}%
\pgfpathlineto{\pgfqpoint{2.865385in}{1.050933in}}%
\pgfpathlineto{\pgfqpoint{2.900879in}{1.066739in}}%
\pgfpathlineto{\pgfqpoint{2.936374in}{1.083508in}}%
\pgfpathlineto{\pgfqpoint{2.971868in}{1.101301in}}%
\pgfpathlineto{\pgfqpoint{3.007362in}{1.120180in}}%
\pgfpathlineto{\pgfqpoint{3.042857in}{1.140198in}}%
\pgfpathlineto{\pgfqpoint{3.078351in}{1.161411in}}%
\pgfpathlineto{\pgfqpoint{3.113845in}{1.183867in}}%
\pgfpathlineto{\pgfqpoint{3.149340in}{1.207612in}}%
\pgfpathlineto{\pgfqpoint{3.184834in}{1.232687in}}%
\pgfpathlineto{\pgfqpoint{3.220328in}{1.259127in}}%
\pgfpathlineto{\pgfqpoint{3.255823in}{1.286963in}}%
\pgfpathlineto{\pgfqpoint{3.291317in}{1.316220in}}%
\pgfpathlineto{\pgfqpoint{3.326811in}{1.346915in}}%
\pgfpathlineto{\pgfqpoint{3.362306in}{1.379062in}}%
\pgfpathlineto{\pgfqpoint{3.397800in}{1.412663in}}%
\pgfpathlineto{\pgfqpoint{3.433295in}{1.447717in}}%
\pgfpathlineto{\pgfqpoint{3.468789in}{1.484212in}}%
\pgfpathlineto{\pgfqpoint{3.504283in}{1.522129in}}%
\pgfpathlineto{\pgfqpoint{3.539778in}{1.561440in}}%
\pgfpathlineto{\pgfqpoint{3.575272in}{1.602108in}}%
\pgfpathlineto{\pgfqpoint{3.610766in}{1.644086in}}%
\pgfpathlineto{\pgfqpoint{3.646261in}{1.687317in}}%
\pgfpathlineto{\pgfqpoint{3.681755in}{1.731733in}}%
\pgfpathlineto{\pgfqpoint{3.717249in}{1.777256in}}%
\pgfpathlineto{\pgfqpoint{3.752744in}{1.823796in}}%
\pgfpathlineto{\pgfqpoint{3.788238in}{1.871252in}}%
\pgfpathlineto{\pgfqpoint{3.823732in}{1.919509in}}%
\pgfpathlineto{\pgfqpoint{3.859227in}{1.968441in}}%
\pgfpathlineto{\pgfqpoint{3.894721in}{2.017910in}}%
\pgfpathlineto{\pgfqpoint{3.930215in}{2.067761in}}%
\pgfpathlineto{\pgfqpoint{3.965710in}{2.117829in}}%
\pgfpathlineto{\pgfqpoint{4.001204in}{2.167931in}}%
\pgfpathlineto{\pgfqpoint{4.036698in}{2.217872in}}%
\pgfpathlineto{\pgfqpoint{4.072193in}{2.267441in}}%
\pgfpathlineto{\pgfqpoint{4.107687in}{2.316411in}}%
\pgfpathlineto{\pgfqpoint{4.143181in}{2.364539in}}%
\pgfpathlineto{\pgfqpoint{4.178676in}{2.411567in}}%
\pgfpathlineto{\pgfqpoint{4.214170in}{2.457217in}}%
\pgfpathlineto{\pgfqpoint{4.249665in}{2.501197in}}%
\pgfpathlineto{\pgfqpoint{4.285159in}{2.543195in}}%
\pgfpathlineto{\pgfqpoint{4.320653in}{2.582883in}}%
\pgfpathlineto{\pgfqpoint{4.356148in}{2.619912in}}%
\pgfpathlineto{\pgfqpoint{4.391642in}{2.653916in}}%
\pgfpathlineto{\pgfqpoint{4.427136in}{2.684507in}}%
\pgfpathlineto{\pgfqpoint{4.462631in}{2.711281in}}%
\pgfpathlineto{\pgfqpoint{4.498125in}{2.733810in}}%
\pgfpathlineto{\pgfqpoint{4.533619in}{2.751647in}}%
\pgfpathlineto{\pgfqpoint{4.569114in}{2.764324in}}%
\pgfpathlineto{\pgfqpoint{4.604608in}{2.771352in}}%
\pgfpathlineto{\pgfqpoint{4.640102in}{2.772218in}}%
\pgfpathlineto{\pgfqpoint{4.675597in}{2.766388in}}%
\pgfpathlineto{\pgfqpoint{4.711091in}{2.753304in}}%
\pgfusepath{stroke}%
\end{pgfscope}%
\begin{pgfscope}%
\pgfsetrectcap%
\pgfsetmiterjoin%
\pgfsetlinewidth{0.803000pt}%
\definecolor{currentstroke}{rgb}{0.000000,0.000000,0.000000}%
\pgfsetstrokecolor{currentstroke}%
\pgfsetdash{}{0pt}%
\pgfpathmoveto{\pgfqpoint{1.016621in}{0.499691in}}%
\pgfpathlineto{\pgfqpoint{1.016621in}{3.194691in}}%
\pgfusepath{stroke}%
\end{pgfscope}%
\begin{pgfscope}%
\pgfsetrectcap%
\pgfsetmiterjoin%
\pgfsetlinewidth{0.803000pt}%
\definecolor{currentstroke}{rgb}{0.000000,0.000000,0.000000}%
\pgfsetstrokecolor{currentstroke}%
\pgfsetdash{}{0pt}%
\pgfpathmoveto{\pgfqpoint{4.891621in}{0.499691in}}%
\pgfpathlineto{\pgfqpoint{4.891621in}{3.194691in}}%
\pgfusepath{stroke}%
\end{pgfscope}%
\begin{pgfscope}%
\pgfsetrectcap%
\pgfsetmiterjoin%
\pgfsetlinewidth{0.803000pt}%
\definecolor{currentstroke}{rgb}{0.000000,0.000000,0.000000}%
\pgfsetstrokecolor{currentstroke}%
\pgfsetdash{}{0pt}%
\pgfpathmoveto{\pgfqpoint{1.016621in}{0.499691in}}%
\pgfpathlineto{\pgfqpoint{4.891621in}{0.499691in}}%
\pgfusepath{stroke}%
\end{pgfscope}%
\begin{pgfscope}%
\pgfsetrectcap%
\pgfsetmiterjoin%
\pgfsetlinewidth{0.803000pt}%
\definecolor{currentstroke}{rgb}{0.000000,0.000000,0.000000}%
\pgfsetstrokecolor{currentstroke}%
\pgfsetdash{}{0pt}%
\pgfpathmoveto{\pgfqpoint{1.016621in}{3.194691in}}%
\pgfpathlineto{\pgfqpoint{4.891621in}{3.194691in}}%
\pgfusepath{stroke}%
\end{pgfscope}%
\end{pgfpicture}%
\makeatother%
\endgroup%

    \caption{Tercer intento de representación de $\varphi$ frente a f (escala logarítmica), ajustando a polinomios de grado \textcolor{Red}{2}, \textcolor{Golden}{3}, \textcolor{Green}{4} y \textcolor{Blue}{5}}
  \end{figure}

  Alguno se acerca, pero parece excesivamente complicado para representar estos valores. Entonces pensamos en una función sigmoide. Por ejemplo, la función logística, generalmente definida cómo $S(x) = \frac{1}{1+e^{-x}}$, es una curva con centro en (0, 0,5) que tiene asíntotas en 0 y en 1. Nuestra función no cumple estas propiedades, sin embargo, su forma parece aproximarse a una sigmoide.

  Si quieremos ajustar a dicha función, primero normalizaremos los datos. Para empezar, la curva logísitica siempre es positiva, así que tomaremos nuestro valor mínimo, -99,4, que redondearemos a 100 para hacerlo más simple, y sumaremos 100 a todos los puntos de la función, moviéndola en el eje vertical. Esto no es completamente correcto, ya que no podemos saber cúal es la asíntota horizontal a la que converge sin más datos, pero haremos una aproximación. Ahora, cómo los datos irán de 0 hasta $100-14,6=85,4$, multiplicaremos la función por $\frac{1}{85,4}$, que de nuevo redondearemos a 0.01 para hacer los cálculos más sencillos, ya que estamos buscando una aproximación. Por último, el valor medio de y ha de estar en $x=0$. Es posible pensar que ese punto de inflexión será la frecuencia de corte, por lo que probaremos a desplazar toda la función $log(1326,29) = 3,1226$ unidades hacia la izquierda. Para todo ello haremos uso de este código de \code{python}:

  \begin{python}
    y3 = (y2 + 100) * 0.01
    x3 = x2 - 3.1226
  \end{python}

  Ahora construiremos una función que ajuste a la sigmoide, haciendo uso de una librería de \code{scypi} llamada \code{optimize}, que nos devolverá los parámetros óptimos (\code{popt}) así como su covarianza (\code{pcov}):

  \begin{python}
    from scipy.optimize import curve_fit

    def sigmoide(x, a, b):
        return 1.0 / (1.0 + np.exp(-a*(x-b)))

    popt, pcov = curve_fit(sigmoide, x3, y3, method='dogbox')
  \end{python}

  Por último, generamos los puntos de la función y volvemos a desnormalizarlos deshaciendo los cambios que aplicamos anteriormente, y dibujamos la función.

  \begin{python}
    xs = np.arange(-3., +3., 0.01)
    ys = (sigmoide(xs, *popt) * 100) - 100
    xs += 3.1226

    plt.plot(xs, ys, color="gold")
  \end{python}

  \begin{figure}[H]
    %\centering
    \hspace{2.5em} %% Creator: Matplotlib, PGF backend
%%
%% To include the figure in your LaTeX document, write
%%   \input{<filename>.pgf}
%%
%% Make sure the required packages are loaded in your preamble
%%   \usepackage{pgf}
%%
%% Figures using additional raster images can only be included by \input if
%% they are in the same directory as the main LaTeX file. For loading figures
%% from other directories you can use the `import` package
%%   \usepackage{import}
%% and then include the figures with
%%   \import{<path to file>}{<filename>.pgf}
%%
%% Matplotlib used the following preamble
%%
\begingroup%
\makeatletter%
\begin{pgfpicture}%
\pgfpathrectangle{\pgfpointorigin}{\pgfqpoint{4.991621in}{3.294691in}}%
\pgfusepath{use as bounding box, clip}%
\begin{pgfscope}%
\pgfsetbuttcap%
\pgfsetmiterjoin%
\definecolor{currentfill}{rgb}{1.000000,1.000000,1.000000}%
\pgfsetfillcolor{currentfill}%
\pgfsetlinewidth{0.000000pt}%
\definecolor{currentstroke}{rgb}{1.000000,1.000000,1.000000}%
\pgfsetstrokecolor{currentstroke}%
\pgfsetdash{}{0pt}%
\pgfpathmoveto{\pgfqpoint{0.000000in}{0.000000in}}%
\pgfpathlineto{\pgfqpoint{4.991621in}{0.000000in}}%
\pgfpathlineto{\pgfqpoint{4.991621in}{3.294691in}}%
\pgfpathlineto{\pgfqpoint{0.000000in}{3.294691in}}%
\pgfpathclose%
\pgfusepath{fill}%
\end{pgfscope}%
\begin{pgfscope}%
\pgfsetbuttcap%
\pgfsetmiterjoin%
\definecolor{currentfill}{rgb}{1.000000,1.000000,1.000000}%
\pgfsetfillcolor{currentfill}%
\pgfsetlinewidth{0.000000pt}%
\definecolor{currentstroke}{rgb}{0.000000,0.000000,0.000000}%
\pgfsetstrokecolor{currentstroke}%
\pgfsetstrokeopacity{0.000000}%
\pgfsetdash{}{0pt}%
\pgfpathmoveto{\pgfqpoint{1.016621in}{0.499691in}}%
\pgfpathlineto{\pgfqpoint{4.891621in}{0.499691in}}%
\pgfpathlineto{\pgfqpoint{4.891621in}{3.194691in}}%
\pgfpathlineto{\pgfqpoint{1.016621in}{3.194691in}}%
\pgfpathclose%
\pgfusepath{fill}%
\end{pgfscope}%
\begin{pgfscope}%
\pgfpathrectangle{\pgfqpoint{1.016621in}{0.499691in}}{\pgfqpoint{3.875000in}{2.695000in}}%
\pgfusepath{clip}%
\pgfsetbuttcap%
\pgfsetroundjoin%
\definecolor{currentfill}{rgb}{0.117647,0.564706,1.000000}%
\pgfsetfillcolor{currentfill}%
\pgfsetlinewidth{1.003750pt}%
\definecolor{currentstroke}{rgb}{0.117647,0.564706,1.000000}%
\pgfsetstrokecolor{currentstroke}%
\pgfsetdash{}{0pt}%
\pgfpathmoveto{\pgfqpoint{1.708757in}{0.594282in}}%
\pgfpathcurveto{\pgfqpoint{1.719807in}{0.594282in}}{\pgfqpoint{1.730407in}{0.598672in}}{\pgfqpoint{1.738220in}{0.606486in}}%
\pgfpathcurveto{\pgfqpoint{1.746034in}{0.614299in}}{\pgfqpoint{1.750424in}{0.624898in}}{\pgfqpoint{1.750424in}{0.635948in}}%
\pgfpathcurveto{\pgfqpoint{1.750424in}{0.646999in}}{\pgfqpoint{1.746034in}{0.657598in}}{\pgfqpoint{1.738220in}{0.665411in}}%
\pgfpathcurveto{\pgfqpoint{1.730407in}{0.673225in}}{\pgfqpoint{1.719807in}{0.677615in}}{\pgfqpoint{1.708757in}{0.677615in}}%
\pgfpathcurveto{\pgfqpoint{1.697707in}{0.677615in}}{\pgfqpoint{1.687108in}{0.673225in}}{\pgfqpoint{1.679295in}{0.665411in}}%
\pgfpathcurveto{\pgfqpoint{1.671481in}{0.657598in}}{\pgfqpoint{1.667091in}{0.646999in}}{\pgfqpoint{1.667091in}{0.635948in}}%
\pgfpathcurveto{\pgfqpoint{1.667091in}{0.624898in}}{\pgfqpoint{1.671481in}{0.614299in}}{\pgfqpoint{1.679295in}{0.606486in}}%
\pgfpathcurveto{\pgfqpoint{1.687108in}{0.598672in}}{\pgfqpoint{1.697707in}{0.594282in}}{\pgfqpoint{1.708757in}{0.594282in}}%
\pgfpathclose%
\pgfusepath{stroke,fill}%
\end{pgfscope}%
\begin{pgfscope}%
\pgfpathrectangle{\pgfqpoint{1.016621in}{0.499691in}}{\pgfqpoint{3.875000in}{2.695000in}}%
\pgfusepath{clip}%
\pgfsetbuttcap%
\pgfsetroundjoin%
\definecolor{currentfill}{rgb}{0.117647,0.564706,1.000000}%
\pgfsetfillcolor{currentfill}%
\pgfsetlinewidth{1.003750pt}%
\definecolor{currentstroke}{rgb}{0.117647,0.564706,1.000000}%
\pgfsetstrokecolor{currentstroke}%
\pgfsetdash{}{0pt}%
\pgfpathmoveto{\pgfqpoint{2.167476in}{0.702161in}}%
\pgfpathcurveto{\pgfqpoint{2.178527in}{0.702161in}}{\pgfqpoint{2.189126in}{0.706551in}}{\pgfqpoint{2.196939in}{0.714365in}}%
\pgfpathcurveto{\pgfqpoint{2.204753in}{0.722178in}}{\pgfqpoint{2.209143in}{0.732777in}}{\pgfqpoint{2.209143in}{0.743828in}}%
\pgfpathcurveto{\pgfqpoint{2.209143in}{0.754878in}}{\pgfqpoint{2.204753in}{0.765477in}}{\pgfqpoint{2.196939in}{0.773290in}}%
\pgfpathcurveto{\pgfqpoint{2.189126in}{0.781104in}}{\pgfqpoint{2.178527in}{0.785494in}}{\pgfqpoint{2.167476in}{0.785494in}}%
\pgfpathcurveto{\pgfqpoint{2.156426in}{0.785494in}}{\pgfqpoint{2.145827in}{0.781104in}}{\pgfqpoint{2.138014in}{0.773290in}}%
\pgfpathcurveto{\pgfqpoint{2.130200in}{0.765477in}}{\pgfqpoint{2.125810in}{0.754878in}}{\pgfqpoint{2.125810in}{0.743828in}}%
\pgfpathcurveto{\pgfqpoint{2.125810in}{0.732777in}}{\pgfqpoint{2.130200in}{0.722178in}}{\pgfqpoint{2.138014in}{0.714365in}}%
\pgfpathcurveto{\pgfqpoint{2.145827in}{0.706551in}}{\pgfqpoint{2.156426in}{0.702161in}}{\pgfqpoint{2.167476in}{0.702161in}}%
\pgfpathclose%
\pgfusepath{stroke,fill}%
\end{pgfscope}%
\begin{pgfscope}%
\pgfpathrectangle{\pgfqpoint{1.016621in}{0.499691in}}{\pgfqpoint{3.875000in}{2.695000in}}%
\pgfusepath{clip}%
\pgfsetbuttcap%
\pgfsetroundjoin%
\definecolor{currentfill}{rgb}{0.117647,0.564706,1.000000}%
\pgfsetfillcolor{currentfill}%
\pgfsetlinewidth{1.003750pt}%
\definecolor{currentstroke}{rgb}{0.117647,0.564706,1.000000}%
\pgfsetstrokecolor{currentstroke}%
\pgfsetdash{}{0pt}%
\pgfpathmoveto{\pgfqpoint{2.343907in}{0.913016in}}%
\pgfpathcurveto{\pgfqpoint{2.354957in}{0.913016in}}{\pgfqpoint{2.365556in}{0.917406in}}{\pgfqpoint{2.373370in}{0.925220in}}%
\pgfpathcurveto{\pgfqpoint{2.381183in}{0.933033in}}{\pgfqpoint{2.385574in}{0.943632in}}{\pgfqpoint{2.385574in}{0.954682in}}%
\pgfpathcurveto{\pgfqpoint{2.385574in}{0.965733in}}{\pgfqpoint{2.381183in}{0.976332in}}{\pgfqpoint{2.373370in}{0.984145in}}%
\pgfpathcurveto{\pgfqpoint{2.365556in}{0.991959in}}{\pgfqpoint{2.354957in}{0.996349in}}{\pgfqpoint{2.343907in}{0.996349in}}%
\pgfpathcurveto{\pgfqpoint{2.332857in}{0.996349in}}{\pgfqpoint{2.322258in}{0.991959in}}{\pgfqpoint{2.314444in}{0.984145in}}%
\pgfpathcurveto{\pgfqpoint{2.306630in}{0.976332in}}{\pgfqpoint{2.302240in}{0.965733in}}{\pgfqpoint{2.302240in}{0.954682in}}%
\pgfpathcurveto{\pgfqpoint{2.302240in}{0.943632in}}{\pgfqpoint{2.306630in}{0.933033in}}{\pgfqpoint{2.314444in}{0.925220in}}%
\pgfpathcurveto{\pgfqpoint{2.322258in}{0.917406in}}{\pgfqpoint{2.332857in}{0.913016in}}{\pgfqpoint{2.343907in}{0.913016in}}%
\pgfpathclose%
\pgfusepath{stroke,fill}%
\end{pgfscope}%
\begin{pgfscope}%
\pgfpathrectangle{\pgfqpoint{1.016621in}{0.499691in}}{\pgfqpoint{3.875000in}{2.695000in}}%
\pgfusepath{clip}%
\pgfsetbuttcap%
\pgfsetroundjoin%
\definecolor{currentfill}{rgb}{0.117647,0.564706,1.000000}%
\pgfsetfillcolor{currentfill}%
\pgfsetlinewidth{1.003750pt}%
\definecolor{currentstroke}{rgb}{0.117647,0.564706,1.000000}%
\pgfsetstrokecolor{currentstroke}%
\pgfsetdash{}{0pt}%
\pgfpathmoveto{\pgfqpoint{2.561504in}{0.954696in}}%
\pgfpathcurveto{\pgfqpoint{2.572554in}{0.954696in}}{\pgfqpoint{2.583154in}{0.959087in}}{\pgfqpoint{2.590967in}{0.966900in}}%
\pgfpathcurveto{\pgfqpoint{2.598781in}{0.974714in}}{\pgfqpoint{2.603171in}{0.985313in}}{\pgfqpoint{2.603171in}{0.996363in}}%
\pgfpathcurveto{\pgfqpoint{2.603171in}{1.007413in}}{\pgfqpoint{2.598781in}{1.018012in}}{\pgfqpoint{2.590967in}{1.025826in}}%
\pgfpathcurveto{\pgfqpoint{2.583154in}{1.033639in}}{\pgfqpoint{2.572554in}{1.038030in}}{\pgfqpoint{2.561504in}{1.038030in}}%
\pgfpathcurveto{\pgfqpoint{2.550454in}{1.038030in}}{\pgfqpoint{2.539855in}{1.033639in}}{\pgfqpoint{2.532042in}{1.025826in}}%
\pgfpathcurveto{\pgfqpoint{2.524228in}{1.018012in}}{\pgfqpoint{2.519838in}{1.007413in}}{\pgfqpoint{2.519838in}{0.996363in}}%
\pgfpathcurveto{\pgfqpoint{2.519838in}{0.985313in}}{\pgfqpoint{2.524228in}{0.974714in}}{\pgfqpoint{2.532042in}{0.966900in}}%
\pgfpathcurveto{\pgfqpoint{2.539855in}{0.959087in}}{\pgfqpoint{2.550454in}{0.954696in}}{\pgfqpoint{2.561504in}{0.954696in}}%
\pgfpathclose%
\pgfusepath{stroke,fill}%
\end{pgfscope}%
\begin{pgfscope}%
\pgfpathrectangle{\pgfqpoint{1.016621in}{0.499691in}}{\pgfqpoint{3.875000in}{2.695000in}}%
\pgfusepath{clip}%
\pgfsetbuttcap%
\pgfsetroundjoin%
\definecolor{currentfill}{rgb}{0.117647,0.564706,1.000000}%
\pgfsetfillcolor{currentfill}%
\pgfsetlinewidth{1.003750pt}%
\definecolor{currentstroke}{rgb}{0.117647,0.564706,1.000000}%
\pgfsetstrokecolor{currentstroke}%
\pgfsetdash{}{0pt}%
\pgfpathmoveto{\pgfqpoint{2.649720in}{1.138581in}}%
\pgfpathcurveto{\pgfqpoint{2.660770in}{1.138581in}}{\pgfqpoint{2.671369in}{1.142972in}}{\pgfqpoint{2.679182in}{1.150785in}}%
\pgfpathcurveto{\pgfqpoint{2.686996in}{1.158599in}}{\pgfqpoint{2.691386in}{1.169198in}}{\pgfqpoint{2.691386in}{1.180248in}}%
\pgfpathcurveto{\pgfqpoint{2.691386in}{1.191298in}}{\pgfqpoint{2.686996in}{1.201897in}}{\pgfqpoint{2.679182in}{1.209711in}}%
\pgfpathcurveto{\pgfqpoint{2.671369in}{1.217524in}}{\pgfqpoint{2.660770in}{1.221915in}}{\pgfqpoint{2.649720in}{1.221915in}}%
\pgfpathcurveto{\pgfqpoint{2.638669in}{1.221915in}}{\pgfqpoint{2.628070in}{1.217524in}}{\pgfqpoint{2.620257in}{1.209711in}}%
\pgfpathcurveto{\pgfqpoint{2.612443in}{1.201897in}}{\pgfqpoint{2.608053in}{1.191298in}}{\pgfqpoint{2.608053in}{1.180248in}}%
\pgfpathcurveto{\pgfqpoint{2.608053in}{1.169198in}}{\pgfqpoint{2.612443in}{1.158599in}}{\pgfqpoint{2.620257in}{1.150785in}}%
\pgfpathcurveto{\pgfqpoint{2.628070in}{1.142972in}}{\pgfqpoint{2.638669in}{1.138581in}}{\pgfqpoint{2.649720in}{1.138581in}}%
\pgfpathclose%
\pgfusepath{stroke,fill}%
\end{pgfscope}%
\begin{pgfscope}%
\pgfpathrectangle{\pgfqpoint{1.016621in}{0.499691in}}{\pgfqpoint{3.875000in}{2.695000in}}%
\pgfusepath{clip}%
\pgfsetbuttcap%
\pgfsetroundjoin%
\definecolor{currentfill}{rgb}{0.117647,0.564706,1.000000}%
\pgfsetfillcolor{currentfill}%
\pgfsetlinewidth{1.003750pt}%
\definecolor{currentstroke}{rgb}{0.117647,0.564706,1.000000}%
\pgfsetstrokecolor{currentstroke}%
\pgfsetdash{}{0pt}%
\pgfpathmoveto{\pgfqpoint{2.720292in}{1.214587in}}%
\pgfpathcurveto{\pgfqpoint{2.731342in}{1.214587in}}{\pgfqpoint{2.741941in}{1.218977in}}{\pgfqpoint{2.749755in}{1.226791in}}%
\pgfpathcurveto{\pgfqpoint{2.757568in}{1.234605in}}{\pgfqpoint{2.761958in}{1.245204in}}{\pgfqpoint{2.761958in}{1.256254in}}%
\pgfpathcurveto{\pgfqpoint{2.761958in}{1.267304in}}{\pgfqpoint{2.757568in}{1.277903in}}{\pgfqpoint{2.749755in}{1.285717in}}%
\pgfpathcurveto{\pgfqpoint{2.741941in}{1.293530in}}{\pgfqpoint{2.731342in}{1.297920in}}{\pgfqpoint{2.720292in}{1.297920in}}%
\pgfpathcurveto{\pgfqpoint{2.709242in}{1.297920in}}{\pgfqpoint{2.698643in}{1.293530in}}{\pgfqpoint{2.690829in}{1.285717in}}%
\pgfpathcurveto{\pgfqpoint{2.683015in}{1.277903in}}{\pgfqpoint{2.678625in}{1.267304in}}{\pgfqpoint{2.678625in}{1.256254in}}%
\pgfpathcurveto{\pgfqpoint{2.678625in}{1.245204in}}{\pgfqpoint{2.683015in}{1.234605in}}{\pgfqpoint{2.690829in}{1.226791in}}%
\pgfpathcurveto{\pgfqpoint{2.698643in}{1.218977in}}{\pgfqpoint{2.709242in}{1.214587in}}{\pgfqpoint{2.720292in}{1.214587in}}%
\pgfpathclose%
\pgfusepath{stroke,fill}%
\end{pgfscope}%
\begin{pgfscope}%
\pgfpathrectangle{\pgfqpoint{1.016621in}{0.499691in}}{\pgfqpoint{3.875000in}{2.695000in}}%
\pgfusepath{clip}%
\pgfsetbuttcap%
\pgfsetroundjoin%
\definecolor{currentfill}{rgb}{0.117647,0.564706,1.000000}%
\pgfsetfillcolor{currentfill}%
\pgfsetlinewidth{1.003750pt}%
\definecolor{currentstroke}{rgb}{0.117647,0.564706,1.000000}%
\pgfsetstrokecolor{currentstroke}%
\pgfsetdash{}{0pt}%
\pgfpathmoveto{\pgfqpoint{2.743816in}{1.290593in}}%
\pgfpathcurveto{\pgfqpoint{2.754866in}{1.290593in}}{\pgfqpoint{2.765465in}{1.294983in}}{\pgfqpoint{2.773279in}{1.302797in}}%
\pgfpathcurveto{\pgfqpoint{2.781092in}{1.310610in}}{\pgfqpoint{2.785482in}{1.321209in}}{\pgfqpoint{2.785482in}{1.332260in}}%
\pgfpathcurveto{\pgfqpoint{2.785482in}{1.343310in}}{\pgfqpoint{2.781092in}{1.353909in}}{\pgfqpoint{2.773279in}{1.361722in}}%
\pgfpathcurveto{\pgfqpoint{2.765465in}{1.369536in}}{\pgfqpoint{2.754866in}{1.373926in}}{\pgfqpoint{2.743816in}{1.373926in}}%
\pgfpathcurveto{\pgfqpoint{2.732766in}{1.373926in}}{\pgfqpoint{2.722167in}{1.369536in}}{\pgfqpoint{2.714353in}{1.361722in}}%
\pgfpathcurveto{\pgfqpoint{2.706539in}{1.353909in}}{\pgfqpoint{2.702149in}{1.343310in}}{\pgfqpoint{2.702149in}{1.332260in}}%
\pgfpathcurveto{\pgfqpoint{2.702149in}{1.321209in}}{\pgfqpoint{2.706539in}{1.310610in}}{\pgfqpoint{2.714353in}{1.302797in}}%
\pgfpathcurveto{\pgfqpoint{2.722167in}{1.294983in}}{\pgfqpoint{2.732766in}{1.290593in}}{\pgfqpoint{2.743816in}{1.290593in}}%
\pgfpathclose%
\pgfusepath{stroke,fill}%
\end{pgfscope}%
\begin{pgfscope}%
\pgfpathrectangle{\pgfqpoint{1.016621in}{0.499691in}}{\pgfqpoint{3.875000in}{2.695000in}}%
\pgfusepath{clip}%
\pgfsetbuttcap%
\pgfsetroundjoin%
\definecolor{currentfill}{rgb}{0.117647,0.564706,1.000000}%
\pgfsetfillcolor{currentfill}%
\pgfsetlinewidth{1.003750pt}%
\definecolor{currentstroke}{rgb}{0.117647,0.564706,1.000000}%
\pgfsetstrokecolor{currentstroke}%
\pgfsetdash{}{0pt}%
\pgfpathmoveto{\pgfqpoint{2.773221in}{1.359243in}}%
\pgfpathcurveto{\pgfqpoint{2.784271in}{1.359243in}}{\pgfqpoint{2.794870in}{1.363634in}}{\pgfqpoint{2.802684in}{1.371447in}}%
\pgfpathcurveto{\pgfqpoint{2.810497in}{1.379261in}}{\pgfqpoint{2.814888in}{1.389860in}}{\pgfqpoint{2.814888in}{1.400910in}}%
\pgfpathcurveto{\pgfqpoint{2.814888in}{1.411960in}}{\pgfqpoint{2.810497in}{1.422559in}}{\pgfqpoint{2.802684in}{1.430373in}}%
\pgfpathcurveto{\pgfqpoint{2.794870in}{1.438186in}}{\pgfqpoint{2.784271in}{1.442577in}}{\pgfqpoint{2.773221in}{1.442577in}}%
\pgfpathcurveto{\pgfqpoint{2.762171in}{1.442577in}}{\pgfqpoint{2.751572in}{1.438186in}}{\pgfqpoint{2.743758in}{1.430373in}}%
\pgfpathcurveto{\pgfqpoint{2.735944in}{1.422559in}}{\pgfqpoint{2.731554in}{1.411960in}}{\pgfqpoint{2.731554in}{1.400910in}}%
\pgfpathcurveto{\pgfqpoint{2.731554in}{1.389860in}}{\pgfqpoint{2.735944in}{1.379261in}}{\pgfqpoint{2.743758in}{1.371447in}}%
\pgfpathcurveto{\pgfqpoint{2.751572in}{1.363634in}}{\pgfqpoint{2.762171in}{1.359243in}}{\pgfqpoint{2.773221in}{1.359243in}}%
\pgfpathclose%
\pgfusepath{stroke,fill}%
\end{pgfscope}%
\begin{pgfscope}%
\pgfpathrectangle{\pgfqpoint{1.016621in}{0.499691in}}{\pgfqpoint{3.875000in}{2.695000in}}%
\pgfusepath{clip}%
\pgfsetbuttcap%
\pgfsetroundjoin%
\definecolor{currentfill}{rgb}{0.117647,0.564706,1.000000}%
\pgfsetfillcolor{currentfill}%
\pgfsetlinewidth{1.003750pt}%
\definecolor{currentstroke}{rgb}{0.117647,0.564706,1.000000}%
\pgfsetstrokecolor{currentstroke}%
\pgfsetdash{}{0pt}%
\pgfpathmoveto{\pgfqpoint{2.796745in}{1.523514in}}%
\pgfpathcurveto{\pgfqpoint{2.807795in}{1.523514in}}{\pgfqpoint{2.818394in}{1.527904in}}{\pgfqpoint{2.826208in}{1.535718in}}%
\pgfpathcurveto{\pgfqpoint{2.834021in}{1.543531in}}{\pgfqpoint{2.838412in}{1.554130in}}{\pgfqpoint{2.838412in}{1.565181in}}%
\pgfpathcurveto{\pgfqpoint{2.838412in}{1.576231in}}{\pgfqpoint{2.834021in}{1.586830in}}{\pgfqpoint{2.826208in}{1.594643in}}%
\pgfpathcurveto{\pgfqpoint{2.818394in}{1.602457in}}{\pgfqpoint{2.807795in}{1.606847in}}{\pgfqpoint{2.796745in}{1.606847in}}%
\pgfpathcurveto{\pgfqpoint{2.785695in}{1.606847in}}{\pgfqpoint{2.775096in}{1.602457in}}{\pgfqpoint{2.767282in}{1.594643in}}%
\pgfpathcurveto{\pgfqpoint{2.759468in}{1.586830in}}{\pgfqpoint{2.755078in}{1.576231in}}{\pgfqpoint{2.755078in}{1.565181in}}%
\pgfpathcurveto{\pgfqpoint{2.755078in}{1.554130in}}{\pgfqpoint{2.759468in}{1.543531in}}{\pgfqpoint{2.767282in}{1.535718in}}%
\pgfpathcurveto{\pgfqpoint{2.775096in}{1.527904in}}{\pgfqpoint{2.785695in}{1.523514in}}{\pgfqpoint{2.796745in}{1.523514in}}%
\pgfpathclose%
\pgfusepath{stroke,fill}%
\end{pgfscope}%
\begin{pgfscope}%
\pgfpathrectangle{\pgfqpoint{1.016621in}{0.499691in}}{\pgfqpoint{3.875000in}{2.695000in}}%
\pgfusepath{clip}%
\pgfsetbuttcap%
\pgfsetroundjoin%
\definecolor{currentfill}{rgb}{0.117647,0.564706,1.000000}%
\pgfsetfillcolor{currentfill}%
\pgfsetlinewidth{1.003750pt}%
\definecolor{currentstroke}{rgb}{0.117647,0.564706,1.000000}%
\pgfsetstrokecolor{currentstroke}%
\pgfsetdash{}{0pt}%
\pgfpathmoveto{\pgfqpoint{2.814388in}{1.474478in}}%
\pgfpathcurveto{\pgfqpoint{2.825438in}{1.474478in}}{\pgfqpoint{2.836037in}{1.478868in}}{\pgfqpoint{2.843851in}{1.486682in}}%
\pgfpathcurveto{\pgfqpoint{2.851664in}{1.494495in}}{\pgfqpoint{2.856055in}{1.505094in}}{\pgfqpoint{2.856055in}{1.516145in}}%
\pgfpathcurveto{\pgfqpoint{2.856055in}{1.527195in}}{\pgfqpoint{2.851664in}{1.537794in}}{\pgfqpoint{2.843851in}{1.545607in}}%
\pgfpathcurveto{\pgfqpoint{2.836037in}{1.553421in}}{\pgfqpoint{2.825438in}{1.557811in}}{\pgfqpoint{2.814388in}{1.557811in}}%
\pgfpathcurveto{\pgfqpoint{2.803338in}{1.557811in}}{\pgfqpoint{2.792739in}{1.553421in}}{\pgfqpoint{2.784925in}{1.545607in}}%
\pgfpathcurveto{\pgfqpoint{2.777112in}{1.537794in}}{\pgfqpoint{2.772721in}{1.527195in}}{\pgfqpoint{2.772721in}{1.516145in}}%
\pgfpathcurveto{\pgfqpoint{2.772721in}{1.505094in}}{\pgfqpoint{2.777112in}{1.494495in}}{\pgfqpoint{2.784925in}{1.486682in}}%
\pgfpathcurveto{\pgfqpoint{2.792739in}{1.478868in}}{\pgfqpoint{2.803338in}{1.474478in}}{\pgfqpoint{2.814388in}{1.474478in}}%
\pgfpathclose%
\pgfusepath{stroke,fill}%
\end{pgfscope}%
\begin{pgfscope}%
\pgfpathrectangle{\pgfqpoint{1.016621in}{0.499691in}}{\pgfqpoint{3.875000in}{2.695000in}}%
\pgfusepath{clip}%
\pgfsetbuttcap%
\pgfsetroundjoin%
\definecolor{currentfill}{rgb}{0.117647,0.564706,1.000000}%
\pgfsetfillcolor{currentfill}%
\pgfsetlinewidth{1.003750pt}%
\definecolor{currentstroke}{rgb}{0.117647,0.564706,1.000000}%
\pgfsetstrokecolor{currentstroke}%
\pgfsetdash{}{0pt}%
\pgfpathmoveto{\pgfqpoint{2.832031in}{1.440153in}}%
\pgfpathcurveto{\pgfqpoint{2.843081in}{1.440153in}}{\pgfqpoint{2.853680in}{1.444543in}}{\pgfqpoint{2.861494in}{1.452357in}}%
\pgfpathcurveto{\pgfqpoint{2.869307in}{1.460170in}}{\pgfqpoint{2.873698in}{1.470769in}}{\pgfqpoint{2.873698in}{1.481819in}}%
\pgfpathcurveto{\pgfqpoint{2.873698in}{1.492870in}}{\pgfqpoint{2.869307in}{1.503469in}}{\pgfqpoint{2.861494in}{1.511282in}}%
\pgfpathcurveto{\pgfqpoint{2.853680in}{1.519096in}}{\pgfqpoint{2.843081in}{1.523486in}}{\pgfqpoint{2.832031in}{1.523486in}}%
\pgfpathcurveto{\pgfqpoint{2.820981in}{1.523486in}}{\pgfqpoint{2.810382in}{1.519096in}}{\pgfqpoint{2.802568in}{1.511282in}}%
\pgfpathcurveto{\pgfqpoint{2.794755in}{1.503469in}}{\pgfqpoint{2.790364in}{1.492870in}}{\pgfqpoint{2.790364in}{1.481819in}}%
\pgfpathcurveto{\pgfqpoint{2.790364in}{1.470769in}}{\pgfqpoint{2.794755in}{1.460170in}}{\pgfqpoint{2.802568in}{1.452357in}}%
\pgfpathcurveto{\pgfqpoint{2.810382in}{1.444543in}}{\pgfqpoint{2.820981in}{1.440153in}}{\pgfqpoint{2.832031in}{1.440153in}}%
\pgfpathclose%
\pgfusepath{stroke,fill}%
\end{pgfscope}%
\begin{pgfscope}%
\pgfpathrectangle{\pgfqpoint{1.016621in}{0.499691in}}{\pgfqpoint{3.875000in}{2.695000in}}%
\pgfusepath{clip}%
\pgfsetbuttcap%
\pgfsetroundjoin%
\definecolor{currentfill}{rgb}{0.117647,0.564706,1.000000}%
\pgfsetfillcolor{currentfill}%
\pgfsetlinewidth{1.003750pt}%
\definecolor{currentstroke}{rgb}{0.117647,0.564706,1.000000}%
\pgfsetstrokecolor{currentstroke}%
\pgfsetdash{}{0pt}%
\pgfpathmoveto{\pgfqpoint{2.849674in}{1.415635in}}%
\pgfpathcurveto{\pgfqpoint{2.860724in}{1.415635in}}{\pgfqpoint{2.871323in}{1.420025in}}{\pgfqpoint{2.879137in}{1.427839in}}%
\pgfpathcurveto{\pgfqpoint{2.886950in}{1.435652in}}{\pgfqpoint{2.891341in}{1.446251in}}{\pgfqpoint{2.891341in}{1.457301in}}%
\pgfpathcurveto{\pgfqpoint{2.891341in}{1.468352in}}{\pgfqpoint{2.886950in}{1.478951in}}{\pgfqpoint{2.879137in}{1.486764in}}%
\pgfpathcurveto{\pgfqpoint{2.871323in}{1.494578in}}{\pgfqpoint{2.860724in}{1.498968in}}{\pgfqpoint{2.849674in}{1.498968in}}%
\pgfpathcurveto{\pgfqpoint{2.838624in}{1.498968in}}{\pgfqpoint{2.828025in}{1.494578in}}{\pgfqpoint{2.820211in}{1.486764in}}%
\pgfpathcurveto{\pgfqpoint{2.812398in}{1.478951in}}{\pgfqpoint{2.808007in}{1.468352in}}{\pgfqpoint{2.808007in}{1.457301in}}%
\pgfpathcurveto{\pgfqpoint{2.808007in}{1.446251in}}{\pgfqpoint{2.812398in}{1.435652in}}{\pgfqpoint{2.820211in}{1.427839in}}%
\pgfpathcurveto{\pgfqpoint{2.828025in}{1.420025in}}{\pgfqpoint{2.838624in}{1.415635in}}{\pgfqpoint{2.849674in}{1.415635in}}%
\pgfpathclose%
\pgfusepath{stroke,fill}%
\end{pgfscope}%
\begin{pgfscope}%
\pgfpathrectangle{\pgfqpoint{1.016621in}{0.499691in}}{\pgfqpoint{3.875000in}{2.695000in}}%
\pgfusepath{clip}%
\pgfsetbuttcap%
\pgfsetroundjoin%
\definecolor{currentfill}{rgb}{0.117647,0.564706,1.000000}%
\pgfsetfillcolor{currentfill}%
\pgfsetlinewidth{1.003750pt}%
\definecolor{currentstroke}{rgb}{0.117647,0.564706,1.000000}%
\pgfsetstrokecolor{currentstroke}%
\pgfsetdash{}{0pt}%
\pgfpathmoveto{\pgfqpoint{2.867317in}{1.538225in}}%
\pgfpathcurveto{\pgfqpoint{2.878367in}{1.538225in}}{\pgfqpoint{2.888966in}{1.542615in}}{\pgfqpoint{2.896780in}{1.550429in}}%
\pgfpathcurveto{\pgfqpoint{2.904593in}{1.558242in}}{\pgfqpoint{2.908984in}{1.568841in}}{\pgfqpoint{2.908984in}{1.579891in}}%
\pgfpathcurveto{\pgfqpoint{2.908984in}{1.590942in}}{\pgfqpoint{2.904593in}{1.601541in}}{\pgfqpoint{2.896780in}{1.609354in}}%
\pgfpathcurveto{\pgfqpoint{2.888966in}{1.617168in}}{\pgfqpoint{2.878367in}{1.621558in}}{\pgfqpoint{2.867317in}{1.621558in}}%
\pgfpathcurveto{\pgfqpoint{2.856267in}{1.621558in}}{\pgfqpoint{2.845668in}{1.617168in}}{\pgfqpoint{2.837854in}{1.609354in}}%
\pgfpathcurveto{\pgfqpoint{2.830041in}{1.601541in}}{\pgfqpoint{2.825650in}{1.590942in}}{\pgfqpoint{2.825650in}{1.579891in}}%
\pgfpathcurveto{\pgfqpoint{2.825650in}{1.568841in}}{\pgfqpoint{2.830041in}{1.558242in}}{\pgfqpoint{2.837854in}{1.550429in}}%
\pgfpathcurveto{\pgfqpoint{2.845668in}{1.542615in}}{\pgfqpoint{2.856267in}{1.538225in}}{\pgfqpoint{2.867317in}{1.538225in}}%
\pgfpathclose%
\pgfusepath{stroke,fill}%
\end{pgfscope}%
\begin{pgfscope}%
\pgfpathrectangle{\pgfqpoint{1.016621in}{0.499691in}}{\pgfqpoint{3.875000in}{2.695000in}}%
\pgfusepath{clip}%
\pgfsetbuttcap%
\pgfsetroundjoin%
\definecolor{currentfill}{rgb}{0.117647,0.564706,1.000000}%
\pgfsetfillcolor{currentfill}%
\pgfsetlinewidth{1.003750pt}%
\definecolor{currentstroke}{rgb}{0.117647,0.564706,1.000000}%
\pgfsetstrokecolor{currentstroke}%
\pgfsetdash{}{0pt}%
\pgfpathmoveto{\pgfqpoint{2.884960in}{1.619134in}}%
\pgfpathcurveto{\pgfqpoint{2.896010in}{1.619134in}}{\pgfqpoint{2.906609in}{1.623524in}}{\pgfqpoint{2.914423in}{1.631338in}}%
\pgfpathcurveto{\pgfqpoint{2.922237in}{1.639152in}}{\pgfqpoint{2.926627in}{1.649751in}}{\pgfqpoint{2.926627in}{1.660801in}}%
\pgfpathcurveto{\pgfqpoint{2.926627in}{1.671851in}}{\pgfqpoint{2.922237in}{1.682450in}}{\pgfqpoint{2.914423in}{1.690264in}}%
\pgfpathcurveto{\pgfqpoint{2.906609in}{1.698077in}}{\pgfqpoint{2.896010in}{1.702467in}}{\pgfqpoint{2.884960in}{1.702467in}}%
\pgfpathcurveto{\pgfqpoint{2.873910in}{1.702467in}}{\pgfqpoint{2.863311in}{1.698077in}}{\pgfqpoint{2.855497in}{1.690264in}}%
\pgfpathcurveto{\pgfqpoint{2.847684in}{1.682450in}}{\pgfqpoint{2.843293in}{1.671851in}}{\pgfqpoint{2.843293in}{1.660801in}}%
\pgfpathcurveto{\pgfqpoint{2.843293in}{1.649751in}}{\pgfqpoint{2.847684in}{1.639152in}}{\pgfqpoint{2.855497in}{1.631338in}}%
\pgfpathcurveto{\pgfqpoint{2.863311in}{1.623524in}}{\pgfqpoint{2.873910in}{1.619134in}}{\pgfqpoint{2.884960in}{1.619134in}}%
\pgfpathclose%
\pgfusepath{stroke,fill}%
\end{pgfscope}%
\begin{pgfscope}%
\pgfpathrectangle{\pgfqpoint{1.016621in}{0.499691in}}{\pgfqpoint{3.875000in}{2.695000in}}%
\pgfusepath{clip}%
\pgfsetbuttcap%
\pgfsetroundjoin%
\definecolor{currentfill}{rgb}{0.117647,0.564706,1.000000}%
\pgfsetfillcolor{currentfill}%
\pgfsetlinewidth{1.003750pt}%
\definecolor{currentstroke}{rgb}{0.117647,0.564706,1.000000}%
\pgfsetstrokecolor{currentstroke}%
\pgfsetdash{}{0pt}%
\pgfpathmoveto{\pgfqpoint{2.902603in}{1.758887in}}%
\pgfpathcurveto{\pgfqpoint{2.913653in}{1.758887in}}{\pgfqpoint{2.924252in}{1.763277in}}{\pgfqpoint{2.932066in}{1.771091in}}%
\pgfpathcurveto{\pgfqpoint{2.939880in}{1.778904in}}{\pgfqpoint{2.944270in}{1.789503in}}{\pgfqpoint{2.944270in}{1.800553in}}%
\pgfpathcurveto{\pgfqpoint{2.944270in}{1.811604in}}{\pgfqpoint{2.939880in}{1.822203in}}{\pgfqpoint{2.932066in}{1.830016in}}%
\pgfpathcurveto{\pgfqpoint{2.924252in}{1.837830in}}{\pgfqpoint{2.913653in}{1.842220in}}{\pgfqpoint{2.902603in}{1.842220in}}%
\pgfpathcurveto{\pgfqpoint{2.891553in}{1.842220in}}{\pgfqpoint{2.880954in}{1.837830in}}{\pgfqpoint{2.873140in}{1.830016in}}%
\pgfpathcurveto{\pgfqpoint{2.865327in}{1.822203in}}{\pgfqpoint{2.860936in}{1.811604in}}{\pgfqpoint{2.860936in}{1.800553in}}%
\pgfpathcurveto{\pgfqpoint{2.860936in}{1.789503in}}{\pgfqpoint{2.865327in}{1.778904in}}{\pgfqpoint{2.873140in}{1.771091in}}%
\pgfpathcurveto{\pgfqpoint{2.880954in}{1.763277in}}{\pgfqpoint{2.891553in}{1.758887in}}{\pgfqpoint{2.902603in}{1.758887in}}%
\pgfpathclose%
\pgfusepath{stroke,fill}%
\end{pgfscope}%
\begin{pgfscope}%
\pgfpathrectangle{\pgfqpoint{1.016621in}{0.499691in}}{\pgfqpoint{3.875000in}{2.695000in}}%
\pgfusepath{clip}%
\pgfsetbuttcap%
\pgfsetroundjoin%
\definecolor{currentfill}{rgb}{0.117647,0.564706,1.000000}%
\pgfsetfillcolor{currentfill}%
\pgfsetlinewidth{1.003750pt}%
\definecolor{currentstroke}{rgb}{0.117647,0.564706,1.000000}%
\pgfsetstrokecolor{currentstroke}%
\pgfsetdash{}{0pt}%
\pgfpathmoveto{\pgfqpoint{2.914365in}{1.727013in}}%
\pgfpathcurveto{\pgfqpoint{2.925415in}{1.727013in}}{\pgfqpoint{2.936014in}{1.731404in}}{\pgfqpoint{2.943828in}{1.739217in}}%
\pgfpathcurveto{\pgfqpoint{2.951642in}{1.747031in}}{\pgfqpoint{2.956032in}{1.757630in}}{\pgfqpoint{2.956032in}{1.768680in}}%
\pgfpathcurveto{\pgfqpoint{2.956032in}{1.779730in}}{\pgfqpoint{2.951642in}{1.790329in}}{\pgfqpoint{2.943828in}{1.798143in}}%
\pgfpathcurveto{\pgfqpoint{2.936014in}{1.805956in}}{\pgfqpoint{2.925415in}{1.810347in}}{\pgfqpoint{2.914365in}{1.810347in}}%
\pgfpathcurveto{\pgfqpoint{2.903315in}{1.810347in}}{\pgfqpoint{2.892716in}{1.805956in}}{\pgfqpoint{2.884902in}{1.798143in}}%
\pgfpathcurveto{\pgfqpoint{2.877089in}{1.790329in}}{\pgfqpoint{2.872699in}{1.779730in}}{\pgfqpoint{2.872699in}{1.768680in}}%
\pgfpathcurveto{\pgfqpoint{2.872699in}{1.757630in}}{\pgfqpoint{2.877089in}{1.747031in}}{\pgfqpoint{2.884902in}{1.739217in}}%
\pgfpathcurveto{\pgfqpoint{2.892716in}{1.731404in}}{\pgfqpoint{2.903315in}{1.727013in}}{\pgfqpoint{2.914365in}{1.727013in}}%
\pgfpathclose%
\pgfusepath{stroke,fill}%
\end{pgfscope}%
\begin{pgfscope}%
\pgfpathrectangle{\pgfqpoint{1.016621in}{0.499691in}}{\pgfqpoint{3.875000in}{2.695000in}}%
\pgfusepath{clip}%
\pgfsetbuttcap%
\pgfsetroundjoin%
\definecolor{currentfill}{rgb}{0.117647,0.564706,1.000000}%
\pgfsetfillcolor{currentfill}%
\pgfsetlinewidth{1.003750pt}%
\definecolor{currentstroke}{rgb}{0.117647,0.564706,1.000000}%
\pgfsetstrokecolor{currentstroke}%
\pgfsetdash{}{0pt}%
\pgfpathmoveto{\pgfqpoint{2.926127in}{1.739272in}}%
\pgfpathcurveto{\pgfqpoint{2.937177in}{1.739272in}}{\pgfqpoint{2.947776in}{1.743663in}}{\pgfqpoint{2.955590in}{1.751476in}}%
\pgfpathcurveto{\pgfqpoint{2.963404in}{1.759290in}}{\pgfqpoint{2.967794in}{1.769889in}}{\pgfqpoint{2.967794in}{1.780939in}}%
\pgfpathcurveto{\pgfqpoint{2.967794in}{1.791989in}}{\pgfqpoint{2.963404in}{1.802588in}}{\pgfqpoint{2.955590in}{1.810402in}}%
\pgfpathcurveto{\pgfqpoint{2.947776in}{1.818215in}}{\pgfqpoint{2.937177in}{1.822606in}}{\pgfqpoint{2.926127in}{1.822606in}}%
\pgfpathcurveto{\pgfqpoint{2.915077in}{1.822606in}}{\pgfqpoint{2.904478in}{1.818215in}}{\pgfqpoint{2.896664in}{1.810402in}}%
\pgfpathcurveto{\pgfqpoint{2.888851in}{1.802588in}}{\pgfqpoint{2.884461in}{1.791989in}}{\pgfqpoint{2.884461in}{1.780939in}}%
\pgfpathcurveto{\pgfqpoint{2.884461in}{1.769889in}}{\pgfqpoint{2.888851in}{1.759290in}}{\pgfqpoint{2.896664in}{1.751476in}}%
\pgfpathcurveto{\pgfqpoint{2.904478in}{1.743663in}}{\pgfqpoint{2.915077in}{1.739272in}}{\pgfqpoint{2.926127in}{1.739272in}}%
\pgfpathclose%
\pgfusepath{stroke,fill}%
\end{pgfscope}%
\begin{pgfscope}%
\pgfpathrectangle{\pgfqpoint{1.016621in}{0.499691in}}{\pgfqpoint{3.875000in}{2.695000in}}%
\pgfusepath{clip}%
\pgfsetbuttcap%
\pgfsetroundjoin%
\definecolor{currentfill}{rgb}{0.117647,0.564706,1.000000}%
\pgfsetfillcolor{currentfill}%
\pgfsetlinewidth{1.003750pt}%
\definecolor{currentstroke}{rgb}{0.117647,0.564706,1.000000}%
\pgfsetstrokecolor{currentstroke}%
\pgfsetdash{}{0pt}%
\pgfpathmoveto{\pgfqpoint{2.937889in}{1.761339in}}%
\pgfpathcurveto{\pgfqpoint{2.948939in}{1.761339in}}{\pgfqpoint{2.959538in}{1.765729in}}{\pgfqpoint{2.967352in}{1.773542in}}%
\pgfpathcurveto{\pgfqpoint{2.975166in}{1.781356in}}{\pgfqpoint{2.979556in}{1.791955in}}{\pgfqpoint{2.979556in}{1.803005in}}%
\pgfpathcurveto{\pgfqpoint{2.979556in}{1.814055in}}{\pgfqpoint{2.975166in}{1.824654in}}{\pgfqpoint{2.967352in}{1.832468in}}%
\pgfpathcurveto{\pgfqpoint{2.959538in}{1.840282in}}{\pgfqpoint{2.948939in}{1.844672in}}{\pgfqpoint{2.937889in}{1.844672in}}%
\pgfpathcurveto{\pgfqpoint{2.926839in}{1.844672in}}{\pgfqpoint{2.916240in}{1.840282in}}{\pgfqpoint{2.908426in}{1.832468in}}%
\pgfpathcurveto{\pgfqpoint{2.900613in}{1.824654in}}{\pgfqpoint{2.896223in}{1.814055in}}{\pgfqpoint{2.896223in}{1.803005in}}%
\pgfpathcurveto{\pgfqpoint{2.896223in}{1.791955in}}{\pgfqpoint{2.900613in}{1.781356in}}{\pgfqpoint{2.908426in}{1.773542in}}%
\pgfpathcurveto{\pgfqpoint{2.916240in}{1.765729in}}{\pgfqpoint{2.926839in}{1.761339in}}{\pgfqpoint{2.937889in}{1.761339in}}%
\pgfpathclose%
\pgfusepath{stroke,fill}%
\end{pgfscope}%
\begin{pgfscope}%
\pgfpathrectangle{\pgfqpoint{1.016621in}{0.499691in}}{\pgfqpoint{3.875000in}{2.695000in}}%
\pgfusepath{clip}%
\pgfsetbuttcap%
\pgfsetroundjoin%
\definecolor{currentfill}{rgb}{0.117647,0.564706,1.000000}%
\pgfsetfillcolor{currentfill}%
\pgfsetlinewidth{1.003750pt}%
\definecolor{currentstroke}{rgb}{0.117647,0.564706,1.000000}%
\pgfsetstrokecolor{currentstroke}%
\pgfsetdash{}{0pt}%
\pgfpathmoveto{\pgfqpoint{2.949651in}{1.837344in}}%
\pgfpathcurveto{\pgfqpoint{2.960701in}{1.837344in}}{\pgfqpoint{2.971300in}{1.841735in}}{\pgfqpoint{2.979114in}{1.849548in}}%
\pgfpathcurveto{\pgfqpoint{2.986928in}{1.857362in}}{\pgfqpoint{2.991318in}{1.867961in}}{\pgfqpoint{2.991318in}{1.879011in}}%
\pgfpathcurveto{\pgfqpoint{2.991318in}{1.890061in}}{\pgfqpoint{2.986928in}{1.900660in}}{\pgfqpoint{2.979114in}{1.908474in}}%
\pgfpathcurveto{\pgfqpoint{2.971300in}{1.916287in}}{\pgfqpoint{2.960701in}{1.920678in}}{\pgfqpoint{2.949651in}{1.920678in}}%
\pgfpathcurveto{\pgfqpoint{2.938601in}{1.920678in}}{\pgfqpoint{2.928002in}{1.916287in}}{\pgfqpoint{2.920188in}{1.908474in}}%
\pgfpathcurveto{\pgfqpoint{2.912375in}{1.900660in}}{\pgfqpoint{2.907985in}{1.890061in}}{\pgfqpoint{2.907985in}{1.879011in}}%
\pgfpathcurveto{\pgfqpoint{2.907985in}{1.867961in}}{\pgfqpoint{2.912375in}{1.857362in}}{\pgfqpoint{2.920188in}{1.849548in}}%
\pgfpathcurveto{\pgfqpoint{2.928002in}{1.841735in}}{\pgfqpoint{2.938601in}{1.837344in}}{\pgfqpoint{2.949651in}{1.837344in}}%
\pgfpathclose%
\pgfusepath{stroke,fill}%
\end{pgfscope}%
\begin{pgfscope}%
\pgfpathrectangle{\pgfqpoint{1.016621in}{0.499691in}}{\pgfqpoint{3.875000in}{2.695000in}}%
\pgfusepath{clip}%
\pgfsetbuttcap%
\pgfsetroundjoin%
\definecolor{currentfill}{rgb}{0.117647,0.564706,1.000000}%
\pgfsetfillcolor{currentfill}%
\pgfsetlinewidth{1.003750pt}%
\definecolor{currentstroke}{rgb}{0.117647,0.564706,1.000000}%
\pgfsetstrokecolor{currentstroke}%
\pgfsetdash{}{0pt}%
\pgfpathmoveto{\pgfqpoint{2.961413in}{1.879025in}}%
\pgfpathcurveto{\pgfqpoint{2.972463in}{1.879025in}}{\pgfqpoint{2.983062in}{1.883415in}}{\pgfqpoint{2.990876in}{1.891229in}}%
\pgfpathcurveto{\pgfqpoint{2.998690in}{1.899042in}}{\pgfqpoint{3.003080in}{1.909641in}}{\pgfqpoint{3.003080in}{1.920692in}}%
\pgfpathcurveto{\pgfqpoint{3.003080in}{1.931742in}}{\pgfqpoint{2.998690in}{1.942341in}}{\pgfqpoint{2.990876in}{1.950154in}}%
\pgfpathcurveto{\pgfqpoint{2.983062in}{1.957968in}}{\pgfqpoint{2.972463in}{1.962358in}}{\pgfqpoint{2.961413in}{1.962358in}}%
\pgfpathcurveto{\pgfqpoint{2.950363in}{1.962358in}}{\pgfqpoint{2.939764in}{1.957968in}}{\pgfqpoint{2.931951in}{1.950154in}}%
\pgfpathcurveto{\pgfqpoint{2.924137in}{1.942341in}}{\pgfqpoint{2.919747in}{1.931742in}}{\pgfqpoint{2.919747in}{1.920692in}}%
\pgfpathcurveto{\pgfqpoint{2.919747in}{1.909641in}}{\pgfqpoint{2.924137in}{1.899042in}}{\pgfqpoint{2.931951in}{1.891229in}}%
\pgfpathcurveto{\pgfqpoint{2.939764in}{1.883415in}}{\pgfqpoint{2.950363in}{1.879025in}}{\pgfqpoint{2.961413in}{1.879025in}}%
\pgfpathclose%
\pgfusepath{stroke,fill}%
\end{pgfscope}%
\begin{pgfscope}%
\pgfpathrectangle{\pgfqpoint{1.016621in}{0.499691in}}{\pgfqpoint{3.875000in}{2.695000in}}%
\pgfusepath{clip}%
\pgfsetbuttcap%
\pgfsetroundjoin%
\definecolor{currentfill}{rgb}{0.117647,0.564706,1.000000}%
\pgfsetfillcolor{currentfill}%
\pgfsetlinewidth{1.003750pt}%
\definecolor{currentstroke}{rgb}{0.117647,0.564706,1.000000}%
\pgfsetstrokecolor{currentstroke}%
\pgfsetdash{}{0pt}%
\pgfpathmoveto{\pgfqpoint{2.973175in}{1.928061in}}%
\pgfpathcurveto{\pgfqpoint{2.984225in}{1.928061in}}{\pgfqpoint{2.994824in}{1.932451in}}{\pgfqpoint{3.002638in}{1.940265in}}%
\pgfpathcurveto{\pgfqpoint{3.010452in}{1.948078in}}{\pgfqpoint{3.014842in}{1.958677in}}{\pgfqpoint{3.014842in}{1.969728in}}%
\pgfpathcurveto{\pgfqpoint{3.014842in}{1.980778in}}{\pgfqpoint{3.010452in}{1.991377in}}{\pgfqpoint{3.002638in}{1.999190in}}%
\pgfpathcurveto{\pgfqpoint{2.994824in}{2.007004in}}{\pgfqpoint{2.984225in}{2.011394in}}{\pgfqpoint{2.973175in}{2.011394in}}%
\pgfpathcurveto{\pgfqpoint{2.962125in}{2.011394in}}{\pgfqpoint{2.951526in}{2.007004in}}{\pgfqpoint{2.943713in}{1.999190in}}%
\pgfpathcurveto{\pgfqpoint{2.935899in}{1.991377in}}{\pgfqpoint{2.931509in}{1.980778in}}{\pgfqpoint{2.931509in}{1.969728in}}%
\pgfpathcurveto{\pgfqpoint{2.931509in}{1.958677in}}{\pgfqpoint{2.935899in}{1.948078in}}{\pgfqpoint{2.943713in}{1.940265in}}%
\pgfpathcurveto{\pgfqpoint{2.951526in}{1.932451in}}{\pgfqpoint{2.962125in}{1.928061in}}{\pgfqpoint{2.973175in}{1.928061in}}%
\pgfpathclose%
\pgfusepath{stroke,fill}%
\end{pgfscope}%
\begin{pgfscope}%
\pgfpathrectangle{\pgfqpoint{1.016621in}{0.499691in}}{\pgfqpoint{3.875000in}{2.695000in}}%
\pgfusepath{clip}%
\pgfsetbuttcap%
\pgfsetroundjoin%
\definecolor{currentfill}{rgb}{0.117647,0.564706,1.000000}%
\pgfsetfillcolor{currentfill}%
\pgfsetlinewidth{1.003750pt}%
\definecolor{currentstroke}{rgb}{0.117647,0.564706,1.000000}%
\pgfsetstrokecolor{currentstroke}%
\pgfsetdash{}{0pt}%
\pgfpathmoveto{\pgfqpoint{2.984937in}{1.932965in}}%
\pgfpathcurveto{\pgfqpoint{2.995987in}{1.932965in}}{\pgfqpoint{3.006587in}{1.937355in}}{\pgfqpoint{3.014400in}{1.945168in}}%
\pgfpathcurveto{\pgfqpoint{3.022214in}{1.952982in}}{\pgfqpoint{3.026604in}{1.963581in}}{\pgfqpoint{3.026604in}{1.974631in}}%
\pgfpathcurveto{\pgfqpoint{3.026604in}{1.985681in}}{\pgfqpoint{3.022214in}{1.996280in}}{\pgfqpoint{3.014400in}{2.004094in}}%
\pgfpathcurveto{\pgfqpoint{3.006587in}{2.011908in}}{\pgfqpoint{2.995987in}{2.016298in}}{\pgfqpoint{2.984937in}{2.016298in}}%
\pgfpathcurveto{\pgfqpoint{2.973887in}{2.016298in}}{\pgfqpoint{2.963288in}{2.011908in}}{\pgfqpoint{2.955475in}{2.004094in}}%
\pgfpathcurveto{\pgfqpoint{2.947661in}{1.996280in}}{\pgfqpoint{2.943271in}{1.985681in}}{\pgfqpoint{2.943271in}{1.974631in}}%
\pgfpathcurveto{\pgfqpoint{2.943271in}{1.963581in}}{\pgfqpoint{2.947661in}{1.952982in}}{\pgfqpoint{2.955475in}{1.945168in}}%
\pgfpathcurveto{\pgfqpoint{2.963288in}{1.937355in}}{\pgfqpoint{2.973887in}{1.932965in}}{\pgfqpoint{2.984937in}{1.932965in}}%
\pgfpathclose%
\pgfusepath{stroke,fill}%
\end{pgfscope}%
\begin{pgfscope}%
\pgfpathrectangle{\pgfqpoint{1.016621in}{0.499691in}}{\pgfqpoint{3.875000in}{2.695000in}}%
\pgfusepath{clip}%
\pgfsetbuttcap%
\pgfsetroundjoin%
\definecolor{currentfill}{rgb}{0.117647,0.564706,1.000000}%
\pgfsetfillcolor{currentfill}%
\pgfsetlinewidth{1.003750pt}%
\definecolor{currentstroke}{rgb}{0.117647,0.564706,1.000000}%
\pgfsetstrokecolor{currentstroke}%
\pgfsetdash{}{0pt}%
\pgfpathmoveto{\pgfqpoint{2.996699in}{1.957483in}}%
\pgfpathcurveto{\pgfqpoint{3.007750in}{1.957483in}}{\pgfqpoint{3.018349in}{1.961873in}}{\pgfqpoint{3.026162in}{1.969686in}}%
\pgfpathcurveto{\pgfqpoint{3.033976in}{1.977500in}}{\pgfqpoint{3.038366in}{1.988099in}}{\pgfqpoint{3.038366in}{1.999149in}}%
\pgfpathcurveto{\pgfqpoint{3.038366in}{2.010199in}}{\pgfqpoint{3.033976in}{2.020798in}}{\pgfqpoint{3.026162in}{2.028612in}}%
\pgfpathcurveto{\pgfqpoint{3.018349in}{2.036426in}}{\pgfqpoint{3.007750in}{2.040816in}}{\pgfqpoint{2.996699in}{2.040816in}}%
\pgfpathcurveto{\pgfqpoint{2.985649in}{2.040816in}}{\pgfqpoint{2.975050in}{2.036426in}}{\pgfqpoint{2.967237in}{2.028612in}}%
\pgfpathcurveto{\pgfqpoint{2.959423in}{2.020798in}}{\pgfqpoint{2.955033in}{2.010199in}}{\pgfqpoint{2.955033in}{1.999149in}}%
\pgfpathcurveto{\pgfqpoint{2.955033in}{1.988099in}}{\pgfqpoint{2.959423in}{1.977500in}}{\pgfqpoint{2.967237in}{1.969686in}}%
\pgfpathcurveto{\pgfqpoint{2.975050in}{1.961873in}}{\pgfqpoint{2.985649in}{1.957483in}}{\pgfqpoint{2.996699in}{1.957483in}}%
\pgfpathclose%
\pgfusepath{stroke,fill}%
\end{pgfscope}%
\begin{pgfscope}%
\pgfpathrectangle{\pgfqpoint{1.016621in}{0.499691in}}{\pgfqpoint{3.875000in}{2.695000in}}%
\pgfusepath{clip}%
\pgfsetbuttcap%
\pgfsetroundjoin%
\definecolor{currentfill}{rgb}{0.117647,0.564706,1.000000}%
\pgfsetfillcolor{currentfill}%
\pgfsetlinewidth{1.003750pt}%
\definecolor{currentstroke}{rgb}{0.117647,0.564706,1.000000}%
\pgfsetstrokecolor{currentstroke}%
\pgfsetdash{}{0pt}%
\pgfpathmoveto{\pgfqpoint{3.002580in}{1.972193in}}%
\pgfpathcurveto{\pgfqpoint{3.013631in}{1.972193in}}{\pgfqpoint{3.024230in}{1.976584in}}{\pgfqpoint{3.032043in}{1.984397in}}%
\pgfpathcurveto{\pgfqpoint{3.039857in}{1.992211in}}{\pgfqpoint{3.044247in}{2.002810in}}{\pgfqpoint{3.044247in}{2.013860in}}%
\pgfpathcurveto{\pgfqpoint{3.044247in}{2.024910in}}{\pgfqpoint{3.039857in}{2.035509in}}{\pgfqpoint{3.032043in}{2.043323in}}%
\pgfpathcurveto{\pgfqpoint{3.024230in}{2.051136in}}{\pgfqpoint{3.013631in}{2.055527in}}{\pgfqpoint{3.002580in}{2.055527in}}%
\pgfpathcurveto{\pgfqpoint{2.991530in}{2.055527in}}{\pgfqpoint{2.980931in}{2.051136in}}{\pgfqpoint{2.973118in}{2.043323in}}%
\pgfpathcurveto{\pgfqpoint{2.965304in}{2.035509in}}{\pgfqpoint{2.960914in}{2.024910in}}{\pgfqpoint{2.960914in}{2.013860in}}%
\pgfpathcurveto{\pgfqpoint{2.960914in}{2.002810in}}{\pgfqpoint{2.965304in}{1.992211in}}{\pgfqpoint{2.973118in}{1.984397in}}%
\pgfpathcurveto{\pgfqpoint{2.980931in}{1.976584in}}{\pgfqpoint{2.991530in}{1.972193in}}{\pgfqpoint{3.002580in}{1.972193in}}%
\pgfpathclose%
\pgfusepath{stroke,fill}%
\end{pgfscope}%
\begin{pgfscope}%
\pgfpathrectangle{\pgfqpoint{1.016621in}{0.499691in}}{\pgfqpoint{3.875000in}{2.695000in}}%
\pgfusepath{clip}%
\pgfsetbuttcap%
\pgfsetroundjoin%
\definecolor{currentfill}{rgb}{0.117647,0.564706,1.000000}%
\pgfsetfillcolor{currentfill}%
\pgfsetlinewidth{1.003750pt}%
\definecolor{currentstroke}{rgb}{0.117647,0.564706,1.000000}%
\pgfsetstrokecolor{currentstroke}%
\pgfsetdash{}{0pt}%
\pgfpathmoveto{\pgfqpoint{3.014342in}{1.977097in}}%
\pgfpathcurveto{\pgfqpoint{3.025393in}{1.977097in}}{\pgfqpoint{3.035992in}{1.981487in}}{\pgfqpoint{3.043805in}{1.989301in}}%
\pgfpathcurveto{\pgfqpoint{3.051619in}{1.997114in}}{\pgfqpoint{3.056009in}{2.007713in}}{\pgfqpoint{3.056009in}{2.018764in}}%
\pgfpathcurveto{\pgfqpoint{3.056009in}{2.029814in}}{\pgfqpoint{3.051619in}{2.040413in}}{\pgfqpoint{3.043805in}{2.048226in}}%
\pgfpathcurveto{\pgfqpoint{3.035992in}{2.056040in}}{\pgfqpoint{3.025393in}{2.060430in}}{\pgfqpoint{3.014342in}{2.060430in}}%
\pgfpathcurveto{\pgfqpoint{3.003292in}{2.060430in}}{\pgfqpoint{2.992693in}{2.056040in}}{\pgfqpoint{2.984880in}{2.048226in}}%
\pgfpathcurveto{\pgfqpoint{2.977066in}{2.040413in}}{\pgfqpoint{2.972676in}{2.029814in}}{\pgfqpoint{2.972676in}{2.018764in}}%
\pgfpathcurveto{\pgfqpoint{2.972676in}{2.007713in}}{\pgfqpoint{2.977066in}{1.997114in}}{\pgfqpoint{2.984880in}{1.989301in}}%
\pgfpathcurveto{\pgfqpoint{2.992693in}{1.981487in}}{\pgfqpoint{3.003292in}{1.977097in}}{\pgfqpoint{3.014342in}{1.977097in}}%
\pgfpathclose%
\pgfusepath{stroke,fill}%
\end{pgfscope}%
\begin{pgfscope}%
\pgfpathrectangle{\pgfqpoint{1.016621in}{0.499691in}}{\pgfqpoint{3.875000in}{2.695000in}}%
\pgfusepath{clip}%
\pgfsetbuttcap%
\pgfsetroundjoin%
\definecolor{currentfill}{rgb}{0.117647,0.564706,1.000000}%
\pgfsetfillcolor{currentfill}%
\pgfsetlinewidth{1.003750pt}%
\definecolor{currentstroke}{rgb}{0.117647,0.564706,1.000000}%
\pgfsetstrokecolor{currentstroke}%
\pgfsetdash{}{0pt}%
\pgfpathmoveto{\pgfqpoint{3.026104in}{2.028585in}}%
\pgfpathcurveto{\pgfqpoint{3.037155in}{2.028585in}}{\pgfqpoint{3.047754in}{2.032975in}}{\pgfqpoint{3.055567in}{2.040789in}}%
\pgfpathcurveto{\pgfqpoint{3.063381in}{2.048602in}}{\pgfqpoint{3.067771in}{2.059201in}}{\pgfqpoint{3.067771in}{2.070251in}}%
\pgfpathcurveto{\pgfqpoint{3.067771in}{2.081302in}}{\pgfqpoint{3.063381in}{2.091901in}}{\pgfqpoint{3.055567in}{2.099714in}}%
\pgfpathcurveto{\pgfqpoint{3.047754in}{2.107528in}}{\pgfqpoint{3.037155in}{2.111918in}}{\pgfqpoint{3.026104in}{2.111918in}}%
\pgfpathcurveto{\pgfqpoint{3.015054in}{2.111918in}}{\pgfqpoint{3.004455in}{2.107528in}}{\pgfqpoint{2.996642in}{2.099714in}}%
\pgfpathcurveto{\pgfqpoint{2.988828in}{2.091901in}}{\pgfqpoint{2.984438in}{2.081302in}}{\pgfqpoint{2.984438in}{2.070251in}}%
\pgfpathcurveto{\pgfqpoint{2.984438in}{2.059201in}}{\pgfqpoint{2.988828in}{2.048602in}}{\pgfqpoint{2.996642in}{2.040789in}}%
\pgfpathcurveto{\pgfqpoint{3.004455in}{2.032975in}}{\pgfqpoint{3.015054in}{2.028585in}}{\pgfqpoint{3.026104in}{2.028585in}}%
\pgfpathclose%
\pgfusepath{stroke,fill}%
\end{pgfscope}%
\begin{pgfscope}%
\pgfpathrectangle{\pgfqpoint{1.016621in}{0.499691in}}{\pgfqpoint{3.875000in}{2.695000in}}%
\pgfusepath{clip}%
\pgfsetbuttcap%
\pgfsetroundjoin%
\definecolor{currentfill}{rgb}{0.117647,0.564706,1.000000}%
\pgfsetfillcolor{currentfill}%
\pgfsetlinewidth{1.003750pt}%
\definecolor{currentstroke}{rgb}{0.117647,0.564706,1.000000}%
\pgfsetstrokecolor{currentstroke}%
\pgfsetdash{}{0pt}%
\pgfpathmoveto{\pgfqpoint{3.031985in}{2.040844in}}%
\pgfpathcurveto{\pgfqpoint{3.043036in}{2.040844in}}{\pgfqpoint{3.053635in}{2.045234in}}{\pgfqpoint{3.061448in}{2.053048in}}%
\pgfpathcurveto{\pgfqpoint{3.069262in}{2.060861in}}{\pgfqpoint{3.073652in}{2.071460in}}{\pgfqpoint{3.073652in}{2.082510in}}%
\pgfpathcurveto{\pgfqpoint{3.073652in}{2.093561in}}{\pgfqpoint{3.069262in}{2.104160in}}{\pgfqpoint{3.061448in}{2.111973in}}%
\pgfpathcurveto{\pgfqpoint{3.053635in}{2.119787in}}{\pgfqpoint{3.043036in}{2.124177in}}{\pgfqpoint{3.031985in}{2.124177in}}%
\pgfpathcurveto{\pgfqpoint{3.020935in}{2.124177in}}{\pgfqpoint{3.010336in}{2.119787in}}{\pgfqpoint{3.002523in}{2.111973in}}%
\pgfpathcurveto{\pgfqpoint{2.994709in}{2.104160in}}{\pgfqpoint{2.990319in}{2.093561in}}{\pgfqpoint{2.990319in}{2.082510in}}%
\pgfpathcurveto{\pgfqpoint{2.990319in}{2.071460in}}{\pgfqpoint{2.994709in}{2.060861in}}{\pgfqpoint{3.002523in}{2.053048in}}%
\pgfpathcurveto{\pgfqpoint{3.010336in}{2.045234in}}{\pgfqpoint{3.020935in}{2.040844in}}{\pgfqpoint{3.031985in}{2.040844in}}%
\pgfpathclose%
\pgfusepath{stroke,fill}%
\end{pgfscope}%
\begin{pgfscope}%
\pgfpathrectangle{\pgfqpoint{1.016621in}{0.499691in}}{\pgfqpoint{3.875000in}{2.695000in}}%
\pgfusepath{clip}%
\pgfsetbuttcap%
\pgfsetroundjoin%
\definecolor{currentfill}{rgb}{0.117647,0.564706,1.000000}%
\pgfsetfillcolor{currentfill}%
\pgfsetlinewidth{1.003750pt}%
\definecolor{currentstroke}{rgb}{0.117647,0.564706,1.000000}%
\pgfsetstrokecolor{currentstroke}%
\pgfsetdash{}{0pt}%
\pgfpathmoveto{\pgfqpoint{3.037866in}{2.072717in}}%
\pgfpathcurveto{\pgfqpoint{3.048917in}{2.072717in}}{\pgfqpoint{3.059516in}{2.077107in}}{\pgfqpoint{3.067329in}{2.084921in}}%
\pgfpathcurveto{\pgfqpoint{3.075143in}{2.092735in}}{\pgfqpoint{3.079533in}{2.103334in}}{\pgfqpoint{3.079533in}{2.114384in}}%
\pgfpathcurveto{\pgfqpoint{3.079533in}{2.125434in}}{\pgfqpoint{3.075143in}{2.136033in}}{\pgfqpoint{3.067329in}{2.143847in}}%
\pgfpathcurveto{\pgfqpoint{3.059516in}{2.151660in}}{\pgfqpoint{3.048917in}{2.156050in}}{\pgfqpoint{3.037866in}{2.156050in}}%
\pgfpathcurveto{\pgfqpoint{3.026816in}{2.156050in}}{\pgfqpoint{3.016217in}{2.151660in}}{\pgfqpoint{3.008404in}{2.143847in}}%
\pgfpathcurveto{\pgfqpoint{3.000590in}{2.136033in}}{\pgfqpoint{2.996200in}{2.125434in}}{\pgfqpoint{2.996200in}{2.114384in}}%
\pgfpathcurveto{\pgfqpoint{2.996200in}{2.103334in}}{\pgfqpoint{3.000590in}{2.092735in}}{\pgfqpoint{3.008404in}{2.084921in}}%
\pgfpathcurveto{\pgfqpoint{3.016217in}{2.077107in}}{\pgfqpoint{3.026816in}{2.072717in}}{\pgfqpoint{3.037866in}{2.072717in}}%
\pgfpathclose%
\pgfusepath{stroke,fill}%
\end{pgfscope}%
\begin{pgfscope}%
\pgfpathrectangle{\pgfqpoint{1.016621in}{0.499691in}}{\pgfqpoint{3.875000in}{2.695000in}}%
\pgfusepath{clip}%
\pgfsetbuttcap%
\pgfsetroundjoin%
\definecolor{currentfill}{rgb}{0.117647,0.564706,1.000000}%
\pgfsetfillcolor{currentfill}%
\pgfsetlinewidth{1.003750pt}%
\definecolor{currentstroke}{rgb}{0.117647,0.564706,1.000000}%
\pgfsetstrokecolor{currentstroke}%
\pgfsetdash{}{0pt}%
\pgfpathmoveto{\pgfqpoint{3.049628in}{2.092332in}}%
\pgfpathcurveto{\pgfqpoint{3.060679in}{2.092332in}}{\pgfqpoint{3.071278in}{2.096722in}}{\pgfqpoint{3.079091in}{2.104535in}}%
\pgfpathcurveto{\pgfqpoint{3.086905in}{2.112349in}}{\pgfqpoint{3.091295in}{2.122948in}}{\pgfqpoint{3.091295in}{2.133998in}}%
\pgfpathcurveto{\pgfqpoint{3.091295in}{2.145048in}}{\pgfqpoint{3.086905in}{2.155647in}}{\pgfqpoint{3.079091in}{2.163461in}}%
\pgfpathcurveto{\pgfqpoint{3.071278in}{2.171275in}}{\pgfqpoint{3.060679in}{2.175665in}}{\pgfqpoint{3.049628in}{2.175665in}}%
\pgfpathcurveto{\pgfqpoint{3.038578in}{2.175665in}}{\pgfqpoint{3.027979in}{2.171275in}}{\pgfqpoint{3.020166in}{2.163461in}}%
\pgfpathcurveto{\pgfqpoint{3.012352in}{2.155647in}}{\pgfqpoint{3.007962in}{2.145048in}}{\pgfqpoint{3.007962in}{2.133998in}}%
\pgfpathcurveto{\pgfqpoint{3.007962in}{2.122948in}}{\pgfqpoint{3.012352in}{2.112349in}}{\pgfqpoint{3.020166in}{2.104535in}}%
\pgfpathcurveto{\pgfqpoint{3.027979in}{2.096722in}}{\pgfqpoint{3.038578in}{2.092332in}}{\pgfqpoint{3.049628in}{2.092332in}}%
\pgfpathclose%
\pgfusepath{stroke,fill}%
\end{pgfscope}%
\begin{pgfscope}%
\pgfpathrectangle{\pgfqpoint{1.016621in}{0.499691in}}{\pgfqpoint{3.875000in}{2.695000in}}%
\pgfusepath{clip}%
\pgfsetbuttcap%
\pgfsetroundjoin%
\definecolor{currentfill}{rgb}{0.117647,0.564706,1.000000}%
\pgfsetfillcolor{currentfill}%
\pgfsetlinewidth{1.003750pt}%
\definecolor{currentstroke}{rgb}{0.117647,0.564706,1.000000}%
\pgfsetstrokecolor{currentstroke}%
\pgfsetdash{}{0pt}%
\pgfpathmoveto{\pgfqpoint{3.055510in}{2.114398in}}%
\pgfpathcurveto{\pgfqpoint{3.066560in}{2.114398in}}{\pgfqpoint{3.077159in}{2.118788in}}{\pgfqpoint{3.084972in}{2.126602in}}%
\pgfpathcurveto{\pgfqpoint{3.092786in}{2.134415in}}{\pgfqpoint{3.097176in}{2.145014in}}{\pgfqpoint{3.097176in}{2.156064in}}%
\pgfpathcurveto{\pgfqpoint{3.097176in}{2.167115in}}{\pgfqpoint{3.092786in}{2.177714in}}{\pgfqpoint{3.084972in}{2.185527in}}%
\pgfpathcurveto{\pgfqpoint{3.077159in}{2.193341in}}{\pgfqpoint{3.066560in}{2.197731in}}{\pgfqpoint{3.055510in}{2.197731in}}%
\pgfpathcurveto{\pgfqpoint{3.044459in}{2.197731in}}{\pgfqpoint{3.033860in}{2.193341in}}{\pgfqpoint{3.026047in}{2.185527in}}%
\pgfpathcurveto{\pgfqpoint{3.018233in}{2.177714in}}{\pgfqpoint{3.013843in}{2.167115in}}{\pgfqpoint{3.013843in}{2.156064in}}%
\pgfpathcurveto{\pgfqpoint{3.013843in}{2.145014in}}{\pgfqpoint{3.018233in}{2.134415in}}{\pgfqpoint{3.026047in}{2.126602in}}%
\pgfpathcurveto{\pgfqpoint{3.033860in}{2.118788in}}{\pgfqpoint{3.044459in}{2.114398in}}{\pgfqpoint{3.055510in}{2.114398in}}%
\pgfpathclose%
\pgfusepath{stroke,fill}%
\end{pgfscope}%
\begin{pgfscope}%
\pgfpathrectangle{\pgfqpoint{1.016621in}{0.499691in}}{\pgfqpoint{3.875000in}{2.695000in}}%
\pgfusepath{clip}%
\pgfsetbuttcap%
\pgfsetroundjoin%
\definecolor{currentfill}{rgb}{0.117647,0.564706,1.000000}%
\pgfsetfillcolor{currentfill}%
\pgfsetlinewidth{1.003750pt}%
\definecolor{currentstroke}{rgb}{0.117647,0.564706,1.000000}%
\pgfsetstrokecolor{currentstroke}%
\pgfsetdash{}{0pt}%
\pgfpathmoveto{\pgfqpoint{3.067272in}{2.158530in}}%
\pgfpathcurveto{\pgfqpoint{3.078322in}{2.158530in}}{\pgfqpoint{3.088921in}{2.162920in}}{\pgfqpoint{3.096734in}{2.170734in}}%
\pgfpathcurveto{\pgfqpoint{3.104548in}{2.178548in}}{\pgfqpoint{3.108938in}{2.189147in}}{\pgfqpoint{3.108938in}{2.200197in}}%
\pgfpathcurveto{\pgfqpoint{3.108938in}{2.211247in}}{\pgfqpoint{3.104548in}{2.221846in}}{\pgfqpoint{3.096734in}{2.229660in}}%
\pgfpathcurveto{\pgfqpoint{3.088921in}{2.237473in}}{\pgfqpoint{3.078322in}{2.241863in}}{\pgfqpoint{3.067272in}{2.241863in}}%
\pgfpathcurveto{\pgfqpoint{3.056221in}{2.241863in}}{\pgfqpoint{3.045622in}{2.237473in}}{\pgfqpoint{3.037809in}{2.229660in}}%
\pgfpathcurveto{\pgfqpoint{3.029995in}{2.221846in}}{\pgfqpoint{3.025605in}{2.211247in}}{\pgfqpoint{3.025605in}{2.200197in}}%
\pgfpathcurveto{\pgfqpoint{3.025605in}{2.189147in}}{\pgfqpoint{3.029995in}{2.178548in}}{\pgfqpoint{3.037809in}{2.170734in}}%
\pgfpathcurveto{\pgfqpoint{3.045622in}{2.162920in}}{\pgfqpoint{3.056221in}{2.158530in}}{\pgfqpoint{3.067272in}{2.158530in}}%
\pgfpathclose%
\pgfusepath{stroke,fill}%
\end{pgfscope}%
\begin{pgfscope}%
\pgfpathrectangle{\pgfqpoint{1.016621in}{0.499691in}}{\pgfqpoint{3.875000in}{2.695000in}}%
\pgfusepath{clip}%
\pgfsetbuttcap%
\pgfsetroundjoin%
\definecolor{currentfill}{rgb}{0.117647,0.564706,1.000000}%
\pgfsetfillcolor{currentfill}%
\pgfsetlinewidth{1.003750pt}%
\definecolor{currentstroke}{rgb}{0.117647,0.564706,1.000000}%
\pgfsetstrokecolor{currentstroke}%
\pgfsetdash{}{0pt}%
\pgfpathmoveto{\pgfqpoint{3.079034in}{2.236988in}}%
\pgfpathcurveto{\pgfqpoint{3.090084in}{2.236988in}}{\pgfqpoint{3.100683in}{2.241378in}}{\pgfqpoint{3.108496in}{2.249192in}}%
\pgfpathcurveto{\pgfqpoint{3.116310in}{2.257005in}}{\pgfqpoint{3.120700in}{2.267604in}}{\pgfqpoint{3.120700in}{2.278654in}}%
\pgfpathcurveto{\pgfqpoint{3.120700in}{2.289705in}}{\pgfqpoint{3.116310in}{2.300304in}}{\pgfqpoint{3.108496in}{2.308117in}}%
\pgfpathcurveto{\pgfqpoint{3.100683in}{2.315931in}}{\pgfqpoint{3.090084in}{2.320321in}}{\pgfqpoint{3.079034in}{2.320321in}}%
\pgfpathcurveto{\pgfqpoint{3.067983in}{2.320321in}}{\pgfqpoint{3.057384in}{2.315931in}}{\pgfqpoint{3.049571in}{2.308117in}}%
\pgfpathcurveto{\pgfqpoint{3.041757in}{2.300304in}}{\pgfqpoint{3.037367in}{2.289705in}}{\pgfqpoint{3.037367in}{2.278654in}}%
\pgfpathcurveto{\pgfqpoint{3.037367in}{2.267604in}}{\pgfqpoint{3.041757in}{2.257005in}}{\pgfqpoint{3.049571in}{2.249192in}}%
\pgfpathcurveto{\pgfqpoint{3.057384in}{2.241378in}}{\pgfqpoint{3.067983in}{2.236988in}}{\pgfqpoint{3.079034in}{2.236988in}}%
\pgfpathclose%
\pgfusepath{stroke,fill}%
\end{pgfscope}%
\begin{pgfscope}%
\pgfpathrectangle{\pgfqpoint{1.016621in}{0.499691in}}{\pgfqpoint{3.875000in}{2.695000in}}%
\pgfusepath{clip}%
\pgfsetbuttcap%
\pgfsetroundjoin%
\definecolor{currentfill}{rgb}{0.117647,0.564706,1.000000}%
\pgfsetfillcolor{currentfill}%
\pgfsetlinewidth{1.003750pt}%
\definecolor{currentstroke}{rgb}{0.117647,0.564706,1.000000}%
\pgfsetstrokecolor{currentstroke}%
\pgfsetdash{}{0pt}%
\pgfpathmoveto{\pgfqpoint{3.090796in}{2.254150in}}%
\pgfpathcurveto{\pgfqpoint{3.101846in}{2.254150in}}{\pgfqpoint{3.112445in}{2.258541in}}{\pgfqpoint{3.120258in}{2.266354in}}%
\pgfpathcurveto{\pgfqpoint{3.128072in}{2.274168in}}{\pgfqpoint{3.132462in}{2.284767in}}{\pgfqpoint{3.132462in}{2.295817in}}%
\pgfpathcurveto{\pgfqpoint{3.132462in}{2.306867in}}{\pgfqpoint{3.128072in}{2.317466in}}{\pgfqpoint{3.120258in}{2.325280in}}%
\pgfpathcurveto{\pgfqpoint{3.112445in}{2.333093in}}{\pgfqpoint{3.101846in}{2.337484in}}{\pgfqpoint{3.090796in}{2.337484in}}%
\pgfpathcurveto{\pgfqpoint{3.079745in}{2.337484in}}{\pgfqpoint{3.069146in}{2.333093in}}{\pgfqpoint{3.061333in}{2.325280in}}%
\pgfpathcurveto{\pgfqpoint{3.053519in}{2.317466in}}{\pgfqpoint{3.049129in}{2.306867in}}{\pgfqpoint{3.049129in}{2.295817in}}%
\pgfpathcurveto{\pgfqpoint{3.049129in}{2.284767in}}{\pgfqpoint{3.053519in}{2.274168in}}{\pgfqpoint{3.061333in}{2.266354in}}%
\pgfpathcurveto{\pgfqpoint{3.069146in}{2.258541in}}{\pgfqpoint{3.079745in}{2.254150in}}{\pgfqpoint{3.090796in}{2.254150in}}%
\pgfpathclose%
\pgfusepath{stroke,fill}%
\end{pgfscope}%
\begin{pgfscope}%
\pgfpathrectangle{\pgfqpoint{1.016621in}{0.499691in}}{\pgfqpoint{3.875000in}{2.695000in}}%
\pgfusepath{clip}%
\pgfsetbuttcap%
\pgfsetroundjoin%
\definecolor{currentfill}{rgb}{0.117647,0.564706,1.000000}%
\pgfsetfillcolor{currentfill}%
\pgfsetlinewidth{1.003750pt}%
\definecolor{currentstroke}{rgb}{0.117647,0.564706,1.000000}%
\pgfsetstrokecolor{currentstroke}%
\pgfsetdash{}{0pt}%
\pgfpathmoveto{\pgfqpoint{3.120201in}{2.281120in}}%
\pgfpathcurveto{\pgfqpoint{3.131251in}{2.281120in}}{\pgfqpoint{3.141850in}{2.285510in}}{\pgfqpoint{3.149663in}{2.293324in}}%
\pgfpathcurveto{\pgfqpoint{3.157477in}{2.301138in}}{\pgfqpoint{3.161867in}{2.311737in}}{\pgfqpoint{3.161867in}{2.322787in}}%
\pgfpathcurveto{\pgfqpoint{3.161867in}{2.333837in}}{\pgfqpoint{3.157477in}{2.344436in}}{\pgfqpoint{3.149663in}{2.352250in}}%
\pgfpathcurveto{\pgfqpoint{3.141850in}{2.360063in}}{\pgfqpoint{3.131251in}{2.364453in}}{\pgfqpoint{3.120201in}{2.364453in}}%
\pgfpathcurveto{\pgfqpoint{3.109151in}{2.364453in}}{\pgfqpoint{3.098552in}{2.360063in}}{\pgfqpoint{3.090738in}{2.352250in}}%
\pgfpathcurveto{\pgfqpoint{3.082924in}{2.344436in}}{\pgfqpoint{3.078534in}{2.333837in}}{\pgfqpoint{3.078534in}{2.322787in}}%
\pgfpathcurveto{\pgfqpoint{3.078534in}{2.311737in}}{\pgfqpoint{3.082924in}{2.301138in}}{\pgfqpoint{3.090738in}{2.293324in}}%
\pgfpathcurveto{\pgfqpoint{3.098552in}{2.285510in}}{\pgfqpoint{3.109151in}{2.281120in}}{\pgfqpoint{3.120201in}{2.281120in}}%
\pgfpathclose%
\pgfusepath{stroke,fill}%
\end{pgfscope}%
\begin{pgfscope}%
\pgfpathrectangle{\pgfqpoint{1.016621in}{0.499691in}}{\pgfqpoint{3.875000in}{2.695000in}}%
\pgfusepath{clip}%
\pgfsetbuttcap%
\pgfsetroundjoin%
\definecolor{currentfill}{rgb}{0.117647,0.564706,1.000000}%
\pgfsetfillcolor{currentfill}%
\pgfsetlinewidth{1.003750pt}%
\definecolor{currentstroke}{rgb}{0.117647,0.564706,1.000000}%
\pgfsetstrokecolor{currentstroke}%
\pgfsetdash{}{0pt}%
\pgfpathmoveto{\pgfqpoint{3.143725in}{2.379192in}}%
\pgfpathcurveto{\pgfqpoint{3.154775in}{2.379192in}}{\pgfqpoint{3.165374in}{2.383582in}}{\pgfqpoint{3.173188in}{2.391396in}}%
\pgfpathcurveto{\pgfqpoint{3.181001in}{2.399210in}}{\pgfqpoint{3.185391in}{2.409809in}}{\pgfqpoint{3.185391in}{2.420859in}}%
\pgfpathcurveto{\pgfqpoint{3.185391in}{2.431909in}}{\pgfqpoint{3.181001in}{2.442508in}}{\pgfqpoint{3.173188in}{2.450322in}}%
\pgfpathcurveto{\pgfqpoint{3.165374in}{2.458135in}}{\pgfqpoint{3.154775in}{2.462525in}}{\pgfqpoint{3.143725in}{2.462525in}}%
\pgfpathcurveto{\pgfqpoint{3.132675in}{2.462525in}}{\pgfqpoint{3.122076in}{2.458135in}}{\pgfqpoint{3.114262in}{2.450322in}}%
\pgfpathcurveto{\pgfqpoint{3.106448in}{2.442508in}}{\pgfqpoint{3.102058in}{2.431909in}}{\pgfqpoint{3.102058in}{2.420859in}}%
\pgfpathcurveto{\pgfqpoint{3.102058in}{2.409809in}}{\pgfqpoint{3.106448in}{2.399210in}}{\pgfqpoint{3.114262in}{2.391396in}}%
\pgfpathcurveto{\pgfqpoint{3.122076in}{2.383582in}}{\pgfqpoint{3.132675in}{2.379192in}}{\pgfqpoint{3.143725in}{2.379192in}}%
\pgfpathclose%
\pgfusepath{stroke,fill}%
\end{pgfscope}%
\begin{pgfscope}%
\pgfpathrectangle{\pgfqpoint{1.016621in}{0.499691in}}{\pgfqpoint{3.875000in}{2.695000in}}%
\pgfusepath{clip}%
\pgfsetbuttcap%
\pgfsetroundjoin%
\definecolor{currentfill}{rgb}{0.117647,0.564706,1.000000}%
\pgfsetfillcolor{currentfill}%
\pgfsetlinewidth{1.003750pt}%
\definecolor{currentstroke}{rgb}{0.117647,0.564706,1.000000}%
\pgfsetstrokecolor{currentstroke}%
\pgfsetdash{}{0pt}%
\pgfpathmoveto{\pgfqpoint{3.226059in}{2.526300in}}%
\pgfpathcurveto{\pgfqpoint{3.237109in}{2.526300in}}{\pgfqpoint{3.247708in}{2.530690in}}{\pgfqpoint{3.255522in}{2.538504in}}%
\pgfpathcurveto{\pgfqpoint{3.263335in}{2.546318in}}{\pgfqpoint{3.267726in}{2.556917in}}{\pgfqpoint{3.267726in}{2.567967in}}%
\pgfpathcurveto{\pgfqpoint{3.267726in}{2.579017in}}{\pgfqpoint{3.263335in}{2.589616in}}{\pgfqpoint{3.255522in}{2.597430in}}%
\pgfpathcurveto{\pgfqpoint{3.247708in}{2.605243in}}{\pgfqpoint{3.237109in}{2.609633in}}{\pgfqpoint{3.226059in}{2.609633in}}%
\pgfpathcurveto{\pgfqpoint{3.215009in}{2.609633in}}{\pgfqpoint{3.204410in}{2.605243in}}{\pgfqpoint{3.196596in}{2.597430in}}%
\pgfpathcurveto{\pgfqpoint{3.188783in}{2.589616in}}{\pgfqpoint{3.184392in}{2.579017in}}{\pgfqpoint{3.184392in}{2.567967in}}%
\pgfpathcurveto{\pgfqpoint{3.184392in}{2.556917in}}{\pgfqpoint{3.188783in}{2.546318in}}{\pgfqpoint{3.196596in}{2.538504in}}%
\pgfpathcurveto{\pgfqpoint{3.204410in}{2.530690in}}{\pgfqpoint{3.215009in}{2.526300in}}{\pgfqpoint{3.226059in}{2.526300in}}%
\pgfpathclose%
\pgfusepath{stroke,fill}%
\end{pgfscope}%
\begin{pgfscope}%
\pgfpathrectangle{\pgfqpoint{1.016621in}{0.499691in}}{\pgfqpoint{3.875000in}{2.695000in}}%
\pgfusepath{clip}%
\pgfsetbuttcap%
\pgfsetroundjoin%
\definecolor{currentfill}{rgb}{0.117647,0.564706,1.000000}%
\pgfsetfillcolor{currentfill}%
\pgfsetlinewidth{1.003750pt}%
\definecolor{currentstroke}{rgb}{0.117647,0.564706,1.000000}%
\pgfsetstrokecolor{currentstroke}%
\pgfsetdash{}{0pt}%
\pgfpathmoveto{\pgfqpoint{3.373084in}{2.673408in}}%
\pgfpathcurveto{\pgfqpoint{3.384134in}{2.673408in}}{\pgfqpoint{3.394733in}{2.677798in}}{\pgfqpoint{3.402547in}{2.685612in}}%
\pgfpathcurveto{\pgfqpoint{3.410361in}{2.693426in}}{\pgfqpoint{3.414751in}{2.704025in}}{\pgfqpoint{3.414751in}{2.715075in}}%
\pgfpathcurveto{\pgfqpoint{3.414751in}{2.726125in}}{\pgfqpoint{3.410361in}{2.736724in}}{\pgfqpoint{3.402547in}{2.744538in}}%
\pgfpathcurveto{\pgfqpoint{3.394733in}{2.752351in}}{\pgfqpoint{3.384134in}{2.756741in}}{\pgfqpoint{3.373084in}{2.756741in}}%
\pgfpathcurveto{\pgfqpoint{3.362034in}{2.756741in}}{\pgfqpoint{3.351435in}{2.752351in}}{\pgfqpoint{3.343621in}{2.744538in}}%
\pgfpathcurveto{\pgfqpoint{3.335808in}{2.736724in}}{\pgfqpoint{3.331418in}{2.726125in}}{\pgfqpoint{3.331418in}{2.715075in}}%
\pgfpathcurveto{\pgfqpoint{3.331418in}{2.704025in}}{\pgfqpoint{3.335808in}{2.693426in}}{\pgfqpoint{3.343621in}{2.685612in}}%
\pgfpathcurveto{\pgfqpoint{3.351435in}{2.677798in}}{\pgfqpoint{3.362034in}{2.673408in}}{\pgfqpoint{3.373084in}{2.673408in}}%
\pgfpathclose%
\pgfusepath{stroke,fill}%
\end{pgfscope}%
\begin{pgfscope}%
\pgfsetbuttcap%
\pgfsetroundjoin%
\definecolor{currentfill}{rgb}{0.000000,0.000000,0.000000}%
\pgfsetfillcolor{currentfill}%
\pgfsetlinewidth{0.803000pt}%
\definecolor{currentstroke}{rgb}{0.000000,0.000000,0.000000}%
\pgfsetstrokecolor{currentstroke}%
\pgfsetdash{}{0pt}%
\pgfsys@defobject{currentmarker}{\pgfqpoint{0.000000in}{-0.048611in}}{\pgfqpoint{0.000000in}{0.000000in}}{%
\pgfpathmoveto{\pgfqpoint{0.000000in}{0.000000in}}%
\pgfpathlineto{\pgfqpoint{0.000000in}{-0.048611in}}%
\pgfusepath{stroke,fill}%
}%
\begin{pgfscope}%
\pgfsys@transformshift{1.120656in}{0.499691in}%
\pgfsys@useobject{currentmarker}{}%
\end{pgfscope}%
\end{pgfscope}%
\begin{pgfscope}%
\definecolor{textcolor}{rgb}{0.000000,0.000000,0.000000}%
\pgfsetstrokecolor{textcolor}%
\pgfsetfillcolor{textcolor}%
\pgftext[x=1.120656in,y=0.402469in,,top]{\color{textcolor}\rmfamily\fontsize{10.000000}{12.000000}\selectfont \(\displaystyle 0\)}%
\end{pgfscope}%
\begin{pgfscope}%
\pgfsetbuttcap%
\pgfsetroundjoin%
\definecolor{currentfill}{rgb}{0.000000,0.000000,0.000000}%
\pgfsetfillcolor{currentfill}%
\pgfsetlinewidth{0.803000pt}%
\definecolor{currentstroke}{rgb}{0.000000,0.000000,0.000000}%
\pgfsetstrokecolor{currentstroke}%
\pgfsetdash{}{0pt}%
\pgfsys@defobject{currentmarker}{\pgfqpoint{0.000000in}{-0.048611in}}{\pgfqpoint{0.000000in}{0.000000in}}{%
\pgfpathmoveto{\pgfqpoint{0.000000in}{0.000000in}}%
\pgfpathlineto{\pgfqpoint{0.000000in}{-0.048611in}}%
\pgfusepath{stroke,fill}%
}%
\begin{pgfscope}%
\pgfsys@transformshift{1.708757in}{0.499691in}%
\pgfsys@useobject{currentmarker}{}%
\end{pgfscope}%
\end{pgfscope}%
\begin{pgfscope}%
\definecolor{textcolor}{rgb}{0.000000,0.000000,0.000000}%
\pgfsetstrokecolor{textcolor}%
\pgfsetfillcolor{textcolor}%
\pgftext[x=1.708757in,y=0.402469in,,top]{\color{textcolor}\rmfamily\fontsize{10.000000}{12.000000}\selectfont \(\displaystyle 1\)}%
\end{pgfscope}%
\begin{pgfscope}%
\pgfsetbuttcap%
\pgfsetroundjoin%
\definecolor{currentfill}{rgb}{0.000000,0.000000,0.000000}%
\pgfsetfillcolor{currentfill}%
\pgfsetlinewidth{0.803000pt}%
\definecolor{currentstroke}{rgb}{0.000000,0.000000,0.000000}%
\pgfsetstrokecolor{currentstroke}%
\pgfsetdash{}{0pt}%
\pgfsys@defobject{currentmarker}{\pgfqpoint{0.000000in}{-0.048611in}}{\pgfqpoint{0.000000in}{0.000000in}}{%
\pgfpathmoveto{\pgfqpoint{0.000000in}{0.000000in}}%
\pgfpathlineto{\pgfqpoint{0.000000in}{-0.048611in}}%
\pgfusepath{stroke,fill}%
}%
\begin{pgfscope}%
\pgfsys@transformshift{2.296859in}{0.499691in}%
\pgfsys@useobject{currentmarker}{}%
\end{pgfscope}%
\end{pgfscope}%
\begin{pgfscope}%
\definecolor{textcolor}{rgb}{0.000000,0.000000,0.000000}%
\pgfsetstrokecolor{textcolor}%
\pgfsetfillcolor{textcolor}%
\pgftext[x=2.296859in,y=0.402469in,,top]{\color{textcolor}\rmfamily\fontsize{10.000000}{12.000000}\selectfont \(\displaystyle 2\)}%
\end{pgfscope}%
\begin{pgfscope}%
\pgfsetbuttcap%
\pgfsetroundjoin%
\definecolor{currentfill}{rgb}{0.000000,0.000000,0.000000}%
\pgfsetfillcolor{currentfill}%
\pgfsetlinewidth{0.803000pt}%
\definecolor{currentstroke}{rgb}{0.000000,0.000000,0.000000}%
\pgfsetstrokecolor{currentstroke}%
\pgfsetdash{}{0pt}%
\pgfsys@defobject{currentmarker}{\pgfqpoint{0.000000in}{-0.048611in}}{\pgfqpoint{0.000000in}{0.000000in}}{%
\pgfpathmoveto{\pgfqpoint{0.000000in}{0.000000in}}%
\pgfpathlineto{\pgfqpoint{0.000000in}{-0.048611in}}%
\pgfusepath{stroke,fill}%
}%
\begin{pgfscope}%
\pgfsys@transformshift{2.884960in}{0.499691in}%
\pgfsys@useobject{currentmarker}{}%
\end{pgfscope}%
\end{pgfscope}%
\begin{pgfscope}%
\definecolor{textcolor}{rgb}{0.000000,0.000000,0.000000}%
\pgfsetstrokecolor{textcolor}%
\pgfsetfillcolor{textcolor}%
\pgftext[x=2.884960in,y=0.402469in,,top]{\color{textcolor}\rmfamily\fontsize{10.000000}{12.000000}\selectfont \(\displaystyle 3\)}%
\end{pgfscope}%
\begin{pgfscope}%
\pgfsetbuttcap%
\pgfsetroundjoin%
\definecolor{currentfill}{rgb}{0.000000,0.000000,0.000000}%
\pgfsetfillcolor{currentfill}%
\pgfsetlinewidth{0.803000pt}%
\definecolor{currentstroke}{rgb}{0.000000,0.000000,0.000000}%
\pgfsetstrokecolor{currentstroke}%
\pgfsetdash{}{0pt}%
\pgfsys@defobject{currentmarker}{\pgfqpoint{0.000000in}{-0.048611in}}{\pgfqpoint{0.000000in}{0.000000in}}{%
\pgfpathmoveto{\pgfqpoint{0.000000in}{0.000000in}}%
\pgfpathlineto{\pgfqpoint{0.000000in}{-0.048611in}}%
\pgfusepath{stroke,fill}%
}%
\begin{pgfscope}%
\pgfsys@transformshift{3.473061in}{0.499691in}%
\pgfsys@useobject{currentmarker}{}%
\end{pgfscope}%
\end{pgfscope}%
\begin{pgfscope}%
\definecolor{textcolor}{rgb}{0.000000,0.000000,0.000000}%
\pgfsetstrokecolor{textcolor}%
\pgfsetfillcolor{textcolor}%
\pgftext[x=3.473061in,y=0.402469in,,top]{\color{textcolor}\rmfamily\fontsize{10.000000}{12.000000}\selectfont \(\displaystyle 4\)}%
\end{pgfscope}%
\begin{pgfscope}%
\pgfsetbuttcap%
\pgfsetroundjoin%
\definecolor{currentfill}{rgb}{0.000000,0.000000,0.000000}%
\pgfsetfillcolor{currentfill}%
\pgfsetlinewidth{0.803000pt}%
\definecolor{currentstroke}{rgb}{0.000000,0.000000,0.000000}%
\pgfsetstrokecolor{currentstroke}%
\pgfsetdash{}{0pt}%
\pgfsys@defobject{currentmarker}{\pgfqpoint{0.000000in}{-0.048611in}}{\pgfqpoint{0.000000in}{0.000000in}}{%
\pgfpathmoveto{\pgfqpoint{0.000000in}{0.000000in}}%
\pgfpathlineto{\pgfqpoint{0.000000in}{-0.048611in}}%
\pgfusepath{stroke,fill}%
}%
\begin{pgfscope}%
\pgfsys@transformshift{4.061163in}{0.499691in}%
\pgfsys@useobject{currentmarker}{}%
\end{pgfscope}%
\end{pgfscope}%
\begin{pgfscope}%
\definecolor{textcolor}{rgb}{0.000000,0.000000,0.000000}%
\pgfsetstrokecolor{textcolor}%
\pgfsetfillcolor{textcolor}%
\pgftext[x=4.061163in,y=0.402469in,,top]{\color{textcolor}\rmfamily\fontsize{10.000000}{12.000000}\selectfont \(\displaystyle 5\)}%
\end{pgfscope}%
\begin{pgfscope}%
\pgfsetbuttcap%
\pgfsetroundjoin%
\definecolor{currentfill}{rgb}{0.000000,0.000000,0.000000}%
\pgfsetfillcolor{currentfill}%
\pgfsetlinewidth{0.803000pt}%
\definecolor{currentstroke}{rgb}{0.000000,0.000000,0.000000}%
\pgfsetstrokecolor{currentstroke}%
\pgfsetdash{}{0pt}%
\pgfsys@defobject{currentmarker}{\pgfqpoint{0.000000in}{-0.048611in}}{\pgfqpoint{0.000000in}{0.000000in}}{%
\pgfpathmoveto{\pgfqpoint{0.000000in}{0.000000in}}%
\pgfpathlineto{\pgfqpoint{0.000000in}{-0.048611in}}%
\pgfusepath{stroke,fill}%
}%
\begin{pgfscope}%
\pgfsys@transformshift{4.649264in}{0.499691in}%
\pgfsys@useobject{currentmarker}{}%
\end{pgfscope}%
\end{pgfscope}%
\begin{pgfscope}%
\definecolor{textcolor}{rgb}{0.000000,0.000000,0.000000}%
\pgfsetstrokecolor{textcolor}%
\pgfsetfillcolor{textcolor}%
\pgftext[x=4.649264in,y=0.402469in,,top]{\color{textcolor}\rmfamily\fontsize{10.000000}{12.000000}\selectfont \(\displaystyle 6\)}%
\end{pgfscope}%
\begin{pgfscope}%
\definecolor{textcolor}{rgb}{0.000000,0.000000,0.000000}%
\pgfsetstrokecolor{textcolor}%
\pgfsetfillcolor{textcolor}%
\pgftext[x=2.954121in,y=0.223457in,,top]{\color{textcolor}\rmfamily\fontsize{10.000000}{12.000000}\selectfont log f}%
\end{pgfscope}%
\begin{pgfscope}%
\pgfsetbuttcap%
\pgfsetroundjoin%
\definecolor{currentfill}{rgb}{0.000000,0.000000,0.000000}%
\pgfsetfillcolor{currentfill}%
\pgfsetlinewidth{0.803000pt}%
\definecolor{currentstroke}{rgb}{0.000000,0.000000,0.000000}%
\pgfsetstrokecolor{currentstroke}%
\pgfsetdash{}{0pt}%
\pgfsys@defobject{currentmarker}{\pgfqpoint{-0.048611in}{0.000000in}}{\pgfqpoint{0.000000in}{0.000000in}}{%
\pgfpathmoveto{\pgfqpoint{0.000000in}{0.000000in}}%
\pgfpathlineto{\pgfqpoint{-0.048611in}{0.000000in}}%
\pgfusepath{stroke,fill}%
}%
\begin{pgfscope}%
\pgfsys@transformshift{1.016621in}{0.621238in}%
\pgfsys@useobject{currentmarker}{}%
\end{pgfscope}%
\end{pgfscope}%
\begin{pgfscope}%
\definecolor{textcolor}{rgb}{0.000000,0.000000,0.000000}%
\pgfsetstrokecolor{textcolor}%
\pgfsetfillcolor{textcolor}%
\pgftext[x=0.603040in,y=0.573012in,left,base]{\color{textcolor}\rmfamily\fontsize{10.000000}{12.000000}\selectfont \(\displaystyle -100\)}%
\end{pgfscope}%
\begin{pgfscope}%
\pgfsetbuttcap%
\pgfsetroundjoin%
\definecolor{currentfill}{rgb}{0.000000,0.000000,0.000000}%
\pgfsetfillcolor{currentfill}%
\pgfsetlinewidth{0.803000pt}%
\definecolor{currentstroke}{rgb}{0.000000,0.000000,0.000000}%
\pgfsetstrokecolor{currentstroke}%
\pgfsetdash{}{0pt}%
\pgfsys@defobject{currentmarker}{\pgfqpoint{-0.048611in}{0.000000in}}{\pgfqpoint{0.000000in}{0.000000in}}{%
\pgfpathmoveto{\pgfqpoint{0.000000in}{0.000000in}}%
\pgfpathlineto{\pgfqpoint{-0.048611in}{0.000000in}}%
\pgfusepath{stroke,fill}%
}%
\begin{pgfscope}%
\pgfsys@transformshift{1.016621in}{1.111598in}%
\pgfsys@useobject{currentmarker}{}%
\end{pgfscope}%
\end{pgfscope}%
\begin{pgfscope}%
\definecolor{textcolor}{rgb}{0.000000,0.000000,0.000000}%
\pgfsetstrokecolor{textcolor}%
\pgfsetfillcolor{textcolor}%
\pgftext[x=0.672484in,y=1.063372in,left,base]{\color{textcolor}\rmfamily\fontsize{10.000000}{12.000000}\selectfont \(\displaystyle -80\)}%
\end{pgfscope}%
\begin{pgfscope}%
\pgfsetbuttcap%
\pgfsetroundjoin%
\definecolor{currentfill}{rgb}{0.000000,0.000000,0.000000}%
\pgfsetfillcolor{currentfill}%
\pgfsetlinewidth{0.803000pt}%
\definecolor{currentstroke}{rgb}{0.000000,0.000000,0.000000}%
\pgfsetstrokecolor{currentstroke}%
\pgfsetdash{}{0pt}%
\pgfsys@defobject{currentmarker}{\pgfqpoint{-0.048611in}{0.000000in}}{\pgfqpoint{0.000000in}{0.000000in}}{%
\pgfpathmoveto{\pgfqpoint{0.000000in}{0.000000in}}%
\pgfpathlineto{\pgfqpoint{-0.048611in}{0.000000in}}%
\pgfusepath{stroke,fill}%
}%
\begin{pgfscope}%
\pgfsys@transformshift{1.016621in}{1.601958in}%
\pgfsys@useobject{currentmarker}{}%
\end{pgfscope}%
\end{pgfscope}%
\begin{pgfscope}%
\definecolor{textcolor}{rgb}{0.000000,0.000000,0.000000}%
\pgfsetstrokecolor{textcolor}%
\pgfsetfillcolor{textcolor}%
\pgftext[x=0.672484in,y=1.553732in,left,base]{\color{textcolor}\rmfamily\fontsize{10.000000}{12.000000}\selectfont \(\displaystyle -60\)}%
\end{pgfscope}%
\begin{pgfscope}%
\pgfsetbuttcap%
\pgfsetroundjoin%
\definecolor{currentfill}{rgb}{0.000000,0.000000,0.000000}%
\pgfsetfillcolor{currentfill}%
\pgfsetlinewidth{0.803000pt}%
\definecolor{currentstroke}{rgb}{0.000000,0.000000,0.000000}%
\pgfsetstrokecolor{currentstroke}%
\pgfsetdash{}{0pt}%
\pgfsys@defobject{currentmarker}{\pgfqpoint{-0.048611in}{0.000000in}}{\pgfqpoint{0.000000in}{0.000000in}}{%
\pgfpathmoveto{\pgfqpoint{0.000000in}{0.000000in}}%
\pgfpathlineto{\pgfqpoint{-0.048611in}{0.000000in}}%
\pgfusepath{stroke,fill}%
}%
\begin{pgfscope}%
\pgfsys@transformshift{1.016621in}{2.092318in}%
\pgfsys@useobject{currentmarker}{}%
\end{pgfscope}%
\end{pgfscope}%
\begin{pgfscope}%
\definecolor{textcolor}{rgb}{0.000000,0.000000,0.000000}%
\pgfsetstrokecolor{textcolor}%
\pgfsetfillcolor{textcolor}%
\pgftext[x=0.672484in,y=2.044092in,left,base]{\color{textcolor}\rmfamily\fontsize{10.000000}{12.000000}\selectfont \(\displaystyle -40\)}%
\end{pgfscope}%
\begin{pgfscope}%
\pgfsetbuttcap%
\pgfsetroundjoin%
\definecolor{currentfill}{rgb}{0.000000,0.000000,0.000000}%
\pgfsetfillcolor{currentfill}%
\pgfsetlinewidth{0.803000pt}%
\definecolor{currentstroke}{rgb}{0.000000,0.000000,0.000000}%
\pgfsetstrokecolor{currentstroke}%
\pgfsetdash{}{0pt}%
\pgfsys@defobject{currentmarker}{\pgfqpoint{-0.048611in}{0.000000in}}{\pgfqpoint{0.000000in}{0.000000in}}{%
\pgfpathmoveto{\pgfqpoint{0.000000in}{0.000000in}}%
\pgfpathlineto{\pgfqpoint{-0.048611in}{0.000000in}}%
\pgfusepath{stroke,fill}%
}%
\begin{pgfscope}%
\pgfsys@transformshift{1.016621in}{2.582678in}%
\pgfsys@useobject{currentmarker}{}%
\end{pgfscope}%
\end{pgfscope}%
\begin{pgfscope}%
\definecolor{textcolor}{rgb}{0.000000,0.000000,0.000000}%
\pgfsetstrokecolor{textcolor}%
\pgfsetfillcolor{textcolor}%
\pgftext[x=0.672484in,y=2.534452in,left,base]{\color{textcolor}\rmfamily\fontsize{10.000000}{12.000000}\selectfont \(\displaystyle -20\)}%
\end{pgfscope}%
\begin{pgfscope}%
\pgfsetbuttcap%
\pgfsetroundjoin%
\definecolor{currentfill}{rgb}{0.000000,0.000000,0.000000}%
\pgfsetfillcolor{currentfill}%
\pgfsetlinewidth{0.803000pt}%
\definecolor{currentstroke}{rgb}{0.000000,0.000000,0.000000}%
\pgfsetstrokecolor{currentstroke}%
\pgfsetdash{}{0pt}%
\pgfsys@defobject{currentmarker}{\pgfqpoint{-0.048611in}{0.000000in}}{\pgfqpoint{0.000000in}{0.000000in}}{%
\pgfpathmoveto{\pgfqpoint{0.000000in}{0.000000in}}%
\pgfpathlineto{\pgfqpoint{-0.048611in}{0.000000in}}%
\pgfusepath{stroke,fill}%
}%
\begin{pgfscope}%
\pgfsys@transformshift{1.016621in}{3.073038in}%
\pgfsys@useobject{currentmarker}{}%
\end{pgfscope}%
\end{pgfscope}%
\begin{pgfscope}%
\definecolor{textcolor}{rgb}{0.000000,0.000000,0.000000}%
\pgfsetstrokecolor{textcolor}%
\pgfsetfillcolor{textcolor}%
\pgftext[x=0.849954in,y=3.024812in,left,base]{\color{textcolor}\rmfamily\fontsize{10.000000}{12.000000}\selectfont \(\displaystyle 0\)}%
\end{pgfscope}%
\begin{pgfscope}%
\definecolor{textcolor}{rgb}{0.000000,0.000000,0.000000}%
\pgfsetstrokecolor{textcolor}%
\pgfsetfillcolor{textcolor}%
\pgftext[x=0.255817in,y=1.847191in,,bottom]{\color{textcolor}\rmfamily\fontsize{10.000000}{12.000000}\selectfont \(\displaystyle \phi\) \textit{(º)}}%
\end{pgfscope}%
\begin{pgfscope}%
\pgfpathrectangle{\pgfqpoint{1.016621in}{0.499691in}}{\pgfqpoint{3.875000in}{2.695000in}}%
\pgfusepath{clip}%
\pgfsetbuttcap%
\pgfsetroundjoin%
\definecolor{currentfill}{rgb}{0.121569,0.466667,0.705882}%
\pgfsetfillcolor{currentfill}%
\pgfsetlinewidth{1.003750pt}%
\definecolor{currentstroke}{rgb}{0.121569,0.466667,0.705882}%
\pgfsetstrokecolor{currentstroke}%
\pgfsetdash{}{0pt}%
\pgfpathmoveto{\pgfqpoint{2.814388in}{1.474478in}}%
\pgfpathcurveto{\pgfqpoint{2.825438in}{1.474478in}}{\pgfqpoint{2.836037in}{1.478868in}}{\pgfqpoint{2.843851in}{1.486682in}}%
\pgfpathcurveto{\pgfqpoint{2.851664in}{1.494495in}}{\pgfqpoint{2.856055in}{1.505094in}}{\pgfqpoint{2.856055in}{1.516145in}}%
\pgfpathcurveto{\pgfqpoint{2.856055in}{1.527195in}}{\pgfqpoint{2.851664in}{1.537794in}}{\pgfqpoint{2.843851in}{1.545607in}}%
\pgfpathcurveto{\pgfqpoint{2.836037in}{1.553421in}}{\pgfqpoint{2.825438in}{1.557811in}}{\pgfqpoint{2.814388in}{1.557811in}}%
\pgfpathcurveto{\pgfqpoint{2.803338in}{1.557811in}}{\pgfqpoint{2.792739in}{1.553421in}}{\pgfqpoint{2.784925in}{1.545607in}}%
\pgfpathcurveto{\pgfqpoint{2.777112in}{1.537794in}}{\pgfqpoint{2.772721in}{1.527195in}}{\pgfqpoint{2.772721in}{1.516145in}}%
\pgfpathcurveto{\pgfqpoint{2.772721in}{1.505094in}}{\pgfqpoint{2.777112in}{1.494495in}}{\pgfqpoint{2.784925in}{1.486682in}}%
\pgfpathcurveto{\pgfqpoint{2.792739in}{1.478868in}}{\pgfqpoint{2.803338in}{1.474478in}}{\pgfqpoint{2.814388in}{1.474478in}}%
\pgfpathclose%
\pgfusepath{stroke,fill}%
\end{pgfscope}%
\begin{pgfscope}%
\pgfpathrectangle{\pgfqpoint{1.016621in}{0.499691in}}{\pgfqpoint{3.875000in}{2.695000in}}%
\pgfusepath{clip}%
\pgfsetbuttcap%
\pgfsetroundjoin%
\definecolor{currentfill}{rgb}{0.121569,0.466667,0.705882}%
\pgfsetfillcolor{currentfill}%
\pgfsetlinewidth{1.003750pt}%
\definecolor{currentstroke}{rgb}{0.121569,0.466667,0.705882}%
\pgfsetstrokecolor{currentstroke}%
\pgfsetdash{}{0pt}%
\pgfpathmoveto{\pgfqpoint{2.832031in}{1.440153in}}%
\pgfpathcurveto{\pgfqpoint{2.843081in}{1.440153in}}{\pgfqpoint{2.853680in}{1.444543in}}{\pgfqpoint{2.861494in}{1.452357in}}%
\pgfpathcurveto{\pgfqpoint{2.869307in}{1.460170in}}{\pgfqpoint{2.873698in}{1.470769in}}{\pgfqpoint{2.873698in}{1.481819in}}%
\pgfpathcurveto{\pgfqpoint{2.873698in}{1.492870in}}{\pgfqpoint{2.869307in}{1.503469in}}{\pgfqpoint{2.861494in}{1.511282in}}%
\pgfpathcurveto{\pgfqpoint{2.853680in}{1.519096in}}{\pgfqpoint{2.843081in}{1.523486in}}{\pgfqpoint{2.832031in}{1.523486in}}%
\pgfpathcurveto{\pgfqpoint{2.820981in}{1.523486in}}{\pgfqpoint{2.810382in}{1.519096in}}{\pgfqpoint{2.802568in}{1.511282in}}%
\pgfpathcurveto{\pgfqpoint{2.794755in}{1.503469in}}{\pgfqpoint{2.790364in}{1.492870in}}{\pgfqpoint{2.790364in}{1.481819in}}%
\pgfpathcurveto{\pgfqpoint{2.790364in}{1.470769in}}{\pgfqpoint{2.794755in}{1.460170in}}{\pgfqpoint{2.802568in}{1.452357in}}%
\pgfpathcurveto{\pgfqpoint{2.810382in}{1.444543in}}{\pgfqpoint{2.820981in}{1.440153in}}{\pgfqpoint{2.832031in}{1.440153in}}%
\pgfpathclose%
\pgfusepath{stroke,fill}%
\end{pgfscope}%
\begin{pgfscope}%
\pgfpathrectangle{\pgfqpoint{1.016621in}{0.499691in}}{\pgfqpoint{3.875000in}{2.695000in}}%
\pgfusepath{clip}%
\pgfsetbuttcap%
\pgfsetroundjoin%
\definecolor{currentfill}{rgb}{0.121569,0.466667,0.705882}%
\pgfsetfillcolor{currentfill}%
\pgfsetlinewidth{1.003750pt}%
\definecolor{currentstroke}{rgb}{0.121569,0.466667,0.705882}%
\pgfsetstrokecolor{currentstroke}%
\pgfsetdash{}{0pt}%
\pgfpathmoveto{\pgfqpoint{2.849674in}{1.415635in}}%
\pgfpathcurveto{\pgfqpoint{2.860724in}{1.415635in}}{\pgfqpoint{2.871323in}{1.420025in}}{\pgfqpoint{2.879137in}{1.427839in}}%
\pgfpathcurveto{\pgfqpoint{2.886950in}{1.435652in}}{\pgfqpoint{2.891341in}{1.446251in}}{\pgfqpoint{2.891341in}{1.457301in}}%
\pgfpathcurveto{\pgfqpoint{2.891341in}{1.468352in}}{\pgfqpoint{2.886950in}{1.478951in}}{\pgfqpoint{2.879137in}{1.486764in}}%
\pgfpathcurveto{\pgfqpoint{2.871323in}{1.494578in}}{\pgfqpoint{2.860724in}{1.498968in}}{\pgfqpoint{2.849674in}{1.498968in}}%
\pgfpathcurveto{\pgfqpoint{2.838624in}{1.498968in}}{\pgfqpoint{2.828025in}{1.494578in}}{\pgfqpoint{2.820211in}{1.486764in}}%
\pgfpathcurveto{\pgfqpoint{2.812398in}{1.478951in}}{\pgfqpoint{2.808007in}{1.468352in}}{\pgfqpoint{2.808007in}{1.457301in}}%
\pgfpathcurveto{\pgfqpoint{2.808007in}{1.446251in}}{\pgfqpoint{2.812398in}{1.435652in}}{\pgfqpoint{2.820211in}{1.427839in}}%
\pgfpathcurveto{\pgfqpoint{2.828025in}{1.420025in}}{\pgfqpoint{2.838624in}{1.415635in}}{\pgfqpoint{2.849674in}{1.415635in}}%
\pgfpathclose%
\pgfusepath{stroke,fill}%
\end{pgfscope}%
\begin{pgfscope}%
\pgfpathrectangle{\pgfqpoint{1.016621in}{0.499691in}}{\pgfqpoint{3.875000in}{2.695000in}}%
\pgfusepath{clip}%
\pgfsetbuttcap%
\pgfsetroundjoin%
\definecolor{currentfill}{rgb}{0.121569,0.466667,0.705882}%
\pgfsetfillcolor{currentfill}%
\pgfsetlinewidth{1.003750pt}%
\definecolor{currentstroke}{rgb}{0.121569,0.466667,0.705882}%
\pgfsetstrokecolor{currentstroke}%
\pgfsetdash{}{0pt}%
\pgfpathmoveto{\pgfqpoint{2.867317in}{1.538225in}}%
\pgfpathcurveto{\pgfqpoint{2.878367in}{1.538225in}}{\pgfqpoint{2.888966in}{1.542615in}}{\pgfqpoint{2.896780in}{1.550429in}}%
\pgfpathcurveto{\pgfqpoint{2.904593in}{1.558242in}}{\pgfqpoint{2.908984in}{1.568841in}}{\pgfqpoint{2.908984in}{1.579891in}}%
\pgfpathcurveto{\pgfqpoint{2.908984in}{1.590942in}}{\pgfqpoint{2.904593in}{1.601541in}}{\pgfqpoint{2.896780in}{1.609354in}}%
\pgfpathcurveto{\pgfqpoint{2.888966in}{1.617168in}}{\pgfqpoint{2.878367in}{1.621558in}}{\pgfqpoint{2.867317in}{1.621558in}}%
\pgfpathcurveto{\pgfqpoint{2.856267in}{1.621558in}}{\pgfqpoint{2.845668in}{1.617168in}}{\pgfqpoint{2.837854in}{1.609354in}}%
\pgfpathcurveto{\pgfqpoint{2.830041in}{1.601541in}}{\pgfqpoint{2.825650in}{1.590942in}}{\pgfqpoint{2.825650in}{1.579891in}}%
\pgfpathcurveto{\pgfqpoint{2.825650in}{1.568841in}}{\pgfqpoint{2.830041in}{1.558242in}}{\pgfqpoint{2.837854in}{1.550429in}}%
\pgfpathcurveto{\pgfqpoint{2.845668in}{1.542615in}}{\pgfqpoint{2.856267in}{1.538225in}}{\pgfqpoint{2.867317in}{1.538225in}}%
\pgfpathclose%
\pgfusepath{stroke,fill}%
\end{pgfscope}%
\begin{pgfscope}%
\pgfpathrectangle{\pgfqpoint{1.016621in}{0.499691in}}{\pgfqpoint{3.875000in}{2.695000in}}%
\pgfusepath{clip}%
\pgfsetbuttcap%
\pgfsetroundjoin%
\definecolor{currentfill}{rgb}{0.121569,0.466667,0.705882}%
\pgfsetfillcolor{currentfill}%
\pgfsetlinewidth{1.003750pt}%
\definecolor{currentstroke}{rgb}{0.121569,0.466667,0.705882}%
\pgfsetstrokecolor{currentstroke}%
\pgfsetdash{}{0pt}%
\pgfpathmoveto{\pgfqpoint{2.884960in}{1.619134in}}%
\pgfpathcurveto{\pgfqpoint{2.896010in}{1.619134in}}{\pgfqpoint{2.906609in}{1.623524in}}{\pgfqpoint{2.914423in}{1.631338in}}%
\pgfpathcurveto{\pgfqpoint{2.922237in}{1.639152in}}{\pgfqpoint{2.926627in}{1.649751in}}{\pgfqpoint{2.926627in}{1.660801in}}%
\pgfpathcurveto{\pgfqpoint{2.926627in}{1.671851in}}{\pgfqpoint{2.922237in}{1.682450in}}{\pgfqpoint{2.914423in}{1.690264in}}%
\pgfpathcurveto{\pgfqpoint{2.906609in}{1.698077in}}{\pgfqpoint{2.896010in}{1.702467in}}{\pgfqpoint{2.884960in}{1.702467in}}%
\pgfpathcurveto{\pgfqpoint{2.873910in}{1.702467in}}{\pgfqpoint{2.863311in}{1.698077in}}{\pgfqpoint{2.855497in}{1.690264in}}%
\pgfpathcurveto{\pgfqpoint{2.847684in}{1.682450in}}{\pgfqpoint{2.843293in}{1.671851in}}{\pgfqpoint{2.843293in}{1.660801in}}%
\pgfpathcurveto{\pgfqpoint{2.843293in}{1.649751in}}{\pgfqpoint{2.847684in}{1.639152in}}{\pgfqpoint{2.855497in}{1.631338in}}%
\pgfpathcurveto{\pgfqpoint{2.863311in}{1.623524in}}{\pgfqpoint{2.873910in}{1.619134in}}{\pgfqpoint{2.884960in}{1.619134in}}%
\pgfpathclose%
\pgfusepath{stroke,fill}%
\end{pgfscope}%
\begin{pgfscope}%
\pgfpathrectangle{\pgfqpoint{1.016621in}{0.499691in}}{\pgfqpoint{3.875000in}{2.695000in}}%
\pgfusepath{clip}%
\pgfsetbuttcap%
\pgfsetroundjoin%
\definecolor{currentfill}{rgb}{0.121569,0.466667,0.705882}%
\pgfsetfillcolor{currentfill}%
\pgfsetlinewidth{1.003750pt}%
\definecolor{currentstroke}{rgb}{0.121569,0.466667,0.705882}%
\pgfsetstrokecolor{currentstroke}%
\pgfsetdash{}{0pt}%
\pgfpathmoveto{\pgfqpoint{2.902603in}{1.758887in}}%
\pgfpathcurveto{\pgfqpoint{2.913653in}{1.758887in}}{\pgfqpoint{2.924252in}{1.763277in}}{\pgfqpoint{2.932066in}{1.771091in}}%
\pgfpathcurveto{\pgfqpoint{2.939880in}{1.778904in}}{\pgfqpoint{2.944270in}{1.789503in}}{\pgfqpoint{2.944270in}{1.800553in}}%
\pgfpathcurveto{\pgfqpoint{2.944270in}{1.811604in}}{\pgfqpoint{2.939880in}{1.822203in}}{\pgfqpoint{2.932066in}{1.830016in}}%
\pgfpathcurveto{\pgfqpoint{2.924252in}{1.837830in}}{\pgfqpoint{2.913653in}{1.842220in}}{\pgfqpoint{2.902603in}{1.842220in}}%
\pgfpathcurveto{\pgfqpoint{2.891553in}{1.842220in}}{\pgfqpoint{2.880954in}{1.837830in}}{\pgfqpoint{2.873140in}{1.830016in}}%
\pgfpathcurveto{\pgfqpoint{2.865327in}{1.822203in}}{\pgfqpoint{2.860936in}{1.811604in}}{\pgfqpoint{2.860936in}{1.800553in}}%
\pgfpathcurveto{\pgfqpoint{2.860936in}{1.789503in}}{\pgfqpoint{2.865327in}{1.778904in}}{\pgfqpoint{2.873140in}{1.771091in}}%
\pgfpathcurveto{\pgfqpoint{2.880954in}{1.763277in}}{\pgfqpoint{2.891553in}{1.758887in}}{\pgfqpoint{2.902603in}{1.758887in}}%
\pgfpathclose%
\pgfusepath{stroke,fill}%
\end{pgfscope}%
\begin{pgfscope}%
\pgfpathrectangle{\pgfqpoint{1.016621in}{0.499691in}}{\pgfqpoint{3.875000in}{2.695000in}}%
\pgfusepath{clip}%
\pgfsetbuttcap%
\pgfsetroundjoin%
\definecolor{currentfill}{rgb}{0.121569,0.466667,0.705882}%
\pgfsetfillcolor{currentfill}%
\pgfsetlinewidth{1.003750pt}%
\definecolor{currentstroke}{rgb}{0.121569,0.466667,0.705882}%
\pgfsetstrokecolor{currentstroke}%
\pgfsetdash{}{0pt}%
\pgfpathmoveto{\pgfqpoint{2.914365in}{1.727013in}}%
\pgfpathcurveto{\pgfqpoint{2.925415in}{1.727013in}}{\pgfqpoint{2.936014in}{1.731404in}}{\pgfqpoint{2.943828in}{1.739217in}}%
\pgfpathcurveto{\pgfqpoint{2.951642in}{1.747031in}}{\pgfqpoint{2.956032in}{1.757630in}}{\pgfqpoint{2.956032in}{1.768680in}}%
\pgfpathcurveto{\pgfqpoint{2.956032in}{1.779730in}}{\pgfqpoint{2.951642in}{1.790329in}}{\pgfqpoint{2.943828in}{1.798143in}}%
\pgfpathcurveto{\pgfqpoint{2.936014in}{1.805956in}}{\pgfqpoint{2.925415in}{1.810347in}}{\pgfqpoint{2.914365in}{1.810347in}}%
\pgfpathcurveto{\pgfqpoint{2.903315in}{1.810347in}}{\pgfqpoint{2.892716in}{1.805956in}}{\pgfqpoint{2.884902in}{1.798143in}}%
\pgfpathcurveto{\pgfqpoint{2.877089in}{1.790329in}}{\pgfqpoint{2.872699in}{1.779730in}}{\pgfqpoint{2.872699in}{1.768680in}}%
\pgfpathcurveto{\pgfqpoint{2.872699in}{1.757630in}}{\pgfqpoint{2.877089in}{1.747031in}}{\pgfqpoint{2.884902in}{1.739217in}}%
\pgfpathcurveto{\pgfqpoint{2.892716in}{1.731404in}}{\pgfqpoint{2.903315in}{1.727013in}}{\pgfqpoint{2.914365in}{1.727013in}}%
\pgfpathclose%
\pgfusepath{stroke,fill}%
\end{pgfscope}%
\begin{pgfscope}%
\pgfpathrectangle{\pgfqpoint{1.016621in}{0.499691in}}{\pgfqpoint{3.875000in}{2.695000in}}%
\pgfusepath{clip}%
\pgfsetbuttcap%
\pgfsetroundjoin%
\definecolor{currentfill}{rgb}{0.121569,0.466667,0.705882}%
\pgfsetfillcolor{currentfill}%
\pgfsetlinewidth{1.003750pt}%
\definecolor{currentstroke}{rgb}{0.121569,0.466667,0.705882}%
\pgfsetstrokecolor{currentstroke}%
\pgfsetdash{}{0pt}%
\pgfpathmoveto{\pgfqpoint{2.926127in}{1.739272in}}%
\pgfpathcurveto{\pgfqpoint{2.937177in}{1.739272in}}{\pgfqpoint{2.947776in}{1.743663in}}{\pgfqpoint{2.955590in}{1.751476in}}%
\pgfpathcurveto{\pgfqpoint{2.963404in}{1.759290in}}{\pgfqpoint{2.967794in}{1.769889in}}{\pgfqpoint{2.967794in}{1.780939in}}%
\pgfpathcurveto{\pgfqpoint{2.967794in}{1.791989in}}{\pgfqpoint{2.963404in}{1.802588in}}{\pgfqpoint{2.955590in}{1.810402in}}%
\pgfpathcurveto{\pgfqpoint{2.947776in}{1.818215in}}{\pgfqpoint{2.937177in}{1.822606in}}{\pgfqpoint{2.926127in}{1.822606in}}%
\pgfpathcurveto{\pgfqpoint{2.915077in}{1.822606in}}{\pgfqpoint{2.904478in}{1.818215in}}{\pgfqpoint{2.896664in}{1.810402in}}%
\pgfpathcurveto{\pgfqpoint{2.888851in}{1.802588in}}{\pgfqpoint{2.884461in}{1.791989in}}{\pgfqpoint{2.884461in}{1.780939in}}%
\pgfpathcurveto{\pgfqpoint{2.884461in}{1.769889in}}{\pgfqpoint{2.888851in}{1.759290in}}{\pgfqpoint{2.896664in}{1.751476in}}%
\pgfpathcurveto{\pgfqpoint{2.904478in}{1.743663in}}{\pgfqpoint{2.915077in}{1.739272in}}{\pgfqpoint{2.926127in}{1.739272in}}%
\pgfpathclose%
\pgfusepath{stroke,fill}%
\end{pgfscope}%
\begin{pgfscope}%
\pgfpathrectangle{\pgfqpoint{1.016621in}{0.499691in}}{\pgfqpoint{3.875000in}{2.695000in}}%
\pgfusepath{clip}%
\pgfsetbuttcap%
\pgfsetroundjoin%
\definecolor{currentfill}{rgb}{0.121569,0.466667,0.705882}%
\pgfsetfillcolor{currentfill}%
\pgfsetlinewidth{1.003750pt}%
\definecolor{currentstroke}{rgb}{0.121569,0.466667,0.705882}%
\pgfsetstrokecolor{currentstroke}%
\pgfsetdash{}{0pt}%
\pgfpathmoveto{\pgfqpoint{2.937889in}{1.761339in}}%
\pgfpathcurveto{\pgfqpoint{2.948939in}{1.761339in}}{\pgfqpoint{2.959538in}{1.765729in}}{\pgfqpoint{2.967352in}{1.773542in}}%
\pgfpathcurveto{\pgfqpoint{2.975166in}{1.781356in}}{\pgfqpoint{2.979556in}{1.791955in}}{\pgfqpoint{2.979556in}{1.803005in}}%
\pgfpathcurveto{\pgfqpoint{2.979556in}{1.814055in}}{\pgfqpoint{2.975166in}{1.824654in}}{\pgfqpoint{2.967352in}{1.832468in}}%
\pgfpathcurveto{\pgfqpoint{2.959538in}{1.840282in}}{\pgfqpoint{2.948939in}{1.844672in}}{\pgfqpoint{2.937889in}{1.844672in}}%
\pgfpathcurveto{\pgfqpoint{2.926839in}{1.844672in}}{\pgfqpoint{2.916240in}{1.840282in}}{\pgfqpoint{2.908426in}{1.832468in}}%
\pgfpathcurveto{\pgfqpoint{2.900613in}{1.824654in}}{\pgfqpoint{2.896223in}{1.814055in}}{\pgfqpoint{2.896223in}{1.803005in}}%
\pgfpathcurveto{\pgfqpoint{2.896223in}{1.791955in}}{\pgfqpoint{2.900613in}{1.781356in}}{\pgfqpoint{2.908426in}{1.773542in}}%
\pgfpathcurveto{\pgfqpoint{2.916240in}{1.765729in}}{\pgfqpoint{2.926839in}{1.761339in}}{\pgfqpoint{2.937889in}{1.761339in}}%
\pgfpathclose%
\pgfusepath{stroke,fill}%
\end{pgfscope}%
\begin{pgfscope}%
\pgfpathrectangle{\pgfqpoint{1.016621in}{0.499691in}}{\pgfqpoint{3.875000in}{2.695000in}}%
\pgfusepath{clip}%
\pgfsetbuttcap%
\pgfsetroundjoin%
\definecolor{currentfill}{rgb}{0.121569,0.466667,0.705882}%
\pgfsetfillcolor{currentfill}%
\pgfsetlinewidth{1.003750pt}%
\definecolor{currentstroke}{rgb}{0.121569,0.466667,0.705882}%
\pgfsetstrokecolor{currentstroke}%
\pgfsetdash{}{0pt}%
\pgfpathmoveto{\pgfqpoint{2.949651in}{1.837344in}}%
\pgfpathcurveto{\pgfqpoint{2.960701in}{1.837344in}}{\pgfqpoint{2.971300in}{1.841735in}}{\pgfqpoint{2.979114in}{1.849548in}}%
\pgfpathcurveto{\pgfqpoint{2.986928in}{1.857362in}}{\pgfqpoint{2.991318in}{1.867961in}}{\pgfqpoint{2.991318in}{1.879011in}}%
\pgfpathcurveto{\pgfqpoint{2.991318in}{1.890061in}}{\pgfqpoint{2.986928in}{1.900660in}}{\pgfqpoint{2.979114in}{1.908474in}}%
\pgfpathcurveto{\pgfqpoint{2.971300in}{1.916287in}}{\pgfqpoint{2.960701in}{1.920678in}}{\pgfqpoint{2.949651in}{1.920678in}}%
\pgfpathcurveto{\pgfqpoint{2.938601in}{1.920678in}}{\pgfqpoint{2.928002in}{1.916287in}}{\pgfqpoint{2.920188in}{1.908474in}}%
\pgfpathcurveto{\pgfqpoint{2.912375in}{1.900660in}}{\pgfqpoint{2.907985in}{1.890061in}}{\pgfqpoint{2.907985in}{1.879011in}}%
\pgfpathcurveto{\pgfqpoint{2.907985in}{1.867961in}}{\pgfqpoint{2.912375in}{1.857362in}}{\pgfqpoint{2.920188in}{1.849548in}}%
\pgfpathcurveto{\pgfqpoint{2.928002in}{1.841735in}}{\pgfqpoint{2.938601in}{1.837344in}}{\pgfqpoint{2.949651in}{1.837344in}}%
\pgfpathclose%
\pgfusepath{stroke,fill}%
\end{pgfscope}%
\begin{pgfscope}%
\pgfpathrectangle{\pgfqpoint{1.016621in}{0.499691in}}{\pgfqpoint{3.875000in}{2.695000in}}%
\pgfusepath{clip}%
\pgfsetbuttcap%
\pgfsetroundjoin%
\definecolor{currentfill}{rgb}{0.121569,0.466667,0.705882}%
\pgfsetfillcolor{currentfill}%
\pgfsetlinewidth{1.003750pt}%
\definecolor{currentstroke}{rgb}{0.121569,0.466667,0.705882}%
\pgfsetstrokecolor{currentstroke}%
\pgfsetdash{}{0pt}%
\pgfpathmoveto{\pgfqpoint{2.961413in}{1.879025in}}%
\pgfpathcurveto{\pgfqpoint{2.972463in}{1.879025in}}{\pgfqpoint{2.983062in}{1.883415in}}{\pgfqpoint{2.990876in}{1.891229in}}%
\pgfpathcurveto{\pgfqpoint{2.998690in}{1.899042in}}{\pgfqpoint{3.003080in}{1.909641in}}{\pgfqpoint{3.003080in}{1.920692in}}%
\pgfpathcurveto{\pgfqpoint{3.003080in}{1.931742in}}{\pgfqpoint{2.998690in}{1.942341in}}{\pgfqpoint{2.990876in}{1.950154in}}%
\pgfpathcurveto{\pgfqpoint{2.983062in}{1.957968in}}{\pgfqpoint{2.972463in}{1.962358in}}{\pgfqpoint{2.961413in}{1.962358in}}%
\pgfpathcurveto{\pgfqpoint{2.950363in}{1.962358in}}{\pgfqpoint{2.939764in}{1.957968in}}{\pgfqpoint{2.931951in}{1.950154in}}%
\pgfpathcurveto{\pgfqpoint{2.924137in}{1.942341in}}{\pgfqpoint{2.919747in}{1.931742in}}{\pgfqpoint{2.919747in}{1.920692in}}%
\pgfpathcurveto{\pgfqpoint{2.919747in}{1.909641in}}{\pgfqpoint{2.924137in}{1.899042in}}{\pgfqpoint{2.931951in}{1.891229in}}%
\pgfpathcurveto{\pgfqpoint{2.939764in}{1.883415in}}{\pgfqpoint{2.950363in}{1.879025in}}{\pgfqpoint{2.961413in}{1.879025in}}%
\pgfpathclose%
\pgfusepath{stroke,fill}%
\end{pgfscope}%
\begin{pgfscope}%
\pgfpathrectangle{\pgfqpoint{1.016621in}{0.499691in}}{\pgfqpoint{3.875000in}{2.695000in}}%
\pgfusepath{clip}%
\pgfsetbuttcap%
\pgfsetroundjoin%
\definecolor{currentfill}{rgb}{0.121569,0.466667,0.705882}%
\pgfsetfillcolor{currentfill}%
\pgfsetlinewidth{1.003750pt}%
\definecolor{currentstroke}{rgb}{0.121569,0.466667,0.705882}%
\pgfsetstrokecolor{currentstroke}%
\pgfsetdash{}{0pt}%
\pgfpathmoveto{\pgfqpoint{2.973175in}{1.928061in}}%
\pgfpathcurveto{\pgfqpoint{2.984225in}{1.928061in}}{\pgfqpoint{2.994824in}{1.932451in}}{\pgfqpoint{3.002638in}{1.940265in}}%
\pgfpathcurveto{\pgfqpoint{3.010452in}{1.948078in}}{\pgfqpoint{3.014842in}{1.958677in}}{\pgfqpoint{3.014842in}{1.969728in}}%
\pgfpathcurveto{\pgfqpoint{3.014842in}{1.980778in}}{\pgfqpoint{3.010452in}{1.991377in}}{\pgfqpoint{3.002638in}{1.999190in}}%
\pgfpathcurveto{\pgfqpoint{2.994824in}{2.007004in}}{\pgfqpoint{2.984225in}{2.011394in}}{\pgfqpoint{2.973175in}{2.011394in}}%
\pgfpathcurveto{\pgfqpoint{2.962125in}{2.011394in}}{\pgfqpoint{2.951526in}{2.007004in}}{\pgfqpoint{2.943713in}{1.999190in}}%
\pgfpathcurveto{\pgfqpoint{2.935899in}{1.991377in}}{\pgfqpoint{2.931509in}{1.980778in}}{\pgfqpoint{2.931509in}{1.969728in}}%
\pgfpathcurveto{\pgfqpoint{2.931509in}{1.958677in}}{\pgfqpoint{2.935899in}{1.948078in}}{\pgfqpoint{2.943713in}{1.940265in}}%
\pgfpathcurveto{\pgfqpoint{2.951526in}{1.932451in}}{\pgfqpoint{2.962125in}{1.928061in}}{\pgfqpoint{2.973175in}{1.928061in}}%
\pgfpathclose%
\pgfusepath{stroke,fill}%
\end{pgfscope}%
\begin{pgfscope}%
\pgfpathrectangle{\pgfqpoint{1.016621in}{0.499691in}}{\pgfqpoint{3.875000in}{2.695000in}}%
\pgfusepath{clip}%
\pgfsetbuttcap%
\pgfsetroundjoin%
\definecolor{currentfill}{rgb}{0.121569,0.466667,0.705882}%
\pgfsetfillcolor{currentfill}%
\pgfsetlinewidth{1.003750pt}%
\definecolor{currentstroke}{rgb}{0.121569,0.466667,0.705882}%
\pgfsetstrokecolor{currentstroke}%
\pgfsetdash{}{0pt}%
\pgfpathmoveto{\pgfqpoint{2.984937in}{1.932965in}}%
\pgfpathcurveto{\pgfqpoint{2.995987in}{1.932965in}}{\pgfqpoint{3.006587in}{1.937355in}}{\pgfqpoint{3.014400in}{1.945168in}}%
\pgfpathcurveto{\pgfqpoint{3.022214in}{1.952982in}}{\pgfqpoint{3.026604in}{1.963581in}}{\pgfqpoint{3.026604in}{1.974631in}}%
\pgfpathcurveto{\pgfqpoint{3.026604in}{1.985681in}}{\pgfqpoint{3.022214in}{1.996280in}}{\pgfqpoint{3.014400in}{2.004094in}}%
\pgfpathcurveto{\pgfqpoint{3.006587in}{2.011908in}}{\pgfqpoint{2.995987in}{2.016298in}}{\pgfqpoint{2.984937in}{2.016298in}}%
\pgfpathcurveto{\pgfqpoint{2.973887in}{2.016298in}}{\pgfqpoint{2.963288in}{2.011908in}}{\pgfqpoint{2.955475in}{2.004094in}}%
\pgfpathcurveto{\pgfqpoint{2.947661in}{1.996280in}}{\pgfqpoint{2.943271in}{1.985681in}}{\pgfqpoint{2.943271in}{1.974631in}}%
\pgfpathcurveto{\pgfqpoint{2.943271in}{1.963581in}}{\pgfqpoint{2.947661in}{1.952982in}}{\pgfqpoint{2.955475in}{1.945168in}}%
\pgfpathcurveto{\pgfqpoint{2.963288in}{1.937355in}}{\pgfqpoint{2.973887in}{1.932965in}}{\pgfqpoint{2.984937in}{1.932965in}}%
\pgfpathclose%
\pgfusepath{stroke,fill}%
\end{pgfscope}%
\begin{pgfscope}%
\pgfpathrectangle{\pgfqpoint{1.016621in}{0.499691in}}{\pgfqpoint{3.875000in}{2.695000in}}%
\pgfusepath{clip}%
\pgfsetbuttcap%
\pgfsetroundjoin%
\definecolor{currentfill}{rgb}{0.121569,0.466667,0.705882}%
\pgfsetfillcolor{currentfill}%
\pgfsetlinewidth{1.003750pt}%
\definecolor{currentstroke}{rgb}{0.121569,0.466667,0.705882}%
\pgfsetstrokecolor{currentstroke}%
\pgfsetdash{}{0pt}%
\pgfpathmoveto{\pgfqpoint{2.996699in}{1.957483in}}%
\pgfpathcurveto{\pgfqpoint{3.007750in}{1.957483in}}{\pgfqpoint{3.018349in}{1.961873in}}{\pgfqpoint{3.026162in}{1.969686in}}%
\pgfpathcurveto{\pgfqpoint{3.033976in}{1.977500in}}{\pgfqpoint{3.038366in}{1.988099in}}{\pgfqpoint{3.038366in}{1.999149in}}%
\pgfpathcurveto{\pgfqpoint{3.038366in}{2.010199in}}{\pgfqpoint{3.033976in}{2.020798in}}{\pgfqpoint{3.026162in}{2.028612in}}%
\pgfpathcurveto{\pgfqpoint{3.018349in}{2.036426in}}{\pgfqpoint{3.007750in}{2.040816in}}{\pgfqpoint{2.996699in}{2.040816in}}%
\pgfpathcurveto{\pgfqpoint{2.985649in}{2.040816in}}{\pgfqpoint{2.975050in}{2.036426in}}{\pgfqpoint{2.967237in}{2.028612in}}%
\pgfpathcurveto{\pgfqpoint{2.959423in}{2.020798in}}{\pgfqpoint{2.955033in}{2.010199in}}{\pgfqpoint{2.955033in}{1.999149in}}%
\pgfpathcurveto{\pgfqpoint{2.955033in}{1.988099in}}{\pgfqpoint{2.959423in}{1.977500in}}{\pgfqpoint{2.967237in}{1.969686in}}%
\pgfpathcurveto{\pgfqpoint{2.975050in}{1.961873in}}{\pgfqpoint{2.985649in}{1.957483in}}{\pgfqpoint{2.996699in}{1.957483in}}%
\pgfpathclose%
\pgfusepath{stroke,fill}%
\end{pgfscope}%
\begin{pgfscope}%
\pgfpathrectangle{\pgfqpoint{1.016621in}{0.499691in}}{\pgfqpoint{3.875000in}{2.695000in}}%
\pgfusepath{clip}%
\pgfsetbuttcap%
\pgfsetroundjoin%
\definecolor{currentfill}{rgb}{0.121569,0.466667,0.705882}%
\pgfsetfillcolor{currentfill}%
\pgfsetlinewidth{1.003750pt}%
\definecolor{currentstroke}{rgb}{0.121569,0.466667,0.705882}%
\pgfsetstrokecolor{currentstroke}%
\pgfsetdash{}{0pt}%
\pgfpathmoveto{\pgfqpoint{3.002580in}{1.972193in}}%
\pgfpathcurveto{\pgfqpoint{3.013631in}{1.972193in}}{\pgfqpoint{3.024230in}{1.976584in}}{\pgfqpoint{3.032043in}{1.984397in}}%
\pgfpathcurveto{\pgfqpoint{3.039857in}{1.992211in}}{\pgfqpoint{3.044247in}{2.002810in}}{\pgfqpoint{3.044247in}{2.013860in}}%
\pgfpathcurveto{\pgfqpoint{3.044247in}{2.024910in}}{\pgfqpoint{3.039857in}{2.035509in}}{\pgfqpoint{3.032043in}{2.043323in}}%
\pgfpathcurveto{\pgfqpoint{3.024230in}{2.051136in}}{\pgfqpoint{3.013631in}{2.055527in}}{\pgfqpoint{3.002580in}{2.055527in}}%
\pgfpathcurveto{\pgfqpoint{2.991530in}{2.055527in}}{\pgfqpoint{2.980931in}{2.051136in}}{\pgfqpoint{2.973118in}{2.043323in}}%
\pgfpathcurveto{\pgfqpoint{2.965304in}{2.035509in}}{\pgfqpoint{2.960914in}{2.024910in}}{\pgfqpoint{2.960914in}{2.013860in}}%
\pgfpathcurveto{\pgfqpoint{2.960914in}{2.002810in}}{\pgfqpoint{2.965304in}{1.992211in}}{\pgfqpoint{2.973118in}{1.984397in}}%
\pgfpathcurveto{\pgfqpoint{2.980931in}{1.976584in}}{\pgfqpoint{2.991530in}{1.972193in}}{\pgfqpoint{3.002580in}{1.972193in}}%
\pgfpathclose%
\pgfusepath{stroke,fill}%
\end{pgfscope}%
\begin{pgfscope}%
\pgfpathrectangle{\pgfqpoint{1.016621in}{0.499691in}}{\pgfqpoint{3.875000in}{2.695000in}}%
\pgfusepath{clip}%
\pgfsetbuttcap%
\pgfsetroundjoin%
\definecolor{currentfill}{rgb}{0.121569,0.466667,0.705882}%
\pgfsetfillcolor{currentfill}%
\pgfsetlinewidth{1.003750pt}%
\definecolor{currentstroke}{rgb}{0.121569,0.466667,0.705882}%
\pgfsetstrokecolor{currentstroke}%
\pgfsetdash{}{0pt}%
\pgfpathmoveto{\pgfqpoint{3.014342in}{1.977097in}}%
\pgfpathcurveto{\pgfqpoint{3.025393in}{1.977097in}}{\pgfqpoint{3.035992in}{1.981487in}}{\pgfqpoint{3.043805in}{1.989301in}}%
\pgfpathcurveto{\pgfqpoint{3.051619in}{1.997114in}}{\pgfqpoint{3.056009in}{2.007713in}}{\pgfqpoint{3.056009in}{2.018764in}}%
\pgfpathcurveto{\pgfqpoint{3.056009in}{2.029814in}}{\pgfqpoint{3.051619in}{2.040413in}}{\pgfqpoint{3.043805in}{2.048226in}}%
\pgfpathcurveto{\pgfqpoint{3.035992in}{2.056040in}}{\pgfqpoint{3.025393in}{2.060430in}}{\pgfqpoint{3.014342in}{2.060430in}}%
\pgfpathcurveto{\pgfqpoint{3.003292in}{2.060430in}}{\pgfqpoint{2.992693in}{2.056040in}}{\pgfqpoint{2.984880in}{2.048226in}}%
\pgfpathcurveto{\pgfqpoint{2.977066in}{2.040413in}}{\pgfqpoint{2.972676in}{2.029814in}}{\pgfqpoint{2.972676in}{2.018764in}}%
\pgfpathcurveto{\pgfqpoint{2.972676in}{2.007713in}}{\pgfqpoint{2.977066in}{1.997114in}}{\pgfqpoint{2.984880in}{1.989301in}}%
\pgfpathcurveto{\pgfqpoint{2.992693in}{1.981487in}}{\pgfqpoint{3.003292in}{1.977097in}}{\pgfqpoint{3.014342in}{1.977097in}}%
\pgfpathclose%
\pgfusepath{stroke,fill}%
\end{pgfscope}%
\begin{pgfscope}%
\pgfpathrectangle{\pgfqpoint{1.016621in}{0.499691in}}{\pgfqpoint{3.875000in}{2.695000in}}%
\pgfusepath{clip}%
\pgfsetbuttcap%
\pgfsetroundjoin%
\definecolor{currentfill}{rgb}{0.121569,0.466667,0.705882}%
\pgfsetfillcolor{currentfill}%
\pgfsetlinewidth{1.003750pt}%
\definecolor{currentstroke}{rgb}{0.121569,0.466667,0.705882}%
\pgfsetstrokecolor{currentstroke}%
\pgfsetdash{}{0pt}%
\pgfpathmoveto{\pgfqpoint{3.026104in}{2.028585in}}%
\pgfpathcurveto{\pgfqpoint{3.037155in}{2.028585in}}{\pgfqpoint{3.047754in}{2.032975in}}{\pgfqpoint{3.055567in}{2.040789in}}%
\pgfpathcurveto{\pgfqpoint{3.063381in}{2.048602in}}{\pgfqpoint{3.067771in}{2.059201in}}{\pgfqpoint{3.067771in}{2.070251in}}%
\pgfpathcurveto{\pgfqpoint{3.067771in}{2.081302in}}{\pgfqpoint{3.063381in}{2.091901in}}{\pgfqpoint{3.055567in}{2.099714in}}%
\pgfpathcurveto{\pgfqpoint{3.047754in}{2.107528in}}{\pgfqpoint{3.037155in}{2.111918in}}{\pgfqpoint{3.026104in}{2.111918in}}%
\pgfpathcurveto{\pgfqpoint{3.015054in}{2.111918in}}{\pgfqpoint{3.004455in}{2.107528in}}{\pgfqpoint{2.996642in}{2.099714in}}%
\pgfpathcurveto{\pgfqpoint{2.988828in}{2.091901in}}{\pgfqpoint{2.984438in}{2.081302in}}{\pgfqpoint{2.984438in}{2.070251in}}%
\pgfpathcurveto{\pgfqpoint{2.984438in}{2.059201in}}{\pgfqpoint{2.988828in}{2.048602in}}{\pgfqpoint{2.996642in}{2.040789in}}%
\pgfpathcurveto{\pgfqpoint{3.004455in}{2.032975in}}{\pgfqpoint{3.015054in}{2.028585in}}{\pgfqpoint{3.026104in}{2.028585in}}%
\pgfpathclose%
\pgfusepath{stroke,fill}%
\end{pgfscope}%
\begin{pgfscope}%
\pgfpathrectangle{\pgfqpoint{1.016621in}{0.499691in}}{\pgfqpoint{3.875000in}{2.695000in}}%
\pgfusepath{clip}%
\pgfsetbuttcap%
\pgfsetroundjoin%
\definecolor{currentfill}{rgb}{0.121569,0.466667,0.705882}%
\pgfsetfillcolor{currentfill}%
\pgfsetlinewidth{1.003750pt}%
\definecolor{currentstroke}{rgb}{0.121569,0.466667,0.705882}%
\pgfsetstrokecolor{currentstroke}%
\pgfsetdash{}{0pt}%
\pgfpathmoveto{\pgfqpoint{3.031985in}{2.040844in}}%
\pgfpathcurveto{\pgfqpoint{3.043036in}{2.040844in}}{\pgfqpoint{3.053635in}{2.045234in}}{\pgfqpoint{3.061448in}{2.053048in}}%
\pgfpathcurveto{\pgfqpoint{3.069262in}{2.060861in}}{\pgfqpoint{3.073652in}{2.071460in}}{\pgfqpoint{3.073652in}{2.082510in}}%
\pgfpathcurveto{\pgfqpoint{3.073652in}{2.093561in}}{\pgfqpoint{3.069262in}{2.104160in}}{\pgfqpoint{3.061448in}{2.111973in}}%
\pgfpathcurveto{\pgfqpoint{3.053635in}{2.119787in}}{\pgfqpoint{3.043036in}{2.124177in}}{\pgfqpoint{3.031985in}{2.124177in}}%
\pgfpathcurveto{\pgfqpoint{3.020935in}{2.124177in}}{\pgfqpoint{3.010336in}{2.119787in}}{\pgfqpoint{3.002523in}{2.111973in}}%
\pgfpathcurveto{\pgfqpoint{2.994709in}{2.104160in}}{\pgfqpoint{2.990319in}{2.093561in}}{\pgfqpoint{2.990319in}{2.082510in}}%
\pgfpathcurveto{\pgfqpoint{2.990319in}{2.071460in}}{\pgfqpoint{2.994709in}{2.060861in}}{\pgfqpoint{3.002523in}{2.053048in}}%
\pgfpathcurveto{\pgfqpoint{3.010336in}{2.045234in}}{\pgfqpoint{3.020935in}{2.040844in}}{\pgfqpoint{3.031985in}{2.040844in}}%
\pgfpathclose%
\pgfusepath{stroke,fill}%
\end{pgfscope}%
\begin{pgfscope}%
\pgfpathrectangle{\pgfqpoint{1.016621in}{0.499691in}}{\pgfqpoint{3.875000in}{2.695000in}}%
\pgfusepath{clip}%
\pgfsetbuttcap%
\pgfsetroundjoin%
\definecolor{currentfill}{rgb}{0.121569,0.466667,0.705882}%
\pgfsetfillcolor{currentfill}%
\pgfsetlinewidth{1.003750pt}%
\definecolor{currentstroke}{rgb}{0.121569,0.466667,0.705882}%
\pgfsetstrokecolor{currentstroke}%
\pgfsetdash{}{0pt}%
\pgfpathmoveto{\pgfqpoint{3.037866in}{2.072717in}}%
\pgfpathcurveto{\pgfqpoint{3.048917in}{2.072717in}}{\pgfqpoint{3.059516in}{2.077107in}}{\pgfqpoint{3.067329in}{2.084921in}}%
\pgfpathcurveto{\pgfqpoint{3.075143in}{2.092735in}}{\pgfqpoint{3.079533in}{2.103334in}}{\pgfqpoint{3.079533in}{2.114384in}}%
\pgfpathcurveto{\pgfqpoint{3.079533in}{2.125434in}}{\pgfqpoint{3.075143in}{2.136033in}}{\pgfqpoint{3.067329in}{2.143847in}}%
\pgfpathcurveto{\pgfqpoint{3.059516in}{2.151660in}}{\pgfqpoint{3.048917in}{2.156050in}}{\pgfqpoint{3.037866in}{2.156050in}}%
\pgfpathcurveto{\pgfqpoint{3.026816in}{2.156050in}}{\pgfqpoint{3.016217in}{2.151660in}}{\pgfqpoint{3.008404in}{2.143847in}}%
\pgfpathcurveto{\pgfqpoint{3.000590in}{2.136033in}}{\pgfqpoint{2.996200in}{2.125434in}}{\pgfqpoint{2.996200in}{2.114384in}}%
\pgfpathcurveto{\pgfqpoint{2.996200in}{2.103334in}}{\pgfqpoint{3.000590in}{2.092735in}}{\pgfqpoint{3.008404in}{2.084921in}}%
\pgfpathcurveto{\pgfqpoint{3.016217in}{2.077107in}}{\pgfqpoint{3.026816in}{2.072717in}}{\pgfqpoint{3.037866in}{2.072717in}}%
\pgfpathclose%
\pgfusepath{stroke,fill}%
\end{pgfscope}%
\begin{pgfscope}%
\pgfpathrectangle{\pgfqpoint{1.016621in}{0.499691in}}{\pgfqpoint{3.875000in}{2.695000in}}%
\pgfusepath{clip}%
\pgfsetbuttcap%
\pgfsetroundjoin%
\definecolor{currentfill}{rgb}{0.121569,0.466667,0.705882}%
\pgfsetfillcolor{currentfill}%
\pgfsetlinewidth{1.003750pt}%
\definecolor{currentstroke}{rgb}{0.121569,0.466667,0.705882}%
\pgfsetstrokecolor{currentstroke}%
\pgfsetdash{}{0pt}%
\pgfpathmoveto{\pgfqpoint{3.049628in}{2.092332in}}%
\pgfpathcurveto{\pgfqpoint{3.060679in}{2.092332in}}{\pgfqpoint{3.071278in}{2.096722in}}{\pgfqpoint{3.079091in}{2.104535in}}%
\pgfpathcurveto{\pgfqpoint{3.086905in}{2.112349in}}{\pgfqpoint{3.091295in}{2.122948in}}{\pgfqpoint{3.091295in}{2.133998in}}%
\pgfpathcurveto{\pgfqpoint{3.091295in}{2.145048in}}{\pgfqpoint{3.086905in}{2.155647in}}{\pgfqpoint{3.079091in}{2.163461in}}%
\pgfpathcurveto{\pgfqpoint{3.071278in}{2.171275in}}{\pgfqpoint{3.060679in}{2.175665in}}{\pgfqpoint{3.049628in}{2.175665in}}%
\pgfpathcurveto{\pgfqpoint{3.038578in}{2.175665in}}{\pgfqpoint{3.027979in}{2.171275in}}{\pgfqpoint{3.020166in}{2.163461in}}%
\pgfpathcurveto{\pgfqpoint{3.012352in}{2.155647in}}{\pgfqpoint{3.007962in}{2.145048in}}{\pgfqpoint{3.007962in}{2.133998in}}%
\pgfpathcurveto{\pgfqpoint{3.007962in}{2.122948in}}{\pgfqpoint{3.012352in}{2.112349in}}{\pgfqpoint{3.020166in}{2.104535in}}%
\pgfpathcurveto{\pgfqpoint{3.027979in}{2.096722in}}{\pgfqpoint{3.038578in}{2.092332in}}{\pgfqpoint{3.049628in}{2.092332in}}%
\pgfpathclose%
\pgfusepath{stroke,fill}%
\end{pgfscope}%
\begin{pgfscope}%
\pgfpathrectangle{\pgfqpoint{1.016621in}{0.499691in}}{\pgfqpoint{3.875000in}{2.695000in}}%
\pgfusepath{clip}%
\pgfsetrectcap%
\pgfsetroundjoin%
\pgfsetlinewidth{1.505625pt}%
\definecolor{currentstroke}{rgb}{1.000000,0.843137,0.000000}%
\pgfsetstrokecolor{currentstroke}%
\pgfsetdash{}{0pt}%
\pgfpathmoveto{\pgfqpoint{1.192757in}{0.622191in}}%
\pgfpathlineto{\pgfqpoint{1.463284in}{0.624449in}}%
\pgfpathlineto{\pgfqpoint{1.627952in}{0.627956in}}%
\pgfpathlineto{\pgfqpoint{1.745572in}{0.632610in}}%
\pgfpathlineto{\pgfqpoint{1.839669in}{0.638549in}}%
\pgfpathlineto{\pgfqpoint{1.916122in}{0.645572in}}%
\pgfpathlineto{\pgfqpoint{1.980813in}{0.653669in}}%
\pgfpathlineto{\pgfqpoint{2.039623in}{0.663306in}}%
\pgfpathlineto{\pgfqpoint{2.092552in}{0.674351in}}%
\pgfpathlineto{\pgfqpoint{2.139600in}{0.686516in}}%
\pgfpathlineto{\pgfqpoint{2.180768in}{0.699352in}}%
\pgfpathlineto{\pgfqpoint{2.221935in}{0.714614in}}%
\pgfpathlineto{\pgfqpoint{2.257221in}{0.729939in}}%
\pgfpathlineto{\pgfqpoint{2.292507in}{0.747645in}}%
\pgfpathlineto{\pgfqpoint{2.327793in}{0.768054in}}%
\pgfpathlineto{\pgfqpoint{2.357198in}{0.787380in}}%
\pgfpathlineto{\pgfqpoint{2.386603in}{0.809041in}}%
\pgfpathlineto{\pgfqpoint{2.416008in}{0.833265in}}%
\pgfpathlineto{\pgfqpoint{2.445413in}{0.860283in}}%
\pgfpathlineto{\pgfqpoint{2.474818in}{0.890331in}}%
\pgfpathlineto{\pgfqpoint{2.504223in}{0.923639in}}%
\pgfpathlineto{\pgfqpoint{2.533628in}{0.960430in}}%
\pgfpathlineto{\pgfqpoint{2.563033in}{1.000905in}}%
\pgfpathlineto{\pgfqpoint{2.592438in}{1.045242in}}%
\pgfpathlineto{\pgfqpoint{2.621844in}{1.093576in}}%
\pgfpathlineto{\pgfqpoint{2.651249in}{1.145994in}}%
\pgfpathlineto{\pgfqpoint{2.680654in}{1.202520in}}%
\pgfpathlineto{\pgfqpoint{2.715940in}{1.275702in}}%
\pgfpathlineto{\pgfqpoint{2.751226in}{1.354486in}}%
\pgfpathlineto{\pgfqpoint{2.786512in}{1.438442in}}%
\pgfpathlineto{\pgfqpoint{2.827679in}{1.542083in}}%
\pgfpathlineto{\pgfqpoint{2.880608in}{1.682267in}}%
\pgfpathlineto{\pgfqpoint{2.974704in}{1.940142in}}%
\pgfpathlineto{\pgfqpoint{3.039396in}{2.114080in}}%
\pgfpathlineto{\pgfqpoint{3.086444in}{2.234256in}}%
\pgfpathlineto{\pgfqpoint{3.127611in}{2.333124in}}%
\pgfpathlineto{\pgfqpoint{3.162897in}{2.412349in}}%
\pgfpathlineto{\pgfqpoint{3.198183in}{2.486000in}}%
\pgfpathlineto{\pgfqpoint{3.233469in}{2.553820in}}%
\pgfpathlineto{\pgfqpoint{3.262874in}{2.605819in}}%
\pgfpathlineto{\pgfqpoint{3.292279in}{2.653740in}}%
\pgfpathlineto{\pgfqpoint{3.321684in}{2.697675in}}%
\pgfpathlineto{\pgfqpoint{3.351089in}{2.737767in}}%
\pgfpathlineto{\pgfqpoint{3.380494in}{2.774192in}}%
\pgfpathlineto{\pgfqpoint{3.409899in}{2.807157in}}%
\pgfpathlineto{\pgfqpoint{3.439304in}{2.836885in}}%
\pgfpathlineto{\pgfqpoint{3.468710in}{2.863607in}}%
\pgfpathlineto{\pgfqpoint{3.498115in}{2.887559in}}%
\pgfpathlineto{\pgfqpoint{3.527520in}{2.908972in}}%
\pgfpathlineto{\pgfqpoint{3.562806in}{2.931632in}}%
\pgfpathlineto{\pgfqpoint{3.598092in}{2.951329in}}%
\pgfpathlineto{\pgfqpoint{3.633378in}{2.968406in}}%
\pgfpathlineto{\pgfqpoint{3.668664in}{2.983180in}}%
\pgfpathlineto{\pgfqpoint{3.709831in}{2.997884in}}%
\pgfpathlineto{\pgfqpoint{3.750998in}{3.010247in}}%
\pgfpathlineto{\pgfqpoint{3.798046in}{3.021958in}}%
\pgfpathlineto{\pgfqpoint{3.850975in}{3.032588in}}%
\pgfpathlineto{\pgfqpoint{3.909786in}{3.041859in}}%
\pgfpathlineto{\pgfqpoint{3.974477in}{3.049646in}}%
\pgfpathlineto{\pgfqpoint{4.050930in}{3.056398in}}%
\pgfpathlineto{\pgfqpoint{4.139145in}{3.061817in}}%
\pgfpathlineto{\pgfqpoint{4.245003in}{3.066051in}}%
\pgfpathlineto{\pgfqpoint{4.386148in}{3.069326in}}%
\pgfpathlineto{\pgfqpoint{4.586102in}{3.071524in}}%
\pgfpathlineto{\pgfqpoint{4.715484in}{3.072191in}}%
\pgfpathlineto{\pgfqpoint{4.715484in}{3.072191in}}%
\pgfusepath{stroke}%
\end{pgfscope}%
\begin{pgfscope}%
\pgfsetrectcap%
\pgfsetmiterjoin%
\pgfsetlinewidth{0.803000pt}%
\definecolor{currentstroke}{rgb}{0.000000,0.000000,0.000000}%
\pgfsetstrokecolor{currentstroke}%
\pgfsetdash{}{0pt}%
\pgfpathmoveto{\pgfqpoint{1.016621in}{0.499691in}}%
\pgfpathlineto{\pgfqpoint{1.016621in}{3.194691in}}%
\pgfusepath{stroke}%
\end{pgfscope}%
\begin{pgfscope}%
\pgfsetrectcap%
\pgfsetmiterjoin%
\pgfsetlinewidth{0.803000pt}%
\definecolor{currentstroke}{rgb}{0.000000,0.000000,0.000000}%
\pgfsetstrokecolor{currentstroke}%
\pgfsetdash{}{0pt}%
\pgfpathmoveto{\pgfqpoint{4.891621in}{0.499691in}}%
\pgfpathlineto{\pgfqpoint{4.891621in}{3.194691in}}%
\pgfusepath{stroke}%
\end{pgfscope}%
\begin{pgfscope}%
\pgfsetrectcap%
\pgfsetmiterjoin%
\pgfsetlinewidth{0.803000pt}%
\definecolor{currentstroke}{rgb}{0.000000,0.000000,0.000000}%
\pgfsetstrokecolor{currentstroke}%
\pgfsetdash{}{0pt}%
\pgfpathmoveto{\pgfqpoint{1.016621in}{0.499691in}}%
\pgfpathlineto{\pgfqpoint{4.891621in}{0.499691in}}%
\pgfusepath{stroke}%
\end{pgfscope}%
\begin{pgfscope}%
\pgfsetrectcap%
\pgfsetmiterjoin%
\pgfsetlinewidth{0.803000pt}%
\definecolor{currentstroke}{rgb}{0.000000,0.000000,0.000000}%
\pgfsetstrokecolor{currentstroke}%
\pgfsetdash{}{0pt}%
\pgfpathmoveto{\pgfqpoint{1.016621in}{3.194691in}}%
\pgfpathlineto{\pgfqpoint{4.891621in}{3.194691in}}%
\pgfusepath{stroke}%
\end{pgfscope}%
\begin{pgfscope}%
\pgfpathrectangle{\pgfqpoint{1.016621in}{0.499691in}}{\pgfqpoint{3.875000in}{2.695000in}}%
\pgfusepath{clip}%
\pgfsetbuttcap%
\pgfsetroundjoin%
\definecolor{currentfill}{rgb}{1.000000,0.388235,0.278431}%
\pgfsetfillcolor{currentfill}%
\pgfsetlinewidth{1.003750pt}%
\definecolor{currentstroke}{rgb}{1.000000,0.388235,0.278431}%
\pgfsetstrokecolor{currentstroke}%
\pgfsetdash{}{0pt}%
\pgfpathmoveto{\pgfqpoint{2.986466in}{1.930599in}}%
\pgfpathcurveto{\pgfqpoint{2.997517in}{1.930599in}}{\pgfqpoint{3.008116in}{1.934989in}}{\pgfqpoint{3.015929in}{1.942803in}}%
\pgfpathcurveto{\pgfqpoint{3.023743in}{1.950617in}}{\pgfqpoint{3.028133in}{1.961216in}}{\pgfqpoint{3.028133in}{1.972266in}}%
\pgfpathcurveto{\pgfqpoint{3.028133in}{1.983316in}}{\pgfqpoint{3.023743in}{1.993915in}}{\pgfqpoint{3.015929in}{2.001729in}}%
\pgfpathcurveto{\pgfqpoint{3.008116in}{2.009542in}}{\pgfqpoint{2.997517in}{2.013932in}}{\pgfqpoint{2.986466in}{2.013932in}}%
\pgfpathcurveto{\pgfqpoint{2.975416in}{2.013932in}}{\pgfqpoint{2.964817in}{2.009542in}}{\pgfqpoint{2.957004in}{2.001729in}}%
\pgfpathcurveto{\pgfqpoint{2.949190in}{1.993915in}}{\pgfqpoint{2.944800in}{1.983316in}}{\pgfqpoint{2.944800in}{1.972266in}}%
\pgfpathcurveto{\pgfqpoint{2.944800in}{1.961216in}}{\pgfqpoint{2.949190in}{1.950617in}}{\pgfqpoint{2.957004in}{1.942803in}}%
\pgfpathcurveto{\pgfqpoint{2.964817in}{1.934989in}}{\pgfqpoint{2.975416in}{1.930599in}}{\pgfqpoint{2.986466in}{1.930599in}}%
\pgfpathclose%
\pgfusepath{stroke,fill}%
\end{pgfscope}%
\end{pgfpicture}%
\makeatother%
\endgroup%

    \caption{Representación de $\varphi$ frente a f (escala logarítmica) con un ajuste a una curva sigmoide}
  \end{figure}

  Se puede observar que el ajuste se acerca mucho a los datos. No podemos asegurar que la curva sigmoide sea la apropiada para ajustar estos datos, y hemos hecho muchas aproximaciones y asunciones que podrían hacer que esta representación sea errónea.

  \subsection{Ajuste correcto}
  \label{sec:arctg}

  Revisando la práctica nos dimos cuenta de un error en este razonamiento. Decidimos mantenerlo porque muestra el camino que seguimos hasta llegar a este punto, pero procederemos a explicar el nuevo camino al que arrivamos.

  En este sección hemos utilizado la ecuación \ref{ec:desfase}, pero no hemos vuelto al origen, la ecuación \ref{ec:faseimp}. Allí se describe la impedancia como un arcotangente. Esto tiene mucho sentido, ya que la función sigmoide y la arcotangente tienen una forma muy similar pese a ser distitnas. Por ejemplo, ambas se utilizan en redes neuronales para la misma función. Esta similaridad nos llevó a la confusión, pese a que la respuesta estaba todo el rato en la ecuación \ref{ec:faseimp}. Podemos adaptar el código anterior para una arcotangente y representarla, primero normalizando nuestros datos. En esta ocasión no hace falta dividir la y por 100 ya que añadiremos ese parámetro a nuestro ajuste por arcotangente como un parámetro a.

  \begin{python}
    def farctan(x, a, b, c):
        xrad = np.deg2rad(x)
        return a * np.rad2deg(np.arctan(b * xrad)) + c

    y3 = (y2 + 45)
    x3 = x2 - 3.1226

    popt, pcov = curve_fit(farctan, x3, y3)

    xs = np.arange(-1.8, +2., 0.01)
    ys = (farctan(xs, *popt)) - 45
    xs += 3.1226
    plt.plot(xs, ys, color="gold")

    for i in range(len(ys)):
        if ys[i] < -44.5 and ys[i] > -45.5:
            plt.scatter(xs[i], ys[i], color="tomato", zorder=4)
            print("Frecuencia de corte:", xs[i], 10**xs[i])
  \end{python}

  \begin{figure}[H]
    %\centering
    \hspace{2.5em} %% Creator: Matplotlib, PGF backend
%%
%% To include the figure in your LaTeX document, write
%%   \input{<filename>.pgf}
%%
%% Make sure the required packages are loaded in your preamble
%%   \usepackage{pgf}
%%
%% Figures using additional raster images can only be included by \input if
%% they are in the same directory as the main LaTeX file. For loading figures
%% from other directories you can use the `import` package
%%   \usepackage{import}
%% and then include the figures with
%%   \import{<path to file>}{<filename>.pgf}
%%
%% Matplotlib used the following preamble
%%
\begingroup%
\makeatletter%
\begin{pgfpicture}%
\pgfpathrectangle{\pgfpointorigin}{\pgfqpoint{4.991621in}{3.294691in}}%
\pgfusepath{use as bounding box, clip}%
\begin{pgfscope}%
\pgfsetbuttcap%
\pgfsetmiterjoin%
\definecolor{currentfill}{rgb}{1.000000,1.000000,1.000000}%
\pgfsetfillcolor{currentfill}%
\pgfsetlinewidth{0.000000pt}%
\definecolor{currentstroke}{rgb}{1.000000,1.000000,1.000000}%
\pgfsetstrokecolor{currentstroke}%
\pgfsetdash{}{0pt}%
\pgfpathmoveto{\pgfqpoint{0.000000in}{0.000000in}}%
\pgfpathlineto{\pgfqpoint{4.991621in}{0.000000in}}%
\pgfpathlineto{\pgfqpoint{4.991621in}{3.294691in}}%
\pgfpathlineto{\pgfqpoint{0.000000in}{3.294691in}}%
\pgfpathclose%
\pgfusepath{fill}%
\end{pgfscope}%
\begin{pgfscope}%
\pgfsetbuttcap%
\pgfsetmiterjoin%
\definecolor{currentfill}{rgb}{1.000000,1.000000,1.000000}%
\pgfsetfillcolor{currentfill}%
\pgfsetlinewidth{0.000000pt}%
\definecolor{currentstroke}{rgb}{0.000000,0.000000,0.000000}%
\pgfsetstrokecolor{currentstroke}%
\pgfsetstrokeopacity{0.000000}%
\pgfsetdash{}{0pt}%
\pgfpathmoveto{\pgfqpoint{1.016621in}{0.499691in}}%
\pgfpathlineto{\pgfqpoint{4.891621in}{0.499691in}}%
\pgfpathlineto{\pgfqpoint{4.891621in}{3.194691in}}%
\pgfpathlineto{\pgfqpoint{1.016621in}{3.194691in}}%
\pgfpathclose%
\pgfusepath{fill}%
\end{pgfscope}%
\begin{pgfscope}%
\pgfpathrectangle{\pgfqpoint{1.016621in}{0.499691in}}{\pgfqpoint{3.875000in}{2.695000in}}%
\pgfusepath{clip}%
\pgfsetbuttcap%
\pgfsetroundjoin%
\definecolor{currentfill}{rgb}{0.117647,0.564706,1.000000}%
\pgfsetfillcolor{currentfill}%
\pgfsetlinewidth{1.003750pt}%
\definecolor{currentstroke}{rgb}{0.117647,0.564706,1.000000}%
\pgfsetstrokecolor{currentstroke}%
\pgfsetdash{}{0pt}%
\pgfpathmoveto{\pgfqpoint{1.488855in}{0.590428in}}%
\pgfpathcurveto{\pgfqpoint{1.499905in}{0.590428in}}{\pgfqpoint{1.510504in}{0.594818in}}{\pgfqpoint{1.518317in}{0.602632in}}%
\pgfpathcurveto{\pgfqpoint{1.526131in}{0.610446in}}{\pgfqpoint{1.530521in}{0.621045in}}{\pgfqpoint{1.530521in}{0.632095in}}%
\pgfpathcurveto{\pgfqpoint{1.530521in}{0.643145in}}{\pgfqpoint{1.526131in}{0.653744in}}{\pgfqpoint{1.518317in}{0.661558in}}%
\pgfpathcurveto{\pgfqpoint{1.510504in}{0.669371in}}{\pgfqpoint{1.499905in}{0.673761in}}{\pgfqpoint{1.488855in}{0.673761in}}%
\pgfpathcurveto{\pgfqpoint{1.477804in}{0.673761in}}{\pgfqpoint{1.467205in}{0.669371in}}{\pgfqpoint{1.459392in}{0.661558in}}%
\pgfpathcurveto{\pgfqpoint{1.451578in}{0.653744in}}{\pgfqpoint{1.447188in}{0.643145in}}{\pgfqpoint{1.447188in}{0.632095in}}%
\pgfpathcurveto{\pgfqpoint{1.447188in}{0.621045in}}{\pgfqpoint{1.451578in}{0.610446in}}{\pgfqpoint{1.459392in}{0.602632in}}%
\pgfpathcurveto{\pgfqpoint{1.467205in}{0.594818in}}{\pgfqpoint{1.477804in}{0.590428in}}{\pgfqpoint{1.488855in}{0.590428in}}%
\pgfpathclose%
\pgfusepath{stroke,fill}%
\end{pgfscope}%
\begin{pgfscope}%
\pgfpathrectangle{\pgfqpoint{1.016621in}{0.499691in}}{\pgfqpoint{3.875000in}{2.695000in}}%
\pgfusepath{clip}%
\pgfsetbuttcap%
\pgfsetroundjoin%
\definecolor{currentfill}{rgb}{0.117647,0.564706,1.000000}%
\pgfsetfillcolor{currentfill}%
\pgfsetlinewidth{1.003750pt}%
\definecolor{currentstroke}{rgb}{0.117647,0.564706,1.000000}%
\pgfsetstrokecolor{currentstroke}%
\pgfsetdash{}{0pt}%
\pgfpathmoveto{\pgfqpoint{2.100821in}{0.697337in}}%
\pgfpathcurveto{\pgfqpoint{2.111871in}{0.697337in}}{\pgfqpoint{2.122470in}{0.701727in}}{\pgfqpoint{2.130283in}{0.709541in}}%
\pgfpathcurveto{\pgfqpoint{2.138097in}{0.717354in}}{\pgfqpoint{2.142487in}{0.727953in}}{\pgfqpoint{2.142487in}{0.739003in}}%
\pgfpathcurveto{\pgfqpoint{2.142487in}{0.750054in}}{\pgfqpoint{2.138097in}{0.760653in}}{\pgfqpoint{2.130283in}{0.768466in}}%
\pgfpathcurveto{\pgfqpoint{2.122470in}{0.776280in}}{\pgfqpoint{2.111871in}{0.780670in}}{\pgfqpoint{2.100821in}{0.780670in}}%
\pgfpathcurveto{\pgfqpoint{2.089770in}{0.780670in}}{\pgfqpoint{2.079171in}{0.776280in}}{\pgfqpoint{2.071358in}{0.768466in}}%
\pgfpathcurveto{\pgfqpoint{2.063544in}{0.760653in}}{\pgfqpoint{2.059154in}{0.750054in}}{\pgfqpoint{2.059154in}{0.739003in}}%
\pgfpathcurveto{\pgfqpoint{2.059154in}{0.727953in}}{\pgfqpoint{2.063544in}{0.717354in}}{\pgfqpoint{2.071358in}{0.709541in}}%
\pgfpathcurveto{\pgfqpoint{2.079171in}{0.701727in}}{\pgfqpoint{2.089770in}{0.697337in}}{\pgfqpoint{2.100821in}{0.697337in}}%
\pgfpathclose%
\pgfusepath{stroke,fill}%
\end{pgfscope}%
\begin{pgfscope}%
\pgfpathrectangle{\pgfqpoint{1.016621in}{0.499691in}}{\pgfqpoint{3.875000in}{2.695000in}}%
\pgfusepath{clip}%
\pgfsetbuttcap%
\pgfsetroundjoin%
\definecolor{currentfill}{rgb}{0.117647,0.564706,1.000000}%
\pgfsetfillcolor{currentfill}%
\pgfsetlinewidth{1.003750pt}%
\definecolor{currentstroke}{rgb}{0.117647,0.564706,1.000000}%
\pgfsetstrokecolor{currentstroke}%
\pgfsetdash{}{0pt}%
\pgfpathmoveto{\pgfqpoint{2.336192in}{0.906295in}}%
\pgfpathcurveto{\pgfqpoint{2.347242in}{0.906295in}}{\pgfqpoint{2.357841in}{0.910685in}}{\pgfqpoint{2.365655in}{0.918498in}}%
\pgfpathcurveto{\pgfqpoint{2.373469in}{0.926312in}}{\pgfqpoint{2.377859in}{0.936911in}}{\pgfqpoint{2.377859in}{0.947961in}}%
\pgfpathcurveto{\pgfqpoint{2.377859in}{0.959011in}}{\pgfqpoint{2.373469in}{0.969610in}}{\pgfqpoint{2.365655in}{0.977424in}}%
\pgfpathcurveto{\pgfqpoint{2.357841in}{0.985238in}}{\pgfqpoint{2.347242in}{0.989628in}}{\pgfqpoint{2.336192in}{0.989628in}}%
\pgfpathcurveto{\pgfqpoint{2.325142in}{0.989628in}}{\pgfqpoint{2.314543in}{0.985238in}}{\pgfqpoint{2.306729in}{0.977424in}}%
\pgfpathcurveto{\pgfqpoint{2.298916in}{0.969610in}}{\pgfqpoint{2.294525in}{0.959011in}}{\pgfqpoint{2.294525in}{0.947961in}}%
\pgfpathcurveto{\pgfqpoint{2.294525in}{0.936911in}}{\pgfqpoint{2.298916in}{0.926312in}}{\pgfqpoint{2.306729in}{0.918498in}}%
\pgfpathcurveto{\pgfqpoint{2.314543in}{0.910685in}}{\pgfqpoint{2.325142in}{0.906295in}}{\pgfqpoint{2.336192in}{0.906295in}}%
\pgfpathclose%
\pgfusepath{stroke,fill}%
\end{pgfscope}%
\begin{pgfscope}%
\pgfpathrectangle{\pgfqpoint{1.016621in}{0.499691in}}{\pgfqpoint{3.875000in}{2.695000in}}%
\pgfusepath{clip}%
\pgfsetbuttcap%
\pgfsetroundjoin%
\definecolor{currentfill}{rgb}{0.117647,0.564706,1.000000}%
\pgfsetfillcolor{currentfill}%
\pgfsetlinewidth{1.003750pt}%
\definecolor{currentstroke}{rgb}{0.117647,0.564706,1.000000}%
\pgfsetstrokecolor{currentstroke}%
\pgfsetdash{}{0pt}%
\pgfpathmoveto{\pgfqpoint{2.626484in}{0.947600in}}%
\pgfpathcurveto{\pgfqpoint{2.637534in}{0.947600in}}{\pgfqpoint{2.648133in}{0.951990in}}{\pgfqpoint{2.655946in}{0.959804in}}%
\pgfpathcurveto{\pgfqpoint{2.663760in}{0.967618in}}{\pgfqpoint{2.668150in}{0.978217in}}{\pgfqpoint{2.668150in}{0.989267in}}%
\pgfpathcurveto{\pgfqpoint{2.668150in}{1.000317in}}{\pgfqpoint{2.663760in}{1.010916in}}{\pgfqpoint{2.655946in}{1.018730in}}%
\pgfpathcurveto{\pgfqpoint{2.648133in}{1.026543in}}{\pgfqpoint{2.637534in}{1.030933in}}{\pgfqpoint{2.626484in}{1.030933in}}%
\pgfpathcurveto{\pgfqpoint{2.615434in}{1.030933in}}{\pgfqpoint{2.604834in}{1.026543in}}{\pgfqpoint{2.597021in}{1.018730in}}%
\pgfpathcurveto{\pgfqpoint{2.589207in}{1.010916in}}{\pgfqpoint{2.584817in}{1.000317in}}{\pgfqpoint{2.584817in}{0.989267in}}%
\pgfpathcurveto{\pgfqpoint{2.584817in}{0.978217in}}{\pgfqpoint{2.589207in}{0.967618in}}{\pgfqpoint{2.597021in}{0.959804in}}%
\pgfpathcurveto{\pgfqpoint{2.604834in}{0.951990in}}{\pgfqpoint{2.615434in}{0.947600in}}{\pgfqpoint{2.626484in}{0.947600in}}%
\pgfpathclose%
\pgfusepath{stroke,fill}%
\end{pgfscope}%
\begin{pgfscope}%
\pgfpathrectangle{\pgfqpoint{1.016621in}{0.499691in}}{\pgfqpoint{3.875000in}{2.695000in}}%
\pgfusepath{clip}%
\pgfsetbuttcap%
\pgfsetroundjoin%
\definecolor{currentfill}{rgb}{0.117647,0.564706,1.000000}%
\pgfsetfillcolor{currentfill}%
\pgfsetlinewidth{1.003750pt}%
\definecolor{currentstroke}{rgb}{0.117647,0.564706,1.000000}%
\pgfsetstrokecolor{currentstroke}%
\pgfsetdash{}{0pt}%
\pgfpathmoveto{\pgfqpoint{2.744169in}{1.129831in}}%
\pgfpathcurveto{\pgfqpoint{2.755220in}{1.129831in}}{\pgfqpoint{2.765819in}{1.134221in}}{\pgfqpoint{2.773632in}{1.142035in}}%
\pgfpathcurveto{\pgfqpoint{2.781446in}{1.149848in}}{\pgfqpoint{2.785836in}{1.160447in}}{\pgfqpoint{2.785836in}{1.171497in}}%
\pgfpathcurveto{\pgfqpoint{2.785836in}{1.182548in}}{\pgfqpoint{2.781446in}{1.193147in}}{\pgfqpoint{2.773632in}{1.200960in}}%
\pgfpathcurveto{\pgfqpoint{2.765819in}{1.208774in}}{\pgfqpoint{2.755220in}{1.213164in}}{\pgfqpoint{2.744169in}{1.213164in}}%
\pgfpathcurveto{\pgfqpoint{2.733119in}{1.213164in}}{\pgfqpoint{2.722520in}{1.208774in}}{\pgfqpoint{2.714707in}{1.200960in}}%
\pgfpathcurveto{\pgfqpoint{2.706893in}{1.193147in}}{\pgfqpoint{2.702503in}{1.182548in}}{\pgfqpoint{2.702503in}{1.171497in}}%
\pgfpathcurveto{\pgfqpoint{2.702503in}{1.160447in}}{\pgfqpoint{2.706893in}{1.149848in}}{\pgfqpoint{2.714707in}{1.142035in}}%
\pgfpathcurveto{\pgfqpoint{2.722520in}{1.134221in}}{\pgfqpoint{2.733119in}{1.129831in}}{\pgfqpoint{2.744169in}{1.129831in}}%
\pgfpathclose%
\pgfusepath{stroke,fill}%
\end{pgfscope}%
\begin{pgfscope}%
\pgfpathrectangle{\pgfqpoint{1.016621in}{0.499691in}}{\pgfqpoint{3.875000in}{2.695000in}}%
\pgfusepath{clip}%
\pgfsetbuttcap%
\pgfsetroundjoin%
\definecolor{currentfill}{rgb}{0.117647,0.564706,1.000000}%
\pgfsetfillcolor{currentfill}%
\pgfsetlinewidth{1.003750pt}%
\definecolor{currentstroke}{rgb}{0.117647,0.564706,1.000000}%
\pgfsetstrokecolor{currentstroke}%
\pgfsetdash{}{0pt}%
\pgfpathmoveto{\pgfqpoint{2.838318in}{1.205153in}}%
\pgfpathcurveto{\pgfqpoint{2.849368in}{1.205153in}}{\pgfqpoint{2.859967in}{1.209543in}}{\pgfqpoint{2.867781in}{1.217357in}}%
\pgfpathcurveto{\pgfqpoint{2.875594in}{1.225170in}}{\pgfqpoint{2.879985in}{1.235769in}}{\pgfqpoint{2.879985in}{1.246819in}}%
\pgfpathcurveto{\pgfqpoint{2.879985in}{1.257869in}}{\pgfqpoint{2.875594in}{1.268469in}}{\pgfqpoint{2.867781in}{1.276282in}}%
\pgfpathcurveto{\pgfqpoint{2.859967in}{1.284096in}}{\pgfqpoint{2.849368in}{1.288486in}}{\pgfqpoint{2.838318in}{1.288486in}}%
\pgfpathcurveto{\pgfqpoint{2.827268in}{1.288486in}}{\pgfqpoint{2.816669in}{1.284096in}}{\pgfqpoint{2.808855in}{1.276282in}}%
\pgfpathcurveto{\pgfqpoint{2.801042in}{1.268469in}}{\pgfqpoint{2.796651in}{1.257869in}}{\pgfqpoint{2.796651in}{1.246819in}}%
\pgfpathcurveto{\pgfqpoint{2.796651in}{1.235769in}}{\pgfqpoint{2.801042in}{1.225170in}}{\pgfqpoint{2.808855in}{1.217357in}}%
\pgfpathcurveto{\pgfqpoint{2.816669in}{1.209543in}}{\pgfqpoint{2.827268in}{1.205153in}}{\pgfqpoint{2.838318in}{1.205153in}}%
\pgfpathclose%
\pgfusepath{stroke,fill}%
\end{pgfscope}%
\begin{pgfscope}%
\pgfpathrectangle{\pgfqpoint{1.016621in}{0.499691in}}{\pgfqpoint{3.875000in}{2.695000in}}%
\pgfusepath{clip}%
\pgfsetbuttcap%
\pgfsetroundjoin%
\definecolor{currentfill}{rgb}{0.117647,0.564706,1.000000}%
\pgfsetfillcolor{currentfill}%
\pgfsetlinewidth{1.003750pt}%
\definecolor{currentstroke}{rgb}{0.117647,0.564706,1.000000}%
\pgfsetstrokecolor{currentstroke}%
\pgfsetdash{}{0pt}%
\pgfpathmoveto{\pgfqpoint{2.869701in}{1.280475in}}%
\pgfpathcurveto{\pgfqpoint{2.880751in}{1.280475in}}{\pgfqpoint{2.891350in}{1.284865in}}{\pgfqpoint{2.899164in}{1.292679in}}%
\pgfpathcurveto{\pgfqpoint{2.906977in}{1.300492in}}{\pgfqpoint{2.911368in}{1.311091in}}{\pgfqpoint{2.911368in}{1.322141in}}%
\pgfpathcurveto{\pgfqpoint{2.911368in}{1.333191in}}{\pgfqpoint{2.906977in}{1.343791in}}{\pgfqpoint{2.899164in}{1.351604in}}%
\pgfpathcurveto{\pgfqpoint{2.891350in}{1.359418in}}{\pgfqpoint{2.880751in}{1.363808in}}{\pgfqpoint{2.869701in}{1.363808in}}%
\pgfpathcurveto{\pgfqpoint{2.858651in}{1.363808in}}{\pgfqpoint{2.848052in}{1.359418in}}{\pgfqpoint{2.840238in}{1.351604in}}%
\pgfpathcurveto{\pgfqpoint{2.832425in}{1.343791in}}{\pgfqpoint{2.828034in}{1.333191in}}{\pgfqpoint{2.828034in}{1.322141in}}%
\pgfpathcurveto{\pgfqpoint{2.828034in}{1.311091in}}{\pgfqpoint{2.832425in}{1.300492in}}{\pgfqpoint{2.840238in}{1.292679in}}%
\pgfpathcurveto{\pgfqpoint{2.848052in}{1.284865in}}{\pgfqpoint{2.858651in}{1.280475in}}{\pgfqpoint{2.869701in}{1.280475in}}%
\pgfpathclose%
\pgfusepath{stroke,fill}%
\end{pgfscope}%
\begin{pgfscope}%
\pgfpathrectangle{\pgfqpoint{1.016621in}{0.499691in}}{\pgfqpoint{3.875000in}{2.695000in}}%
\pgfusepath{clip}%
\pgfsetbuttcap%
\pgfsetroundjoin%
\definecolor{currentfill}{rgb}{0.117647,0.564706,1.000000}%
\pgfsetfillcolor{currentfill}%
\pgfsetlinewidth{1.003750pt}%
\definecolor{currentstroke}{rgb}{0.117647,0.564706,1.000000}%
\pgfsetstrokecolor{currentstroke}%
\pgfsetdash{}{0pt}%
\pgfpathmoveto{\pgfqpoint{2.908929in}{1.348507in}}%
\pgfpathcurveto{\pgfqpoint{2.919980in}{1.348507in}}{\pgfqpoint{2.930579in}{1.352898in}}{\pgfqpoint{2.938392in}{1.360711in}}%
\pgfpathcurveto{\pgfqpoint{2.946206in}{1.368525in}}{\pgfqpoint{2.950596in}{1.379124in}}{\pgfqpoint{2.950596in}{1.390174in}}%
\pgfpathcurveto{\pgfqpoint{2.950596in}{1.401224in}}{\pgfqpoint{2.946206in}{1.411823in}}{\pgfqpoint{2.938392in}{1.419637in}}%
\pgfpathcurveto{\pgfqpoint{2.930579in}{1.427451in}}{\pgfqpoint{2.919980in}{1.431841in}}{\pgfqpoint{2.908929in}{1.431841in}}%
\pgfpathcurveto{\pgfqpoint{2.897879in}{1.431841in}}{\pgfqpoint{2.887280in}{1.427451in}}{\pgfqpoint{2.879467in}{1.419637in}}%
\pgfpathcurveto{\pgfqpoint{2.871653in}{1.411823in}}{\pgfqpoint{2.867263in}{1.401224in}}{\pgfqpoint{2.867263in}{1.390174in}}%
\pgfpathcurveto{\pgfqpoint{2.867263in}{1.379124in}}{\pgfqpoint{2.871653in}{1.368525in}}{\pgfqpoint{2.879467in}{1.360711in}}%
\pgfpathcurveto{\pgfqpoint{2.887280in}{1.352898in}}{\pgfqpoint{2.897879in}{1.348507in}}{\pgfqpoint{2.908929in}{1.348507in}}%
\pgfpathclose%
\pgfusepath{stroke,fill}%
\end{pgfscope}%
\begin{pgfscope}%
\pgfpathrectangle{\pgfqpoint{1.016621in}{0.499691in}}{\pgfqpoint{3.875000in}{2.695000in}}%
\pgfusepath{clip}%
\pgfsetbuttcap%
\pgfsetroundjoin%
\definecolor{currentfill}{rgb}{0.117647,0.564706,1.000000}%
\pgfsetfillcolor{currentfill}%
\pgfsetlinewidth{1.003750pt}%
\definecolor{currentstroke}{rgb}{0.117647,0.564706,1.000000}%
\pgfsetstrokecolor{currentstroke}%
\pgfsetdash{}{0pt}%
\pgfpathmoveto{\pgfqpoint{2.940312in}{1.511300in}}%
\pgfpathcurveto{\pgfqpoint{2.951362in}{1.511300in}}{\pgfqpoint{2.961962in}{1.515690in}}{\pgfqpoint{2.969775in}{1.523504in}}%
\pgfpathcurveto{\pgfqpoint{2.977589in}{1.531318in}}{\pgfqpoint{2.981979in}{1.541917in}}{\pgfqpoint{2.981979in}{1.552967in}}%
\pgfpathcurveto{\pgfqpoint{2.981979in}{1.564017in}}{\pgfqpoint{2.977589in}{1.574616in}}{\pgfqpoint{2.969775in}{1.582430in}}%
\pgfpathcurveto{\pgfqpoint{2.961962in}{1.590243in}}{\pgfqpoint{2.951362in}{1.594633in}}{\pgfqpoint{2.940312in}{1.594633in}}%
\pgfpathcurveto{\pgfqpoint{2.929262in}{1.594633in}}{\pgfqpoint{2.918663in}{1.590243in}}{\pgfqpoint{2.910850in}{1.582430in}}%
\pgfpathcurveto{\pgfqpoint{2.903036in}{1.574616in}}{\pgfqpoint{2.898646in}{1.564017in}}{\pgfqpoint{2.898646in}{1.552967in}}%
\pgfpathcurveto{\pgfqpoint{2.898646in}{1.541917in}}{\pgfqpoint{2.903036in}{1.531318in}}{\pgfqpoint{2.910850in}{1.523504in}}%
\pgfpathcurveto{\pgfqpoint{2.918663in}{1.515690in}}{\pgfqpoint{2.929262in}{1.511300in}}{\pgfqpoint{2.940312in}{1.511300in}}%
\pgfpathclose%
\pgfusepath{stroke,fill}%
\end{pgfscope}%
\begin{pgfscope}%
\pgfpathrectangle{\pgfqpoint{1.016621in}{0.499691in}}{\pgfqpoint{3.875000in}{2.695000in}}%
\pgfusepath{clip}%
\pgfsetbuttcap%
\pgfsetroundjoin%
\definecolor{currentfill}{rgb}{0.117647,0.564706,1.000000}%
\pgfsetfillcolor{currentfill}%
\pgfsetlinewidth{1.003750pt}%
\definecolor{currentstroke}{rgb}{0.117647,0.564706,1.000000}%
\pgfsetstrokecolor{currentstroke}%
\pgfsetdash{}{0pt}%
\pgfpathmoveto{\pgfqpoint{2.963850in}{1.462705in}}%
\pgfpathcurveto{\pgfqpoint{2.974900in}{1.462705in}}{\pgfqpoint{2.985499in}{1.467096in}}{\pgfqpoint{2.993312in}{1.474909in}}%
\pgfpathcurveto{\pgfqpoint{3.001126in}{1.482723in}}{\pgfqpoint{3.005516in}{1.493322in}}{\pgfqpoint{3.005516in}{1.504372in}}%
\pgfpathcurveto{\pgfqpoint{3.005516in}{1.515422in}}{\pgfqpoint{3.001126in}{1.526021in}}{\pgfqpoint{2.993312in}{1.533835in}}%
\pgfpathcurveto{\pgfqpoint{2.985499in}{1.541648in}}{\pgfqpoint{2.974900in}{1.546039in}}{\pgfqpoint{2.963850in}{1.546039in}}%
\pgfpathcurveto{\pgfqpoint{2.952799in}{1.546039in}}{\pgfqpoint{2.942200in}{1.541648in}}{\pgfqpoint{2.934387in}{1.533835in}}%
\pgfpathcurveto{\pgfqpoint{2.926573in}{1.526021in}}{\pgfqpoint{2.922183in}{1.515422in}}{\pgfqpoint{2.922183in}{1.504372in}}%
\pgfpathcurveto{\pgfqpoint{2.922183in}{1.493322in}}{\pgfqpoint{2.926573in}{1.482723in}}{\pgfqpoint{2.934387in}{1.474909in}}%
\pgfpathcurveto{\pgfqpoint{2.942200in}{1.467096in}}{\pgfqpoint{2.952799in}{1.462705in}}{\pgfqpoint{2.963850in}{1.462705in}}%
\pgfpathclose%
\pgfusepath{stroke,fill}%
\end{pgfscope}%
\begin{pgfscope}%
\pgfpathrectangle{\pgfqpoint{1.016621in}{0.499691in}}{\pgfqpoint{3.875000in}{2.695000in}}%
\pgfusepath{clip}%
\pgfsetbuttcap%
\pgfsetroundjoin%
\definecolor{currentfill}{rgb}{0.117647,0.564706,1.000000}%
\pgfsetfillcolor{currentfill}%
\pgfsetlinewidth{1.003750pt}%
\definecolor{currentstroke}{rgb}{0.117647,0.564706,1.000000}%
\pgfsetstrokecolor{currentstroke}%
\pgfsetdash{}{0pt}%
\pgfpathmoveto{\pgfqpoint{2.987387in}{1.428689in}}%
\pgfpathcurveto{\pgfqpoint{2.998437in}{1.428689in}}{\pgfqpoint{3.009036in}{1.433079in}}{\pgfqpoint{3.016849in}{1.440893in}}%
\pgfpathcurveto{\pgfqpoint{3.024663in}{1.448706in}}{\pgfqpoint{3.029053in}{1.459305in}}{\pgfqpoint{3.029053in}{1.470356in}}%
\pgfpathcurveto{\pgfqpoint{3.029053in}{1.481406in}}{\pgfqpoint{3.024663in}{1.492005in}}{\pgfqpoint{3.016849in}{1.499818in}}%
\pgfpathcurveto{\pgfqpoint{3.009036in}{1.507632in}}{\pgfqpoint{2.998437in}{1.512022in}}{\pgfqpoint{2.987387in}{1.512022in}}%
\pgfpathcurveto{\pgfqpoint{2.976337in}{1.512022in}}{\pgfqpoint{2.965738in}{1.507632in}}{\pgfqpoint{2.957924in}{1.499818in}}%
\pgfpathcurveto{\pgfqpoint{2.950110in}{1.492005in}}{\pgfqpoint{2.945720in}{1.481406in}}{\pgfqpoint{2.945720in}{1.470356in}}%
\pgfpathcurveto{\pgfqpoint{2.945720in}{1.459305in}}{\pgfqpoint{2.950110in}{1.448706in}}{\pgfqpoint{2.957924in}{1.440893in}}%
\pgfpathcurveto{\pgfqpoint{2.965738in}{1.433079in}}{\pgfqpoint{2.976337in}{1.428689in}}{\pgfqpoint{2.987387in}{1.428689in}}%
\pgfpathclose%
\pgfusepath{stroke,fill}%
\end{pgfscope}%
\begin{pgfscope}%
\pgfpathrectangle{\pgfqpoint{1.016621in}{0.499691in}}{\pgfqpoint{3.875000in}{2.695000in}}%
\pgfusepath{clip}%
\pgfsetbuttcap%
\pgfsetroundjoin%
\definecolor{currentfill}{rgb}{0.117647,0.564706,1.000000}%
\pgfsetfillcolor{currentfill}%
\pgfsetlinewidth{1.003750pt}%
\definecolor{currentstroke}{rgb}{0.117647,0.564706,1.000000}%
\pgfsetstrokecolor{currentstroke}%
\pgfsetdash{}{0pt}%
\pgfpathmoveto{\pgfqpoint{3.010924in}{1.404391in}}%
\pgfpathcurveto{\pgfqpoint{3.021974in}{1.404391in}}{\pgfqpoint{3.032573in}{1.408782in}}{\pgfqpoint{3.040387in}{1.416595in}}%
\pgfpathcurveto{\pgfqpoint{3.048200in}{1.424409in}}{\pgfqpoint{3.052590in}{1.435008in}}{\pgfqpoint{3.052590in}{1.446058in}}%
\pgfpathcurveto{\pgfqpoint{3.052590in}{1.457108in}}{\pgfqpoint{3.048200in}{1.467707in}}{\pgfqpoint{3.040387in}{1.475521in}}%
\pgfpathcurveto{\pgfqpoint{3.032573in}{1.483335in}}{\pgfqpoint{3.021974in}{1.487725in}}{\pgfqpoint{3.010924in}{1.487725in}}%
\pgfpathcurveto{\pgfqpoint{2.999874in}{1.487725in}}{\pgfqpoint{2.989275in}{1.483335in}}{\pgfqpoint{2.981461in}{1.475521in}}%
\pgfpathcurveto{\pgfqpoint{2.973647in}{1.467707in}}{\pgfqpoint{2.969257in}{1.457108in}}{\pgfqpoint{2.969257in}{1.446058in}}%
\pgfpathcurveto{\pgfqpoint{2.969257in}{1.435008in}}{\pgfqpoint{2.973647in}{1.424409in}}{\pgfqpoint{2.981461in}{1.416595in}}%
\pgfpathcurveto{\pgfqpoint{2.989275in}{1.408782in}}{\pgfqpoint{2.999874in}{1.404391in}}{\pgfqpoint{3.010924in}{1.404391in}}%
\pgfpathclose%
\pgfusepath{stroke,fill}%
\end{pgfscope}%
\begin{pgfscope}%
\pgfpathrectangle{\pgfqpoint{1.016621in}{0.499691in}}{\pgfqpoint{3.875000in}{2.695000in}}%
\pgfusepath{clip}%
\pgfsetbuttcap%
\pgfsetroundjoin%
\definecolor{currentfill}{rgb}{0.117647,0.564706,1.000000}%
\pgfsetfillcolor{currentfill}%
\pgfsetlinewidth{1.003750pt}%
\definecolor{currentstroke}{rgb}{0.117647,0.564706,1.000000}%
\pgfsetstrokecolor{currentstroke}%
\pgfsetdash{}{0pt}%
\pgfpathmoveto{\pgfqpoint{3.034461in}{1.525879in}}%
\pgfpathcurveto{\pgfqpoint{3.045511in}{1.525879in}}{\pgfqpoint{3.056110in}{1.530269in}}{\pgfqpoint{3.063924in}{1.538082in}}%
\pgfpathcurveto{\pgfqpoint{3.071737in}{1.545896in}}{\pgfqpoint{3.076128in}{1.556495in}}{\pgfqpoint{3.076128in}{1.567545in}}%
\pgfpathcurveto{\pgfqpoint{3.076128in}{1.578595in}}{\pgfqpoint{3.071737in}{1.589194in}}{\pgfqpoint{3.063924in}{1.597008in}}%
\pgfpathcurveto{\pgfqpoint{3.056110in}{1.604822in}}{\pgfqpoint{3.045511in}{1.609212in}}{\pgfqpoint{3.034461in}{1.609212in}}%
\pgfpathcurveto{\pgfqpoint{3.023411in}{1.609212in}}{\pgfqpoint{3.012812in}{1.604822in}}{\pgfqpoint{3.004998in}{1.597008in}}%
\pgfpathcurveto{\pgfqpoint{2.997185in}{1.589194in}}{\pgfqpoint{2.992794in}{1.578595in}}{\pgfqpoint{2.992794in}{1.567545in}}%
\pgfpathcurveto{\pgfqpoint{2.992794in}{1.556495in}}{\pgfqpoint{2.997185in}{1.545896in}}{\pgfqpoint{3.004998in}{1.538082in}}%
\pgfpathcurveto{\pgfqpoint{3.012812in}{1.530269in}}{\pgfqpoint{3.023411in}{1.525879in}}{\pgfqpoint{3.034461in}{1.525879in}}%
\pgfpathclose%
\pgfusepath{stroke,fill}%
\end{pgfscope}%
\begin{pgfscope}%
\pgfpathrectangle{\pgfqpoint{1.016621in}{0.499691in}}{\pgfqpoint{3.875000in}{2.695000in}}%
\pgfusepath{clip}%
\pgfsetbuttcap%
\pgfsetroundjoin%
\definecolor{currentfill}{rgb}{0.117647,0.564706,1.000000}%
\pgfsetfillcolor{currentfill}%
\pgfsetlinewidth{1.003750pt}%
\definecolor{currentstroke}{rgb}{0.117647,0.564706,1.000000}%
\pgfsetstrokecolor{currentstroke}%
\pgfsetdash{}{0pt}%
\pgfpathmoveto{\pgfqpoint{3.057998in}{1.606060in}}%
\pgfpathcurveto{\pgfqpoint{3.069048in}{1.606060in}}{\pgfqpoint{3.079647in}{1.610450in}}{\pgfqpoint{3.087461in}{1.618264in}}%
\pgfpathcurveto{\pgfqpoint{3.095275in}{1.626078in}}{\pgfqpoint{3.099665in}{1.636677in}}{\pgfqpoint{3.099665in}{1.647727in}}%
\pgfpathcurveto{\pgfqpoint{3.099665in}{1.658777in}}{\pgfqpoint{3.095275in}{1.669376in}}{\pgfqpoint{3.087461in}{1.677189in}}%
\pgfpathcurveto{\pgfqpoint{3.079647in}{1.685003in}}{\pgfqpoint{3.069048in}{1.689393in}}{\pgfqpoint{3.057998in}{1.689393in}}%
\pgfpathcurveto{\pgfqpoint{3.046948in}{1.689393in}}{\pgfqpoint{3.036349in}{1.685003in}}{\pgfqpoint{3.028535in}{1.677189in}}%
\pgfpathcurveto{\pgfqpoint{3.020722in}{1.669376in}}{\pgfqpoint{3.016331in}{1.658777in}}{\pgfqpoint{3.016331in}{1.647727in}}%
\pgfpathcurveto{\pgfqpoint{3.016331in}{1.636677in}}{\pgfqpoint{3.020722in}{1.626078in}}{\pgfqpoint{3.028535in}{1.618264in}}%
\pgfpathcurveto{\pgfqpoint{3.036349in}{1.610450in}}{\pgfqpoint{3.046948in}{1.606060in}}{\pgfqpoint{3.057998in}{1.606060in}}%
\pgfpathclose%
\pgfusepath{stroke,fill}%
\end{pgfscope}%
\begin{pgfscope}%
\pgfpathrectangle{\pgfqpoint{1.016621in}{0.499691in}}{\pgfqpoint{3.875000in}{2.695000in}}%
\pgfusepath{clip}%
\pgfsetbuttcap%
\pgfsetroundjoin%
\definecolor{currentfill}{rgb}{0.117647,0.564706,1.000000}%
\pgfsetfillcolor{currentfill}%
\pgfsetlinewidth{1.003750pt}%
\definecolor{currentstroke}{rgb}{0.117647,0.564706,1.000000}%
\pgfsetstrokecolor{currentstroke}%
\pgfsetdash{}{0pt}%
\pgfpathmoveto{\pgfqpoint{3.081535in}{1.744555in}}%
\pgfpathcurveto{\pgfqpoint{3.092585in}{1.744555in}}{\pgfqpoint{3.103184in}{1.748946in}}{\pgfqpoint{3.110998in}{1.756759in}}%
\pgfpathcurveto{\pgfqpoint{3.118812in}{1.764573in}}{\pgfqpoint{3.123202in}{1.775172in}}{\pgfqpoint{3.123202in}{1.786222in}}%
\pgfpathcurveto{\pgfqpoint{3.123202in}{1.797272in}}{\pgfqpoint{3.118812in}{1.807871in}}{\pgfqpoint{3.110998in}{1.815685in}}%
\pgfpathcurveto{\pgfqpoint{3.103184in}{1.823498in}}{\pgfqpoint{3.092585in}{1.827889in}}{\pgfqpoint{3.081535in}{1.827889in}}%
\pgfpathcurveto{\pgfqpoint{3.070485in}{1.827889in}}{\pgfqpoint{3.059886in}{1.823498in}}{\pgfqpoint{3.052073in}{1.815685in}}%
\pgfpathcurveto{\pgfqpoint{3.044259in}{1.807871in}}{\pgfqpoint{3.039869in}{1.797272in}}{\pgfqpoint{3.039869in}{1.786222in}}%
\pgfpathcurveto{\pgfqpoint{3.039869in}{1.775172in}}{\pgfqpoint{3.044259in}{1.764573in}}{\pgfqpoint{3.052073in}{1.756759in}}%
\pgfpathcurveto{\pgfqpoint{3.059886in}{1.748946in}}{\pgfqpoint{3.070485in}{1.744555in}}{\pgfqpoint{3.081535in}{1.744555in}}%
\pgfpathclose%
\pgfusepath{stroke,fill}%
\end{pgfscope}%
\begin{pgfscope}%
\pgfpathrectangle{\pgfqpoint{1.016621in}{0.499691in}}{\pgfqpoint{3.875000in}{2.695000in}}%
\pgfusepath{clip}%
\pgfsetbuttcap%
\pgfsetroundjoin%
\definecolor{currentfill}{rgb}{0.117647,0.564706,1.000000}%
\pgfsetfillcolor{currentfill}%
\pgfsetlinewidth{1.003750pt}%
\definecolor{currentstroke}{rgb}{0.117647,0.564706,1.000000}%
\pgfsetstrokecolor{currentstroke}%
\pgfsetdash{}{0pt}%
\pgfpathmoveto{\pgfqpoint{3.097227in}{1.712969in}}%
\pgfpathcurveto{\pgfqpoint{3.108277in}{1.712969in}}{\pgfqpoint{3.118876in}{1.717359in}}{\pgfqpoint{3.126690in}{1.725173in}}%
\pgfpathcurveto{\pgfqpoint{3.134503in}{1.732986in}}{\pgfqpoint{3.138893in}{1.743585in}}{\pgfqpoint{3.138893in}{1.754635in}}%
\pgfpathcurveto{\pgfqpoint{3.138893in}{1.765685in}}{\pgfqpoint{3.134503in}{1.776284in}}{\pgfqpoint{3.126690in}{1.784098in}}%
\pgfpathcurveto{\pgfqpoint{3.118876in}{1.791912in}}{\pgfqpoint{3.108277in}{1.796302in}}{\pgfqpoint{3.097227in}{1.796302in}}%
\pgfpathcurveto{\pgfqpoint{3.086177in}{1.796302in}}{\pgfqpoint{3.075578in}{1.791912in}}{\pgfqpoint{3.067764in}{1.784098in}}%
\pgfpathcurveto{\pgfqpoint{3.059950in}{1.776284in}}{\pgfqpoint{3.055560in}{1.765685in}}{\pgfqpoint{3.055560in}{1.754635in}}%
\pgfpathcurveto{\pgfqpoint{3.055560in}{1.743585in}}{\pgfqpoint{3.059950in}{1.732986in}}{\pgfqpoint{3.067764in}{1.725173in}}%
\pgfpathcurveto{\pgfqpoint{3.075578in}{1.717359in}}{\pgfqpoint{3.086177in}{1.712969in}}{\pgfqpoint{3.097227in}{1.712969in}}%
\pgfpathclose%
\pgfusepath{stroke,fill}%
\end{pgfscope}%
\begin{pgfscope}%
\pgfpathrectangle{\pgfqpoint{1.016621in}{0.499691in}}{\pgfqpoint{3.875000in}{2.695000in}}%
\pgfusepath{clip}%
\pgfsetbuttcap%
\pgfsetroundjoin%
\definecolor{currentfill}{rgb}{0.117647,0.564706,1.000000}%
\pgfsetfillcolor{currentfill}%
\pgfsetlinewidth{1.003750pt}%
\definecolor{currentstroke}{rgb}{0.117647,0.564706,1.000000}%
\pgfsetstrokecolor{currentstroke}%
\pgfsetdash{}{0pt}%
\pgfpathmoveto{\pgfqpoint{3.112918in}{1.725117in}}%
\pgfpathcurveto{\pgfqpoint{3.123968in}{1.725117in}}{\pgfqpoint{3.134567in}{1.729508in}}{\pgfqpoint{3.142381in}{1.737321in}}%
\pgfpathcurveto{\pgfqpoint{3.150195in}{1.745135in}}{\pgfqpoint{3.154585in}{1.755734in}}{\pgfqpoint{3.154585in}{1.766784in}}%
\pgfpathcurveto{\pgfqpoint{3.154585in}{1.777834in}}{\pgfqpoint{3.150195in}{1.788433in}}{\pgfqpoint{3.142381in}{1.796247in}}%
\pgfpathcurveto{\pgfqpoint{3.134567in}{1.804060in}}{\pgfqpoint{3.123968in}{1.808451in}}{\pgfqpoint{3.112918in}{1.808451in}}%
\pgfpathcurveto{\pgfqpoint{3.101868in}{1.808451in}}{\pgfqpoint{3.091269in}{1.804060in}}{\pgfqpoint{3.083455in}{1.796247in}}%
\pgfpathcurveto{\pgfqpoint{3.075642in}{1.788433in}}{\pgfqpoint{3.071251in}{1.777834in}}{\pgfqpoint{3.071251in}{1.766784in}}%
\pgfpathcurveto{\pgfqpoint{3.071251in}{1.755734in}}{\pgfqpoint{3.075642in}{1.745135in}}{\pgfqpoint{3.083455in}{1.737321in}}%
\pgfpathcurveto{\pgfqpoint{3.091269in}{1.729508in}}{\pgfqpoint{3.101868in}{1.725117in}}{\pgfqpoint{3.112918in}{1.725117in}}%
\pgfpathclose%
\pgfusepath{stroke,fill}%
\end{pgfscope}%
\begin{pgfscope}%
\pgfpathrectangle{\pgfqpoint{1.016621in}{0.499691in}}{\pgfqpoint{3.875000in}{2.695000in}}%
\pgfusepath{clip}%
\pgfsetbuttcap%
\pgfsetroundjoin%
\definecolor{currentfill}{rgb}{0.117647,0.564706,1.000000}%
\pgfsetfillcolor{currentfill}%
\pgfsetlinewidth{1.003750pt}%
\definecolor{currentstroke}{rgb}{0.117647,0.564706,1.000000}%
\pgfsetstrokecolor{currentstroke}%
\pgfsetdash{}{0pt}%
\pgfpathmoveto{\pgfqpoint{3.128610in}{1.746985in}}%
\pgfpathcurveto{\pgfqpoint{3.139660in}{1.746985in}}{\pgfqpoint{3.150259in}{1.751375in}}{\pgfqpoint{3.158072in}{1.759189in}}%
\pgfpathcurveto{\pgfqpoint{3.165886in}{1.767003in}}{\pgfqpoint{3.170276in}{1.777602in}}{\pgfqpoint{3.170276in}{1.788652in}}%
\pgfpathcurveto{\pgfqpoint{3.170276in}{1.799702in}}{\pgfqpoint{3.165886in}{1.810301in}}{\pgfqpoint{3.158072in}{1.818114in}}%
\pgfpathcurveto{\pgfqpoint{3.150259in}{1.825928in}}{\pgfqpoint{3.139660in}{1.830318in}}{\pgfqpoint{3.128610in}{1.830318in}}%
\pgfpathcurveto{\pgfqpoint{3.117559in}{1.830318in}}{\pgfqpoint{3.106960in}{1.825928in}}{\pgfqpoint{3.099147in}{1.818114in}}%
\pgfpathcurveto{\pgfqpoint{3.091333in}{1.810301in}}{\pgfqpoint{3.086943in}{1.799702in}}{\pgfqpoint{3.086943in}{1.788652in}}%
\pgfpathcurveto{\pgfqpoint{3.086943in}{1.777602in}}{\pgfqpoint{3.091333in}{1.767003in}}{\pgfqpoint{3.099147in}{1.759189in}}%
\pgfpathcurveto{\pgfqpoint{3.106960in}{1.751375in}}{\pgfqpoint{3.117559in}{1.746985in}}{\pgfqpoint{3.128610in}{1.746985in}}%
\pgfpathclose%
\pgfusepath{stroke,fill}%
\end{pgfscope}%
\begin{pgfscope}%
\pgfpathrectangle{\pgfqpoint{1.016621in}{0.499691in}}{\pgfqpoint{3.875000in}{2.695000in}}%
\pgfusepath{clip}%
\pgfsetbuttcap%
\pgfsetroundjoin%
\definecolor{currentfill}{rgb}{0.117647,0.564706,1.000000}%
\pgfsetfillcolor{currentfill}%
\pgfsetlinewidth{1.003750pt}%
\definecolor{currentstroke}{rgb}{0.117647,0.564706,1.000000}%
\pgfsetstrokecolor{currentstroke}%
\pgfsetdash{}{0pt}%
\pgfpathmoveto{\pgfqpoint{3.144301in}{1.822307in}}%
\pgfpathcurveto{\pgfqpoint{3.155351in}{1.822307in}}{\pgfqpoint{3.165950in}{1.826697in}}{\pgfqpoint{3.173764in}{1.834511in}}%
\pgfpathcurveto{\pgfqpoint{3.181577in}{1.842324in}}{\pgfqpoint{3.185968in}{1.852924in}}{\pgfqpoint{3.185968in}{1.863974in}}%
\pgfpathcurveto{\pgfqpoint{3.185968in}{1.875024in}}{\pgfqpoint{3.181577in}{1.885623in}}{\pgfqpoint{3.173764in}{1.893436in}}%
\pgfpathcurveto{\pgfqpoint{3.165950in}{1.901250in}}{\pgfqpoint{3.155351in}{1.905640in}}{\pgfqpoint{3.144301in}{1.905640in}}%
\pgfpathcurveto{\pgfqpoint{3.133251in}{1.905640in}}{\pgfqpoint{3.122652in}{1.901250in}}{\pgfqpoint{3.114838in}{1.893436in}}%
\pgfpathcurveto{\pgfqpoint{3.107025in}{1.885623in}}{\pgfqpoint{3.102634in}{1.875024in}}{\pgfqpoint{3.102634in}{1.863974in}}%
\pgfpathcurveto{\pgfqpoint{3.102634in}{1.852924in}}{\pgfqpoint{3.107025in}{1.842324in}}{\pgfqpoint{3.114838in}{1.834511in}}%
\pgfpathcurveto{\pgfqpoint{3.122652in}{1.826697in}}{\pgfqpoint{3.133251in}{1.822307in}}{\pgfqpoint{3.144301in}{1.822307in}}%
\pgfpathclose%
\pgfusepath{stroke,fill}%
\end{pgfscope}%
\begin{pgfscope}%
\pgfpathrectangle{\pgfqpoint{1.016621in}{0.499691in}}{\pgfqpoint{3.875000in}{2.695000in}}%
\pgfusepath{clip}%
\pgfsetbuttcap%
\pgfsetroundjoin%
\definecolor{currentfill}{rgb}{0.117647,0.564706,1.000000}%
\pgfsetfillcolor{currentfill}%
\pgfsetlinewidth{1.003750pt}%
\definecolor{currentstroke}{rgb}{0.117647,0.564706,1.000000}%
\pgfsetstrokecolor{currentstroke}%
\pgfsetdash{}{0pt}%
\pgfpathmoveto{\pgfqpoint{3.159992in}{1.863613in}}%
\pgfpathcurveto{\pgfqpoint{3.171043in}{1.863613in}}{\pgfqpoint{3.181642in}{1.868003in}}{\pgfqpoint{3.189455in}{1.875816in}}%
\pgfpathcurveto{\pgfqpoint{3.197269in}{1.883630in}}{\pgfqpoint{3.201659in}{1.894229in}}{\pgfqpoint{3.201659in}{1.905279in}}%
\pgfpathcurveto{\pgfqpoint{3.201659in}{1.916329in}}{\pgfqpoint{3.197269in}{1.926928in}}{\pgfqpoint{3.189455in}{1.934742in}}%
\pgfpathcurveto{\pgfqpoint{3.181642in}{1.942556in}}{\pgfqpoint{3.171043in}{1.946946in}}{\pgfqpoint{3.159992in}{1.946946in}}%
\pgfpathcurveto{\pgfqpoint{3.148942in}{1.946946in}}{\pgfqpoint{3.138343in}{1.942556in}}{\pgfqpoint{3.130530in}{1.934742in}}%
\pgfpathcurveto{\pgfqpoint{3.122716in}{1.926928in}}{\pgfqpoint{3.118326in}{1.916329in}}{\pgfqpoint{3.118326in}{1.905279in}}%
\pgfpathcurveto{\pgfqpoint{3.118326in}{1.894229in}}{\pgfqpoint{3.122716in}{1.883630in}}{\pgfqpoint{3.130530in}{1.875816in}}%
\pgfpathcurveto{\pgfqpoint{3.138343in}{1.868003in}}{\pgfqpoint{3.148942in}{1.863613in}}{\pgfqpoint{3.159992in}{1.863613in}}%
\pgfpathclose%
\pgfusepath{stroke,fill}%
\end{pgfscope}%
\begin{pgfscope}%
\pgfpathrectangle{\pgfqpoint{1.016621in}{0.499691in}}{\pgfqpoint{3.875000in}{2.695000in}}%
\pgfusepath{clip}%
\pgfsetbuttcap%
\pgfsetroundjoin%
\definecolor{currentfill}{rgb}{0.117647,0.564706,1.000000}%
\pgfsetfillcolor{currentfill}%
\pgfsetlinewidth{1.003750pt}%
\definecolor{currentstroke}{rgb}{0.117647,0.564706,1.000000}%
\pgfsetstrokecolor{currentstroke}%
\pgfsetdash{}{0pt}%
\pgfpathmoveto{\pgfqpoint{3.175684in}{1.912207in}}%
\pgfpathcurveto{\pgfqpoint{3.186734in}{1.912207in}}{\pgfqpoint{3.197333in}{1.916598in}}{\pgfqpoint{3.205147in}{1.924411in}}%
\pgfpathcurveto{\pgfqpoint{3.212960in}{1.932225in}}{\pgfqpoint{3.217351in}{1.942824in}}{\pgfqpoint{3.217351in}{1.953874in}}%
\pgfpathcurveto{\pgfqpoint{3.217351in}{1.964924in}}{\pgfqpoint{3.212960in}{1.975523in}}{\pgfqpoint{3.205147in}{1.983337in}}%
\pgfpathcurveto{\pgfqpoint{3.197333in}{1.991150in}}{\pgfqpoint{3.186734in}{1.995541in}}{\pgfqpoint{3.175684in}{1.995541in}}%
\pgfpathcurveto{\pgfqpoint{3.164634in}{1.995541in}}{\pgfqpoint{3.154035in}{1.991150in}}{\pgfqpoint{3.146221in}{1.983337in}}%
\pgfpathcurveto{\pgfqpoint{3.138407in}{1.975523in}}{\pgfqpoint{3.134017in}{1.964924in}}{\pgfqpoint{3.134017in}{1.953874in}}%
\pgfpathcurveto{\pgfqpoint{3.134017in}{1.942824in}}{\pgfqpoint{3.138407in}{1.932225in}}{\pgfqpoint{3.146221in}{1.924411in}}%
\pgfpathcurveto{\pgfqpoint{3.154035in}{1.916598in}}{\pgfqpoint{3.164634in}{1.912207in}}{\pgfqpoint{3.175684in}{1.912207in}}%
\pgfpathclose%
\pgfusepath{stroke,fill}%
\end{pgfscope}%
\begin{pgfscope}%
\pgfpathrectangle{\pgfqpoint{1.016621in}{0.499691in}}{\pgfqpoint{3.875000in}{2.695000in}}%
\pgfusepath{clip}%
\pgfsetbuttcap%
\pgfsetroundjoin%
\definecolor{currentfill}{rgb}{0.117647,0.564706,1.000000}%
\pgfsetfillcolor{currentfill}%
\pgfsetlinewidth{1.003750pt}%
\definecolor{currentstroke}{rgb}{0.117647,0.564706,1.000000}%
\pgfsetstrokecolor{currentstroke}%
\pgfsetdash{}{0pt}%
\pgfpathmoveto{\pgfqpoint{3.191375in}{1.917067in}}%
\pgfpathcurveto{\pgfqpoint{3.202425in}{1.917067in}}{\pgfqpoint{3.213024in}{1.921457in}}{\pgfqpoint{3.220838in}{1.929271in}}%
\pgfpathcurveto{\pgfqpoint{3.228652in}{1.937084in}}{\pgfqpoint{3.233042in}{1.947683in}}{\pgfqpoint{3.233042in}{1.958734in}}%
\pgfpathcurveto{\pgfqpoint{3.233042in}{1.969784in}}{\pgfqpoint{3.228652in}{1.980383in}}{\pgfqpoint{3.220838in}{1.988196in}}%
\pgfpathcurveto{\pgfqpoint{3.213024in}{1.996010in}}{\pgfqpoint{3.202425in}{2.000400in}}{\pgfqpoint{3.191375in}{2.000400in}}%
\pgfpathcurveto{\pgfqpoint{3.180325in}{2.000400in}}{\pgfqpoint{3.169726in}{1.996010in}}{\pgfqpoint{3.161913in}{1.988196in}}%
\pgfpathcurveto{\pgfqpoint{3.154099in}{1.980383in}}{\pgfqpoint{3.149709in}{1.969784in}}{\pgfqpoint{3.149709in}{1.958734in}}%
\pgfpathcurveto{\pgfqpoint{3.149709in}{1.947683in}}{\pgfqpoint{3.154099in}{1.937084in}}{\pgfqpoint{3.161913in}{1.929271in}}%
\pgfpathcurveto{\pgfqpoint{3.169726in}{1.921457in}}{\pgfqpoint{3.180325in}{1.917067in}}{\pgfqpoint{3.191375in}{1.917067in}}%
\pgfpathclose%
\pgfusepath{stroke,fill}%
\end{pgfscope}%
\begin{pgfscope}%
\pgfpathrectangle{\pgfqpoint{1.016621in}{0.499691in}}{\pgfqpoint{3.875000in}{2.695000in}}%
\pgfusepath{clip}%
\pgfsetbuttcap%
\pgfsetroundjoin%
\definecolor{currentfill}{rgb}{0.117647,0.564706,1.000000}%
\pgfsetfillcolor{currentfill}%
\pgfsetlinewidth{1.003750pt}%
\definecolor{currentstroke}{rgb}{0.117647,0.564706,1.000000}%
\pgfsetstrokecolor{currentstroke}%
\pgfsetdash{}{0pt}%
\pgfpathmoveto{\pgfqpoint{3.207067in}{1.941364in}}%
\pgfpathcurveto{\pgfqpoint{3.218117in}{1.941364in}}{\pgfqpoint{3.228716in}{1.945755in}}{\pgfqpoint{3.236530in}{1.953568in}}%
\pgfpathcurveto{\pgfqpoint{3.244343in}{1.961382in}}{\pgfqpoint{3.248733in}{1.971981in}}{\pgfqpoint{3.248733in}{1.983031in}}%
\pgfpathcurveto{\pgfqpoint{3.248733in}{1.994081in}}{\pgfqpoint{3.244343in}{2.004680in}}{\pgfqpoint{3.236530in}{2.012494in}}%
\pgfpathcurveto{\pgfqpoint{3.228716in}{2.020307in}}{\pgfqpoint{3.218117in}{2.024698in}}{\pgfqpoint{3.207067in}{2.024698in}}%
\pgfpathcurveto{\pgfqpoint{3.196017in}{2.024698in}}{\pgfqpoint{3.185418in}{2.020307in}}{\pgfqpoint{3.177604in}{2.012494in}}%
\pgfpathcurveto{\pgfqpoint{3.169790in}{2.004680in}}{\pgfqpoint{3.165400in}{1.994081in}}{\pgfqpoint{3.165400in}{1.983031in}}%
\pgfpathcurveto{\pgfqpoint{3.165400in}{1.971981in}}{\pgfqpoint{3.169790in}{1.961382in}}{\pgfqpoint{3.177604in}{1.953568in}}%
\pgfpathcurveto{\pgfqpoint{3.185418in}{1.945755in}}{\pgfqpoint{3.196017in}{1.941364in}}{\pgfqpoint{3.207067in}{1.941364in}}%
\pgfpathclose%
\pgfusepath{stroke,fill}%
\end{pgfscope}%
\begin{pgfscope}%
\pgfpathrectangle{\pgfqpoint{1.016621in}{0.499691in}}{\pgfqpoint{3.875000in}{2.695000in}}%
\pgfusepath{clip}%
\pgfsetbuttcap%
\pgfsetroundjoin%
\definecolor{currentfill}{rgb}{0.117647,0.564706,1.000000}%
\pgfsetfillcolor{currentfill}%
\pgfsetlinewidth{1.003750pt}%
\definecolor{currentstroke}{rgb}{0.117647,0.564706,1.000000}%
\pgfsetstrokecolor{currentstroke}%
\pgfsetdash{}{0pt}%
\pgfpathmoveto{\pgfqpoint{3.214912in}{1.955943in}}%
\pgfpathcurveto{\pgfqpoint{3.225963in}{1.955943in}}{\pgfqpoint{3.236562in}{1.960333in}}{\pgfqpoint{3.244375in}{1.968147in}}%
\pgfpathcurveto{\pgfqpoint{3.252189in}{1.975960in}}{\pgfqpoint{3.256579in}{1.986559in}}{\pgfqpoint{3.256579in}{1.997609in}}%
\pgfpathcurveto{\pgfqpoint{3.256579in}{2.008660in}}{\pgfqpoint{3.252189in}{2.019259in}}{\pgfqpoint{3.244375in}{2.027072in}}%
\pgfpathcurveto{\pgfqpoint{3.236562in}{2.034886in}}{\pgfqpoint{3.225963in}{2.039276in}}{\pgfqpoint{3.214912in}{2.039276in}}%
\pgfpathcurveto{\pgfqpoint{3.203862in}{2.039276in}}{\pgfqpoint{3.193263in}{2.034886in}}{\pgfqpoint{3.185450in}{2.027072in}}%
\pgfpathcurveto{\pgfqpoint{3.177636in}{2.019259in}}{\pgfqpoint{3.173246in}{2.008660in}}{\pgfqpoint{3.173246in}{1.997609in}}%
\pgfpathcurveto{\pgfqpoint{3.173246in}{1.986559in}}{\pgfqpoint{3.177636in}{1.975960in}}{\pgfqpoint{3.185450in}{1.968147in}}%
\pgfpathcurveto{\pgfqpoint{3.193263in}{1.960333in}}{\pgfqpoint{3.203862in}{1.955943in}}{\pgfqpoint{3.214912in}{1.955943in}}%
\pgfpathclose%
\pgfusepath{stroke,fill}%
\end{pgfscope}%
\begin{pgfscope}%
\pgfpathrectangle{\pgfqpoint{1.016621in}{0.499691in}}{\pgfqpoint{3.875000in}{2.695000in}}%
\pgfusepath{clip}%
\pgfsetbuttcap%
\pgfsetroundjoin%
\definecolor{currentfill}{rgb}{0.117647,0.564706,1.000000}%
\pgfsetfillcolor{currentfill}%
\pgfsetlinewidth{1.003750pt}%
\definecolor{currentstroke}{rgb}{0.117647,0.564706,1.000000}%
\pgfsetstrokecolor{currentstroke}%
\pgfsetdash{}{0pt}%
\pgfpathmoveto{\pgfqpoint{3.230604in}{1.960802in}}%
\pgfpathcurveto{\pgfqpoint{3.241654in}{1.960802in}}{\pgfqpoint{3.252253in}{1.965192in}}{\pgfqpoint{3.260067in}{1.973006in}}%
\pgfpathcurveto{\pgfqpoint{3.267880in}{1.980820in}}{\pgfqpoint{3.272271in}{1.991419in}}{\pgfqpoint{3.272271in}{2.002469in}}%
\pgfpathcurveto{\pgfqpoint{3.272271in}{2.013519in}}{\pgfqpoint{3.267880in}{2.024118in}}{\pgfqpoint{3.260067in}{2.031932in}}%
\pgfpathcurveto{\pgfqpoint{3.252253in}{2.039745in}}{\pgfqpoint{3.241654in}{2.044136in}}{\pgfqpoint{3.230604in}{2.044136in}}%
\pgfpathcurveto{\pgfqpoint{3.219554in}{2.044136in}}{\pgfqpoint{3.208955in}{2.039745in}}{\pgfqpoint{3.201141in}{2.031932in}}%
\pgfpathcurveto{\pgfqpoint{3.193328in}{2.024118in}}{\pgfqpoint{3.188937in}{2.013519in}}{\pgfqpoint{3.188937in}{2.002469in}}%
\pgfpathcurveto{\pgfqpoint{3.188937in}{1.991419in}}{\pgfqpoint{3.193328in}{1.980820in}}{\pgfqpoint{3.201141in}{1.973006in}}%
\pgfpathcurveto{\pgfqpoint{3.208955in}{1.965192in}}{\pgfqpoint{3.219554in}{1.960802in}}{\pgfqpoint{3.230604in}{1.960802in}}%
\pgfpathclose%
\pgfusepath{stroke,fill}%
\end{pgfscope}%
\begin{pgfscope}%
\pgfpathrectangle{\pgfqpoint{1.016621in}{0.499691in}}{\pgfqpoint{3.875000in}{2.695000in}}%
\pgfusepath{clip}%
\pgfsetbuttcap%
\pgfsetroundjoin%
\definecolor{currentfill}{rgb}{0.117647,0.564706,1.000000}%
\pgfsetfillcolor{currentfill}%
\pgfsetlinewidth{1.003750pt}%
\definecolor{currentstroke}{rgb}{0.117647,0.564706,1.000000}%
\pgfsetstrokecolor{currentstroke}%
\pgfsetdash{}{0pt}%
\pgfpathmoveto{\pgfqpoint{3.246295in}{2.011827in}}%
\pgfpathcurveto{\pgfqpoint{3.257345in}{2.011827in}}{\pgfqpoint{3.267945in}{2.016217in}}{\pgfqpoint{3.275758in}{2.024031in}}%
\pgfpathcurveto{\pgfqpoint{3.283572in}{2.031844in}}{\pgfqpoint{3.287962in}{2.042443in}}{\pgfqpoint{3.287962in}{2.053493in}}%
\pgfpathcurveto{\pgfqpoint{3.287962in}{2.064544in}}{\pgfqpoint{3.283572in}{2.075143in}}{\pgfqpoint{3.275758in}{2.082956in}}%
\pgfpathcurveto{\pgfqpoint{3.267945in}{2.090770in}}{\pgfqpoint{3.257345in}{2.095160in}}{\pgfqpoint{3.246295in}{2.095160in}}%
\pgfpathcurveto{\pgfqpoint{3.235245in}{2.095160in}}{\pgfqpoint{3.224646in}{2.090770in}}{\pgfqpoint{3.216833in}{2.082956in}}%
\pgfpathcurveto{\pgfqpoint{3.209019in}{2.075143in}}{\pgfqpoint{3.204629in}{2.064544in}}{\pgfqpoint{3.204629in}{2.053493in}}%
\pgfpathcurveto{\pgfqpoint{3.204629in}{2.042443in}}{\pgfqpoint{3.209019in}{2.031844in}}{\pgfqpoint{3.216833in}{2.024031in}}%
\pgfpathcurveto{\pgfqpoint{3.224646in}{2.016217in}}{\pgfqpoint{3.235245in}{2.011827in}}{\pgfqpoint{3.246295in}{2.011827in}}%
\pgfpathclose%
\pgfusepath{stroke,fill}%
\end{pgfscope}%
\begin{pgfscope}%
\pgfpathrectangle{\pgfqpoint{1.016621in}{0.499691in}}{\pgfqpoint{3.875000in}{2.695000in}}%
\pgfusepath{clip}%
\pgfsetbuttcap%
\pgfsetroundjoin%
\definecolor{currentfill}{rgb}{0.117647,0.564706,1.000000}%
\pgfsetfillcolor{currentfill}%
\pgfsetlinewidth{1.003750pt}%
\definecolor{currentstroke}{rgb}{0.117647,0.564706,1.000000}%
\pgfsetstrokecolor{currentstroke}%
\pgfsetdash{}{0pt}%
\pgfpathmoveto{\pgfqpoint{3.254141in}{2.023976in}}%
\pgfpathcurveto{\pgfqpoint{3.265191in}{2.023976in}}{\pgfqpoint{3.275790in}{2.028366in}}{\pgfqpoint{3.283604in}{2.036179in}}%
\pgfpathcurveto{\pgfqpoint{3.291417in}{2.043993in}}{\pgfqpoint{3.295808in}{2.054592in}}{\pgfqpoint{3.295808in}{2.065642in}}%
\pgfpathcurveto{\pgfqpoint{3.295808in}{2.076692in}}{\pgfqpoint{3.291417in}{2.087291in}}{\pgfqpoint{3.283604in}{2.095105in}}%
\pgfpathcurveto{\pgfqpoint{3.275790in}{2.102919in}}{\pgfqpoint{3.265191in}{2.107309in}}{\pgfqpoint{3.254141in}{2.107309in}}%
\pgfpathcurveto{\pgfqpoint{3.243091in}{2.107309in}}{\pgfqpoint{3.232492in}{2.102919in}}{\pgfqpoint{3.224678in}{2.095105in}}%
\pgfpathcurveto{\pgfqpoint{3.216865in}{2.087291in}}{\pgfqpoint{3.212474in}{2.076692in}}{\pgfqpoint{3.212474in}{2.065642in}}%
\pgfpathcurveto{\pgfqpoint{3.212474in}{2.054592in}}{\pgfqpoint{3.216865in}{2.043993in}}{\pgfqpoint{3.224678in}{2.036179in}}%
\pgfpathcurveto{\pgfqpoint{3.232492in}{2.028366in}}{\pgfqpoint{3.243091in}{2.023976in}}{\pgfqpoint{3.254141in}{2.023976in}}%
\pgfpathclose%
\pgfusepath{stroke,fill}%
\end{pgfscope}%
\begin{pgfscope}%
\pgfpathrectangle{\pgfqpoint{1.016621in}{0.499691in}}{\pgfqpoint{3.875000in}{2.695000in}}%
\pgfusepath{clip}%
\pgfsetbuttcap%
\pgfsetroundjoin%
\definecolor{currentfill}{rgb}{0.117647,0.564706,1.000000}%
\pgfsetfillcolor{currentfill}%
\pgfsetlinewidth{1.003750pt}%
\definecolor{currentstroke}{rgb}{0.117647,0.564706,1.000000}%
\pgfsetstrokecolor{currentstroke}%
\pgfsetdash{}{0pt}%
\pgfpathmoveto{\pgfqpoint{3.261987in}{2.055562in}}%
\pgfpathcurveto{\pgfqpoint{3.273037in}{2.055562in}}{\pgfqpoint{3.283636in}{2.059952in}}{\pgfqpoint{3.291450in}{2.067766in}}%
\pgfpathcurveto{\pgfqpoint{3.299263in}{2.075580in}}{\pgfqpoint{3.303653in}{2.086179in}}{\pgfqpoint{3.303653in}{2.097229in}}%
\pgfpathcurveto{\pgfqpoint{3.303653in}{2.108279in}}{\pgfqpoint{3.299263in}{2.118878in}}{\pgfqpoint{3.291450in}{2.126692in}}%
\pgfpathcurveto{\pgfqpoint{3.283636in}{2.134505in}}{\pgfqpoint{3.273037in}{2.138895in}}{\pgfqpoint{3.261987in}{2.138895in}}%
\pgfpathcurveto{\pgfqpoint{3.250937in}{2.138895in}}{\pgfqpoint{3.240338in}{2.134505in}}{\pgfqpoint{3.232524in}{2.126692in}}%
\pgfpathcurveto{\pgfqpoint{3.224710in}{2.118878in}}{\pgfqpoint{3.220320in}{2.108279in}}{\pgfqpoint{3.220320in}{2.097229in}}%
\pgfpathcurveto{\pgfqpoint{3.220320in}{2.086179in}}{\pgfqpoint{3.224710in}{2.075580in}}{\pgfqpoint{3.232524in}{2.067766in}}%
\pgfpathcurveto{\pgfqpoint{3.240338in}{2.059952in}}{\pgfqpoint{3.250937in}{2.055562in}}{\pgfqpoint{3.261987in}{2.055562in}}%
\pgfpathclose%
\pgfusepath{stroke,fill}%
\end{pgfscope}%
\begin{pgfscope}%
\pgfpathrectangle{\pgfqpoint{1.016621in}{0.499691in}}{\pgfqpoint{3.875000in}{2.695000in}}%
\pgfusepath{clip}%
\pgfsetbuttcap%
\pgfsetroundjoin%
\definecolor{currentfill}{rgb}{0.117647,0.564706,1.000000}%
\pgfsetfillcolor{currentfill}%
\pgfsetlinewidth{1.003750pt}%
\definecolor{currentstroke}{rgb}{0.117647,0.564706,1.000000}%
\pgfsetstrokecolor{currentstroke}%
\pgfsetdash{}{0pt}%
\pgfpathmoveto{\pgfqpoint{3.277678in}{2.075000in}}%
\pgfpathcurveto{\pgfqpoint{3.288728in}{2.075000in}}{\pgfqpoint{3.299327in}{2.079390in}}{\pgfqpoint{3.307141in}{2.087204in}}%
\pgfpathcurveto{\pgfqpoint{3.314955in}{2.095018in}}{\pgfqpoint{3.319345in}{2.105617in}}{\pgfqpoint{3.319345in}{2.116667in}}%
\pgfpathcurveto{\pgfqpoint{3.319345in}{2.127717in}}{\pgfqpoint{3.314955in}{2.138316in}}{\pgfqpoint{3.307141in}{2.146130in}}%
\pgfpathcurveto{\pgfqpoint{3.299327in}{2.153943in}}{\pgfqpoint{3.288728in}{2.158333in}}{\pgfqpoint{3.277678in}{2.158333in}}%
\pgfpathcurveto{\pgfqpoint{3.266628in}{2.158333in}}{\pgfqpoint{3.256029in}{2.153943in}}{\pgfqpoint{3.248215in}{2.146130in}}%
\pgfpathcurveto{\pgfqpoint{3.240402in}{2.138316in}}{\pgfqpoint{3.236012in}{2.127717in}}{\pgfqpoint{3.236012in}{2.116667in}}%
\pgfpathcurveto{\pgfqpoint{3.236012in}{2.105617in}}{\pgfqpoint{3.240402in}{2.095018in}}{\pgfqpoint{3.248215in}{2.087204in}}%
\pgfpathcurveto{\pgfqpoint{3.256029in}{2.079390in}}{\pgfqpoint{3.266628in}{2.075000in}}{\pgfqpoint{3.277678in}{2.075000in}}%
\pgfpathclose%
\pgfusepath{stroke,fill}%
\end{pgfscope}%
\begin{pgfscope}%
\pgfpathrectangle{\pgfqpoint{1.016621in}{0.499691in}}{\pgfqpoint{3.875000in}{2.695000in}}%
\pgfusepath{clip}%
\pgfsetbuttcap%
\pgfsetroundjoin%
\definecolor{currentfill}{rgb}{0.117647,0.564706,1.000000}%
\pgfsetfillcolor{currentfill}%
\pgfsetlinewidth{1.003750pt}%
\definecolor{currentstroke}{rgb}{0.117647,0.564706,1.000000}%
\pgfsetstrokecolor{currentstroke}%
\pgfsetdash{}{0pt}%
\pgfpathmoveto{\pgfqpoint{3.285524in}{2.096868in}}%
\pgfpathcurveto{\pgfqpoint{3.296574in}{2.096868in}}{\pgfqpoint{3.307173in}{2.101258in}}{\pgfqpoint{3.314987in}{2.109072in}}%
\pgfpathcurveto{\pgfqpoint{3.322800in}{2.116885in}}{\pgfqpoint{3.327191in}{2.127484in}}{\pgfqpoint{3.327191in}{2.138534in}}%
\pgfpathcurveto{\pgfqpoint{3.327191in}{2.149585in}}{\pgfqpoint{3.322800in}{2.160184in}}{\pgfqpoint{3.314987in}{2.167997in}}%
\pgfpathcurveto{\pgfqpoint{3.307173in}{2.175811in}}{\pgfqpoint{3.296574in}{2.180201in}}{\pgfqpoint{3.285524in}{2.180201in}}%
\pgfpathcurveto{\pgfqpoint{3.274474in}{2.180201in}}{\pgfqpoint{3.263875in}{2.175811in}}{\pgfqpoint{3.256061in}{2.167997in}}%
\pgfpathcurveto{\pgfqpoint{3.248248in}{2.160184in}}{\pgfqpoint{3.243857in}{2.149585in}}{\pgfqpoint{3.243857in}{2.138534in}}%
\pgfpathcurveto{\pgfqpoint{3.243857in}{2.127484in}}{\pgfqpoint{3.248248in}{2.116885in}}{\pgfqpoint{3.256061in}{2.109072in}}%
\pgfpathcurveto{\pgfqpoint{3.263875in}{2.101258in}}{\pgfqpoint{3.274474in}{2.096868in}}{\pgfqpoint{3.285524in}{2.096868in}}%
\pgfpathclose%
\pgfusepath{stroke,fill}%
\end{pgfscope}%
\begin{pgfscope}%
\pgfpathrectangle{\pgfqpoint{1.016621in}{0.499691in}}{\pgfqpoint{3.875000in}{2.695000in}}%
\pgfusepath{clip}%
\pgfsetbuttcap%
\pgfsetroundjoin%
\definecolor{currentfill}{rgb}{0.117647,0.564706,1.000000}%
\pgfsetfillcolor{currentfill}%
\pgfsetlinewidth{1.003750pt}%
\definecolor{currentstroke}{rgb}{0.117647,0.564706,1.000000}%
\pgfsetstrokecolor{currentstroke}%
\pgfsetdash{}{0pt}%
\pgfpathmoveto{\pgfqpoint{3.301215in}{2.140603in}}%
\pgfpathcurveto{\pgfqpoint{3.312266in}{2.140603in}}{\pgfqpoint{3.322865in}{2.144993in}}{\pgfqpoint{3.330678in}{2.152807in}}%
\pgfpathcurveto{\pgfqpoint{3.338492in}{2.160621in}}{\pgfqpoint{3.342882in}{2.171220in}}{\pgfqpoint{3.342882in}{2.182270in}}%
\pgfpathcurveto{\pgfqpoint{3.342882in}{2.193320in}}{\pgfqpoint{3.338492in}{2.203919in}}{\pgfqpoint{3.330678in}{2.211733in}}%
\pgfpathcurveto{\pgfqpoint{3.322865in}{2.219546in}}{\pgfqpoint{3.312266in}{2.223936in}}{\pgfqpoint{3.301215in}{2.223936in}}%
\pgfpathcurveto{\pgfqpoint{3.290165in}{2.223936in}}{\pgfqpoint{3.279566in}{2.219546in}}{\pgfqpoint{3.271753in}{2.211733in}}%
\pgfpathcurveto{\pgfqpoint{3.263939in}{2.203919in}}{\pgfqpoint{3.259549in}{2.193320in}}{\pgfqpoint{3.259549in}{2.182270in}}%
\pgfpathcurveto{\pgfqpoint{3.259549in}{2.171220in}}{\pgfqpoint{3.263939in}{2.160621in}}{\pgfqpoint{3.271753in}{2.152807in}}%
\pgfpathcurveto{\pgfqpoint{3.279566in}{2.144993in}}{\pgfqpoint{3.290165in}{2.140603in}}{\pgfqpoint{3.301215in}{2.140603in}}%
\pgfpathclose%
\pgfusepath{stroke,fill}%
\end{pgfscope}%
\begin{pgfscope}%
\pgfpathrectangle{\pgfqpoint{1.016621in}{0.499691in}}{\pgfqpoint{3.875000in}{2.695000in}}%
\pgfusepath{clip}%
\pgfsetbuttcap%
\pgfsetroundjoin%
\definecolor{currentfill}{rgb}{0.117647,0.564706,1.000000}%
\pgfsetfillcolor{currentfill}%
\pgfsetlinewidth{1.003750pt}%
\definecolor{currentstroke}{rgb}{0.117647,0.564706,1.000000}%
\pgfsetstrokecolor{currentstroke}%
\pgfsetdash{}{0pt}%
\pgfpathmoveto{\pgfqpoint{3.316907in}{2.218355in}}%
\pgfpathcurveto{\pgfqpoint{3.327957in}{2.218355in}}{\pgfqpoint{3.338556in}{2.222745in}}{\pgfqpoint{3.346370in}{2.230559in}}%
\pgfpathcurveto{\pgfqpoint{3.354183in}{2.238372in}}{\pgfqpoint{3.358573in}{2.248971in}}{\pgfqpoint{3.358573in}{2.260021in}}%
\pgfpathcurveto{\pgfqpoint{3.358573in}{2.271072in}}{\pgfqpoint{3.354183in}{2.281671in}}{\pgfqpoint{3.346370in}{2.289484in}}%
\pgfpathcurveto{\pgfqpoint{3.338556in}{2.297298in}}{\pgfqpoint{3.327957in}{2.301688in}}{\pgfqpoint{3.316907in}{2.301688in}}%
\pgfpathcurveto{\pgfqpoint{3.305857in}{2.301688in}}{\pgfqpoint{3.295258in}{2.297298in}}{\pgfqpoint{3.287444in}{2.289484in}}%
\pgfpathcurveto{\pgfqpoint{3.279630in}{2.281671in}}{\pgfqpoint{3.275240in}{2.271072in}}{\pgfqpoint{3.275240in}{2.260021in}}%
\pgfpathcurveto{\pgfqpoint{3.275240in}{2.248971in}}{\pgfqpoint{3.279630in}{2.238372in}}{\pgfqpoint{3.287444in}{2.230559in}}%
\pgfpathcurveto{\pgfqpoint{3.295258in}{2.222745in}}{\pgfqpoint{3.305857in}{2.218355in}}{\pgfqpoint{3.316907in}{2.218355in}}%
\pgfpathclose%
\pgfusepath{stroke,fill}%
\end{pgfscope}%
\begin{pgfscope}%
\pgfpathrectangle{\pgfqpoint{1.016621in}{0.499691in}}{\pgfqpoint{3.875000in}{2.695000in}}%
\pgfusepath{clip}%
\pgfsetbuttcap%
\pgfsetroundjoin%
\definecolor{currentfill}{rgb}{0.117647,0.564706,1.000000}%
\pgfsetfillcolor{currentfill}%
\pgfsetlinewidth{1.003750pt}%
\definecolor{currentstroke}{rgb}{0.117647,0.564706,1.000000}%
\pgfsetstrokecolor{currentstroke}%
\pgfsetdash{}{0pt}%
\pgfpathmoveto{\pgfqpoint{3.332598in}{2.235363in}}%
\pgfpathcurveto{\pgfqpoint{3.343648in}{2.235363in}}{\pgfqpoint{3.354247in}{2.239753in}}{\pgfqpoint{3.362061in}{2.247567in}}%
\pgfpathcurveto{\pgfqpoint{3.369875in}{2.255381in}}{\pgfqpoint{3.374265in}{2.265980in}}{\pgfqpoint{3.374265in}{2.277030in}}%
\pgfpathcurveto{\pgfqpoint{3.374265in}{2.288080in}}{\pgfqpoint{3.369875in}{2.298679in}}{\pgfqpoint{3.362061in}{2.306492in}}%
\pgfpathcurveto{\pgfqpoint{3.354247in}{2.314306in}}{\pgfqpoint{3.343648in}{2.318696in}}{\pgfqpoint{3.332598in}{2.318696in}}%
\pgfpathcurveto{\pgfqpoint{3.321548in}{2.318696in}}{\pgfqpoint{3.310949in}{2.314306in}}{\pgfqpoint{3.303135in}{2.306492in}}%
\pgfpathcurveto{\pgfqpoint{3.295322in}{2.298679in}}{\pgfqpoint{3.290932in}{2.288080in}}{\pgfqpoint{3.290932in}{2.277030in}}%
\pgfpathcurveto{\pgfqpoint{3.290932in}{2.265980in}}{\pgfqpoint{3.295322in}{2.255381in}}{\pgfqpoint{3.303135in}{2.247567in}}%
\pgfpathcurveto{\pgfqpoint{3.310949in}{2.239753in}}{\pgfqpoint{3.321548in}{2.235363in}}{\pgfqpoint{3.332598in}{2.235363in}}%
\pgfpathclose%
\pgfusepath{stroke,fill}%
\end{pgfscope}%
\begin{pgfscope}%
\pgfpathrectangle{\pgfqpoint{1.016621in}{0.499691in}}{\pgfqpoint{3.875000in}{2.695000in}}%
\pgfusepath{clip}%
\pgfsetbuttcap%
\pgfsetroundjoin%
\definecolor{currentfill}{rgb}{0.117647,0.564706,1.000000}%
\pgfsetfillcolor{currentfill}%
\pgfsetlinewidth{1.003750pt}%
\definecolor{currentstroke}{rgb}{0.117647,0.564706,1.000000}%
\pgfsetstrokecolor{currentstroke}%
\pgfsetdash{}{0pt}%
\pgfpathmoveto{\pgfqpoint{3.371827in}{2.262090in}}%
\pgfpathcurveto{\pgfqpoint{3.382877in}{2.262090in}}{\pgfqpoint{3.393476in}{2.266480in}}{\pgfqpoint{3.401290in}{2.274294in}}%
\pgfpathcurveto{\pgfqpoint{3.409103in}{2.282108in}}{\pgfqpoint{3.413494in}{2.292707in}}{\pgfqpoint{3.413494in}{2.303757in}}%
\pgfpathcurveto{\pgfqpoint{3.413494in}{2.314807in}}{\pgfqpoint{3.409103in}{2.325406in}}{\pgfqpoint{3.401290in}{2.333220in}}%
\pgfpathcurveto{\pgfqpoint{3.393476in}{2.341033in}}{\pgfqpoint{3.382877in}{2.345423in}}{\pgfqpoint{3.371827in}{2.345423in}}%
\pgfpathcurveto{\pgfqpoint{3.360777in}{2.345423in}}{\pgfqpoint{3.350178in}{2.341033in}}{\pgfqpoint{3.342364in}{2.333220in}}%
\pgfpathcurveto{\pgfqpoint{3.334550in}{2.325406in}}{\pgfqpoint{3.330160in}{2.314807in}}{\pgfqpoint{3.330160in}{2.303757in}}%
\pgfpathcurveto{\pgfqpoint{3.330160in}{2.292707in}}{\pgfqpoint{3.334550in}{2.282108in}}{\pgfqpoint{3.342364in}{2.274294in}}%
\pgfpathcurveto{\pgfqpoint{3.350178in}{2.266480in}}{\pgfqpoint{3.360777in}{2.262090in}}{\pgfqpoint{3.371827in}{2.262090in}}%
\pgfpathclose%
\pgfusepath{stroke,fill}%
\end{pgfscope}%
\begin{pgfscope}%
\pgfpathrectangle{\pgfqpoint{1.016621in}{0.499691in}}{\pgfqpoint{3.875000in}{2.695000in}}%
\pgfusepath{clip}%
\pgfsetbuttcap%
\pgfsetroundjoin%
\definecolor{currentfill}{rgb}{0.117647,0.564706,1.000000}%
\pgfsetfillcolor{currentfill}%
\pgfsetlinewidth{1.003750pt}%
\definecolor{currentstroke}{rgb}{0.117647,0.564706,1.000000}%
\pgfsetstrokecolor{currentstroke}%
\pgfsetdash{}{0pt}%
\pgfpathmoveto{\pgfqpoint{3.403210in}{2.359280in}}%
\pgfpathcurveto{\pgfqpoint{3.414260in}{2.359280in}}{\pgfqpoint{3.424859in}{2.363670in}}{\pgfqpoint{3.432673in}{2.371484in}}%
\pgfpathcurveto{\pgfqpoint{3.440486in}{2.379297in}}{\pgfqpoint{3.444876in}{2.389896in}}{\pgfqpoint{3.444876in}{2.400946in}}%
\pgfpathcurveto{\pgfqpoint{3.444876in}{2.411997in}}{\pgfqpoint{3.440486in}{2.422596in}}{\pgfqpoint{3.432673in}{2.430409in}}%
\pgfpathcurveto{\pgfqpoint{3.424859in}{2.438223in}}{\pgfqpoint{3.414260in}{2.442613in}}{\pgfqpoint{3.403210in}{2.442613in}}%
\pgfpathcurveto{\pgfqpoint{3.392160in}{2.442613in}}{\pgfqpoint{3.381561in}{2.438223in}}{\pgfqpoint{3.373747in}{2.430409in}}%
\pgfpathcurveto{\pgfqpoint{3.365933in}{2.422596in}}{\pgfqpoint{3.361543in}{2.411997in}}{\pgfqpoint{3.361543in}{2.400946in}}%
\pgfpathcurveto{\pgfqpoint{3.361543in}{2.389896in}}{\pgfqpoint{3.365933in}{2.379297in}}{\pgfqpoint{3.373747in}{2.371484in}}%
\pgfpathcurveto{\pgfqpoint{3.381561in}{2.363670in}}{\pgfqpoint{3.392160in}{2.359280in}}{\pgfqpoint{3.403210in}{2.359280in}}%
\pgfpathclose%
\pgfusepath{stroke,fill}%
\end{pgfscope}%
\begin{pgfscope}%
\pgfpathrectangle{\pgfqpoint{1.016621in}{0.499691in}}{\pgfqpoint{3.875000in}{2.695000in}}%
\pgfusepath{clip}%
\pgfsetbuttcap%
\pgfsetroundjoin%
\definecolor{currentfill}{rgb}{0.117647,0.564706,1.000000}%
\pgfsetfillcolor{currentfill}%
\pgfsetlinewidth{1.003750pt}%
\definecolor{currentstroke}{rgb}{0.117647,0.564706,1.000000}%
\pgfsetstrokecolor{currentstroke}%
\pgfsetdash{}{0pt}%
\pgfpathmoveto{\pgfqpoint{3.513050in}{2.505064in}}%
\pgfpathcurveto{\pgfqpoint{3.524100in}{2.505064in}}{\pgfqpoint{3.534699in}{2.509455in}}{\pgfqpoint{3.542513in}{2.517268in}}%
\pgfpathcurveto{\pgfqpoint{3.550326in}{2.525082in}}{\pgfqpoint{3.554716in}{2.535681in}}{\pgfqpoint{3.554716in}{2.546731in}}%
\pgfpathcurveto{\pgfqpoint{3.554716in}{2.557781in}}{\pgfqpoint{3.550326in}{2.568380in}}{\pgfqpoint{3.542513in}{2.576194in}}%
\pgfpathcurveto{\pgfqpoint{3.534699in}{2.584007in}}{\pgfqpoint{3.524100in}{2.588398in}}{\pgfqpoint{3.513050in}{2.588398in}}%
\pgfpathcurveto{\pgfqpoint{3.502000in}{2.588398in}}{\pgfqpoint{3.491401in}{2.584007in}}{\pgfqpoint{3.483587in}{2.576194in}}%
\pgfpathcurveto{\pgfqpoint{3.475773in}{2.568380in}}{\pgfqpoint{3.471383in}{2.557781in}}{\pgfqpoint{3.471383in}{2.546731in}}%
\pgfpathcurveto{\pgfqpoint{3.471383in}{2.535681in}}{\pgfqpoint{3.475773in}{2.525082in}}{\pgfqpoint{3.483587in}{2.517268in}}%
\pgfpathcurveto{\pgfqpoint{3.491401in}{2.509455in}}{\pgfqpoint{3.502000in}{2.505064in}}{\pgfqpoint{3.513050in}{2.505064in}}%
\pgfpathclose%
\pgfusepath{stroke,fill}%
\end{pgfscope}%
\begin{pgfscope}%
\pgfpathrectangle{\pgfqpoint{1.016621in}{0.499691in}}{\pgfqpoint{3.875000in}{2.695000in}}%
\pgfusepath{clip}%
\pgfsetbuttcap%
\pgfsetroundjoin%
\definecolor{currentfill}{rgb}{0.117647,0.564706,1.000000}%
\pgfsetfillcolor{currentfill}%
\pgfsetlinewidth{1.003750pt}%
\definecolor{currentstroke}{rgb}{0.117647,0.564706,1.000000}%
\pgfsetstrokecolor{currentstroke}%
\pgfsetdash{}{0pt}%
\pgfpathmoveto{\pgfqpoint{3.709193in}{2.650849in}}%
\pgfpathcurveto{\pgfqpoint{3.720243in}{2.650849in}}{\pgfqpoint{3.730842in}{2.655239in}}{\pgfqpoint{3.738655in}{2.663053in}}%
\pgfpathcurveto{\pgfqpoint{3.746469in}{2.670866in}}{\pgfqpoint{3.750859in}{2.681465in}}{\pgfqpoint{3.750859in}{2.692515in}}%
\pgfpathcurveto{\pgfqpoint{3.750859in}{2.703566in}}{\pgfqpoint{3.746469in}{2.714165in}}{\pgfqpoint{3.738655in}{2.721978in}}%
\pgfpathcurveto{\pgfqpoint{3.730842in}{2.729792in}}{\pgfqpoint{3.720243in}{2.734182in}}{\pgfqpoint{3.709193in}{2.734182in}}%
\pgfpathcurveto{\pgfqpoint{3.698143in}{2.734182in}}{\pgfqpoint{3.687544in}{2.729792in}}{\pgfqpoint{3.679730in}{2.721978in}}%
\pgfpathcurveto{\pgfqpoint{3.671916in}{2.714165in}}{\pgfqpoint{3.667526in}{2.703566in}}{\pgfqpoint{3.667526in}{2.692515in}}%
\pgfpathcurveto{\pgfqpoint{3.667526in}{2.681465in}}{\pgfqpoint{3.671916in}{2.670866in}}{\pgfqpoint{3.679730in}{2.663053in}}%
\pgfpathcurveto{\pgfqpoint{3.687544in}{2.655239in}}{\pgfqpoint{3.698143in}{2.650849in}}{\pgfqpoint{3.709193in}{2.650849in}}%
\pgfpathclose%
\pgfusepath{stroke,fill}%
\end{pgfscope}%
\begin{pgfscope}%
\pgfsetbuttcap%
\pgfsetroundjoin%
\definecolor{currentfill}{rgb}{0.000000,0.000000,0.000000}%
\pgfsetfillcolor{currentfill}%
\pgfsetlinewidth{0.803000pt}%
\definecolor{currentstroke}{rgb}{0.000000,0.000000,0.000000}%
\pgfsetstrokecolor{currentstroke}%
\pgfsetdash{}{0pt}%
\pgfsys@defobject{currentmarker}{\pgfqpoint{0.000000in}{-0.048611in}}{\pgfqpoint{0.000000in}{0.000000in}}{%
\pgfpathmoveto{\pgfqpoint{0.000000in}{0.000000in}}%
\pgfpathlineto{\pgfqpoint{0.000000in}{-0.048611in}}%
\pgfusepath{stroke,fill}%
}%
\begin{pgfscope}%
\pgfsys@transformshift{1.488855in}{0.499691in}%
\pgfsys@useobject{currentmarker}{}%
\end{pgfscope}%
\end{pgfscope}%
\begin{pgfscope}%
\definecolor{textcolor}{rgb}{0.000000,0.000000,0.000000}%
\pgfsetstrokecolor{textcolor}%
\pgfsetfillcolor{textcolor}%
\pgftext[x=1.488855in,y=0.402469in,,top]{\color{textcolor}\rmfamily\fontsize{10.000000}{12.000000}\selectfont \(\displaystyle 1\)}%
\end{pgfscope}%
\begin{pgfscope}%
\pgfsetbuttcap%
\pgfsetroundjoin%
\definecolor{currentfill}{rgb}{0.000000,0.000000,0.000000}%
\pgfsetfillcolor{currentfill}%
\pgfsetlinewidth{0.803000pt}%
\definecolor{currentstroke}{rgb}{0.000000,0.000000,0.000000}%
\pgfsetstrokecolor{currentstroke}%
\pgfsetdash{}{0pt}%
\pgfsys@defobject{currentmarker}{\pgfqpoint{0.000000in}{-0.048611in}}{\pgfqpoint{0.000000in}{0.000000in}}{%
\pgfpathmoveto{\pgfqpoint{0.000000in}{0.000000in}}%
\pgfpathlineto{\pgfqpoint{0.000000in}{-0.048611in}}%
\pgfusepath{stroke,fill}%
}%
\begin{pgfscope}%
\pgfsys@transformshift{2.273426in}{0.499691in}%
\pgfsys@useobject{currentmarker}{}%
\end{pgfscope}%
\end{pgfscope}%
\begin{pgfscope}%
\definecolor{textcolor}{rgb}{0.000000,0.000000,0.000000}%
\pgfsetstrokecolor{textcolor}%
\pgfsetfillcolor{textcolor}%
\pgftext[x=2.273426in,y=0.402469in,,top]{\color{textcolor}\rmfamily\fontsize{10.000000}{12.000000}\selectfont \(\displaystyle 2\)}%
\end{pgfscope}%
\begin{pgfscope}%
\pgfsetbuttcap%
\pgfsetroundjoin%
\definecolor{currentfill}{rgb}{0.000000,0.000000,0.000000}%
\pgfsetfillcolor{currentfill}%
\pgfsetlinewidth{0.803000pt}%
\definecolor{currentstroke}{rgb}{0.000000,0.000000,0.000000}%
\pgfsetstrokecolor{currentstroke}%
\pgfsetdash{}{0pt}%
\pgfsys@defobject{currentmarker}{\pgfqpoint{0.000000in}{-0.048611in}}{\pgfqpoint{0.000000in}{0.000000in}}{%
\pgfpathmoveto{\pgfqpoint{0.000000in}{0.000000in}}%
\pgfpathlineto{\pgfqpoint{0.000000in}{-0.048611in}}%
\pgfusepath{stroke,fill}%
}%
\begin{pgfscope}%
\pgfsys@transformshift{3.057998in}{0.499691in}%
\pgfsys@useobject{currentmarker}{}%
\end{pgfscope}%
\end{pgfscope}%
\begin{pgfscope}%
\definecolor{textcolor}{rgb}{0.000000,0.000000,0.000000}%
\pgfsetstrokecolor{textcolor}%
\pgfsetfillcolor{textcolor}%
\pgftext[x=3.057998in,y=0.402469in,,top]{\color{textcolor}\rmfamily\fontsize{10.000000}{12.000000}\selectfont \(\displaystyle 3\)}%
\end{pgfscope}%
\begin{pgfscope}%
\pgfsetbuttcap%
\pgfsetroundjoin%
\definecolor{currentfill}{rgb}{0.000000,0.000000,0.000000}%
\pgfsetfillcolor{currentfill}%
\pgfsetlinewidth{0.803000pt}%
\definecolor{currentstroke}{rgb}{0.000000,0.000000,0.000000}%
\pgfsetstrokecolor{currentstroke}%
\pgfsetdash{}{0pt}%
\pgfsys@defobject{currentmarker}{\pgfqpoint{0.000000in}{-0.048611in}}{\pgfqpoint{0.000000in}{0.000000in}}{%
\pgfpathmoveto{\pgfqpoint{0.000000in}{0.000000in}}%
\pgfpathlineto{\pgfqpoint{0.000000in}{-0.048611in}}%
\pgfusepath{stroke,fill}%
}%
\begin{pgfscope}%
\pgfsys@transformshift{3.842570in}{0.499691in}%
\pgfsys@useobject{currentmarker}{}%
\end{pgfscope}%
\end{pgfscope}%
\begin{pgfscope}%
\definecolor{textcolor}{rgb}{0.000000,0.000000,0.000000}%
\pgfsetstrokecolor{textcolor}%
\pgfsetfillcolor{textcolor}%
\pgftext[x=3.842570in,y=0.402469in,,top]{\color{textcolor}\rmfamily\fontsize{10.000000}{12.000000}\selectfont \(\displaystyle 4\)}%
\end{pgfscope}%
\begin{pgfscope}%
\pgfsetbuttcap%
\pgfsetroundjoin%
\definecolor{currentfill}{rgb}{0.000000,0.000000,0.000000}%
\pgfsetfillcolor{currentfill}%
\pgfsetlinewidth{0.803000pt}%
\definecolor{currentstroke}{rgb}{0.000000,0.000000,0.000000}%
\pgfsetstrokecolor{currentstroke}%
\pgfsetdash{}{0pt}%
\pgfsys@defobject{currentmarker}{\pgfqpoint{0.000000in}{-0.048611in}}{\pgfqpoint{0.000000in}{0.000000in}}{%
\pgfpathmoveto{\pgfqpoint{0.000000in}{0.000000in}}%
\pgfpathlineto{\pgfqpoint{0.000000in}{-0.048611in}}%
\pgfusepath{stroke,fill}%
}%
\begin{pgfscope}%
\pgfsys@transformshift{4.627142in}{0.499691in}%
\pgfsys@useobject{currentmarker}{}%
\end{pgfscope}%
\end{pgfscope}%
\begin{pgfscope}%
\definecolor{textcolor}{rgb}{0.000000,0.000000,0.000000}%
\pgfsetstrokecolor{textcolor}%
\pgfsetfillcolor{textcolor}%
\pgftext[x=4.627142in,y=0.402469in,,top]{\color{textcolor}\rmfamily\fontsize{10.000000}{12.000000}\selectfont \(\displaystyle 5\)}%
\end{pgfscope}%
\begin{pgfscope}%
\definecolor{textcolor}{rgb}{0.000000,0.000000,0.000000}%
\pgfsetstrokecolor{textcolor}%
\pgfsetfillcolor{textcolor}%
\pgftext[x=2.954121in,y=0.223457in,,top]{\color{textcolor}\rmfamily\fontsize{10.000000}{12.000000}\selectfont log f}%
\end{pgfscope}%
\begin{pgfscope}%
\pgfsetbuttcap%
\pgfsetroundjoin%
\definecolor{currentfill}{rgb}{0.000000,0.000000,0.000000}%
\pgfsetfillcolor{currentfill}%
\pgfsetlinewidth{0.803000pt}%
\definecolor{currentstroke}{rgb}{0.000000,0.000000,0.000000}%
\pgfsetstrokecolor{currentstroke}%
\pgfsetdash{}{0pt}%
\pgfsys@defobject{currentmarker}{\pgfqpoint{-0.048611in}{0.000000in}}{\pgfqpoint{0.000000in}{0.000000in}}{%
\pgfpathmoveto{\pgfqpoint{0.000000in}{0.000000in}}%
\pgfpathlineto{\pgfqpoint{-0.048611in}{0.000000in}}%
\pgfusepath{stroke,fill}%
}%
\begin{pgfscope}%
\pgfsys@transformshift{1.016621in}{0.617516in}%
\pgfsys@useobject{currentmarker}{}%
\end{pgfscope}%
\end{pgfscope}%
\begin{pgfscope}%
\definecolor{textcolor}{rgb}{0.000000,0.000000,0.000000}%
\pgfsetstrokecolor{textcolor}%
\pgfsetfillcolor{textcolor}%
\pgftext[x=0.603040in,y=0.569291in,left,base]{\color{textcolor}\rmfamily\fontsize{10.000000}{12.000000}\selectfont \(\displaystyle -100\)}%
\end{pgfscope}%
\begin{pgfscope}%
\pgfsetbuttcap%
\pgfsetroundjoin%
\definecolor{currentfill}{rgb}{0.000000,0.000000,0.000000}%
\pgfsetfillcolor{currentfill}%
\pgfsetlinewidth{0.803000pt}%
\definecolor{currentstroke}{rgb}{0.000000,0.000000,0.000000}%
\pgfsetstrokecolor{currentstroke}%
\pgfsetdash{}{0pt}%
\pgfsys@defobject{currentmarker}{\pgfqpoint{-0.048611in}{0.000000in}}{\pgfqpoint{0.000000in}{0.000000in}}{%
\pgfpathmoveto{\pgfqpoint{0.000000in}{0.000000in}}%
\pgfpathlineto{\pgfqpoint{-0.048611in}{0.000000in}}%
\pgfusepath{stroke,fill}%
}%
\begin{pgfscope}%
\pgfsys@transformshift{1.016621in}{1.103465in}%
\pgfsys@useobject{currentmarker}{}%
\end{pgfscope}%
\end{pgfscope}%
\begin{pgfscope}%
\definecolor{textcolor}{rgb}{0.000000,0.000000,0.000000}%
\pgfsetstrokecolor{textcolor}%
\pgfsetfillcolor{textcolor}%
\pgftext[x=0.672484in,y=1.055239in,left,base]{\color{textcolor}\rmfamily\fontsize{10.000000}{12.000000}\selectfont \(\displaystyle -80\)}%
\end{pgfscope}%
\begin{pgfscope}%
\pgfsetbuttcap%
\pgfsetroundjoin%
\definecolor{currentfill}{rgb}{0.000000,0.000000,0.000000}%
\pgfsetfillcolor{currentfill}%
\pgfsetlinewidth{0.803000pt}%
\definecolor{currentstroke}{rgb}{0.000000,0.000000,0.000000}%
\pgfsetstrokecolor{currentstroke}%
\pgfsetdash{}{0pt}%
\pgfsys@defobject{currentmarker}{\pgfqpoint{-0.048611in}{0.000000in}}{\pgfqpoint{0.000000in}{0.000000in}}{%
\pgfpathmoveto{\pgfqpoint{0.000000in}{0.000000in}}%
\pgfpathlineto{\pgfqpoint{-0.048611in}{0.000000in}}%
\pgfusepath{stroke,fill}%
}%
\begin{pgfscope}%
\pgfsys@transformshift{1.016621in}{1.589413in}%
\pgfsys@useobject{currentmarker}{}%
\end{pgfscope}%
\end{pgfscope}%
\begin{pgfscope}%
\definecolor{textcolor}{rgb}{0.000000,0.000000,0.000000}%
\pgfsetstrokecolor{textcolor}%
\pgfsetfillcolor{textcolor}%
\pgftext[x=0.672484in,y=1.541188in,left,base]{\color{textcolor}\rmfamily\fontsize{10.000000}{12.000000}\selectfont \(\displaystyle -60\)}%
\end{pgfscope}%
\begin{pgfscope}%
\pgfsetbuttcap%
\pgfsetroundjoin%
\definecolor{currentfill}{rgb}{0.000000,0.000000,0.000000}%
\pgfsetfillcolor{currentfill}%
\pgfsetlinewidth{0.803000pt}%
\definecolor{currentstroke}{rgb}{0.000000,0.000000,0.000000}%
\pgfsetstrokecolor{currentstroke}%
\pgfsetdash{}{0pt}%
\pgfsys@defobject{currentmarker}{\pgfqpoint{-0.048611in}{0.000000in}}{\pgfqpoint{0.000000in}{0.000000in}}{%
\pgfpathmoveto{\pgfqpoint{0.000000in}{0.000000in}}%
\pgfpathlineto{\pgfqpoint{-0.048611in}{0.000000in}}%
\pgfusepath{stroke,fill}%
}%
\begin{pgfscope}%
\pgfsys@transformshift{1.016621in}{2.075361in}%
\pgfsys@useobject{currentmarker}{}%
\end{pgfscope}%
\end{pgfscope}%
\begin{pgfscope}%
\definecolor{textcolor}{rgb}{0.000000,0.000000,0.000000}%
\pgfsetstrokecolor{textcolor}%
\pgfsetfillcolor{textcolor}%
\pgftext[x=0.672484in,y=2.027136in,left,base]{\color{textcolor}\rmfamily\fontsize{10.000000}{12.000000}\selectfont \(\displaystyle -40\)}%
\end{pgfscope}%
\begin{pgfscope}%
\pgfsetbuttcap%
\pgfsetroundjoin%
\definecolor{currentfill}{rgb}{0.000000,0.000000,0.000000}%
\pgfsetfillcolor{currentfill}%
\pgfsetlinewidth{0.803000pt}%
\definecolor{currentstroke}{rgb}{0.000000,0.000000,0.000000}%
\pgfsetstrokecolor{currentstroke}%
\pgfsetdash{}{0pt}%
\pgfsys@defobject{currentmarker}{\pgfqpoint{-0.048611in}{0.000000in}}{\pgfqpoint{0.000000in}{0.000000in}}{%
\pgfpathmoveto{\pgfqpoint{0.000000in}{0.000000in}}%
\pgfpathlineto{\pgfqpoint{-0.048611in}{0.000000in}}%
\pgfusepath{stroke,fill}%
}%
\begin{pgfscope}%
\pgfsys@transformshift{1.016621in}{2.561309in}%
\pgfsys@useobject{currentmarker}{}%
\end{pgfscope}%
\end{pgfscope}%
\begin{pgfscope}%
\definecolor{textcolor}{rgb}{0.000000,0.000000,0.000000}%
\pgfsetstrokecolor{textcolor}%
\pgfsetfillcolor{textcolor}%
\pgftext[x=0.672484in,y=2.513084in,left,base]{\color{textcolor}\rmfamily\fontsize{10.000000}{12.000000}\selectfont \(\displaystyle -20\)}%
\end{pgfscope}%
\begin{pgfscope}%
\pgfsetbuttcap%
\pgfsetroundjoin%
\definecolor{currentfill}{rgb}{0.000000,0.000000,0.000000}%
\pgfsetfillcolor{currentfill}%
\pgfsetlinewidth{0.803000pt}%
\definecolor{currentstroke}{rgb}{0.000000,0.000000,0.000000}%
\pgfsetstrokecolor{currentstroke}%
\pgfsetdash{}{0pt}%
\pgfsys@defobject{currentmarker}{\pgfqpoint{-0.048611in}{0.000000in}}{\pgfqpoint{0.000000in}{0.000000in}}{%
\pgfpathmoveto{\pgfqpoint{0.000000in}{0.000000in}}%
\pgfpathlineto{\pgfqpoint{-0.048611in}{0.000000in}}%
\pgfusepath{stroke,fill}%
}%
\begin{pgfscope}%
\pgfsys@transformshift{1.016621in}{3.047258in}%
\pgfsys@useobject{currentmarker}{}%
\end{pgfscope}%
\end{pgfscope}%
\begin{pgfscope}%
\definecolor{textcolor}{rgb}{0.000000,0.000000,0.000000}%
\pgfsetstrokecolor{textcolor}%
\pgfsetfillcolor{textcolor}%
\pgftext[x=0.849954in,y=2.999032in,left,base]{\color{textcolor}\rmfamily\fontsize{10.000000}{12.000000}\selectfont \(\displaystyle 0\)}%
\end{pgfscope}%
\begin{pgfscope}%
\definecolor{textcolor}{rgb}{0.000000,0.000000,0.000000}%
\pgfsetstrokecolor{textcolor}%
\pgfsetfillcolor{textcolor}%
\pgftext[x=0.255817in,y=1.847191in,,bottom]{\color{textcolor}\rmfamily\fontsize{10.000000}{12.000000}\selectfont \(\displaystyle \phi\) \textit{(º)}}%
\end{pgfscope}%
\begin{pgfscope}%
\pgfpathrectangle{\pgfqpoint{1.016621in}{0.499691in}}{\pgfqpoint{3.875000in}{2.695000in}}%
\pgfusepath{clip}%
\pgfsetbuttcap%
\pgfsetroundjoin%
\definecolor{currentfill}{rgb}{0.121569,0.466667,0.705882}%
\pgfsetfillcolor{currentfill}%
\pgfsetlinewidth{1.003750pt}%
\definecolor{currentstroke}{rgb}{0.121569,0.466667,0.705882}%
\pgfsetstrokecolor{currentstroke}%
\pgfsetdash{}{0pt}%
\pgfpathmoveto{\pgfqpoint{2.963850in}{1.462705in}}%
\pgfpathcurveto{\pgfqpoint{2.974900in}{1.462705in}}{\pgfqpoint{2.985499in}{1.467096in}}{\pgfqpoint{2.993312in}{1.474909in}}%
\pgfpathcurveto{\pgfqpoint{3.001126in}{1.482723in}}{\pgfqpoint{3.005516in}{1.493322in}}{\pgfqpoint{3.005516in}{1.504372in}}%
\pgfpathcurveto{\pgfqpoint{3.005516in}{1.515422in}}{\pgfqpoint{3.001126in}{1.526021in}}{\pgfqpoint{2.993312in}{1.533835in}}%
\pgfpathcurveto{\pgfqpoint{2.985499in}{1.541648in}}{\pgfqpoint{2.974900in}{1.546039in}}{\pgfqpoint{2.963850in}{1.546039in}}%
\pgfpathcurveto{\pgfqpoint{2.952799in}{1.546039in}}{\pgfqpoint{2.942200in}{1.541648in}}{\pgfqpoint{2.934387in}{1.533835in}}%
\pgfpathcurveto{\pgfqpoint{2.926573in}{1.526021in}}{\pgfqpoint{2.922183in}{1.515422in}}{\pgfqpoint{2.922183in}{1.504372in}}%
\pgfpathcurveto{\pgfqpoint{2.922183in}{1.493322in}}{\pgfqpoint{2.926573in}{1.482723in}}{\pgfqpoint{2.934387in}{1.474909in}}%
\pgfpathcurveto{\pgfqpoint{2.942200in}{1.467096in}}{\pgfqpoint{2.952799in}{1.462705in}}{\pgfqpoint{2.963850in}{1.462705in}}%
\pgfpathclose%
\pgfusepath{stroke,fill}%
\end{pgfscope}%
\begin{pgfscope}%
\pgfpathrectangle{\pgfqpoint{1.016621in}{0.499691in}}{\pgfqpoint{3.875000in}{2.695000in}}%
\pgfusepath{clip}%
\pgfsetbuttcap%
\pgfsetroundjoin%
\definecolor{currentfill}{rgb}{0.121569,0.466667,0.705882}%
\pgfsetfillcolor{currentfill}%
\pgfsetlinewidth{1.003750pt}%
\definecolor{currentstroke}{rgb}{0.121569,0.466667,0.705882}%
\pgfsetstrokecolor{currentstroke}%
\pgfsetdash{}{0pt}%
\pgfpathmoveto{\pgfqpoint{2.987387in}{1.428689in}}%
\pgfpathcurveto{\pgfqpoint{2.998437in}{1.428689in}}{\pgfqpoint{3.009036in}{1.433079in}}{\pgfqpoint{3.016849in}{1.440893in}}%
\pgfpathcurveto{\pgfqpoint{3.024663in}{1.448706in}}{\pgfqpoint{3.029053in}{1.459305in}}{\pgfqpoint{3.029053in}{1.470356in}}%
\pgfpathcurveto{\pgfqpoint{3.029053in}{1.481406in}}{\pgfqpoint{3.024663in}{1.492005in}}{\pgfqpoint{3.016849in}{1.499818in}}%
\pgfpathcurveto{\pgfqpoint{3.009036in}{1.507632in}}{\pgfqpoint{2.998437in}{1.512022in}}{\pgfqpoint{2.987387in}{1.512022in}}%
\pgfpathcurveto{\pgfqpoint{2.976337in}{1.512022in}}{\pgfqpoint{2.965738in}{1.507632in}}{\pgfqpoint{2.957924in}{1.499818in}}%
\pgfpathcurveto{\pgfqpoint{2.950110in}{1.492005in}}{\pgfqpoint{2.945720in}{1.481406in}}{\pgfqpoint{2.945720in}{1.470356in}}%
\pgfpathcurveto{\pgfqpoint{2.945720in}{1.459305in}}{\pgfqpoint{2.950110in}{1.448706in}}{\pgfqpoint{2.957924in}{1.440893in}}%
\pgfpathcurveto{\pgfqpoint{2.965738in}{1.433079in}}{\pgfqpoint{2.976337in}{1.428689in}}{\pgfqpoint{2.987387in}{1.428689in}}%
\pgfpathclose%
\pgfusepath{stroke,fill}%
\end{pgfscope}%
\begin{pgfscope}%
\pgfpathrectangle{\pgfqpoint{1.016621in}{0.499691in}}{\pgfqpoint{3.875000in}{2.695000in}}%
\pgfusepath{clip}%
\pgfsetbuttcap%
\pgfsetroundjoin%
\definecolor{currentfill}{rgb}{0.121569,0.466667,0.705882}%
\pgfsetfillcolor{currentfill}%
\pgfsetlinewidth{1.003750pt}%
\definecolor{currentstroke}{rgb}{0.121569,0.466667,0.705882}%
\pgfsetstrokecolor{currentstroke}%
\pgfsetdash{}{0pt}%
\pgfpathmoveto{\pgfqpoint{3.010924in}{1.404391in}}%
\pgfpathcurveto{\pgfqpoint{3.021974in}{1.404391in}}{\pgfqpoint{3.032573in}{1.408782in}}{\pgfqpoint{3.040387in}{1.416595in}}%
\pgfpathcurveto{\pgfqpoint{3.048200in}{1.424409in}}{\pgfqpoint{3.052590in}{1.435008in}}{\pgfqpoint{3.052590in}{1.446058in}}%
\pgfpathcurveto{\pgfqpoint{3.052590in}{1.457108in}}{\pgfqpoint{3.048200in}{1.467707in}}{\pgfqpoint{3.040387in}{1.475521in}}%
\pgfpathcurveto{\pgfqpoint{3.032573in}{1.483335in}}{\pgfqpoint{3.021974in}{1.487725in}}{\pgfqpoint{3.010924in}{1.487725in}}%
\pgfpathcurveto{\pgfqpoint{2.999874in}{1.487725in}}{\pgfqpoint{2.989275in}{1.483335in}}{\pgfqpoint{2.981461in}{1.475521in}}%
\pgfpathcurveto{\pgfqpoint{2.973647in}{1.467707in}}{\pgfqpoint{2.969257in}{1.457108in}}{\pgfqpoint{2.969257in}{1.446058in}}%
\pgfpathcurveto{\pgfqpoint{2.969257in}{1.435008in}}{\pgfqpoint{2.973647in}{1.424409in}}{\pgfqpoint{2.981461in}{1.416595in}}%
\pgfpathcurveto{\pgfqpoint{2.989275in}{1.408782in}}{\pgfqpoint{2.999874in}{1.404391in}}{\pgfqpoint{3.010924in}{1.404391in}}%
\pgfpathclose%
\pgfusepath{stroke,fill}%
\end{pgfscope}%
\begin{pgfscope}%
\pgfpathrectangle{\pgfqpoint{1.016621in}{0.499691in}}{\pgfqpoint{3.875000in}{2.695000in}}%
\pgfusepath{clip}%
\pgfsetbuttcap%
\pgfsetroundjoin%
\definecolor{currentfill}{rgb}{0.121569,0.466667,0.705882}%
\pgfsetfillcolor{currentfill}%
\pgfsetlinewidth{1.003750pt}%
\definecolor{currentstroke}{rgb}{0.121569,0.466667,0.705882}%
\pgfsetstrokecolor{currentstroke}%
\pgfsetdash{}{0pt}%
\pgfpathmoveto{\pgfqpoint{3.034461in}{1.525879in}}%
\pgfpathcurveto{\pgfqpoint{3.045511in}{1.525879in}}{\pgfqpoint{3.056110in}{1.530269in}}{\pgfqpoint{3.063924in}{1.538082in}}%
\pgfpathcurveto{\pgfqpoint{3.071737in}{1.545896in}}{\pgfqpoint{3.076128in}{1.556495in}}{\pgfqpoint{3.076128in}{1.567545in}}%
\pgfpathcurveto{\pgfqpoint{3.076128in}{1.578595in}}{\pgfqpoint{3.071737in}{1.589194in}}{\pgfqpoint{3.063924in}{1.597008in}}%
\pgfpathcurveto{\pgfqpoint{3.056110in}{1.604822in}}{\pgfqpoint{3.045511in}{1.609212in}}{\pgfqpoint{3.034461in}{1.609212in}}%
\pgfpathcurveto{\pgfqpoint{3.023411in}{1.609212in}}{\pgfqpoint{3.012812in}{1.604822in}}{\pgfqpoint{3.004998in}{1.597008in}}%
\pgfpathcurveto{\pgfqpoint{2.997185in}{1.589194in}}{\pgfqpoint{2.992794in}{1.578595in}}{\pgfqpoint{2.992794in}{1.567545in}}%
\pgfpathcurveto{\pgfqpoint{2.992794in}{1.556495in}}{\pgfqpoint{2.997185in}{1.545896in}}{\pgfqpoint{3.004998in}{1.538082in}}%
\pgfpathcurveto{\pgfqpoint{3.012812in}{1.530269in}}{\pgfqpoint{3.023411in}{1.525879in}}{\pgfqpoint{3.034461in}{1.525879in}}%
\pgfpathclose%
\pgfusepath{stroke,fill}%
\end{pgfscope}%
\begin{pgfscope}%
\pgfpathrectangle{\pgfqpoint{1.016621in}{0.499691in}}{\pgfqpoint{3.875000in}{2.695000in}}%
\pgfusepath{clip}%
\pgfsetbuttcap%
\pgfsetroundjoin%
\definecolor{currentfill}{rgb}{0.121569,0.466667,0.705882}%
\pgfsetfillcolor{currentfill}%
\pgfsetlinewidth{1.003750pt}%
\definecolor{currentstroke}{rgb}{0.121569,0.466667,0.705882}%
\pgfsetstrokecolor{currentstroke}%
\pgfsetdash{}{0pt}%
\pgfpathmoveto{\pgfqpoint{3.057998in}{1.606060in}}%
\pgfpathcurveto{\pgfqpoint{3.069048in}{1.606060in}}{\pgfqpoint{3.079647in}{1.610450in}}{\pgfqpoint{3.087461in}{1.618264in}}%
\pgfpathcurveto{\pgfqpoint{3.095275in}{1.626078in}}{\pgfqpoint{3.099665in}{1.636677in}}{\pgfqpoint{3.099665in}{1.647727in}}%
\pgfpathcurveto{\pgfqpoint{3.099665in}{1.658777in}}{\pgfqpoint{3.095275in}{1.669376in}}{\pgfqpoint{3.087461in}{1.677189in}}%
\pgfpathcurveto{\pgfqpoint{3.079647in}{1.685003in}}{\pgfqpoint{3.069048in}{1.689393in}}{\pgfqpoint{3.057998in}{1.689393in}}%
\pgfpathcurveto{\pgfqpoint{3.046948in}{1.689393in}}{\pgfqpoint{3.036349in}{1.685003in}}{\pgfqpoint{3.028535in}{1.677189in}}%
\pgfpathcurveto{\pgfqpoint{3.020722in}{1.669376in}}{\pgfqpoint{3.016331in}{1.658777in}}{\pgfqpoint{3.016331in}{1.647727in}}%
\pgfpathcurveto{\pgfqpoint{3.016331in}{1.636677in}}{\pgfqpoint{3.020722in}{1.626078in}}{\pgfqpoint{3.028535in}{1.618264in}}%
\pgfpathcurveto{\pgfqpoint{3.036349in}{1.610450in}}{\pgfqpoint{3.046948in}{1.606060in}}{\pgfqpoint{3.057998in}{1.606060in}}%
\pgfpathclose%
\pgfusepath{stroke,fill}%
\end{pgfscope}%
\begin{pgfscope}%
\pgfpathrectangle{\pgfqpoint{1.016621in}{0.499691in}}{\pgfqpoint{3.875000in}{2.695000in}}%
\pgfusepath{clip}%
\pgfsetbuttcap%
\pgfsetroundjoin%
\definecolor{currentfill}{rgb}{0.121569,0.466667,0.705882}%
\pgfsetfillcolor{currentfill}%
\pgfsetlinewidth{1.003750pt}%
\definecolor{currentstroke}{rgb}{0.121569,0.466667,0.705882}%
\pgfsetstrokecolor{currentstroke}%
\pgfsetdash{}{0pt}%
\pgfpathmoveto{\pgfqpoint{3.081535in}{1.744555in}}%
\pgfpathcurveto{\pgfqpoint{3.092585in}{1.744555in}}{\pgfqpoint{3.103184in}{1.748946in}}{\pgfqpoint{3.110998in}{1.756759in}}%
\pgfpathcurveto{\pgfqpoint{3.118812in}{1.764573in}}{\pgfqpoint{3.123202in}{1.775172in}}{\pgfqpoint{3.123202in}{1.786222in}}%
\pgfpathcurveto{\pgfqpoint{3.123202in}{1.797272in}}{\pgfqpoint{3.118812in}{1.807871in}}{\pgfqpoint{3.110998in}{1.815685in}}%
\pgfpathcurveto{\pgfqpoint{3.103184in}{1.823498in}}{\pgfqpoint{3.092585in}{1.827889in}}{\pgfqpoint{3.081535in}{1.827889in}}%
\pgfpathcurveto{\pgfqpoint{3.070485in}{1.827889in}}{\pgfqpoint{3.059886in}{1.823498in}}{\pgfqpoint{3.052073in}{1.815685in}}%
\pgfpathcurveto{\pgfqpoint{3.044259in}{1.807871in}}{\pgfqpoint{3.039869in}{1.797272in}}{\pgfqpoint{3.039869in}{1.786222in}}%
\pgfpathcurveto{\pgfqpoint{3.039869in}{1.775172in}}{\pgfqpoint{3.044259in}{1.764573in}}{\pgfqpoint{3.052073in}{1.756759in}}%
\pgfpathcurveto{\pgfqpoint{3.059886in}{1.748946in}}{\pgfqpoint{3.070485in}{1.744555in}}{\pgfqpoint{3.081535in}{1.744555in}}%
\pgfpathclose%
\pgfusepath{stroke,fill}%
\end{pgfscope}%
\begin{pgfscope}%
\pgfpathrectangle{\pgfqpoint{1.016621in}{0.499691in}}{\pgfqpoint{3.875000in}{2.695000in}}%
\pgfusepath{clip}%
\pgfsetbuttcap%
\pgfsetroundjoin%
\definecolor{currentfill}{rgb}{0.121569,0.466667,0.705882}%
\pgfsetfillcolor{currentfill}%
\pgfsetlinewidth{1.003750pt}%
\definecolor{currentstroke}{rgb}{0.121569,0.466667,0.705882}%
\pgfsetstrokecolor{currentstroke}%
\pgfsetdash{}{0pt}%
\pgfpathmoveto{\pgfqpoint{3.097227in}{1.712969in}}%
\pgfpathcurveto{\pgfqpoint{3.108277in}{1.712969in}}{\pgfqpoint{3.118876in}{1.717359in}}{\pgfqpoint{3.126690in}{1.725173in}}%
\pgfpathcurveto{\pgfqpoint{3.134503in}{1.732986in}}{\pgfqpoint{3.138893in}{1.743585in}}{\pgfqpoint{3.138893in}{1.754635in}}%
\pgfpathcurveto{\pgfqpoint{3.138893in}{1.765685in}}{\pgfqpoint{3.134503in}{1.776284in}}{\pgfqpoint{3.126690in}{1.784098in}}%
\pgfpathcurveto{\pgfqpoint{3.118876in}{1.791912in}}{\pgfqpoint{3.108277in}{1.796302in}}{\pgfqpoint{3.097227in}{1.796302in}}%
\pgfpathcurveto{\pgfqpoint{3.086177in}{1.796302in}}{\pgfqpoint{3.075578in}{1.791912in}}{\pgfqpoint{3.067764in}{1.784098in}}%
\pgfpathcurveto{\pgfqpoint{3.059950in}{1.776284in}}{\pgfqpoint{3.055560in}{1.765685in}}{\pgfqpoint{3.055560in}{1.754635in}}%
\pgfpathcurveto{\pgfqpoint{3.055560in}{1.743585in}}{\pgfqpoint{3.059950in}{1.732986in}}{\pgfqpoint{3.067764in}{1.725173in}}%
\pgfpathcurveto{\pgfqpoint{3.075578in}{1.717359in}}{\pgfqpoint{3.086177in}{1.712969in}}{\pgfqpoint{3.097227in}{1.712969in}}%
\pgfpathclose%
\pgfusepath{stroke,fill}%
\end{pgfscope}%
\begin{pgfscope}%
\pgfpathrectangle{\pgfqpoint{1.016621in}{0.499691in}}{\pgfqpoint{3.875000in}{2.695000in}}%
\pgfusepath{clip}%
\pgfsetbuttcap%
\pgfsetroundjoin%
\definecolor{currentfill}{rgb}{0.121569,0.466667,0.705882}%
\pgfsetfillcolor{currentfill}%
\pgfsetlinewidth{1.003750pt}%
\definecolor{currentstroke}{rgb}{0.121569,0.466667,0.705882}%
\pgfsetstrokecolor{currentstroke}%
\pgfsetdash{}{0pt}%
\pgfpathmoveto{\pgfqpoint{3.112918in}{1.725117in}}%
\pgfpathcurveto{\pgfqpoint{3.123968in}{1.725117in}}{\pgfqpoint{3.134567in}{1.729508in}}{\pgfqpoint{3.142381in}{1.737321in}}%
\pgfpathcurveto{\pgfqpoint{3.150195in}{1.745135in}}{\pgfqpoint{3.154585in}{1.755734in}}{\pgfqpoint{3.154585in}{1.766784in}}%
\pgfpathcurveto{\pgfqpoint{3.154585in}{1.777834in}}{\pgfqpoint{3.150195in}{1.788433in}}{\pgfqpoint{3.142381in}{1.796247in}}%
\pgfpathcurveto{\pgfqpoint{3.134567in}{1.804060in}}{\pgfqpoint{3.123968in}{1.808451in}}{\pgfqpoint{3.112918in}{1.808451in}}%
\pgfpathcurveto{\pgfqpoint{3.101868in}{1.808451in}}{\pgfqpoint{3.091269in}{1.804060in}}{\pgfqpoint{3.083455in}{1.796247in}}%
\pgfpathcurveto{\pgfqpoint{3.075642in}{1.788433in}}{\pgfqpoint{3.071251in}{1.777834in}}{\pgfqpoint{3.071251in}{1.766784in}}%
\pgfpathcurveto{\pgfqpoint{3.071251in}{1.755734in}}{\pgfqpoint{3.075642in}{1.745135in}}{\pgfqpoint{3.083455in}{1.737321in}}%
\pgfpathcurveto{\pgfqpoint{3.091269in}{1.729508in}}{\pgfqpoint{3.101868in}{1.725117in}}{\pgfqpoint{3.112918in}{1.725117in}}%
\pgfpathclose%
\pgfusepath{stroke,fill}%
\end{pgfscope}%
\begin{pgfscope}%
\pgfpathrectangle{\pgfqpoint{1.016621in}{0.499691in}}{\pgfqpoint{3.875000in}{2.695000in}}%
\pgfusepath{clip}%
\pgfsetbuttcap%
\pgfsetroundjoin%
\definecolor{currentfill}{rgb}{0.121569,0.466667,0.705882}%
\pgfsetfillcolor{currentfill}%
\pgfsetlinewidth{1.003750pt}%
\definecolor{currentstroke}{rgb}{0.121569,0.466667,0.705882}%
\pgfsetstrokecolor{currentstroke}%
\pgfsetdash{}{0pt}%
\pgfpathmoveto{\pgfqpoint{3.128610in}{1.746985in}}%
\pgfpathcurveto{\pgfqpoint{3.139660in}{1.746985in}}{\pgfqpoint{3.150259in}{1.751375in}}{\pgfqpoint{3.158072in}{1.759189in}}%
\pgfpathcurveto{\pgfqpoint{3.165886in}{1.767003in}}{\pgfqpoint{3.170276in}{1.777602in}}{\pgfqpoint{3.170276in}{1.788652in}}%
\pgfpathcurveto{\pgfqpoint{3.170276in}{1.799702in}}{\pgfqpoint{3.165886in}{1.810301in}}{\pgfqpoint{3.158072in}{1.818114in}}%
\pgfpathcurveto{\pgfqpoint{3.150259in}{1.825928in}}{\pgfqpoint{3.139660in}{1.830318in}}{\pgfqpoint{3.128610in}{1.830318in}}%
\pgfpathcurveto{\pgfqpoint{3.117559in}{1.830318in}}{\pgfqpoint{3.106960in}{1.825928in}}{\pgfqpoint{3.099147in}{1.818114in}}%
\pgfpathcurveto{\pgfqpoint{3.091333in}{1.810301in}}{\pgfqpoint{3.086943in}{1.799702in}}{\pgfqpoint{3.086943in}{1.788652in}}%
\pgfpathcurveto{\pgfqpoint{3.086943in}{1.777602in}}{\pgfqpoint{3.091333in}{1.767003in}}{\pgfqpoint{3.099147in}{1.759189in}}%
\pgfpathcurveto{\pgfqpoint{3.106960in}{1.751375in}}{\pgfqpoint{3.117559in}{1.746985in}}{\pgfqpoint{3.128610in}{1.746985in}}%
\pgfpathclose%
\pgfusepath{stroke,fill}%
\end{pgfscope}%
\begin{pgfscope}%
\pgfpathrectangle{\pgfqpoint{1.016621in}{0.499691in}}{\pgfqpoint{3.875000in}{2.695000in}}%
\pgfusepath{clip}%
\pgfsetbuttcap%
\pgfsetroundjoin%
\definecolor{currentfill}{rgb}{0.121569,0.466667,0.705882}%
\pgfsetfillcolor{currentfill}%
\pgfsetlinewidth{1.003750pt}%
\definecolor{currentstroke}{rgb}{0.121569,0.466667,0.705882}%
\pgfsetstrokecolor{currentstroke}%
\pgfsetdash{}{0pt}%
\pgfpathmoveto{\pgfqpoint{3.144301in}{1.822307in}}%
\pgfpathcurveto{\pgfqpoint{3.155351in}{1.822307in}}{\pgfqpoint{3.165950in}{1.826697in}}{\pgfqpoint{3.173764in}{1.834511in}}%
\pgfpathcurveto{\pgfqpoint{3.181577in}{1.842324in}}{\pgfqpoint{3.185968in}{1.852924in}}{\pgfqpoint{3.185968in}{1.863974in}}%
\pgfpathcurveto{\pgfqpoint{3.185968in}{1.875024in}}{\pgfqpoint{3.181577in}{1.885623in}}{\pgfqpoint{3.173764in}{1.893436in}}%
\pgfpathcurveto{\pgfqpoint{3.165950in}{1.901250in}}{\pgfqpoint{3.155351in}{1.905640in}}{\pgfqpoint{3.144301in}{1.905640in}}%
\pgfpathcurveto{\pgfqpoint{3.133251in}{1.905640in}}{\pgfqpoint{3.122652in}{1.901250in}}{\pgfqpoint{3.114838in}{1.893436in}}%
\pgfpathcurveto{\pgfqpoint{3.107025in}{1.885623in}}{\pgfqpoint{3.102634in}{1.875024in}}{\pgfqpoint{3.102634in}{1.863974in}}%
\pgfpathcurveto{\pgfqpoint{3.102634in}{1.852924in}}{\pgfqpoint{3.107025in}{1.842324in}}{\pgfqpoint{3.114838in}{1.834511in}}%
\pgfpathcurveto{\pgfqpoint{3.122652in}{1.826697in}}{\pgfqpoint{3.133251in}{1.822307in}}{\pgfqpoint{3.144301in}{1.822307in}}%
\pgfpathclose%
\pgfusepath{stroke,fill}%
\end{pgfscope}%
\begin{pgfscope}%
\pgfpathrectangle{\pgfqpoint{1.016621in}{0.499691in}}{\pgfqpoint{3.875000in}{2.695000in}}%
\pgfusepath{clip}%
\pgfsetbuttcap%
\pgfsetroundjoin%
\definecolor{currentfill}{rgb}{0.121569,0.466667,0.705882}%
\pgfsetfillcolor{currentfill}%
\pgfsetlinewidth{1.003750pt}%
\definecolor{currentstroke}{rgb}{0.121569,0.466667,0.705882}%
\pgfsetstrokecolor{currentstroke}%
\pgfsetdash{}{0pt}%
\pgfpathmoveto{\pgfqpoint{3.159992in}{1.863613in}}%
\pgfpathcurveto{\pgfqpoint{3.171043in}{1.863613in}}{\pgfqpoint{3.181642in}{1.868003in}}{\pgfqpoint{3.189455in}{1.875816in}}%
\pgfpathcurveto{\pgfqpoint{3.197269in}{1.883630in}}{\pgfqpoint{3.201659in}{1.894229in}}{\pgfqpoint{3.201659in}{1.905279in}}%
\pgfpathcurveto{\pgfqpoint{3.201659in}{1.916329in}}{\pgfqpoint{3.197269in}{1.926928in}}{\pgfqpoint{3.189455in}{1.934742in}}%
\pgfpathcurveto{\pgfqpoint{3.181642in}{1.942556in}}{\pgfqpoint{3.171043in}{1.946946in}}{\pgfqpoint{3.159992in}{1.946946in}}%
\pgfpathcurveto{\pgfqpoint{3.148942in}{1.946946in}}{\pgfqpoint{3.138343in}{1.942556in}}{\pgfqpoint{3.130530in}{1.934742in}}%
\pgfpathcurveto{\pgfqpoint{3.122716in}{1.926928in}}{\pgfqpoint{3.118326in}{1.916329in}}{\pgfqpoint{3.118326in}{1.905279in}}%
\pgfpathcurveto{\pgfqpoint{3.118326in}{1.894229in}}{\pgfqpoint{3.122716in}{1.883630in}}{\pgfqpoint{3.130530in}{1.875816in}}%
\pgfpathcurveto{\pgfqpoint{3.138343in}{1.868003in}}{\pgfqpoint{3.148942in}{1.863613in}}{\pgfqpoint{3.159992in}{1.863613in}}%
\pgfpathclose%
\pgfusepath{stroke,fill}%
\end{pgfscope}%
\begin{pgfscope}%
\pgfpathrectangle{\pgfqpoint{1.016621in}{0.499691in}}{\pgfqpoint{3.875000in}{2.695000in}}%
\pgfusepath{clip}%
\pgfsetbuttcap%
\pgfsetroundjoin%
\definecolor{currentfill}{rgb}{0.121569,0.466667,0.705882}%
\pgfsetfillcolor{currentfill}%
\pgfsetlinewidth{1.003750pt}%
\definecolor{currentstroke}{rgb}{0.121569,0.466667,0.705882}%
\pgfsetstrokecolor{currentstroke}%
\pgfsetdash{}{0pt}%
\pgfpathmoveto{\pgfqpoint{3.175684in}{1.912207in}}%
\pgfpathcurveto{\pgfqpoint{3.186734in}{1.912207in}}{\pgfqpoint{3.197333in}{1.916598in}}{\pgfqpoint{3.205147in}{1.924411in}}%
\pgfpathcurveto{\pgfqpoint{3.212960in}{1.932225in}}{\pgfqpoint{3.217351in}{1.942824in}}{\pgfqpoint{3.217351in}{1.953874in}}%
\pgfpathcurveto{\pgfqpoint{3.217351in}{1.964924in}}{\pgfqpoint{3.212960in}{1.975523in}}{\pgfqpoint{3.205147in}{1.983337in}}%
\pgfpathcurveto{\pgfqpoint{3.197333in}{1.991150in}}{\pgfqpoint{3.186734in}{1.995541in}}{\pgfqpoint{3.175684in}{1.995541in}}%
\pgfpathcurveto{\pgfqpoint{3.164634in}{1.995541in}}{\pgfqpoint{3.154035in}{1.991150in}}{\pgfqpoint{3.146221in}{1.983337in}}%
\pgfpathcurveto{\pgfqpoint{3.138407in}{1.975523in}}{\pgfqpoint{3.134017in}{1.964924in}}{\pgfqpoint{3.134017in}{1.953874in}}%
\pgfpathcurveto{\pgfqpoint{3.134017in}{1.942824in}}{\pgfqpoint{3.138407in}{1.932225in}}{\pgfqpoint{3.146221in}{1.924411in}}%
\pgfpathcurveto{\pgfqpoint{3.154035in}{1.916598in}}{\pgfqpoint{3.164634in}{1.912207in}}{\pgfqpoint{3.175684in}{1.912207in}}%
\pgfpathclose%
\pgfusepath{stroke,fill}%
\end{pgfscope}%
\begin{pgfscope}%
\pgfpathrectangle{\pgfqpoint{1.016621in}{0.499691in}}{\pgfqpoint{3.875000in}{2.695000in}}%
\pgfusepath{clip}%
\pgfsetbuttcap%
\pgfsetroundjoin%
\definecolor{currentfill}{rgb}{0.121569,0.466667,0.705882}%
\pgfsetfillcolor{currentfill}%
\pgfsetlinewidth{1.003750pt}%
\definecolor{currentstroke}{rgb}{0.121569,0.466667,0.705882}%
\pgfsetstrokecolor{currentstroke}%
\pgfsetdash{}{0pt}%
\pgfpathmoveto{\pgfqpoint{3.191375in}{1.917067in}}%
\pgfpathcurveto{\pgfqpoint{3.202425in}{1.917067in}}{\pgfqpoint{3.213024in}{1.921457in}}{\pgfqpoint{3.220838in}{1.929271in}}%
\pgfpathcurveto{\pgfqpoint{3.228652in}{1.937084in}}{\pgfqpoint{3.233042in}{1.947683in}}{\pgfqpoint{3.233042in}{1.958734in}}%
\pgfpathcurveto{\pgfqpoint{3.233042in}{1.969784in}}{\pgfqpoint{3.228652in}{1.980383in}}{\pgfqpoint{3.220838in}{1.988196in}}%
\pgfpathcurveto{\pgfqpoint{3.213024in}{1.996010in}}{\pgfqpoint{3.202425in}{2.000400in}}{\pgfqpoint{3.191375in}{2.000400in}}%
\pgfpathcurveto{\pgfqpoint{3.180325in}{2.000400in}}{\pgfqpoint{3.169726in}{1.996010in}}{\pgfqpoint{3.161913in}{1.988196in}}%
\pgfpathcurveto{\pgfqpoint{3.154099in}{1.980383in}}{\pgfqpoint{3.149709in}{1.969784in}}{\pgfqpoint{3.149709in}{1.958734in}}%
\pgfpathcurveto{\pgfqpoint{3.149709in}{1.947683in}}{\pgfqpoint{3.154099in}{1.937084in}}{\pgfqpoint{3.161913in}{1.929271in}}%
\pgfpathcurveto{\pgfqpoint{3.169726in}{1.921457in}}{\pgfqpoint{3.180325in}{1.917067in}}{\pgfqpoint{3.191375in}{1.917067in}}%
\pgfpathclose%
\pgfusepath{stroke,fill}%
\end{pgfscope}%
\begin{pgfscope}%
\pgfpathrectangle{\pgfqpoint{1.016621in}{0.499691in}}{\pgfqpoint{3.875000in}{2.695000in}}%
\pgfusepath{clip}%
\pgfsetbuttcap%
\pgfsetroundjoin%
\definecolor{currentfill}{rgb}{0.121569,0.466667,0.705882}%
\pgfsetfillcolor{currentfill}%
\pgfsetlinewidth{1.003750pt}%
\definecolor{currentstroke}{rgb}{0.121569,0.466667,0.705882}%
\pgfsetstrokecolor{currentstroke}%
\pgfsetdash{}{0pt}%
\pgfpathmoveto{\pgfqpoint{3.207067in}{1.941364in}}%
\pgfpathcurveto{\pgfqpoint{3.218117in}{1.941364in}}{\pgfqpoint{3.228716in}{1.945755in}}{\pgfqpoint{3.236530in}{1.953568in}}%
\pgfpathcurveto{\pgfqpoint{3.244343in}{1.961382in}}{\pgfqpoint{3.248733in}{1.971981in}}{\pgfqpoint{3.248733in}{1.983031in}}%
\pgfpathcurveto{\pgfqpoint{3.248733in}{1.994081in}}{\pgfqpoint{3.244343in}{2.004680in}}{\pgfqpoint{3.236530in}{2.012494in}}%
\pgfpathcurveto{\pgfqpoint{3.228716in}{2.020307in}}{\pgfqpoint{3.218117in}{2.024698in}}{\pgfqpoint{3.207067in}{2.024698in}}%
\pgfpathcurveto{\pgfqpoint{3.196017in}{2.024698in}}{\pgfqpoint{3.185418in}{2.020307in}}{\pgfqpoint{3.177604in}{2.012494in}}%
\pgfpathcurveto{\pgfqpoint{3.169790in}{2.004680in}}{\pgfqpoint{3.165400in}{1.994081in}}{\pgfqpoint{3.165400in}{1.983031in}}%
\pgfpathcurveto{\pgfqpoint{3.165400in}{1.971981in}}{\pgfqpoint{3.169790in}{1.961382in}}{\pgfqpoint{3.177604in}{1.953568in}}%
\pgfpathcurveto{\pgfqpoint{3.185418in}{1.945755in}}{\pgfqpoint{3.196017in}{1.941364in}}{\pgfqpoint{3.207067in}{1.941364in}}%
\pgfpathclose%
\pgfusepath{stroke,fill}%
\end{pgfscope}%
\begin{pgfscope}%
\pgfpathrectangle{\pgfqpoint{1.016621in}{0.499691in}}{\pgfqpoint{3.875000in}{2.695000in}}%
\pgfusepath{clip}%
\pgfsetbuttcap%
\pgfsetroundjoin%
\definecolor{currentfill}{rgb}{0.121569,0.466667,0.705882}%
\pgfsetfillcolor{currentfill}%
\pgfsetlinewidth{1.003750pt}%
\definecolor{currentstroke}{rgb}{0.121569,0.466667,0.705882}%
\pgfsetstrokecolor{currentstroke}%
\pgfsetdash{}{0pt}%
\pgfpathmoveto{\pgfqpoint{3.214912in}{1.955943in}}%
\pgfpathcurveto{\pgfqpoint{3.225963in}{1.955943in}}{\pgfqpoint{3.236562in}{1.960333in}}{\pgfqpoint{3.244375in}{1.968147in}}%
\pgfpathcurveto{\pgfqpoint{3.252189in}{1.975960in}}{\pgfqpoint{3.256579in}{1.986559in}}{\pgfqpoint{3.256579in}{1.997609in}}%
\pgfpathcurveto{\pgfqpoint{3.256579in}{2.008660in}}{\pgfqpoint{3.252189in}{2.019259in}}{\pgfqpoint{3.244375in}{2.027072in}}%
\pgfpathcurveto{\pgfqpoint{3.236562in}{2.034886in}}{\pgfqpoint{3.225963in}{2.039276in}}{\pgfqpoint{3.214912in}{2.039276in}}%
\pgfpathcurveto{\pgfqpoint{3.203862in}{2.039276in}}{\pgfqpoint{3.193263in}{2.034886in}}{\pgfqpoint{3.185450in}{2.027072in}}%
\pgfpathcurveto{\pgfqpoint{3.177636in}{2.019259in}}{\pgfqpoint{3.173246in}{2.008660in}}{\pgfqpoint{3.173246in}{1.997609in}}%
\pgfpathcurveto{\pgfqpoint{3.173246in}{1.986559in}}{\pgfqpoint{3.177636in}{1.975960in}}{\pgfqpoint{3.185450in}{1.968147in}}%
\pgfpathcurveto{\pgfqpoint{3.193263in}{1.960333in}}{\pgfqpoint{3.203862in}{1.955943in}}{\pgfqpoint{3.214912in}{1.955943in}}%
\pgfpathclose%
\pgfusepath{stroke,fill}%
\end{pgfscope}%
\begin{pgfscope}%
\pgfpathrectangle{\pgfqpoint{1.016621in}{0.499691in}}{\pgfqpoint{3.875000in}{2.695000in}}%
\pgfusepath{clip}%
\pgfsetbuttcap%
\pgfsetroundjoin%
\definecolor{currentfill}{rgb}{0.121569,0.466667,0.705882}%
\pgfsetfillcolor{currentfill}%
\pgfsetlinewidth{1.003750pt}%
\definecolor{currentstroke}{rgb}{0.121569,0.466667,0.705882}%
\pgfsetstrokecolor{currentstroke}%
\pgfsetdash{}{0pt}%
\pgfpathmoveto{\pgfqpoint{3.230604in}{1.960802in}}%
\pgfpathcurveto{\pgfqpoint{3.241654in}{1.960802in}}{\pgfqpoint{3.252253in}{1.965192in}}{\pgfqpoint{3.260067in}{1.973006in}}%
\pgfpathcurveto{\pgfqpoint{3.267880in}{1.980820in}}{\pgfqpoint{3.272271in}{1.991419in}}{\pgfqpoint{3.272271in}{2.002469in}}%
\pgfpathcurveto{\pgfqpoint{3.272271in}{2.013519in}}{\pgfqpoint{3.267880in}{2.024118in}}{\pgfqpoint{3.260067in}{2.031932in}}%
\pgfpathcurveto{\pgfqpoint{3.252253in}{2.039745in}}{\pgfqpoint{3.241654in}{2.044136in}}{\pgfqpoint{3.230604in}{2.044136in}}%
\pgfpathcurveto{\pgfqpoint{3.219554in}{2.044136in}}{\pgfqpoint{3.208955in}{2.039745in}}{\pgfqpoint{3.201141in}{2.031932in}}%
\pgfpathcurveto{\pgfqpoint{3.193328in}{2.024118in}}{\pgfqpoint{3.188937in}{2.013519in}}{\pgfqpoint{3.188937in}{2.002469in}}%
\pgfpathcurveto{\pgfqpoint{3.188937in}{1.991419in}}{\pgfqpoint{3.193328in}{1.980820in}}{\pgfqpoint{3.201141in}{1.973006in}}%
\pgfpathcurveto{\pgfqpoint{3.208955in}{1.965192in}}{\pgfqpoint{3.219554in}{1.960802in}}{\pgfqpoint{3.230604in}{1.960802in}}%
\pgfpathclose%
\pgfusepath{stroke,fill}%
\end{pgfscope}%
\begin{pgfscope}%
\pgfpathrectangle{\pgfqpoint{1.016621in}{0.499691in}}{\pgfqpoint{3.875000in}{2.695000in}}%
\pgfusepath{clip}%
\pgfsetbuttcap%
\pgfsetroundjoin%
\definecolor{currentfill}{rgb}{0.121569,0.466667,0.705882}%
\pgfsetfillcolor{currentfill}%
\pgfsetlinewidth{1.003750pt}%
\definecolor{currentstroke}{rgb}{0.121569,0.466667,0.705882}%
\pgfsetstrokecolor{currentstroke}%
\pgfsetdash{}{0pt}%
\pgfpathmoveto{\pgfqpoint{3.246295in}{2.011827in}}%
\pgfpathcurveto{\pgfqpoint{3.257345in}{2.011827in}}{\pgfqpoint{3.267945in}{2.016217in}}{\pgfqpoint{3.275758in}{2.024031in}}%
\pgfpathcurveto{\pgfqpoint{3.283572in}{2.031844in}}{\pgfqpoint{3.287962in}{2.042443in}}{\pgfqpoint{3.287962in}{2.053493in}}%
\pgfpathcurveto{\pgfqpoint{3.287962in}{2.064544in}}{\pgfqpoint{3.283572in}{2.075143in}}{\pgfqpoint{3.275758in}{2.082956in}}%
\pgfpathcurveto{\pgfqpoint{3.267945in}{2.090770in}}{\pgfqpoint{3.257345in}{2.095160in}}{\pgfqpoint{3.246295in}{2.095160in}}%
\pgfpathcurveto{\pgfqpoint{3.235245in}{2.095160in}}{\pgfqpoint{3.224646in}{2.090770in}}{\pgfqpoint{3.216833in}{2.082956in}}%
\pgfpathcurveto{\pgfqpoint{3.209019in}{2.075143in}}{\pgfqpoint{3.204629in}{2.064544in}}{\pgfqpoint{3.204629in}{2.053493in}}%
\pgfpathcurveto{\pgfqpoint{3.204629in}{2.042443in}}{\pgfqpoint{3.209019in}{2.031844in}}{\pgfqpoint{3.216833in}{2.024031in}}%
\pgfpathcurveto{\pgfqpoint{3.224646in}{2.016217in}}{\pgfqpoint{3.235245in}{2.011827in}}{\pgfqpoint{3.246295in}{2.011827in}}%
\pgfpathclose%
\pgfusepath{stroke,fill}%
\end{pgfscope}%
\begin{pgfscope}%
\pgfpathrectangle{\pgfqpoint{1.016621in}{0.499691in}}{\pgfqpoint{3.875000in}{2.695000in}}%
\pgfusepath{clip}%
\pgfsetbuttcap%
\pgfsetroundjoin%
\definecolor{currentfill}{rgb}{0.121569,0.466667,0.705882}%
\pgfsetfillcolor{currentfill}%
\pgfsetlinewidth{1.003750pt}%
\definecolor{currentstroke}{rgb}{0.121569,0.466667,0.705882}%
\pgfsetstrokecolor{currentstroke}%
\pgfsetdash{}{0pt}%
\pgfpathmoveto{\pgfqpoint{3.254141in}{2.023976in}}%
\pgfpathcurveto{\pgfqpoint{3.265191in}{2.023976in}}{\pgfqpoint{3.275790in}{2.028366in}}{\pgfqpoint{3.283604in}{2.036179in}}%
\pgfpathcurveto{\pgfqpoint{3.291417in}{2.043993in}}{\pgfqpoint{3.295808in}{2.054592in}}{\pgfqpoint{3.295808in}{2.065642in}}%
\pgfpathcurveto{\pgfqpoint{3.295808in}{2.076692in}}{\pgfqpoint{3.291417in}{2.087291in}}{\pgfqpoint{3.283604in}{2.095105in}}%
\pgfpathcurveto{\pgfqpoint{3.275790in}{2.102919in}}{\pgfqpoint{3.265191in}{2.107309in}}{\pgfqpoint{3.254141in}{2.107309in}}%
\pgfpathcurveto{\pgfqpoint{3.243091in}{2.107309in}}{\pgfqpoint{3.232492in}{2.102919in}}{\pgfqpoint{3.224678in}{2.095105in}}%
\pgfpathcurveto{\pgfqpoint{3.216865in}{2.087291in}}{\pgfqpoint{3.212474in}{2.076692in}}{\pgfqpoint{3.212474in}{2.065642in}}%
\pgfpathcurveto{\pgfqpoint{3.212474in}{2.054592in}}{\pgfqpoint{3.216865in}{2.043993in}}{\pgfqpoint{3.224678in}{2.036179in}}%
\pgfpathcurveto{\pgfqpoint{3.232492in}{2.028366in}}{\pgfqpoint{3.243091in}{2.023976in}}{\pgfqpoint{3.254141in}{2.023976in}}%
\pgfpathclose%
\pgfusepath{stroke,fill}%
\end{pgfscope}%
\begin{pgfscope}%
\pgfpathrectangle{\pgfqpoint{1.016621in}{0.499691in}}{\pgfqpoint{3.875000in}{2.695000in}}%
\pgfusepath{clip}%
\pgfsetbuttcap%
\pgfsetroundjoin%
\definecolor{currentfill}{rgb}{0.121569,0.466667,0.705882}%
\pgfsetfillcolor{currentfill}%
\pgfsetlinewidth{1.003750pt}%
\definecolor{currentstroke}{rgb}{0.121569,0.466667,0.705882}%
\pgfsetstrokecolor{currentstroke}%
\pgfsetdash{}{0pt}%
\pgfpathmoveto{\pgfqpoint{3.261987in}{2.055562in}}%
\pgfpathcurveto{\pgfqpoint{3.273037in}{2.055562in}}{\pgfqpoint{3.283636in}{2.059952in}}{\pgfqpoint{3.291450in}{2.067766in}}%
\pgfpathcurveto{\pgfqpoint{3.299263in}{2.075580in}}{\pgfqpoint{3.303653in}{2.086179in}}{\pgfqpoint{3.303653in}{2.097229in}}%
\pgfpathcurveto{\pgfqpoint{3.303653in}{2.108279in}}{\pgfqpoint{3.299263in}{2.118878in}}{\pgfqpoint{3.291450in}{2.126692in}}%
\pgfpathcurveto{\pgfqpoint{3.283636in}{2.134505in}}{\pgfqpoint{3.273037in}{2.138895in}}{\pgfqpoint{3.261987in}{2.138895in}}%
\pgfpathcurveto{\pgfqpoint{3.250937in}{2.138895in}}{\pgfqpoint{3.240338in}{2.134505in}}{\pgfqpoint{3.232524in}{2.126692in}}%
\pgfpathcurveto{\pgfqpoint{3.224710in}{2.118878in}}{\pgfqpoint{3.220320in}{2.108279in}}{\pgfqpoint{3.220320in}{2.097229in}}%
\pgfpathcurveto{\pgfqpoint{3.220320in}{2.086179in}}{\pgfqpoint{3.224710in}{2.075580in}}{\pgfqpoint{3.232524in}{2.067766in}}%
\pgfpathcurveto{\pgfqpoint{3.240338in}{2.059952in}}{\pgfqpoint{3.250937in}{2.055562in}}{\pgfqpoint{3.261987in}{2.055562in}}%
\pgfpathclose%
\pgfusepath{stroke,fill}%
\end{pgfscope}%
\begin{pgfscope}%
\pgfpathrectangle{\pgfqpoint{1.016621in}{0.499691in}}{\pgfqpoint{3.875000in}{2.695000in}}%
\pgfusepath{clip}%
\pgfsetbuttcap%
\pgfsetroundjoin%
\definecolor{currentfill}{rgb}{0.121569,0.466667,0.705882}%
\pgfsetfillcolor{currentfill}%
\pgfsetlinewidth{1.003750pt}%
\definecolor{currentstroke}{rgb}{0.121569,0.466667,0.705882}%
\pgfsetstrokecolor{currentstroke}%
\pgfsetdash{}{0pt}%
\pgfpathmoveto{\pgfqpoint{3.277678in}{2.075000in}}%
\pgfpathcurveto{\pgfqpoint{3.288728in}{2.075000in}}{\pgfqpoint{3.299327in}{2.079390in}}{\pgfqpoint{3.307141in}{2.087204in}}%
\pgfpathcurveto{\pgfqpoint{3.314955in}{2.095018in}}{\pgfqpoint{3.319345in}{2.105617in}}{\pgfqpoint{3.319345in}{2.116667in}}%
\pgfpathcurveto{\pgfqpoint{3.319345in}{2.127717in}}{\pgfqpoint{3.314955in}{2.138316in}}{\pgfqpoint{3.307141in}{2.146130in}}%
\pgfpathcurveto{\pgfqpoint{3.299327in}{2.153943in}}{\pgfqpoint{3.288728in}{2.158333in}}{\pgfqpoint{3.277678in}{2.158333in}}%
\pgfpathcurveto{\pgfqpoint{3.266628in}{2.158333in}}{\pgfqpoint{3.256029in}{2.153943in}}{\pgfqpoint{3.248215in}{2.146130in}}%
\pgfpathcurveto{\pgfqpoint{3.240402in}{2.138316in}}{\pgfqpoint{3.236012in}{2.127717in}}{\pgfqpoint{3.236012in}{2.116667in}}%
\pgfpathcurveto{\pgfqpoint{3.236012in}{2.105617in}}{\pgfqpoint{3.240402in}{2.095018in}}{\pgfqpoint{3.248215in}{2.087204in}}%
\pgfpathcurveto{\pgfqpoint{3.256029in}{2.079390in}}{\pgfqpoint{3.266628in}{2.075000in}}{\pgfqpoint{3.277678in}{2.075000in}}%
\pgfpathclose%
\pgfusepath{stroke,fill}%
\end{pgfscope}%
\begin{pgfscope}%
\pgfpathrectangle{\pgfqpoint{1.016621in}{0.499691in}}{\pgfqpoint{3.875000in}{2.695000in}}%
\pgfusepath{clip}%
\pgfsetrectcap%
\pgfsetroundjoin%
\pgfsetlinewidth{1.505625pt}%
\definecolor{currentstroke}{rgb}{1.000000,0.843137,0.000000}%
\pgfsetstrokecolor{currentstroke}%
\pgfsetdash{}{0pt}%
\pgfpathmoveto{\pgfqpoint{1.192757in}{0.622191in}}%
\pgfpathlineto{\pgfqpoint{1.341826in}{0.637266in}}%
\pgfpathlineto{\pgfqpoint{1.475203in}{0.652903in}}%
\pgfpathlineto{\pgfqpoint{1.592889in}{0.668784in}}%
\pgfpathlineto{\pgfqpoint{1.702729in}{0.685769in}}%
\pgfpathlineto{\pgfqpoint{1.804723in}{0.703818in}}%
\pgfpathlineto{\pgfqpoint{1.898872in}{0.722846in}}%
\pgfpathlineto{\pgfqpoint{1.985175in}{0.742701in}}%
\pgfpathlineto{\pgfqpoint{2.063632in}{0.763159in}}%
\pgfpathlineto{\pgfqpoint{2.134243in}{0.783906in}}%
\pgfpathlineto{\pgfqpoint{2.197009in}{0.804537in}}%
\pgfpathlineto{\pgfqpoint{2.259775in}{0.827586in}}%
\pgfpathlineto{\pgfqpoint{2.314695in}{0.850060in}}%
\pgfpathlineto{\pgfqpoint{2.369615in}{0.875035in}}%
\pgfpathlineto{\pgfqpoint{2.416689in}{0.898734in}}%
\pgfpathlineto{\pgfqpoint{2.463763in}{0.924856in}}%
\pgfpathlineto{\pgfqpoint{2.510838in}{0.953744in}}%
\pgfpathlineto{\pgfqpoint{2.550066in}{0.980214in}}%
\pgfpathlineto{\pgfqpoint{2.589295in}{1.009140in}}%
\pgfpathlineto{\pgfqpoint{2.628524in}{1.040812in}}%
\pgfpathlineto{\pgfqpoint{2.667752in}{1.075555in}}%
\pgfpathlineto{\pgfqpoint{2.706981in}{1.113721in}}%
\pgfpathlineto{\pgfqpoint{2.746209in}{1.155688in}}%
\pgfpathlineto{\pgfqpoint{2.777592in}{1.192262in}}%
\pgfpathlineto{\pgfqpoint{2.808975in}{1.231720in}}%
\pgfpathlineto{\pgfqpoint{2.840358in}{1.274252in}}%
\pgfpathlineto{\pgfqpoint{2.871741in}{1.320027in}}%
\pgfpathlineto{\pgfqpoint{2.903124in}{1.369172in}}%
\pgfpathlineto{\pgfqpoint{2.934507in}{1.421759in}}%
\pgfpathlineto{\pgfqpoint{2.973735in}{1.492310in}}%
\pgfpathlineto{\pgfqpoint{3.012964in}{1.567963in}}%
\pgfpathlineto{\pgfqpoint{3.052192in}{1.648155in}}%
\pgfpathlineto{\pgfqpoint{3.107112in}{1.766258in}}%
\pgfpathlineto{\pgfqpoint{3.264027in}{2.109183in}}%
\pgfpathlineto{\pgfqpoint{3.303255in}{2.188529in}}%
\pgfpathlineto{\pgfqpoint{3.342484in}{2.263196in}}%
\pgfpathlineto{\pgfqpoint{3.381712in}{2.332687in}}%
\pgfpathlineto{\pgfqpoint{3.420941in}{2.396803in}}%
\pgfpathlineto{\pgfqpoint{3.452324in}{2.444250in}}%
\pgfpathlineto{\pgfqpoint{3.483707in}{2.488385in}}%
\pgfpathlineto{\pgfqpoint{3.515090in}{2.529357in}}%
\pgfpathlineto{\pgfqpoint{3.546473in}{2.567349in}}%
\pgfpathlineto{\pgfqpoint{3.585701in}{2.610946in}}%
\pgfpathlineto{\pgfqpoint{3.624930in}{2.650586in}}%
\pgfpathlineto{\pgfqpoint{3.664158in}{2.686654in}}%
\pgfpathlineto{\pgfqpoint{3.703387in}{2.719514in}}%
\pgfpathlineto{\pgfqpoint{3.742615in}{2.749501in}}%
\pgfpathlineto{\pgfqpoint{3.781844in}{2.776920in}}%
\pgfpathlineto{\pgfqpoint{3.828918in}{2.806815in}}%
\pgfpathlineto{\pgfqpoint{3.875993in}{2.833819in}}%
\pgfpathlineto{\pgfqpoint{3.923067in}{2.858291in}}%
\pgfpathlineto{\pgfqpoint{3.977987in}{2.884051in}}%
\pgfpathlineto{\pgfqpoint{4.032907in}{2.907203in}}%
\pgfpathlineto{\pgfqpoint{4.087827in}{2.928099in}}%
\pgfpathlineto{\pgfqpoint{4.150593in}{2.949593in}}%
\pgfpathlineto{\pgfqpoint{4.221204in}{2.971166in}}%
\pgfpathlineto{\pgfqpoint{4.291816in}{2.990390in}}%
\pgfpathlineto{\pgfqpoint{4.370273in}{3.009418in}}%
\pgfpathlineto{\pgfqpoint{4.456576in}{3.027956in}}%
\pgfpathlineto{\pgfqpoint{4.550724in}{3.045789in}}%
\pgfpathlineto{\pgfqpoint{4.652719in}{3.062769in}}%
\pgfpathlineto{\pgfqpoint{4.715484in}{3.072191in}}%
\pgfpathlineto{\pgfqpoint{4.715484in}{3.072191in}}%
\pgfusepath{stroke}%
\end{pgfscope}%
\begin{pgfscope}%
\pgfsetrectcap%
\pgfsetmiterjoin%
\pgfsetlinewidth{0.803000pt}%
\definecolor{currentstroke}{rgb}{0.000000,0.000000,0.000000}%
\pgfsetstrokecolor{currentstroke}%
\pgfsetdash{}{0pt}%
\pgfpathmoveto{\pgfqpoint{1.016621in}{0.499691in}}%
\pgfpathlineto{\pgfqpoint{1.016621in}{3.194691in}}%
\pgfusepath{stroke}%
\end{pgfscope}%
\begin{pgfscope}%
\pgfsetrectcap%
\pgfsetmiterjoin%
\pgfsetlinewidth{0.803000pt}%
\definecolor{currentstroke}{rgb}{0.000000,0.000000,0.000000}%
\pgfsetstrokecolor{currentstroke}%
\pgfsetdash{}{0pt}%
\pgfpathmoveto{\pgfqpoint{4.891621in}{0.499691in}}%
\pgfpathlineto{\pgfqpoint{4.891621in}{3.194691in}}%
\pgfusepath{stroke}%
\end{pgfscope}%
\begin{pgfscope}%
\pgfsetrectcap%
\pgfsetmiterjoin%
\pgfsetlinewidth{0.803000pt}%
\definecolor{currentstroke}{rgb}{0.000000,0.000000,0.000000}%
\pgfsetstrokecolor{currentstroke}%
\pgfsetdash{}{0pt}%
\pgfpathmoveto{\pgfqpoint{1.016621in}{0.499691in}}%
\pgfpathlineto{\pgfqpoint{4.891621in}{0.499691in}}%
\pgfusepath{stroke}%
\end{pgfscope}%
\begin{pgfscope}%
\pgfsetrectcap%
\pgfsetmiterjoin%
\pgfsetlinewidth{0.803000pt}%
\definecolor{currentstroke}{rgb}{0.000000,0.000000,0.000000}%
\pgfsetstrokecolor{currentstroke}%
\pgfsetdash{}{0pt}%
\pgfpathmoveto{\pgfqpoint{1.016621in}{3.194691in}}%
\pgfpathlineto{\pgfqpoint{4.891621in}{3.194691in}}%
\pgfusepath{stroke}%
\end{pgfscope}%
\begin{pgfscope}%
\pgfpathrectangle{\pgfqpoint{1.016621in}{0.499691in}}{\pgfqpoint{3.875000in}{2.695000in}}%
\pgfusepath{clip}%
\pgfsetbuttcap%
\pgfsetroundjoin%
\definecolor{currentfill}{rgb}{1.000000,0.388235,0.278431}%
\pgfsetfillcolor{currentfill}%
\pgfsetlinewidth{1.003750pt}%
\definecolor{currentstroke}{rgb}{1.000000,0.388235,0.278431}%
\pgfsetstrokecolor{currentstroke}%
\pgfsetdash{}{0pt}%
\pgfpathmoveto{\pgfqpoint{3.193415in}{1.915794in}}%
\pgfpathcurveto{\pgfqpoint{3.204465in}{1.915794in}}{\pgfqpoint{3.215064in}{1.920184in}}{\pgfqpoint{3.222878in}{1.927998in}}%
\pgfpathcurveto{\pgfqpoint{3.230692in}{1.935811in}}{\pgfqpoint{3.235082in}{1.946410in}}{\pgfqpoint{3.235082in}{1.957460in}}%
\pgfpathcurveto{\pgfqpoint{3.235082in}{1.968510in}}{\pgfqpoint{3.230692in}{1.979109in}}{\pgfqpoint{3.222878in}{1.986923in}}%
\pgfpathcurveto{\pgfqpoint{3.215064in}{1.994737in}}{\pgfqpoint{3.204465in}{1.999127in}}{\pgfqpoint{3.193415in}{1.999127in}}%
\pgfpathcurveto{\pgfqpoint{3.182365in}{1.999127in}}{\pgfqpoint{3.171766in}{1.994737in}}{\pgfqpoint{3.163952in}{1.986923in}}%
\pgfpathcurveto{\pgfqpoint{3.156139in}{1.979109in}}{\pgfqpoint{3.151749in}{1.968510in}}{\pgfqpoint{3.151749in}{1.957460in}}%
\pgfpathcurveto{\pgfqpoint{3.151749in}{1.946410in}}{\pgfqpoint{3.156139in}{1.935811in}}{\pgfqpoint{3.163952in}{1.927998in}}%
\pgfpathcurveto{\pgfqpoint{3.171766in}{1.920184in}}{\pgfqpoint{3.182365in}{1.915794in}}{\pgfqpoint{3.193415in}{1.915794in}}%
\pgfpathclose%
\pgfusepath{stroke,fill}%
\end{pgfscope}%
\end{pgfpicture}%
\makeatother%
\endgroup%

    \caption{Representación de $\varphi$ frente a f (escala logarítmica) con un ajuste a una curva arcotangente}
  \end{figure}

  Un problema que tuvimos con nuestros datos que evitamos comentar hasta ahora fue que para los valores cercanos a la frecuencia de corte obtivimos ángulos no centrados en 45º. Para 1300 Hz obtuvimos un ángulo de 48,6º, mientras que para 1360 Hz obtuvimos 47º. Comentamos este detalle con el profesor de la práctica, y repetimos estas mediciones para asegurar que no las habíamos tomado mal, pero el resultado apenas varió. Discutiremos esto más en profundidad en las conclusiones, sin embargo, esto afecta directamente a este apartado ya que utilizando este método aproximamos la frecuencia de corte a 1487.99 Hz, dónde la curva vale 45º. Esto difiere mucho de nuestra estimación, pero es un problema de los ángulos obtenidos para los datos que tomamos, no necesariamente del ajuste que hicimos de ellos.


  \newpage
  \section{Conclusiones}

  \subsection{Impedancia (Z) frente a frecuencia (f)}

  Tras mediciones de voltajes ($V_m$, $V_{mR}$ y $V_{mC}$) variando la resistencia, representamos el módulo de la impedancia (Z) frente a la frecuencia (f) en una escala logarítmica. Observamos que forman una hipérbola cuyas asíntotas son las rectas R (horizontal) y C (oblicua). Estas se cortan en la frecuencia de corte, que coincidía con la calculada teóricamente ($1326, 29 Hz$).

  \subsection{$\frac{V_{mR}}{V_{mC}}$ frente a frecuencia (f)}

  Representamos el cociente entre los voltajes en bornes de la resistencia y el condensador ($V_{mR}/V_{mC}$) frente a la frecuencia. Cúando los primeros resultaban 1 obteníamos la frecuencia de corte, en este caso de $1308.90 Hz$, razonablemente próxima a la teórica y a la averiguada por el método anterior.

  \subsection{Desfase ($\varphi$) frente a frecuencia (f)}

  Utilizamos el método de las dos trazas para estudiar la variación del argumento de la impedancia ($\varphi$) al variar la frecuencia, entre las señales del potencial total $V_m$ y el potencial en bornes de la resistencia $V_{mR}$. Añadimos mediciones de frecuencias muy altas y muy bajas, aunque su incertidumbre sería bastante elevada.

  Representamos $\varphi$ frente a la escala logarítmica de la frecuencia, y tras varios intentos de ajuste concluímos que se trataba de una función arcotangente. Tuvimos un problema con los valores del ángulo que medimos, ya que en la frecuencia de corte este valía en torno a 47-49º, cúando tenía que ser 45º. Repetimos las mediciones pero obtuvimos resultados muy similares. El resto de datos coinciden con la función que aproximamos y las relaciones entre los valores relativos parecen tener sentido. Concluimos por tanto que lo más probable es que se debiera a un \textbf{error sistemático}, o bien por parte de las herramientas de medida o por nuestra cuenta, ya que todos los valores parecen influenciados en la misma dirección. Solucionar este problema requeriría una inspección más próxima y una repetición completa de la práctica, con el mismo osciloscopio y con otro distinto, para discernir el origen del error.

  Sin embargo, podemos comprobar que el desfase es de 45º para la frecuencia de corte si utilizamos los datos de la primera sección, que no están influenciados por este error sistemático. Utilizando las ecuaciones \ref{ec:faseimp} y \ref{ec:fcorte} podemos deducir que:

  \begin{equation}
    \varphi = arctg\left(-\frac{1}{wRC}\right) \qquad \frac{V_{mC}}{V_{mR}} = 1 = \frac{1}{wRC} \qquad \varphi = arctg\left(-\frac{V_{mC}}{{V_{mR}}}\right) \nonumber
  \end{equation}

  Por lo tanto, podemos utilizar los datos de $\frac{V_{mC}}{V_{mR}}$ de la tabla \ref{tb:ac1} y representar los desfases ($\varphi$) de esos datos:

  \begin{figure}[H]
    %\centering
    \hspace{2.5em} %% Creator: Matplotlib, PGF backend
%%
%% To include the figure in your LaTeX document, write
%%   \input{<filename>.pgf}
%%
%% Make sure the required packages are loaded in your preamble
%%   \usepackage{pgf}
%%
%% Figures using additional raster images can only be included by \input if
%% they are in the same directory as the main LaTeX file. For loading figures
%% from other directories you can use the `import` package
%%   \usepackage{import}
%% and then include the figures with
%%   \import{<path to file>}{<filename>.pgf}
%%
%% Matplotlib used the following preamble
%%
\begingroup%
\makeatletter%
\begin{pgfpicture}%
\pgfpathrectangle{\pgfpointorigin}{\pgfqpoint{4.922176in}{3.294691in}}%
\pgfusepath{use as bounding box, clip}%
\begin{pgfscope}%
\pgfsetbuttcap%
\pgfsetmiterjoin%
\definecolor{currentfill}{rgb}{1.000000,1.000000,1.000000}%
\pgfsetfillcolor{currentfill}%
\pgfsetlinewidth{0.000000pt}%
\definecolor{currentstroke}{rgb}{1.000000,1.000000,1.000000}%
\pgfsetstrokecolor{currentstroke}%
\pgfsetdash{}{0pt}%
\pgfpathmoveto{\pgfqpoint{0.000000in}{0.000000in}}%
\pgfpathlineto{\pgfqpoint{4.922176in}{0.000000in}}%
\pgfpathlineto{\pgfqpoint{4.922176in}{3.294691in}}%
\pgfpathlineto{\pgfqpoint{0.000000in}{3.294691in}}%
\pgfpathclose%
\pgfusepath{fill}%
\end{pgfscope}%
\begin{pgfscope}%
\pgfsetbuttcap%
\pgfsetmiterjoin%
\definecolor{currentfill}{rgb}{1.000000,1.000000,1.000000}%
\pgfsetfillcolor{currentfill}%
\pgfsetlinewidth{0.000000pt}%
\definecolor{currentstroke}{rgb}{0.000000,0.000000,0.000000}%
\pgfsetstrokecolor{currentstroke}%
\pgfsetstrokeopacity{0.000000}%
\pgfsetdash{}{0pt}%
\pgfpathmoveto{\pgfqpoint{0.947176in}{0.499691in}}%
\pgfpathlineto{\pgfqpoint{4.822176in}{0.499691in}}%
\pgfpathlineto{\pgfqpoint{4.822176in}{3.194691in}}%
\pgfpathlineto{\pgfqpoint{0.947176in}{3.194691in}}%
\pgfpathclose%
\pgfusepath{fill}%
\end{pgfscope}%
\begin{pgfscope}%
\pgfsetbuttcap%
\pgfsetroundjoin%
\definecolor{currentfill}{rgb}{0.000000,0.000000,0.000000}%
\pgfsetfillcolor{currentfill}%
\pgfsetlinewidth{0.803000pt}%
\definecolor{currentstroke}{rgb}{0.000000,0.000000,0.000000}%
\pgfsetstrokecolor{currentstroke}%
\pgfsetdash{}{0pt}%
\pgfsys@defobject{currentmarker}{\pgfqpoint{0.000000in}{-0.048611in}}{\pgfqpoint{0.000000in}{0.000000in}}{%
\pgfpathmoveto{\pgfqpoint{0.000000in}{0.000000in}}%
\pgfpathlineto{\pgfqpoint{0.000000in}{-0.048611in}}%
\pgfusepath{stroke,fill}%
}%
\begin{pgfscope}%
\pgfsys@transformshift{0.950489in}{0.499691in}%
\pgfsys@useobject{currentmarker}{}%
\end{pgfscope}%
\end{pgfscope}%
\begin{pgfscope}%
\definecolor{textcolor}{rgb}{0.000000,0.000000,0.000000}%
\pgfsetstrokecolor{textcolor}%
\pgfsetfillcolor{textcolor}%
\pgftext[x=0.950489in,y=0.402469in,,top]{\color{textcolor}\rmfamily\fontsize{10.000000}{12.000000}\selectfont \(\displaystyle 1.5\)}%
\end{pgfscope}%
\begin{pgfscope}%
\pgfsetbuttcap%
\pgfsetroundjoin%
\definecolor{currentfill}{rgb}{0.000000,0.000000,0.000000}%
\pgfsetfillcolor{currentfill}%
\pgfsetlinewidth{0.803000pt}%
\definecolor{currentstroke}{rgb}{0.000000,0.000000,0.000000}%
\pgfsetstrokecolor{currentstroke}%
\pgfsetdash{}{0pt}%
\pgfsys@defobject{currentmarker}{\pgfqpoint{0.000000in}{-0.048611in}}{\pgfqpoint{0.000000in}{0.000000in}}{%
\pgfpathmoveto{\pgfqpoint{0.000000in}{0.000000in}}%
\pgfpathlineto{\pgfqpoint{0.000000in}{-0.048611in}}%
\pgfusepath{stroke,fill}%
}%
\begin{pgfscope}%
\pgfsys@transformshift{1.655316in}{0.499691in}%
\pgfsys@useobject{currentmarker}{}%
\end{pgfscope}%
\end{pgfscope}%
\begin{pgfscope}%
\definecolor{textcolor}{rgb}{0.000000,0.000000,0.000000}%
\pgfsetstrokecolor{textcolor}%
\pgfsetfillcolor{textcolor}%
\pgftext[x=1.655316in,y=0.402469in,,top]{\color{textcolor}\rmfamily\fontsize{10.000000}{12.000000}\selectfont \(\displaystyle 2.0\)}%
\end{pgfscope}%
\begin{pgfscope}%
\pgfsetbuttcap%
\pgfsetroundjoin%
\definecolor{currentfill}{rgb}{0.000000,0.000000,0.000000}%
\pgfsetfillcolor{currentfill}%
\pgfsetlinewidth{0.803000pt}%
\definecolor{currentstroke}{rgb}{0.000000,0.000000,0.000000}%
\pgfsetstrokecolor{currentstroke}%
\pgfsetdash{}{0pt}%
\pgfsys@defobject{currentmarker}{\pgfqpoint{0.000000in}{-0.048611in}}{\pgfqpoint{0.000000in}{0.000000in}}{%
\pgfpathmoveto{\pgfqpoint{0.000000in}{0.000000in}}%
\pgfpathlineto{\pgfqpoint{0.000000in}{-0.048611in}}%
\pgfusepath{stroke,fill}%
}%
\begin{pgfscope}%
\pgfsys@transformshift{2.360144in}{0.499691in}%
\pgfsys@useobject{currentmarker}{}%
\end{pgfscope}%
\end{pgfscope}%
\begin{pgfscope}%
\definecolor{textcolor}{rgb}{0.000000,0.000000,0.000000}%
\pgfsetstrokecolor{textcolor}%
\pgfsetfillcolor{textcolor}%
\pgftext[x=2.360144in,y=0.402469in,,top]{\color{textcolor}\rmfamily\fontsize{10.000000}{12.000000}\selectfont \(\displaystyle 2.5\)}%
\end{pgfscope}%
\begin{pgfscope}%
\pgfsetbuttcap%
\pgfsetroundjoin%
\definecolor{currentfill}{rgb}{0.000000,0.000000,0.000000}%
\pgfsetfillcolor{currentfill}%
\pgfsetlinewidth{0.803000pt}%
\definecolor{currentstroke}{rgb}{0.000000,0.000000,0.000000}%
\pgfsetstrokecolor{currentstroke}%
\pgfsetdash{}{0pt}%
\pgfsys@defobject{currentmarker}{\pgfqpoint{0.000000in}{-0.048611in}}{\pgfqpoint{0.000000in}{0.000000in}}{%
\pgfpathmoveto{\pgfqpoint{0.000000in}{0.000000in}}%
\pgfpathlineto{\pgfqpoint{0.000000in}{-0.048611in}}%
\pgfusepath{stroke,fill}%
}%
\begin{pgfscope}%
\pgfsys@transformshift{3.064971in}{0.499691in}%
\pgfsys@useobject{currentmarker}{}%
\end{pgfscope}%
\end{pgfscope}%
\begin{pgfscope}%
\definecolor{textcolor}{rgb}{0.000000,0.000000,0.000000}%
\pgfsetstrokecolor{textcolor}%
\pgfsetfillcolor{textcolor}%
\pgftext[x=3.064971in,y=0.402469in,,top]{\color{textcolor}\rmfamily\fontsize{10.000000}{12.000000}\selectfont \(\displaystyle 3.0\)}%
\end{pgfscope}%
\begin{pgfscope}%
\pgfsetbuttcap%
\pgfsetroundjoin%
\definecolor{currentfill}{rgb}{0.000000,0.000000,0.000000}%
\pgfsetfillcolor{currentfill}%
\pgfsetlinewidth{0.803000pt}%
\definecolor{currentstroke}{rgb}{0.000000,0.000000,0.000000}%
\pgfsetstrokecolor{currentstroke}%
\pgfsetdash{}{0pt}%
\pgfsys@defobject{currentmarker}{\pgfqpoint{0.000000in}{-0.048611in}}{\pgfqpoint{0.000000in}{0.000000in}}{%
\pgfpathmoveto{\pgfqpoint{0.000000in}{0.000000in}}%
\pgfpathlineto{\pgfqpoint{0.000000in}{-0.048611in}}%
\pgfusepath{stroke,fill}%
}%
\begin{pgfscope}%
\pgfsys@transformshift{3.769798in}{0.499691in}%
\pgfsys@useobject{currentmarker}{}%
\end{pgfscope}%
\end{pgfscope}%
\begin{pgfscope}%
\definecolor{textcolor}{rgb}{0.000000,0.000000,0.000000}%
\pgfsetstrokecolor{textcolor}%
\pgfsetfillcolor{textcolor}%
\pgftext[x=3.769798in,y=0.402469in,,top]{\color{textcolor}\rmfamily\fontsize{10.000000}{12.000000}\selectfont \(\displaystyle 3.5\)}%
\end{pgfscope}%
\begin{pgfscope}%
\pgfsetbuttcap%
\pgfsetroundjoin%
\definecolor{currentfill}{rgb}{0.000000,0.000000,0.000000}%
\pgfsetfillcolor{currentfill}%
\pgfsetlinewidth{0.803000pt}%
\definecolor{currentstroke}{rgb}{0.000000,0.000000,0.000000}%
\pgfsetstrokecolor{currentstroke}%
\pgfsetdash{}{0pt}%
\pgfsys@defobject{currentmarker}{\pgfqpoint{0.000000in}{-0.048611in}}{\pgfqpoint{0.000000in}{0.000000in}}{%
\pgfpathmoveto{\pgfqpoint{0.000000in}{0.000000in}}%
\pgfpathlineto{\pgfqpoint{0.000000in}{-0.048611in}}%
\pgfusepath{stroke,fill}%
}%
\begin{pgfscope}%
\pgfsys@transformshift{4.474626in}{0.499691in}%
\pgfsys@useobject{currentmarker}{}%
\end{pgfscope}%
\end{pgfscope}%
\begin{pgfscope}%
\definecolor{textcolor}{rgb}{0.000000,0.000000,0.000000}%
\pgfsetstrokecolor{textcolor}%
\pgfsetfillcolor{textcolor}%
\pgftext[x=4.474626in,y=0.402469in,,top]{\color{textcolor}\rmfamily\fontsize{10.000000}{12.000000}\selectfont \(\displaystyle 4.0\)}%
\end{pgfscope}%
\begin{pgfscope}%
\definecolor{textcolor}{rgb}{0.000000,0.000000,0.000000}%
\pgfsetstrokecolor{textcolor}%
\pgfsetfillcolor{textcolor}%
\pgftext[x=2.884676in,y=0.223457in,,top]{\color{textcolor}\rmfamily\fontsize{10.000000}{12.000000}\selectfont log f}%
\end{pgfscope}%
\begin{pgfscope}%
\pgfsetbuttcap%
\pgfsetroundjoin%
\definecolor{currentfill}{rgb}{0.000000,0.000000,0.000000}%
\pgfsetfillcolor{currentfill}%
\pgfsetlinewidth{0.803000pt}%
\definecolor{currentstroke}{rgb}{0.000000,0.000000,0.000000}%
\pgfsetstrokecolor{currentstroke}%
\pgfsetdash{}{0pt}%
\pgfsys@defobject{currentmarker}{\pgfqpoint{-0.048611in}{0.000000in}}{\pgfqpoint{0.000000in}{0.000000in}}{%
\pgfpathmoveto{\pgfqpoint{0.000000in}{0.000000in}}%
\pgfpathlineto{\pgfqpoint{-0.048611in}{0.000000in}}%
\pgfusepath{stroke,fill}%
}%
\begin{pgfscope}%
\pgfsys@transformshift{0.947176in}{0.902709in}%
\pgfsys@useobject{currentmarker}{}%
\end{pgfscope}%
\end{pgfscope}%
\begin{pgfscope}%
\definecolor{textcolor}{rgb}{0.000000,0.000000,0.000000}%
\pgfsetstrokecolor{textcolor}%
\pgfsetfillcolor{textcolor}%
\pgftext[x=0.603040in,y=0.854484in,left,base]{\color{textcolor}\rmfamily\fontsize{10.000000}{12.000000}\selectfont \(\displaystyle -80\)}%
\end{pgfscope}%
\begin{pgfscope}%
\pgfsetbuttcap%
\pgfsetroundjoin%
\definecolor{currentfill}{rgb}{0.000000,0.000000,0.000000}%
\pgfsetfillcolor{currentfill}%
\pgfsetlinewidth{0.803000pt}%
\definecolor{currentstroke}{rgb}{0.000000,0.000000,0.000000}%
\pgfsetstrokecolor{currentstroke}%
\pgfsetdash{}{0pt}%
\pgfsys@defobject{currentmarker}{\pgfqpoint{-0.048611in}{0.000000in}}{\pgfqpoint{0.000000in}{0.000000in}}{%
\pgfpathmoveto{\pgfqpoint{0.000000in}{0.000000in}}%
\pgfpathlineto{\pgfqpoint{-0.048611in}{0.000000in}}%
\pgfusepath{stroke,fill}%
}%
\begin{pgfscope}%
\pgfsys@transformshift{0.947176in}{1.484302in}%
\pgfsys@useobject{currentmarker}{}%
\end{pgfscope}%
\end{pgfscope}%
\begin{pgfscope}%
\definecolor{textcolor}{rgb}{0.000000,0.000000,0.000000}%
\pgfsetstrokecolor{textcolor}%
\pgfsetfillcolor{textcolor}%
\pgftext[x=0.603040in,y=1.436076in,left,base]{\color{textcolor}\rmfamily\fontsize{10.000000}{12.000000}\selectfont \(\displaystyle -60\)}%
\end{pgfscope}%
\begin{pgfscope}%
\pgfsetbuttcap%
\pgfsetroundjoin%
\definecolor{currentfill}{rgb}{0.000000,0.000000,0.000000}%
\pgfsetfillcolor{currentfill}%
\pgfsetlinewidth{0.803000pt}%
\definecolor{currentstroke}{rgb}{0.000000,0.000000,0.000000}%
\pgfsetstrokecolor{currentstroke}%
\pgfsetdash{}{0pt}%
\pgfsys@defobject{currentmarker}{\pgfqpoint{-0.048611in}{0.000000in}}{\pgfqpoint{0.000000in}{0.000000in}}{%
\pgfpathmoveto{\pgfqpoint{0.000000in}{0.000000in}}%
\pgfpathlineto{\pgfqpoint{-0.048611in}{0.000000in}}%
\pgfusepath{stroke,fill}%
}%
\begin{pgfscope}%
\pgfsys@transformshift{0.947176in}{2.065894in}%
\pgfsys@useobject{currentmarker}{}%
\end{pgfscope}%
\end{pgfscope}%
\begin{pgfscope}%
\definecolor{textcolor}{rgb}{0.000000,0.000000,0.000000}%
\pgfsetstrokecolor{textcolor}%
\pgfsetfillcolor{textcolor}%
\pgftext[x=0.603040in,y=2.017669in,left,base]{\color{textcolor}\rmfamily\fontsize{10.000000}{12.000000}\selectfont \(\displaystyle -40\)}%
\end{pgfscope}%
\begin{pgfscope}%
\pgfsetbuttcap%
\pgfsetroundjoin%
\definecolor{currentfill}{rgb}{0.000000,0.000000,0.000000}%
\pgfsetfillcolor{currentfill}%
\pgfsetlinewidth{0.803000pt}%
\definecolor{currentstroke}{rgb}{0.000000,0.000000,0.000000}%
\pgfsetstrokecolor{currentstroke}%
\pgfsetdash{}{0pt}%
\pgfsys@defobject{currentmarker}{\pgfqpoint{-0.048611in}{0.000000in}}{\pgfqpoint{0.000000in}{0.000000in}}{%
\pgfpathmoveto{\pgfqpoint{0.000000in}{0.000000in}}%
\pgfpathlineto{\pgfqpoint{-0.048611in}{0.000000in}}%
\pgfusepath{stroke,fill}%
}%
\begin{pgfscope}%
\pgfsys@transformshift{0.947176in}{2.647486in}%
\pgfsys@useobject{currentmarker}{}%
\end{pgfscope}%
\end{pgfscope}%
\begin{pgfscope}%
\definecolor{textcolor}{rgb}{0.000000,0.000000,0.000000}%
\pgfsetstrokecolor{textcolor}%
\pgfsetfillcolor{textcolor}%
\pgftext[x=0.603040in,y=2.599261in,left,base]{\color{textcolor}\rmfamily\fontsize{10.000000}{12.000000}\selectfont \(\displaystyle -20\)}%
\end{pgfscope}%
\begin{pgfscope}%
\definecolor{textcolor}{rgb}{0.000000,0.000000,0.000000}%
\pgfsetstrokecolor{textcolor}%
\pgfsetfillcolor{textcolor}%
\pgftext[x=0.255817in,y=1.847191in,,bottom]{\color{textcolor}\rmfamily\fontsize{10.000000}{12.000000}\selectfont \(\displaystyle \phi\) \textit{(º)}}%
\end{pgfscope}%
\begin{pgfscope}%
\pgfpathrectangle{\pgfqpoint{0.947176in}{0.499691in}}{\pgfqpoint{3.875000in}{2.695000in}}%
\pgfusepath{clip}%
\pgfsetbuttcap%
\pgfsetroundjoin%
\definecolor{currentfill}{rgb}{0.121569,0.466667,0.705882}%
\pgfsetfillcolor{currentfill}%
\pgfsetlinewidth{1.003750pt}%
\definecolor{currentstroke}{rgb}{0.121569,0.466667,0.705882}%
\pgfsetstrokecolor{currentstroke}%
\pgfsetdash{}{0pt}%
\pgfpathmoveto{\pgfqpoint{1.655316in}{0.719796in}}%
\pgfpathcurveto{\pgfqpoint{1.666366in}{0.719796in}}{\pgfqpoint{1.676965in}{0.724187in}}{\pgfqpoint{1.684779in}{0.732000in}}%
\pgfpathcurveto{\pgfqpoint{1.692593in}{0.739814in}}{\pgfqpoint{1.696983in}{0.750413in}}{\pgfqpoint{1.696983in}{0.761463in}}%
\pgfpathcurveto{\pgfqpoint{1.696983in}{0.772513in}}{\pgfqpoint{1.692593in}{0.783112in}}{\pgfqpoint{1.684779in}{0.790926in}}%
\pgfpathcurveto{\pgfqpoint{1.676965in}{0.798739in}}{\pgfqpoint{1.666366in}{0.803130in}}{\pgfqpoint{1.655316in}{0.803130in}}%
\pgfpathcurveto{\pgfqpoint{1.644266in}{0.803130in}}{\pgfqpoint{1.633667in}{0.798739in}}{\pgfqpoint{1.625853in}{0.790926in}}%
\pgfpathcurveto{\pgfqpoint{1.618040in}{0.783112in}}{\pgfqpoint{1.613650in}{0.772513in}}{\pgfqpoint{1.613650in}{0.761463in}}%
\pgfpathcurveto{\pgfqpoint{1.613650in}{0.750413in}}{\pgfqpoint{1.618040in}{0.739814in}}{\pgfqpoint{1.625853in}{0.732000in}}%
\pgfpathcurveto{\pgfqpoint{1.633667in}{0.724187in}}{\pgfqpoint{1.644266in}{0.719796in}}{\pgfqpoint{1.655316in}{0.719796in}}%
\pgfpathclose%
\pgfusepath{stroke,fill}%
\end{pgfscope}%
\begin{pgfscope}%
\pgfpathrectangle{\pgfqpoint{0.947176in}{0.499691in}}{\pgfqpoint{3.875000in}{2.695000in}}%
\pgfusepath{clip}%
\pgfsetbuttcap%
\pgfsetroundjoin%
\definecolor{currentfill}{rgb}{0.121569,0.466667,0.705882}%
\pgfsetfillcolor{currentfill}%
\pgfsetlinewidth{1.003750pt}%
\definecolor{currentstroke}{rgb}{0.121569,0.466667,0.705882}%
\pgfsetstrokecolor{currentstroke}%
\pgfsetdash{}{0pt}%
\pgfpathmoveto{\pgfqpoint{1.937247in}{0.785637in}}%
\pgfpathcurveto{\pgfqpoint{1.948297in}{0.785637in}}{\pgfqpoint{1.958896in}{0.790027in}}{\pgfqpoint{1.966710in}{0.797841in}}%
\pgfpathcurveto{\pgfqpoint{1.974524in}{0.805654in}}{\pgfqpoint{1.978914in}{0.816253in}}{\pgfqpoint{1.978914in}{0.827303in}}%
\pgfpathcurveto{\pgfqpoint{1.978914in}{0.838354in}}{\pgfqpoint{1.974524in}{0.848953in}}{\pgfqpoint{1.966710in}{0.856766in}}%
\pgfpathcurveto{\pgfqpoint{1.958896in}{0.864580in}}{\pgfqpoint{1.948297in}{0.868970in}}{\pgfqpoint{1.937247in}{0.868970in}}%
\pgfpathcurveto{\pgfqpoint{1.926197in}{0.868970in}}{\pgfqpoint{1.915598in}{0.864580in}}{\pgfqpoint{1.907784in}{0.856766in}}%
\pgfpathcurveto{\pgfqpoint{1.899971in}{0.848953in}}{\pgfqpoint{1.895580in}{0.838354in}}{\pgfqpoint{1.895580in}{0.827303in}}%
\pgfpathcurveto{\pgfqpoint{1.895580in}{0.816253in}}{\pgfqpoint{1.899971in}{0.805654in}}{\pgfqpoint{1.907784in}{0.797841in}}%
\pgfpathcurveto{\pgfqpoint{1.915598in}{0.790027in}}{\pgfqpoint{1.926197in}{0.785637in}}{\pgfqpoint{1.937247in}{0.785637in}}%
\pgfpathclose%
\pgfusepath{stroke,fill}%
\end{pgfscope}%
\begin{pgfscope}%
\pgfpathrectangle{\pgfqpoint{0.947176in}{0.499691in}}{\pgfqpoint{3.875000in}{2.695000in}}%
\pgfusepath{clip}%
\pgfsetbuttcap%
\pgfsetroundjoin%
\definecolor{currentfill}{rgb}{0.121569,0.466667,0.705882}%
\pgfsetfillcolor{currentfill}%
\pgfsetlinewidth{1.003750pt}%
\definecolor{currentstroke}{rgb}{0.121569,0.466667,0.705882}%
\pgfsetstrokecolor{currentstroke}%
\pgfsetdash{}{0pt}%
\pgfpathmoveto{\pgfqpoint{2.134599in}{0.850808in}}%
\pgfpathcurveto{\pgfqpoint{2.145649in}{0.850808in}}{\pgfqpoint{2.156248in}{0.855198in}}{\pgfqpoint{2.164062in}{0.863012in}}%
\pgfpathcurveto{\pgfqpoint{2.171875in}{0.870826in}}{\pgfqpoint{2.176265in}{0.881425in}}{\pgfqpoint{2.176265in}{0.892475in}}%
\pgfpathcurveto{\pgfqpoint{2.176265in}{0.903525in}}{\pgfqpoint{2.171875in}{0.914124in}}{\pgfqpoint{2.164062in}{0.921938in}}%
\pgfpathcurveto{\pgfqpoint{2.156248in}{0.929751in}}{\pgfqpoint{2.145649in}{0.934141in}}{\pgfqpoint{2.134599in}{0.934141in}}%
\pgfpathcurveto{\pgfqpoint{2.123549in}{0.934141in}}{\pgfqpoint{2.112950in}{0.929751in}}{\pgfqpoint{2.105136in}{0.921938in}}%
\pgfpathcurveto{\pgfqpoint{2.097322in}{0.914124in}}{\pgfqpoint{2.092932in}{0.903525in}}{\pgfqpoint{2.092932in}{0.892475in}}%
\pgfpathcurveto{\pgfqpoint{2.092932in}{0.881425in}}{\pgfqpoint{2.097322in}{0.870826in}}{\pgfqpoint{2.105136in}{0.863012in}}%
\pgfpathcurveto{\pgfqpoint{2.112950in}{0.855198in}}{\pgfqpoint{2.123549in}{0.850808in}}{\pgfqpoint{2.134599in}{0.850808in}}%
\pgfpathclose%
\pgfusepath{stroke,fill}%
\end{pgfscope}%
\begin{pgfscope}%
\pgfpathrectangle{\pgfqpoint{0.947176in}{0.499691in}}{\pgfqpoint{3.875000in}{2.695000in}}%
\pgfusepath{clip}%
\pgfsetbuttcap%
\pgfsetroundjoin%
\definecolor{currentfill}{rgb}{0.121569,0.466667,0.705882}%
\pgfsetfillcolor{currentfill}%
\pgfsetlinewidth{1.003750pt}%
\definecolor{currentstroke}{rgb}{0.121569,0.466667,0.705882}%
\pgfsetstrokecolor{currentstroke}%
\pgfsetdash{}{0pt}%
\pgfpathmoveto{\pgfqpoint{2.289661in}{0.931050in}}%
\pgfpathcurveto{\pgfqpoint{2.300711in}{0.931050in}}{\pgfqpoint{2.311310in}{0.935440in}}{\pgfqpoint{2.319124in}{0.943253in}}%
\pgfpathcurveto{\pgfqpoint{2.326937in}{0.951067in}}{\pgfqpoint{2.331327in}{0.961666in}}{\pgfqpoint{2.331327in}{0.972716in}}%
\pgfpathcurveto{\pgfqpoint{2.331327in}{0.983766in}}{\pgfqpoint{2.326937in}{0.994365in}}{\pgfqpoint{2.319124in}{1.002179in}}%
\pgfpathcurveto{\pgfqpoint{2.311310in}{1.009993in}}{\pgfqpoint{2.300711in}{1.014383in}}{\pgfqpoint{2.289661in}{1.014383in}}%
\pgfpathcurveto{\pgfqpoint{2.278611in}{1.014383in}}{\pgfqpoint{2.268012in}{1.009993in}}{\pgfqpoint{2.260198in}{1.002179in}}%
\pgfpathcurveto{\pgfqpoint{2.252384in}{0.994365in}}{\pgfqpoint{2.247994in}{0.983766in}}{\pgfqpoint{2.247994in}{0.972716in}}%
\pgfpathcurveto{\pgfqpoint{2.247994in}{0.961666in}}{\pgfqpoint{2.252384in}{0.951067in}}{\pgfqpoint{2.260198in}{0.943253in}}%
\pgfpathcurveto{\pgfqpoint{2.268012in}{0.935440in}}{\pgfqpoint{2.278611in}{0.931050in}}{\pgfqpoint{2.289661in}{0.931050in}}%
\pgfpathclose%
\pgfusepath{stroke,fill}%
\end{pgfscope}%
\begin{pgfscope}%
\pgfpathrectangle{\pgfqpoint{0.947176in}{0.499691in}}{\pgfqpoint{3.875000in}{2.695000in}}%
\pgfusepath{clip}%
\pgfsetbuttcap%
\pgfsetroundjoin%
\definecolor{currentfill}{rgb}{0.121569,0.466667,0.705882}%
\pgfsetfillcolor{currentfill}%
\pgfsetlinewidth{1.003750pt}%
\definecolor{currentstroke}{rgb}{0.121569,0.466667,0.705882}%
\pgfsetstrokecolor{currentstroke}%
\pgfsetdash{}{0pt}%
\pgfpathmoveto{\pgfqpoint{2.402433in}{1.025117in}}%
\pgfpathcurveto{\pgfqpoint{2.413483in}{1.025117in}}{\pgfqpoint{2.424082in}{1.029507in}}{\pgfqpoint{2.431896in}{1.037321in}}%
\pgfpathcurveto{\pgfqpoint{2.439710in}{1.045135in}}{\pgfqpoint{2.444100in}{1.055734in}}{\pgfqpoint{2.444100in}{1.066784in}}%
\pgfpathcurveto{\pgfqpoint{2.444100in}{1.077834in}}{\pgfqpoint{2.439710in}{1.088433in}}{\pgfqpoint{2.431896in}{1.096247in}}%
\pgfpathcurveto{\pgfqpoint{2.424082in}{1.104060in}}{\pgfqpoint{2.413483in}{1.108450in}}{\pgfqpoint{2.402433in}{1.108450in}}%
\pgfpathcurveto{\pgfqpoint{2.391383in}{1.108450in}}{\pgfqpoint{2.380784in}{1.104060in}}{\pgfqpoint{2.372970in}{1.096247in}}%
\pgfpathcurveto{\pgfqpoint{2.365157in}{1.088433in}}{\pgfqpoint{2.360767in}{1.077834in}}{\pgfqpoint{2.360767in}{1.066784in}}%
\pgfpathcurveto{\pgfqpoint{2.360767in}{1.055734in}}{\pgfqpoint{2.365157in}{1.045135in}}{\pgfqpoint{2.372970in}{1.037321in}}%
\pgfpathcurveto{\pgfqpoint{2.380784in}{1.029507in}}{\pgfqpoint{2.391383in}{1.025117in}}{\pgfqpoint{2.402433in}{1.025117in}}%
\pgfpathclose%
\pgfusepath{stroke,fill}%
\end{pgfscope}%
\begin{pgfscope}%
\pgfpathrectangle{\pgfqpoint{0.947176in}{0.499691in}}{\pgfqpoint{3.875000in}{2.695000in}}%
\pgfusepath{clip}%
\pgfsetbuttcap%
\pgfsetroundjoin%
\definecolor{currentfill}{rgb}{0.121569,0.466667,0.705882}%
\pgfsetfillcolor{currentfill}%
\pgfsetlinewidth{1.003750pt}%
\definecolor{currentstroke}{rgb}{0.121569,0.466667,0.705882}%
\pgfsetstrokecolor{currentstroke}%
\pgfsetdash{}{0pt}%
\pgfpathmoveto{\pgfqpoint{2.501109in}{1.086255in}}%
\pgfpathcurveto{\pgfqpoint{2.512159in}{1.086255in}}{\pgfqpoint{2.522758in}{1.090645in}}{\pgfqpoint{2.530572in}{1.098459in}}%
\pgfpathcurveto{\pgfqpoint{2.538385in}{1.106272in}}{\pgfqpoint{2.542776in}{1.116871in}}{\pgfqpoint{2.542776in}{1.127921in}}%
\pgfpathcurveto{\pgfqpoint{2.542776in}{1.138972in}}{\pgfqpoint{2.538385in}{1.149571in}}{\pgfqpoint{2.530572in}{1.157384in}}%
\pgfpathcurveto{\pgfqpoint{2.522758in}{1.165198in}}{\pgfqpoint{2.512159in}{1.169588in}}{\pgfqpoint{2.501109in}{1.169588in}}%
\pgfpathcurveto{\pgfqpoint{2.490059in}{1.169588in}}{\pgfqpoint{2.479460in}{1.165198in}}{\pgfqpoint{2.471646in}{1.157384in}}%
\pgfpathcurveto{\pgfqpoint{2.463833in}{1.149571in}}{\pgfqpoint{2.459442in}{1.138972in}}{\pgfqpoint{2.459442in}{1.127921in}}%
\pgfpathcurveto{\pgfqpoint{2.459442in}{1.116871in}}{\pgfqpoint{2.463833in}{1.106272in}}{\pgfqpoint{2.471646in}{1.098459in}}%
\pgfpathcurveto{\pgfqpoint{2.479460in}{1.090645in}}{\pgfqpoint{2.490059in}{1.086255in}}{\pgfqpoint{2.501109in}{1.086255in}}%
\pgfpathclose%
\pgfusepath{stroke,fill}%
\end{pgfscope}%
\begin{pgfscope}%
\pgfpathrectangle{\pgfqpoint{0.947176in}{0.499691in}}{\pgfqpoint{3.875000in}{2.695000in}}%
\pgfusepath{clip}%
\pgfsetbuttcap%
\pgfsetroundjoin%
\definecolor{currentfill}{rgb}{0.121569,0.466667,0.705882}%
\pgfsetfillcolor{currentfill}%
\pgfsetlinewidth{1.003750pt}%
\definecolor{currentstroke}{rgb}{0.121569,0.466667,0.705882}%
\pgfsetstrokecolor{currentstroke}%
\pgfsetdash{}{0pt}%
\pgfpathmoveto{\pgfqpoint{2.585688in}{1.145990in}}%
\pgfpathcurveto{\pgfqpoint{2.596738in}{1.145990in}}{\pgfqpoint{2.607337in}{1.150381in}}{\pgfqpoint{2.615151in}{1.158194in}}%
\pgfpathcurveto{\pgfqpoint{2.622965in}{1.166008in}}{\pgfqpoint{2.627355in}{1.176607in}}{\pgfqpoint{2.627355in}{1.187657in}}%
\pgfpathcurveto{\pgfqpoint{2.627355in}{1.198707in}}{\pgfqpoint{2.622965in}{1.209306in}}{\pgfqpoint{2.615151in}{1.217120in}}%
\pgfpathcurveto{\pgfqpoint{2.607337in}{1.224933in}}{\pgfqpoint{2.596738in}{1.229324in}}{\pgfqpoint{2.585688in}{1.229324in}}%
\pgfpathcurveto{\pgfqpoint{2.574638in}{1.229324in}}{\pgfqpoint{2.564039in}{1.224933in}}{\pgfqpoint{2.556226in}{1.217120in}}%
\pgfpathcurveto{\pgfqpoint{2.548412in}{1.209306in}}{\pgfqpoint{2.544022in}{1.198707in}}{\pgfqpoint{2.544022in}{1.187657in}}%
\pgfpathcurveto{\pgfqpoint{2.544022in}{1.176607in}}{\pgfqpoint{2.548412in}{1.166008in}}{\pgfqpoint{2.556226in}{1.158194in}}%
\pgfpathcurveto{\pgfqpoint{2.564039in}{1.150381in}}{\pgfqpoint{2.574638in}{1.145990in}}{\pgfqpoint{2.585688in}{1.145990in}}%
\pgfpathclose%
\pgfusepath{stroke,fill}%
\end{pgfscope}%
\begin{pgfscope}%
\pgfpathrectangle{\pgfqpoint{0.947176in}{0.499691in}}{\pgfqpoint{3.875000in}{2.695000in}}%
\pgfusepath{clip}%
\pgfsetbuttcap%
\pgfsetroundjoin%
\definecolor{currentfill}{rgb}{0.121569,0.466667,0.705882}%
\pgfsetfillcolor{currentfill}%
\pgfsetlinewidth{1.003750pt}%
\definecolor{currentstroke}{rgb}{0.121569,0.466667,0.705882}%
\pgfsetstrokecolor{currentstroke}%
\pgfsetdash{}{0pt}%
\pgfpathmoveto{\pgfqpoint{2.670268in}{1.204223in}}%
\pgfpathcurveto{\pgfqpoint{2.681318in}{1.204223in}}{\pgfqpoint{2.691917in}{1.208613in}}{\pgfqpoint{2.699730in}{1.216427in}}%
\pgfpathcurveto{\pgfqpoint{2.707544in}{1.224241in}}{\pgfqpoint{2.711934in}{1.234840in}}{\pgfqpoint{2.711934in}{1.245890in}}%
\pgfpathcurveto{\pgfqpoint{2.711934in}{1.256940in}}{\pgfqpoint{2.707544in}{1.267539in}}{\pgfqpoint{2.699730in}{1.275353in}}%
\pgfpathcurveto{\pgfqpoint{2.691917in}{1.283166in}}{\pgfqpoint{2.681318in}{1.287556in}}{\pgfqpoint{2.670268in}{1.287556in}}%
\pgfpathcurveto{\pgfqpoint{2.659217in}{1.287556in}}{\pgfqpoint{2.648618in}{1.283166in}}{\pgfqpoint{2.640805in}{1.275353in}}%
\pgfpathcurveto{\pgfqpoint{2.632991in}{1.267539in}}{\pgfqpoint{2.628601in}{1.256940in}}{\pgfqpoint{2.628601in}{1.245890in}}%
\pgfpathcurveto{\pgfqpoint{2.628601in}{1.234840in}}{\pgfqpoint{2.632991in}{1.224241in}}{\pgfqpoint{2.640805in}{1.216427in}}%
\pgfpathcurveto{\pgfqpoint{2.648618in}{1.208613in}}{\pgfqpoint{2.659217in}{1.204223in}}{\pgfqpoint{2.670268in}{1.204223in}}%
\pgfpathclose%
\pgfusepath{stroke,fill}%
\end{pgfscope}%
\begin{pgfscope}%
\pgfpathrectangle{\pgfqpoint{0.947176in}{0.499691in}}{\pgfqpoint{3.875000in}{2.695000in}}%
\pgfusepath{clip}%
\pgfsetbuttcap%
\pgfsetroundjoin%
\definecolor{currentfill}{rgb}{0.121569,0.466667,0.705882}%
\pgfsetfillcolor{currentfill}%
\pgfsetlinewidth{1.003750pt}%
\definecolor{currentstroke}{rgb}{0.121569,0.466667,0.705882}%
\pgfsetstrokecolor{currentstroke}%
\pgfsetdash{}{0pt}%
\pgfpathmoveto{\pgfqpoint{2.726654in}{1.288584in}}%
\pgfpathcurveto{\pgfqpoint{2.737704in}{1.288584in}}{\pgfqpoint{2.748303in}{1.292974in}}{\pgfqpoint{2.756117in}{1.300788in}}%
\pgfpathcurveto{\pgfqpoint{2.763930in}{1.308601in}}{\pgfqpoint{2.768320in}{1.319200in}}{\pgfqpoint{2.768320in}{1.330250in}}%
\pgfpathcurveto{\pgfqpoint{2.768320in}{1.341301in}}{\pgfqpoint{2.763930in}{1.351900in}}{\pgfqpoint{2.756117in}{1.359713in}}%
\pgfpathcurveto{\pgfqpoint{2.748303in}{1.367527in}}{\pgfqpoint{2.737704in}{1.371917in}}{\pgfqpoint{2.726654in}{1.371917in}}%
\pgfpathcurveto{\pgfqpoint{2.715604in}{1.371917in}}{\pgfqpoint{2.705005in}{1.367527in}}{\pgfqpoint{2.697191in}{1.359713in}}%
\pgfpathcurveto{\pgfqpoint{2.689377in}{1.351900in}}{\pgfqpoint{2.684987in}{1.341301in}}{\pgfqpoint{2.684987in}{1.330250in}}%
\pgfpathcurveto{\pgfqpoint{2.684987in}{1.319200in}}{\pgfqpoint{2.689377in}{1.308601in}}{\pgfqpoint{2.697191in}{1.300788in}}%
\pgfpathcurveto{\pgfqpoint{2.705005in}{1.292974in}}{\pgfqpoint{2.715604in}{1.288584in}}{\pgfqpoint{2.726654in}{1.288584in}}%
\pgfpathclose%
\pgfusepath{stroke,fill}%
\end{pgfscope}%
\begin{pgfscope}%
\pgfpathrectangle{\pgfqpoint{0.947176in}{0.499691in}}{\pgfqpoint{3.875000in}{2.695000in}}%
\pgfusepath{clip}%
\pgfsetbuttcap%
\pgfsetroundjoin%
\definecolor{currentfill}{rgb}{0.121569,0.466667,0.705882}%
\pgfsetfillcolor{currentfill}%
\pgfsetlinewidth{1.003750pt}%
\definecolor{currentstroke}{rgb}{0.121569,0.466667,0.705882}%
\pgfsetstrokecolor{currentstroke}%
\pgfsetdash{}{0pt}%
\pgfpathmoveto{\pgfqpoint{2.797137in}{1.329366in}}%
\pgfpathcurveto{\pgfqpoint{2.808187in}{1.329366in}}{\pgfqpoint{2.818786in}{1.333756in}}{\pgfqpoint{2.826599in}{1.341570in}}%
\pgfpathcurveto{\pgfqpoint{2.834413in}{1.349383in}}{\pgfqpoint{2.838803in}{1.359982in}}{\pgfqpoint{2.838803in}{1.371032in}}%
\pgfpathcurveto{\pgfqpoint{2.838803in}{1.382082in}}{\pgfqpoint{2.834413in}{1.392681in}}{\pgfqpoint{2.826599in}{1.400495in}}%
\pgfpathcurveto{\pgfqpoint{2.818786in}{1.408309in}}{\pgfqpoint{2.808187in}{1.412699in}}{\pgfqpoint{2.797137in}{1.412699in}}%
\pgfpathcurveto{\pgfqpoint{2.786086in}{1.412699in}}{\pgfqpoint{2.775487in}{1.408309in}}{\pgfqpoint{2.767674in}{1.400495in}}%
\pgfpathcurveto{\pgfqpoint{2.759860in}{1.392681in}}{\pgfqpoint{2.755470in}{1.382082in}}{\pgfqpoint{2.755470in}{1.371032in}}%
\pgfpathcurveto{\pgfqpoint{2.755470in}{1.359982in}}{\pgfqpoint{2.759860in}{1.349383in}}{\pgfqpoint{2.767674in}{1.341570in}}%
\pgfpathcurveto{\pgfqpoint{2.775487in}{1.333756in}}{\pgfqpoint{2.786086in}{1.329366in}}{\pgfqpoint{2.797137in}{1.329366in}}%
\pgfpathclose%
\pgfusepath{stroke,fill}%
\end{pgfscope}%
\begin{pgfscope}%
\pgfpathrectangle{\pgfqpoint{0.947176in}{0.499691in}}{\pgfqpoint{3.875000in}{2.695000in}}%
\pgfusepath{clip}%
\pgfsetbuttcap%
\pgfsetroundjoin%
\definecolor{currentfill}{rgb}{0.121569,0.466667,0.705882}%
\pgfsetfillcolor{currentfill}%
\pgfsetlinewidth{1.003750pt}%
\definecolor{currentstroke}{rgb}{0.121569,0.466667,0.705882}%
\pgfsetstrokecolor{currentstroke}%
\pgfsetdash{}{0pt}%
\pgfpathmoveto{\pgfqpoint{2.853523in}{1.408053in}}%
\pgfpathcurveto{\pgfqpoint{2.864573in}{1.408053in}}{\pgfqpoint{2.875172in}{1.412444in}}{\pgfqpoint{2.882986in}{1.420257in}}%
\pgfpathcurveto{\pgfqpoint{2.890799in}{1.428071in}}{\pgfqpoint{2.895189in}{1.438670in}}{\pgfqpoint{2.895189in}{1.449720in}}%
\pgfpathcurveto{\pgfqpoint{2.895189in}{1.460770in}}{\pgfqpoint{2.890799in}{1.471369in}}{\pgfqpoint{2.882986in}{1.479183in}}%
\pgfpathcurveto{\pgfqpoint{2.875172in}{1.486996in}}{\pgfqpoint{2.864573in}{1.491387in}}{\pgfqpoint{2.853523in}{1.491387in}}%
\pgfpathcurveto{\pgfqpoint{2.842473in}{1.491387in}}{\pgfqpoint{2.831874in}{1.486996in}}{\pgfqpoint{2.824060in}{1.479183in}}%
\pgfpathcurveto{\pgfqpoint{2.816246in}{1.471369in}}{\pgfqpoint{2.811856in}{1.460770in}}{\pgfqpoint{2.811856in}{1.449720in}}%
\pgfpathcurveto{\pgfqpoint{2.811856in}{1.438670in}}{\pgfqpoint{2.816246in}{1.428071in}}{\pgfqpoint{2.824060in}{1.420257in}}%
\pgfpathcurveto{\pgfqpoint{2.831874in}{1.412444in}}{\pgfqpoint{2.842473in}{1.408053in}}{\pgfqpoint{2.853523in}{1.408053in}}%
\pgfpathclose%
\pgfusepath{stroke,fill}%
\end{pgfscope}%
\begin{pgfscope}%
\pgfpathrectangle{\pgfqpoint{0.947176in}{0.499691in}}{\pgfqpoint{3.875000in}{2.695000in}}%
\pgfusepath{clip}%
\pgfsetbuttcap%
\pgfsetroundjoin%
\definecolor{currentfill}{rgb}{0.121569,0.466667,0.705882}%
\pgfsetfillcolor{currentfill}%
\pgfsetlinewidth{1.003750pt}%
\definecolor{currentstroke}{rgb}{0.121569,0.466667,0.705882}%
\pgfsetstrokecolor{currentstroke}%
\pgfsetdash{}{0pt}%
\pgfpathmoveto{\pgfqpoint{2.895812in}{1.458356in}}%
\pgfpathcurveto{\pgfqpoint{2.906863in}{1.458356in}}{\pgfqpoint{2.917462in}{1.462746in}}{\pgfqpoint{2.925275in}{1.470559in}}%
\pgfpathcurveto{\pgfqpoint{2.933089in}{1.478373in}}{\pgfqpoint{2.937479in}{1.488972in}}{\pgfqpoint{2.937479in}{1.500022in}}%
\pgfpathcurveto{\pgfqpoint{2.937479in}{1.511072in}}{\pgfqpoint{2.933089in}{1.521671in}}{\pgfqpoint{2.925275in}{1.529485in}}%
\pgfpathcurveto{\pgfqpoint{2.917462in}{1.537299in}}{\pgfqpoint{2.906863in}{1.541689in}}{\pgfqpoint{2.895812in}{1.541689in}}%
\pgfpathcurveto{\pgfqpoint{2.884762in}{1.541689in}}{\pgfqpoint{2.874163in}{1.537299in}}{\pgfqpoint{2.866350in}{1.529485in}}%
\pgfpathcurveto{\pgfqpoint{2.858536in}{1.521671in}}{\pgfqpoint{2.854146in}{1.511072in}}{\pgfqpoint{2.854146in}{1.500022in}}%
\pgfpathcurveto{\pgfqpoint{2.854146in}{1.488972in}}{\pgfqpoint{2.858536in}{1.478373in}}{\pgfqpoint{2.866350in}{1.470559in}}%
\pgfpathcurveto{\pgfqpoint{2.874163in}{1.462746in}}{\pgfqpoint{2.884762in}{1.458356in}}{\pgfqpoint{2.895812in}{1.458356in}}%
\pgfpathclose%
\pgfusepath{stroke,fill}%
\end{pgfscope}%
\begin{pgfscope}%
\pgfpathrectangle{\pgfqpoint{0.947176in}{0.499691in}}{\pgfqpoint{3.875000in}{2.695000in}}%
\pgfusepath{clip}%
\pgfsetbuttcap%
\pgfsetroundjoin%
\definecolor{currentfill}{rgb}{0.121569,0.466667,0.705882}%
\pgfsetfillcolor{currentfill}%
\pgfsetlinewidth{1.003750pt}%
\definecolor{currentstroke}{rgb}{0.121569,0.466667,0.705882}%
\pgfsetstrokecolor{currentstroke}%
\pgfsetdash{}{0pt}%
\pgfpathmoveto{\pgfqpoint{2.938102in}{1.518802in}}%
\pgfpathcurveto{\pgfqpoint{2.949152in}{1.518802in}}{\pgfqpoint{2.959751in}{1.523192in}}{\pgfqpoint{2.967565in}{1.531005in}}%
\pgfpathcurveto{\pgfqpoint{2.975378in}{1.538819in}}{\pgfqpoint{2.979769in}{1.549418in}}{\pgfqpoint{2.979769in}{1.560468in}}%
\pgfpathcurveto{\pgfqpoint{2.979769in}{1.571518in}}{\pgfqpoint{2.975378in}{1.582117in}}{\pgfqpoint{2.967565in}{1.589931in}}%
\pgfpathcurveto{\pgfqpoint{2.959751in}{1.597745in}}{\pgfqpoint{2.949152in}{1.602135in}}{\pgfqpoint{2.938102in}{1.602135in}}%
\pgfpathcurveto{\pgfqpoint{2.927052in}{1.602135in}}{\pgfqpoint{2.916453in}{1.597745in}}{\pgfqpoint{2.908639in}{1.589931in}}%
\pgfpathcurveto{\pgfqpoint{2.900826in}{1.582117in}}{\pgfqpoint{2.896435in}{1.571518in}}{\pgfqpoint{2.896435in}{1.560468in}}%
\pgfpathcurveto{\pgfqpoint{2.896435in}{1.549418in}}{\pgfqpoint{2.900826in}{1.538819in}}{\pgfqpoint{2.908639in}{1.531005in}}%
\pgfpathcurveto{\pgfqpoint{2.916453in}{1.523192in}}{\pgfqpoint{2.927052in}{1.518802in}}{\pgfqpoint{2.938102in}{1.518802in}}%
\pgfpathclose%
\pgfusepath{stroke,fill}%
\end{pgfscope}%
\begin{pgfscope}%
\pgfpathrectangle{\pgfqpoint{0.947176in}{0.499691in}}{\pgfqpoint{3.875000in}{2.695000in}}%
\pgfusepath{clip}%
\pgfsetbuttcap%
\pgfsetroundjoin%
\definecolor{currentfill}{rgb}{0.121569,0.466667,0.705882}%
\pgfsetfillcolor{currentfill}%
\pgfsetlinewidth{1.003750pt}%
\definecolor{currentstroke}{rgb}{0.121569,0.466667,0.705882}%
\pgfsetstrokecolor{currentstroke}%
\pgfsetdash{}{0pt}%
\pgfpathmoveto{\pgfqpoint{2.980392in}{1.553780in}}%
\pgfpathcurveto{\pgfqpoint{2.991442in}{1.553780in}}{\pgfqpoint{3.002041in}{1.558170in}}{\pgfqpoint{3.009854in}{1.565984in}}%
\pgfpathcurveto{\pgfqpoint{3.017668in}{1.573797in}}{\pgfqpoint{3.022058in}{1.584396in}}{\pgfqpoint{3.022058in}{1.595446in}}%
\pgfpathcurveto{\pgfqpoint{3.022058in}{1.606497in}}{\pgfqpoint{3.017668in}{1.617096in}}{\pgfqpoint{3.009854in}{1.624909in}}%
\pgfpathcurveto{\pgfqpoint{3.002041in}{1.632723in}}{\pgfqpoint{2.991442in}{1.637113in}}{\pgfqpoint{2.980392in}{1.637113in}}%
\pgfpathcurveto{\pgfqpoint{2.969342in}{1.637113in}}{\pgfqpoint{2.958743in}{1.632723in}}{\pgfqpoint{2.950929in}{1.624909in}}%
\pgfpathcurveto{\pgfqpoint{2.943115in}{1.617096in}}{\pgfqpoint{2.938725in}{1.606497in}}{\pgfqpoint{2.938725in}{1.595446in}}%
\pgfpathcurveto{\pgfqpoint{2.938725in}{1.584396in}}{\pgfqpoint{2.943115in}{1.573797in}}{\pgfqpoint{2.950929in}{1.565984in}}%
\pgfpathcurveto{\pgfqpoint{2.958743in}{1.558170in}}{\pgfqpoint{2.969342in}{1.553780in}}{\pgfqpoint{2.980392in}{1.553780in}}%
\pgfpathclose%
\pgfusepath{stroke,fill}%
\end{pgfscope}%
\begin{pgfscope}%
\pgfpathrectangle{\pgfqpoint{0.947176in}{0.499691in}}{\pgfqpoint{3.875000in}{2.695000in}}%
\pgfusepath{clip}%
\pgfsetbuttcap%
\pgfsetroundjoin%
\definecolor{currentfill}{rgb}{0.121569,0.466667,0.705882}%
\pgfsetfillcolor{currentfill}%
\pgfsetlinewidth{1.003750pt}%
\definecolor{currentstroke}{rgb}{0.121569,0.466667,0.705882}%
\pgfsetstrokecolor{currentstroke}%
\pgfsetdash{}{0pt}%
\pgfpathmoveto{\pgfqpoint{3.022681in}{1.609956in}}%
\pgfpathcurveto{\pgfqpoint{3.033731in}{1.609956in}}{\pgfqpoint{3.044330in}{1.614346in}}{\pgfqpoint{3.052144in}{1.622160in}}%
\pgfpathcurveto{\pgfqpoint{3.059958in}{1.629973in}}{\pgfqpoint{3.064348in}{1.640572in}}{\pgfqpoint{3.064348in}{1.651622in}}%
\pgfpathcurveto{\pgfqpoint{3.064348in}{1.662673in}}{\pgfqpoint{3.059958in}{1.673272in}}{\pgfqpoint{3.052144in}{1.681085in}}%
\pgfpathcurveto{\pgfqpoint{3.044330in}{1.688899in}}{\pgfqpoint{3.033731in}{1.693289in}}{\pgfqpoint{3.022681in}{1.693289in}}%
\pgfpathcurveto{\pgfqpoint{3.011631in}{1.693289in}}{\pgfqpoint{3.001032in}{1.688899in}}{\pgfqpoint{2.993219in}{1.681085in}}%
\pgfpathcurveto{\pgfqpoint{2.985405in}{1.673272in}}{\pgfqpoint{2.981015in}{1.662673in}}{\pgfqpoint{2.981015in}{1.651622in}}%
\pgfpathcurveto{\pgfqpoint{2.981015in}{1.640572in}}{\pgfqpoint{2.985405in}{1.629973in}}{\pgfqpoint{2.993219in}{1.622160in}}%
\pgfpathcurveto{\pgfqpoint{3.001032in}{1.614346in}}{\pgfqpoint{3.011631in}{1.609956in}}{\pgfqpoint{3.022681in}{1.609956in}}%
\pgfpathclose%
\pgfusepath{stroke,fill}%
\end{pgfscope}%
\begin{pgfscope}%
\pgfpathrectangle{\pgfqpoint{0.947176in}{0.499691in}}{\pgfqpoint{3.875000in}{2.695000in}}%
\pgfusepath{clip}%
\pgfsetbuttcap%
\pgfsetroundjoin%
\definecolor{currentfill}{rgb}{0.121569,0.466667,0.705882}%
\pgfsetfillcolor{currentfill}%
\pgfsetlinewidth{1.003750pt}%
\definecolor{currentstroke}{rgb}{0.121569,0.466667,0.705882}%
\pgfsetstrokecolor{currentstroke}%
\pgfsetdash{}{0pt}%
\pgfpathmoveto{\pgfqpoint{3.064971in}{1.653021in}}%
\pgfpathcurveto{\pgfqpoint{3.076021in}{1.653021in}}{\pgfqpoint{3.086620in}{1.657411in}}{\pgfqpoint{3.094434in}{1.665225in}}%
\pgfpathcurveto{\pgfqpoint{3.102247in}{1.673039in}}{\pgfqpoint{3.106638in}{1.683638in}}{\pgfqpoint{3.106638in}{1.694688in}}%
\pgfpathcurveto{\pgfqpoint{3.106638in}{1.705738in}}{\pgfqpoint{3.102247in}{1.716337in}}{\pgfqpoint{3.094434in}{1.724151in}}%
\pgfpathcurveto{\pgfqpoint{3.086620in}{1.731964in}}{\pgfqpoint{3.076021in}{1.736354in}}{\pgfqpoint{3.064971in}{1.736354in}}%
\pgfpathcurveto{\pgfqpoint{3.053921in}{1.736354in}}{\pgfqpoint{3.043322in}{1.731964in}}{\pgfqpoint{3.035508in}{1.724151in}}%
\pgfpathcurveto{\pgfqpoint{3.027695in}{1.716337in}}{\pgfqpoint{3.023304in}{1.705738in}}{\pgfqpoint{3.023304in}{1.694688in}}%
\pgfpathcurveto{\pgfqpoint{3.023304in}{1.683638in}}{\pgfqpoint{3.027695in}{1.673039in}}{\pgfqpoint{3.035508in}{1.665225in}}%
\pgfpathcurveto{\pgfqpoint{3.043322in}{1.657411in}}{\pgfqpoint{3.053921in}{1.653021in}}{\pgfqpoint{3.064971in}{1.653021in}}%
\pgfpathclose%
\pgfusepath{stroke,fill}%
\end{pgfscope}%
\begin{pgfscope}%
\pgfpathrectangle{\pgfqpoint{0.947176in}{0.499691in}}{\pgfqpoint{3.875000in}{2.695000in}}%
\pgfusepath{clip}%
\pgfsetbuttcap%
\pgfsetroundjoin%
\definecolor{currentfill}{rgb}{0.121569,0.466667,0.705882}%
\pgfsetfillcolor{currentfill}%
\pgfsetlinewidth{1.003750pt}%
\definecolor{currentstroke}{rgb}{0.121569,0.466667,0.705882}%
\pgfsetstrokecolor{currentstroke}%
\pgfsetdash{}{0pt}%
\pgfpathmoveto{\pgfqpoint{3.107261in}{1.714580in}}%
\pgfpathcurveto{\pgfqpoint{3.118311in}{1.714580in}}{\pgfqpoint{3.128910in}{1.718971in}}{\pgfqpoint{3.136723in}{1.726784in}}%
\pgfpathcurveto{\pgfqpoint{3.144537in}{1.734598in}}{\pgfqpoint{3.148927in}{1.745197in}}{\pgfqpoint{3.148927in}{1.756247in}}%
\pgfpathcurveto{\pgfqpoint{3.148927in}{1.767297in}}{\pgfqpoint{3.144537in}{1.777896in}}{\pgfqpoint{3.136723in}{1.785710in}}%
\pgfpathcurveto{\pgfqpoint{3.128910in}{1.793523in}}{\pgfqpoint{3.118311in}{1.797914in}}{\pgfqpoint{3.107261in}{1.797914in}}%
\pgfpathcurveto{\pgfqpoint{3.096210in}{1.797914in}}{\pgfqpoint{3.085611in}{1.793523in}}{\pgfqpoint{3.077798in}{1.785710in}}%
\pgfpathcurveto{\pgfqpoint{3.069984in}{1.777896in}}{\pgfqpoint{3.065594in}{1.767297in}}{\pgfqpoint{3.065594in}{1.756247in}}%
\pgfpathcurveto{\pgfqpoint{3.065594in}{1.745197in}}{\pgfqpoint{3.069984in}{1.734598in}}{\pgfqpoint{3.077798in}{1.726784in}}%
\pgfpathcurveto{\pgfqpoint{3.085611in}{1.718971in}}{\pgfqpoint{3.096210in}{1.714580in}}{\pgfqpoint{3.107261in}{1.714580in}}%
\pgfpathclose%
\pgfusepath{stroke,fill}%
\end{pgfscope}%
\begin{pgfscope}%
\pgfpathrectangle{\pgfqpoint{0.947176in}{0.499691in}}{\pgfqpoint{3.875000in}{2.695000in}}%
\pgfusepath{clip}%
\pgfsetbuttcap%
\pgfsetroundjoin%
\definecolor{currentfill}{rgb}{0.121569,0.466667,0.705882}%
\pgfsetfillcolor{currentfill}%
\pgfsetlinewidth{1.003750pt}%
\definecolor{currentstroke}{rgb}{0.121569,0.466667,0.705882}%
\pgfsetstrokecolor{currentstroke}%
\pgfsetdash{}{0pt}%
\pgfpathmoveto{\pgfqpoint{3.135454in}{1.763187in}}%
\pgfpathcurveto{\pgfqpoint{3.146504in}{1.763187in}}{\pgfqpoint{3.157103in}{1.767578in}}{\pgfqpoint{3.164916in}{1.775391in}}%
\pgfpathcurveto{\pgfqpoint{3.172730in}{1.783205in}}{\pgfqpoint{3.177120in}{1.793804in}}{\pgfqpoint{3.177120in}{1.804854in}}%
\pgfpathcurveto{\pgfqpoint{3.177120in}{1.815904in}}{\pgfqpoint{3.172730in}{1.826503in}}{\pgfqpoint{3.164916in}{1.834317in}}%
\pgfpathcurveto{\pgfqpoint{3.157103in}{1.842130in}}{\pgfqpoint{3.146504in}{1.846521in}}{\pgfqpoint{3.135454in}{1.846521in}}%
\pgfpathcurveto{\pgfqpoint{3.124404in}{1.846521in}}{\pgfqpoint{3.113805in}{1.842130in}}{\pgfqpoint{3.105991in}{1.834317in}}%
\pgfpathcurveto{\pgfqpoint{3.098177in}{1.826503in}}{\pgfqpoint{3.093787in}{1.815904in}}{\pgfqpoint{3.093787in}{1.804854in}}%
\pgfpathcurveto{\pgfqpoint{3.093787in}{1.793804in}}{\pgfqpoint{3.098177in}{1.783205in}}{\pgfqpoint{3.105991in}{1.775391in}}%
\pgfpathcurveto{\pgfqpoint{3.113805in}{1.767578in}}{\pgfqpoint{3.124404in}{1.763187in}}{\pgfqpoint{3.135454in}{1.763187in}}%
\pgfpathclose%
\pgfusepath{stroke,fill}%
\end{pgfscope}%
\begin{pgfscope}%
\pgfpathrectangle{\pgfqpoint{0.947176in}{0.499691in}}{\pgfqpoint{3.875000in}{2.695000in}}%
\pgfusepath{clip}%
\pgfsetbuttcap%
\pgfsetroundjoin%
\definecolor{currentfill}{rgb}{0.121569,0.466667,0.705882}%
\pgfsetfillcolor{currentfill}%
\pgfsetlinewidth{1.003750pt}%
\definecolor{currentstroke}{rgb}{0.121569,0.466667,0.705882}%
\pgfsetstrokecolor{currentstroke}%
\pgfsetdash{}{0pt}%
\pgfpathmoveto{\pgfqpoint{3.163647in}{1.800378in}}%
\pgfpathcurveto{\pgfqpoint{3.174697in}{1.800378in}}{\pgfqpoint{3.185296in}{1.804768in}}{\pgfqpoint{3.193110in}{1.812582in}}%
\pgfpathcurveto{\pgfqpoint{3.200923in}{1.820396in}}{\pgfqpoint{3.205313in}{1.830995in}}{\pgfqpoint{3.205313in}{1.842045in}}%
\pgfpathcurveto{\pgfqpoint{3.205313in}{1.853095in}}{\pgfqpoint{3.200923in}{1.863694in}}{\pgfqpoint{3.193110in}{1.871507in}}%
\pgfpathcurveto{\pgfqpoint{3.185296in}{1.879321in}}{\pgfqpoint{3.174697in}{1.883711in}}{\pgfqpoint{3.163647in}{1.883711in}}%
\pgfpathcurveto{\pgfqpoint{3.152597in}{1.883711in}}{\pgfqpoint{3.141998in}{1.879321in}}{\pgfqpoint{3.134184in}{1.871507in}}%
\pgfpathcurveto{\pgfqpoint{3.126370in}{1.863694in}}{\pgfqpoint{3.121980in}{1.853095in}}{\pgfqpoint{3.121980in}{1.842045in}}%
\pgfpathcurveto{\pgfqpoint{3.121980in}{1.830995in}}{\pgfqpoint{3.126370in}{1.820396in}}{\pgfqpoint{3.134184in}{1.812582in}}%
\pgfpathcurveto{\pgfqpoint{3.141998in}{1.804768in}}{\pgfqpoint{3.152597in}{1.800378in}}{\pgfqpoint{3.163647in}{1.800378in}}%
\pgfpathclose%
\pgfusepath{stroke,fill}%
\end{pgfscope}%
\begin{pgfscope}%
\pgfpathrectangle{\pgfqpoint{0.947176in}{0.499691in}}{\pgfqpoint{3.875000in}{2.695000in}}%
\pgfusepath{clip}%
\pgfsetbuttcap%
\pgfsetroundjoin%
\definecolor{currentfill}{rgb}{0.121569,0.466667,0.705882}%
\pgfsetfillcolor{currentfill}%
\pgfsetlinewidth{1.003750pt}%
\definecolor{currentstroke}{rgb}{0.121569,0.466667,0.705882}%
\pgfsetstrokecolor{currentstroke}%
\pgfsetdash{}{0pt}%
\pgfpathmoveto{\pgfqpoint{3.191840in}{1.853458in}}%
\pgfpathcurveto{\pgfqpoint{3.202890in}{1.853458in}}{\pgfqpoint{3.213489in}{1.857849in}}{\pgfqpoint{3.221303in}{1.865662in}}%
\pgfpathcurveto{\pgfqpoint{3.229116in}{1.873476in}}{\pgfqpoint{3.233507in}{1.884075in}}{\pgfqpoint{3.233507in}{1.895125in}}%
\pgfpathcurveto{\pgfqpoint{3.233507in}{1.906175in}}{\pgfqpoint{3.229116in}{1.916774in}}{\pgfqpoint{3.221303in}{1.924588in}}%
\pgfpathcurveto{\pgfqpoint{3.213489in}{1.932401in}}{\pgfqpoint{3.202890in}{1.936792in}}{\pgfqpoint{3.191840in}{1.936792in}}%
\pgfpathcurveto{\pgfqpoint{3.180790in}{1.936792in}}{\pgfqpoint{3.170191in}{1.932401in}}{\pgfqpoint{3.162377in}{1.924588in}}%
\pgfpathcurveto{\pgfqpoint{3.154563in}{1.916774in}}{\pgfqpoint{3.150173in}{1.906175in}}{\pgfqpoint{3.150173in}{1.895125in}}%
\pgfpathcurveto{\pgfqpoint{3.150173in}{1.884075in}}{\pgfqpoint{3.154563in}{1.873476in}}{\pgfqpoint{3.162377in}{1.865662in}}%
\pgfpathcurveto{\pgfqpoint{3.170191in}{1.857849in}}{\pgfqpoint{3.180790in}{1.853458in}}{\pgfqpoint{3.191840in}{1.853458in}}%
\pgfpathclose%
\pgfusepath{stroke,fill}%
\end{pgfscope}%
\begin{pgfscope}%
\pgfpathrectangle{\pgfqpoint{0.947176in}{0.499691in}}{\pgfqpoint{3.875000in}{2.695000in}}%
\pgfusepath{clip}%
\pgfsetbuttcap%
\pgfsetroundjoin%
\definecolor{currentfill}{rgb}{0.121569,0.466667,0.705882}%
\pgfsetfillcolor{currentfill}%
\pgfsetlinewidth{1.003750pt}%
\definecolor{currentstroke}{rgb}{0.121569,0.466667,0.705882}%
\pgfsetstrokecolor{currentstroke}%
\pgfsetdash{}{0pt}%
\pgfpathmoveto{\pgfqpoint{3.220033in}{1.878829in}}%
\pgfpathcurveto{\pgfqpoint{3.231083in}{1.878829in}}{\pgfqpoint{3.241682in}{1.883219in}}{\pgfqpoint{3.249496in}{1.891033in}}%
\pgfpathcurveto{\pgfqpoint{3.257309in}{1.898847in}}{\pgfqpoint{3.261700in}{1.909446in}}{\pgfqpoint{3.261700in}{1.920496in}}%
\pgfpathcurveto{\pgfqpoint{3.261700in}{1.931546in}}{\pgfqpoint{3.257309in}{1.942145in}}{\pgfqpoint{3.249496in}{1.949959in}}%
\pgfpathcurveto{\pgfqpoint{3.241682in}{1.957772in}}{\pgfqpoint{3.231083in}{1.962162in}}{\pgfqpoint{3.220033in}{1.962162in}}%
\pgfpathcurveto{\pgfqpoint{3.208983in}{1.962162in}}{\pgfqpoint{3.198384in}{1.957772in}}{\pgfqpoint{3.190570in}{1.949959in}}%
\pgfpathcurveto{\pgfqpoint{3.182757in}{1.942145in}}{\pgfqpoint{3.178366in}{1.931546in}}{\pgfqpoint{3.178366in}{1.920496in}}%
\pgfpathcurveto{\pgfqpoint{3.178366in}{1.909446in}}{\pgfqpoint{3.182757in}{1.898847in}}{\pgfqpoint{3.190570in}{1.891033in}}%
\pgfpathcurveto{\pgfqpoint{3.198384in}{1.883219in}}{\pgfqpoint{3.208983in}{1.878829in}}{\pgfqpoint{3.220033in}{1.878829in}}%
\pgfpathclose%
\pgfusepath{stroke,fill}%
\end{pgfscope}%
\begin{pgfscope}%
\pgfpathrectangle{\pgfqpoint{0.947176in}{0.499691in}}{\pgfqpoint{3.875000in}{2.695000in}}%
\pgfusepath{clip}%
\pgfsetbuttcap%
\pgfsetroundjoin%
\definecolor{currentfill}{rgb}{0.121569,0.466667,0.705882}%
\pgfsetfillcolor{currentfill}%
\pgfsetlinewidth{1.003750pt}%
\definecolor{currentstroke}{rgb}{0.121569,0.466667,0.705882}%
\pgfsetstrokecolor{currentstroke}%
\pgfsetdash{}{0pt}%
\pgfpathmoveto{\pgfqpoint{3.248226in}{1.903450in}}%
\pgfpathcurveto{\pgfqpoint{3.259276in}{1.903450in}}{\pgfqpoint{3.269875in}{1.907840in}}{\pgfqpoint{3.277689in}{1.915654in}}%
\pgfpathcurveto{\pgfqpoint{3.285502in}{1.923468in}}{\pgfqpoint{3.289893in}{1.934067in}}{\pgfqpoint{3.289893in}{1.945117in}}%
\pgfpathcurveto{\pgfqpoint{3.289893in}{1.956167in}}{\pgfqpoint{3.285502in}{1.966766in}}{\pgfqpoint{3.277689in}{1.974580in}}%
\pgfpathcurveto{\pgfqpoint{3.269875in}{1.982393in}}{\pgfqpoint{3.259276in}{1.986783in}}{\pgfqpoint{3.248226in}{1.986783in}}%
\pgfpathcurveto{\pgfqpoint{3.237176in}{1.986783in}}{\pgfqpoint{3.226577in}{1.982393in}}{\pgfqpoint{3.218763in}{1.974580in}}%
\pgfpathcurveto{\pgfqpoint{3.210950in}{1.966766in}}{\pgfqpoint{3.206559in}{1.956167in}}{\pgfqpoint{3.206559in}{1.945117in}}%
\pgfpathcurveto{\pgfqpoint{3.206559in}{1.934067in}}{\pgfqpoint{3.210950in}{1.923468in}}{\pgfqpoint{3.218763in}{1.915654in}}%
\pgfpathcurveto{\pgfqpoint{3.226577in}{1.907840in}}{\pgfqpoint{3.237176in}{1.903450in}}{\pgfqpoint{3.248226in}{1.903450in}}%
\pgfpathclose%
\pgfusepath{stroke,fill}%
\end{pgfscope}%
\begin{pgfscope}%
\pgfpathrectangle{\pgfqpoint{0.947176in}{0.499691in}}{\pgfqpoint{3.875000in}{2.695000in}}%
\pgfusepath{clip}%
\pgfsetbuttcap%
\pgfsetroundjoin%
\definecolor{currentfill}{rgb}{0.121569,0.466667,0.705882}%
\pgfsetfillcolor{currentfill}%
\pgfsetlinewidth{1.003750pt}%
\definecolor{currentstroke}{rgb}{0.121569,0.466667,0.705882}%
\pgfsetstrokecolor{currentstroke}%
\pgfsetdash{}{0pt}%
\pgfpathmoveto{\pgfqpoint{3.276419in}{1.935151in}}%
\pgfpathcurveto{\pgfqpoint{3.287469in}{1.935151in}}{\pgfqpoint{3.298068in}{1.939541in}}{\pgfqpoint{3.305882in}{1.947354in}}%
\pgfpathcurveto{\pgfqpoint{3.313696in}{1.955168in}}{\pgfqpoint{3.318086in}{1.965767in}}{\pgfqpoint{3.318086in}{1.976817in}}%
\pgfpathcurveto{\pgfqpoint{3.318086in}{1.987867in}}{\pgfqpoint{3.313696in}{1.998466in}}{\pgfqpoint{3.305882in}{2.006280in}}%
\pgfpathcurveto{\pgfqpoint{3.298068in}{2.014094in}}{\pgfqpoint{3.287469in}{2.018484in}}{\pgfqpoint{3.276419in}{2.018484in}}%
\pgfpathcurveto{\pgfqpoint{3.265369in}{2.018484in}}{\pgfqpoint{3.254770in}{2.014094in}}{\pgfqpoint{3.246956in}{2.006280in}}%
\pgfpathcurveto{\pgfqpoint{3.239143in}{1.998466in}}{\pgfqpoint{3.234753in}{1.987867in}}{\pgfqpoint{3.234753in}{1.976817in}}%
\pgfpathcurveto{\pgfqpoint{3.234753in}{1.965767in}}{\pgfqpoint{3.239143in}{1.955168in}}{\pgfqpoint{3.246956in}{1.947354in}}%
\pgfpathcurveto{\pgfqpoint{3.254770in}{1.939541in}}{\pgfqpoint{3.265369in}{1.935151in}}{\pgfqpoint{3.276419in}{1.935151in}}%
\pgfpathclose%
\pgfusepath{stroke,fill}%
\end{pgfscope}%
\begin{pgfscope}%
\pgfpathrectangle{\pgfqpoint{0.947176in}{0.499691in}}{\pgfqpoint{3.875000in}{2.695000in}}%
\pgfusepath{clip}%
\pgfsetbuttcap%
\pgfsetroundjoin%
\definecolor{currentfill}{rgb}{0.121569,0.466667,0.705882}%
\pgfsetfillcolor{currentfill}%
\pgfsetlinewidth{1.003750pt}%
\definecolor{currentstroke}{rgb}{0.121569,0.466667,0.705882}%
\pgfsetstrokecolor{currentstroke}%
\pgfsetdash{}{0pt}%
\pgfpathmoveto{\pgfqpoint{3.304612in}{1.973038in}}%
\pgfpathcurveto{\pgfqpoint{3.315662in}{1.973038in}}{\pgfqpoint{3.326261in}{1.977429in}}{\pgfqpoint{3.334075in}{1.985242in}}%
\pgfpathcurveto{\pgfqpoint{3.341889in}{1.993056in}}{\pgfqpoint{3.346279in}{2.003655in}}{\pgfqpoint{3.346279in}{2.014705in}}%
\pgfpathcurveto{\pgfqpoint{3.346279in}{2.025755in}}{\pgfqpoint{3.341889in}{2.036354in}}{\pgfqpoint{3.334075in}{2.044168in}}%
\pgfpathcurveto{\pgfqpoint{3.326261in}{2.051981in}}{\pgfqpoint{3.315662in}{2.056372in}}{\pgfqpoint{3.304612in}{2.056372in}}%
\pgfpathcurveto{\pgfqpoint{3.293562in}{2.056372in}}{\pgfqpoint{3.282963in}{2.051981in}}{\pgfqpoint{3.275149in}{2.044168in}}%
\pgfpathcurveto{\pgfqpoint{3.267336in}{2.036354in}}{\pgfqpoint{3.262946in}{2.025755in}}{\pgfqpoint{3.262946in}{2.014705in}}%
\pgfpathcurveto{\pgfqpoint{3.262946in}{2.003655in}}{\pgfqpoint{3.267336in}{1.993056in}}{\pgfqpoint{3.275149in}{1.985242in}}%
\pgfpathcurveto{\pgfqpoint{3.282963in}{1.977429in}}{\pgfqpoint{3.293562in}{1.973038in}}{\pgfqpoint{3.304612in}{1.973038in}}%
\pgfpathclose%
\pgfusepath{stroke,fill}%
\end{pgfscope}%
\begin{pgfscope}%
\pgfpathrectangle{\pgfqpoint{0.947176in}{0.499691in}}{\pgfqpoint{3.875000in}{2.695000in}}%
\pgfusepath{clip}%
\pgfsetbuttcap%
\pgfsetroundjoin%
\definecolor{currentfill}{rgb}{0.121569,0.466667,0.705882}%
\pgfsetfillcolor{currentfill}%
\pgfsetlinewidth{1.003750pt}%
\definecolor{currentstroke}{rgb}{0.121569,0.466667,0.705882}%
\pgfsetstrokecolor{currentstroke}%
\pgfsetdash{}{0pt}%
\pgfpathmoveto{\pgfqpoint{3.332805in}{2.009090in}}%
\pgfpathcurveto{\pgfqpoint{3.343855in}{2.009090in}}{\pgfqpoint{3.354455in}{2.013480in}}{\pgfqpoint{3.362268in}{2.021294in}}%
\pgfpathcurveto{\pgfqpoint{3.370082in}{2.029108in}}{\pgfqpoint{3.374472in}{2.039707in}}{\pgfqpoint{3.374472in}{2.050757in}}%
\pgfpathcurveto{\pgfqpoint{3.374472in}{2.061807in}}{\pgfqpoint{3.370082in}{2.072406in}}{\pgfqpoint{3.362268in}{2.080220in}}%
\pgfpathcurveto{\pgfqpoint{3.354455in}{2.088033in}}{\pgfqpoint{3.343855in}{2.092423in}}{\pgfqpoint{3.332805in}{2.092423in}}%
\pgfpathcurveto{\pgfqpoint{3.321755in}{2.092423in}}{\pgfqpoint{3.311156in}{2.088033in}}{\pgfqpoint{3.303343in}{2.080220in}}%
\pgfpathcurveto{\pgfqpoint{3.295529in}{2.072406in}}{\pgfqpoint{3.291139in}{2.061807in}}{\pgfqpoint{3.291139in}{2.050757in}}%
\pgfpathcurveto{\pgfqpoint{3.291139in}{2.039707in}}{\pgfqpoint{3.295529in}{2.029108in}}{\pgfqpoint{3.303343in}{2.021294in}}%
\pgfpathcurveto{\pgfqpoint{3.311156in}{2.013480in}}{\pgfqpoint{3.321755in}{2.009090in}}{\pgfqpoint{3.332805in}{2.009090in}}%
\pgfpathclose%
\pgfusepath{stroke,fill}%
\end{pgfscope}%
\begin{pgfscope}%
\pgfpathrectangle{\pgfqpoint{0.947176in}{0.499691in}}{\pgfqpoint{3.875000in}{2.695000in}}%
\pgfusepath{clip}%
\pgfsetbuttcap%
\pgfsetroundjoin%
\definecolor{currentfill}{rgb}{0.121569,0.466667,0.705882}%
\pgfsetfillcolor{currentfill}%
\pgfsetlinewidth{1.003750pt}%
\definecolor{currentstroke}{rgb}{0.121569,0.466667,0.705882}%
\pgfsetstrokecolor{currentstroke}%
\pgfsetdash{}{0pt}%
\pgfpathmoveto{\pgfqpoint{3.346902in}{2.043405in}}%
\pgfpathcurveto{\pgfqpoint{3.357952in}{2.043405in}}{\pgfqpoint{3.368551in}{2.047795in}}{\pgfqpoint{3.376365in}{2.055609in}}%
\pgfpathcurveto{\pgfqpoint{3.384178in}{2.063422in}}{\pgfqpoint{3.388569in}{2.074021in}}{\pgfqpoint{3.388569in}{2.085071in}}%
\pgfpathcurveto{\pgfqpoint{3.388569in}{2.096121in}}{\pgfqpoint{3.384178in}{2.106720in}}{\pgfqpoint{3.376365in}{2.114534in}}%
\pgfpathcurveto{\pgfqpoint{3.368551in}{2.122348in}}{\pgfqpoint{3.357952in}{2.126738in}}{\pgfqpoint{3.346902in}{2.126738in}}%
\pgfpathcurveto{\pgfqpoint{3.335852in}{2.126738in}}{\pgfqpoint{3.325253in}{2.122348in}}{\pgfqpoint{3.317439in}{2.114534in}}%
\pgfpathcurveto{\pgfqpoint{3.309626in}{2.106720in}}{\pgfqpoint{3.305235in}{2.096121in}}{\pgfqpoint{3.305235in}{2.085071in}}%
\pgfpathcurveto{\pgfqpoint{3.305235in}{2.074021in}}{\pgfqpoint{3.309626in}{2.063422in}}{\pgfqpoint{3.317439in}{2.055609in}}%
\pgfpathcurveto{\pgfqpoint{3.325253in}{2.047795in}}{\pgfqpoint{3.335852in}{2.043405in}}{\pgfqpoint{3.346902in}{2.043405in}}%
\pgfpathclose%
\pgfusepath{stroke,fill}%
\end{pgfscope}%
\begin{pgfscope}%
\pgfpathrectangle{\pgfqpoint{0.947176in}{0.499691in}}{\pgfqpoint{3.875000in}{2.695000in}}%
\pgfusepath{clip}%
\pgfsetbuttcap%
\pgfsetroundjoin%
\definecolor{currentfill}{rgb}{0.121569,0.466667,0.705882}%
\pgfsetfillcolor{currentfill}%
\pgfsetlinewidth{1.003750pt}%
\definecolor{currentstroke}{rgb}{0.121569,0.466667,0.705882}%
\pgfsetstrokecolor{currentstroke}%
\pgfsetdash{}{0pt}%
\pgfpathmoveto{\pgfqpoint{3.375095in}{2.069670in}}%
\pgfpathcurveto{\pgfqpoint{3.386145in}{2.069670in}}{\pgfqpoint{3.396744in}{2.074060in}}{\pgfqpoint{3.404558in}{2.081874in}}%
\pgfpathcurveto{\pgfqpoint{3.412371in}{2.089687in}}{\pgfqpoint{3.416762in}{2.100286in}}{\pgfqpoint{3.416762in}{2.111337in}}%
\pgfpathcurveto{\pgfqpoint{3.416762in}{2.122387in}}{\pgfqpoint{3.412371in}{2.132986in}}{\pgfqpoint{3.404558in}{2.140799in}}%
\pgfpathcurveto{\pgfqpoint{3.396744in}{2.148613in}}{\pgfqpoint{3.386145in}{2.153003in}}{\pgfqpoint{3.375095in}{2.153003in}}%
\pgfpathcurveto{\pgfqpoint{3.364045in}{2.153003in}}{\pgfqpoint{3.353446in}{2.148613in}}{\pgfqpoint{3.345632in}{2.140799in}}%
\pgfpathcurveto{\pgfqpoint{3.337819in}{2.132986in}}{\pgfqpoint{3.333428in}{2.122387in}}{\pgfqpoint{3.333428in}{2.111337in}}%
\pgfpathcurveto{\pgfqpoint{3.333428in}{2.100286in}}{\pgfqpoint{3.337819in}{2.089687in}}{\pgfqpoint{3.345632in}{2.081874in}}%
\pgfpathcurveto{\pgfqpoint{3.353446in}{2.074060in}}{\pgfqpoint{3.364045in}{2.069670in}}{\pgfqpoint{3.375095in}{2.069670in}}%
\pgfpathclose%
\pgfusepath{stroke,fill}%
\end{pgfscope}%
\begin{pgfscope}%
\pgfpathrectangle{\pgfqpoint{0.947176in}{0.499691in}}{\pgfqpoint{3.875000in}{2.695000in}}%
\pgfusepath{clip}%
\pgfsetbuttcap%
\pgfsetroundjoin%
\definecolor{currentfill}{rgb}{0.121569,0.466667,0.705882}%
\pgfsetfillcolor{currentfill}%
\pgfsetlinewidth{1.003750pt}%
\definecolor{currentstroke}{rgb}{0.121569,0.466667,0.705882}%
\pgfsetstrokecolor{currentstroke}%
\pgfsetdash{}{0pt}%
\pgfpathmoveto{\pgfqpoint{3.403288in}{2.101096in}}%
\pgfpathcurveto{\pgfqpoint{3.414338in}{2.101096in}}{\pgfqpoint{3.424937in}{2.105486in}}{\pgfqpoint{3.432751in}{2.113299in}}%
\pgfpathcurveto{\pgfqpoint{3.440565in}{2.121113in}}{\pgfqpoint{3.444955in}{2.131712in}}{\pgfqpoint{3.444955in}{2.142762in}}%
\pgfpathcurveto{\pgfqpoint{3.444955in}{2.153812in}}{\pgfqpoint{3.440565in}{2.164411in}}{\pgfqpoint{3.432751in}{2.172225in}}%
\pgfpathcurveto{\pgfqpoint{3.424937in}{2.180039in}}{\pgfqpoint{3.414338in}{2.184429in}}{\pgfqpoint{3.403288in}{2.184429in}}%
\pgfpathcurveto{\pgfqpoint{3.392238in}{2.184429in}}{\pgfqpoint{3.381639in}{2.180039in}}{\pgfqpoint{3.373825in}{2.172225in}}%
\pgfpathcurveto{\pgfqpoint{3.366012in}{2.164411in}}{\pgfqpoint{3.361621in}{2.153812in}}{\pgfqpoint{3.361621in}{2.142762in}}%
\pgfpathcurveto{\pgfqpoint{3.361621in}{2.131712in}}{\pgfqpoint{3.366012in}{2.121113in}}{\pgfqpoint{3.373825in}{2.113299in}}%
\pgfpathcurveto{\pgfqpoint{3.381639in}{2.105486in}}{\pgfqpoint{3.392238in}{2.101096in}}{\pgfqpoint{3.403288in}{2.101096in}}%
\pgfpathclose%
\pgfusepath{stroke,fill}%
\end{pgfscope}%
\begin{pgfscope}%
\pgfpathrectangle{\pgfqpoint{0.947176in}{0.499691in}}{\pgfqpoint{3.875000in}{2.695000in}}%
\pgfusepath{clip}%
\pgfsetbuttcap%
\pgfsetroundjoin%
\definecolor{currentfill}{rgb}{0.121569,0.466667,0.705882}%
\pgfsetfillcolor{currentfill}%
\pgfsetlinewidth{1.003750pt}%
\definecolor{currentstroke}{rgb}{0.121569,0.466667,0.705882}%
\pgfsetstrokecolor{currentstroke}%
\pgfsetdash{}{0pt}%
\pgfpathmoveto{\pgfqpoint{3.417385in}{2.113247in}}%
\pgfpathcurveto{\pgfqpoint{3.428435in}{2.113247in}}{\pgfqpoint{3.439034in}{2.117637in}}{\pgfqpoint{3.446847in}{2.125451in}}%
\pgfpathcurveto{\pgfqpoint{3.454661in}{2.133264in}}{\pgfqpoint{3.459051in}{2.143863in}}{\pgfqpoint{3.459051in}{2.154913in}}%
\pgfpathcurveto{\pgfqpoint{3.459051in}{2.165964in}}{\pgfqpoint{3.454661in}{2.176563in}}{\pgfqpoint{3.446847in}{2.184376in}}%
\pgfpathcurveto{\pgfqpoint{3.439034in}{2.192190in}}{\pgfqpoint{3.428435in}{2.196580in}}{\pgfqpoint{3.417385in}{2.196580in}}%
\pgfpathcurveto{\pgfqpoint{3.406335in}{2.196580in}}{\pgfqpoint{3.395735in}{2.192190in}}{\pgfqpoint{3.387922in}{2.184376in}}%
\pgfpathcurveto{\pgfqpoint{3.380108in}{2.176563in}}{\pgfqpoint{3.375718in}{2.165964in}}{\pgfqpoint{3.375718in}{2.154913in}}%
\pgfpathcurveto{\pgfqpoint{3.375718in}{2.143863in}}{\pgfqpoint{3.380108in}{2.133264in}}{\pgfqpoint{3.387922in}{2.125451in}}%
\pgfpathcurveto{\pgfqpoint{3.395735in}{2.117637in}}{\pgfqpoint{3.406335in}{2.113247in}}{\pgfqpoint{3.417385in}{2.113247in}}%
\pgfpathclose%
\pgfusepath{stroke,fill}%
\end{pgfscope}%
\begin{pgfscope}%
\pgfpathrectangle{\pgfqpoint{0.947176in}{0.499691in}}{\pgfqpoint{3.875000in}{2.695000in}}%
\pgfusepath{clip}%
\pgfsetbuttcap%
\pgfsetroundjoin%
\definecolor{currentfill}{rgb}{0.121569,0.466667,0.705882}%
\pgfsetfillcolor{currentfill}%
\pgfsetlinewidth{1.003750pt}%
\definecolor{currentstroke}{rgb}{0.121569,0.466667,0.705882}%
\pgfsetstrokecolor{currentstroke}%
\pgfsetdash{}{0pt}%
\pgfpathmoveto{\pgfqpoint{3.431481in}{2.159592in}}%
\pgfpathcurveto{\pgfqpoint{3.442531in}{2.159592in}}{\pgfqpoint{3.453130in}{2.163982in}}{\pgfqpoint{3.460944in}{2.171796in}}%
\pgfpathcurveto{\pgfqpoint{3.468758in}{2.179610in}}{\pgfqpoint{3.473148in}{2.190209in}}{\pgfqpoint{3.473148in}{2.201259in}}%
\pgfpathcurveto{\pgfqpoint{3.473148in}{2.212309in}}{\pgfqpoint{3.468758in}{2.222908in}}{\pgfqpoint{3.460944in}{2.230722in}}%
\pgfpathcurveto{\pgfqpoint{3.453130in}{2.238535in}}{\pgfqpoint{3.442531in}{2.242925in}}{\pgfqpoint{3.431481in}{2.242925in}}%
\pgfpathcurveto{\pgfqpoint{3.420431in}{2.242925in}}{\pgfqpoint{3.409832in}{2.238535in}}{\pgfqpoint{3.402018in}{2.230722in}}%
\pgfpathcurveto{\pgfqpoint{3.394205in}{2.222908in}}{\pgfqpoint{3.389815in}{2.212309in}}{\pgfqpoint{3.389815in}{2.201259in}}%
\pgfpathcurveto{\pgfqpoint{3.389815in}{2.190209in}}{\pgfqpoint{3.394205in}{2.179610in}}{\pgfqpoint{3.402018in}{2.171796in}}%
\pgfpathcurveto{\pgfqpoint{3.409832in}{2.163982in}}{\pgfqpoint{3.420431in}{2.159592in}}{\pgfqpoint{3.431481in}{2.159592in}}%
\pgfpathclose%
\pgfusepath{stroke,fill}%
\end{pgfscope}%
\begin{pgfscope}%
\pgfpathrectangle{\pgfqpoint{0.947176in}{0.499691in}}{\pgfqpoint{3.875000in}{2.695000in}}%
\pgfusepath{clip}%
\pgfsetbuttcap%
\pgfsetroundjoin%
\definecolor{currentfill}{rgb}{0.121569,0.466667,0.705882}%
\pgfsetfillcolor{currentfill}%
\pgfsetlinewidth{1.003750pt}%
\definecolor{currentstroke}{rgb}{0.121569,0.466667,0.705882}%
\pgfsetstrokecolor{currentstroke}%
\pgfsetdash{}{0pt}%
\pgfpathmoveto{\pgfqpoint{3.459674in}{2.176086in}}%
\pgfpathcurveto{\pgfqpoint{3.470724in}{2.176086in}}{\pgfqpoint{3.481323in}{2.180476in}}{\pgfqpoint{3.489137in}{2.188290in}}%
\pgfpathcurveto{\pgfqpoint{3.496951in}{2.196103in}}{\pgfqpoint{3.501341in}{2.206702in}}{\pgfqpoint{3.501341in}{2.217752in}}%
\pgfpathcurveto{\pgfqpoint{3.501341in}{2.228803in}}{\pgfqpoint{3.496951in}{2.239402in}}{\pgfqpoint{3.489137in}{2.247215in}}%
\pgfpathcurveto{\pgfqpoint{3.481323in}{2.255029in}}{\pgfqpoint{3.470724in}{2.259419in}}{\pgfqpoint{3.459674in}{2.259419in}}%
\pgfpathcurveto{\pgfqpoint{3.448624in}{2.259419in}}{\pgfqpoint{3.438025in}{2.255029in}}{\pgfqpoint{3.430212in}{2.247215in}}%
\pgfpathcurveto{\pgfqpoint{3.422398in}{2.239402in}}{\pgfqpoint{3.418008in}{2.228803in}}{\pgfqpoint{3.418008in}{2.217752in}}%
\pgfpathcurveto{\pgfqpoint{3.418008in}{2.206702in}}{\pgfqpoint{3.422398in}{2.196103in}}{\pgfqpoint{3.430212in}{2.188290in}}%
\pgfpathcurveto{\pgfqpoint{3.438025in}{2.180476in}}{\pgfqpoint{3.448624in}{2.176086in}}{\pgfqpoint{3.459674in}{2.176086in}}%
\pgfpathclose%
\pgfusepath{stroke,fill}%
\end{pgfscope}%
\begin{pgfscope}%
\pgfpathrectangle{\pgfqpoint{0.947176in}{0.499691in}}{\pgfqpoint{3.875000in}{2.695000in}}%
\pgfusepath{clip}%
\pgfsetbuttcap%
\pgfsetroundjoin%
\definecolor{currentfill}{rgb}{0.121569,0.466667,0.705882}%
\pgfsetfillcolor{currentfill}%
\pgfsetlinewidth{1.003750pt}%
\definecolor{currentstroke}{rgb}{0.121569,0.466667,0.705882}%
\pgfsetstrokecolor{currentstroke}%
\pgfsetdash{}{0pt}%
\pgfpathmoveto{\pgfqpoint{3.473771in}{2.197369in}}%
\pgfpathcurveto{\pgfqpoint{3.484821in}{2.197369in}}{\pgfqpoint{3.495420in}{2.201759in}}{\pgfqpoint{3.503234in}{2.209573in}}%
\pgfpathcurveto{\pgfqpoint{3.511047in}{2.217387in}}{\pgfqpoint{3.515438in}{2.227986in}}{\pgfqpoint{3.515438in}{2.239036in}}%
\pgfpathcurveto{\pgfqpoint{3.515438in}{2.250086in}}{\pgfqpoint{3.511047in}{2.260685in}}{\pgfqpoint{3.503234in}{2.268498in}}%
\pgfpathcurveto{\pgfqpoint{3.495420in}{2.276312in}}{\pgfqpoint{3.484821in}{2.280702in}}{\pgfqpoint{3.473771in}{2.280702in}}%
\pgfpathcurveto{\pgfqpoint{3.462721in}{2.280702in}}{\pgfqpoint{3.452122in}{2.276312in}}{\pgfqpoint{3.444308in}{2.268498in}}%
\pgfpathcurveto{\pgfqpoint{3.436494in}{2.260685in}}{\pgfqpoint{3.432104in}{2.250086in}}{\pgfqpoint{3.432104in}{2.239036in}}%
\pgfpathcurveto{\pgfqpoint{3.432104in}{2.227986in}}{\pgfqpoint{3.436494in}{2.217387in}}{\pgfqpoint{3.444308in}{2.209573in}}%
\pgfpathcurveto{\pgfqpoint{3.452122in}{2.201759in}}{\pgfqpoint{3.462721in}{2.197369in}}{\pgfqpoint{3.473771in}{2.197369in}}%
\pgfpathclose%
\pgfusepath{stroke,fill}%
\end{pgfscope}%
\begin{pgfscope}%
\pgfpathrectangle{\pgfqpoint{0.947176in}{0.499691in}}{\pgfqpoint{3.875000in}{2.695000in}}%
\pgfusepath{clip}%
\pgfsetbuttcap%
\pgfsetroundjoin%
\definecolor{currentfill}{rgb}{0.121569,0.466667,0.705882}%
\pgfsetfillcolor{currentfill}%
\pgfsetlinewidth{1.003750pt}%
\definecolor{currentstroke}{rgb}{0.121569,0.466667,0.705882}%
\pgfsetstrokecolor{currentstroke}%
\pgfsetdash{}{0pt}%
\pgfpathmoveto{\pgfqpoint{3.501964in}{2.227851in}}%
\pgfpathcurveto{\pgfqpoint{3.513014in}{2.227851in}}{\pgfqpoint{3.523613in}{2.232241in}}{\pgfqpoint{3.531427in}{2.240055in}}%
\pgfpathcurveto{\pgfqpoint{3.539240in}{2.247869in}}{\pgfqpoint{3.543631in}{2.258468in}}{\pgfqpoint{3.543631in}{2.269518in}}%
\pgfpathcurveto{\pgfqpoint{3.543631in}{2.280568in}}{\pgfqpoint{3.539240in}{2.291167in}}{\pgfqpoint{3.531427in}{2.298981in}}%
\pgfpathcurveto{\pgfqpoint{3.523613in}{2.306794in}}{\pgfqpoint{3.513014in}{2.311185in}}{\pgfqpoint{3.501964in}{2.311185in}}%
\pgfpathcurveto{\pgfqpoint{3.490914in}{2.311185in}}{\pgfqpoint{3.480315in}{2.306794in}}{\pgfqpoint{3.472501in}{2.298981in}}%
\pgfpathcurveto{\pgfqpoint{3.464688in}{2.291167in}}{\pgfqpoint{3.460297in}{2.280568in}}{\pgfqpoint{3.460297in}{2.269518in}}%
\pgfpathcurveto{\pgfqpoint{3.460297in}{2.258468in}}{\pgfqpoint{3.464688in}{2.247869in}}{\pgfqpoint{3.472501in}{2.240055in}}%
\pgfpathcurveto{\pgfqpoint{3.480315in}{2.232241in}}{\pgfqpoint{3.490914in}{2.227851in}}{\pgfqpoint{3.501964in}{2.227851in}}%
\pgfpathclose%
\pgfusepath{stroke,fill}%
\end{pgfscope}%
\begin{pgfscope}%
\pgfpathrectangle{\pgfqpoint{0.947176in}{0.499691in}}{\pgfqpoint{3.875000in}{2.695000in}}%
\pgfusepath{clip}%
\pgfsetrectcap%
\pgfsetroundjoin%
\pgfsetlinewidth{1.505625pt}%
\definecolor{currentstroke}{rgb}{1.000000,0.843137,0.000000}%
\pgfsetstrokecolor{currentstroke}%
\pgfsetdash{}{0pt}%
\pgfpathmoveto{\pgfqpoint{1.123312in}{0.622191in}}%
\pgfpathlineto{\pgfqpoint{1.247362in}{0.644470in}}%
\pgfpathlineto{\pgfqpoint{1.361544in}{0.667196in}}%
\pgfpathlineto{\pgfqpoint{1.468678in}{0.690752in}}%
\pgfpathlineto{\pgfqpoint{1.568763in}{0.714999in}}%
\pgfpathlineto{\pgfqpoint{1.663210in}{0.740148in}}%
\pgfpathlineto{\pgfqpoint{1.752019in}{0.766080in}}%
\pgfpathlineto{\pgfqpoint{1.835188in}{0.792638in}}%
\pgfpathlineto{\pgfqpoint{1.914129in}{0.820134in}}%
\pgfpathlineto{\pgfqpoint{1.988841in}{0.848459in}}%
\pgfpathlineto{\pgfqpoint{2.060733in}{0.878071in}}%
\pgfpathlineto{\pgfqpoint{2.129806in}{0.908941in}}%
\pgfpathlineto{\pgfqpoint{2.196060in}{0.941015in}}%
\pgfpathlineto{\pgfqpoint{2.259494in}{0.974214in}}%
\pgfpathlineto{\pgfqpoint{2.320109in}{1.008430in}}%
\pgfpathlineto{\pgfqpoint{2.379315in}{1.044410in}}%
\pgfpathlineto{\pgfqpoint{2.437111in}{1.082176in}}%
\pgfpathlineto{\pgfqpoint{2.493497in}{1.121730in}}%
\pgfpathlineto{\pgfqpoint{2.548473in}{1.163049in}}%
\pgfpathlineto{\pgfqpoint{2.602040in}{1.206083in}}%
\pgfpathlineto{\pgfqpoint{2.655607in}{1.251996in}}%
\pgfpathlineto{\pgfqpoint{2.709174in}{1.300905in}}%
\pgfpathlineto{\pgfqpoint{2.762741in}{1.352893in}}%
\pgfpathlineto{\pgfqpoint{2.816308in}{1.408001in}}%
\pgfpathlineto{\pgfqpoint{2.871284in}{1.467778in}}%
\pgfpathlineto{\pgfqpoint{2.927671in}{1.532368in}}%
\pgfpathlineto{\pgfqpoint{2.988286in}{1.605257in}}%
\pgfpathlineto{\pgfqpoint{3.054540in}{1.688529in}}%
\pgfpathlineto{\pgfqpoint{3.132071in}{1.789734in}}%
\pgfpathlineto{\pgfqpoint{3.249072in}{1.946682in}}%
\pgfpathlineto{\pgfqpoint{3.377350in}{2.117794in}}%
\pgfpathlineto{\pgfqpoint{3.453472in}{2.215604in}}%
\pgfpathlineto{\pgfqpoint{3.518316in}{2.295415in}}%
\pgfpathlineto{\pgfqpoint{3.578931in}{2.366511in}}%
\pgfpathlineto{\pgfqpoint{3.635317in}{2.429293in}}%
\pgfpathlineto{\pgfqpoint{3.690294in}{2.487237in}}%
\pgfpathlineto{\pgfqpoint{3.743861in}{2.540538in}}%
\pgfpathlineto{\pgfqpoint{3.797428in}{2.590735in}}%
\pgfpathlineto{\pgfqpoint{3.850994in}{2.637897in}}%
\pgfpathlineto{\pgfqpoint{3.904561in}{2.682127in}}%
\pgfpathlineto{\pgfqpoint{3.959538in}{2.724611in}}%
\pgfpathlineto{\pgfqpoint{4.014514in}{2.764305in}}%
\pgfpathlineto{\pgfqpoint{4.070901in}{2.802297in}}%
\pgfpathlineto{\pgfqpoint{4.128696in}{2.838573in}}%
\pgfpathlineto{\pgfqpoint{4.187902in}{2.873141in}}%
\pgfpathlineto{\pgfqpoint{4.249927in}{2.906762in}}%
\pgfpathlineto{\pgfqpoint{4.313361in}{2.938633in}}%
\pgfpathlineto{\pgfqpoint{4.379615in}{2.969444in}}%
\pgfpathlineto{\pgfqpoint{4.448688in}{2.999122in}}%
\pgfpathlineto{\pgfqpoint{4.520580in}{3.027614in}}%
\pgfpathlineto{\pgfqpoint{4.596702in}{3.055387in}}%
\pgfpathlineto{\pgfqpoint{4.646040in}{3.072191in}}%
\pgfpathlineto{\pgfqpoint{4.646040in}{3.072191in}}%
\pgfusepath{stroke}%
\end{pgfscope}%
\begin{pgfscope}%
\pgfsetrectcap%
\pgfsetmiterjoin%
\pgfsetlinewidth{0.803000pt}%
\definecolor{currentstroke}{rgb}{0.000000,0.000000,0.000000}%
\pgfsetstrokecolor{currentstroke}%
\pgfsetdash{}{0pt}%
\pgfpathmoveto{\pgfqpoint{0.947176in}{0.499691in}}%
\pgfpathlineto{\pgfqpoint{0.947176in}{3.194691in}}%
\pgfusepath{stroke}%
\end{pgfscope}%
\begin{pgfscope}%
\pgfsetrectcap%
\pgfsetmiterjoin%
\pgfsetlinewidth{0.803000pt}%
\definecolor{currentstroke}{rgb}{0.000000,0.000000,0.000000}%
\pgfsetstrokecolor{currentstroke}%
\pgfsetdash{}{0pt}%
\pgfpathmoveto{\pgfqpoint{4.822176in}{0.499691in}}%
\pgfpathlineto{\pgfqpoint{4.822176in}{3.194691in}}%
\pgfusepath{stroke}%
\end{pgfscope}%
\begin{pgfscope}%
\pgfsetrectcap%
\pgfsetmiterjoin%
\pgfsetlinewidth{0.803000pt}%
\definecolor{currentstroke}{rgb}{0.000000,0.000000,0.000000}%
\pgfsetstrokecolor{currentstroke}%
\pgfsetdash{}{0pt}%
\pgfpathmoveto{\pgfqpoint{0.947176in}{0.499691in}}%
\pgfpathlineto{\pgfqpoint{4.822176in}{0.499691in}}%
\pgfusepath{stroke}%
\end{pgfscope}%
\begin{pgfscope}%
\pgfsetrectcap%
\pgfsetmiterjoin%
\pgfsetlinewidth{0.803000pt}%
\definecolor{currentstroke}{rgb}{0.000000,0.000000,0.000000}%
\pgfsetstrokecolor{currentstroke}%
\pgfsetdash{}{0pt}%
\pgfpathmoveto{\pgfqpoint{0.947176in}{3.194691in}}%
\pgfpathlineto{\pgfqpoint{4.822176in}{3.194691in}}%
\pgfusepath{stroke}%
\end{pgfscope}%
\begin{pgfscope}%
\pgfpathrectangle{\pgfqpoint{0.947176in}{0.499691in}}{\pgfqpoint{3.875000in}{2.695000in}}%
\pgfusepath{clip}%
\pgfsetbuttcap%
\pgfsetroundjoin%
\definecolor{currentfill}{rgb}{1.000000,0.388235,0.278431}%
\pgfsetfillcolor{currentfill}%
\pgfsetlinewidth{1.003750pt}%
\definecolor{currentstroke}{rgb}{1.000000,0.388235,0.278431}%
\pgfsetstrokecolor{currentstroke}%
\pgfsetdash{}{0pt}%
\pgfpathmoveto{\pgfqpoint{3.229337in}{1.878406in}}%
\pgfpathcurveto{\pgfqpoint{3.240387in}{1.878406in}}{\pgfqpoint{3.250986in}{1.882796in}}{\pgfqpoint{3.258799in}{1.890610in}}%
\pgfpathcurveto{\pgfqpoint{3.266613in}{1.898424in}}{\pgfqpoint{3.271003in}{1.909023in}}{\pgfqpoint{3.271003in}{1.920073in}}%
\pgfpathcurveto{\pgfqpoint{3.271003in}{1.931123in}}{\pgfqpoint{3.266613in}{1.941722in}}{\pgfqpoint{3.258799in}{1.949535in}}%
\pgfpathcurveto{\pgfqpoint{3.250986in}{1.957349in}}{\pgfqpoint{3.240387in}{1.961739in}}{\pgfqpoint{3.229337in}{1.961739in}}%
\pgfpathcurveto{\pgfqpoint{3.218287in}{1.961739in}}{\pgfqpoint{3.207688in}{1.957349in}}{\pgfqpoint{3.199874in}{1.949535in}}%
\pgfpathcurveto{\pgfqpoint{3.192060in}{1.941722in}}{\pgfqpoint{3.187670in}{1.931123in}}{\pgfqpoint{3.187670in}{1.920073in}}%
\pgfpathcurveto{\pgfqpoint{3.187670in}{1.909023in}}{\pgfqpoint{3.192060in}{1.898424in}}{\pgfqpoint{3.199874in}{1.890610in}}%
\pgfpathcurveto{\pgfqpoint{3.207688in}{1.882796in}}{\pgfqpoint{3.218287in}{1.878406in}}{\pgfqpoint{3.229337in}{1.878406in}}%
\pgfpathclose%
\pgfusepath{stroke,fill}%
\end{pgfscope}%
\end{pgfpicture}%
\makeatother%
\endgroup%

    \caption{Representación de $\varphi$ frente a f (escala logarítmica) con un ajuste a una curva arcotangente}
  \end{figure}

  De este modo obtenemos una frecuencia de corte de 1307,98 Hz, mucho más razonable en base a nuestras estimaciones. Vuelvo a reiterar que estos no son los mismos datos que la segunda parte, esos siguen teniendo un error sistemático que nos impedirá realizar el cálculo con precisión. Sin embargo, de esta manera, podemos probar de otro modo que para la frecuencia de corte la fase es de 45º.

  Por lo general, consideramos la primera parte de la práctica exitosa, ya que aproximamos muy bien el valor de la frecuencia de corte, mientras que esta segunda parte sólo parcialmente, ya que no pudimos acercarnos todo lo deseado a ella con los datos que obtuvimos.



  \newpage
  \begin{appendices}
    \addtocontents{toc}{\protect\setcounter{tocdepth}{2}}
    \makeatletter
    \addtocontents{toc}{%
    \begingroup
    \let\protect\l@chapter\protect\l@section
    \let\protect\l@section\protect\l@subsection
    }

    \section{Datos extra}

    \subsection{Corriente alterna: Circuito RC}
    \label{extra1}

    \begin{table}[H]
    \centering
    \resizebox{!}{4in}{
    \csvreader[
      tabular=|c|c|c|c|c|c|c|c|c|,
      table head=\hline Medida & $f (Hz)$ & $\log f$ & $V_m (V)$ & $V_mR (V)$ & $V_mC (V)$ & $Z (\Omega)$ & $20\log Z$ & $V_{mR}/V_{mC}$ \\ \hline,
      late after last line=\\\hline,
      separator=semicolon
      ]{CA4AltAll.csv}
      {f=\f, logf=\logf, Vm=\vm, VmR=\vmr, VmC=\vmc, Z=\z, 20logZ=\logz, VmRVmC=\vmbyvm}
      {\thecsvrow & \f & \logf & \vm & \vmr & \vmc & \z & \logz & \vmbyvm}}
    \caption{Mediciones extra de potenciales frente a frecuencia}
    \end{table}

    \subsection{Corriente alterna: Desfase entre señales}
    \label{extra2}

    \begin{table}[H]
    \centering
    \resizebox{!}{4.8in}{
    \csvreader[
      tabular=|c|c|c|c|c|c|,
      table head=\hline Medida & $f (Hz)$ & $\log f$ & $\Delta t (s)$ & $\varphi (rad)$ & $\varphi$ \textit{(º)} \\ \hline,
      late after last line=\\\hline,
      separator=semicolon
      ]{CA5All.csv}
      {f=\f, logf=\logf, deltat=\dt, phirad=\phir, phideg=\phid}
      {\thecsvrow & \f & \logf & \dt & \phir & \phid}}
    \caption{Mediciones extra de $\Delta t$ según la frecuencia}
    \end{table}


    \section{Bibliografía}

    \subsection{Facultad de física}

    Amigo, Alfredo. \textit{Análisis de incertidumbres.} Apuntes de Técnicas Experimentales I, Facultad de Física, 2019-20.

    \textit{Práctica 1. Circuitos de Corriente Continua.} Apuntes de Técnicas Experimentales I, Facultad de Física, 2019-20.

    \textit{Práctica 2. Circuitos de Corriente Alterna.} Apuntes de Técnicas Experimentales I, Facultad de Física, 2019-20.

    \textit{Anexo Práctica 1. El Polímetro.} Apuntes de Técnicas Experimentales I, Facultad de Física, 2019-20.

    \textit{Anexo Práctica 2. El Osciloscopio.} Apuntes de Técnicas Experimentales I, Facultad de Física, 2019-20.

    \subsection{Otras fuentes}

    Wikipedia. \textit{Función sigmoide.} \url{https://en.wikipedia.org/wiki/Sigmoid_function}

    SciPy. \textit{Scypy - Optimize - Curve Fit.} \url{https://docs.scipy.org/doc/scipy/reference/generated/scipy.optimize.curve_fit.html}

    SciPy. \textit{Numpy - Polynomial - Fit.} \url{https://docs.scipy.org/doc/numpy/reference/generated/numpy.polynomial.polynomial.Polynomial.fit.html}

    \subsection{LaTeX}

    Redaelli, Massimo. \textit{CircuiTikZ.} \url{http://texdoc.net/texmf-dist/doc/latex/circuitikz/circuitikzmanual.pdf}

    Sturm, Thomas. \textit{The csvsimple package.} \url{https://osl.ugr.es/CTAN/macros/latex/contrib/csvsimple/csvsimple.pdf}

    \section{Código y datos}

    Todo el código y los datos utilizados para la creación de esta memoria se encuentran en un repositorio hospedado en GitHub. Puede accederse desde el siguiente enlace:

    \url{https://github.com/josekoalas/andromeda/tree/master/Instrumentacion}

    Dentro de la carpeta \code{Instrumentación} pueden encontrarse los siguientes ficheros:

    \begin{itemize}[label=$-$]
      \item Memoria.tex y Memoria.pdf: Archivo LaTeX y pdf generado para esta memoria.
      \item Graphics*.py: Archivos con código python para crear las gráficas en formato .pgf.
      \item CA*.pgf, CC*.pgf y Osciloscope*.pgf: Las distintas gráficas en formato vectorial adecuado para su uso en LaTeX, de corriente alterna y corriente continua.
      \item CA*.csv y CC*.csv: Tablas con los datos de la práctica utilizadas tanto en LaTeX cómo en python.
      \item R*.png: Imágenes con las resistencias con colores.
    \end{itemize}

    \addtocontents{toc}{\endgroup}
  \end{appendices}


\end{document}
